\setstretch{1.145}
\titleclass{\chapter}{straight}
\titlespacing{\chapter}{0pt}{1em}{1em}

\mychapbig{I.}

Antaŭ ĉio mi parolos kelkajn vortojn pri tiuj kritikoj, kiujn mi ĝis hodiaŭ aŭdis aŭ legis, en gazetoj aŭ en leteroj al mi, kvankam mi devas antaŭsciigi la leganton, ke tiu ĉi punkto estas en miaj okuloj tre grava kaj poste mi ankoraŭ parolos pri ĝi pli vaste. Mi ne volus fari ian\footnote{\emph{Komento el kompostanto:} Ne klaras ĉi tie, ĉu Zamenhofo intencis la modernan "iam" aŭ la modernan "ian".  Ĉu li \emph{neniam} volus fari premon?  Aŭ ĉu li volus \emph{nenian} premon?  Ĉar ne klaras, mi forlasas la originalan "ian" ĉi tie.} premon sur la juĝo de l' publiko, kaj mi volus, ke la mondo kreu mem sian decidon en la afero, kiun mi proponis. Sed kelkaj kritikoj estis tiel skribitaj, ke mi ne povas tute silenti pri ili.

\emph{a}) Unuj parolis pri l' aŭtoro, anstataŭ paroli pri l' afero. Ili aŭ ŝutis komplimentojn al la aŭtoro, rigardigis, kiom da malfacila laboro kredeble la afero min kostis, kaj, laŭdante la \emph{aŭtoron}, ili preskaŭ tute forgesis paroli pri l' utileco kaj la signifo de l' \emph{afero} kaj decidigi la publikon labori por ĝi; aliaj, ne trovante en mia verko la instruitan miksaĵon kaj la instruita-teorian filozofadon, kiujn ili kutimis renkonti en ĉia grava verko, timis, ke la pseŭdonima aŭtoro eble estas ne sufiĉe instruita aŭ ne sufiĉe merita, kaj ili timis esprimi decidan juĝon, pli multe penante malkovri, kiu estas la pseŭdonima aŭtoro. Por igi la kritikistojn tute apartigi la \emph{aferon} de la \emph{aŭtoro}, mi publike diras mem, ke mi ne estas multege instruita lingvisto, ke mi estas tute senmerita kaj ne konata en la mondo. Mi scias, ke mia konfeso malvarmigos multajn por la afero, sed mi volas, ke oni juĝu ne l' aŭtoron, sed la verkon. Se la verko estas bona, prenu ĝin; se ĝi estas malbona---ĵetu ĝin. Per kia vojo mi venis al la kreo de mia lingvo kaj laŭ kiaj metodoj mi laboris,---mi ankoraŭ parolos, sed en unu de la \emph{venontaj} kajeroj; ĉar laŭ mi tiu ĉi demando estas por la publiko sen signifo: por la mondo estas gravaj sole la \emph{rezultatoj}.

\emph{b}) Aliaj ekbrilis per senfinaj filozofadoj kaj skribis instruitajn artikulojn, tute ne pensinte kaj ne demandinte sin, ĉu ili parolas logike kaj afertuŝante. Anstataŭ provi praktike (kion fari estas tre facile), ĉu la lingvo, proponita de mi, taŭgas por internacia kompreniĝo, ĉu ĝi efektive al ĉiu donas la eblon esti komprenata de personoj alinaciaj,---ili parolis pri la fiziologio kaj historio de l' lingvoj vivaj; anstataŭ provi per ilia propra orelo, ĉu mia lingvo estas bonsona aŭ ne,---ili teorie parolis pri leĝoj de bonsoneco; anstataŭ analizi, ĉu mi bone kreis la vortaron kaj ĉu oni ne povus fari ĝin ankoraŭ pli komprenebla kaj pli praktika, ili diris, ke la vortaro devas esti farita el radikoj Sanskritaj aŭ el vortoj, prenitaj mikse el ĉiuj lingvoj de l' mondo. (La lingvo multe per tio ĉi \emph{perdus}, fariĝinte tute ne komprenebla; sed kion ĝi \emph{gajnus}, esceptinte la sennecesan instruitan eksteron? tion ĉi ili tute forgesis sin demandi.)

\emph{c}) Aliaj skribis kritikon pri mia afero, eĉ ne leginte bone mian malgrandan broŝuron kaj eĉ ne peninte kompreni la aferon. Tiel ekzemple la unuatempajn signetojn inter la partoj de l' vortoj ili tute ne komprenis, kaj skribante ekzemple \emph{\glqq{}ensong, oprinc, in, o, nmivid, is\grqq{}} (anstataŭ: \emph{\glqq{}en sonĝ\,o princ\,in\,o\,n mi vid\,is\grqq{}}), ili rigardigis iliajn legantojn, \glqq{}kiel malbonsona kaj nekomprenebla la lingvo estas\grqq{}! La projekton de l' tutmonda voĉdono, kiu kun la efektiva kaj senkondiĉa signifo de l' lingvo tute ne estas kunligita, kaj kiu estas proponita sole por tio, ke la lingvo pli rapide el \emph{internacia} fariĝu \emph{tutmonda},---ili prenis por la plej grava kaj fonda parto de l' afero kaj komprenigis la legantojn, ke \glqq{}ĉar dek milionoj adeptoj (!) neniam estos kolektitaj, tial la afero tute ne havas estontecon\grqq{}! Kelkajn fojojn mi eĉ legis longajn artikulojn pri mia afero, kie estis videble, ke la aŭtoroj eĉ ne vidis mian verkon.

\emph{ĉ}) Aliaj, anstataŭ paroli pri la utileco aŭ la senutileco de mia lingvo, donis sole sensencajn ŝercojn, kiuj de iliaj legantoj estis eble prenataj por kritiko, ĉar multaj legantoj propran juĝon ne havas, kaj la plej malsaĝaj ŝercoj je ia afero estas por ili sufiĉa vidigo, ke la afero estas \glqq{}ridinda\grqq{} kaj taŭgas por nenio.

Mi ne deziras laŭdon, mi volas, ke oni min helpu forigi la erarojn, kiujn mi faris, kaj ju la kritikoj de mia lingvo estas pli severaj, des pli danke mi ilin alprenas, se ili nur havas la celon montri al mi la erarojn de mia afero, ke mi ilin bonigu, sed ne ridi sen senco aŭ insulti sen kaŭzo. Mi scias tre bone, ke la verko de \emph{unu homo} ne povas esti senerara, se tiu homo eĉ estus la plej genia kaj multe pli instruita ol mi. Tial mi ne donis ankoraŭ al mia lingvo la finan formon; mi ne parolas: \glqq{}jen la lingvo estas kreita kaj preta, tiel mi volas, tia ĝi estu kaj tia ĝi restu!\grqq{} Ĉio bonigebla estos bonigata per la konsiloj de l' mondo. Mi ne volas esti \emph{kreinto} de l' lingvo, mi volas nur esti \emph{iniciatoro}. Tio ĉi estu ankaŭ respondo al tiuj amikoj de l' lingvo internacia, kiuj estas neatendemaj kaj volus jam vidi librojn kaj gazetojn en la lingvo internacia, plenajn vortarojn, vortarojn nacia-internaciajn kaj cetere. Ne malfacile estus por mi kontentigi tiujn ĉi amikojn; sed ili ne forgesu, ke tio ĉi estus danĝera por la afero mem, kiu estas tiel grava, ke estus nepardoneble faradi laŭ la propra decido de unu homo. Mi ne povas diri, ke la lingvo estas preta, ĝis ĝi estos trairinta la juĝon de l' publiko. Unu jaro ne estas eterno, kaj tamen tiu ĉi jaro estas tre grava por l' afero. Tiel ankaŭ mi ne povas fari iajn ŝanĝojn en la lingvo tuj post la ricevo de la konsiloj, se tiuj ĉi konsiloj eĉ estus la plej seneraraj kaj venus de la plej kompetentaj personoj. En la daŭro de la tuta jaro 1888 la lingvo restos \emph{tute sen ŝanĝo}; sed kiam la jaro estos finita, tiam ĉiuj necesaj ŝanĝoj, antaŭe analizitaj kaj provitaj, estos publikigitaj, la lingvo ricevos la finan formon, kaj tiam komencos ĝia plena funkciado. Juĝante laŭ la konsiloj, kiuj estas senditaj al mi ĝis hodiaŭ, mi pensas, ke la lingvo kredeble estos ŝanĝita tre malmulte, ĉar la plej granda parto de tiuj konsiloj estas ne praktika kaj kaŭzita de ne sufiĉa pripensado kaj provado de l' afero; sed diri, ke la lingvo tute ne estos ŝanĝita, mi tamen ne povas. Cetere, ĉiuj proponoj, kiujn mi ricevas, kune kun mia juĝo pri ili, estos prezentataj al la juĝo de l' publiko aŭ de ia el la jam konataj instruitaj akademioj, se inter tiuj ĉi estos trovita unu, kiu volos preni tiun ĉi laboron. Se ia kompetenta akademio min sciigos, ke ĝi volas preni tiun ĉi laboron, mi tuj sendos al ĝi la tutan materialon, kiu estas ĉe mi, mi fordonos al ĝi la tutan aferon, mi foriros kun la plej granda ĝojo je eterne de l' sceno, kaj el aŭtoro kaj iniciatoro mi fariĝos simpla amiko de l' lingvo internacia, kiel ĉiu alia amiko. Se tamen nenia el la instruitaj akademioj volos preni mian aferon, tiam mi daŭrigos la publikigadon de l' proponoj, sendataj al mi, kaj laŭ mia propra pensado kaj laŭ la pensoj de l' publiko, sendataj al mi pri tiuj proponoj, mi mem antaŭ la fino de l' jaro decidos la finan formon de l' lingvo kaj mi sciigos, ke la lingvo estas preta. 

\mychapbig{II.}

La nombro 10,000,000, pri kiu estas parolita en mia unua verko, ŝajnas al multaj absolute ne ricevebla. La plej granda parto de l' mondo efektive kredeble estos tiel senmova, ke ĝi de si mem ne donos voĉon, malgraŭ ke la afero estas tiel grava kaj la laboro de l' voĉdono tiel malgrandega. Sed se l' amikoj de l' lingvo internacia, anstataŭ timegi la nombron, laboros por la afero kaj kolektos tiom voĉojn, kiom ili povos, tiam la necesa nombro da voĉoj povas esti ricevita en la plej mallonga tempo.

Kiam mi proponis la voĉdonon, mi profunde kredis, ke pli frue aŭ pli malfrue 10,000,000 voĉoj estos kolektitaj. La rezultatoj, kiuj sin montris ĝis hodiaŭ, ankoraŭ plifortigas mian kredon. Sed ni prenu, ke mi fantazias, ke mi eraras, ke mi tro multe esperas,---ke sur la tuta tero ne estos kolektita eĉ unu miliono da voĉoj \ldots{} kion l' afero tiam perdos? Kelkaj konsilas al mi, ke mi forĵetu la voĉdonon aŭ ke mi malgrandigu la nombron da postulataj voĉoj ĝis unu miliono; \glqq{}ĉar\grqq{}, ili diras, \glqq{}danke la fantazian punkton de l' voĉdono, afero per si mem tiel utila, povas fari fiaskon.\grqq{} Sed kie, sinjoroj, vi prenis, ke la veno al celo de l' afero dependas de l' rezultatoj de la voĉdono? Tiuj, kiuj trovas, ke mia lingvo estas inda je lerno, sendas al mi promesojn senkondiĉajn kaj lernas la lingvon sen ia atendo. Sendepende de l' iro de la voĉdono en tiu ĉi lingvo estos eldonataj libroj kaj gazetoj, kaj la afero sin movos antaŭen. La voĉdonon mi proponis sole por tio, ke al la afero povu esti altiritaj per \emph{unu fojo} tutaj \emph{amasoj} da homoj, ĉar mi scias, ke preni ian laboron, eĉ la plej malgrandan, ne ĉiu konsentos, sed helpi aferon tre utilan, kie estas postulata nek laboro, nek mono,---ne multaj malkonsentos, tiom pli, se troviĝos memorigantoj. Mi ripetas: profunde mi kredas, ke pli frue aŭ pli malfrue 10,000,000 voĉoj estos kolektitaj, kaj tiel je unu bela tago ni sciiĝos, ke la lingvo internacia fariĝis tutmonda; sed se eĉ la nombro de l' voĉoj neniam venus al dek milionoj,---la afero pro tio ĉi tute ne estos perdita.

Kelkaj provis montri al mi, ke mia projekto de l' voĉdono estas \emph{matematike} ne ebla; tiel ekzemple unu faris jenan kalkulon: \glqq{}se ni prenos, ke la enskribado de ĉiu promesanto okupos ne pli multe ol unu minuton, kaj vi, forĵetinte ĉian alian laboron, vin okupos sole je tiu ĉi afero, laborante sen ripozo 15 horojn ĉiutage,---tiam la pretigo de la libro de l' voĉoj okupos 30 jarojn, kaj por eldoni ĝin vi bezonos la riĉecon de Krezo!\grqq{} La kalkulo ŝajne estas tute prava kaj povas timigi ĉiun,---tamen se la skribinto de tiu ĉi kalkulo bone pensus pri ĝi, li tre facile ekvidus, ke tie ĉi estas sofismo, kaj se nur efektive estos alsenditaj dek milionoj promesoj, la libron de l' voĉdono oni povos pretigi kaj eldoni en kelkaj monatoj kaj sen iaj riĉecoj de Krezo. Ĉar kiu diras, ke la tuta libro devas esti propramane skribita de unu persono? Ke ĉe ĉiu pli granda afero estas uzata \emph{divido de laboro}, la skribinto tute forgesis! Tiaj \glqq{}timigaj\grqq{} libroj estas eldonataj ĉiutage en granda nombro, kaj tio ne sole ne estas neebla, sed neniu eĉ en tio vidas ion grandegan, mirindan. Se vi kolektos la numerojn de ia ĉiutaga gazeto por unu jaro, vi ricevos libron, kiu laŭ grandeco kaj kosto egalas la elirontan libron de l' voĉoj, kaj laŭ la malfacileco de l' pretigo multe superas mian libron, de kiu la pretigo estas laboro pure meĥanika. Tiel ĉiujare en la mondo estas eldonataj miloj kaj dekmiloj da tiaj \glqq{}neeblaj\grqq{} libroj, kaj tamen neniu el la redaktoroj estas mirindaĵisto. Sinjoroj la kalkulantoj forgesis tiun simplan leĝon, kiun ili povas vidi sur ĉiu paŝo, ke tio, kio ĉe unu homo postulas 30 jarojn, ĉe cent homoj okupos sole 4 monatojn, kaj tio, kio estas neebla por unu persono, estas ludilo por grupo da personoj.

Al ĉiuj amikoj de l' lingvo internacia mi ripetas ankoraŭ mian peton: ne forgesu la promesojn kaj kolektu ilin kie kaj kiom vi povas. Multaj pensas, ke ili ne devas sendi promeson, ĉar \glqq{}la aŭtoro eĉ scias, ke ili ellernos aŭ jam ellernis la lingvon\grqq{}! Sed la promeso estas necesa ne por mi, sed por la statistiko. Se iu eĉ skribis al mi kelkajn leterojn en la lingvo internacia, mi ne povas lin nomi internaciisto, ĝis li ne sendis al mi sian promeson. Ne diru, ke de unu aŭ kelkaj promesoj la grandega nombro ne pleniĝos: ĉiu maro estas kreita de apartaj gutoj, kaj la plej granda nombro devas kaj povas esti ricevita el apartaj unuoj. Memoru, ke se eĉ la esperata nombro estas ne ricevota, vi nenion perdas, sendante la promeson. 

\mychapbig{III.}

La venontajn apartajn pecojn mi donas, ke la lernantoj povu ripeti praktike la regulojn de l' gramatiko internacia kaj kompreni bone la signifon kaj la uzon de l' sufiksoj kaj prefiksoj.
