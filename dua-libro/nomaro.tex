\newgeometry{top=2cm,bottom=2cm,inner=2cm,outer=2cm}
\cleardoublepage
\renewcommand\,{\textquotesingle}
\thispagestyle{plain}
\phantomsection
\addcontentsline{toc}{chapter}{Nomaro de verkoj}
{\centering\scalebox{2}[1]{\large\didone\bf NOM\,AR\,O}\par}

{\centering de l' verk\,o\,j pri la lingv\,o inter\,naci\,a (Esperant\,a),\\
kiu\,j el\,ir\,is ĝis Februar\,o 1889.\par}

\small
\renewcommand*{\arraystretch}{1.15}
\begin{longtblr}[theme=plain,label=none]{
  colspec={Q[r,h]Q[co=1,j,f]Q[c,f]},
  stretch=0
}
\scalebox{2}[0.8]{\textnumero} &
\hspace{1em}\scalebox{1.5}[0.8]{\didone\bf Nom\,o de l' verk\,o:} &
\scalebox{1.5}[0.8]{\didone{\textbf{Kosto:}}} \\

\SetCell[r=2]{h} 1.
&
\SetCell[r=2]{h} \hangindent=0.7em \textit{D\textsuperscript{ro} Esperanto.} Lingv\,\-o in\-ter\,\-na\-ci\,a.   Antaŭ\,\-par\-ol\,\-o kaj plen\,\-a lern\,\-o\-\,libr\,o en la lingv\,\-o rus\,\-a \Dotfill
&
\SetCell[r=2]{f} {\rotatebox[origin=c]{90}{\parbox{2.5em}{\scriptsize\it rubl\,o\,j\\ \centering ko-\\ pek\,o\,j}} \\ 0,\thinspace{}15} \\

 & & \\

2. & — en la lingv\,o pol\,a \Dotfill & 0,\thinspace{}15 \\

3. & — en la lingv\,o franc\,a \Dotfill & 0,\thinspace{}20 \\

4. & — en la lingv\,o german\,a \Dotfill & 0,\thinspace{}20 \\

5. & — en la lingv\,o angl\,a \Dotfill & 0,\thinspace{}20 \\

6. & \textit{D\textsuperscript{ro} Esperanto.} {Mal\,grand\,a vort\,ar\,o \\ \hphantom{\buffer}inter\,naci\,a-rus\,a \Dotfill} & 0,\thinspace{}03 \\

7. & \buffer{}inter\,naci\,a-pol\,a \Dotfill & 0,\thinspace{}03 \\

8. & \buffer{}inter\,naci\,a-franc\,a \Dotfill & 0,\thinspace{}03 \\

9. & \buffer{}inter\,naci\,a-german\,a \Dotfill & 0,\thinspace{}03 \\

10. & \buffer{}inter\,naci\,a-angl\,a \Dotfill & 0,\thinspace{}03 \\

11. & 
\hangindent=0.7em \textit{D\textsuperscript{ro} Esperanto.} Du\,a Libr\,o de l’ ling\-v\,o inter\,\-na\-ci\,a (skrib\,it\,a {\didone\bf inter\,naci\,e}) \Dotfill & 0,\thinspace{}25 \\

12. & \hangindent=0.7em
— Al\,don\,o al la Du\,\-a Libr\,o de l’ ling\-v\,o inter\,\-naci\,a. ({\didone\bf inter\,naci\,e}) \Dotfill & 0,\thinspace{}10 \\

13. & \hangindent=0.7em \textit{Hanez.} Safah achath lekulanu (lern\,o\,-libr\,\-o de l' lingv\,o inter\,naci\,a Esper\-ant\,a \newline en la lingv\,o hebre\,a)\Dotfill & 0,\thinspace{}20 \\

14. & \hangindent=0.7em
\textit{A. Grabowski.} La Neĝ\,a Blov\,ad\,o. Ra\-kont\,o de A. Puŝkin. ({\didone\bf inter\,naci\,e}) \Dotfill & 0,\thinspace{}15 \\

15. & \hangindent=0.7em
\textit{L. Einstein.} La lingvo inter\-nacia als beste Lö\-sung des inter\-nati\-on\-alen Welt\-spra\-che-\-Prob\-lems. Vor\-wort, Gram\-ma\-tik und Styl nebst Stamm\-wört\-er-\-Ver\-zeich\-niss nach dem Ent\-wurf des pseu\-do\-nymen Dr. Esper\-anto \Dotfill & 0,\thinspace{}50 \\

16. & \hangindent=0.7em
\textit{N. N.} Rus\,a traduk\,o de la Du\,a Libr\,o de l’ lingv\,o inter\,naci\,a \Dotfill & 0,\thinspace{}25 \\

17. & \hangindent=0.7em
\textit{D\textsuperscript{ro} Esperanto.} Plen\,a vort\,\-ar\,\-o rus\,\-a-in\-ter\,na\-ci\,a \Dotfill & 1,\thinspace{}— \\

18. & \hangindent=0.7em
\textit{N. N.} Rus\,a traduk\,o de la \glqq{}Al\,\-don\,\-o al la Du\,\-a Libr\,\-o\grqq{} \Dotfill & 0,\thinspace{}10 \\

19. & \hangindent=0.7em
\textit{A. Grabowski.} La Gefratoj. Komedio de Göthe. ({\didone\bf inter\,naci\,e}) \Dotfill & 
 0,\thinspace{}15 \\

20. & \hangindent=0.7em
\textit{L. Einstein.} Weltsprachliche Zeit-und Streitfragen. I. Vo\-la\-pük und Ling\-vo in\-ter\-na\-ci\-a \Dotfill & 0,\thinspace{}25

\end{longtblr}

\thispagestyle{plain}

{\centering\rule{3cm}{0.4pt}\par}

Ĉiu\,j supr\,e skrib\,it\,a\,j verk\,o\,j pov\,as est\,i ricev\,at\,a\,j en la libr\,ej\,o\,j kaj ankaŭ ĉe \textit{\sansfont{d-r\,o L. Zamenhof}} \textit{(}\textit{\sansfont{Varsovi\thinspace{}\,\thinspace{}o,}} strat\,o Przejazd N. 9\textit{)}. Anstataŭ mon\,o oni pov\,as send\,i sign\,o\,j\,n de poŝt\,o (de ĉia land\,o). Por la poŝt\,a trans\,send\,o oni dev\,as al\,don\,i po 20\% de la kost\,o de l’ verk\,o\,j. Kiu aĉet\,as ne mal\,pli ol por unu rubl\,o, tiu por la trans\,send\,o ne pag\,as.

{\centering\rule{3cm}{0.4pt}\par}

En la komenc\,o de ĉiu monat\,o est\,as pres\,at\,a nov\,a nom\,ar\,o de ĉiu\,j verk\,o\,j pri la lingv\,o inter\,naci\,a (de kiu ajn ili est\,as el\,don\,it\,a\,j), kiu\,j el\,ir\,is de la komenc\,o mem ĝis tiu monat\,o. \leftpointright{}~{\didone\bf Kiu dezir\,as regul\,e ĉiam sci\,i pri la progres\,ad\,o de l’ lingv\,o inter\,naci\,a,}\rightpointleft{} tiu pov\,as send\,i 30 kopek\,o\,j\,n por jar\,o al la el\,don\,ant\,o de la dir\,it\,a\,j nom\,ar\,o\,j {\didone\bf (L. Zamenhof, Varsovi\,o}, strat\,o Przejazd N. 9), kaj tiam li akurat\,e ricev\,ad\,os per la poŝt\,o ĉiu\,n nov\,a\,n nom\,ar\,o\,n tuj kiam ĝi est\,os pres\,it\,a.

{\centering\pgfornament[width=0.3\textwidth]{83}\par}

\normalsize
\newpage