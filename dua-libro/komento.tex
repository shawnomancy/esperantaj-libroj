%
% Kolofono
%
\kolofono

\small Kiel mia \XeLaTeX{}-versio de \emph{Unua Libro}, mi provis rekrei la 19e-jarcentan «senton» de \emph{Dua Libro} tiel fidele kiel eble, sed denove, mi ne perfekte povis imiti ĉiun tiparon, ornaman lineon, tiparan grandon, k.t.p.

Denove mi elektis A5 por la moderna standarda papergrando plej sama kiel la originala.

La unua eldono de \emph{Dua Libro} estis publikigita dum la jaro, je kiu Zamenhof ankoraŭ ricevis proponitajn plibonigojn.  Kiam \emph{Unua Libro} unue estis publikigita, la korelativoj (tabelvortoj) pri tempo finas per «-an» anstataŭ «-am».  Do, la unua eldono de \emph{Dua Libro} uzis tiujn originalajn formojn (\emph{kian, tian, ian, ĉian, nenian}).  Por ke la moderna Esperantisto ne konfuziĝos, mi anstataŭ uzis la modernajn formojn (\emph{kiam, tiam, iam, ĉiam, neniam}).  Je ĉiuj fojoj krom unu, mi povis decidi, ĉu la intencita vorto estis la korelativo de tempo (nune «-am»), aŭ la akuzativa formo de la korelativa adjektivo (nune «-an»), kaj mi uzis la konvenan vorton.  Je la unu fojo, en kiu ĝi ne klaras, mi faris piednoton kaj forlasis la originalan vorton (\emph{ian}) neŝanĝita.\\[1ex] 

{\setlength{\parindent}{0em}
Shawn \fsc{Knight} (angle elparolata \emph{ŝan najt})\\
\hodiau}
