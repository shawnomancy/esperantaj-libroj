\mysect{II.}

The solution of the second problem was effected thus:

1) I introduced a complete dismemberment of ideas into independent words, so that the whole language consists, not of words in different states of grammatical inflexion, but of unchangeable words. If the reader will turn to one of the pages of this book written in my language, he will perceive that each word always retains its original unalterable form, ---namely, that under which it appears in the vocabulary. The various grammatical inflexions, the reciprocal relations of the members of a sentence, are expressed by the junction of immutable syllables. But the structure of such a synthetic language being altogether strange to the chief European nations, and consequently difficult for them to become accustomed to, I have adapted this principle of dismemberment to the spirit of the European languages, in such a manner that anyone learning my tongue from grammar alone, without having previously read this introduction, --- which is quite unnecessary for the learner, --- will never perceive that the structure of the language differs in any respect from that of his mother-tongue. So, for example, the derivation of \emph{frat\,in\,o}, which is in reality a compound of \emph{frat} “child of the same parents as one’s self”, \emph{in} “female”, \emph{o} “an entity”, “that which exists”, i.~e., “that which exists as a female child of the same parents as one’s self” = “a sister”,--- is explained by the grammar thus: the root for “brother” is \emph{frat}, the termination of substantives in the nominative case is \emph{o}, hence \emph{frat\,o} is the equivalent of “brother”; the feminine gender is formed by the suffix \emph{in}, hence \emph{frat\,in\,o} = “sister”. (The little strokes, between certain letters, are added in accordance with a rule of the grammar, which requires their insertion between each component part of every complete word). Thus the learner experiences no difficulty, and never even imagines that what he calls terminations, suffixes, etc.,---are complete and independent words, which always keep their own proper significations, whether placed at the beginning or end of a word, in the middle, or alone. The result of this construction of the language is, that everything written in it can be immediately and perfectly understood by the help of the vocabulary, --- or even almost without it, --- by anyone who has not only not learnt the language before, but even has never heard of its very existence. Let me illustrate this by an example: --- I am amongst Englishmen, and have not the slightest knowledge of the English language; I am absolutely in need of making myself understood, and write in the international tongue, may be, as follows:

    \emph{Mi ne sci\,as ki\,e mi las\,is la baston\,o\,n; ĉu vi ĝi\,n ne vid\,is?}

I hold out to one of the strangers an International--English vocabulary, and point to the title, where the following sentence appears in large letters: “Everything written in the international language can be translated by the help of this vocabulary. If several words together express but a single idea, they are written as one word, but separated by commas; e.~g., \emph{frat\,in\,o}, though a single idea is yet composed of three words which must be looked for separately in the vocabulary”. If my companion has never heard of the international language he will probably favour me at first with a vacant stare, will then take the paper offered to him, and searching for the words in the vocabulary, as directed, will make out something of this kind:



\begingroup
\footnotesize
\renewcommand{\arraystretch}{1}
\begin{longtblr}[theme=plain,label=none]{
  colspec={cclclcc},
  columns={rightsep=0.5ex},
  column{1}={font=\itshape},
  column{2}={leftsep=0ex},
  column{3}={font=\itshape},
  stretch=0,
}

Mi & \tlba & mi & = & I & \trba & I \\

ne & \tlba & ne & = & not & \trba & not \\

\SetCell[r=2]{c} sci\,as & \SetCell[r=2]{c} \tlbb & sci & = & know & \SetCell[r=2]{c} \trbb & \SetCell[r=2]{c} do know \\
 & & as & = & sign of the present tense \\
 
kie & \tlba & kie & = & where & \trba & where \\

mi & \tlba & mi & = & I & \trba & I \\

\SetCell[r=2]{c} las\,is & \SetCell[r=2]{c} \tlbb & las & = & leave & \SetCell[r=2]{c} \trbb & \SetCell[r=2]{c} have left \\
 & & is & = & sign of the past tense \\
 
la & \tlba & la & = & the & \trba & the \\

\SetCell[r=3]{c} baston\,o\,n; & \SetCell[r=3]{c} \tlbc & baston & = & stick & \SetCell[r=3]{c} \trbc & \SetCell[r=3]{c} stick; \\
 & & o & = & sign of a substantive \\
 & & n & = & sign of the objective case \\
 
\SetCell[r=2]{c} ĉu & \SetCell[r=2]{c} \tlbb & ĉu & = & whether, if, & \SetCell[r=2]{c} \trbb & \SetCell[r=2]{c} whether \\
 & & & & employed in questions & & \\

vi & \tlba & vi & = & you, thou & \trba & you \\

\SetCell[r=2]{c} ĝi\,n & \SetCell[r=2]{c} \tlbb & ĝi & = & it, this & \SetCell[r=2]{c} \trbb & \SetCell[r=2]{c} it \\
 & & n & = & sign of the objective case \\
 
ne & \tlba & ne & = & not & \trba & not \\

\SetCell[r=2]{c} vid\,is? & \SetCell[r=2]{c} \tlbb & vid & = & see & \SetCell[r=2]{c} \trbb & \SetCell[r=2]{c} have seen? \\
 & & is & = & sign of the past tense \\
 
\end{longtblr}
\endgroup

And thus the Englishman will easily understand what it is I desire. If he wishes to reply, I show him an English--International vocabulary, on which are printed these words: “To express anything by means of this vocabulary, in the international language, look for the words required, in the vocabulary itself; and for the terminations necessary to distinguish the grammatical forms, look in the grammatical appendix, under the respective headings of the parts of speech which you desire to express”. Since the explanation of the whole grammatical structure of the language is comprised in a few lines,---as a glance at the grammar will show,---the finding of the required terminations occupies no longer time than the turning up a word in the dictionary.

I would now direct the attention of my readers to another matter, at first sight a trifling one, but, in truth, of immense importance. Everyone knows the impossibility of communicating intelligibly with a foreigner, by the aid of even the best of dictionaries, if one has no previous acquaintance with the language. In order to find any given word in a dictionary, we must know its derivation, for when words are arranged in sentences, nearly every one of them undergoes some grammatical change. After this alteration, a word often bears not the least resemblance to its primary form, so that without knowing something of the language beforehand, we are able to find hardly any of the words occurring in a given phrase, and even those we do find will give no connected sense. Suppose, for example, I had written the simple sentence adduced above, in German: \glqq Ich weiss nicht wo ich den Stock gelassen habe; haben Sie ihn nicht gesehen?\grqq{} Anyone who did not speak or understand German, after searching for each word separately in a dictionary, would produce the following farrago of nonsense: “I; white; not; where; I; --- ; stick; dispassionate; property; to have; she, they, you; --- ; not; --- ?” I need scarcely point out that a lexicon of a modern language is usually a tome of a certain bulk, and the search for any number of words one by one is in itself a most laborious undertaking, not to speak of the different significations attaching to the same word amongst which there is but a bare possibility of the student selecting the right one. The international vocabulary, owing to the highly synthetic structure of the language, is a mere leaflet, which one might carry in one’s note-book, or the waistcoat-pocket. Granted that we \emph{had} a language with a grammar simplified to the utmost, and whose every word had a definite fixed meaning, the person addressed would require not only to have beforehand some knowledge of the grammar, to be able, even with the vocabulary at hand, to understand anything addressed to him, but would also need some previous acquaintance with the vocabulary itself, in order to be able to distinguish between the primitive word and its grammatically-altered derivatives. The utility, again, of such a language would wholly depend upon the number of its adepts, for when sitting, for instance, in a railway-carriage, and wishing to ask a fellow-traveller, “How long do we stop at ---?”, it is scarcely to be expected that he will undertake to learn the grammar of the language before replying! By using, on the other hand, the international language, we are set in possibility of communicating directly with a person of any nationality, even though he may never have heard of the existence of the language before.

Anything whatever, written in the international tongue, can be translated, without difficulty, by means of the vocabulary alone, no previous study being requisite. The reader may easily convince himself of the truth of this assertion, by experimenting for himself with the specimens of the language appended to this pamphlet. A person of good education will seldom need to refer to the vocabulary, a linguist scarcely at all.

Let us suppose that you have to write to a Spaniard, who neither knows your language nor you his. You think that probably he has never heard of the international tongue. --- No matter, write boldly to him in that language, and be sure he will understand you perfectly. The complete vocabulary required for everyday use, being but a single sheet of paper, can be bought for a few pence, in any language you please, easily enclosed in the smallest envelope, and forwarded with your letter. The person to whom it is addressed will without doubt understand what you have written, the vocabulary being not only a clue to, but a complete explanation of your letter. The wonderful power of combination possessed by the words of the international language renders this lilliputian lexicon amply sufficient for the expression of every want of daily life; but words seldom met with, technical terms, and foreign words familiar to all nations, as, “tobacco”, “theatre”, “fabric”, etc., are not included in it. If such words, therefore, are needed, and it is impossible to express them by some equivalent terms, the larger vocabulary must be consulted.

2) It has now been shown how, by means of the peculiar structure of the international tongue, any one may enter into an intelligible correspondence with another person of a different nationality. The sole drawback, until the language becomes more widely known, is the necessity under which the writer is placed of waiting until the person addressed shall have analysed his thoughts. In order to remove this obstacle, as far as practicable, at least for persons of education, recourse was had to the following expedient. Such words as are common to the languages of all civilised peoples, together with the so-called “foreign” words, and technical terms, were left unaltered. If a word has a different sound in different languages, that sound has been chosen which is common to at least two or three of the most important European tongues, or which, if found in one language only, has become familiar to other nations. When the required word has a different sound in every language, some word was sought for, having only a relative likeness in meaning to the other, or one which, though seldom used, is yet well-known to the leading nations, e.~g., the word for ``near'' is different in every European language, but if one consider for a moment the word ``proximus'' (nearest), it will be noticed that some modified form of the word is in use in all important tongues. If, then, I call ``near'', \emph{proksim}, the meaning will be apparent to every educated man. In other emergencies words were drawn from the Latin, as being a quasi-international language. Deviations from these rules were only made in exceptional cases, as for the avoidance of homonyms, simplicity of orthography, etc. In this manner, being in communication with a European of fair education, who has never learnt the international tongue, one may make sure of being immediately understood, without the person addressed having to refer continually to the vocabulary.

In order that the reader may prove for himself the truth of all that has been set forth above, a few specimens of the international language are subjoined.\footnote{In correspondence with persons who have learnt the language, as well as in works written for them exclusively, the commas, separating parts of words, are omitted.}
