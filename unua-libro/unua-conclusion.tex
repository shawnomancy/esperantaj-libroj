\mysect{III.}

I have now completed my analysis of the more remarkable features of my international language. I have shown the advantages to be derived from a study of it, and proved that its ultimate success is altogether independent of the opinions that may be formed as to its right to the title “international”. For even should the language never come into general use, it gives to every one who \emph{has} learned it, the possibility of being understood by foreigners, if only they be able to read and write. But my tongue has yet another object; not content with internationality, it aims at universality, and aspires to being \emph{spoken} by the majority of educated people. To count on the aid of the public in a scheme of this nature would indeed be to build on a tottering, --- nay rather, an imaginary, --- foundation. The larger part of the public does not care to aid anyone, it prefers to have its wishes gratified without inconvenience to itself. On this account I made my best endeavours to discover some means of accomplishing my object, independently of the help of the public. One of my plans, of which I shall now speak more at large, is a kind of “universal vote”.

If the reader consider all that has been said above, he must come to the conclusion that the study of the international language is practically useful, and completely remunerates the learner for the small amount of trouble he has to expend on it. For my own part, I am naturally wishful that the whole of mankind should take up my language, but I had rather be prepared for the worst, than form too sanguine anticipations. I suppose therefore, that, just at first, very few will consider my language worth the learning, so far as practical usefulness is concerned, and for abstract principles no one will lose even a single hour.

Most of my readers will, either pay not the slightest attention to my proposition, or, doubting whether the language be of any use, never “screw up their courage to the sticking-point” of learning it, fearing that they may be dubbed “dreamers”, a sobriquet dreaded by most people more than fire. What, then, is to be done, to dispose this mass of indifferent and undecided beings to master the international language? Could we, in imagination, look for a moment into the mind of each of these indifferent ones, we should find their thoughts to be taking somewhat of the following form. In principle, no one has anything to oppose to the introduction of an international dialect; on the contrary, all would give it their fullest approval, but each wishes to see the greater part of the civilized world able to speak the language, and himself able to comprehend it, without any preliminary “wearisome bitterness of learning”, on his own part. \emph{Then}, of course, even the most indifferent would set to work, because to shirk the small amount of labour necessary for learning a language possessed of such valuable qualities, and above all, considered \emph{``the thing''} by all the educated, would be regarded as simple stupidity.

In order to supply a language ready for immediate use, without any one having to initiate the study, and to see on every hand people either already proficient in the tongue, or having promised to take it up, we must proceed somewhat in the following manner. Doubtless this little book will be scattered through various countries, and fall into the hands of various readers. I do not ask any of my readers to spend time, labour, or money on the subject now brought to their notice. I merely beg of you, the present reader of the pamphlet, to take up your pen for a moment, fill in one of the appended \emph{\glqq Promes\,o\,j\grqq} (below) and send it to me (Dr. Esperanto, $\mathrm{^c\!/\!_o}$ Dr. L. Samenhof. Warsaw, Poland). The \emph{\glqq Promes\,o\grqq} is to this effect:

\begin{adjustwidth}{1cm}{1cm}
\hspace{1.5em} “I, the undersigned, promise to learn the international language, proposed by Dr. Es\-per\-anto, if it shall be shown that ten million similar promises have been publicly given”.
\end{adjustwidth}

If you have any objections to make to the present form of the language, strike out the words of the promise, and write \emph{\glqq kontraŭ\grqq} (against), beneath them. If you undertake to learn the language unconditionally, i.~e., without reference to the number of other students, strike out the latter words of the \glqq Promes\,o\grqq, and write \emph{\glqq{}sen\,kondiĉ\,e\grqq{}}, (unconditionally). On the back of the promise write name and address. The signing of this promise lays no obligations upon the person signing, and does not bind him to the smallest sacrifice or work. It merely puts him under an obligation to study the language, when ten million other persons shall be doing the same. When that time arrives, there will be no talking about ``sacrifice'', everyone will be ready to study the language, without having signed any promises.

On the other hand, every person signing one of these \glqq{}Promes\,o\,j\grqq{}, will, --- without any greater inconvenience to himself than dipping a pen in ink, --- be hastening on the realization of the traditional ideal of mankind, the universal language. When the number of promises has reached ten millions, a list of the names of those who have signed will be published, and with it, the question of an international language --- decided.

Nothing actually \emph{prevents} people from inducing their friends and acquaintances to sign a promise in any cause, yet how few, as a fact, ever do sign anything, be the object ever so important and advantageous to mankind. More especially, when, as in the present instance, the act of signing, while contributing to the realization of a sublime ideal, at the same time requires no moral nor material sacrifice, can one see no very clear grounds for a refusal.

Doubtless, no one has anything to say, in general, against the introduction of an international language; but, if anyone does not approve of the present form of the language, by all means let him send me, instead of his “Promise”, his “Protest”. For it is, manifestly, the duty of every person able to read and write, of every age, sex, or profession, to give his opinion in this great undertaking; the more so, as it requires no greater sacrifice than that of a few moments for filling in the promise, and a few pence for sending it to me.

I would here beg of all editors of newspapers and magazines to make known the cause to their readers, and at the same time, I would request \emph{my} readers to mention the subject to all their friends.

I need not say any more. I am not so conceited as to suppose that my language is so perfect as to be incapable of improvement, but I make bold to think that I have satisfied all the conditions required in a language claiming to be styled “international”. It is only after having solved successfully all the problems I had proposed to myself,---concerning the more important of which only, I have been able to speak above, owing to the small compass of this pamphlet,---and after many years spent in a careful study of the subject that I venture to appear in public. I am but human; I may have erred, I may have committed unpardonable faults. I may even have omitted to give to my language the very thing most important to it. For these reasons, before printing complete vocabularies and bringing out books and magazines, I lay my work before the public, for the space of one year, addressing myself to the whole intelligent world with the earnest request to send me opinions on the proposed international language. I invite everyone to communicate with me as to the changes, corrections, etc., which he deems advisable. All such observations sent to me, I will gratefully make use of, if they appear really advantageous, and at the same time, not subversive of the fundamental principles of the structure of the language:---that is to say, simplicity, and adaptability to international communication whether adopted \emph{universally} or not.

At the end of the alloted time, an abstract of the proposed changes will be published and the language will receive its final form. But if, even then, anyone should find the language not altogether satisfactory to himself, he should not forget that the language is by no means proof against all further changes, only that the right of alteration will be no longer the author’s personal privilege, but that of an academy of the tongue.

It is no easy task to invent an international language, but it is a still less easy one to persuade the public to make use of it. Hence, it is of the utmost importance that every possible effort be made for its furtherance. When the form of the language has been decided, and the language itself has come into general use, a special academy can introduce, --- gradually and imperceptibly, --- all necessary changes, even should the result be a total alteration of the form of the language. On this account, I would pray those of my readers, who may be, for whatever reasons, dissatisfied with my language, to send in their protests only in the event of their having serious cause for it, such as the finding in the language objectionable features, unalterable in the future.

This little work, which has cost much labour and health, I now commend to the kindly attention of the public, hoping that all, to whom the public weal is dear, will aid me to the best of their ability. Circumstances will show each one in what way he can be of use; I will only direct the attention of all friends of the international language, to that most important object, towards which all eyes must be turned, the success of the voting. Let each do what he can, and in a short time we shall have, that which men have been dreaming of so long, --- \emph{``A Universal Tongue''}.

\sectionline

\emph{NB.} The author requests his reader to fill in one of the ``Promises'' on the following page, and send it to him, and to distribute the others amongst friends and acquaintances for the same purpose.

\vspace{1ex}

Author's Address:

\vspace{1ex}

\hspace{1.5em} Dr. Esperanto,

\vspace{1ex}

\hspace{3em} $\mathrm{^c\!/\!_o}$ Dr. L. Samenhof,

\vspace{1ex}

\hspace{4.5em} Warsaw,

\vspace{1ex}

\hspace{6em} Russ-Poland.

\sectionline
