\mysect{I.}

The first of the problems was solved in the following manner:

1) I simplified the grammar to the utmost, and while, on the one hand, I carried out my object in the spirit of the existing modern languages, in order to make the study as free from difficulties as possible, on the other hand I did not deprive it of clearness, exactness, and flexibility. My whole grammar can be learned perfectly in \inbold{one hour}. The immense alleviation given to the study of a language, by such a grammar, must be self-evident to everyone.

2) I established rules for the formation of new words, and at the same time, reduced to a very small compass the list of words absolutely necessary to be learned, without, however, depriving the language of the means of becoming a rich one. On the contrary, thanks to the possibility of forming from one root-word any number of compounds, expressive of every conceivable shade of idea, I made it the richest of the rich amongst modern tongues. This I accomplished by the introduction of numerous prefixes and suffixes, by whose aid the student is enabled to create new words for himself, without the necessity of having previously to learn them, e.~g.

1) The prefix \emph{mal} denotes the direct opposite of any idea. If, for instance, we know the word for “good”, \emph{bon\,a}, we can immediately form that for “bad”, \emph{mal\,bon\,a}, and hence the necessity of a special word for “bad” is obviated. In like manner, \emph{alt\,a}, “high”, “tall”, \emph{mal\,alt\,a,} “low”, “short”; \emph{estim\,i,} “to respect”, \emph{mal\,estim\,i}, “to despise”, etc. Consequently, if one has learned this single word \emph{mal} he is relieved of leaning a long string of words such as “hard” (premising that he knows “soft”), “cold”, “old”, “dirty”, “distant”, “darkness”, “shame”, “to hate”, etc., etc.

2) The suffix \emph{in} marks the feminine gender, and thus if we know the word “brother”, \emph{frat\,o}, we can form “sister”, \emph{frat\,in\,o}: so also, “father”, \emph{patr\,o}; “mother”, \emph{patr\,in\,o}. By this device words like “grandmother”, “bride”, “girl”, “hen”, “cow”, etc., are done away with.

3) The suffix \emph{il} indicates an instrument for a given purpose, e.~g., \emph{tranĉ\,i}, “to cut”, \emph{tranĉ\,il\,o}, “a knife”; so words like “comb”, “axe”, “bell”, etc., are rendered unnecessary.

In the same manner are employed many other affixes, --- some fifty in all, --- which the reader will find in the vocabulary at end of this tractate.\footnote{To facilitate the finding of these affixes they are entered in the vocabulary as separate words.} Moreover, as I have laid it down as a general rule, that every word already regarded as international, --- the so-called “foreign” words, for example, --- undergoes no change in my language, except such as may be necessary to bring it into conformity with the international orthography, innumerable words become superfluous, e.~g., “locomotive”, “telegraph”, “nerve”, “temperature”, “centre”, “form”, “public”, “platinum”, “figure”, “waggon”, “comedy”, and hundreds more.

By the help of these rules, and others, which will be found in the grammar, the language is rendered so exceedingly simple that the whole labour in learning consists in committing to memory some 900 words, --- which number includes all the grammatical inflexions, prefixes, etc. --- With the assistance of the rules given in the grammar, any one of ordinary intellectual capacity, may form for himself all the words, expressions, and idioms in ordinary use. Even these 900 words, as will be shown directly, are so chosen, that the learning them offers no difficulty to a well-educated person.

Thus the acquirement of this rich, mellifluous, universally-comprehensible language, is not a matter of years of laborious study, but the mere light amusement of a few days. 
