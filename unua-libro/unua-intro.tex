\chapter*{\protect\scalebox{0.8}[1]{INTRODUCTION.}}
\addcontentsline{toc}{chapter}{Introduction.}
\fancyhead[C]{--- \thepage{} ---}

\begin{center}
\rule{5em}{0.4pt}
\end{center}

The reader will doubtless take up this little work with an incredulous smile, supposing that he is about to peruse the impracticable schemes of some good citizen of Utopia. I would, therefore, in the first place, beg of him to lay aside all prejudice, and treat seriously and critically the question brought before him. 

I need not here point out the considerable importance to humanity of an international language,---a language unconditionally accepted by everyone, and the common property of the whole world. How much time and labour we spend in learning foreign tongues, and yet when travelling in foreign countries, we are, as a rule, unable to converse with other human beings in their own language. How much time, labour, and money are wasted in translating the literary productions of one nation into the language of another, and yet, if we rely on translations alone, we can become acquainted with but a tithe of foreign literature.

Were there but an international language, all translations would be made into it alone, as into a tongue intelligible to all, and works of an international character would be written in it in the first instance.

The Chinese wall dividing literatures would disappear, and the works of other nations would be as readily intelligible to us as those of our own authors. Books being the same for everyone, education, ideals, convictions, aims, would be the same too, and all nations would be united in a common brotherhood. Being compelled, as we now are, to devote our time to the study of several different languages, we cannot study any of them sufficiently well, and there are but few persons who can even boast a complete mastery of their mother-tongue; on the other hand, languages cannot progress towards perfection, and we are often obliged, even in speaking our own language, to borrow words and expressions from foreigners, or to express our thoughts inexactly.

How different would the case be, had we but two languages to learn; we should know them infinitely better, and the languages themselves would grow richer, and reach a higher degrees of perfection than is found in any of those now existing. And yet, though language is the prime motor of civilisation, and to it alone we owe the having raised ourselves above the level of other animals, difference of speech is a cause of antipathy, nay even of hatred, between people, as being the first thing to strike us on meeting. Not being understood we keep aloof, and the first notion that occurs to our minds is, not to find out whether the others are of our own political opinions, or whence their ancestors came from thousands of years ago, but to dislike the strange sound of their language. Any one, who has lived for a length of time in a commercial city, whose inhabitants were of different unfriendly nations, will easily understand what a boon would be conferred on mankind by the adoption of an international idiom, which, without interfering with domestic affairs or the private-life of nations, would play the part of an official and commercial dialect, at any rate in countries inhabited by people of different nationalities.

The immense importance, which it may well be imagined, an international language would acquire in science, commerce, etc., I will not here expatiate on: whoever has but once bestowed a thought on the subject will surely acknowledge that no sacrifice would be too great, if by it we could obtain a universal tongue. It is, therefore, imperative that the slightest effort in that direction should be attended to. The best years of my life have been devoted to the momentous cause which I am now bringing before the public, and I hope that, on account of the importance of the subject, my readers will peruse this pamphlet attentively to the end.

I shall not here enter upon an analysis of the various attempts already made to give the public a universal language, but will content myself with remarking that these efforts have amounted, either to a short system of mutually-intelligible signs, or to a natural simplification of the grammar of existing modern languages, with a change of their words into arbitrarily-formed ones. The attempts of the first category were quickly seen to be too complicated for practical use, and so faded into oblivion; those of the second were, perhaps, entitled to the name of “languages”, but certainly not “international” languages. The inventors called their tongues “universal”, I know not why, possibly, because no one in the whole world except themselves could understand a single word, written or spoken in any of them. If a language, in order to become universal, has but to be named so, then, forsooth, the wish of any single individual can frame out of any existing dialect a universal tongue. As these authors na\"{i}vely imagined that their essays would be enthusiastically welcomed and taken up by the whole world, and as this unanimous welcome is precisely what the cold and indifferent world declines to give, when there is no chance of realising any immediate benefit, it is not much to be marvelled at, if these brilliant attempts came to nothing. The greater part of the world was not in the slightest degree interested in the prospect of a new language, and the persons who really cared about the matter thought it scarcely worth while to learn a tongue which none but the inventor could understand. When the whole world, said they, has learnt this language, or at least several million people, we will do the same. And so a scheme, which had it but been able to number some thousands of adepts before its appearance in public, would have been enthusiastically hailed, came into the world an utter fiasco. If the \glqq{}Volapük\grqq{}, one of the latest attempts at a universal tongue, has indeed its adepts, it owes its popularity solely to the idea of its being a “universal language”, and that idea has in itself something so attractive and sublime, that true enthusiasts, leaders in every new discovery, are ready to devote their time, in the hope that they may, perchance, win the cause.

But the number of enthusiasts, after having risen to a certain number, will remain stationary\footnote{One cannot, of course, reckon the number of those who learned the language as equal to the number of instruction-books sold.} and as the unfeeling and indifferent world will never consent to take any pains in order to speak with the few, this attempt will, like its predecessors, disappear without having achieved any practical victory.

I have always been interested in the question of a universal language, but as I did not feel myself better qualified for the work than the authors of so many other fruitless attempts, I did not risk running into print, and merely occupied myself with imaginary schemes and a minute study of the problem. At length, however, some happy ideas, the fruits of my reflections, incited me to further work, and induced me to essay the systematic conquest of the many obstacles, which beset the path of the inventor of a new rational universal language. As it appears to me that I have almost succeeded in my undertaking, I am now venturing to lay before the critical public, the results of my long and assiduous labours.

The principal difficulties to be overcome were:

1) To render the study of the language so easy as to make its acquisition mere play to the learner.

2) To enable the learner to make direct use of his knowledge with persons of any nationality, whether the language be universally accepted or not; in other words, the language is to be directly a means of international communication.

3) To find some means of overcoming the natural indifference of mankind, and disposing them, in the quickest manner possible, and \emph{en masse}, to learn and use the proposed language as a living one, and not only in last extremities, and with the key at hand.

Amongst the numberless projects submitted at various times to the public, often under the high-sounding but unaccountable name of “universal languages”, no one has solved at once more than \inbold{one} of the above-mentioned problems, and even that but partially. (Many other problems, of course, presented themselves, in addition to those here noticed, but these, as being of but secondary importance, I shall not in this place discuss).

Before proceeding to enlighten the reader as to the means employed for the solution of the problems, I would ask of him to reconsider the exact significance of each separately, so that he may not be inclined to carp at my methods of solution, merely because they may appear to him perhaps too simple. I do this, because I am well aware that the majority of mankind feel disposed to bestow their consideration on any subject the more carefully, in proportion as it is enigmatical and incomprehensible. Such persons, at the sight of so short a grammar, with rules so simple, and so readily intelligible, will be ready to regard it with a contemptuous glance, never considering the fact, --- of which a little further reflection would convince them, --- that this simplification and bringing of each detail out of its original complicated form into the simplest and easiest conceivable, was, in fact, the most insuperable obstacle to be coped with. 
