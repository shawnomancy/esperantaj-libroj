\cleardoublepage
\titlespacing*{\chapter}{0pt}{0pt}{0pt}
\titleformat{\section}[display]{\centering}{}{0pt}{}
\chapter*{\tuscan COMPLETE GRAMMAR}

\begin{center}
\scalebox{0.7}[1]{\LARGE OF THE INTERNATIONAL LANGUAGE.}
\addcontentsline{toc}{chapter}{Complete Grammar of the International Language}
\sectionline

%
% Alphabet
%
\gramsect{A}{The Alphabet}

\begin{tabu} to\textwidth{+Y@{}ZY@{}ZY@{}ZY@{}ZY@{}ZY}
\rowfont\LARGE A a, & B b, & C c, & Ĉ ĉ, & D d, & E e,  \\ 
\rowfont\footnotesize \emph{a} as in ``last'' & \emph{b} as in ``be'' & \emph{ts} as in ``wits'' & \emph{ch} as in ``church'' & \emph{d} as in ``do'' & \emph{e} as in ``make'' \\[2ex]
\rowfont\LARGE F f, & G g, & Ĝ ĝ, & H h, & Ĥ ĥ, & I i, \\
\rowfont\footnotesize \emph{f} as in ``fly'' & \emph{g} as in ``gun'' & \emph{j} as in ``join'' & \emph{h} as in ``half'' & strongly aspirated h, \emph{ch} as in ``loch'' (Scotch) & \emph{i} as in ``marine''  \\[2ex]
\rowfont\LARGE J j, & Ĵ ĵ, & K k, & L l, & M m, & N n,  \\
\rowfont\footnotesize \emph{y} as in ``yoke'' & \emph{z} as in ``azure'' & \emph{k} as in ``key'' & \emph{l} as in ``line'' & \emph{m} as in ``make'' & \emph{n} as in ``now'' \\[2ex]
\rowfont\LARGE O o, & P p, & R r, & S s, & Ŝ ŝ, & T t,   \\
\rowfont\footnotesize \emph{o} as in ``not'' & \emph{p} as in ``pair'' & \emph{r} as in ``rare'' & \emph{s} as in ``see'' & \emph{sh} as in ``show'' & \emph{t} as in ``tea'' \\[2ex]
\rowfont\LARGE & U u, & Ŭ ŭ, & V v, & Z z. \\
\rowfont\footnotesize & \emph{u} as in ``bull'' & \emph{u} as in ``mount'' (used in diphthongs) & \emph{v} as in ``very'' & \emph{z} as in ``zeal'' 
\end{tabu}
\end{center}

If it be found impractical to print works with the diacritical signs ( ˆ, ˘ ), the letter \emph{h} may be substituted for the sign ( ˆ ), and the sign ( ˘ ) may be altogether omitted; but at the beginning of works so printed there should be this note: ``NB ch \texttt{=} ĉ; gh \texttt{=} ĝ; hh \texttt{=} ĥ; jh \texttt{=} ĵ; sh \texttt{=} ŝ.''

When it is necessary to make use of the ``internal'' sign ( \, ) care should be taken that it can not be mistaken for a comma.  Instead of ( \, ) may be printed ( \textquotesingle{} ) or ( $\cdot$ ), e.~g. \emph{sign\,et\,o, sign\textquotesingle{}et\textquotesingle{}o,} or \emph{sign$\cdot$et$\cdot$o}.

%
% Parts of Speech
%
\gramsect{B}{Parts of Speech}

1. There is no indefinite, and only one definite, article, \emph{la}, for all genders, numbers, and cases.

2. Substantives are formed by adding \emph{o} to the root. For the plural, the letter \emph{j} must be added to the singular. There are two cases: the nominative and the objective (accusative). The root with the added \emph{o} is the nominative, the objective adds an \emph{n} after the \emph{o}. Other cases are formed by prepositions; thus, the possessive (genitive) by \emph{de}, “of”; the dative by \emph{al}, “to”; the instrumental (ablative) by \emph{kun}, “with”, or other preposition as the sense demands. E.~g., root \emph{patr}, “father”; \emph{la patr\,o}, “the father”; \emph{patr\,o\,n}, “father” (objective), \emph{de la patr\,o}, “of the father”, \emph{al la patr\,o}, “to the father”, \emph{kun la patr\,o}, “with the father”; \emph{la patro\,j}, “the fathers”; \emph{la patro\,j\,n}, “the fathers” (obj.), \emph{por la patr\,o\,j}, “for the fathers”.

3. Adjectives are formed by adding \emph{a} to the root. The numbers and cases are the same as in substantives. The comparative degree is formed by prefixing \emph{pli} (more); the superlative by \emph{plej} (most). The word “than” is rendered by \emph{ol}, e.~g., \emph{pli blank\,a ol neĝ\,o}, “whiter than snow”.

4. The cardinal numerals do not change their forms for the different cases. They are:

\begin{center}
\begin{tabular}{rlccrl}
1 & \emph{unu} & \hspace{2em} & \hspace{2em} & 7 & \emph{sep} \\
2 & \emph{du} & & & 8 & \emph{ok} \\
3 & \emph{tri} & & & 9 & \emph{naŭ} \\
4 & \emph{kvar} & & & 10 & \emph{dek} \\
5 & \emph{kvin} & & & 100 & \emph{cent} \\
6 & \emph{ses} & & & 1000 & \emph{mil} 
\end{tabular}
\end{center}

The tens and hundreds are formed by simple junction of the numerals, e.~g., 533 = \emph{kvin\,cent tri\,dek tri}.

Ordinals are formed by adding the adjectival \emph{a} to the cardinals, e.~g., \emph{unu\,a}, “first”; \emph{du\,a}, “second”, etc.

Multiplicatives (as “threefold”, “fourfold”, etc.) add \emph{obl}, e.~g., \emph{tri\,obl\,a}, “threefold”.

Fractionals add \emph{on}, as \emph{du\,on\,o}, “a half”, \emph{kvar\,on,o}, “a quarter”. Collective numerals add \emph{op}, as \emph{kvar\,op\,e}, “four together”.

Distributives prefix \emph{po}, e.~g., \emph{po kvin}, “five apiece”.

Adverbials take \emph{e}, e.~g., \emph{unu\,e}, “firstly”, etc.

5. The Personal Pronouns are: \emph{mi}, I; \emph{vi}, thou, you; \emph{li}, he; \emph{ŝi}, she; \emph{ĝi}, it; \emph{si}, “self”; \emph{ni}, “we”; \emph{ili}, “they”; \emph{oni}, “one”, “people”, (French “on”).

Possessive pronouns are formed by suffixing to the required personal, the adjectival termination. The declension of the pronouns is identical with that of substantives. E.~g., \emph{mi}, “I”; \emph{mi\,n}, “me” (obj.); \emph{mi\,a}, “my”, “mine”.

6. The verb does not change its form for numbers or persons, e.~g., \emph{mi far\,as}, “I do”; \emph{la patr\,o far\,as}, “the father does”; \emph{ili far\,as}, “they do”.

Forms of the Verb:

a) The present tense ends in \emph{as}, e.~g., \emph{mi far\,as}, “I do”.

b) The past tense ends in \emph{is}, e.~g., \emph{li far\,is}, “he did”.

c) The future tense ends in \emph{os}, e.~g., \emph{ili far\,os}, “they will do”.

ĉ) The subjunctive mood ends in \emph{us}, e.~g., \emph{ŝi far\,us}, “she may do”.

d) The imperative mood ends in \emph{u}, e.~g., \emph{ni far\,u}, “let us do”.

e) The infinitive mood ends in \emph{i}, e.~g., \emph{far\,i}, “to do”.

There are two forms of the participle in the international language, the changeable or adjectival, and the unchangeable or adverbial. 

f) The present participle active ends in \emph{ant}, e.~g., \emph{far\,ant\,a}, “he who is doing”; \emph{far\,ant\,e}, “doing”.

g) The past participle active ends in \emph{int}, e.~g., \emph{far\,int\,a}, “he who has done”; \emph{far\,int\,e}, “having done”.

ĝ) The future participle active ends in \emph{ont}, e.~g., \emph{far\,ont\,a}, “he who will do”; \emph{far\,ont\,e}, “about to do”.

h) The present participle passive ends in \emph{at}, e.~g., \emph{far\,at\,e}, “being done”.

ĥ) The past participle passive ends in \emph{it}, e.~g., \emph{far\,it\,a}, “that which has been done”; \emph{far\,it\,e}, “having been done”.

i) The future participle passive ends in \emph{ot}, e.~g., \emph{far\,ot\,a}, “that which will be done”; \emph{far\,ot\,e}, “about to be done”.

All forms of the passive are rendered by the respective forms of the verb \emph{est} (to be) and the present participle passive of the required verb; the preposition used is \emph{de}, “by”. E.~g., \emph{ŝi est\,as am\,at\,a de ĉiu\,j}, “she is loved by every one.”

7. Adverbs are formed by adding \emph{e} to the root. The degrees of comparison are the same as in adjectives, e.~g., \emph{mi\,a frat\,o kant\,as pli bon\,e ol mi}, “my brother sings better than I”.

8. All prepositions govern the nominative case.

%
% General Rules
%
\gramsect{C}{General Rules}

1. Every word is to be read exactly as written, there are no silent letters.

2. The accent falls on the last syllable but one, (penultimate).

3. Compound words are formed by the simple junction of roots, (the principal word standing last), which are written as a single word, but, in elementary works, separated by a small line (\, or {\relsize{-2.5}$'$}). Grammatical terminations are considered as independent words, e.~g., \emph{vapor\,ŝip\,o}, “steamboat”, is composed of the roots \emph{vapor}, “steam”, and \emph{ŝip}, “a boat”, with the substantival termination \emph{o}.

4. If there be one negative in a clause, a second is not admissible.

5. In phrases answering the question “where?” (meaning direction), the words take the termination of the objective case; e.~g., \emph{kie\,n vi ir\,as?} “where are you going?” \emph{dom\,o\,n}, “home”; \emph{London\,o\,n}, “to London”; etc.

6. Every preposition in the international language has a definite fixed meaning. If it be necessary to employ some preposition, and it is not quite evident from the sense which it should be, the word \emph{je} is used, which has no definite meaning; for example, \emph{ĝoj\,i je tio}, “to rejoice \emph{over} it”; \emph{rid\,i je tio} “to laugh \emph{at} it”; \emph{enu\,o je la patr\,uj\,o}, “a longing \emph{for} one’s fatherland”. In every language different prepositions, sanctioned by usage, are employed in these dubious cases, in the international language, one word, \emph{je}, suffices for all. Instead of \emph{je}, the objective without a preposition may be used, when no confusion is to be feared.

7. The so-called “foreign” words, i.~e., words which the greater number of languages have derived from the same source, undergo no change in the international language, beyond conforming to its system of orthography.---Such is the rule with regard to primary words, derivatives are better formed (from the primary word) according to the rules of the international grammar: e.~g., \emph{teatr\,o}, “theater”, but \emph{teatr\,a}, “theatrical”, (not \emph{teatrical\,a}), etc.

8. The \emph{a} of the article, and the final \emph{o} of substantives, may be sometimes dropped euphoniae gratia, e.~g., \emph{de l’ mond\,o} for \emph{de la mond\,o}; \emph{Ŝiller’} for \emph{Ŝiller\,o}; in such cases an apostrophe should be substituted for the discarded vowel. 
