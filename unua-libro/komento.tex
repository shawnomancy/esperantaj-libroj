\kolofono

\small Kvankam mi provis rekrei la 19e-jarcentan «senton» de \emph{Unua Libro} tiel fidele kiel eble, mi ne perfekte povis imiti ĉiun tiparon, ornaman lineon, tiparan grandon, ktp.  Ankaŭ:

\begin{enumerate}

\item La rezigno per D-ro Zamenhofo de siaj aŭtoraj rajtoj ne aperas en la angla eldono, kiun mi inspektis; tamen mi elektis aldoni ĝin al la cenzurpaĝo je la komenco, kaj aranĝis ĝin tiel, kiel ĝi estas en la eldonoj de la aliaj lingvoj, por eksponi la karakteron de la D-ro kaj de la internacia lingvo.

\item Por klareco, tiu ĉi versio ignoras la vorton «kian», kiun D-ro Zamenhofo ŝanĝis al «kiam» dum la unua jaro (1888) post la publikigo de la lingvo, kaj ankaŭ ignoras la aliaj korelativoj (tabelvortoj) kiun estis ŝanĝita el «-an» al «-am». Kvankam la originala angla eldono de Geoghegan (1889) enhavis ambaŭ formojn (ekz. «\emph{kiam (kian)} when»).

\item Mi elektis komposti la libron por A5 papergrando, kiel la plej proksima al la originala.  Tio precipe influas la \emph{Vortaron}, kiu uzis faldatan paĝon en la originala, kiu estas tre pli granda ol la aliaj paĝoj de la libro.

\item Por anglaj vortoj kaj frazoj, mi uzis anglajn citilojn (``kiel ĉi tiuj''), anstataŭ la germanaj (\glqq{}kiel ĉi tiuj\grqq{}) de la originala.  Mi konservis la germanajn citilojn por esperantaj kaj germanaj vortoj kaj frazoj, kaj por la nomo de la lingvo Volap\"{u}k.

\item Mi silente korektis malgravajn, evidentajn mistajpaĵojn, ekzemple la misliterumado de ``certain'' kiel ``certian'' ĉe la malsupro de paĝo 12 de la originalo.

\item Mi uzas modernajn nombritajn piednotojn, anstataŭ la nekonsenkvencaj ``(*)'' kaj ``*)'' el la originala.

\end{enumerate}

Mi dankas Gene-on K\fakesc{EYES}, kies antaŭa PDF de \emph{Unua Libro} estas uzita, kiel la tajpata fundamento de la teksto.  Ĉiuj eraroj aŭ nesufiĉaĵoj, tamen, estas la miaj.\\[1ex]

{\setlength{\parindent}{0em}
Shawn \fsc{Knight} (angle elparolata \emph{ŝan najt})\\
\hodiau}