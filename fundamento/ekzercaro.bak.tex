%
% Ekzercaro
%
\label{ekzercaro}
\markboth{FUNDAMENTO DE ESPERANTO}{EKZERCARO}
\thispagestyle{plain}
\begin{center}
\narrow{\huge\bf \spaceoutless{EKZERCARO}}

\bookman{de la lingvo internacia «~Esperanto~»}
\end{center}

§1

ALFABETO

Aa, Bb, Cc, Ĉĉ, Dd, Ee, Ff, Gg, Ĝĝ, Hh, Ĥĥ, Ii, Jj, Ĵĵ, Kk, Ll, Mm, Nn, Oo, Pp, Rr, Ss, Ŝŝ, Tt, Uu, Ŭŭ, Vv, Zz.

Aa, Bb, Cc, Ĉĉ, Dd, Ee, Ff, Gg, Ĝĝ, Hh, Ĥĥ, Ii, Jj, Ĵĵ, Kk, Ll, Mm, Nn, Oo, Pp, Rr, Ss, Ŝŝ, Tt, Uu, Ŭŭ, Vv, Zz.

\textbf{Nomoj de la literoj:} a, bo, co, ĉo, do, e, fo, go, ĝo, ho, ĥo, i, jo, ĵo, ko, lo, mo, no, o, po, ro, so, ŝo, to, u, ŭo, vo, zo.

§2

Ekzerco de legado.

Al. Bá-lo. Pát-ro. Nú-bo. Cé-lo. Ci-tró-no. Cén-to. Sén-to. Scé-no. Scí-o. Có-lo. Kó-lo. O-fi-cí-ro. Fa-cí-la. Lá-ca. Pa-cú-lo. Ĉar. Ĉe-mí-zo. Ĉi-ká-no. Ĉi-é-lo. Ĉu. Fe-lí-ĉa. Cí-a. Ĉí-a. Pro-cé-so. Sen-ĉé-sa Ec. Eĉ. Ek. Da. Lú-do. Dén-to. Plén-di. El. En. De. Té-ni. Sen. Vé-ro. Fá-li. Fi-dé-la. Trá-fi. Gá-lo. Grán-da. Gén-to. Gíp-so. Gús-to. Lé-gi. Pá-go. Pá-ĝo. Ĝis. Ĝús-ta. Ré-ĝi. Ĝar-dé-no. Lón-ga. Rég-no. Síg-ni. Gvar-dí-o. Lín-gvo. Ĝu-á-do. Há-ro. Hi-rún-do. Há-ki. Ne-hé-la. Pac-hó-ro. Ses-hó-ra Bat-hú-fo. Hó-ro. Ĥó-ro. Kó-ro. Ĥo-lé-ro. Ĥe-mí-o. I-mí-ti. Fí-lo. Bír-do. Tró-vi. Prin-tém-po. Min. Fo-í-ro. Fe-í-no. I-el. I-am. In. Jam. Ju. Jes. Ju-ris-to. Kra-jó-no. Ma-jés-ta. Tuj. Dó-moj. Ru-í-no. Prúj-no. Ba-lá-i. Pá-laj. De-í-no. Véj-no. Pe-ré-i. Mál-plej. Jús-ta. Ĵus. Ĵé-ti. Ĵa-lú-za. Ĵur-nálo. Má-jo. Bo-ná-ĵo. Ká-po. Ma-kú-lo. Kés-to. Su-ké-ro. Ak-vo. Ko-ké-to. Li-kvó-ro. Pac-ká-po.

§3

Ekzerco de legado.

Lá-vi. Le-ví-lo. Pa-ró-li. Mem. Im-plí-ki. Em-ba-rá-so. Nó-mo. In-di-fe-rén-ta. In-ter-na-cí-a. Ol. He-ró-i. He-ro-í-no. Fój-no. Pí-a. Pál-pi. Ri-pé-ti. Ar-bá-ro. Sá-ma. Stá-ri. Si-gé-lo. Sis-té-mo. Pe-sí-lo. Pe-zí-lo. Sén-ti. So-fís-mo. Ci-pré-so. Ŝi. Pá-ŝo. Stá-lo. Ŝtá-lo. Vés-to. Véŝ-to. Dis-ŝí-ri. Ŝan-cé-li. Ta-pí-ŝo Te-o-rí-o. Pa-tén-to. U-tí-la. Un-go. Plú-mo. Tu-múl-to. Plu. Lú-i. Kí-u. Ba-lá-u. Tra-ú-lo. Pe-ré-u. Ne-ú-lo. Fráŭ-lo. Paŭ-lí-no. Láŭ-di. Eŭ-ró-po. Tro-ú-zi. Ho-dí-aŭ. Vá-na. Vér-so. Sól-vi. Zór-gi. Ze-ní-to. Zo-o-lo-gí-o. A-zé-no. Me-zú-ro. Ná-zo. Tre-zó-ro. Mez-nók-to. Zú-mo. Sú-mo. Zó-no. Só-no. Pé-zo. Pé-co. Pé-so. Ne-ní-o. A-dí-aŭ. Fi-zí-ko. Ge-o-gra-fí-o. Spi-rí-to. Lip-há-ro. In-díg-ni. Ne-ní-el. Spe-gú-lo. Ŝpí-no. Né-i. Ré-e. He-ró-o. Kon-scí-i. Tra-e-té-ra. He-ro-é-to. Lú-e. Mó-le. Pá-le. Tra-í-re. Pa-sí-e. Me-tí-o. In-ĝe-ni-é-ro. In-sék-to. Re-sér-vi. Re-zér-vi.

§4

Ekzerco de legado.

Citrono. Cento. Sceno. Scio. Balau. Ŝanceli. Neniel. Embaraso. Zoologio. Reservi. Traire. Hodiaŭ. Disŝiri. Neulo. Majesta. Packapo. Heroino. Pezo. Internacia. Seshora. Cipreso. Stalo. Feino. Plu. Sukero. Gento. Indigni. Sigelo. Krajono. Ruino. Pesilo. Lipharo. Metio. Ĝardeno. Sono. Laŭdi. Pale. Facila. Insekto. Kiu. Zorgi. Cikano. Traetera. Sofismo. Domoj. Spino. Majo. Signi. Ec. Bonaĵo. Legi. Iel. Juristo. Ĉielo. Ĥemio.

§5

Patro kaj frato. ― Leono estas besto. ― Rozo estas floro kaj kolombo estas birdo. ― La rozo apartenas al Teodoro. ― La suno brilas. ― La patro estas sana. ― La patro estas tajloro.

patro père | father | Vater | отецъ | ojciec.
o marque le substantif | ending of nouns (substantive) | bezeichnet das Substantiv | означаетъ существительное | oznacza rzeczownik.
kaj et | and | und | и | i.
frato frère | brother | Bruder | братъ | brat.
leono lion | lion | Löwe | левъ | lew.
esti être | be | sein | быть | być.
as marque le présent d’un verbe | ending of the present tense in verbs | bezeichnet das Präsens | означаетъ настоящее время глагола | oznacza czas teraźniejszy.
besto animal | beast | Thier | животное | zwierzę.
rozo rose | rose | Rose | роза | róża.
floro fleur | flouwer | Blume | цвѣтъ, цвѣтокъ | kwiat.
kolombo pigeon | dove | Taube | голубь | gołąb’.
birdo oiseau | bird | Vogel | птица | ptak.
la article défini (le, la, les) | the | bestimmter Artikel (der, die, das) | членъ опредѣленный (по русски не переводится) | przedimek określny (nie tłómaczy się).
aparteni appartenir | belong | gehören | принадлежать | należeć.
al à | to | zu (ersetzt zugleich den Dativ) | къ (замѣняетъ также дательный падежъ) | do (zastępuje też przypadek trzeci).
suno soleil | sun | Sonne | солнце | słońce.
brili briller | shine | glänzen | блистать | błyszczeć.
sana sain, en santé | well, healthy | gesund | здоровый | zdrowy.
a marque l’adjectif | termination of adjectives | bezeichnet das Adjektiv | означаетъ прилагательное | oznacza przymiotnik.
tajloro tailleur | tailor | Schneider | портной | krawiec.

§6

Infano ne estas matura homo. ― La infano jam ne ploras. ― La ĉielo estas blua. ― Kie estas la libro kaj la krajono? ― La libro estas sur la tablo, kaj la krajono kuŝas sur la fenestro. ― Sur la fenestro kuŝas krajono kaj plumo. ― Jen estas pomo. ― Jen estas la pomo, kiun mi trovis. ― Sur la tero kuŝas ŝtono.

infano enfant | child | Kind | дитя | dziecię.
ne non, ne, ne... pas | no, not | nicht, nein | не, нѣтъ | nie.
matura mûr | mature, ripe | reif | зрѣлый | dojrzały.
homo homme | man | Mensch | человѣкъ | człowiek.
jam déjà | already | schon | уже | juź.
plori pleurer | mourn, weep | weinen | плакать | płakać.
ĉielo ciel | heaven | Himmel | небо | niebo.
blua bleu | blue | blau | синій | niebieski.
kie où | where | wo | гдѣ | gdzie.
libro livre | book | Buch | книга | księga, książka.
krajono crayon | pencil | Bleistift | карандашъ | ołówek.
sur sur | upon, on | auf | на | na.
tablo table | table | Tisch | столъ | stół.
kuŝi être couché | lie (down) | liegen | лежать | leżeć.
fenestro fenêtre | window | Fenster | окно | okno.
plumo plume | pen | Feder | перо | pióro.
jen voici, voilà | behold, lo | da, siehe | вотъ | otóż.
pomo pomme | apple | Apfel | яблоко | jabłko.
kiu qui lequel, laquelle | who, which | wer, welcher | кто, который | kto, który.
n marque l’accusatif ou complément direct | ending of the objective | bezeichnet den Accusativ | означаетъ винительный падежъ | oznacza przypadek czwarty.
mi je, moi | I | ich | я | ja.
trovi trouver | find | finden | находить | znajdować.
is marque le passé | ending of past tense in verbs | bezeichnet die vergangene Zeit | означаетъ прошедшее время | oznaczca czas przeszły.
tero terre | earth | Erde | земля | ziemia.
ŝtono pierre | stone | Stein | камень | kamień.

§7

Leono estas forta. ― La dentoj de leono estas akraj. ― Al leono ne donu la manon. ― Mi vidas leonon. ― Resti kun leono estas danĝere. ― Kiu kuraĝas rajdi sur leono? ― Mi parolas pri leono.

forta fort | strong | stark, kräftig | сильный | silny, mocny.
dento dent | tooth | Zahn | зубъ | ząb.
j marque le pluriel | sign of the plural | bezeichnet die Mehrzahl | означаетъ множественное число | oznacza liczbę mnogą.
de de | of, from | von; ersetzt auch den Genitiv | отъ; замѣняетъ также родительный падежъ | od; zastępuje też przypadek drugi.
akra aigu | sharp | scharf | острый | ostry.
doni donner | give | geben | давать | dawać.
u marque l’impératif | ending of the imperative in verbs | bezeichnet den Imperativ | означаетъ повелительное наклоненіе | oznacza tryb rozkazujący.
mano main | hand | Hand | рука | ręka.
vidi voir | see | sehen | видѣть | widzieć.
resti rester | remain | bleiben | оставаться | pozostawać.
kun avec | with | mit | съ | z.
danĝero danger | danger | Gefahr | опасность | niebezpieczeństwo.
e marque l’adverbe | ending of adverbs | Endung des Adverbs | окончаніе нарѣчія | zakończenie przysłówka.
kuraĝa courageux | courageous, daring | kühn, dreist | смѣлый | śmiały.
rajdi aller à cheval | ride | reiten | ѣздить верхомъ | jeździć konno.
i marque l’infinitif | termination of the infinitive in verbs | bezeichnet den Infinitiv | означаетъ неопредѣленное наклоненіе | oznacza tryb bezokoliczny słowa.
paroli parler | speak | sprechen | говорить | mówić.
pri sur, touchant, de | concerning, about | von, über | о, объ | o.

§8

La patro estas bona. ― Jen kuŝas la ĉapelo de la patro. ― Diru al la patro, ke mi estas diligenta. ― Mi amas la patron. ― Venu kune kun la patro. ― La filo staras apud la patro. ― La mano de Johano estas pura. ― Mi konas Johanon. ― Ludoviko, donu al mi panon. ― Mi manĝas per la buŝo kaj flaras per la nazo. ― Antaŭ la domo staras arbo. ― La patro estas en la ĉambro.

bona bon | good | gut | добрый | dobry.
ĉapelo chapeau | hat | Hut | шляпа | kapelusz.
diri dire | say | sagen | сказать | powiadać.
ke que | that (conj.) | dass | что | że.
diligenta diligent, assidu | diligent | fleissig | прилежный | pilny.
ami aimer | love | lieben | любить | lubić, kochać.
veni venir | come | kommen | приходить | przychodzić.
kune ensemble | together | zusammen | вмѣстѣ | razem, wraz.
filo fils | son | Sohn | сынъ | syn.
stari être debout | stand | stehen | стоять | stać.
apud auprès de | near by | neben, an | при, возлѣ | przy, obok.
pura pur, propre | clean, pure | rein | чистый | czysty.
koni connaître | know, recognise | kennen | знать (быть знакомымъ) | znać.
pano pain | bread | Brot | хлѣбъ | chleb.
manĝi manger | eat | essen | ѣсть | jeść.
per par, au moyen de | through, by means of | mittelst, vermittelst, durch | посредствомъ | przez, za pomocą.
buŝo bouche | mouth | Mund | ротъ | usta.
flari flairer, sentir | smell | riechen, schnupfen | нюхать, обонять | wąchać.
nazo nez | nose | Nase | носъ | nos.
antaŭ devant | before | vor | предъ | przed.
domo maison | house | Haus | домъ | dom.
arbo arbre | tree | Baum | дерево | drzewo.
ĉambro chambre | room | Zimmer | комната | pokój.

§9

La birdoj flugas. ― La kanto de la birdoj estas agrabla. ― Donu al la birdoj akvon, ĉar ili volas trinki. ― La knabo forpelis la birdojn. ― Ni vidas per la okuloj kaj aŭdas per la oreloj. ― Bonaj infanoj lernas diligente. ― Aleksandro ne volas lerni, kaj tial mi batas Aleksandron. ― De la patro mi ricevis libron, kaj de la frato mi ricevis plumon. ― Mi venas de la avo, kaj mi iras nun al la onklo. ― Mi legas libron. ― La patro ne legas libron, sed li skribas leteron.

flugi voler (avec des ailes) | fly (vb.) | fliegen | летать | latać.
kanti chanter | sing | singen | пѣть | śpiewać.
agrabla agréable | agreeable | angenehm | пріятный | przyjemny.
akvo eau | water | Wasser | вода | woda.
ĉar car, parce que | for | weil, da, denn | ибо, такъ какъ | albowiem, ponieważ.
ili ils, elles | they | sie (Mehrzahl) | они, онѣ | oni, one.
voli vouloir | wish, will | wollen | хотѣть | chcieć.
trinki boire | drink | trinken | пить | pić.
knabo garçon | boy | Knabe | мальчикъ | chłopiec.
for loin, hors | forth, out | fort | прочь | precz.
peli chasser, renvoyer | pursue, chase out | jagen, treiben | гнать | gonić.
ni nous | we | wir | мы | my.
okulo œil | eye | Auge | глазъ | oko.
aŭdi entendre | hear | hören | слышать | słyszeć.
orelo oreille | ear | Ohr | ухо | ucho.
lerni apprendre | learn | lernen | учиться | uczyć się.
tial c’est pourquoi | therefore | darum, deshalb | потому | dla tego.
bati battre | beat | schlagen | бить | bić.
ricevi recevoir, obtenir | obtain, get, receive | bekommen, erhalten | получать | otrzymywać.
avo grand-père | grandfather | Grossvater | дѣдъ, дѣдушка | dziad, dziadek.
iri aller | go | gehen | идти | iść.
nun maintenant | now | jetzt | теперь | teraz.
onklo oncle | uncle | Onkel | дядя | wuj, stryj.
legi lire | read | lesen | читать | czytać.
sed mais | but | aber, sondern | но, а | lecz.
li il, lui | he | er | онъ | on.
skribi écrire | write | schreiben | писать | pisać.
letero lettre, épître | letter | Brief | письмо | list.

§10

Papero estas blanka. ― Blanka papero kuŝas sur la tablo. ― La blanka papero jam ne kuŝas sur la tablo. ― Jen estas la kajero de la juna fraŭlino. ― La patro donis al mi dolĉan pomon. ― Rakontu al mia juna amiko belan historion. ― Mi ne amas obstinajn homojn. ― Mi deziras al vi bonan tagon, sinjoro! ― Bonan matenon! ― Ĝojan feston! (mi deziras al vi). ― Kia ĝoja festo! (estas hodiaŭ). ― Sur la ĉielo staras la bela suno. ― En la tago ni vidas la helan sunon, kaj en la nokto ni vidas la palan lunon kaj la belajn stelojn. ― La papero estas tre blanka, sed la neĝo estas pli blanka. ― Lakto estas pli nutra, ol vino. ― Mi havas pli freŝan panon, ol vi. ― Ne, vi eraras, sinjoro: via pano estas malpli freŝa, ol mia. ― El ĉiuj miaj infanoj Ernesto estas la plej juna. ― Mi estas tiel forta, kiel vi. ― El ĉiuj siaj fratoj Antono estas la malplej saĝa.

papero papier | paper | Papier | бумага | papier.
blanka blanc | white | weiss | бѣлый | biały.
kajero cahier | copy book | Heft | тетрадь | kajet.
juna jeune | young | jung | молодой | młody.
fraŭlo homme non marié | bachelor | unverheiratheter Herr | холостой господинъ | kawaler.
in marque le féminin; ex.: patro père ― patrino mère | ending of feminine words; e.g. patro father ― patrino mother | bezeichnet das weibliche Geschlecht; z.B. paro Vater ― patrino Mutter; fianĉo Bräutigam ― fianĉino Braut | означаетъ женскій полъ; напр. patro отецъ ― patrino мать; fianĉo женихъ ― fianĉino невѣста | oznacza płeć żeńską; np. patro ojciec ― patrino matka; koko kogut ― kokino kura.
(fraŭlino demoiselle, mademoiselle | miss | Fräulein | барышня | panna.)
dolĉa doux | sweet | süss | сладкій | słodki.
rakonti raconter | tell, relate | erzählen | разсказывать | opowiadać.
mia mon | my | mein | мой | mój.
amiko ami | friend | Freund | другъ | przyjaciel.
bela beau | beautiful | schön, hübsch | красивый, прекрасный | piękny, ładny.
historio histoire | history, story | Geschichte | исторія | historja.
obstina entêté, obstiné | obstinate | eigensinnig | упрямый | uparty.
deziri désirer | desire | wünschen | желать | życzyć.
vi vous, toi, tu | you | Ihr, du, Sie | вы, ты | wy, ty.
tago jour | day | Tag | день | dzień.
sinjoro monsieur | Sir, Mr | Herr | господинъ | pan.
mateno matin | morning | Morgen | утро | poranek.
ĝoji se réjouir | rejoice | sich freuen | радоваться | cieszyć się.
festi fêter | feast | feiern | праздновать | świętować.
kia quel | of what kind, what a | was für ein, welcher | какой | jaki.
hodiaŭ aujourd’hui | to-day | heute | сегодня | dziś.
en en, dans | in | in, ein- | въ | w.
hela clair (qui n’est pas obscur) | clear, glaring | hell, grell | яркій | jasny, jaskrawy.
nokto nuit | night | Nacht | ночь | noc.
pala pâle | pale | bleich, blass | блѣдный | blady.
luno lune | moon | Mond | луна | księźyć.
stelo étoile | star | Stern | звѣзда | gwiazda.
neĝo neige | snow | Schnee | снѣгъ | śnieg.
pli plus | more | mehr | болѣе, больше | więcej.
lakto lait | milk | Milch | молоко | mleko.
nutri nourrir | nourish | nähren | питать | karmić, pożywiać.
ol que (dans une comparaison) | than | als | чѣмъ | niź.
vino vin | vine | Wein | вино | wino.
havi avoir | have | haben | имѣть | mieć.
freŝa frais, récent | fresh | frisch | свѣжій | świeźy.
erari errer | err, mistake | irren | ошибаться, блуждать | błądzić, mylić się.
mal marque les contraires; ex. bona bon ― malbona mauvais; estimi estimer ― malestimi mépriser | denotes opposites; e.g. bona good ― malbona evil; estimi esteem ― malestimi despise | bezeichnet einen geraden Gegensatz; z.B. bona gut ― malbona schlecht; estimi schätzen ― malestimi verachten | прямо противоположно; напр. bona хорошій ― malbona дурной; estimi уважать ― malestimi презирать | oznacza preciwieństwo; np. bona dobry ― malbona zły; estimi poważać ― malestimi gardzić.
el de, d’entre, é-, ex- | from, out from | aus | изъ | z.
ĉiu chacun | each, every one | jedermann | всякій, каждый | wszystek, każdy.
(ĉiuj tous | all | alle | всѣ | wszyscy.)
plej le plus | most | am meisten | наиболѣе | najwięcej.
tiel ainsi, de cette manière | thus, so | so | такъ | tak.
kiel comment | how, as | wie | какъ | jak.
si soi, se | one’s self | sich | себя | siebie.
(sia son, sa | one’s | sein | свой | swój.)
saĝa sage, sensé | wise | klug, vernünftig | умный | mądry.

§11

La feino.

Unu vidvino havis du filinojn. La pli maljuna estis tiel simila al la patrino per sia karaktero kaj vizaĝo, ke ĉiu, kiu ŝin vidis, povis pensi, ke li vidas la patrinon; ili ambaŭ estis tiel malagrablaj kaj tiel fieraj, ke oni ne povis vivi kun ili. La pli juna filino, kiu estis la plena portreto de sia patro laŭ sia boneco kaj honesteco, estis krom tio unu el la plej belaj knabinoj, kiujn oni povis trovi.

feino fée | fairy | Fee | фея | wieszczka.
unu un | one | ein, eins | одинъ | jeden.
vidvo veuf | widower | Wittwer | вдовецъ | wdowiec.
du deux | two | zwei | два | dwa.
simila semblable | like, similar | ähnlich | похожій | podobny.
karaktero caractère | character | Charakter | характеръ | charakter.
vizaĝo visage | face | Gesicht | лицо | twarz.
povi pouvoir | be able, can | können | мочь | módz.
pensi penser | think | denken | думать | myśleć.
ambaŭ l’un et l’autre | both | beide | оба | obaj.
fiera fier, orgueilleux | proud | stolz | гордый | dumny.
oni on | one, people, they | man | безличное мѣстоименіе множественнаго числа | zaimek nieosobisty liczby mnogiej.
vivi vivre | live | leben | жить | żyć.
plena plein | full, complete | voll | полный | pełny.
portreto portrait | portrait | Portrait | портретъ | portret.
laŭ selon, d’après | according to | nach, gemäss | по, согласно | według.
ec marque la qualité (abstraitement); ex. bona bon ― boneco bonté; viro homme ― vireco virilité | denotes qualities; e.g. bona good ― boneco goodness; viro man ― vireco manliness; virino woman ― virineco womanliness | Eigenschaft; z.B. bona gut ― boneco Güte; virino Weib ― virineco Weiblichkeit | качество или состояніе; напр. bona добрый ― boneco доброта; virino женщина ― virineco женственность | przymiot; np. bona dobry ― boneco dobroć; infano dziecię ― infaneco dziecinśtwo.
honesta honnête | honest | ehrlich | честный | uczciwy.
krom hors, hormis, excepté | besides, without, except | ausser | кромѣ | oprócz.
tio cela | that, that one | jenes, das | то, это | to, tamto.

§12

Du homoj povas pli multe fari ol unu. ― Mi havas nur unu buŝon, sed mi havas du orelojn. ― Li promenas kun tri hundoj. ― Li faris ĉion per la dek fingroj de siaj manoj. ― El ŝiaj multaj infanoj unuj estas bonaj kaj aliaj estas malbonaj. ― Kvin kaj sep faras dek du. ― Dek kaj dek faras dudek. ― Kvar kaj dek ok faras dudek du. ― Tridek kaj kvardek kvin faras sepdek kvin. ― Mil okcent naŭdek tri. ― Li havas dek unu infanojn. ― Sesdek minutoj faras unu horon, kaj unu minuto konsistas el sesdek sekundoj. ― Januaro estas la unua monato de la jaro, Aprilo estas la kvara, Novembro estas la dek-unua, Decembro estas la dek-dua. ― La dudeka (tago) de Februaro estas la kvindek-unua tago de la jaro. ― La sepan tagon de la semajno Dio elektis, ke ĝi estu pli sankta, ol la ses unuaj tagoj. ― Kion Dio kreis en la sesa tago? ― Kiun daton ni havas hodiaŭ? ― Hodiaŭ estas la dudek sepa (tago) de Marto. ― Georgo Vaŝington estis naskita la dudek duan de Februaro de la jaro mil sepcent tridek dua.

multe beaucoup, nombreux | much, many | viel | много | wiele.
fari faire | do | thun, machen | дѣлать | robić.
nur seulement, ne... que | only (adv.) | nur | только | tylko.
promeni se promener | walk, promenade | spazieren | прогуливаться | spacerować.
tri trois | three | drei | три | trzy.
hundo chien | dog | Hund | песъ, собака | pies.
ĉio tout | everything | alles | все | wszystko.
dek dix | ten | zehn | десять | dziesięć.
fingro doigt | finger | Finger | палецъ | palec.
alia autre | other | ander | иной | inny.
kvin cinq | five | fünf | пять | pięć.
sep sept | seven | sieben | семь | siedm.
kvar quatre | four | vier | четыре | cztery.
ok huit | eight | acht | восемь | ośm.
mil mille (nombre) | thousand | tausend | тысяча | tysiąc.
cent cent | hundred | hundert | сто | sto.
naŭ neuf (9) | nine | neun | девять | dziewięć.
ses six | six | sechs | шесть | sześć.
minuto minute | minute | Minute | минута | minuta.
horo heure | hour | Stunde | часъ | godzina.
konsisti consister | consist | bestehen | состоять | składać się.
sekundo seconde | second | Sekunde | секунда | sekunda.
Januaro Janvier | January | Januar | Январь | Styczeń.
monato mois | month | Monat | мѣсяцъ | miesiąc.
jaro année | year | Jahr | годъ | rok.
Aprilo Avril | April | April | Апрѣль | Kwiecień.
Novembro Novembre | November | November | Ноябрь | Listopad.
Decembro Décembre | December | December | Декабрь | Grudzień.
Februaro Février | February | Februar | Февраль | Luty.
semajno semaine | week | Woche | недѣля | tydzień.
Dio Dieu | God | Gott | Богъ | Bóg.
elekti choisir | choose | wählen | выбирать | wybierać.
ĝi cela, il, elle | it | es, dieses | оно, это | ono, to.
sankta saint | holy | heilig | святой, священный | święty.
krei créer | create | schaffen, erschaffen | создавать | stwarzać.
dato date | date | Datum | число (мѣсяца) | data.
Marto Mars | March | März | Мартъ | Marzec.
naski enfanter, faire naître | bear, produce | gebären | рождать | rodzić.
it marque le participe passé passif | ending of past part. pass. in verbs | bezeichnet das Participium perfecti passivi | означаетъ причастіе прошедшаго времени страдат. залога | oznacza imiesłów bierny czasu przeszłego.

§13

La feino (Daŭrigo).

Ĉar ĉiu amas ordinare personon, kiu estas simila al li, tial tiu ĉi patrino varmege amis sian pli maljunan filinon, kaj en tiu sama tempo ŝi havis teruran malamon kontraŭ la pli juna. Ŝi devigis ŝin manĝi en la kuirejo kaj laboradi senĉese. Inter aliaj aferoj tiu ĉi malfeliĉa infano devis du fojojn en ĉiu tago iri ĉerpi akvon en tre malproksima loko kaj alporti domen plenan grandan kruĉon.

daŭri durer | endure, last | dauern | продолжаться | trwać.
ig faire...; ex. pura pur, propre ― purigi nettoyer; morti mourir ― mortigi tuer (faire mourir) | cause to be; e.g. pura pure ― purigi purify; sidi sit ― sidigi seat | zu etwas machen, lassen; z.B. pura rein ― purigi reinigen; bruli brennen (selbst) ― bruligi brennen (etwas) | дѣлать чѣмъ нибудь, заставить дѣлать; напр. pura чистый ― purigi чистить; bruli горѣть ― bruligi жечь | robić czemś; np. pura czysty ― purigi czyścić; bruli palić się ― bruligi palić.
ordinara ordinaire | ordinary | gewöhnlich | обыкновенный | zwyczajny.
persono personne | person | Person | особа, лицо | osoba.
tiu celui-là | that | jener | тотъ | tamten.
ĉi ce qui est le plus près; ex. tiu celui-là ― tiu ĉi celui-ci | denotes proximity; e. g. tiu that ― tiu ĉi this; tie there ― tie ĉi here | die nächste Hinweisung; z. B. tiu jener ― tiu ĉi dieser; tie dort ― tie ĉi hier | ближайшее указаніе; напр. tiu тотъ ― tiu ĉi этотъ; tie тамъ ― tie ĉi здѣсь | wskazanie najbliższe; np. tiu tamten ― tiu ĉi ten; tie tam ― tie ĉi tu.
varma chaud | warm | warm | теплый | ciepły.
eg marque augmentation, plus haut degré; ex. pordo porte ― pordego grande porte; peti prier ― petegi supplier | denotes increase of degree; e.g. varma warm ― varmega hot | bezeichnet eine Vergösserung oder Steigerung; z.B. pordo Thür ― pordego Thor; varma warm ― varmega heiss | означаетъ увеличеніе или усиленіе степени; напр. mano рука ― manego ручище; varma теплый ― varmega горячій | oznacza zwiększenie lub wzmocnienie stopnia; np. mano ręka ― manego łapa; varma ciepły ― varmega gorący.
sama même (qui n’est pas autre) | same | selb, selbst (z. B. derselbe, daselbst) | же, самый (напр. тамъ же, тотъ самый) | źe, sam (np. tam że, ten sam).
tempo temps (durée) | time | Zeit | время | czas.
teruro terreur, effroi | terror | Schrecken | ужасъ | przerażenie.
kontraŭ contre | against | gegen | противъ | przeciw.
devi devoir | ought, must | müssen | долженствовать | musieć.
kuiri faire cuire | cook | kochen | варить | gotować.
ej marque le lieu spécialement affecté à... ex. preĝi prier ― preĝejo église; kuiri faire cuire ― kuirejo cuisine | place of an action; e.g. kuiri cook ― kuirejo kitchen | Ort für...; z.B. kuiri kochen ― kuirejo Küche; preĝi beten ― preĝejo Kirche | мѣсто для...; напр. kuiri варить ― kuirejo кухня; preĝi молиться ― preĝejo церковь | miejsce dla...; np. kuiri gotować ― kuirejo kuchnia; preĝi modlić się ― preĝejo kościół.
labori travailler | labor, work | arbeiten | работать | pracować.
ad marque durée dans l’action; ex. pafo coup de fusil ― pafado fusillade | denotes duration of action; e.g. danco dance ― dancado dancing | bezeichnet die Dauer der Thätigkeit; z.B. danco der Tanz ― dancado das Tanzen | означаетъ продолжительность дѣйствія: напр. iri идти ― iradi ходить, хаживать | oznacza trwanie czynności; np. iri iść -- iradi chodzić.
sen sans | without | ohne | безъ | bez.
ĉesi cesser | cease, desist | aufhören | переставать | przestawać.
inter entre, parmi | between, among | zwischen | между | między.
afero affaire | affair | Sache, Angelegenheit | дѣло | sprawa.
feliĉa heureux | happy | glücklich | счастливый | szczęśliwy.
fojo fois | time (e. g. three times etc.) | Mal | разъ | raz.
ĉerpi puiser | draw | schöpfen (z. B. Wasser) | черпать | czerpać.
tre très | very | sehr | очень | bardzo.
proksima proche, près de | near | nahe | близкій | blizki.
loko place, lieu | place | Ort | мѣсто | miejsce.
porti porter | pack, carry | tragen | носить | nosić.
n marque l’accusatif et le lieu ou où l’on va | ending of the objective, also marks direction towards | bezeichnet den Accusativ, auch die Richtung | означаетъ винит. падежъ, а также направленіе | oznacza przypadek czwarty, również kierunek.
kruĉo cruche | pitcher | Krug | кувшинъ | dzban.

§14

Mi havas cent pomojn. ― Mi havas centon da pomoj. ― Tiu ĉi urbo havas milionon da loĝantoj. ― Mi aĉetis dekduon (aŭ dek-duon) da kuleroj kaj du dekduojn da forkoj. ― Mil jaroj (aŭ milo da jaroj) faras miljaron. ― Unue mi redonas al vi la monon, kiun vi pruntis al mi; due mi dankas vin por la prunto; trie mi petas vin ankaŭ poste prunti al mi, kiam mi bezonos monon. ― Por ĉiu tago mi ricevas kvin frankojn, sed por la hodiaŭa tago mi ricevis duoblan pagon, t. e. (= tio estas) dek frankojn. ― Kvinoble sep estas tridek kvin. ― Tri estas duono de ses. ― Ok estas kvar kvinonoj de dek. ― Kvar metroj da tiu ĉi ŝtofo kostas naŭ frankojn; tial du metroj kostas kvar kaj duonon frankojn (aŭ da frankoj). ― Unu tago estas tricent-sesdek-kvinono aŭ tricent-sesdek-sesono de jaro. ― Tiuj ĉi du amikoj promenas ĉiam duope. ― Kvinope ili sin ĵetis sur min, sed mi venkis ĉiujn kvin atakantojn. ― Por miaj kvar infanoj mi aĉetis dek du pomojn, kaj al ĉiu el la infanoj mi donis po tri pomoj. ― Tiu ĉi libro havas sesdek paĝojn; tial, se mi legos en ĉiu tago po dek kvin paĝoj, mi finos la tutan libron en kvar tagoj.

on marque les nombres fractionnaires; ex. kvar quatre ― kvarono le quart | marks fractions; e.g. kvar four ― kvarono a fourth, quarter | Bruchzahlwort; z.B. kvar vier ― kvarono Viertel | означаетъ числительное дробное; напр. kvar четыре ― kvarono четверть | liczebnik ułamkowy; np. kvar cztery ― kvarono ćwierć.
da de (après les mots marquant mesure, poids, nombre) | is used instead of de after words expressing weight or measure | ersetzt den Genitiv nach Mass, Gewicht u. drgl bezeichnenden Wörtern | замѣняетъ родительный падежъ послѣ словъ, означающихъ мѣру, вѣсъ и т. п. | zastępuje przypadek drugi po słowach oznaczających miarę, wagę i. t. p.
urbo ville | town | Stadt | городъ | miasto.
loĝi habiter, loger | lodge | wohnen | жить, квартировать | mieszkać.
ant marque le participe actif | ending of pres. part. act. in verbs | bezeichnet das Participium praes. act. | означаетъ причасіе настоящаго времени дѣйств. залога | oznacza imieslów czynny czasu teraźniejsz.
aĉeti acheter | buy | kaufen | покупать | kupować.
aŭ ou | or | oder | или | albo, lub.
kulero cuillère | spoon | Löffel | ложка | łyżka.
forko fourchette | fork | Gabel | вилы, вилка | widły, widelec.
re de nouveau, de retour | again, back | wieder, zurück | снова, назадъ | znowu, napowrót.
mono argent (monnaie) | money | Geld | деньги | pieniądze.
prunti prêter | lend, borrow | leihen, borgen | взаймы давать или брать | pożyczać.
danki remercier | thank | danken | благодарить | dziękować.
por pour | for | für | для, за | dla, za.
peti prier | request, beg | bitten | просить | prosić.
ankaŭ aussi | also | auch | также | także.
post après | after, behind | nach, hinter | послѣ, за | po, za, potem.
kiam quand, lorsque | when | wann | когда | kiedy.
bezoni avoir besoin de | need, want | brauchen | нуждаться | potrzebować.
obl marque l’adjectif numéral multiplicatif; ex. du deux ― duobla double | ...fold; e.g. du two ― duobla twofold, duplex | bezeichnet das Vervielfachungszahlwort; z.B. du zwei ― duobla zweifach | означаетъ числительное множительное; напр. du два ― duobla двойной | oznacza liczebnik wieloraki; np. du dwa ― duobla podwójny.
pagi payer | pay | zahlen | платить | płacić.
ŝtofo étoffe | stuff, matter, goods | Stoff | вещество, матерія | materja, materjał.
kosti coûter | cost | kosten | стоить | kosztować.
ĉiam toujours | always | immer | всегда | zawsze.
op marque ládjectif numéral collectif; ex. du deux ― duope à deux | marks collective numerals; e.g. tri three ― triope three together | Sammelzahlwort; z.B. du zwei ― duope selbander, zwei zusammen | означаетъ числительное собирательное; напр. du два ― duope вдвоемъ | oznacza liczebnik zbiorowy; np. du dwa ― duope we dwoje.
ĵeti jeter | throw | werfen | бросать | rzucać.
venki vaincre | conquer | siegen | побѣждать | zwyciężać.
ataki attaquer | attack | angreifen | нападать | atakować.
paĝo page (d’un livre) | page | Seite (Buch-) | страница | stronica.
se si | if | wenn | если | jeżeli.
fini finir | end, finish | enden, beendigen | кончать | kończyć.
tuta entier, total | whole | ganz | цѣлый, весь | cały.

§15

La feino (Daŭrigo).

En unu tago, kiam ŝi estis apud tiu fonto, venis al ŝi malriĉa virino, kiu petis ŝin, ke ŝi donu al ŝi trinki. “Tre volonte, mia bona,”; diris la bela knabino. Kaj ŝi tuj lavis sian kruĉon kaj ĉerpis akvon en la plej pura loko de la fonto kaj alportis al la virino, ĉiam subtenante la kruĉon, por ke la virino povu trinki pli oportune. Kiam la bona virino trankviligis sian soifon, ŝi diris al la knabino: “Vi estas tiel bela, tiel bona kaj tiel honesta, ke mi devas fari al vi donacon” (ĉar tio ĉi estis feino, kiu prenis sur sin la formon de malriĉa vilaĝa virino, por vidi, kiel granda estos la ĝentileco de tiu ĉi juna knabino). “Mi faras al vi donacon,” daŭrigis la feino, “ke ĉe ĉiu vorto, kiun vi diros, el via buŝo eliros aŭ floro aŭ multekosta ŝtono.”

fonto source | fountain | Quelle | источникъ | żródło.
riĉa riche | rich | reich | богатый | bogaty.
viro homme (sexe) | man | Mann | мужчина, мужъ | mężczyzna, mąż.
volonte volontiers | willingly | gern | охотно | chętnie.
tuj tout de suite, aussitôt | immediate | bald, sogleich | сейчасъ | natychmiast.
lavi laver | wash | waschen | мыть | myć.
sub sous | under, beneath, below | unter | подъ | pod.
teni tenir | hold, grasp | halten | держать | trzymać.
oportuna commode, qui est à propos | opportune, suitable | bequem | удобный | wygodny.
trankvila tranquille | quiet | ruhig | спокойный | spokojny.
soifi avoir soif | thirst | dursten | жаждать | pragnąć.
donaci faire cadeau | make a present | schenken | дарить | darować.
preni prendre | take | nehmen | брать | brać.
formo forme | form | Form | форма | forma, kształt.
vilaĝo village | village | Dorf | деревня | wieś.
ĝentila gentil, poli | polite, gentle | höflich | вѣжливый | grzeczny.
ĉe chez | at | bei | у, при | u, przy.

§16

Mi legas. ― Ci skribas (anstataŭ „ci” oni uzas ordinare „vi”). ― Li estas knabo, kaj ŝi estas knabino. ― La tranĉilo tranĉas bone, ĉar ĝi estas akra. ― Ni estas homoj. ― Vi estas infanoj. ― Ili estas rusoj. ― Kie estas la knaboj? ― Ili estas en la ĝardeno. ― Kie estas la knabinoj? ― Ili ankaŭ estas en la ĝardeno. ― Kie estas la tranĉiloj? ― Ili kuŝas sur la tablo. ― Mi vokas la knabon, kaj li venas. ― Mi vokas la knabinon, kaj ŝi venas. ― La infano ploras, ĉar ĝi volas manĝi. ― La infanoj ploras, ĉar ili volas manĝi. ― Knabo, vi estas neĝentila. ― Sinjoro, vi estas neĝentila. ― Sinjoroj, vi estas neĝentilaj. ― Mia hundo, vi estas tre fidela. ― Oni diras, ke la vero ĉiam venkas. ― En la vintro oni hejtas la fornojn. ― Kiam oni estas riĉa (aŭ riĉaj), oni havas multajn amikojn.

ci tu, toi, | thou | du | ты | ty.
anstataŭ au lieu de | instead | anstatt, statt | вмѣсто | zamiast.
uzi employer | use | gebrauchen | употреблять | używać.
tranĉi trancher, couper | cut | schneiden | рѣзать | rżnąć.
il instrument; ex. tondi tondre ― tondilo ciseaux; pafi tirer (coup de feu) ― pafilo fusil | instrument; e.g. tondi shear ― tondilo scissors | Werkzeug; z.B. tondi scheeren ― tondilo Scheere; pafi schieesen ― pafilo Flinte | орудіе; напр. tondi стричь ― tondilo ножницы; pafi стрѣлять ― pafilo ружье | narzędzie; np. tondi strzydz ― tondilo nożyce; pafi strzelać ― pafilo fuzya.
ruso russe | Russian | Russe | русскій | rossjanin.
ĝardeno jardin | garden | Garten | садъ | ogród.
voki appeler | call | rufen | звать | wołać.
voli vouloir | wish, will | wollen | хотѣть | chcieć.
fidela fidèle | faithful | treu | вѣрный | wierny.
vero vérité | true | Wahrheit | истина | prawda.
vintro hiver | winter | Winter | зима | zima.
hejti chauffer, faire du feu | heat (vb.) | heizen | топить (печку) | palić (w piecu).
forno fourneau, poële, four | stove | Ofen | печь, печка | piec.

§17

La feino (Daŭrigo).

Kiam tiu ĉi bela knabino venis domen, ŝia patrino insultis ŝin, kial ŝi revenis tiel malfrue de la fonto. “Pardonu al mi, patrino,” diris la malfeliĉa knabino, “ke mi restis tiel longe”. Kaj kiam ŝi parolis tiujn ĉi vortojn, elsaltis el ŝia buŝo tri rozoj, tri perloj kaj tri grandaj diamantoj. “Kion mi vidas!” diris ŝia patrino kun grandega miro. “Ŝajnas al mi, ke el ŝia buŝo elsaltas perloj kaj diamantoj! De kio tio ĉi venas, mia filino?” (Tio ĉi estis la unua fojo, ke ŝi nomis ŝin sia filino). La malfeliĉa infano rakontis al ŝi naive ĉion, kio okazis al ŝi, kaj, dum ŝi parolis, elfalis el ŝia buŝo multego da diamantoj. “Se estas tiel,” diris la patrino, “mi devas tien sendi mian filinon. Marinjo, rigardu, kio eliras el la buŝo de via fratino, kiam ŝi parolas; ĉu ne estus al vi agrable havi tian saman kapablon? Vi devas nur iri al la fonto ĉerpi akvon; kaj kiam malriĉa virino petos de vi trinki, vi donos ĝin al ŝi ĝentile.”

insulti injurier | insult | schelten, schimpfen | ругать | besztać, łajać.
kial pourquoi | because, wherefore | warum | почему | dlaczego.
frue de bonne heure | early | früh | рано | rano, wcześnie.
pardoni pardonner | forgive | verzeihen | прощать | przebaczać.
longa long | long | lang | долгій, длинный | długi.
salti sauter, bondir | leap, jump | springen | прыгать | skakać.
perlo perle | pearl | Perle | жемчугъ | perła.
granda grand | great, tall | gross | большой, великій | wielki, duźy.
diamanto diamant | diamond | Diamant | алмазъ | djament.
miri s’étonner, admirer | wonder | sich wundern | удивляться | dziwić się.
ŝajni sembler | seem | scheinen | казаться | wydawać się.
nomi nommer, appeler | name, nominate | nennen | называть | naźywać.
naiva naïf | naïve | naiv | наивный | naiwny.
okazi avoir lieu, arriver | happen | vorfallen | случаться | zdarzać się.
dum pendant, tandis que | while | während | пока, между тѣмъ какъ | póki.
sendi envoyer | send | senden, schicken | посылать | posyłać.
kapabla capable, apte | capable | fähig | способный | zdolny.

§18

Li amas min, sed mi lin ne amas. ― Mi volis lin bati, sed li forkuris de mi. ― Diru al mi vian nomon. ― Ne skribu al mi tiajn longajn leterojn. ― Venu al mi hodiaŭ vespere. ― Mi rakontos al vi historion. ― Ĉu vi diros al mi la veron? ― La domo apartenas al li. ― Li estas mia onklo, ĉar mia patro estas lia frato. ― Sinjoro Petro kaj lia edzino tre amas miajn infanojn; mi ankaŭ tre amas iliajn (infanojn). ― Montru al ili vian novan veston. ― Mi amas min mem, vi amas vin mem, li amas sin mem, kaj ĉiu homo amas sin mem. ― Mia frato diris al Stefano, ke li amas lin pli, ol sin mem. ― Mi zorgas pri ŝi tiel, kiel mi zorgas pri mi mem; sed ŝi mem tute ne zorgas pri si kaj tute sin ne gardas. ― Miaj fratoj havis hodiaŭ gastojn; post la vespermanĝo niaj fratoj eliris kun la gastoj el sia domo kaj akompanis ilin ĝis ilia domo. ― Mi jam havas mian ĉapelon; nun serĉu vi vian. ― Mi lavis min en mia ĉambro, kaj ŝi lavis sin en sia ĉambro. ― La infano serĉis sian pupon; mi montris al la infano, kie kuŝas ĝia pupo. ― Oni ne forgesas facile sian unuan amon.

kuri courir | run | laufen | бѣгать | biegać, lecieć.
vespero soir | evening | Abend | вечеръ | wieczór.
ĉu est-ce que | whether | ob | ли, развѣ | czy.
edzo mari, époux | married person, husband | Gemahl | супругъ | małżonek.
montri montrer | show | zeigen | показывать | pokazywać.
nova nouveau | new | neu | новый | nowy.
vesti vêtir, habiller | clothe | ankleiden | одѣвать | odziewać, ubierać.
mem même (moi-, toi-, etc.) | self | selbst | самъ | sam.
zorgi avoir soin | care, be anxious | sorgen | заботиться | troszczyć się.
gardi garder | guard | hüten | стеречь, беречь | strzedz.
gasto hôte | guest | Gast | гость | gość.
akompani accompagner | accompany | begleiten | сопровождать | towarzyszyć.
ĝis jusqu’à | up to, until | bis | до | do, aż.
serĉi chercher | search | suchen | искать | szukać.
pupo poupée | doll | Puppe | кукла | lalka.
forgesi oublier | forget | vergessen | забывать | zapominać.
facila facile | easy | leicht | легкій | łatwy, lekki.

§19

La feino (Daŭrigo).

“Estus tre bele,” respondis la filino malĝentile, “ke mi iru al la fonto!” ― “Mi volas ke vi tien iru,” diris la patrino, “kaj iru tuj!” La filino iris, sed ĉiam murmurante. Ŝi prenis la plej belan arĝentan vazon, kiu estis en la loĝejo. Apenaŭ ŝi venis al la fonto, ŝi vidis unu sinjorinon, tre riĉe vestitan, kiu eliris el la arbaro kaj petis de ŝi trinki (tio ĉi estis tiu sama feino, kiu prenis sur sin la formon kaj la vestojn de princino, por vidi, kiel granda estos la malboneco de tiu ĉi knabino). “Ĉu mi venis tien ĉi,” diris al ŝi la malĝentila kaj fiera knabino, “por doni al vi trinki? Certe, mi alportis arĝentan vazon speciale por tio, por doni trinki al tiu ĉi sinjorino! Mia opinio estas: prenu mem akvon, se vi volas trinki.” ― “Vi tute ne estas ĝentila,” diris la feino sen kolero. “Bone, ĉar vi estas tiel servema, mi faras al vi donacon, ke ĉe ĉiu vorto, kiun vi parolos, eliros el via buŝo aŭ serpento aŭ rano.”

us marque le conditionnel (ou le subjonctif) | ending of conditional in verbs | bezeichnet den Konditionalis (oder Konjunktiv) | означаетъ условное наклоненіе (или сослагательное) | oznacza tryb warunkowy.
murmuri murmurer, grommeler | murmur | murren, brummen | ворчать | mruczеć.
vazo vase | vase | Gefäss | сосудъ | naczynie.
arĝento argent (métal) | silver | Silber | серебро | srebro.
apenaŭ à peine | scarcely | kaum | едва | ledwie.
ar une réunion de certains objets; ex. arbo arbre ― arbaro forêt | a collection of objects; e.g. arbo tree ― arbaro forest; ŝtupo step ― ŝtuparo stairs | Sammlung gewisser Gegenstände; z.B. arbo Baum ― arbaro Wald; ŝtupo Stufe ― ŝtuparo Treppe, Leiter | собраніе данныхъ предметовъ; напр. arbo дерево ― arbaro лѣсъ; ŝtupo ступень ― ŝtuparo лѣстница | zbiór danych przedmiotów; np. arbo drzewo ― arbaro las; ŝtupo szczebel ― ŝtuparo drabina.
princo prince | prince | Fürst, Prinz | принцъ, князь | książe.
certa certain | certain, sure | sicher, gewiss | вѣрный, извѣстный | pewny.
speciala spécial | special | speciell | спеціальный | specjalny.
opinio opinion | opinion | Meinung | мнѣніе | opinja.
koleri se fâcher | be angry | zürnen | сердиться | gniewać się.
servi servir | serve | dienen | служить | słuźyć.
em qui a le penchant, l’habitude; ex. babili babiller ― babilema babillard | inclined to; e.g. babili chatter ― babilema talkative | geneigt, gewohnt; z.B. babili plaudern ― babilema geschwätzig | склонный, имѣющій привычку; напр. babili болтать ― babilema болтливый | skłonny, przyzwyczajony; np. babili paplać ― babilema gadula.
serpento serpent | serpent | Schlange | змѣя | wąź.
rano grenouille | frog | Frosch | лягушка | żaba.

§20

Nun mi legas, vi legas kaj li legas; ni ĉiuj legas. ― Vi skribas, kaj la infanoj skribas; ili ĉiuj sidas silente kaj skribas. ― Hieraŭ mi renkontis vian filon, kaj li ĝentile salutis min. ― Hodiaŭ estas sabato, kaj morgaŭ estos dimanĉo. ― Hieraŭ estis vendredo, kaj postmorgaŭ estos lundo. ― Antaŭ tri tagoj mi vizitis vian kuzon kaj mia vizito faris al li plezuron. ― Ĉu vi jam trovis vian horloĝon? ― Mi ĝin ankoraŭ ne serĉis; kiam mi finos mian laboron, mi serĉos mian horloĝon, sed mi timas, ke mi ĝin jam ne trovos. ― Kiam mi venis al li, li dormis; sed mi lin vekis. ― Se mi estus sana, mi estus feliĉa. ― Se li scius, ke mi estas tie ĉi, li tuj venus al mi. ― Se la lernanto scius bone sian lecionon, la instruanto lin ne punus. ― Kial vi ne respondas al mi? ― Ĉu vi estas surda aŭ muta? ― Iru for! ― Infano, ne tuŝu la spegulon! ― Karaj infanoj, estu ĉiam honestaj! ― Li venu, kaj mi pardonos al li. Ordonu al li, ke li ne babilu. ― Petu ŝin, ke ŝi sendu al mi kandelon. ― Ni estu gajaj, ni uzu bone la vivon, ĉar la vivo ne estas longa. ― Ŝi volas danci. ― Morti pro la patrujo estas agrable. ― La infano ne ĉesas petoli.

sidi être assis | sit | sitzen | сидѣть | siedzieć.
silenti se taire | be silent | schweigen | молчать | milczeć.
hieraŭ hier | yesterday | gestern | вчера | wczoraj.
renkonti rencontrer | meet | begegnen | встрѣчать | spotykać.
saluti saluer | salute, greet | grüssen | кланяться | kłaniać się.
sabato samedi | Saturday | Sonnabend | суббота | sobota.
morgaŭ demain | to-morrow | morgen | завтра | jutro.
dimanĉo dimanche | Sunday | Sonntag | воскресенье | niedziela.
vendredo vendredi | Friday | Freitag | пятница | piątek.
lundo lundi | Monday | Montag | понедѣльникъ | poniedziałek.
viziti visiter | visit | besuchen | посѣщать | odwiedzać.
kuzo cousin | cousin | Vetter, Cousin | двоюродный братъ | kuzyn.
plezuro plaisir | pleasure | Vergnügen | удовольствіе | przyjemność.
horloĝo horloge, montre | clock | Uhr | часы | zegar.
timi craindre | fear | fürchten | бояться | obawiać się.
dormi dormir | sleep | schlafen | спать | spać.
veki réveiller, éveiller | wake, arouse | wecken | будить | budzić.
scii savoir | know | wissen | знать | wiedzieć.
leciono leçon | lesson | Lektion | урокъ | lekcya.
instrui instruire, enseigner | instruct, teach | lehren | учить | uczyć.
puni punir | punish | strafen | наказывать | karać.
surda sourd | deaf | taub | глухой | głuchy.
muta muet | dumb | stumm | нѣмой | niemy.
tuŝi toucher | touch | rühren | трогать | ruszać, dotykać.
spegulo miroir | looking-glass | Spiegel | зеркало | zwierciadło.
kara cher | dear | theuer | дорогой | drogi.
ordoni ordonner | order, command | befehlen | приказывать | rozkazywać.
babili babiller | chatter | schwatzen, plaudern | болтать | paplać.
kandelo chandelle | candle | Licht, Kerze | свѣча | świeca.
gaja gai | gay, glad | lustig | веселый | wesoły.
danci danser | dance | tanzen | танцовать | tańczyć.
morti mourir | die | sterben | умирать | umierać.
petoli faire le polisson, faire des bêtises | be mischievous | muthwillig sein | шалить | swawolić.
uj qui porte, qui contient, qui est peuplé de; ex. pomo pomme ― pomujo pommier; cigaro cigare ― cigarujo porte-cigares; Turko Turc ― Turkujo Turquie | containing, filled with; e. g. cigaro cigar ― cigarujo cigar-case; pomo apple ― pomujo apple-tree; Turko Turk ― Turkujo Turkey | Behälter, Träger (d. h. Gegenstand worin... aufbewahrt wird,... Früchte tragende Pflanze, von... bevölkertes Land); z. B. cigaro Cigarre ― cigarujo Cigarrenbüchse; pomo Apfel ― pomujo Apfelbaum; Turko Türke ― Turkujo Türkei | вмѣститель, носитель (т. е. вещь, въ которой храниться...; растеніе несущее... или страна, заселенная...); напр. cigaro сигара ― cigarujo портъ-сигаръ; pomo яблоко ― pomujo яблоня; Turko Турокъ ― Turkujo Турція | zawierający, noszący (t. j. przedmiot, w którym się coś przechowuje, roślina, która wydaje owoc, lub kraj, względem zaludniających go mieszkańców; np. cigaro cygaro ― cigarujo cygarnica; pomo jabłko ― pomujo jabłoń; Turko turek ― Turkujo Turcya.

§21

La feino (Daŭrigo).

Apenaŭ ŝia patrino ŝin rimarkis, ŝi kriis al ŝi: “Nu, mia filino?” ― “Jes, patrino”, respondis al ŝi la malĝentilulino, elĵetante unu serpenton kaj unu ranon. ― “Ho, ĉielo!” ekkriis la patrino, “kion mi vidas? Ŝia fratino en ĉio estas kulpa; mi pagos al ŝi por tio ĉi!” Kaj ŝi tuj kuris bati ŝin. La malfeliĉa infano forkuris kaj kaŝis sin en la plej proksima arbaro. La filo de la reĝo, kiu revenis de ĉaso, ŝin renkontis; kaj, vidante, ke ŝi estas tiel bela, li demandis ŝin, kion ŝi faras tie ĉi tute sola kaj pro kio ŝi ploras. ― “Ho ve, sinjoro, mia patrino forpelis min el la domo”.

rimarki remarquer | remark | merken, bemerken | замѣчать | postrzegać, zauwaźać.
krii crier | cry | schreien | кричать | krzyczeć.
nu eh bien! | well! | nu! | nun | ну! | no!
jes oui | yes | ja | да | tak.
ek indique une action qui commence ou qui est momentanée; ex. kanti chanter ― ekkanti commencer à chanter; krii crier ― ekkrii s’écrier | denotes sudden or momentary action; e. g. krii cry ― ekkrii cry out | bezeichnet eine anfangende oder momentane Handlung; z. B. kanti singen ― ekkanti einen Gesang anstimmen; krii schreien ― ekkrii aufschreien | начало или мгновенность; напр. kanti пѣть ― ekkanti запѣть; krii кричать ― ekkrii вскрикнуть | oznacza początek lub chwilowość; np. kanti śpiewać ― ekkanti zaśpiewać; krii krzyczeć ― ekkrii krzyknąć.
kulpa coupable | blameable | schuldig | виноватый | winny.
kaŝi cacher | hide (vb.) | verbergen | прятать | chować.
reĝo roi | king | König | король, царь | król.
ĉasi chasser (vénerie) | hunt | jagen, Jagd machen | охотиться | polować.
demandi demander, questionner | demand, ask | fragen | спрашивать | pytać.
sola seul | only, alone | einzig, allein | единственный | jedyny.
pro à cause de, pour | for the sake of | wegen | ради | dla.
ho oh! | oh! | o! och! | о! охъ! | o! och!.
ve malheur! | woe! | wehe! | увы! | biada! nestety!.

§22

Fluanta akvo estas pli pura, ol akvo staranta senmove. ― Promenante sur la strato, mi falis. ― Kiam Nikodemo batas Jozefon, tiam Nikodemo estas la batanto kaj Jozefo estas la batato. ― Al homo, pekinta senintence, Dio facile pardonas. ― Trovinte pomon, mi ĝin manĝis. ― La falinta homo ne povis sin levi. Ne riproĉu vian amikon, ĉar vi mem plimulte meritas riproĉon; li estas nur unufoja mensoginto dum vi estas ankoraŭ nun ĉiam mensoganto. ― La tempo pasinta jam neniam revenos; la tempon venontan neniu ankoraŭ konas. ― Venu, ni atendas vin, Savonto de la mondo. ― En la lingvo "Esperanto" ni vidas la estontan lingvon de la tuta mondo. ― Aŭgusto estas mia plej amata filo. ― Mono havata estas pli grava ol havita. ― Pasero kaptita estas pli bona, ol aglo kaptota. ― La soldatoj kondukis la arestitojn tra la stratoj. ― Li venis al mi tute ne atendite. ― Homo, kiun oni devas juĝi, estas juĝoto.

flui couler | flow | fliessen | течь | płynąć, cieknąć.
movi mouvoir | move | bewegen | двигать | ruszać.
strato rue | street | Strasse | улица | ulica.
fali tomber | fall | fallen | падать | padać.
at marque le participe présent passif | ending of pres. part. pass. in verbs | bezeichnet das Participium praes. passivi | означаеть причастіе настоящаго времени страд. залога | oznacza imiesłów bierny czasu teraźniejszego.
peki pécher | sin | sündigen | грѣшить | grzeszyć.
int marque le participe passé du verbe actif | ending of past part. act. in verbs | bezeichnet das Participium perfecti activi | означаетъ причастіе прошедшаго времени дѣйствит. залога | oznacza imiesłów czynny czasu przeszłego.
intenci se proposer de | intend | beabsichtigen | намѣреваться | zamierzać.
levi lever | lift, raise | aufheben | поднимать | podnosić.
riproĉi reprocher | reproach | vorwerfen | упрекать | zarzucać.
meriti mériter | merit | verdienen | заслуживать | zasługiwać.
mensogi mentir | tell a lie | lügen | лгать | kłamać.
pasi passer | pass | vergehen | проходить | przechodzić.
neniam ne... jamais | never | niemals | никогда | nigdy.
ont marque le participe futur d’un verbe actif | ending of fut. part. act. in verbs | bezeichnet das Participium fut. act. | означаетъ причастіе будущаго времени дѣйствит. залога | oznacza imiesłów czynny czasu przyszłego.
neniu personne | nobody | Niemand | никто | nikt.
atendi attendre | wait, expect | warten, erwarten | ждать, ожидать | czekać.
savi sauver | save | retten | спасать | ratować.
mondo monde | world | Welt | міръ | świat.
lingvo langue, langage | language | Sprache | языкъ, рѣчь | język, mowa.
grava grave, important | important | wichtig | важный | ważny.
pasero passereau | sparrow | Sperling | воробей | wróbel.
kapti attraper | catch | fangen | ловить | chwytać.
aglo aigle | eagle | Adler | орелъ | orzeł.
ot marque le participe futur d’un verbe passif | ending of fut. part. pass. in verbs | bezeichnet das Participium fut. pass. | означаетъ причастіе будущ. времени страд. залога | oznacza imiesłów bierny czasu przyszłego.
soldato soldat | soldier | Soldat | солдатъ | żolnierz.
konduki conduire | conduct | führen | вести | prowadzić.
aresti arrêter | arrest | verhaften | арестовать | aresztować.
tra à travers | through | durch | черезъ, сквозь | przez (wskroś).
juĝi juger | judge | richten, urtheilen | судить | sądzić.

§23

La feino (Fino)

La reĝido, kiu vidis, ke el ŝia buŝo eliris kelke da perloj kaj kelke da diamantoj, petis ŝin, ke ŝi diru al li, de kie tio ĉi venas. Ŝi rakontis al li sian tutan aventuron. La reĝido konsideris, ke tia kapablo havas pli grandan indon, ol ĉio, kion oni povus doni dote al alia fraŭlino, forkondukis ŝin al la palaco de sia patro, la reĝo, kie li edziĝis je ŝi. Sed pri ŝia fratino ni povas diri, ke ŝi fariĝis tiel malaminda, ke ŝia propra patrino ŝin forpelis de si; kaj la malfeliĉa knabino, multe kurinte kaj trovinte neniun, kiu volus ŝin akcepti, baldaŭ mortis en angulo de arbaro.

kelke quelque | some | mancher, einige | нѣкоторый, нѣсколько | niektóry, kilka.
aventuro aventure | adventure | Abenteuer | приключеніе | przygoda.
konsideri considérer | consider | betrachten, erwägen | соображать | zastanawiać się.
inda mérite, qui mérite, est digne | worthy, valuable | würdig, werth | достойный | godny, wart.
doto dot | dowry | Mitgift | приданое | posag.
palaco palais | palace | Schloss (Gebäude) | дворецъ | pałac.
iĝ se faire, devenir...; ex. pala pâle ― paliĝi pâlir; sidi être assis ― sidiĝi s’asseoir | to become; e. g. pala pale ― paliĝi turn pale; sidi sit ― sidiĝi become seated | zu etwas werden, sich zu etwas veranlassen; z. B. pala blass ― paliĝi erblassen; sidi sitzen ― sidiĝi sich setzen | дѣлаться чѣмъ нибудь, заставить себя; напр. pala блѣдный ― paliĝi блѣднѣть; sidi сидѣть ― sidiĝi сѣсть | stawać się czemś; np. pala blady ― paliĝi blednąć; sidi siedzieć ― sidiĝi usiąść.
je se traduit par différentes prépositions | can be rendered by various prepositions | kann durch verschiedene Präpositionen übersetzt werden | можетъ быть переведено различным предлогами | może być przetłomaczone za pomocą różnych przyimków.
propra propre (à soi) | own (one’s own) | eigen | собственный | własny.
akcepti accepter | accept | annehmen | принимать | przyjmować.
baldaŭ bientôt | soon | bald | сейчасъ, скоро | zaraz.
angulo coin, angle | corner, angle | Winkel | уголъ | kąt.

§24

Nun li diras al mi la veron. ― Hieraŭ li diris al mi la veron. ― Li ĉiam diradis al mi la veron. ― Kiam vi vidis nin en la salono, li jam antaŭe diris al mi la veron (aŭ li estis dirinta al mi la veron). ― Li diros al mi la veron. ― Kiam vi venos al mi, li jam antaŭe diros al mi la veron (aŭ li estos dirinta al mi la veron; aŭ antaŭ ol vi venos al mi, li diros al mi la veron). ― Se mi petus lin, li dirus al mi la veron. ― Mi ne farus la eraron, se li antaŭe dirus al mi la veron (aŭ se li estus dirinta al mi la veron). ― Kiam mi venos, diru al mi la veron. ― Kiam mia patro venos, diru al mi antaŭe la veron (aŭ estu dirinta al mi la veron). ― Mi volas diri al vi la veron. ― Mi volas, ke tio, kion mi diris, estu vera (aŭ mi volas esti dirinta la veron).

salono salоn | saloon | Salon | залъ | salon.
os marque le futur | ending of future tense in verbs | bezeichnet das Futur | означаетъ будущее время | oznacza czas przyszły.

§25

Mi estas amata. Mi estis amata. Mi estos amata. Mi estus amata. Estu amata. Esti amata. ― Vi estas lavita. Vi estis lavita. Vi estos lavita. Vi estus lavita. Estu lavita. Esti lavita. ― Li estas invitota. Li estis invitota. Li estos invitota. Li estus invitota. Estu invitota. Esti invitota. ― Tiu ĉi komercaĵo estas ĉiam volonte aĉetata de mi. ― La surtuto estas aĉetita de mi, sekve ĝi apartenas al mi. ― Kiam via domo estis konstruata, mia domo estis jam longe konstruita. ― Mi sciigas, ke de nun la ŝuldoj de mia filo ne estos pagataj de mi. ― Estu trankvila, mia tuta ŝuldo estos pagita al vi baldaŭ. ― Mia ora ringo ne estus nun tiel longe serĉata, se ĝi ne estus tiel lerte kaŝita de vi. ― Laŭ la projekto de la inĝenieroj tiu ĉi fervojo estas konstruota en la daŭro de du jaroj; sed mi pensas, ke ĝi estos konstruata pli ol tri jarojn. ― Honesta homo agas honeste. ― La pastro, kiu mortis antaŭ nelonge (aŭ antaŭ nelonga tempo), loĝis longe en nia urbo. ― Ĉu hodiaŭ estas varme aŭ malvarme? ― Sur la kameno inter du potoj staras fera kaldrono; el la kaldrono, en kiu sin trovas bolanta akvo, eliras vaporo; tra la fenestro, kiu sin trovas apud la pordo, la vaporo iras sur la korton.

inviti inviter | invite | einladen | приглашать | zapraszać.
komerci commercer | trade | handeln, Handel treiben | торговать | handlować.
aĵ quelque chose possédant une certaine qualité ou fait d’une certaine matière: ex. molа mou ― molaĵo partie molle d’une chose | made from or possessing the quality of; e. g. malnova old ― malnovaĵo old things frukto fruit ― fruktaĵo something made from fruit | etwas von einer gewissen Eigenschaft, oder aus einem gewissen Stoffe; z. B. malnova alt ― malnovaĵo altes Zeug; frukto Frucht ― fruktaĵo etwas aus Früchten bereitetes | нѣчто съ даннымъ качествомъ или изъ даннаго матеріала; напр. molа мягкій ― molaĵо мякишъ; fruktо плодъ ― fruktaĵо нѣчто приготовленное изъ плодовъ | oznacza przedmiot posiadający pewną własność albo zrobiony z pewnego materjału; np. malnova stary ― malnovaĵo starzyzna; frukto owoc ― fruktaĵo coś zrobineego z owoców.
sekvi suivre | follow | folgen | слѣдовать | nastąpić.
konstrui construire | construct, build | bauen | строить | budować.
ŝuldi devoir (dette) | owe | schulden | быть должнымъ | być dłużnym.
oro or (métal) | gold | Gold | золото | złoto.
ringo anneau | ring (subst.) | Ring | кольцо | pierścień.
lerta adroit, habile | skilful | geschickt, gewandt | ловкій | zręczny.
projekto projet | project | Entwurf | проектъ | projekt.
inĝeniero ingénieur | engineer | Ingenieur | инженеръ | inżynier.
fero fer | iron | Eisen | желѣзо | żelazo.
vojo route, voie | way, road | Weg | дорога | droga.
agi agir | act | handeln, verfahren | поступать | postępować.
pastro prêtre, pasteur | priest, pastor | Priester | жрецъ, священникъ | kapłan.
kameno cheminée | fire-place | Kamin | каминъ | kominek.
poto pot | pot | Topf | горшокъ | garnek.
kaldrono chaudron | kettle | Kessel | котелъ | kocioł.
boli bouillir | boil | sieden | кипѣть | kipieć, wrzeć.
vaporo vapeur | steam | Dampf | паръ | para.
pordo porte | door | Thür | дверь | drzwi.
korto cour | yard, court | Hof | дворъ | podwórze.

§26

Kie vi estas? ― Mi estas en la ĝardeno. ― Kien vi iras? ― Mi iras en la ĝardenon. ― La birdo flugas en la ĉambro (= ĝi estas en la ĉambro kaj flugas en ĝi). ― La birdo flugas en la ĉambron (= ĝi estas ekster la ĉambro kaj flugas nun en ĝin). ― Mi vojaĝas en Hispanujo. ― Mi vojaĝas en Hispanujon. ― Mi sidas sur seĝo kaj tenas la piedojn sur benketo. ― Mi metis la manon sur la tablon. ― El sub la kanapo la muso kuris sub la liton, kaj nun ĝi kuras sub la lito. ― Super la tero sin trovas aero. ― Anstataŭ kafo li donis al mi teon kun sukero, sed sen kremo. ― Mi staras ekster la domo, kaj li estas interne. ― En la salono estis neniu krom li kaj lia fianĉino. ― La hirundo flugis trans la riveron, ĉar trans la rivero sin trovis aliaj hirundoj. ― Mi restas tie ĉi laŭ la ordono de mia estro. ― Kiam li estis ĉe mi, li staris tutan horon apud la fenestro. ― Li diras, ke mi estas atenta. ― Li petas, ke mi estu atenta. ― Kvankam vi estas riĉa, mi dubas, ĉu vi estas feliĉa. ― Se vi scius, kiu li estas, vi lin pli estimus. ― Se li jam venis, petu lin al mi. ― Ho, Dio! kion vi faras! ― Ha, kiel bele! ― For de tie ĉi! ― Fi, kiel abomene! ― Nu, iru pli rapide!

ekster hors, en dehors de | outside, besides | ausser, ausserhalb | внѣ | zewnątrz.
vojaĝi voyager | voyage | reisen | путешествовать | podróżować.
piedo pied | foot | Fuss, Bein | нога | noga.
benko banc | bench | Bank | скамья | ławka.
et marque diminution, décroissance; ex. muro mur ― mureto petit mur; ridi rire ― rideti sourire | denotes diminution of degree; e. g. ridi laugh ― rideti smile | bezeichnet eine Verkleinerung oder Schwächung; z. B. muro Wand ― mureto Wändchen; ridi lachen ― rideti lächeln | означаетъ уменьшеніе или ослабленіе степени; напр. muro стѣна ― mureto стѣнка; ridi смѣяться ― rideti улыбаться | oznacza zmniejszenie lub osłabienie stopnia; np. muro ściana ― mureto ścianka; ridi śmiać się ― rideti uśmiechać się.
meti mettre | put, place | hinthun | дѣть; класть | podziać.
kanapo canapé | sofa, lounge | Kanapee | диванъ | kanapa.
muso souris | mouse | Maus | мышь | mysź.
lito lit | bed | Bett | кровать | łóżko.
super au dessus de | over, above | über, oberhalb | надъ | nad.
aero air | air | Luft | воздухъ | powietrze.
kafo café | coffee | Kaffee | кофе | kawa.
teo thé | tea | Thee | чай | herbata.
sukero sucre | sugar | Zucker | сахаръ | cukier.
kremo crème | cream | Schmant, Sahne | сливки | śmietana.
interne à l’intérieur, dedans | within | innerhalb | внутри | wewnątrz.
fianĉo fiancé | betrothed person | Bräutigam | женихъ | narzeczony.
hirundo hirondelle | swallow | Schwalbe | ласточка | jaskółka.
trans au delà | across | jenseit | черезъ, надъ | przez.
rivero rivière, fleuve | river | Fluss | рѣка | rzeka.
estro chef | chief | Vorsteher | начальникъ | zwierzchnik.
atenta attentif | attentive | aufmerksam | внимательный | uważny.
kvankam quoique | although | obgleich | хотя | chociaź.
dubi douter | doubt | zweifeln | сомнѣваться | wątpić.
estimi estimer | esteem | schätzen, achten | уважать | szanować.
fi fi donc! | fie! | pfui! | фи, тьфу | fe!.
abomeno abomination | abomination | Abscheu | отвращеніе | odraza.
rapida rapide, vite | quick, rapid | schnell | быстрый | prędki.

§27

La artikolo „la” estas uzata tiam, kiam ni parolas pri personoj aŭ objektoj konataj. Ĝia uzado estas tia sama kiel en la aliaj lingvoj. La personoj, kiuj ne komprenas la uzadon de la artikolo (ekzemple rusoj aŭ poloj, kiuj ne scias alian lingvon krom sia propra), povas en la unua tempo tute ne uzi la artikolon, ĉar ĝi estas oportuna sed ne necesa. Anstataŭ „la” oni povas ankaŭ diri „l’” (sed nur post prepozicio, kiu finiĝas per vokalo). ― Vortoj kunmetitaj estas kreataj per simpla kunligado de vortoj; oni prenas ordinare la purajn radikojn, sed, se la bonsoneco aŭ la klareco postulas, oni povas ankaŭ preni la tutan vorton, t. e. la radikon kune kun ĝia gramatika finiĝo. Ekzemploj: skribtablo aŭ skribotablo (= tablo, sur kiu oni skribas); internacia (= kiu estas inter diversaj nacioj); tutmonda (= de la tuta mondo); unutaga (= kiu daŭras unu tagon); unuataga (= kiu estas en la unua tago); vaporŝipo (= ŝipo, kiu sin movas per vaporo); matenmanĝi, tagmanĝi, vespermanĝi; abonpago (= pago por la abono).

artikolo article | article | Artikel | членъ, статья | artykuł, przedimek.
tiam alors | then | dann | тогда | wtedy.
objekto objet | object | Gegenstand | предметъ | przedmiot.
tia tel | such | solcher | такой | taki.
kompreni comprendre | understand | verstehen | понимать | rozumieć.
ekzemplo exemple | example | Beispiel | примѣръ | przykład.
polo Polonais | Pole | Pole | Полякъ | Polak.
necesa nécessaire | necessary | nothwendig | необходимый | niezbędny.
prepozicio préposition | preposition | Vorwort | предлогъ | przyimek.
vokalo voyelle | vowel | Vokal | гласная | samogłoska.
kunmeti composer | compound | zusammensetzen | слагать | składać.
simpla simple | simple | einfach | простой | prosty, zwyczajny.
ligi lier | bind, tie | binden | связывать | wiązać.
radiko racine | root | Wurzel | корень | korzeń.
soni sonner, rendre des sons | sound | tönen, lauten | звучать | brzmieć.
klara clair | clear | klar | ясный | jasny.
postuli exiger, requérir | require, claim | fordern | требовать | żądać.
gramatiko grammaire | grammar | Grammatik | грамматика | gramatyka.
nacio nation | nation | Nation | нація, народъ | naród, nacja.
diversa divers | various, diverse | verschieden | различный | różny.
ŝipo navire | ship | Schiff | корабль | okręt.
matenmanĝi déjeuner | breakfast | frühstücken | завтракать | śniadać.
aboni abonner | subscribe | abonniren | подписываться | prenumerować.

§28

Ĉiuj prepozicioj per si mem postulas ĉiam nur la nominativon. Se ni iam post prepozicio uzas la akuzativon, la akuzativo tie dependas ne de la prepozicio, sed de aliaj kaŭzoj. Ekzemple: por esprimi direkton, ni aldonas al la vorto la finon „n”; sekve: tie (= en tiu loko), tien (= al tiu loko); tiel same ni ankaŭ diras: “la birdo flugis en la ĝardenon, sur la tablon”, kaj la vortoj „ĝardenon”, „tablon” staras tie ĉi en akuzativo ne ĉar la prepozicioj „en” kaj „sur” tion ĉi postulas, sed nur ĉar ni volis esprimi direkton, t. e. montri, ke la birdo sin ne trovis antaŭe en la ĝardeno aŭ sur la tablo kaj tie flugis, sed ke ĝi de alia loko flugis al la ĝardeno, al la tablo (ni volas montri, ke la ĝardeno kaj tablo ne estis la loko de la flugado, sed nur la celo de la flugado); en tiaj okazoj ni uzus la finiĝon „n” tute egale ĉu ia prepozicio starus aŭ ne. ― Morgaŭ mi veturos Parizon (aŭ en Parizon). ― Mi restos hodiaŭ dome. ― Jam estas tempo iri domen. ― Ni disiĝis kaj iris en diversajn flankojn: mi iris dekstren, kaj li iris maldekstren. ― Flanken, sinjoro! ― Mi konas neniun en tiu ĉi urbo. ― Mi neniel povas kompreni, kion vi parolas. ― Mi renkontis nek lin, nek lian fraton (aŭ mi ne renkontis lin, nek lian fraton).

nominativo nominatif | nominative | Nominativ | именительный падежъ | pierwszy przypadek.
iam jamais, un jour | at any time, ever | irgend wann | когда-нибудь | kiedyś.
akuzativo accusatif | accusative | Accusativ | винительный падежъ | czwarty przypadek.
tie là-bas, là, y | there | dort | тамъ | tam.
dependi dépendre | depend | abhängen | зависѣть | zależeć.
kaŭzo cause | cause | Ursache | причина | przyczyna.
esprimi exprimer | express | ausdrücken | выражать | wyrażać.
direkti diriger | direct | richten | направлять | kierować.
celi viser | aim | zielen | цѣлиться | celować.
egala égal | equal | gleich | одинаковый | jednakowy.
ia quelconque | of any kind | irgend welcher | какой-нибудь | jakiś.
veturi aller, partir | journey, travel | fahren | ѣхать | jechać.
dis marque division, dissémination; ex. iri aller ― disiri se séparer, aller chacun de son côté | has the same force as the English prefix dis; e. g. semi sow ― dissemi disseminate; ŝiri tear ― disŝiri tear to pieces | zer-; z. B. ŝiri reissen ― disŝiri zerreissen | раз-; напр. ŝiri рвать ― disŝiri разрывать | roz-; np. ŝiri rwać ― disŝiri rozrywać.
flanko côté | side | Seite | сторона | strona.
dekstra droit, droite | right-hand | recht | правый | prawy.
neniel nullement, en aucune façon | nohow | keineswegs, auf keine Weise | никакъ | w żaden sposób.
nek --- nek ni ― ni | neither ― nor | weder ― noch | ни ― ни | ani ― ani.

§29

Se ni bezonas uzi prepozicion kaj la senco ne montras al ni, kian prepozicion uzi, tiam ni povas uzi la komunan prepozicion „je”. Sed estas bone uzadi la vorton „je” kiel eble pli malofte. Anstataŭ la vorto „je” ni povas ankaŭ uzi akuzativon sen prepozicio. ― Mi ridas je lia naiveco (aŭ mi ridas pro lia naiveco, aŭ: mi ridas lian naivecon). ― Je la lasta fojo mi vidas lin ĉe vi (aŭ: la lastan fojon). ― Mi veturis du tagojn kaj unu nokton. ― Mi sopiras je mia perdita feliĉo (aŭ: mian perditan feliĉon). ― El la dirita regulo sekvas, ke se ni pri ia verbo ne scias, ĉu ĝi postulas post si la akuzativon (t. e. ĉu ĝi estas aktiva) aŭ ne, ni povas ĉiam uzi la akuzativon. Ekzemple, ni povas diri “obei al la patro” kaj “obei la patron” (anstataŭ “obei je la patro”). Sed ni ne uzas la akuzativon tiam, kiam la klareco de la senco tion ĉi malpermesas; ekzemple: ni povas diri “pardoni al la malamiko” kaj “pardoni la malamikon”, sed ni devas diri ĉiam “pardoni al la malamiko lian kulpon”.

senco sens, acception | sense | Sinn | смыслъ | sens, znaczenie.
komuna commun | common | gemeinsam | общій | ogólny, wspólny.
ebla possible | able, possible | möglich | возможный | możliwy.
ofte souvent | often | oft | часто | często.
ridi rire | laugh | lachen | смѣяться | śmiać się.
lasta dernier | last, latest | letzt | послѣдній | ostatni.
sopiri soupirer | sigh, long for | sich sehnen | тосковать | tęsknić.
regulo règle | rule | Regel | правило | prawidło.
verbo verbe | verb | Zeitwort | глаголъ | czasownik.
obei obéir | obey | gehorchen | повиноваться | być posłusznym.
permesi permettre | permit, allow | erlauben | позволять | pozwalać.

§30

Ia, ial, iam, ie, iel, ies, io, iom, iu. ― La montritajn naŭ vortojn ni konsilas bone ellerni, ĉar el ili ĉiu povas jam fari al si grandan serion da aliaj pronomoj kaj adverboj. Se ni aldonas al ili la literon „k”, ni ricevas vortojn demandajn aŭ rilatajn: kia, kial, kiam, kie, kiel, kies, kio, kiom, kiu. Se ni aldonas la literon „t”, ni ricevas vortojn montrajn: tia, tial, tiam, tie, tiel, ties, tio, tiom, tiu. Aldonante la literon „ĉ”, ni ricevas vortojn komunajn: ĉia, ĉial, ĉiam, ĉie, ĉiel, ĉies, ĉio, ĉiom, ĉiu. Aldonante la prefikson „nen”, ni ricevas vortojn neajn: nenia, nenial, neniam, nenie, neniel, nenies, nenio, neniom, neniu. Aldonante al la vortoj montraj la vorton „ĉi”, ni ricevas montron pli proksiman; ekzemple: tiu (pli malproksima), tiu ĉi (aŭ ĉi tiu) (pli proksima); tie (malproksime), tie ĉi aŭ ĉi tie (proksime). Aldonante al la vortoj demandaj la vorton „ajn”, ni ricevas vortojn sendiferencajn: kia ajn, kial ajn, kiam ajn, kie ajn, kiel ajn, kies ajn, kio ajn, kiom ajn, kiu ajn. Ekster tio el la diritaj vortoj ni povas ankoraŭ fari aliajn vortojn, per helpo de gramatikaj finiĝoj kaj aliaj vortoj (sufiksoj); ekzemple: tiama, ĉiama, kioma, tiea, ĉi-tiea, tieulo, tiamulo k. t. p. (= kaj tiel plu).

ia quelconque, quelque | of any kind | irgend welcher | какой-нибудь | jakiś.
ial pour une raison quelconque | for any cause | irgend warum | почему-нибудь | dla jakiejś przyczyny.
iam jamais, un jour | at any time, ever | irgend wann, einst | когда-нибудь | kiedyś.
ie quelque part | any where | irgend wo | гдѣ-нибудь | gdzieś.
iel d’une manière quelconque | anyhow | irgend wie | какъ-нибудь | jakoś.
ies de quelqu’un | anyone’s | irgend jemandes | чей-нибудь | czyjś.
io quelque chose | anything | etwas | что-нибудъ | coś.
iom quelque peu | any quantity | ein wenig, irgend wie viel | сколько-нибудь | ilekolwiek.
iu quelqu’un | any one | jemand | кто-нибудь | ktoś.
konsili conseiller | advise, counsel | rathen | совѣтовать | radzić.
serio série | series | Reihe | рядъ, серія | serya.
pronomo pronom | pronoun | Fürwort | мѣстоименіе | zaimek.
adverbo adverbe | adverb | Nebenwort | нарѣчіе | przysłówek.
litero lettre (de l’alphabet) | letter (of the alphabet) | Buchstabe | буква | litera.
rilati concerner; avoir rapport à | be related to | sich beziehen | относиться | odnosić się.
prefikso préfixe | prefix | Präfix | приставка | przybranka.
ajn que ce soit | ever | auch nur | бы-ни | kolwiek, bądź.
diferenci différer (v. n.) | differ | sich unterscheiden | различаться | różnic się.
helpi aider | help | helfen | помогать | pomagać.
sufikso suffixe | suffix | Suffix | суффиксъ | przyrostek.

§31

Lia kolero longe daŭris. ― Li estas hodiaŭ en kolera humoro. ― Li koleras kaj insultas. ― Li fermis kolere la pordon. ― Lia filo mortis kaj estas nun malviva. ― La korpo estas morta, la animo estas senmorta. ― Li estas morte malsana, li ne vivos pli, ol unu tagon. ― Li parolas, kaj lia parolo fluas dolĉe kaj agrable. ― Ni faris la kontrakton ne skribe, sed parole. ― Li estas bona parolanto. ― Starante ekstere, li povis vidi nur la eksteran flankon de nia domo. ― Li loĝas ekster la urbo. ― La ekstero de tiu ĉi homo estas pli bona, ol lia interno. ― Li tuj faris, kion mi volis, kaj mi dankis lin por la tuja plenumo de mia deziro. ― Kia granda brulo! kio brulas? ― Ligno estas bona brula materialo. ― La fera bastono, kiu kuŝis en la forno, estas nun brule varmega. ― Ĉu li donis al vi jesan respondon aŭ nean? Li eliris el la dormoĉambro kaj eniris en la manĝoĉambron. ― La birdo ne forflugis: ĝi nur deflugis de la arbo, alflugis al la domo kaj surflugis sur la tegmenton. ― Por ĉiu aĉetita funto da teo tiu ĉi komercisto aldonas senpage funton da sukero. ― Lernolibron oni devas ne tralegi, sed tralerni. ― Li portas rozokoloran superveston kaj teleroforman ĉapelon. ― En mia skribotablo sin trovas kvar tirkestoj. ― Liaj lipharoj estas pli grizaj, ol liaj vangharoj.

humoro humeur | humor | Laune | расположеніе духа | humor.
fermi fermer | shut | schliessen, zumachen | запирать | zamykać.
korpo corps | body | Körper | тѣло | ciało.
animo âme | soul | Seele | душа | dusza.
kontrakti contracter | contract | einen Vertrag abschliessen | заключать договоръ | zawierać umowę.
um suffixe peu employé, et qui reçoit différents sens aisément suggérés par le contexte et la signification de la racine à laquelle il est joint | this syllable has no fixed meaning | Suffix von verschiedener Bedeutung | суффиксъ безъ постояннаго значенія | przyrostek nie mający stlałego znaczenia.
(plenumi accomplir | fulfil, accomplish | erfüllen | исполнять | spełniać.)
bruli brûler (être en feu) | burn (v. n.) | brennen (v. n.) | горѣть | palić się.
ligno bois | wood (the substance) | Holz | дерево, дрова | drzewo, drwa.
materialo matière | material | Stoff | матеріялъ | materjał.
bastono bâton | stick | Stock | палка | kij, laska.
tegmento toit | roof | Dach | крыша | dach.
funto livre | pound | Pfund | фунтъ | funt.
ist marque la profession; ex. boto botte ― botisto bottier; maro mer ― maristo marin | person occupied with; e. g. boto boot ― botisto boot-maker; maro sea ― maristo sailor | sich mit etwas beschäftigend; z. B. boto Stiefel ― botisto Schuster; maro Meer ― maristo Seeman | занимающійся; напр. boto сапогъ ― botisto сапожникъ; maro море ― maristo морякъ | zajmujący się; np. boto but ― botisto szewc; maro morze ― maristo marynarz.
koloro couleur | color | Farbe | краска, цвѣтъ | kolor.
supre en haut | above, upper | oben | вверху | na górze.
telero assiette | plate | Teller | тарелка | talerz.
tero terre | earth | Erde | земля | ziemia.
kesto caisse, coffre | chest, box | Kiste, Kasten, Lade | ящикъ | skrzynia.
lipo lèvre | lip | Lippe | губа | warga.
haro cheveu | hair | Haar | волосъ | włos.
griza gris | grey | grau | сѣрый, сѣдой | szary, siwy.
vango joue | check | Wange | щека | policzek.

§32

Teatramanto ofte vizitas la teatron kaj ricevas baldaŭ teatrajn manierojn. ― Kiu okupas sin je meĥaniko, estas meĥanikisto, kaj kiu okupas sin je ĥemio, estas ĥemiisto. ― Diplomatiiston oni povas ankaŭ nomi diplomato, sed fizikiston oni ne povas nomi fiziko, ĉar fiziko estas la nomo de la scienco mem. ― La fotografisto fotografis min, kaj mi sendis mian fotografaĵon al mia patro. ― Glaso de vino estas glaso, en kiu antaŭe sin trovis vino, aŭ kiun oni uzas por vino; glaso da vino estas glaso plena je vino. ― Alportu al mi metron da nigra drapo (Metro de drapo signifus metron, kiu kuŝis sur drapo, aŭ kiu estas uzata por drapo). ― Mi aĉetis dekon da ovoj. ― Tiu ĉi rivero havas ducent kilometrojn da longo. ― Sur la bordo de la maro staris amaso da homoj. ― Multaj birdoj flugas en la aŭtuno en pli varmajn landojn. ― Sur la arbo sin trovis multe (aŭ multo) da birdoj. ― Kelkaj homoj sentas sin la plej feliĉaj, kiam ili vidas la suferojn de siaj najbaroj. ― En la ĉambro sidis nur kelke da homoj. ― „Da” post ia vorto montras, ke tiu ĉi vorto havas signifon de mezuro.

teatro theâtre | theatre | Theater | театръ | teatr.
maniero manière, façon | manner | Manier, Weise, Art | способъ, манера | sposób, manjera.
okupi occuper | occupy | einnehmen, beschäftigen | занимать | zajmować.
meĥaniko mécanique | mechanics | Mechanik | механика | mechanika.
ĥemio chimie | chemistry | Chemie | химія | chemia.
diplomatio diplomatie | diplomacy | Diplomatie | дипломатія | dyplomacja.
fiziko phyique | phyics | Phyik | физика | fizyka.
scienco science | science | Wissenschaft | наука | nauka.
glaso verre (à boire) | glass | Glas (Gefäss) | стаканъ | szklanka.
nigra noir | black | schwarz | черный | czarny.
drapo drap | woollen goods | Tuch (wollenes Gewebe) | сукно | sukno.
signifi signifier | signify, mean | bezeichnen, bedeuten | означать | oznaczać.
ovo œuf | egg | Ei | яйцо | jajko.
bordo bord, rivage | shore | Ufer | берегъ | brzeg.
maro mer | sea | Meer | море | morze.
amaso amas, foule | crowd, mass | Haufen, Menge | куча, толпа | kupa, tłum.
aŭtuno automne | autumn | Haufen, Menge | осень | jesień.
lando pays | land, country | Land | страна | kraj.
suferi souffrir, endurer | suffer | leiden | страдать | cierpieć.
najbaro voisin | neighbour | Nachbar | сосѣдъ | sąsiad.
mezuri mesurer | measure | messen | мѣрить | mierzyć.

§33

Mia frato ne estas granda, sed li ne estas ankaŭ malgranda: li estas de meza kresko. ― Li estas tiel dika, ke li ne povas trairi tra nia mallarĝa pordo. ― Haro estas tre maldika. ― La nokto estis tiel malluma, ke ni nenion povis vidi eĉ antaŭ nia nazo. ― Tiu ĉi malfreŝa pano estas malmola, kiel ŝtono. ― Malbonaj infanoj amas turmenti bestojn. ― Li sentis sin tiel malfeliĉa, ke li malbenis la tagon, en kiu li estis naskita. ― Ni forte malestimas tiun ĉi malnoblan homon. ― La fenestro longe estis nefermita; mi ĝin fermis, sed mia frato tuj ĝin denove malfermis. ― Rekta vojo estas pli mallonga, ol kurba. ― La tablo staras malrekte kaj kredeble baldaŭ renversiĝos. ― Li staras supre sur la monto kaj rigardas malsupren sur la kampon. ― Malamiko venis en nian landon. ― Oni tiel malhelpis al mi, ke mi malbonigis mian tutan laboron. ― La edzino de mia patro estas mia patrino kaj la avino de miaj infanoj. ― Sur la korto staras koko kun tri kokinoj. ― Mia fratino estas tre bela knabino. ― Mia onklino estas bona virino. ― Mi vidis vian avinon kun ŝiaj kvar nepinoj kaj kun mia nevino. ― Lia duonpatrino estas mia bofratino. ― Mi havas bovon kaj bovinon. ― La juna vidvino fariĝis denove fianĉino.

mezo milieu | middle | Mitte | средина | środek.
kreski croître | grow, increase | wachsen | рости | rosnąć.
dika gros | thick, fat | dick | толстый | gruby.
larĝa large | broad | breit | широкій | szeroki.
lumi luire | light | leuchten | свѣтить | świecić.
mola mou | soft | weich | мягкій | miękki.
turmenti tourmenter | torment | quälen, martern | мучить | męczyć.
senti ressentir, éprouver | feel, perceive | fühlen | чувствовать | czuć.
beni bénir | bless | segnen | благословлять | błogosławić.
nobla noble | noble | edel | благородный | szlachetny.
rekta droit, direct | straight | gerade | прямой | prosty.
kurba courbe, tortueux | curved | krumm | кривой | krzywy.
kredi croire | believe | glauben | вѣрить | wierzyć.
renversi renverser | upset | umwerfen, umstürzen | опрокидывать | przewracać.
monto montagne | mountain | Berg | гора | góra.
kampo champ, campagne | field | Feld | поле | pole.
koko coq | cock | Hahn | пѣтухъ | kogut.
nepo petit-fils | grandson | Enkel | внукъ | wnuk.
nevo neveu | nephew | Neffe | племянникъ | siostrzeniec, bratanek.
bo marque la parenté résultant du mariage; ex. patro père ― bopatro beau-père | relation by marriage; e. g. patrino mother ― bopatrino mother-in-law | durch Heirath erworben; z. B. patro Vater ― bopatro Schwiegervater; frato Bruder ― bofrato Schwager | пріобрѣтенный бракомъ; напр. patro отецъ ― bopatro тесть, свекоръ; frato братъ ― bofrato шуринъ, зять, деверь, | nabyty przez małżeństwo; np. patro ojciec ― bopatro teść; frato brat ― bofrato szwagier.
duonpatro beau-père | step-father | Stiefvater | отчимъ | ojczym.
bovo bœuf | ox | Ochs | быкъ | byk.

§34

La tranĉilo estis tiel malakra, ke mi ne povis tranĉi per ĝi la viandon kaj mi devis uzi mian poŝan tranĉilon. ― Ĉu vi havas korktirilon, por malŝtopi la botelon? ― Mi volis ŝlosi la pordon, sed mi perdis la ŝlosilon. ― Ŝi kombas al si la harojn per arĝenta kombilo. ― En somero ni veturas per diversaj veturiloj, kaj en vintro ni veturas per glitveturilo. ― Hodiaŭ estas bela frosta vetero, tial mi prenos miajn glitilojn kaj iros gliti. ― Per hakilo ni hakas, per segilo ni segas, per fosilo ni fosas, per kudrilo ni kudras, per tondilo ni tondas, per sonorilo ni sonoras, per fajfilo ni fajfas. ― Mia skribilaro konsistas el inkujo, sablujo, kelke da plumoj, krajono kaj inksorbilo. ― Oni metis antaŭ mi manĝilaron, kiu konsistis el telero, kulero, tranĉilo, forko, glaseto por brando, glaso por vino kaj telertuketo. ― En varmega tago mi amas promeni en arbaro. ― Nia lando venkos, ĉar nia militistaro estas granda kaj brava. ― Sur kruta ŝtuparo li levis sin al la tegmento de la domo. ― Mi ne scias la lingvon hispanan, sed per helpo de vortaro hispana-germana mi tamen komprenis iom vian leteron. ― Sur tiuj ĉi vastaj kaj herboriĉaj kampoj paŝtas sin grandaj brutaroj, precipe aroj da bellanaj ŝafoj.

viando viande | meat, flesh | Fleisch | мясо | mięso.
poŝo poche | pocket | Tasche | карманъ | kieszeń.
korko bouchon | cork | Kork | пробка | korek.
tiri tirer | draw, pull, drag | ziehen | тянуть | ciągnąć.
ŝtopi boucher | stop, fasten down | stopfen | затыкать | zatykać.
botelo bouteille | bottle | Flasche | бутылка | butelka.
ŝlosi fermer à clef | lock, fasten | schliessen | запирать на ключъ | zamykać na klucz.
kombi peigner | comb | kämmen | чесать | czesać.
somero été | summer | Sommer | лѣто | lato.
gliti glisser | sakte | gleiten, glitschen | скользить, кататься | ślizgać się.
frosto gelée | frost | Frost | морозъ | mróz.
vetero temps (température) | weather | Wetter | погода | pogoda.
haki hacher, abattre | hew, chop | hauen, hacken | рубить | rąbać.
segi scier | saw | sägen | пилить | piłować.
fosi creuser | dig | graben | копать | kopać.
kudri coudre | sew | nähen | шить | szyć.
tondi tondre | clip, shear | scheeren | стричь | strzydz.
sonori tinter | give out a sound (as a bell) | klingen | звенѣть | brzęczeć, dzwonić.
fajfi siffler | whistle | pfeifen | свистать | świstać.
inko encre | ink | Tinte | чернила | atrament.
sablo sable | sand | Sand | песокъ | piasek.
sorbi humer | sip | schlürfen | хлебать | chlipać.
brando eau-de-vie | brandy | Branntwein | водка | wódka.
tuko mouchoir | cloth | Tuch (Hals-, Schnupf- etc.) | платокъ | chustka.
militi guerroyer | fight | Krieg führen | воевать | wojować.
brava brave, solide | valliant, brave | tüchtig | дѣльный, удалый | dzielny, chwacki.
kruta roide, escarpé | steep | steil | крутой | stromy.
ŝtupo marche, échelon | step | Stufe | ступень | stopień.
Hispano Espagnol | Spaniard | Spanier | Испанецъ | Hiszpan.
Germano Allemand | German | Deutscher | Нѣмецъ | Niemiec.
tamen pourtant, néanmoins | however, nevertheless | doch, jedoch | однако | jednak.
vasta vaste, étendu | wide, vast | weit, geräumig | обширный, просторный | obszerny.
herbo herbe | grass | Gras | трава | trawa.
paŝti paître | pasture, feed animals | weiden lassen | пасти | paść.
bruto brute, bétail | brute | Vieh | скотъ | bydło.
precipe principalement, surtout | particularly | besonders, vorzüglich | преимущественно | szczególnie.
lano laine | wool | Wolle | шерсть | wełna.
ŝafo mouton | sheep | Schaf | овца | owca.

§35

Vi parolas sensencaĵon, mia amiko. ― Mi trinkis teon kun kuko kaj konfitaĵo. ― Akvo estas fluidaĵo. ― Mi ne volis trinki la vinon, ĉar ĝi enhavis en si ian suspektan malklaraĵon. ― Sur la tablo staris diversaj sukeraĵoj. ― En tiuj ĉi boteletoj sin trovas diversaj acidoj: vinagro, sulfuracido, azotacido kaj aliaj. ― Via vino estas nur ia abomena acidaĵo. ― La acideco de tiu ĉi vinagro estas tre malforta. ― Mi manĝis bongustan ovaĵon. ― Tiu ĉi granda altaĵo ne estas natura monto. ― La alteco de tiu monto ne estas tre granda. ― Kiam mi ien veturas, mi neniam prenas kun mi multon da pakaĵo. ― Ĉemizojn, kolumojn, manumojn kaj ceterajn similajn objektojn ni nomas tolaĵo, kvankam ili ne ĉiam estas faritaj el tolo. ― Glaciaĵo estas dolĉa glaciigita frandaĵo. ― La riĉeco de tiu ĉi homo estas granda, sed lia malsaĝeco estas ankoraŭ pli granda. ― Li amas tiun ĉi knabinon pro ŝia beleco kaj boneco. ― Lia heroeco tre plaĉis al mi. ― La tuta supraĵo de la lago estis kovrita per naĝantaj folioj kaj diversaj aliaj kreskaĵoj. ― Mi vivas kun li en granda amikeco.

kuko gâteau | cаке | Kuchen | пирогъ | pieroźek.
konfiti confire | preserve with sugar | einmachen (mit Zucker) | варить въ сахарѣ | smażyć w cukrze.
fluida liquide | fluid | flüssig | жидкій | płynny.
suspekti suspecter, soupçonner | suspect | verdächtigen | подозрѣвать | podejrzewać.
acida aigre | sour | sauer | кислый | kwaśny.
vinagro vinaigre | vinegar | Essig | уксусъ | ocet.
sulfuro soufre | sulphur | Schwefel | сѣра | siara.
azoto azote | azotе | Stickstoff | азоть | azot.
gusto goût | taste | Geschmack | вкусъ | smak, gust.
alta haut | high | hoch | высокій | wysoki.
naturo nature | nature | Natur | природа | przyroda.
paki empaqueter, emballer | pack, put ut | packen, einpacken | укладывать, упаковывать | pakować.
ĉemizo chemise | shirt | Hemd | сорочка | koszula.
kolo cou | neck | Hals | шея | szyja.
cetera autre (le reste) | rest, remainder | übrig | прочій | pozostały.
tolo toile | linen | Leinwand | полотно | płótno.
glacio glace | ice | Eis | ледъ | lód.
frandi goûter par friandise | dainty | naschen | лакомиться | złakomić się.
heroo héros | hero, champion | Held | герой | bohater.
plaĉi plaire | please | gefallen | нравиться | podobać się.
lago lac | lake | See (der) | озеро | jezioro.
kovri couvrir | cover | verdecken, verhüllen | закрывать | zakrywać.
naĝi nager | swim | schwimmen | плавать | pływać.
folio feuille | leaf | Blatt, Bogen | листъ | liść, arkusz.

§36

Patro kaj patrino kune estas nomataj gepatroj. ― Petro, Anno kaj Elizabeto estas miaj gefratoj. ― Gesinjoroj N. hodiaŭ vespere venos al ni. ― Mi gratulis telegrafe la junajn geedzojn. ― La gefianĉoj staris apud la altaro. ― La patro de mia edzino estas mia bopatro, mi estas lia bofilo, kaj mia patro estas la bopatro de mia edzino. ― Ĉiuj parencoj de mia edzino estas miaj boparencoj, sekve ŝia frato estas mia bofrato, ŝia fratino estas mia bofratino; mia frato kaj fratino (gefratoj) estas la bogefratoj de mia edzino. ― La edzino de mia nevo kaj la nevino de mia edzino estas miaj bonevinoj. ― Virino, kiu kuracas, estas kuracistino; edzino de kuracisto estas kuracistedzino. ― La doktoredzino A. vizitis hodiaŭ la gedoktorojn P. ― Li ne estas lavisto, li estas lavistinedzo. ― La filoj, nepoj kaj pranepoj de reĝo estas reĝidoj. ― La hebreoj estas Izraelidoj, ĉar ili devenas de Izraelo. ― Ĉevalido estas nematura ĉevalo, kokido ― nematura koko, bovido ― nematura bovo, birdido ― nematura birdo.

ge les deux sexes réunis; ex. patro père ― gepatroj les parents (père et mère) | of both sexes; e. g. patro father ― gepatroj parents | beiderlei Geschlechtes; z. B. patro Vater ― gepatroj Eltern; mastro Wirth ― gemastroj Wirth und Wirthin | обоего пола, напр. patro отецъ ― gepatroj родители; mastro хозяинъ ― gemastroj хозяинъ съ хозяйкой | obojej płci, np. patro ojciec ― gepatroj rodzice; mastro gospodarz ― gemastroj gospodarstwo (gospodarz i gospodyni).
gratuli féliciter | congratulate | gratuliren | поздравлять | winszować.
altaro autel | altar | Altar | алтарь | ołtarz.
kuraci traiter (une maladie) | cure, heal | kuriren, heilen | лѣчить | leczyć.
doktoro docteur | doctor | Doctor | докторъ | doktór.
pra bis-, arrière- | great-, primordial | ur- | пра- | pra-.
id enfant, descendant; ex. bovo bœuf ― bovido veau; Izraelo Israël ― Izraelido Israëlite | descendant, young one; e. g. bovo ox ― bovido calf | Kind, Nachkomme; z. B. bovo Ochs ― bovido Kalb; Izraelo Israel ― Izraelido Israelit | дитя, потомокъ; напр. bovo быкъ ― bovido теленокъ; Izraelo Израиль ― Izraelido Израильтянинъ | dziecię, potomek; np. bovo byk ― bovido cielę; Izraelo Izrael ― Izraelido Izraelita.
hebreo juif | Jew | Jude | еврей | żyd.
ĉevalo cheval | horse | Pferd | конь | koń.

§37

La ŝipanoj devas obei la ŝipestron. ― Ĉiuj loĝantoj de regno estas regnanoj. ― Urbanoj estas ordinare pli ruzaj, ol vilaĝanoj. ― La regnestro de nia lando estas bona kaj saĝa reĝo. ― La Parizanoj estas gajaj homoj. ― Nia provincestro estas severa, sed justa. ― Nia urbo havas bonajn policanojn, sed ne sufiĉe energian policestron. ― Luteranoj kaj Kalvinanoj estas kristanoj. ― Germanoj kaj francoj, kiuj loĝas en Rusujo, estas Rusujanoj, kvankam ili ne estas rusoj. ― Li estas nelerta kaj naiva provincano. ― La loĝantoj de unu regno estas samregnanoj, la loĝantoj de unu urbo estas samurbanoj, la konfesantoj de unu religio estas samreligianoj. ― Nia regimentestro estas por siaj soldatoj kiel bona patro. ― La botisto faras botojn kaj ŝuojn. ― La lignisto vendas lignon, kaj la lignaĵisto faras tablojn, seĝojn kaj aliajn lignajn objektojn. - Ŝteliston neniu lasas en sian domon. ― La kuraĝa maristo dronis en la maro. ― Verkisto verkas librojn, kaj skribisto simple transskribas paperojn. ― Ni havas diversajn servantojn: kuiriston, ĉambristinon, infanistinon kaj veturigiston. ― La riĉulo havas multe da mono. ― Malsaĝulon ĉiu batas. ― Timulo timas eĉ sian propran ombron. ― Li estas mensogisto kaj malnoblulo. ― Preĝu al la Sankta Virgulino.

an membre, habitant, partisan; ex. regno l’état ― regnano citoyen | inhabitant, member; e. g. Nov-Jorko New York ― Nov-Jorkano New Yorker | Mitglied, Einwohner, Anhänger; z. B. regno Staat ― regnano Bürger; Varsoviano Warschauer | членъ, житель, приверженец; напр. regno государство ― regnano гражданинъ; Varsoviano Варшавянинъ | członek, mieszkaniec, zwolennik; np. regno państwo ― regnano obywatel; Varsoviano Warszawianin.
regno l’Etat | kingdom | Staat | государство | państwo.
vilaĝano paysan | caountryman | Bauer | крестьянинъ | wieśniak.
provinco province | province | Provinz | область, провинція | prowincya.
severa sévère | severe | streng | строгій | surowy, srogi, ostry.
justo juste | just, righteous | gerecht | справедливый | sprawiedliwy.
polico police | police | Polizei | полиція | policya.
sufiĉe suffisant | enough | genug | довольно, достаточно | dosyć, dostatecznie.
Kristo Christ | Christ | Christus | Христосъ | Chrystus.
Franco Français | Frenchman | Franzose | Французъ | Francuz.
konfesi avouer | confess | bekennen, gestehen | признавать, исповѣдывать | przyznawać.
religio religion | religion | Religion | вѣра, религія | religia.
regimento regiment | regiment | Regiment | полкъ | półk.
boto botte | boot | Stiefel | сапогъ | but.
ŝuo soulier | shoe | Schuh | башмакъ | trzewik.
lasi laisser, abandonner | leave, let alone | lassen | пускать, оставлять | puszczać, zostawiać.
droni se noyer | drown | ertrinken | тонуть | tonąć.
verki composer, faire des ouvrages (littér.) | work (literary) | verfassen | сочинять | tworzyć, pisać.
ul qui est caractérisé par telle ou telle qualité; ex. bela beau ― belulo bel homme | person noted for...; e. g. avara covetous ― avarulo miser, covetous person | Person, die sich durch... unterscheidet; z. B. juna jung ― junulo Jüngling | особа, отличающаяся даннымъ качествомъ; напр. bela красивый ― belulo красавецъ | człowiek, posiadający dany przymiot; np. riĉa bogaty ― riĉulo bogacz.
eĉ même, jusqu’à | even | sogar | даже | nawet.
ombro ombre | shadow | Schatten | тѣнь | cień.
preĝi prier (Dieu) | pray | beten | молиться | modlić się.
virga virginal | virginal | jungfräulich | дѣвственный | dziewiczy.

§38

Mi aĉetis por la infanoj tableton kaj kelke da seĝetoj. ― En nia lando sin ne trovas montoj, sed nur montetoj. ― Tuj post la hejto la forno estis varmega, post unu horo ĝi estis jam nur varma, post du horoj ĝi estis nur iom varmeta, kaj post tri horoj ĝi estis jam tute malvarma. ― En somero ni trovas malvarmeton en densaj arbaroj. ― Li sidas apud la tablo kaj dormetas. ― Mallarĝa vojeto kondukas tra tiu ĉi kampo al nia domo. ― Sur lia vizaĝo mi vidis ĝojan rideton. ― Kun bruo oni malfermis la pordegon, kaj la kaleŝo enveturis en la korton. Tio ĉi estis jam ne simpla pluvo, sed pluvego. ― Grandega hundo metis sur min sian antaŭan piedegon, kaj mi de teruro ne sciis, kion fari, ― Antaŭ nia militistaro staris granda serio da pafilegoj. ― Johanon, Nikolaon, Erneston, Vilhelmon, Marion, Klaron kaj Sofion iliaj gepatroj nomas Johanĉjo (aŭ Joĉjo), Nikolĉjo (aŭ Nikoĉjo aŭ Nikĉjo aŭ Niĉjo), Erneĉjo (aŭ Erĉjo), Vilhelĉjo (aŭ Vilheĉjo aŭ Vilĉjo aŭ Viĉjo), Manjo (aŭ Marinjo), Klanjo kaj Sonjo (aŭ Sofinjo).

densa épais, dense | dense | dicht | густой | gęsty.
brui faire du bruit | make a noise | lärmen, brausen | шумѣть | szumieć, hałasować.
kaleŝo carosse, calèche | carriage | Wagen | коляска | powóz.
pluvo pluie | rain | Regen | дождь | deszcz.
pafi tirer, faire feu | shoot | schiessen | стрѣлять | strzelać.
ĉj, nj après les 1-6 premières lettres d’un prénom masculin (nj ― féminin) lui donne un caractère diminutif et caressant | affectionate diminutive of masculine (nj ― feminine) names | den ersten 1-6 Buchstaben eines männlichen (nj ― weiblichen) Eigennamens beigefügt, verwandelt diesen in ein Kosewort | приставленное къ первымъ 1-6 буквамъ имени собственнаго мужескаго (nj ― женскаго) пола, превращаетъ его въ ласкательное | dodane do pierwszych 1-6 liter imienia własnego męskiego (nj ― żenńskiego) rodzaju zmienia je w pieszczotliwe.

§39

En la kota vetero mia vesto forte malpuriĝis; tial mi prenis broson kaj purigis la veston. ― Li paliĝis de timo kaj poste li ruĝiĝis de honto. ― Li fianĉiĝis kun fraŭlino Berto; post tri monatoj estos la edziĝo; la edziĝa soleno estos en la nova preĝejo, kaj la edziĝa festo estos en la domo de liaj estontaj bogepatroj. ― Tiu ĉi maljunulo tute malsaĝiĝis kaj infaniĝis. ― Post infekta malsano oni ofte bruligas la vestojn de la malsanulo. ― Forigu vian fraton, ĉar li malhelpas al ni. ― Ŝi edziniĝis kun sia kuzo, kvankam ŝiaj gepatroj volis ŝin edzinigi kun alia persono. ― En la printempo la glacio kaj la neĝo fluidiĝas. ― Venigu la kuraciston, ĉar mi estas malsana. ― Li venigis al si el Berlino multajn librojn. ― Mia onklo ne mortis per natura morto, sed li tamen ne mortigis sin mem kaj ankaŭ estis mortigita de neniu; unu tagon, promenante apud la reloj de fervojo, li falis sub la radojn de veturanta vagonaro kaj mortiĝis. ― Mi ne pendigis mian ĉapon sur tiu ĉi arbeto; sed la vento forblovis de mia kapo la ĉapon, kaj ĝi, flugante, pendiĝis sur la branĉoj de la arbeto. ― Sidigu vin (aŭ sidiĝu), sinjoro! ― La junulo aliĝis al nia militistaro kaj kuraĝe batalis kune kun ni kontraŭ niaj malamikoj.

koto boue | dirt | Koth, Schmutz | грязь | błoto.
broso brosse | brush | Bürste | щетка | szczotka.
ruĝa rouge | red | roth | красный | czerwony.
honti avoir honte | be ashamed | sich schämen | стыдиться | wstydzić się.
solena solennel | solemn | feierlich | торжественный | uroczysty.
infekti infecter | infect | anstecken | заражать | zarażać.
printempo printemps | spring time | Frühling | весна | wiosna.
relo rail | rail | Schiene | рельса | szyna.
rado roue | wheel | Rad | колесо | koło (od woza i t. p.).
pendi pendre, être suspendu | hang | hängen (v. n.) | висѣть | wisieć.
ĉapo bonnet | bonnet | Mütze | шапка | czapka.
vento vent | wind | Wind | вѣтеръ | wiatr.
blovi souffler | blow | blasen, wehen | дуть | dąć, dmuchać.
kapo tête | head | Kopf | голова | głowa.
branĉo branche | branch | Zweig | вѣтвь | gałaź.

§40

En la daŭro de kelke da minutoj mi aŭdis du pafojn. ― La pafado daŭris tre longe. ― Mi eksaltis de surprizo. ― Mi saltas tre lerte. ― Mi saltadis la tutan tagon de loko al loko. ― Lia hieraŭa parolo estis tre bela, sed la tro multa parolado lacigas lin. ― Kiam vi ekparolis, ni atendis aŭdi ion novan, sed baldaŭ ni vidis, ke ni trompiĝis, ― Li kantas tre belan kanton. ― La kantado estas agrabla okupo. ― La diamanto havas belan brilon. ― Du ekbriloj de fulmo trakuris tra la malluma ĉielo. ― La domo, en kiu oni lernas, estas lernejo, kaj la domo, en kiu oni preĝas, estas preĝejo. ― La kuiristo sidas en la kuirejo. ― La kuracisto konsilis al mi iri en ŝvitbanejon. ― Magazeno, en kiu oni vendas cigarojn, aŭ ĉambro, en kiu oni tenas cigarojn, estas cigarejo; skatoleto aŭ alia objekto, en kiu oni tenas cigarojn, estas cigarujo; tubeto, en kiun oni metas cigaron, kiam oni ĝin fumas, estas cigaringo. ― Skatolo, en kiu oni tenas plumojn, estas plumujo, kaj bastoneto, sur kiu oni tenas plumon por skribado, estas plumingo. ― En la kandelingo sidis brulanta kandelo. ― En la poŝo de mia pantalono mi portas monujon, kaj en la poŝo de mia surtuto mi portas paperujon; pli grandan paperujon mi portas sub la brako. ― La rusoj loĝas en Rusujo kaj la germanoj en Germanujo.

surprizi surprendre | surprise | überraschen | дѣлать сюрпризъ | robić niespodzianki.
laca las, fatigué | weary | müde | усталый | zmęczony.
trompi tromper, duper | deceive, cheat | betrügen | обманывать | oszukiwać.
fulmo éclair | lightning | Blitz | молнія | błyskawica.
ŝviti suer | perspire | schwitzen | потѣть | pocić się.
bani baigner | bath | baden | купать | kąpać.
magazeno magazin | store | Kaufladen | лавка, магазинъ | sklep, magazyn.
vendi vendre | sell | verkaufen | продавать | sprzedawać.
cigaro cigare | cigar | Cigarre | сигара | cygaro.
tubo tuyau | tube | Röhre | труба | rura.
fumo fumée | smoke | Rauch | дымъ | dym.
ing marque l’objet dans lequel se met, ou mieux s’introduit...; ex. kandelo chandelle ― kandelingo chandelier | holder for; e. g. kandelo candle ― kandelingo candle-stick | Gegenstand, in den etwas eingestellt, eingesetzt wird; z. B. kandelo Kerze ― kandelingo Leuchter | вещь, въ которую вставляется, всаживается; напр. kandelo свѣча ― kandelingo подсвѣчникъ | przedmiot, w który się coś wsadza, wstawia; np. kandelo świeca ― kandelingo lichtarz.
skatolo boîte | small box, case | Büchse, Schachtel | коробка | pudełko.
pantalono pantalon | pantaloons, trowsers | Hosen | брюки | spodnie.
surtuto redingote | over-coat | Rock | сюртукъ | surdut.
brako bras | arm | Arm | рука, объятія | ramię.

§41

Ŝtalo estas fleksebla, sed fero ne estas fleksebla. ― Vitro estas rompebla kaj travidebla. ― Ne ĉiu kreskaĵo estas manĝebla. ― Via parolo estas tute nekomprenebla kaj viaj leteroj estas ĉiam skribitaj tute nelegeble. ― Rakontu al mi vian malfeliĉon, ĉar eble mi povos helpi al vi. ― Li rakontis al mi historion tute nekredeblan. ― Ĉu vi amas vian patron? Kia demando! kompreneble, ke mi lin amas. ― Mi kredeble ne povos veni al vi hodiaŭ, ĉar mi pensas, ke mi mem havos hodiaŭ gastojn. ― Li estas homo ne kredinda. ― Via ago estas tre laŭdinda. ― Tiu ĉi grava tago restos por mi ĉiam memorinda. ― Lia edzino estas tre laborema kaj ŝparema, sed ŝi estas ankaŭ tre babilema kaj kriema. Li estas tre ekkolerema kaj ekscitiĝas ofte ĉe la plej malgranda bagatelo; tamen li estas tre pardonema, li ne portas longe la koleron kaj li tute ne estas venĝema. ― Li estas tre kredema: eĉ la plej nekredeblajn aferojn, kiujn rakontas al li la plej nekredindaj homoj, li tuj kredas. ― Centimo, pfenigo kaj kopeko estas moneroj. ― Sablero enfalis en mian okulon. ― Li estas tre purema, kaj eĉ unu polveron vi ne trovos sur lia vesto. ― Unu fajrero estas sufiĉa, por eksplodigi pulvon.

ŝtalo acier | steet | Stahl | сталь | stal.
fleksi fléchir, ployer | bend | biegen | гнуть | giąć.
vitro verre (matière) | glass (substance) | Glas | стекло | szkło.
rompi rompre, casser | break | brechen | ломать | łamać.
laŭdi louer, vanter | praise | loben | хвалить | chwalić.
memori se souvenir, se rappeler | remember | im Gedächtniss behalten, sich erinnern | помнить | pamiętać.
ŝpari ménager, épargner | be sparing | sparen | сберегать | oszczędzać.
bagatelo bagatelle | trifle, toy | Kleinigkeit | мелочь, бездѣлица | drobnostka.
venĝi se venger | revenge | rächen | мстить | mścić się.
eksciti exciter, émouvoir | excite | erregen | возбуждать | wzbudzać.
er marque l’unité; ex. sablo sable ― sablero un grain de sable | one of many objects of the same kind; e. g. sablo sand ― sablero grain of sand | ein einziges; z. B. sablo Sand ― sablero Sandkörnchen | отдѣльная единица; напр. sablo песокъ ― sablero песчинка | oddzielna jednostka; np. sablo piasek ― sablero ziarnko piasku.
polvo poussière | dust | Staub | пыль | kurz.
fajro feu | fire | Feuer | огонь | ogień.
eksplodi faire explosion | explode | explodiren | взрывать | wybuchać.
pulvo poudre à tirer | gunpowder | Pulver (Schiess-) | порохъ | proch.

§42

Ni ĉiuj kunvenis, por priparoli tre gravan aferon; sed ni ne povis atingi ian rezultaton, kaj ni disiris. ― Malfeliĉo ofte kunigas la homojn, kaj feliĉo ofte disigas ilin. ― Mi disŝiris la leteron kaj disĵetis ĝiajn pecetojn en ĉiujn angulojn de la ĉambro. ― Li donis al mi monon, sed mi ĝin tuj redonis al li. ― Mi foriras, sed atendu min, ĉar mi baldaŭ revenos. ― La suno rebrilas en la klara akvo de la rivero. ― Mi diris al la reĝo: via reĝa moŝto, pardonu min! ― El la tri leteroj unu estis adresita: al Lia Episkopa Moŝto, Sinjoro N.; la dua: al Lia Grafa Moŝto, Sinjoro P.; la tria: al Lia Moŝto, Sinjoro D. ― La sufikso «um» ne havas difinitan signifon, kaj tial la (tre malmultajn) vortojn kun «um» oni devas lerni, kiel simplajn vortojn. Ekzemple: plenumi, kolumo, manumo. ― Mi volonte plenumis lian deziron. ― En malbona vetero oni povas facile malvarmumi. ― Sano, sana, sane, sani, sanu, saniga, saneco, sanilo, sanigi, saniĝi, sanejo, sanisto, sanulo, malsano, malsana, malsane, malsani, malsanulo, malsaniga, malsaniĝi, malsaneta, malsanema, malsanulejo, malsanulisto, malsanero, malsaneraro, sanigebla, sanigisto, sanigilo, resanigi, resaniĝanto, sanigilejo, sanigejo, malsanemulo, sanilaro, malsanaro, malsanulido, nesana, malsanado, sanulaĵo, malsaneco, malsanemeco, saniginda, sanilujo, sanigilujo, remalsano, remalsaniĝo, malsanulino, sanigista, sanigilista, sanilista, malsanulista k. t. p.

atingi atteindre | attain, reach | erlangen, erreichen | достигать | dosięgać.
rezultato résultat | result | Ergebniss | результатъ | rezultat.
ŝiri déchirer | tear, rend | reissen | рвать | rwać.
peco morceau | piece | Stück | кусокъ | kawał.
moŝto titre commun | universal title | allgemeiner Titel | общій титулъ | Mość.
episkopo évêque | bishop | Bischof | епископъ | biskup.
grafo comte | earl, count | Graf | графъ | hrabia.
difini définir, déterminer | define | bestimmen | опредѣлять | wyznaczać, określać.

\cleardoublepage
