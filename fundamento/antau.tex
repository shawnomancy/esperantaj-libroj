\label{antau}
\fancyhead[LE,RO]{\footnotesize\thepage}
\fancyhead[CE]{\footnotesize\narrow{\leftmark}}
\fancyhead[CO]{\footnotesize\narrow{\rightmark}}
\markboth{FUNDAMENTO DE ESPERANTO}{ANTAŬPAROLO}
\thispagestyle{plain}

\vspace*{1.35em}

\begin{center}

\scalebox{0.6}[1]{\Huge\bf FUNDAMENTO DE ESPERANTO}

\vspace{1em}

\rule{0.9\textwidth}{0.4pt}

\vspace{3em}

\Large\bookman{Antaŭparolo}
\end{center}
\vspace{1ex}

Por ke lingvo internacia povu bone kaj regule progresadi kaj por ke ĝi havu plenan certecon, ke ĝi neniam disfalos kaj ia facilanima paŝo de ĝiaj amikoj estontaj ne detruos la laborojn de ĝiaj amikoj estintaj, ― estas plej necesa antaŭ ĉio unu kondiĉo: la ekzistado de klare difinita, neniam tuŝebla kaj neniam ŝanĝebla \textbf{Fundamento} de la lingvo. Kiam nia lingvo estos oficiale akceptita de la \emph{registaroj} de la plej ĉefaj regnoj kaj tiuj ĉi registaroj per speciala \emph{leĝo} garantios al Esperanto tute certan vivon kaj uzatecon kaj plenan sendanĝerecon kontraŭ ĉiuj personaj kapricoj aŭ disputoj, tiam aŭtoritata komitato, interkonsente elektita de tiuj registaroj, havos la rajton fari en la fundamento de la lingvo unu fojon por ĉiam ĉiujn deziritajn ŝanĝojn, \emph{se} tiaj ŝanĝoj montriĝos necesaj; sed \emph{ĝis tiu tempo} la fundamento de Esperanto devas plej severe resti absolute senŝanĝa, ĉar severa netuŝebleco de nia fundamento estas la plej grava kaŭzo de nia ĝisnuna progresado kaj la plej grava kondiĉo por nia regula kaj paca progresado estonta. \emph{Neniu persono kaj neniu societo devas havi la rajton arbitre fari en nia Fundamento iun eĉ plej malgrandan ŝanĝon!} Tiun ĉi tre gravan principon la esperantistoj volu ĉiam bone memori kaj kontraŭ la ektuŝo de tiu ĉi principo ili volu ĉiam energie batali, ĉar la momento, en kiu ni ektuŝus tiun principon, estus la komenco de nia morto.

Laŭ silenta interkonsento de ĉiuj esperantistoj jam de tre longa tempo la sekvantaj tri verkoj estas rigardataj kiel fundamento de Esperanto: 1.) La 16-regula \emph{gramatiko}; 2) la «~\emph{Universala Vortaro}~»; 3) la «~\emph{Ekzercaro}~». Tiujn ĉi tri verkojn la aŭtoro de Esperanto rigardadis ĉiam kiel \emph{leĝojn} por li, kaj malgraŭ oftaj tentoj kaj delogoj li neniam permesis al si (almenaŭ \emph{konscie}) eĉ la plej malgrandan pekon kontraŭ tiuj ĉi leĝoj; li esperas, ke pro la bono de nia afero ankaŭ ĉiuj aliaj esperantistoj ĉiam rigardados tiujn ĉi tri verkojn kiel la solan leĝan kaj netuŝeblan fundamenton de Esperanto.

Por ke ia regno estu forta kaj glora kaj povu sane disvolviĝadi, estas necese, ke ĉiu regnano sciu, ke li neniam dependos de la kapricoj de tiu aŭ alia persono, sed devas obei ĉiam nur klarajn, tute difinitajn fundamentajn \emph{leĝojn} de sia lando, kiuj estas egale devigaj por la regantoj kaj regatoj kaj en kiuj neniu havas la rajton fari arbitre laŭ persona bontrovo ian ŝanĝon aŭ aldonon. Tiel same por ke nia afero bone progresadu, estas necese, ke ĉiu esperantisto havu la plenan certecon, ke leĝodonanto por li ĉiam estos ne ia \emph{persono}, sed ia klare difinita \emph{verko}. Tial, por meti finon al ĉiuj malkompreniĝoj kaj disputoj, kaj por ke ĉiu esperantisto sciu tute klare, per kio li devas en ĉio sin gvidi, la aŭtoro de Esperanto decidis nun eldoni en formo de unu libro tiujn tri verkojn, kiuj laŭ silenta interkonsento de ĉiuj esperantistoj jam de longe fariĝis fundamento por Esperanto, kaj li petas, ke la okuloj de ĉiuj esperantistoj estu ĉiam turnataj ne al li, sed al \emph{tiu ĉi libro.} Ĝis la tempo, kiam ia por ĉiuj aŭtoritata kaj nedisputebla institucio decidos alie, ĉio, kio troviĝas en tiu ĉi libro, devas esti rigardata kiel deviga por ĉiuj; ĉio, kio estas kontraŭ tiu ĉi libro, devas esti rigardata kiel malbona, se ĝi eĉ apartenus al la plumo de la aŭtoro de Esperanto mem. Nur la supre nomitaj tri verkoj publikigitaj en la libro «~Fundamento de Esperanto~», devas esti rigardataj kiel oficialaj; ĉio alia, kion mi verkis aŭ verkos, konsilas, korektas, aprobas k.~t.~p., estas nur verkoj \emph{privataj}, kiujn la esperantistoj ― se ili trovas tion ĉi utila por la unueco de nia afero ― povas rigardadi kiel \emph{modela}, sed ne kiel \emph{deviga}.

Havante la karakteron de \emph{fundamento}, la tri verkoj represitaj en tiu ĉi libro devas antaŭ ĉio esti \emph{netuŝeblaj}. Tial la legantoj ne miru, ke ili trovos en la nacia traduko de diversaj vortoj en tiu ĉi libro (precipe en la angla parto) tute nekorektite tiujn samajn \emph{erarojn}, kiuj sin trovis en la unua eldono de la «~Universala Vortaro~». Mi permesis al mi nur korekti la \emph{preserarojn}; sed se ia vorto estis erare aŭ nelerte \emph{tradukita}, mi ĝin lasis en tiu ĉi libro tute senŝanĝe; ĉar se mi volus plibonigi, tio ĉi jam estus \emph{ŝanĝo}, kiu povus kaŭzi disputojn kaj kiu en verko fundamenta ne povas esti tolerata. \emph{La fundamento devas resti severe netuŝebla eĉ kune kun siaj eraroj.} La \emph{erareco} en la nacia traduko de tiu aŭ alia vorto ne prezentas grandan malfeliĉon, ĉar, komparante la kuntekstan tradukon en la aliaj lingvoj, oni facile trovos la veran sencon de ĉiu vorto; sed senkompare pli grandan danĝeron prezentus la \emph{ŝanĝado} de la traduko de ia vorto, ĉar, perdinte la severan netuŝeblecon, la verko perdus sian eksterordinare necesan karakteron de dogma fundamenteco, kaj, trovante en unu eldono alian tradukon ol en alia, la uzanto ne havus la certecon, ke mi morgaŭ ne faros ian alian ŝanĝon, kaj li perdus sian konfidon kaj apogon. Al ĉiu, kiu montros al mi ian nebonan esprimon en la Fundamenta libro, mi respondos trankvile: jes, ĝi estas eraro, sed ĝi devas resti netuŝebla, ĉar ĝi apartenas al la fundamenta dokumento, en kiu neniu havas la rajton fari ian ŝanĝon.

La «~Fundamento de Esperanto~» tute ne devas esti rigardata kiel la plej bona lernolibro kaj vortaro de Esperanto. Ho, ne! Kiu volas \emph{perfektiĝi} en Esperanto, al tiu mi rekomendas la diversajn lernolibrojn kaj vortarojn, multe \emph{pli bonajn} kaj \emph{pli vastajn}, kiuj estas eldonitaj de niaj plej kompetentaj amikoj por ĉiu nacio aparte kaj el kiuj la plej gravaj estas eldonitaj tre bone kaj zorgeme, sub mia persona kontrolo kaj kunhelpo. Sed la «~Fundamento de Esperanto~» devas troviĝi en la manoj de ĉiu bona esperantisto kiel konstanta \emph{gvida dokumento}, por ke li bone ellernu kaj per ofta enrigardado konstante memorigadu al si, kio en nia lingvo estas oficiala kaj netuŝebla, por ke li povu ĉiam bone distingi la vortojn kaj regulojn \emph{oficialajn}, kiuj devas troviĝi en ĉiuj lernoverkoj de Esperanto, de la vortoj kaj reguloj rekomendataj \emph{private}, kiuj eble ne al ĉiuj esperantistoj estas konataj aŭ eble ne de ĉiuj estas aprobataj. La «~Fundamento de Esperanto~» devas troviĝi en la manoj de ĉiu esperantisto kiel konstanta \emph{kontrolilo}, kiu gardos lin de deflankiĝado de la vojo de unueco.

Mi diris, ke la fundamento de nia lingvo devas esti absolute netuŝebla, se eĉ ŝajnus al ni, ke tiu aŭ alia punkto estas sendube erara. Tio ĉi povus naski la penson, ke nia lingvo restos ĉiam rigida kaj neniam disvolviĝos... Ho, ne! Malgraŭ la severa netuŝebleco de la fundamento, nia lingvo havos la plenan eblon ne sole konstante riĉiĝadi, sed eĉ konstante \emph{pliboniĝadi} kaj \emph{perfektiĝadi}; la netuŝebleco de la fundamento nur garantios al ni konstante, ke tiu perfektiĝado fariĝados ne per arbitra, interbatala kaj ruiniga \emph{rompado} kaj \emph{ŝanĝado}, ne per nuligado aŭ sentaŭgigado de nia ĝisnuna literaturo, sed per vojo \emph{natura}, senkonfuza kaj sendanĝera. Pli detale mi parolos pri tio ĉi en la Bulonja kongreso; nun mi diros pri tio ĉi nur kelkajn vortojn, por ke mia opinio ne ŝajnu tro paradoksa:

1) \emph{Riĉigadi} la lingvon per novaj vortoj oni povas jam \emph{nun}, per konsiliĝado kun tiuj personoj, kiuj estas rigardataj kiel la plej aŭtoritataj en nia lingvo, kaj zorgante pri tio, ke ĉiuj uzu tiujn vortojn en la sama formo; sed tiuj ĉi vortoj devas esti nur rekomendataj, ne altrudataj; oni devas ilin uzadi nur en la \emph{literaturo}; sed en korespondado kun personoj \emph{nekonataj} estas bone ĉiam peni uzadi nur vortojn el la «~Fundamento~» ĉar nur pri tiaj vortoj ni povas esti certaj, ke nia adresato ilin nepre trovos en sia vortaro. Nur iam poste, kiam la plej granda parto de la novaj vortoj estos jam tute matura, ia aŭtoritata institucio enkondukos ilin en la vortaron \emph{oficialan}, kiel «~Aldonon al la Fundamento~»

2) Se ia aŭtoritata centra institucio trovos, ke tiu aŭ alia vorto aŭ regulo en nia lingvo estas \emph{tro neoportuna}, ĝi ne devos \emph{forigi} aŭ \emph{ŝanĝi} la diritan formon, sed ĝi povos proponi formon novan, kiun ĝi rekomendos uzadi \emph{paralele} kun la formo malnova. Kun la tempo la formo nova iom post iom elpuŝos la formon malnovan, kiu fariĝos \emph{arĥaismo}, kiel ni tion ĉi vidas en ĉiu natura lingvo. Sed, prezentante parton de la \emph{fundamento}, tiuj ĉi arĥaismoj neniam estos elĵetitaj, sed ĉiam estos presataj en ĉiuj lernolibroj kaj vortaroj samtempe kun la formoj novaj, kaj tiamaniere ni havos la certecon, ke eĉ ĉe la plej granda perfektiĝado la unueco de Esperanto neniam estos rompata kaj neniu verko Esperanta eĉ el la plej frua tempo iam perdos sian valoron kaj kompreneblecon por la estontaj generacioj.

Mi montris en \emph{principo}, kiamaniere la severa netuŝebleco de la «~Fundamento~» gardos ĉiam la unuecon de nia lingvo, ne malhelpante tamen al la lingvo ne sole riĉiĝadi, sed eĉ konstante \emph{perfektiĝadi}. Sed en la \emph{praktiko} ni (pro kaŭzoj jam multajn fojojn priparolitaj) devas kompreneble esti \emph{tre singardaj} kun ĉia «~perfektigado~» de la lingvo: a) ni devas tion ĉi fari ne facilanime, sed nur en okazoj de efektiva \emph{neceseco}; b) fari tion ĉi (post matura prijuĝado) povas ne apartaj personoj, sed nur ia centra institucio, kiu havos nedisputeblan aŭtoritatecon por la tuta esperantistaro.

Mi finas do per la jenaj vortoj:

1. pro la unueco de nia afero ĉiu bona esperantisto devas antaŭ ĉio bone koni la \emph{fundamenton} de nia lingvo;

2. la fundamento de nia lingvo devas resti por ĉiam \emph{netuŝebla};

3. ĝis la tempo kiam aŭtoritata centra institucio decidos \emph{pligrandigi} (neniam \emph{ŝanĝi}!) la ĝisnunan fundamenton per oficialigo de novaj vortoj aŭ reguloj, ĉio bona, kio ne troviĝas en la «~Fundamento de Esperanto~», devas esti rigardata ne kiel deviga, sed nur kiel rekomendata.

La ideoj, kiujn mi supre esprimis pri la Fundamento de Esperanto, prezentas dume nur mian \emph{privatan} opinion. Leĝan sankcion ili ricevos nur en tia okazo, se ili estos akceptitaj de la unua internacia kongreso de esperantistoj, al kiu tiu ĉi verko kune kun sia antaŭparolo estos prezentita.

\begin{flushright}
\small \fauxsc{L. Zamenhof}.~~~~~~~~~~~~~
\end{flushright}

{\footnotesize \hspace{3em} Varsovio, Julio 1905.}

\cleardoublepage

