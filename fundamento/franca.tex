%
% Esperanta gramatiko por francoj
%
\mainmatter
\label{gram:franca}
\markright{GRAMMAIRE}
\thispagestyle{plain}
\begin{center}
\phantomsection
\narrow{\huge\bf \spaceoutless{FUNDAMENTA GRAMATIKO}}
\addcontentsline{toc}{chapter}{Fundamenta Gramatiko}
\vspace{1em}

\narrow{DE LA LINGVO ESPERANTO}
\vspace{1em}

\narrow{\LARGE {\spaceoutmed{EN KVIN LINGVOJ}}}

\rule{0.9\textwidth}{0.4pt}
\vspace{2em}

\phantomsection
\narrow{\LARGE\bf \spaceoutless{GRAMMAIRE}}
\addcontentsline{toc}{section}{Grammaire (Gramatiko Franca)}
\selectlanguage{french}

\rule{13mm}{0.4pt}\\[1em]

{\large \gramsec{A) ALPHABET}}
\vspace{1em}

\setstretch{1}
\begin{tabu} to \textwidth{+Y@{}ZY@{}ZY@{}ZY@{}ZY@{}ZY@{}ZY}
\rowstyle{\Large\arbfont} Aa, & Bb, & Cc, & Ĉĉ, & Dd, & Ee, & Ff, \\
\rowstyle{\footnotesize} â & b & ts (tsar) & tch (tchèque) & d & é & f \\[1ex]
\rowstyle{\Large\arbfont} Gg, & Ĝĝ, & Hh, & Ĥĥ, & Ii, & Jj, & Ĵĵ, \\
\rowstyle{\footnotesize} g dur~(gant) & dj (adjudant) & h légère\-ment aspiré & h~forte\-ment aspiré & i & y (yeux) & j \\[1ex]
\rowstyle{\Large\arbfont} Kk, & Ll, & Mm, & Nn, & Oo, & Pp, & Rr, \\
\rowstyle{\footnotesize} k & l & m & n & ô & p & r \\[1ex]
\rowstyle{\Large\arbfont} Ss, & Ŝŝ, & Tt, & Uu, & Ŭŭ, & Vv, & Zz. \\
\rowstyle{\footnotesize} ss, ç & ch (chat) & t & ou & ou bref (dans~l‘alle\-mand~„laut„) & v & z 
\end{tabu}
\end{center}

{\footnotesize \br{Remarque.} ― Les typographies qui n’ont pas les caractères \emph{ĉ, ĝ, ĥ, ĵ, ŝ, ŭ,} peuvent les remplacer par \emph{ch, gh, hh, jh, sh, u.}}
\begin{center}
\large \gramsec{B) PARTIES DU DISCOURS}
\end{center}

\textbf{1.} L’Esperanto n’a qu’un \textbf{article défini} (\emph{la}), invariable pour tous les genres, nombres et cas. Il n’a pas d’article indéfini.

{\footnotesize \br{Remarque.} ― L’emploi de l’article est le même qu’en français ou en allemand. Mais les personnes auxquelles il présenterait quelque difficulté peuvent fort bien ne pas s’en servir.}

\textbf{2.} Le \textbf{substantif} finit toujours par \emph{o}. Pour former le pluriel on ajoute \emph{j} au singulier. La langue n’a que deux cas: le \emph{nominatif} et \emph{l’accusatif}. Ce dernier se forme du nominatif par l’addition d’un \emph{n}. Les autres cas sont marqués par des prépositions: le \emph{génitif} par \emph{de} (de), le \emph{datif} par \emph{al} (à), l’\emph{ablatif} par \emph{per} (par, au moyen de) ou par d’autres prépositions, selon le sens. Ex.: \emph{la patr$'$o} ― le père; \emph{al la patr$'$o} ― au père, \emph{de la patr$'$o} ― du père, \emph{la patr$'$o$'$n} ― le père (à l’accusatif, c.-à-d. complément direct), \emph{per la patr$'$o$'$j} ― par les pères ou au moyen des pères, \emph{la patr$'$o$'$j$'$n} ― les pères (accus. plur.), \emph{por la patr$'$o} ― pour le père, \emph{kun la patr$'$o} ― avec le père, etc.

\textbf{3. L’adjectif} finit toujours par \emph{a}. Ses cas et ses nombres se marquent de la même manière que ceux du substantif. Le \emph{comparatif} se forme à l’aide du mot \emph{pli} ― plus, et le \emph{superlatif} à l’aide du mot \emph{plej} ― le plus. Le „que“ du comparatif se traduit par \emph{„ol“} et le „de“ du superlatif par \emph{„el“} (d’entre). Ex.: \emph{pli blank$'$a ol neĝ$'$o} ― plus blanc que neige; \emph{mi hav$'$as la plej bel$'$a$'$n patr$'$in$'$o$'$n el ĉiu$'$j} ― j’ai la plus belle mère de toutes.

\textbf{4.} Les \emph{adjectifs} \textbf{numéraux} \emph{cardinaux} sont invariables: \emph{unu} (1), \emph{du} (2), \emph{tri} (3), \emph{kvar} (4), \emph{kvin} (5), \emph{ses} (6), \emph{sep} (7), \emph{ok} (8), \emph{naŭ} (9), \emph{dek} (10), \emph{cent} (100), \emph{mil} (1000). Le dizaines et les centaines se forment par la simple réunion des dix premiers nombres. Aux adjectifs numéraux cardinaux on ajoute: la terminaison (\emph{a}) de l’adjectif, pour les \emph{numéraux ordinaux;} \emph{obl}, pour les numéraux \emph{multiplicatifs}; \emph{on}, pour les numéraux \emph{fractionnaires}; \emph{op}, pour les numéraux \emph{collectifs}. On met \emph{po} avant ces nombres pour marquer les numéraux \emph{distributifs}. Enfin, dans la langue, les adjectifs numéraux peuvent s’employer substantivement ou adverbialement. Ex.: \emph{Kvin$'$cent tri$'$dek tri} ― 533; \emph{kvar$'$a} ― 4\textsuperscript{me}; \emph{tri$'$obl$'$a} ― triple; \emph{kvar$'$on$'$o} ― un quart; \emph{du$'$op$'$e} ― à deux; \emph{po kvin} ― au taux de cinq (chacun); \emph{unu$'$o} ― (l’) unité; \emph{sep$'$e} ― septièmement.

\textbf{5.} Les \textbf{pronoms} personnels sont: \emph{mi} (je, moi), \emph{vi} (vous, tu, toi), \emph{li} (il, lui), \emph{ŝi} (elle), \emph{ĝi} (il, elle, pour les animaux ou les choses), \emph{si} (soi), \emph{ni} (nous), \emph{ili} (ils, elles), \emph{oni} (on). Pour en faire des adjectifs ou des pronoms possessifs, on ajoute la terminaison (\emph{a}) de l’adjectif. Les pronoms se déclinent comme le substantif. Ex.: \emph{mi$'$n} ― moi, me (accus.), \emph{mi$'$a} ― mon, \emph{la vi$'$a$'$j} ― les vôtres.

\textbf{6.} Le \textbf{verbe} ne change ni pour les personnes, ni pour les nombres. Ex.: \emph{mi far$'$as} ― je fais, \emph{la patr$'$o far$'$as} ― le père fait, \emph{ili far$'$as} ― ils font.

\begin{center}
\bf Formes du verbe:
\end{center}

a) Le \emph{présent} est caractérisé par \emph{as}; ex.: \emph{mi far$'$as} ― je fais.

b) Le \emph{passé}, par \emph{is}: \emph{vi far$'$is} ― vous faisiez, vous avez fait.

c) Le \emph{futur}, par \emph{os}: \emph{ili far$'$os} ― ils feront.

ĉ) Le \emph{conditionnel}, par \emph{us}: \emph{ŝi far$'$us} ― elle ferait.

d) L’\emph{impératif}, par \emph{u}: \emph{far$'$u} ― fais, faites; \emph{ni far$'$u} ― faisons.

e) L’\emph{infinitif}, par \emph{i}: \emph{far$'$i} ― faire.

f) Le \emph{participe présent actif}, par \emph{ant}: \emph{far$'$ant$'$a} ― faisant, \emph{far$'$ant$'$e} ― en faisant.

g) Le \emph{participe passé actif}, par \emph{int}: \emph{far$'$int$'$a} ― ayant fait.

ĝ) Le \emph{participe futur actif}, par \emph{ont}: \emph{far$'$ont$'$a} ― devant faire, qui fera.

h) Le \emph{participe présent passif}, par \emph{at}: \emph{far$'$at$'$a} ― étant fait, qu’on fait.

ĥ) Le \emph{participe passé passif}, par \emph{it}: \emph{far$'$it$'$a} - ayant été fait, qu’on a fait.

i) Le \emph{participe futur passif}, par \emph{ot}: \emph{far$'$ot$'$a} ― devant être fait, qu’on fera.

La voix passive n’est que la combinaison du verbe \emph{est} (être) et du participe présent ou passé du verbe passif donné. Le „de“ ou le „par“ du complément indirect se rendent par \emph{de}. Ex.: \emph{ŝi est$'$as am$'$at$'$a de ĉiu$'$j} ― elle est aimée de tous (part. prés.: la chose se fait). \emph{La pord$'$o est$'$as ferm$'$it$'$a} ― la porte est fermée (part. pas.: la chose a été faite).

\textbf{7.} L’\textbf{adverbe} est caractérisé par \emph{e}. Ses degrés de comparaison se marquent de la même manière que ceux de l’adjectif. Ex.: \emph{mi$'$a frat$'$o pli bon$'$e kant$'$as ol mi} ― mon frère chante mieux que moi.

\textbf{8.} Toutes les \textbf{prépositions} veulent, par elles-mêmes, le nominatif.

\begin{center}
\large \gramsec{C) RÈGLES GÉNÉRALES}
\end{center}

\textbf{9.} Chaque mot se prononce absolument comme il est écrit.

\textbf{10.} L’accent tonique se place toujours sur l’avant-dernière syllabe.

\textbf{11.} Les mots composés s’obtiennent par la simple réunion des éléments qui les forment, écrits ensemble, mais séparés par de petits traits\footnote{Dans les lettres ou dans les ouvrages, qui s’adressent à des personnes connaissant déjà la langue, on peut omettre ces petits traits. Ils ont pour but de permettre à tous de trouver aisément, dans le dictionnaire, le sens précis de chacun des éléments du mot, et d’en obtenir ainsi la signification complète, sans aucune étude préalable de la grammaire.}). Le mot fondamental doit toujours être à la fin. Les terminaisons grammaticales sont considérées comme des mots. Ex.: \emph{vapor$'$ŝip$'$o} (bateau à vapeur) est formé de: \emph{vapor} ― vapeur, \emph{ŝip} ― bateau, \emph{o} ― terminaison caractéristique du substantif.

\textbf{12.} S’il y a dans la phrase un autre mot de sens négatif, l’adverbe „ne“ se supprime. Ex.: \emph{mi neniam vid$'$is} ― je n’ai jamais vu.

\textbf{13.} Si le mot marque le lieu où l’on va, il prend la terminaison de l’accusatif. Ex.: \emph{kie vi est$'$as?} ― où êtes-vous? \emph{kie$'$n vi ir$'$as?} ― où allez-vous? \emph{Mi ir$'$as Pariz$'$o$'$n} ― je vais à Paris.

\textbf{14.} Chaque préposition possède, en Esperanto, un sens immuable et bien déterminé, qui en fixe l’emploi. Cependant, si le choix de celle-ci plutôt que de celle-là ne s’impose pas clairement à l’esprit, on fait usage de la préposition \emph{je} qui n’a pas de signification propre. Ex.: \emph{ĝoj$'$i je tio} ― s’en réjouir, \emph{rid$'$i je tio} ― en rire, \emph{enu$'$o je la patr$'$uj$'$o} ― regret de la patrie.

La clarté de la langue n’en souffre aucunement, car, dans toutes, on emploie, en pareil cas, une préposition quelconque, pourvu qu’elle soit sanctionnée par l’usage. L’Esperanto adopte pour cet office la seule préposition \emph{je}.

A sa place on peut cependant employer aussi l’accusatif sans préposition, quand aucune amphibologie n’est à craindre.

\textbf{15.} Les mots „étrangers“ c.-à-d. ceux que la plupart des langues ont empruntés à la même source, ne changent pas en Esperanto. Ils prennent seulement l’orthographe et les terminaisons grammaticales de la langue. Mais quand, dans une catégorie, plusieurs mots différents dérivent de la même racine, il vaut mieux n’employer que le mot fondamental, sans altération, et former les autres d’après les règles de la langue internationale. Ex.: tragédie ― \emph{tragedi$'$o}, tragique ― \emph{tragedi$'$a}.

\textbf{16.} Les terminaisons des substantifs et de l’article peuvent se supprimer et se remplacer par une apostrophe. Ex.: \emph{Ŝiller’} (Schiller) au lieu de \emph{Ŝiller$'$o;} \emph{de l’ mond$'$o} au lieu de \emph{de la mond$'$o}.

\newpage
