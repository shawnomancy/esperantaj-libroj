\fancyhead[C]{}
\titleformat{\chapter}[display]{\centering\sansfont}{\chaptertitlename}{0pt}{\Large}
\phantomsection
\chapter*{KOMENTO DE LA KOMPOSTANTO.}
\addcontentsline{toc}{chapter}{Komento de la kompostanto}

\begin{center}
\rule[0.5ex]{0.5\textwidth}{0.4pt}

\emph{Jen versio \laversio{} de ĉi tiu} \XeLaTeX{} \emph{versio.}
\end{center}

En miaj aliaj \XeLaTeX-aj versioj de la ĉefaj libroj de Esperanto (\emph{Unua Libro, Dua Libro, Fundamenta Krestomatio}), mia komento sekvas la reprodukton; sed la specialeco de la \emph{Fundamento} postulas, ke mia komento iru antaŭ la reprodukto, por zorge klarigi la aferon. Jen la plej grava punkto: Ĉi tiu dokumento \textbf{ne estas} la vera \emph{Fundamento de Esperanto} mem!

La \emph{Fundamento} estas netuŝebla; tial la Akademio de Esperanto diras (ĉe ilia retejo) «oni eldonas la Fundamenton normale nur en faksimila (fotokopiita) formo. Rekompostado ĉiam riskus falsi la tekston.» Jes ja, sed ankoraŭ, ĉar mi faris \XeLaTeX-ajn versiojn de la aliaj libroj, imitante la originalan tipografion, mi elektis fari ĉi tiun reprodukton ankaŭ de la \emph{Fundamento}, \textbf{kun la averto supre.}

Piednotoj en la \emph{Fundamenta Gramatiko}, kiuj estas montritaj per numeroj, estas en la originala. Piednotoj en la tuta libro, kiuj estas montritaj per nenumeraj simboloj (ekz. «\**» aŭ «†»), estas el mi.  Mi montras ŝajnajn erarojn en la originala, kaj diversajn lokojn, kie la \emph{Ekzercaro} en la \emph{Fundamento} diferencis el la fruaj eldonoj de la \emph{Ekzercaro} kiel memstaranta libro.  La pli novaj eldonoj de la \emph{Fundamento} montras la aferojn pli klare.

\vspace{1ex}

{\setlength{\parindent}{0em}
Shawn KNIGHT (angle elparolata \emph{ŝan najt})\\
\hodiau}
\titleformat{\chapter}[display]{\centering}{}{0pt}{\Large\bookman}