%
% Ekzercaro
%
\markboth{FUNDAMENTO DE ESPERANTO}{EKZERCARO}
\label{ekzercaro}

\thispagestyle{plain}
\vspace*{1em}
\begin{center}
\phantomsection
\narrow{\huge\bf \spaceoutless{EKZERCARO}}
\addcontentsline{toc}{chapter}{Ekzercaro}
\vspace{1em}

\bookman{\Large de la lingvo internacia «~Esperanto~»}
\selectlanguage{esperanto}

\rule{0.9\textwidth}{0.4pt}
\end{center}

\ekzsec[\vskip 1ex{\large\chunk\textbf{ALFABETO}}]{§1.}

\begin{center}
\setstretch{1}
\begin{tabu} to \textwidth{+Y@{}ZY@{}ZY@{}ZY@{}ZY@{}ZY@{}ZY}
\rowstyle{\Large\arbfont} Aa, & Bb, & Cc, & Ĉĉ, & Dd, & Ee, & Ff, \\[1ex]
\rowstyle{\Large\arbfont} Gg, & Ĝĝ, & Hh, & Ĥĥ, & Ii, & Jj, & Ĵĵ, \\[1ex]
\rowstyle{\Large\arbfont} Kk, & Ll, & Mm, & Nn, & Oo, & Pp, & Rr, \\[1ex]
\rowstyle{\Large\arbfont} Ss, & Ŝŝ, & Tt, & Uu, & Ŭŭ, & Vv, & Zz. \\[1ex]

\rowstyle{\LARGE\curve} A a, & B b, & C c, & \scriptC{} \scriptc{}, & D d, & E e, & F f, \\[1ex]
\rowstyle{\LARGE\curve} G g, & \scriptG{} \scriptg{}, & H h, & \scriptH{} \scripth{}, & I i, & J j, & \scriptJ{} \scriptj{}, \\[1ex]
\rowstyle{\LARGE\curve} K k, & L l, & M m, & N n, & O o, & P p, & R r, \\[1ex]
\rowstyle{\LARGE\curve} S s, & \scriptS{} \scripts{}, & T t, & U u, & \scriptU{} \scriptu{}, & V v, & Z z.
\end{tabu}
\end{center}

\textbf{Nomoj de la literoj:} a, bo, co, ĉo, do, e, fo, go, ĝo, ho, ĥo, i, jo, ĵo, ko, lo, mo, no, o, po, ro, so, ŝo, to, u, ŭo, vo, zo.

\ekzsec[\textbf{Ekzerco de legado.}]{§2.}

Al. Bá-lo. Pát-ro. Nú-bo. Cé-lo. Ci-tró-no. Cén-to. Sén-to. Scé-no. Scí-o. Có-lo. Kó-lo. O-fi-cí-ro. Fa-cí-la. Lá-ca. Pa-cú-lo. Ĉar. Ĉe-mí-zo. Ĉi-ká-no. Ĉi-é-lo. Ĉu. Fe-lí-ĉa. Cí-a. Ĉí-a. Pro-cé-so. Sen-ĉé-sa Ec. Eĉ. Ek. Da. Lú-do. Dén-to. Plén-di. El. En. De. Té-ni. Sen. Vé-ro. Fá-li. Fi-dé-la. Trá-fi. Gá-lo. Grán-da. Gén-to. Gíp-so. Gús-to. Lé-gi. Pá-go. Pá-ĝo.\footnoteE{La fruaj eldonoj de la \emph{Ekzercaro} enhavis ``Lé-ĝo'' ĉi tie.} Ĝis. Ĝús-ta. Ré-ĝi. Ĝar-dé-no. Lón-ga. Rég-no. Síg-ni. Gvar-dí-o. Lín-gvo. Ĝu-á-do. Há-ro. Hi-rún-do. Há-ki. Ne-hé-la. Pac-hó-ro. Ses-hó-ra Bat-hú-fo. Hó-ro. Ĥó-ro. Kó-ro. Ĥo-lé-ro. Ĥe-mí-o. I-mí-ti. Fí-lo. Bír-do. Tró-vi. Prin-tém-po. Min. Fo-í-ro. Fe-í-no. I-el. I-am. In. Jam. Ju. Jes. Ju-ris-to. Kra-jó-no. Ma-jés-ta. Tuj. Dó-moj. Ru-í-no. Prúj-no. Ba-lá-i. Pá-laj. De-í-no\footnoteE{Ĝi ŝajne devus esti aŭ «Fe-í-no» aŭ «Di-í-no.»}. Véj-no. Pe-ré-i. Mál-plej. Jús-ta. Ĵus. Ĵé-ti. Ĵa-lú-za. Ĵur-nálo. Má-jo. Bo-ná-ĵo. Ká-po. Ma-kú-lo. Kés-to. Su-ké-ro. Ak-vo. Ko-ké-to. Li-kvó-ro. Pac-ká-po.

\ekzsec[\textbf{Ekzerco de legado.}]{§3.}

Lá-vi. Le-ví-lo. Pa-ró-li. Mem. Im-plí-ki. Em-ba-rá-so. Nó-mo. In-di-fe-rén-ta. In-ter-na-cí-a. Ol. He-ró-i. He-ro-í-no. Fój-no. Pí-a. Pál-pi. Ri-pé-ti. Ar-bá-ro. Sá-ma. Stá-ri. Si-gé-lo. Sis-té-mo. Pe-sí-lo. Pe-zí-lo. Sén-ti. So-fís-mo. Ci-pré-so. Ŝi. Pá-ŝo. Stá-lo. Ŝtá-lo. Vés-to. Véŝ-to. Dis-ŝí-ri. Ŝan-cé-li. Ta-pí-ŝo Te-o-rí-o. Pa-tén-to. U-tí-la. Un-go. Plú-mo. Tu-múl-to. Plu. Lú-i. Kí-u. Ba-lá-u. Tra-ú-lo. Pe-ré-u. Ne-ú-lo. Fráŭ-lo. Paŭ-lí-no. Láŭ-di. Eŭ-ró-po. Tro-ú-zi. Ho-dí-aŭ. Vá-na. Vér-so. Sól-vi. Zór-gi. Ze-ní-to. Zo-o-lo-gí-o. A-zé-no. Me-zú-ro. Ná-zo. Tre-zó-ro. Mez-nók-to. Zú-mo. Sú-mo. Zó-no. Só-no. Pé-zo. Pé-co. Pé-so. Ne-ní-o. A-dí-aŭ. Fi-zí-ko. Ge-o-gra-fí-o. Spi-rí-to. Lip-há-ro. In-díg-ni. Ne-ní-el. Spe-gú-lo. Ŝpí-no.\footnoteE{La fruaj eldonoj de la \emph{Ekzercaro} enhavis ``Spí-no'' antaŭ ``Ŝpí-no''.} Né-i. Ré-e. He-ró-o. Kon-scí-i. Tra-e-té-ra. He-ro-é-to. Lú-e. Mó-le. Pá-le. Tra-í-re. Pa-sí-e. Me-tí-o. In-ĝe-ni-é-ro. In-sék-to. Re-sér-vi. Re-zér-vi.

\ekzsec[\textbf{Ekzerco de legado.}]{§4.}

Citrono. Cento. Sceno. Scio. Balau. Ŝanceli. Neniel. Embaraso. Zoologio. Reservi. Traire. Hodiaŭ. Disŝiri. Neulo. Majesta. Packapo. Heroino. Pezo. Internacia. Seshora. Cipreso. Stalo. Feino. Plu. Sukero. Gento. Indigni. Sigelo. Krajono. Ruino. Pesilo. Lipharo. Metio. Ĝardeno. Sono. Laŭdi. Pale. Facila. Insekto. Kiu. Zorgi. Cikano.\footnoteE{La fruaj eldonoj de la \emph{Ekzercaro} havis ``Ĉikano'', kiu vorto egalus tiun vorton en Ekzerco 2.} Traetera. Sofismo. Domoj. Spino. Majo. Signi. Ec. Bonaĵo. Legi. Iel. Juristo. Ĉielo. Ĥemio.

\ekzsec{§5.}

Patro kaj frato. ― Leono estas besto. ― Rozo estas floro kaj kolombo estas birdo. ― La rozo apartenas al Teodoro. ― La suno brilas. ― La patro estas sana. ― La patro estas tajloro.

\begin{ekzvocab}{1em}
\uventry{patro} père | father | Vater | отецъ | ojciec.

\uventry{o} marque le substantif | ending of nouns (substantive) | bezeichnet das Substantiv | означаетъ существительное | oznacza rzeczownik.

\uventry{kaj} et | and | und | и | i.

\uventry{frato} frère | brother | Bruder | братъ | brat.

\uventry{leono} lion | lion | Löwe | левъ | lew.

\uventry{esti} être | be | sein | быть | być.

\uventry{as} marque le présent d’un verbe | ending of the present tense in verbs | bezeichnet das Präsens | означаетъ настоящее время глагола | oznacza czas teraźniejszy.

\uventry{besto} animal | beast | Thier | животное | zwierzę.

\uventry{rozo} rose | rose | Rose | роза | róża.

\uventry{floro} fleur | flouwer | Blume | цвѣтъ, цвѣтокъ | kwiat.

\uventry{kolombo} pigeon | dove | Taube | голубь | gołąb’.

\uventry{birdo} oiseau | bird | Vogel | птица | ptak.

\uventry{la} article défini (le, la, les) | the | bestimmter Artikel (der, die, das) | членъ опредѣленный (по русски не переводится) | przedimek określny (nie tłómaczy się).

\uventry{aparteni} appartenir | belong | gehören | принадлежать | należeć.

\uventry{al} à | to | zu (ersetzt zugleich den Dativ) | къ (замѣняетъ также дательный падежъ) | do (zastępuje też przypadek trzeci).

\uventry{suno} soleil | sun | Sonne | солнце | słońce.

\uventry{brili} briller | shine | glänzen | блистать | błyszczeć.

\uventry{sana} sain, en santé | well, healthy | gesund | здоровый | zdrowy.

\uventry{a} marque l’adjectif | termination of adjectives | bezeichnet das Adjektiv | означаетъ прилагательное | oznacza przymiotnik.

\uventry{tajloro} tailleur | tailor | Schneider | портной | krawiec.

\end{ekzvocab}


\ekzsec{§6.}

Infano ne estas matura homo. ― La infano jam ne ploras. ― La ĉielo estas blua. ― Kie estas la libro kaj la krajono? ― La libro estas sur la tablo, kaj la krajono kuŝas sur la fenestro. ― Sur la fenestro kuŝas krajono kaj plumo. ― Jen estas pomo. ― Jen estas la pomo, kiun mi trovis. ― Sur la tero kuŝas ŝtono.

\begin{ekzvocab}{1em}
\uventry{infano} enfant | child | Kind | дитя | dziecię.

\uventry{ne} non, ne, ne\ldots{} pas | no, not | nicht, nein | не, нѣтъ | nie.

\uventry{matura} mûr | mature, ripe | reif | зрѣлый | dojrzały.

\uventry{homo} homme | man | Mensch | человѣкъ | człowiek.

\uventry{jam} déjà | already | schon | уже | juź.

\uventry{plori} pleurer | mourn, weep | weinen | плакать | płakać.

\uventry{ĉielo} ciel | heaven | Himmel | небо | niebo.

\uventry{blua} bleu | blue | blau | синій | nibieski\eraro{niebieski}.

\uventry{kie} où | where | wo | гдѣ | gdzie.

\uventry{libro} livre | book | Buch | книга | księga, książka.

\uventry{krajono} crayon | pencil | Bleistift | карандашъ | ołówek.

\uventry{sur} sur | upon, on | auf | на | na.

\uventry{tablo} table | table | Tisch | столъ | stół.

\uventry{kuŝi} être couché | lie (down) | liegen | лежать | leżeć.

\uventry{fenestro} fenêtre | window | Fenster | окно | okno.

\uventry{plumo} plume | pen | Feder | перо | pióro.

\uventry{jen} voici, voilà | behold, lo | da, siehe | вотъ | otóż.

\uventry{pomo} pomme | apple | Apfel | яблоко | jabłko.

\uventry{kiu} qui lequel, laquelle | who, which | wer, welcher | кто, который | kto, który.

\uventry{n} marque l’accusatif ou complément direct | ending of the objective | bezeichnet den Accusativ | означаетъ винительный падежъ | oznacza przypadek czwarty.

\uventry{mi} je, moi | I | ich | я | ja.

\uventry{trovi} trouver | find | finden | находить | znajdować.

\uventry{is} marque le passé | ending of past tense in verbs | bezeichnet die vergangene Zeit | означаетъ прошедшее время | oznaczca czas przeszły.

\uventry{tero} terre | earth | Erde | земля | ziemia.

\uventry{ŝtono} pierre | stone | Stein | камень | kamień.

\end{ekzvocab}


\ekzsec{§7.}

Leono estas forta. ― La dentoj de leono estas akraj. ― Al leono ne donu la manon. ― Mi vidas leonon. ― Resti kun leono estas danĝere. ― Kiu kuraĝas rajdi sur leono? ― Mi parolas pri leono.\enlargethispage{-\baselineskip}

\begin{ekzvocab}{1em}
\uventry{forta} fort | strong | stark, kräftig | сильный | silny, mocny.

\uventry{dento} dent | tooth | Zahn | зубъ | ząb.

\uventry{j} marque le pluriel | sign of the plural | bezeichnet die Mehrzahl | означаетъ множественное число | oznacza liczbę mnogą.

\uventry{de} de | of, from | von; ersetzt auch den Genitiv | отъ; замѣняетъ также родительный падежъ | od; zastępuje też przypadek drugi.

\uventry{akra} aigu | sharp | scharf | острый | ostry.

\uventry{doni} donner | give | geben | давать | dawać.

\uventry{u} marque l’impératif | ending of the imperative in verbs | bezeichnet den Imperativ | означаетъ повелительное наклоненіе | oznacza tryb rozkazujący.

\uventry{mano} main | hand | Hand | рука | ręka.

\uventry{vidi} voir | see | sehen | видѣть | widzieć.

\uventry{resti} rester | remain | bleiben | оставаться | pozostawać.

\uventry{kun} avec | with | mit | съ | z.

\uventry{danĝero} danger | danger | Gefahr | опасность | niebezpieczeństwo.

\uventry{e} marque l’adverbe | ending of adverbs | Endung des Adverbs | окончаніе нарѣчія | zakończenie przysłówka.

\uventry{kuraĝa} courageux | courageous, daring | kühn, dreist | смѣлый | śmiały.

\uventry{rajdi} aller à cheval | ride | reiten | ѣздить верхомъ | jeździć konno.

\uventry{i} marque l’infinitif | termination of the infinitive in verbs | bezeichnet den Infinitiv | означаетъ неопредѣленное наклоненіе | oznacza tryb bezokoliczny słowa.

\uventry{paroli} parler | speak | sprechen | говорить | mówić.

\uventry{pri} sur, touchant, de | concerning, about | von, über | о, объ | o.

\end{ekzvocab}

\begin{samepage}
\ekzsec{§8.}

La patro estas bona. ― Jen kuŝas la ĉapelo de la patro. ― Diru al la patro, ke mi estas diligenta. ― Mi amas la patron. ― Venu kune kun la patro. ― La filo staras apud la patro. ― La mano de Johano estas pura. ― Mi konas Johanon. ― Ludoviko, donu al mi panon. ― Mi manĝas per la buŝo kaj flaras per la nazo. ― Antaŭ la domo staras arbo. ― La patro estas en la ĉambro.
\end{samepage}

\begin{ekzvocab}{1em}
\uventry{bona} bon | good | gut | добрый | dobry.

\uventry{ĉapelo} chapeau | hat | Hut | шляпа | kapelusz.

\uventry{diri} dire | say | sagen | сказать | powiadać.

\uventry{ke} que | that (conj.) | dass | что | że.

\uventry{diligenta} diligent, assidu | diligent | fleissig | прилежный | pilny.

\uventry{ami} aimer | love | lieben | любить | lubić, kochać.

\uventry{veni} venir | come | kommen | приходить | przychodzić.

\uventry{kune} ensemble | together | zusammen | вмѣстѣ | razem, wraz.

\uventry{filo} fils | son | Sohn | сынъ | syn.

\uventry{stari} être debout | stand | stehen | стоять | stać.

\uventry{apud} auprès de | near by | neben, an | при, возлѣ | przy, obok.

\uventry{pura} pur, propre | clean, pure | rein | чистый | czysty.

\uventry{koni} connaître | know, recognise | kennen | знать (быть знакомымъ) | znać.

\uventry{pano} pain | bread | Brot | хлѣбъ | chleb.

\uventry{manĝi} manger | eat | essen | ѣсть | jeść.

\uventry{per} par, au moyen de | through, by means of | mittelst, vermittelst, durch | посредствомъ | przez, za pomocą.

\uventry{buŝo} bouche | mouth | Mund | ротъ | usta.

\uventry{flari} flairer, sentir | smell | riechen, schnupfen | нюхать, обонять | wąchać.

\uventry{nazo} nez | nose | Nase | носъ | nos.

\uventry{antaŭ} devant | before | vor | предъ | przed.

\uventry{domo} maison | house | Haus | домъ | dom.

\uventry{arbo} arbre | tree | Baum | дерево | drzewo.

\uventry{ĉambro} chambre | room | Zimmer | комната | pokój.

\end{ekzvocab}

%\newpage % necesa

\ekzsec{§9.}

La birdoj flugas. ― La kanto de la birdoj estas agrabla. ― Donu al la birdoj akvon, ĉar ili volas trinki. ― La knabo forpelis la birdojn. ― Ni vidas per la okuloj kaj aŭdas per la oreloj. ― Bonaj infanoj lernas diligente. ― Aleksandro ne volas lerni, kaj tial mi batas Aleksandron. ― De la patro mi ricevis libron, kaj de la frato mi ricevis plumon. ― Mi venas de la avo, kaj mi iras nun al la onklo. ― Mi legas libron. ― La patro ne legas libron, sed li skribas leteron.

\begin{ekzvocab}{1em}
\uventry{flugi} voler (avec des ailes) | fly (vb.) | fliegen | летать | latać.

\uventry{kanti} chanter | sing | singen | пѣть | śpiewać.

\uventry{agrabla} agréable | agreeable | angenehm | пріятный | przyjemny.

\uventry{akvo} eau | water | Wasser | вода | woda.

\uventry{ĉar} car, parce que | for | weil, da, denn | ибо, такъ какъ | albowiem, ponieważ.

\uventry{ili} ils, elles | they | sie (Mehrzahl) | они, онѣ | oni, one.

\uventry{voli} vouloir | wish, will | wollen | хотѣть | chcieć.

\uventry{trinki} boire | drink | trinken | пить | pić.

\uventry{knabo} garçon | boy | Knabe | мальчикъ | chłopiec.

\uventry{for} loin, hors | forth, out | fort | прочь | precz.

\uventry{peli} chasser, renvoyer | pursue, chase out | jagen, treiben | гнать | gonić.

\uventry{ni} nous | we | wir | мы | my.

\uventry{okulo} œil | eye | Auge | глазъ | oko.

\uventry{aŭdi} entendre | hear | hören | слышать | słyszeć.

\uventry{orelo} oreille | ear | Ohr | ухо | ucho.

\uventry{lerni} apprendre | learn | lernen | учиться | uczyć się.

\uventry{tial} c’est pourquoi | therefore | darum, deshalb | потому | dla tego.

\uventry{bati} battre | beat | schlagen | бить | bić.

\uventry{ricevi} recevoir, obtenir | obtain, get, receive | bekommen, erhalten | получать | otrzymywać.

\uventry{avo} grand-père | grandfather | Grossvater | дѣдъ, дѣдушка | dziad, dziadek.

\uventry{iri} aller | go | gehen | идти | iść.

\uventry{nun} maintenant | now | jetzt | теперь | teraz.

\uventry{onklo} oncle | uncle | Onkel | дядя | wuj, stryj.

\uventry{legi} lire | read | lesen | читать | czytać.

\uventry{sed} mais | but | aber, sondern | но, а | lecz.

\uventry{li} il, lui | he | er | онъ | on.

\uventry{skribi} écrire | write | schreiben | писать | pisać.

\uventry{letero} lettre, épître | letter | Brief | письмо | list.

\end{ekzvocab}


\ekzsec{§10.}

Papero estas blanka. ― Blanka papero kuŝas sur la tablo. ― La blanka papero jam ne kuŝas sur la tablo. ― Jen estas la kajero de la juna fraŭlino. ― La patro donis al mi dolĉan pomon. ― Rakontu al mia juna amiko belan historion. ― Mi ne amas obstinajn homojn. ― Mi deziras al vi bonan tagon, sinjoro! ― Bonan matenon! ― Ĝojan feston! (mi deziras al vi). ― Kia ĝoja festo! (estas hodiaŭ). ― Sur la ĉielo staras la bela suno. ― En la tago ni vidas la helan sunon, kaj en la nokto ni vidas la palan lunon kaj la belajn stelojn. ― La papero estas tre blanka, sed la neĝo estas pli blanka. ― Lakto estas pli nutra, ol vino. ― Mi havas pli freŝan panon, ol vi. ― Ne, vi eraras, sinjoro: via pano estas malpli freŝa, ol mia. ― El ĉiuj miaj infanoj Ernesto estas la plej juna. ― Mi estas tiel forta, kiel vi. ― El ĉiuj siaj fratoj Antono estas la malplej saĝa.

\begin{ekzvocab}{1em}
\uventry{papero} papier | paper | Papier | бумага | papier.

\uventry{blanka} blanc | white | weiss | бѣлый | biały.

\uventry{kajero} cahier | copy book | Heft | тетрадь | kajet.

\uventry{juna} jeune | young | jung | молодой | młody.

\uventry{fraŭlo} homme non marié | bachelor | unverheiratheter Herr | холостой господинъ | kawaler.

\uventry{in} marque le féminin; ex.: \uventry{patro} père ― \uventry{patrino} mère | ending of feminine words; e.~g. \uventry{patro} father ― \uventry{patrino} mother | bezeichnet das weibliche Geschlecht; z.~B. \uventry{patro} Vater ― \uventry{patrino} Mutter; \uventry{fianĉo} Bräutigam ― \uventry{fianĉino} Braut | означаетъ женскій полъ; напр. \uventry{patro} отецъ ― patrino \uventry{мать}; \uventry{fianĉo} женихъ ― \uventry{fianĉino} невѣста | oznacza płeć żeńską; np. \uventry{patro} ojciec ― \uventry{patrino} matka; \uventry{koko} kogut ― \uventry{kokino} kura.

\uventry{(fraŭlino} demoiselle, mademoiselle | miss | Fräulein | барышня | panna.)

\uventry{dolĉa} doux | sweet | süss | сладкій | słodki.

\uventry{rakonti} raconter | tell, relate | erzählen | разсказывать | opowiadać.

\uventry{mia} mon | my | mein | мой | mój.

\uventry{amiko} ami | friend | Freund | другъ | przyjaciel.

\uventry{bela} beau | beautiful | schön, hübsch | красивый, прекрасный | piękny, ładny.

\uventry{historio} histoire | history, story | Geschichte | исторія | historja.

\uventry{obstina} entêté, obstiné | obstinate | eigensinnig | упрямый | uparty.

\uventry{deziri} désirer | desire | wünschen | желать | życzyć.

\uventry{vi} vous, toi, tu | you | Ihr, du, Sie | вы, ты | wy, ty.

\uventry{tago} jour | day | Tag | день | dzień.

\uventry{sinjoro} monsieur | Sir, Mr | Herr | господинъ | pan.

\uventry{mateno} matin | morning | Morgen | утро | poranek.

\uventry{ĝoji} se réjouir | rejoice | sich freuen | радоваться | cieszyć się.

\uventry{festi} fêter | feast | feiern | праздновать | świętować.

\uventry{kia} quel | of what kind, what a | was für ein, welcher | какой | jaki.

\uventry{hodiaŭ} aujourd’hui | to-day | heute | сегодня | dziś.

\uventry{en} en, dans | in | in, ein- | въ | w.

\uventry{hela} clair (qui n’est pas obscur) | clear, glaring | hell, grell | яркій | jasny, jaskrawy.

\uventry{nokto} nuit | night | Nacht | ночь | noc.

\uventry{pala} pâle | pale | bleich, blass | блѣдный | blady.

\uventry{luno} lune | moon | Mond | луна | księźyć.

\uventry{stelo} étoile | star | Stern | звѣзда | gwiazda.

\uventry{neĝo} neige | snow | Schnee | снѣгъ | śnieg.

\uventry{pli} plus | more | mehr | болѣе, больше | więcej.

\uventry{lakto} lait | milk | Milch | молоко | mleko.

\uventry{nutri} nourrir | nourish | nähren | питать | karmić, pożywiać.

\uventry{ol} que (dans une comparaison) | than | als | чѣмъ | niź.

\uventry{vino} vin | vine | Wein | вино | wino.

\uventry{havi} avoir | have | haben | имѣть | mieć.

\uventry{freŝa} frais, récent | fresh | frisch | свѣжій | świeźy.

\uventry{erari} errer | err, mistake | irren | ошибаться, блуждать | błądzić, mylić się.

\uventry{mal} marque les contraires; ex. \uventry{bona} bon ― \uventry{malbona} mauvais; \uventry{estimi} estimer ― \uventry{malestimi} mépriser | denotes opposites; e.~g. \uventry{bona} good ― \uventry{malbona} evil; \uventry{estimi} esteem ― \uventry{malestimi} despise | bezeichnet einen geraden Gegensatz; z.~B. \uventry{bona} gut ― \uventry{malbona} schlecht; \uventry{estimi} schätzen ― \uventry{malestimi} verachten | прямо противоположно; напр. \uventry{bona} хорошій ― \uventry{malbona} дурной; \uventry{estimi} уважать ― \uventry{malestimi} презирать | oznacza preciwieństwo; np. \uventry{bona} dobry ― \uventry{malbona} zły; \uventry{estimi} poważać ― \uventry{malestimi} gardzić.

\uventry{el} de, d’entre, é-, ex- | from, out from | aus | изъ | z.

\uventry{ĉiu} chacun | each, every one | jedermann | всякій, каждый | wszystek, każdy.

\uventry{(ĉiuj} tous | all | alle | всѣ | wszyscy.)

\uventry{plej} le plus | most | am meisten | наиболѣе | najwięcej.

\uventry{tiel} ainsi, de cette manière | thus, so | so | такъ | tak.

\uventry{kiel} comment | how, as | wie | какъ | jak.

\uventry{si} soi, se | one’s self | sich | себя | siebie.

\uventry{(sia} son, sa | one’s | sein | свой | swój.)

\uventry{saĝa} sage, sensé | wise | klug, vernünftig | умный | mądry.

\end{ekzvocab}

\begin{samepage}
\ekzsec[\vskip 1ex\Large\bookman{La feino.}]{§11.}

Unu vidvino havis du filinojn. La pli maljuna estis tiel simila al la patrino per sia karaktero kaj vizaĝo, ke ĉiu, kiu ŝin vidis, povis pensi, ke li vidas la patrinon; ili ambaŭ estis tiel malagrablaj kaj tiel fieraj, ke oni ne povis vivi kun ili. La pli juna filino, kiu estis la plena portreto de sia patro laŭ sia boneco kaj honesteco, estis krom tio unu el la plej belaj knabinoj, kiujn oni povis trovi.
\end{samepage}

\begin{ekzvocab}{1em}
\uventry{feino} fée | fairy | Fee | фея | wieszczka.

\uventry{unu} un | one | ein, eins | одинъ | jeden.

\uventry{vidvo} veuf | widower | Wittwer | вдовецъ | wdowiec.

\uventry{du} deux | two | zwei | два | dwa.

\uventry{simila} semblable | like, similar | ähnlich | похожій | podobny.

\uventry{karaktero} caractère | character | Charakter | характеръ | charakter.

\uventry{vizaĝo} visage | face | Gesicht | лицо | twarz.

\uventry{povi} pouvoir | be able, can | können | мочь | módz.

\uventry{pensi} penser | think | denken | думать | myśleć.

\uventry{ambaŭ} l’un et l’autre | both | beide | оба | obaj.

\uventry{fiera} fier, orgueilleux | proud | stolz | гордый | dumny.

\uventry{oni} on | one, people, they | man | безличное мѣстоименіе множественнаго числа | zaimek nieosobisty liczby mnogiej.

\uventry{vivi} vivre | live | leben | жить | żyć.

\uventry{plena} plein | full, complete | voll | полный | pełny.

\uventry{portreto} portrait | portrait | Portrait | портретъ | portret.

\uventry{laŭ} selon, d’après | according to | nach, gemäss | по, согласно | według.

\uventry{ec} marque la qualité (abstraitement); ex. \uventry{bona} bon ― \uventry{boneco} bonté; \uventry{viro} homme ― \uventry{vireco} virilité | denotes qualities; e.~g. \uventry{bona} good ― \uventry{boneco} goodness; \uventry{viro} man ― \uventry{vireco} manliness; \uventry{virino} woman ― \uventry{virineco} womanliness | Eigenschaft; z.~B. \uventry{bona} gut ― \uventry{boneco} Güte; \uventry{virino} Weib ― \uventry{virineco} Weiblichkeit | качество или состояніе; напр. \uventry{bona} добрый ― \uventry{boneco} доброта; \uventry{virino} женщина ― \uventry{virineco} женственность | przymiot; np. \uventry{bona} dobry ― \uventry{boneco} dobroć; \uventry{infano} dziecię ― \uventry{infaneco} dziecinśtwo.

\uventry{honesta} honnête | honest | ehrlich | честный | uczciwy.

\uventry{krom} hors, hormis, excepté | besides, without, except | ausser | кромѣ | oprócz.

\uventry{tio} cela | that, that one | jenes, das | то, это | to, tamto.

\end{ekzvocab}


\ekzsec{§12.}

Du homoj povas pli multe fari ol unu. ― Mi havas nur unu buŝon, sed mi havas du orelojn. ― Li promenas kun tri hundoj. ― Li faris ĉion per la dek fingroj de siaj manoj. ― El ŝiaj multaj infanoj unuj estas bonaj kaj aliaj estas malbonaj. ― Kvin kaj sep faras dek du. ― Dek kaj dek faras dudek. ― Kvar kaj dek ok faras dudek du. ― Tridek kaj kvardek kvin faras sepdek kvin. ― Mil okcent naŭdek tri. ― Li havas dek unu infanojn. ― Sesdek minutoj faras unu horon, kaj unu minuto konsistas el sesdek sekundoj. ― Januaro estas la unua monato de la jaro, Aprilo estas la kvara, Novembro estas la dek-unua, Decembro estas la dek-dua. ― La dudeka (tago) de Februaro estas la kvindek-unua tago de la jaro. ― La sepan tagon de la semajno Dio elektis, ke ĝi estu pli sankta, ol la ses unuaj tagoj. ― Kion Dio kreis en la sesa tago? ― Kiun daton ni havas hodiaŭ? ― Hodiaŭ estas la dudek sepa\footnoteE{Ĉi tie ne estas streketo (ne estas \emph{dudek-sepa}), sed aliaj antaŭaj nombroj havas ĝin.} (tago) de Marto. ― Georgo Vaŝington estis naskita la dudek duan\footnoteE{Sama rimarko kiel la antaŭa; ĝi povus anstataŭiĝi \emph{dudek-duan}.} de Februaro de la jaro mil sepcent tridek dua.

\begin{ekzvocab}{1em}
\uventry{multe} beaucoup, nombreux | much, many | viel | много | wiele.

\uventry{fari} faire | do | thun, machen | дѣлать | robić.

\uventry{nur} seulement, ne\ldots{} que | only (adv.) | nur | только | tylko.

\uventry{promeni} se promener | walk, promenade | spazieren | прогуливаться | spacerować.

\uventry{tri} trois | three | drei | три | trzy.

\uventry{hundo} chien | dog | Hund | песъ, собака | pies.

\uventry{ĉio} tout | everything | alles | все | wszystko.

\uventry{dek} dix | ten | zehn | десять | dziesięć.

\uventry{fingro} doigt | finger | Finger | палецъ | palec.

\uventry{alia} autre | other | ander | иной | inny.

\uventry{kvin} cinq | five | fünf | пять | pięć.

\uventry{sep} sept | seven | sieben | семь | siedm.

\uventry{kvar} quatre | four | vier | четыре | cztery.

\uventry{ok} huit | eight | acht | восемь | ośm.

\uventry{mil} mille (nombre) | thousand | tausend | тысяча | tysiąc.

\uventry{cent} cent | hundred | hundert | сто | sto.

\uventry{naŭ} neuf (9) | nine | neun | девять | dziewięć.

\uventry{ses} six | six | sechs | шесть | sześć.

\uventry{minuto} minute | minute | Minute | минута | minuta.

\uventry{horo} heure | hour | Stunde | часъ | godzina.

\uventry{konsisti} consister | consist | bestehen | состоять | składać się.

\uventry{sekundo} seconde | second | Sekunde | секунда | sekunda.

\uventry{Januaro} Janvier | January | Januar | Январь | Styczeń.

\uventry{monato} mois | month | Monat | мѣсяцъ | miesiąc.

\uventry{jaro} année | year | Jahr | годъ | rok.

\uventry{Aprilo} Avril | April | April | Апрѣль | Kwiecień.

\uventry{Novembro} Novembre | November | November | Ноябрь | Listopad.

\uventry{Decembro} Décembre | December | December | Декабрь | Grudzień.

\uventry{Februaro} Février | February | Februar | Февраль | Luty.

\uventry{semajno} semaine | week | Woche | недѣля | tydzień.

\uventry{Dio} Dieu | God | Gott | Богъ | Bóg.

\uventry{elekti} choisir | choose | wählen | выбирать | wybierać.

\uventry{ĝi} cela, il, elle | it | es, dieses | оно, это | ono, to.

\uventry{sankta} saint | holy | heilig | святой, священный | święty.

\uventry{krei} créer | create | schaffen, erschaffen | создавать | stwarzać.

\uventry{dato} date | date | Datum | число (мѣсяца) | data.

\uventry{Marto} Mars | March | März | Мартъ | Marzec.

\uventry{naski} enfanter, faire naître | bear, produce | gebären | рождать | rodzić.

\uventry{it} marque le participe passé passif | ending of past part. pass. in verbs | bezeichnet das Participium perfecti passivi | означаетъ причастіе прошедшаго времени страдат. залога | oznacza imiesłów bierny czasu przeszłego.

\end{ekzvocab}

\ekzsec[\vskip 1ex\Large{\bookman La feino} \large(Daŭrigo).]{§13.}


Ĉar ĉiu amas ordinare personon, kiu estas simila al li, tial tiu ĉi patrino varmege amis sian pli maljunan filinon, kaj en tiu sama tempo ŝi havis teruran malamon kontraŭ la pli juna. Ŝi devigis ŝin manĝi en la kuirejo kaj laboradi senĉese. Inter aliaj aferoj tiu ĉi malfeliĉa infano devis du fojojn en ĉiu tago iri ĉerpi akvon en tre malproksima loko kaj alporti domen plenan grandan kruĉon.

\begin{ekzvocab}{1em}
\uventry{daŭri} durer | endure, last | dauern | продолжаться | trwać.

\uventry{ig} faire\ldots{}; ex. \uventry{pura} pur, propre ― \uventry{purigi} nettoyer; \uventry{morti} mourir ― \uventry{mortigi} tuer (faire mourir) | cause to be; e.~g. \uventry{pura} pure ― \uventry{purigi} purify; \uventry{sidi} sit ― \uventry{sidigi} seat | zu etwas machen, lassen; z.~B. \uventry{pura} rein ― \uventry{purigi} reinigen; \uventry{bruli} brennen (selbst) ― \uventry{bruligi} brennen (etwas) | дѣлать чѣмъ нибудь, заставить дѣлать; напр. \uventry{pura} чистый ― \uventry{purigi} чистить; \uventry{bruli} горѣть ― \uventry{bruligi} жечь | robić czemś; np. \uventry{pura} czysty ― \uventry{purigi} czyścić; \uventry{bruli} palić się ― \uventry{bruligi} palić.

\uventry{ordinara} ordinaire | ordinary | gewöhnlich | обыкновенный | zwyczajny.

\uventry{persono} personne | person | Person | особа, лицо | osoba.

\uventry{tiu} celui-là | that | jener | тотъ | tamten.

\uventry{ĉi} ce qui est le plus près; ex. \uventry{tiu} celui-là ― \uventry{tiu ĉi} celui-ci | denotes proximity; e.~g. \uventry{tiu} that ― \uventry{tiu ĉi} this; \uventry{tie} there ― \uventry{tie ĉi} here | die nächste Hinweisung; z.~B. \uventry{tiu} jener ― \uventry{tiu ĉi} dieser; \uventry{tie} dort ― \uventry{tie ĉi} hier | ближайшее указаніе; напр. \uventry{tiu} тотъ ― \uventry{tiu ĉi} этотъ; \uventry{tie} тамъ ― \uventry{tie ĉi} здѣсь | wskazanie najbliższe; np. \uventry{tiu} tamten ― \uventry{tiu ĉi} ten; \uventry{tie} tam ― \uventry{tie ĉi} tu.

\uventry{varma} chaud | warm | warm | теплый | ciepły.

\uventry{eg} marque augmentation, plus haut degré; ex. \uventry{pordo} porte ― \uventry{pordego} grande porte; \uventry{peti} prier ― \uventry{petegi} supplier | denotes increase of degree; e.~g. \uventry{varma} warm ― \uventry{varmega} hot | bezeichnet eine Vergösserung oder Steigerung; z.~B. \uventry{pordo} Thür ― \uventry{pordego} Thor; \uventry{varma} warm ― \uventry{varmega} heiss | означаетъ увеличеніе или усиленіе степени; напр. \uventry{mano} рука ― \uventry{manego} ручище; \uventry{varma} теплый ― \uventry{varmega} горячій | oznacza zwiększenie lub wzmocnienie stopnia; np. \uventry{mano} ręka ― \uventry{manego} łapa; \uventry{varma} ciepły ― \uventry{varmega} gorący.

\uventry{sama} même (qui n’est pas autre) | same | selb, selbst (z.~B. derselbe, daselbst) | же, самый (напр. тамъ же, тотъ самый) | źe, sam (np. tam że, ten sam).

\uventry{tempo} temps (durée) | time | Zeit | время | czas.

\uventry{teruro} terreur, effroi | terror | Schrecken | ужасъ | przerażenie.

\uventry{kontraŭ} contre | against | gegen | противъ | przeciw.

\uventry{devi} devoir | ought, must | müssen | долженствовать | musieć.

\uventry{kuiri} faire cuire | cook | kochen | варить | gotować.

\uventry{ej} marque le lieu spécialement affecté à\ldots{} ex. \uventry{preĝi} prier ― \uventry{preĝejo} église; \uventry{kuiri} faire cuire ― \uventry{kuirejo} cuisine | place of an action; e.~g. \uventry{kuiri} cook ― \uventry{kuirejo} kitchen | Ort für\ldots{}; z.~B. \uventry{kuiri} kochen ― \uventry{kuirejo} Küche; \uventry{preĝi} beten ― \uventry{preĝejo} Kirche | мѣсто для\ldots{}; напр. \uventry{kuiri} варить ― \uventry{kuirejo} кухня; \uventry{preĝi} молиться ― \uventry{preĝejo} церковь | miejsce dla\ldots{}; np. \uventry{kuiri} gotować ― \uventry{kuirejo} kuchnia; \uventry{preĝi} modlić się ― \uventry{preĝejo} kościół.

\uventry{labori} travailler | labor, work | arbeiten | работать | pracować.

\uventry{ad} marque durée dans l’action; ex. \uventry{pafo} coup de fusil ― \uventry{pafado} fusillade | denotes duration of action; e.~g. \uventry{danco} dance ― \uventry{dancado} dancing | bezeichnet die Dauer der Thätigkeit; z.~B. \uventry{danco} der Tanz ― \uventry{dancado} das Tanzen | означаетъ продолжительность дѣйствія: напр. \uventry{iri} идти ― \uventry{iradi} ходить, хаживать | oznacza trwanie czynności; np. \uventry{iri} iść -- \uventry{iradi} chodzić.

\uventry{sen} sans | without | ohne | безъ | bez.

\uventry{ĉesi} cesser | cease, desist | aufhören | переставать | przestawać.

\uventry{inter} entre, parmi | between, among | zwischen | между | miedzy\eraro{między}.

\uventry{afero} affaire | affair | Sache, Angelegenheit | дѣло | sprawa.

\uventry{feliĉa} heureux | happy | glücklich | счастливый | szczęśliwy.

\uventry{fojo} fois | time (e.~g. three times etc.) | Mal | разъ | raz.

\uventry{ĉerpi} puiser | draw | schöpfen (z.~B. Wasser) | черпать | czerpać.

\uventry{tre} très | very | sehr | очень | bardzo.

\uventry{proksima} proche, près de | near | nahe | близкій | blizki.

\uventry{loko} place, lieu | place | Ort | мѣсто | miejsce.

\uventry{porti} porter | pack, carry | tragen | носить | nosić.

\uventry{n} marque l’accusatif et le lieu ou où l’on va | ending of the objective, also marks direction towards | bezeichnet den Accusativ, auch die Richtung | означаетъ винит. падежъ, а также направленіе | oznacza przypadek czwarty, również kierunek.

\uventry{kruĉo} cruche | pitcher | Krug | кувшинъ | dzban.

\end{ekzvocab}


\ekzsec{§14.}

Mi havas cent pomojn. ― Mi havas centon da pomoj. ― Tiu ĉi urbo havas milionon da loĝantoj. ― Mi aĉetis dekduon (aŭ dek-duon) da kuleroj kaj du dekduojn da forkoj. ― Mil jaroj (aŭ milo da jaroj) faras miljaron. ― Unue mi redonas al vi la monon, kiun vi pruntis al mi; due mi dankas vin por la prunto; trie mi petas vin ankaŭ poste prunti al mi, kiam mi bezonos monon. ― Por ĉiu tago mi ricevas kvin frankojn, sed por la hodiaŭa tago mi ricevis duoblan pagon, t. e. (= tio estas) dek frankojn. ― Kvinoble sep estas tridek kvin. ― Tri estas duono de ses. ― Ok estas kvar kvinonoj de dek. ― Kvar metroj da tiu ĉi ŝtofo kostas naŭ frankojn; tial du metroj kostas kvar kaj duonon frankojn (aŭ da frankoj). ― Unu tago estas tricent-sesdek-kvinono aŭ tricent-sesdek-sesono de jaro. ― Tiuj ĉi du amikoj promenas ĉiam duope. ― Kvinope ili sin ĵetis sur min, sed mi venkis ĉiujn kvin atakantojn. ― Por miaj kvar infanoj mi aĉetis dek du pomojn, kaj al ĉiu el la infanoj mi donis po tri pomoj. ― Tiu ĉi libro havas sesdek paĝojn; tial, se mi legos en ĉiu tago po dek kvin paĝoj, mi finos la tutan libron en kvar tagoj.

\begin{ekzvocab}{1em}
\uventry{on} marque les nombres fractionnaires; ex. \uventry{kvar} quatre ― \uventry{kvarono} le quart | marks fractions; e.~g. \uventry{kvar} four ― \uventry{kvarono} a fourth, quarter | Bruchzahlwort; z.~B. \uventry{kvar} vier ― \uventry{kvarono} Viertel | означаетъ числительное дробное; напр. \uventry{kvar} четыре ― \uventry{kvarono} четверть | liczebnik ułamkowy; np. \uventry{kvar} cztery ― \uventry{kvarono} ćwierć.

\uventry{da} de (après les mots marquant mesure, poids, nombre) | is used instead of de after words expressing weight or measure | ersetzt den Genitiv nach Mass, Gewicht u. drgl bezeichnenden Wörtern | замѣняетъ родительный падежъ послѣ словъ, означающихъ мѣру, вѣсъ и т. п. | zastępuje przypadek drugi po słowach oznaczających miarę, wagę i. t. p.

\uventry{urbo} ville | town | Stadt | городъ | miasto.

\uventry{loĝi} habiter, loger | lodge | wohnen | жить, квартировать | mieszkać.

\uventry{ant} marque le participe actif | ending of pres. part. act. in verbs | bezeichnet das Participium praes. act. | означаетъ причасіе настоящаго времени дѣйств. залога | oznacza imieslów czynny czasu teraźniejsz.

\uventry{aĉeti} acheter | buy | kaufen | покупать | kupować.

\uventry{aŭ} ou | or | oder | или | albo, lub.

\uventry{kulero} cuillère | spoon | Löffel | ложка | łyżka.

\uventry{forko} fourchette | fork | Gabel | вилы, вилка | widły, widelec.

\uventry{re} de nouveau, de retour | again, back | wieder, zurück | снова, назадъ | znowu, napowrót.

\uventry{mono} argent (monnaie) | money | Geld | деньги | pieniądze.

\uventry{prunti} prêter | lend, borrow | leihen, borgen | взаймы давать или брать | pożyczać.

\uventry{danki} remercier | thank | danken | благодарить | dziękować.

\uventry{por} pour | for | für | для, за | dla, za.

\uventry{peti} prier | request, beg | bitten | просить | prosić.

\uventry{ankaŭ} aussi | also | auch | также | także.

\uventry{post} après | after, behind | nach, hinter | послѣ, за | po, za, potem.

\uventry{kiam} quand, lorsque | when | wann | когда | kiedy.

\uventry{bezoni} avoir besoin de | need, want | brauchen | нуждаться | potrzebować.

\uventry{obl} marque l’adjectif numéral multiplicatif; ex. \uventry{du} deux ― \uventry{duobla} double | \ldots{}fold; e.~g. \uventry{du} two ― \uventry{duobla} twofold, duplex | bezeichnet das Vervielfachungszahlwort; z.~B. \uventry{du} zwei ― \uventry{duobla} zweifach | означаетъ числительное множительное; напр. \uventry{du} два ― \uventry{duobla} двойной | oznacza liczebnik wieloraki; np. \uventry{du} dwa ― \uventry{duobla} podwójny.

\uventry{pagi} payer | pay | zahlen | платить | płacić.

\uventry{ŝtofo} étoffe | stuff, matter, goods | Stoff | вещество, матерія | materja, materjał.

\uventry{kosti} coûter | cost | kosten | стоить | kosztować.

\uventry{ĉiam} toujours | always | immer | всегда | zawsze.

\uventry{op} marque ládjectif numéral collectif; ex. \uventry{du} deux ― \uventry{duope} à deux | marks collective numerals; e.~g. \uventry{tri} three ― \uventry{triope} three together | Sammelzahlwort; z.~B. \uventry{du} zwei ― \uventry{duope} selbander, zwei zusammen | означаетъ числительное собирательное; напр. \uventry{du} два ― \uventry{duope} вдвоемъ | oznacza liczebnik zbiorowy; np. \uventry{du} dwa ― \uventry{duope} we dwoje.

\uventry{ĵeti} jeter | throw | werfen | бросать | rzucać.

\uventry{venki} vaincre | conquer | siegen | побѣждать | zwyciężać.

\uventry{ataki} attaquer | attack | angreifen | нападать | atakować.

\uventry{paĝo} page (d’un livre) | page | Seite (Buch-) | страница | stronica.

\uventry{se} si | if | wenn | если | jeżeli.

\uventry{fini} finir | end, finish | enden, beendigen | кончать | kończyć.

\uventry{tuta} entier, total | whole | ganz | цѣлый, весь | cały.

\end{ekzvocab}


\ekzsec[\vskip 1ex\Large{\bookman La feino} \large(Daŭrigo).]{§15.}

En unu tago, kiam ŝi estis apud tiu fonto, venis al ŝi malriĉa virino, kiu petis ŝin, ke ŝi donu al ŝi trinki. “Tre volonte, mia bona,”; diris la bela knabino. Kaj ŝi tuj lavis sian kruĉon kaj ĉerpis akvon en la plej pura loko de la fonto kaj alportis al la virino, ĉiam subtenante la kruĉon, por ke la virino povu trinki pli oportune. Kiam la bona virino trankviligis sian soifon, ŝi diris al la knabino: “Vi estas tiel bela, tiel bona kaj tiel honesta, ke mi devas fari al vi donacon” (ĉar tio ĉi estis feino, kiu prenis sur sin la formon de malriĉa vilaĝa virino, por vidi, kiel granda estos la ĝentileco de tiu ĉi juna knabino). “Mi faras al vi donacon,” daŭrigis la feino, “ke ĉe ĉiu vorto, kiun vi diros, el via buŝo eliros aŭ floro aŭ multekosta ŝtono.”

\begin{ekzvocab}{1em}
\uventry{fonto} source | fountain | Quelle | источникъ | żródło.

\uventry{riĉa} riche | rich | reich | богатый | bogaty.

\uventry{viro} homme (sexe) | man | Mann | мужчина, мужъ | mężczyzna, mąż.

\uventry{volonte} volontiers | willingly | gern | охотно | chętnie.

\uventry{tuj} tout de suite, aussitôt | immediate | bald, sogleich | сейчасъ | natychmiast.

\uventry{lavi} laver | wasch\eraro{wash} | waschen | мыть | myć.

\uventry{sub} sous | under, beneath, below | unter | подъ | pod.

\uventry{teni} tenir | hold, grasp | halten | держать | trzymać.

\uventry{oportuna} commode, qui est à propos | opportune, suitable | bequem | удобный | wygodny.

\uventry{trankvila} tranquille | quiet | ruhig | спокойный | spokojny.

\uventry{soifi} avoir soif | thirst | dursten | жаждать | pragnąć.

\uventry{donaci} faire cadeau | make a present | schenken | дарить | darować.

\uventry{preni} prendre | take | nehmen | брать | brać.

\uventry{formo} forme | form | Form | форма | forma, kształt.

\uventry{vilaĝo} village | village | Dorf | деревня | wieś.

\uventry{ĝentila} gentil, poli | polite, gentle | höflich | вѣжливый | grzeczny.

\uventry{ĉe} chez | at | bei | у, при | u, przy.

\end{ekzvocab}


\ekzsec{§16.}

Mi legas. ― Ci skribas (anstataŭ ``ci" oni uzas ordinare ``vi"). ― Li estas knabo, kaj ŝi estas knabino. ― La tranĉilo tranĉas bone, ĉar ĝi estas akra. ― Ni estas homoj. ― Vi estas infanoj. ― Ili estas rusoj. ― Kie estas la knaboj? ― Ili estas en la ĝardeno. ― Kie estas la knabinoj? ― Ili ankaŭ estas en la ĝardeno. ― Kie estas la tranĉiloj? ― Ili kuŝas sur la tablo. ― Mi vokas la knabon, kaj li venas. ― Mi vokas la knabinon, kaj ŝi venas. ― La infano ploras, ĉar ĝi volas manĝi. ― La infanoj ploras, ĉar ili volas manĝi. ― Knabo, vi estas neĝentila. ― Sinjoro, vi estas neĝentila. ― Sinjoroj, vi estas neĝentilaj. ― Mia hundo, vi estas tre fidela. ― Oni diras, ke la vero ĉiam venkas. ― En la vintro oni hejtas la fornojn. ― Kiam oni estas riĉa (aŭ riĉaj), oni havas multajn amikojn.

\begin{ekzvocab}{1em}
\uventry{ci} tu, toi, | thou | du | ты | ty.

\uventry{anstataŭ} au lieu de | instead | anstatt, statt | вмѣсто | zamiast.

\uventry{uzi} employer | use | gebrauchen | употреблять | używać.

\uventry{tranĉi} trancher, couper | cut | schneiden | рѣзать | rżnąć.

\uventry{il} instrument; ex. \uventry{tondi} tondre ― \uventry{tondilo} ciseaux; \uventry{pafi} tirer (coup de feu) ― \uventry{pafilo} fusil | instrument; e.~g. \uventry{tondi} shear ― \uventry{tondilo} scissors | Werkzeug; z.~B. \uventry{tondi} scheeren ― \uventry{tondilo} Scheere; \uventry{pafi} schiessen ― \uventry{pafilo} Flinte | орудіе; напр. \uventry{tondi} стричь ― \uventry{tondilo} ножницы; \uventry{pafi} стрѣлять ― \uventry{pafilo} ружье | narzędzie; np. \uventry{tondi} strzydz ― \uventry{tondilo} nożyce; \uventry{pafi} strzelać ― \uventry{pafilo} fuzya.

\uventry{ruso} russe | Russian | Russe | русскій | rossjanin.

\uventry{ĝardeno} jardin | garden | Garten | садъ | ogród.

\uventry{voki} appeler | call | rufen | звать | wołać.

\uventry{voli} vouloir | wish, will | wollen | хотѣть | chcieć.

\uventry{fidela} fidèle | faithful | treu | вѣрный | wierny.

\uventry{vero} vérité | true | Wahrheit | истина | prawda.

\uventry{vintro} hiver | winter | Winter | зима | zima.

\uventry{hejti} chauffer, faire du feu | heat (vb.) | heizen | топить (печку) | palić (w piecu).

\uventry{forno} fourneau, poële, four | stove | Ofen | печь, печка | piec.

\end{ekzvocab}


\ekzsec[\vskip 1ex\Large{\bookman La feino} \large(Daŭrigo).]{§17.}

Kiam tiu ĉi bela knabino venis domen, ŝia patrino insultis ŝin, kial ŝi revenis tiel malfrue de la fonto. “Pardonu al mi, patrino,” diris la malfeliĉa knabino, “ke mi restis tiel longe”. Kaj kiam ŝi parolis tiujn ĉi vortojn, elsaltis el ŝia buŝo tri rozoj, tri perloj kaj tri grandaj diamantoj. “Kion mi vidas!” diris ŝia patrino kun grandega miro. “Ŝajnas al mi, ke el ŝia buŝo elsaltas perloj kaj diamantoj! De kio tio ĉi venas, mia filino?” (Tio ĉi estis la unua fojo, ke ŝi nomis ŝin sia filino). La malfeliĉa infano rakontis al ŝi naive ĉion, kio okazis al ŝi, kaj, dum ŝi parolis, elfalis el ŝia buŝo multego da diamantoj. “Se estas tiel,” diris la patrino, “mi devas tien sendi mian filinon. Marinjo, rigardu, kio eliras el la buŝo de via fratino, kiam ŝi parolas; ĉu ne estus al vi agrable havi tian saman kapablon? Vi devas nur iri al la fonto ĉerpi akvon; kaj kiam malriĉa virino petos de vi trinki, vi donos ĝin al ŝi ĝentile.”

\begin{ekzvocab}{1em}
\uventry{insulti} injurier | insult | schelten, schimpfen | ругать | besztać, łajać.

\uventry{kial} pourquoi | because, wherefore | warum | почему | dlaczego.

\uventry{frue} de bonne heure | early | früh | рано | rano, wcześnie.

\uventry{pardoni} pardonner | forgive | verzeihen | прощать | przebaczać.

\uventry{longa} long | long | lang | долгій, длинный | długi.

\uventry{salti} sauter, bondir | leap, jump | springen | прыгать | skakać.

\uventry{perlo} perle | pearl | Perle | жемчугъ | perła.

\uventry{granda} grand | great, tall | gross | большой, великій | wielki, duźy.

\uventry{diamanto} diamant | diamond | Diamant | алмазъ | djament.

\uventry{miri} s’étonner, admirer | wonder | sich wundern | удивляться | dziwić się.

\uventry{ŝajni} sembler | seem | scheinen | казаться | wydawać się.

\uventry{nomi} nommer, appeler | name, nominate | nennen | называть | naźywać.

\uventry{naiva} naïf | naïve | naiv | наивный | naiwny.

\uventry{okazi} avoir lieu, arriver | happen | vorfallen | случаться | zdarzać się.

\uventry{dum} pendant, tandis que | while | während | пока, между тѣмъ какъ | póki.

\uventry{sendi} envoyer | send | senden, schicken | посылать | posyłać.

\uventry{kapabla} capable, apte | capable | fähig | способный | zdolny.

\end{ekzvocab}


\ekzsec{§18.}

Li amas min, sed mi lin ne amas. ― Mi volis lin bati, sed li forkuris de mi. ― Diru al mi vian nomon. ― Ne skribu al mi tiajn longajn leterojn. ― Venu al mi hodiaŭ vespere. ― Mi rakontos al vi historion. ― Ĉu vi diros al mi la veron? ― La domo apartenas al li. ― Li estas mia onklo, ĉar mia patro estas lia frato. ― Sinjoro Petro kaj lia edzino tre amas miajn infanojn; mi ankaŭ tre amas iliajn (infanojn). ― Montru al ili vian novan veston. ― Mi amas min mem, vi amas vin mem, li amas sin mem, kaj ĉiu homo amas sin mem. ― Mia frato diris al Stefano, ke li amas lin pli, ol sin mem. ― Mi zorgas pri ŝi tiel, kiel mi zorgas pri mi mem; sed ŝi mem tute ne zorgas pri si kaj tute sin ne gardas. ― Miaj fratoj havis hodiaŭ gastojn; post la vespermanĝo niaj fratoj eliris kun la gastoj el sia domo kaj akompanis ilin ĝis ilia domo. ― Mi jam havas mian ĉapelon; nun serĉu vi vian. ― Mi lavis min en mia ĉambro, kaj ŝi lavis sin en sia ĉambro. ― La infano serĉis sian pupon; mi montris al la infano, kie kuŝas ĝia pupo. ― Oni ne forgesas facile sian unuan amon.

\begin{ekzvocab}{1em}
\uventry{kuri} courir | run | laufen | бѣгать | biegać, lecieć.

\uventry{vespero} soir | evening | Abend | вечеръ | wieczór.

\uventry{ĉu} est-ce que | whether | ob | ли, развѣ | czy.

\uventry{edzo} mari, époux | married person, husband | Gemahl | супругъ | małżonek.

\uventry{montri} montrer | show | zeigen | показывать | pokazywać.

\uventry{nova} nouveau | new | neu | новый | nowy.

\uventry{vesti} vêtir, habiller | clothe | ankleiden | одѣвать | odziewać, ubierać.

\uventry{mem} même (moi-, toi-, etc.) | self | selbst | самъ | sam.

\uventry{zorgi} avoir soin | care, be anxious | sorgen | заботиться | troszczyć się.

\uventry{gardi} garder | guard | hüten | стеречь, беречь | strzedz.

\uventry{gasto} hôte | guest | Gast | гость | gość.

\uventry{akompani} accompagner | accompany | begleiten | сопровождать | towarzyszyć.

\uventry{ĝis} jusqu’à | up to, until | bis | до | do, aż.

\uventry{serĉi} chercher | search | suchen | искать | szukać.

\uventry{pupo} poupée | doll | Puppe | кукла | lalka.

\uventry{forgesi} oublier | forget | vergessen | забывать | zapominać.

\uventry{facila} facile | easy | leicht | легкій | łatwy, lekki.

\end{ekzvocab}

\ekzsec[\vskip 1ex\Large{\bookman La feino} \large(Daŭrigo).]{§19.}

“Estus tre bele,” respondis la filino malĝentile, “ke mi iru al la fonto!” ― “Mi volas\unuakomon{} ke vi tien iru,” diris la patrino, “kaj iru tuj!” La filino iris, sed ĉiam murmurante. Ŝi prenis la plej belan arĝentan vazon, kiu estis en la loĝejo. Apenaŭ ŝi venis al la fonto, ŝi vidis unu sinjorinon, tre riĉe vestitan, kiu eliris el la arbaro kaj petis de ŝi trinki (tio ĉi estis tiu sama feino, kiu prenis sur sin la formon kaj la vestojn de princino, por vidi, kiel granda estos la malboneco de tiu ĉi knabino). “Ĉu mi venis tien ĉi,” diris al ŝi la malĝentila kaj fiera knabino, “por doni al vi trinki? Certe, mi alportis arĝentan vazon speciale por tio, por doni trinki al tiu ĉi sinjorino! Mia opinio estas: prenu mem akvon, se vi volas trinki.” ― “Vi tute ne estas ĝentila,” diris la feino sen kolero. “Bone, ĉar vi estas tiel servema, mi faras al vi donacon, ke ĉe ĉiu vorto, kiun vi parolos, eliros el via buŝo aŭ serpento aŭ rano.”

\begin{ekzvocab}{1em}
\uventry{us} marque le conditionnel (ou le subjonctif) | ending of conditional in verbs | bezeichnet den Konditionalis (oder Konjunktiv) | означаетъ условное наклоненіе (или сослагательное) | oznacza tryb warunkowy.

\uventry{murmuri} murmurer, grommeler | murmur | murren, brummen | ворчать | mruczеć.

\uventry{vazo} vase | vase | Gefäss | сосудъ | naczynie.

\uventry{arĝento} argent (métal) | silver | Silber | серебро | srebro.

\uventry{apenaŭ} à peine | scarcely | kaum | едва | ledwie.

\uventry{ar} une réunion de certains objets; ex. \uventry{arbo} arbre ― \uventry{arbaro} forêt | a collection of objects; e.~g. \uventry{arbo} tree ― \uventry{arbaro} forest; \uventry{ŝtupo} step ― \uventry{ŝtuparo} stairs | Sammlung gewisser Gegenstände; z.~B. \uventry{arbo} Baum ― \uventry{arbaro} Wald; \uventry{ŝtupo} Stufe ― \uventry{ŝtuparo} Treppe, Leiter | собраніе данныхъ предметовъ; напр. \uventry{arbo} дерево ― \uventry{arbaro} лѣсъ; \uventry{ŝtupo} ступень ― \uventry{ŝtuparo} лѣстница | zbiór danych przedmiotów; np. \uventry{arbo} drzewo ― \uventry{arbaro} las; \uventry{ŝtupo} szczebel ― \uventry{ŝtuparo} drabina.

\uventry{princo} prince | prince | Fürst, Prinz | принцъ, князь | książe.

\uventry{certa} certain | certain, sure | sicher, gewiss | вѣрный, извѣстный | pewny.

\uventry{speciala} spécial | special | speciell | спеціальный | specjalny.

\uventry{opinio} opinion | opinion | Meinung | мнѣніе | opinja.

\uventry{koleri} se fâcher | be angry | zürnen | сердиться | gniewać się.

\uventry{servi} servir | serve | dienen | служить | słuźyć.

\uventry{em} qui a le penchant, l’habitude; ex. \uventry{babili} babiller ― \uventry{babilema} babillard | inclined to; e.~g. \uventry{babili} chatter ― \uventry{babilema} talkative | geneigt, gewohnt; z.~B. \uventry{babili} plaudern ― \uventry{babilema} geschwätzig | склонный, имѣющій привычку; напр. \uventry{babili} болтать ― \uventry{babilema} болтливый | skłonny, przyzwyczajony; np. \uventry{babili} paplać ― \uventry{babilema} gadula.

\uventry{serpento} serpent | serpent | Schlange | змѣя | wąź.

\uventry{rano} grenouille | frog | Frosch | лягушка | żaba.

\end{ekzvocab}

\newpage % necesa

\ekzsec{§20.}

Nun mi legas, vi legas kaj li legas; ni ĉiuj legas. ― Vi skribas, kaj la infanoj skribas; ili ĉiuj sidas silente kaj skribas. ― Hieraŭ mi renkontis vian filon, kaj li ĝentile salutis min. ― Hodiaŭ estas sabato, kaj morgaŭ estos dimanĉo. ― Hieraŭ estis vendredo, kaj postmorgaŭ estos lundo. ― Antaŭ tri tagoj mi vizitis vian kuzon kaj mia vizito faris al li plezuron. ― Ĉu vi jam trovis vian horloĝon? ― Mi ĝin ankoraŭ ne serĉis; kiam mi finos mian laboron, mi serĉos mian horloĝon, sed mi timas, ke mi ĝin jam ne trovos. ― Kiam mi venis al li, li dormis; sed mi lin vekis. ― Se mi estus sana, mi estus feliĉa. ― Se li scius, ke mi estas tie ĉi, li tuj venus al mi. ― Se la lernanto scius bone sian lecionon, la instruanto lin ne punus. ― Kial vi ne respondas al mi? ― Ĉu vi estas surda aŭ muta? ― Iru for! ― Infano, ne tuŝu la spegulon! ― Karaj infanoj, estu ĉiam honestaj! ― Li venu, kaj mi pardonos al li. Ordonu al li, ke li ne babilu. ― Petu ŝin, ke ŝi sendu al mi kandelon. ― Ni estu gajaj, ni uzu bone la vivon, ĉar la vivo ne estas longa. ― Ŝi volas danci. ― Morti pro la patrujo estas agrable. ― La infano ne ĉesas petoli.

\begin{ekzvocab}{1em}
\uventry{sidi} être assis | sit | sitzen | сидѣть | siedzieć.

\uventry{silenti} se taire | be silent | schweigen | молчать | milczeć.

\uventry{hieraŭ} hier | yesterday | gestern | вчера | wczoraj.

\uventry{renkonti} rencontrer | meet | begegnen | встрѣчать | spotykać.

\uventry{saluti} saluer | salute, greet | grüssen | кланяться | kłaniać się.

\uventry{sabato} samedi | Saturday | Sonnabend | суббота | sobota.

\uventry{morgaŭ} demain | to-morrow | morgen | завтра | jutro.

\uventry{dimanĉo} dimanche | Sunday | Sonntag | воскресенье | niedziela.

\uventry{vendredo} vendredi | Friday | Freitag | пятница | piątek.

\uventry{lundo} lundi | Monday | Montag | понедѣльникъ | poniedziałek.

\uventry{viziti} visiter | visit | besuchen | посѣщать | odwiedzać.

\uventry{kuzo} cousin | cousin | Vetter, Cousin | двоюродный братъ | kuzyn.

\uventry{plezuro} plaisir | pleasure | Vergnügen | удовольствіе | przyjemność.

\uventry{horloĝo} horloge, montre | clock | Uhr | часы | zegar.

\uventry{timi} craindre | fear | fürchten | бояться | obawiać się.

\uventry{dormi} dormir | sleep | schlafen | спать | spać.

\uventry{veki} réveiller, éveiller | wake, arouse | wecken | будить | budzić.

\uventry{scii} savoir | know | wissen | знать | wiedzieć.

\uventry{leciono} leçon | lesson | Lektion | урокъ | lekcya.

\uventry{instrui} instruire, enseigner | instruct, teach | lehren | учить | uczyć.

\uventry{puni} punir | punish | strafen | наказывать | karać.

\uventry{surda} sourd | deaf | taub | глухой | głuchy.

\uventry{muta} muet | dumb | stumm | нѣмой | niemy.

\uventry{tuŝi} toucher | touch | rühren | трогать | ruszać, dotykać.

\uventry{spegulo} miroir | looking-glass | Spiegel | зеркало | zwierciadło.

\uventry{kara} cher | dear | theuer | дорогой | drogi.

\uventry{ordoni} ordonner | order, command | befehlen | приказывать | rozkazywać.

\uventry{babili} babiller | chatter | schwatzen, plaudern | болтать | paplać.

\uventry{kandelo} chandelle | candle | Licht, Kerze | свѣча | świeca.

\uventry{gaja} gai | gay, glad | lustig | веселый | wesoły.

\uventry{danci} danser | dance | tanzen | танцовать | tańczyć.

\uventry{morti} mourir | die | sterben | умирать | umierać.

\uventry{petoli} faire le polisson, faire des bêtises | be mischievous | muthwillig sein | шалить | swawolić.

\uventry{uj} qui porte, qui contient, qui est peuplé de; ex. \uventry{pomo} pomme ― \uventry{pomujo} pommier; \uventry{cigaro} cigare ― \uventry{cigarujo} porte-cigares; \uventry{Turko} Turc ― \uventry{Turkujo} Turquie | containing, filled with; e.~g. \uventry{cigaro} cigar ― \uventry{cigarujo} cigar-case; \uventry{pomo} apple ― \uventry{pomujo} apple-tree; \uventry{Turko} Turk ― \uventry{Turkujo} Turkey | Behälter, Träger (d. h. Gegenstand worin\ldots{} aufbewahrt wird,\ldots{} Früchte tragende Pflanze, von\ldots{} bevölkertes Land); z.~B. \uventry{cigaro} Cigarre ― \uventry{cigarujo} Cigarrenbüchse; \uventry{pomo} Apfel ― \uventry{pomujo} Apfelbaum; \uventry{Turko} Türke ― \uventry{Turkujo} Türkei | вмѣститель, носитель (т.~е. вещь, въ которой храниться\ldots{}; растеніе несущее\ldots{} или страна, заселенная\ldots{}); напр. \uventry{cigaro} сигара ― \uventry{cigarujo} портъ-сигаръ; \uventry{pomo} яблоко ― \uventry{pomujo} яблоня; \uventry{Turko} Турокъ ― Turkujo \uventry{Турція} | zawierający, noszący (t.~j. przedmiot, w którym się coś przechowuje, roślina, która wydaje owoc, lub kraj, względem zaludniających go mieszkańców; np. \uventry{cigaro} cygaro ― \uventry{cigarujo} cygarnica; \uventry{pomo} jabłko ― \uventry{pomujo} jabłoń; \uventry{Turko} turek ― \uventry{Turkujo} Turcya.

\end{ekzvocab}

\ekzsec[\vskip 1ex\Large{\bookman La feino} \large(Daŭrigo).]{§21.}

Apenaŭ ŝia patrino ŝin rimarkis, ŝi kriis al ŝi: «Nu, mia filino?» ― «Jes, patrino», respondis al ŝi la malĝentilulino, elĵetante unu serpenton kaj unu ranon. ― «Ho, ĉielo!» ekkriis la patrino, «kion mi vidas? Ŝia fratino en ĉio estas kulpa; mi pagos al ŝi por tio ĉi!» Kaj ŝi tuj kuris bati ŝin. La malfeliĉa infano forkuris kaj kaŝis sin en la plej proksima arbaro. La filo de la reĝo, kiu revenis de ĉaso, ŝin renkontis; kaj, vidante, ke ŝi estas tiel bela, li demandis ŝin, kion ŝi faras tie ĉi tute sola kaj pro kio ŝi ploras. ― «Ho ve, sinjoro, mia patrino forpelis min el la domo».

\begin{ekzvocab}{1em}
\uventry{rimarki} remarquer | remark | merken, bemerken | замѣчать | postrzegać, zauwaźać.

\uventry{krii} crier | cry | schreien | кричать | krzyczeć.

\uventry{nu} eh bien! | well! | nu! | nun\footnoteE{Ŝajnerare, ĉar estas \emph{ses} tradukoj anstataŭ kvin; la germana traduko estus ``nu!'' \emph{aŭ} ``nun'', sed la originala disigis ĝin, tiel oni vidas.} | ну! | no!

\uventry{jes} oui | yes | ja | да | tak.

\uventry{ek} indique une action qui commence ou qui est momentanée; ex. \uventry{kanti} chanter ― \uventry{ekkanti} commencer à chanter; \uventry{krii} crier ― \uventry{ekkrii} s’écrier | denotes sudden or momentary action; e.~g. \uventry{krii} cry ― \uventry{ekkrii} cry out | bezeichnet eine anfangende oder momentane Handlung; z.~B. \uventry{kanti} singen ― \uventry{ekkanti} einen Gesang anstimmen; \uventry{krii} schreien ― \uventry{ekkrii} aufschreien | начало или мгновенность; напр. \uventry{kanti} пѣть ― \uventry{ekkanti} запѣть; \uventry{krii} кричать ― \uventry{ekkrii} вскрикнуть | oznacza początek lub chwilowość; np. \uventry{kanti} śpiewać ― \uventry{ekkanti} zaśpiewać; \uventry{krii} krzyczeć ― \uventry{ekkrii} krzyknąć.

\uventry{kulpa} coupable | blameable | schuldig | виноватый | winny.

\uventry{kaŝi} cacher | hide (vb.) | verbergen | прятать | chować.

\uventry{reĝo} roi | king | König | король, царь | król.

\uventry{ĉasi} chasser (vénerie) | hunt | jagen, Jagd machen | охотиться | polować.

\uventry{demandi} demander, questionner | demand, ask | fragen | спрашивать | pytać.

\uventry{sola} seul | only, alone | einzig, allein | единственный | jedyny.

\uventry{pro} à cause de, pour | for the sake of | wegen | ради | dla.

\uventry{ho} oh! | oh! | o! och! | о! охъ! | o! och!.

\uventry{ve} malheur! | woe! | wehe! | увы! | biada! nestety!.

\end{ekzvocab}

\ekzsec{§22.}

Fluanta akvo estas pli pura, ol akvo staranta senmove. ― Promenante sur la strato, mi falis. ― Kiam Nikodemo batas Jozefon, tiam Nikodemo estas la batanto kaj Jozefo estas la batato. ― Al homo, pekinta senintence, Dio facile pardonas. ― Trovinte pomon, mi ĝin manĝis. ― La falinta homo ne povis sin levi. Ne riproĉu vian amikon, ĉar vi mem plimulte\footnoteE{Laŭ fruaj eldonoj de \emph{Ekzercaro}, ĝi estas \emph{pli multe}. } meritas riproĉon; li estas nur unufoja mensoginto\unuakomon{} dum vi estas ankoraŭ nun ĉiam mensoganto. ― La tempo pasinta jam neniam revenos; la tempon venontan neniu ankoraŭ konas. ― Venu, ni atendas vin, Savonto de la mondo. ― En la lingvo «Esperanto» ni vidas la estontan lingvon de la tuta mondo. ― Aŭgusto estas mia plej amata filo. ― Mono havata estas pli grava ol havita. ― Pasero kaptita estas pli bona, ol aglo kaptota. ― La soldatoj kondukis la arestitojn tra la stratoj. ― Li venis al mi tute ne atendite. ― Homo, kiun oni devas juĝi, estas juĝoto.

\begin{ekzvocab}{1em}
\uventry{flui} couler | flow | fliessen | течь | płynąć, cieknąć.

\uventry{movi} mouvoir | move | bewegen | двигать | ruszać.

\uventry{strato} rue | street | Strasse | улица | ulica.

\uventry{fali} tomber | fall | fallen | падать | padać.

\uventry{at} marque le participe présent passif | ending of pres. part. pass. in verbs | bezeichnet das Participium praes. passivi | означаеть причастіе настоящаго времени страд. залога | oznacza imiesłów bierny czasu teraźniejszego.

\uventry{peki} pécher | sin | sündigen | грѣшить | grzeszyć.

\uventry{int} marque le participe passé du verbe actif | ending of past part. act. in verbs | bezeichnet das Participium perfecti activi | означаетъ причастіе прошедшаго времени дѣйствит. залога | oznacza imiesłów czynny czasu przeszłego.

\uventry{intenci} se proposer de | intend | beabsichtigen | намѣреваться | zamierzać.

\uventry{levi} lever | lift, raise | aufheben | поднимать | podnosić.

\uventry{riproĉi} reprocher | reproach | vorwerfen | упрекать | zarzucać.

\uventry{meriti} mériter | merit | verdienen | заслуживать | zasługiwać.

\uventry{mensogi} mentir | tell a lie | lügen | лгать | kłamać.

\uventry{pasi} passer | pass | vergehen | проходить | przechodzić.

\uventry{neniam} ne\ldots{} jamais | never | niemals | никогда | nigdy.

\uventry{ont} marque le participe futur d’un verbe actif | ending of fut. part. act. in verbs | bezeichnet das Participium fut. act. | означаетъ причастіе будущаго времени дѣйствит. залога | oznacza imiesłów czynny czasu przyszłego.

\uventry{neniu} personne | nobody | Niemand | никто | nikt.

\uventry{atendi} attendre | wait, expect | warten, erwarten | ждать, ожидать | czekać.

\uventry{savi} sauver | save | retten | спасать | ratować.

\uventry{mondo} monde | world | Welt | міръ | świat.

\uventry{lingvo} langue, langage | language | Sprache | языкъ, рѣчь | język, mowa.

\uventry{grava} grave, important | important | wichtig | важный | ważny.

\uventry{pasero} passereau | sparrow | Sperling | воробей | wróbel.

\uventry{kapti} attraper | catch | fangen | ловить | chwytać.

\uventry{aglo} aigle | eagle | Adler | орелъ | orzeł.

\uventry{ot} marque le participe futur d’un verbe passif | ending of fut. part. pass. in verbs | bezeichnet das Participium fut. pass. | означаетъ причастіе будущ. времени страд. залога | oznacza imiesłów bierny czasu przyszłego.

\uventry{soldato} soldat | soldier | Soldat | солдатъ | żolnierz.

\uventry{konduki} conduire | conduct | führen | вести | prowadzić.

\uventry{aresti} arrêter | arrest | verhaften | арестовать | aresztować.

\uventry{tra} à travers | through | durch | черезъ, сквозь | przez (wskroś).

\uventry{juĝi} juger | judge | richten, urtheilen | судить | sądzić.

\end{ekzvocab}

\begin{samepage}
\ekzsec[\vskip 1ex\Large{\bookman La feino} \large(Fino)]{§23.}

La reĝido, kiu vidis, ke el ŝia buŝo eliris kelke da perloj kaj kelke da diamantoj, petis ŝin, ke ŝi diru al li, de kie tio ĉi venas. Ŝi rakontis al li sian tutan aventuron. La reĝido konsideris, ke tia kapablo havas pli grandan indon, ol ĉio, kion oni povus doni dote al alia fraŭlino, forkondukis ŝin al la palaco de sia patro, la reĝo, kie li edziĝis je ŝi. Sed pri ŝia fratino ni povas diri, ke ŝi fariĝis tiel malaminda, ke ŝia propra patrino ŝin forpelis de si; kaj la malfeliĉa knabino, multe kurinte kaj trovinte neniun, kiu volus ŝin akcepti, baldaŭ mortis en angulo de arbaro.
\end{samepage}

\begin{ekzvocab}{1em}
\uventry{kelke} quelque | some | mancher, einige | нѣкоторый, нѣсколько | niektóry, kilka.

\uventry{aventuro} aventure | adventure | Abenteuer | приключеніе | przygoda.

\uventry{konsideri} considérer | consider | betrachten, erwägen | соображать | zastanawiać się.

\uventry{inda} mérite, qui mérite, est digne | worthy, valuable | würdig, werth | достойный | godny, wart.

\uventry{doto} dot | dowry | Mitgift | приданое | posag.

\uventry{palaco} palais | palace | Schloss (Gebäude) | дворецъ | pałac.

\uventry{iĝ} se faire, devenir\ldots{}; ex. \uventry{pala} pâle ― \uventry{paliĝi} pâlir; \uventry{sidi} être assis ― \uventry{sidiĝi} s’asseoir | to become; e.~g. \uventry{pala} pale ― \uventry{paliĝi} turn pale; \uventry{sidi} sit ― \uventry{sidiĝi} become seated | zu etwas werden, sich zu etwas veranlassen; z.~B. \uventry{pala} blass ― \uventry{paliĝi} erblassen; \uventry{sidi} sitzen ― \uventry{sidiĝi} sich setzen | дѣлаться чѣмъ нибудь, заставить себя; напр. \uventry{pala} блѣдный ― \uventry{paliĝi} блѣднѣть; \uventry{sidi} сидѣть ― \uventry{sidiĝi} сѣсть | stawać się czemś; np. \uventry{pala} blady ― \uventry{paliĝi} blednąć; \uventry{sidi} siedzieć ― \uventry{sidiĝi} usiąść.

\uventry{je} se traduit par différentes prépositions | can be rendered by various prepositions | kann durch verschiedene Präpositionen übersetzt werden | можетъ быть переведено различным предлогами | może być przetłomaczone za pomocą różnych przyimków.

\uventry{propra} propre (à soi) | own (one’s own) | eigen | собственный | własny.

\uventry{akcepti} accepter | accept | annehmen | принимать | przyjmować.

\uventry{baldaŭ} bientôt | soon | bald | сейчасъ, скоро | zaraz.

\uventry{angulo} coin, angle | corner, angle | Winkel | уголъ | kąt.

\end{ekzvocab}


\ekzsec{§24.}

Nun li diras al mi la veron. ― Hieraŭ li diris al mi la veron. ― Li ĉiam diradis al mi la veron. ― Kiam vi vidis nin en la salono, li jam antaŭe diris al mi la veron (aŭ li estis dirinta al mi la veron). ― Li diros al mi la veron. ― Kiam vi venos al mi, li jam antaŭe diros al mi la veron (aŭ li estos dirinta al mi la veron; aŭ antaŭ ol vi venos al mi, li diros al mi la veron). ― Se mi petus lin, li dirus al mi la veron. ― Mi ne farus la eraron, se li antaŭe dirus al mi la veron (aŭ se li estus dirinta al mi la veron). ― Kiam mi venos, diru al mi la veron. ― Kiam mia patro venos, diru al mi antaŭe la veron (aŭ estu dirinta al mi la veron). ― Mi volas diri al vi la veron. ― Mi volas, ke tio, kion mi diris, estu vera (aŭ mi volas esti dirinta la veron).

\begin{ekzvocab}{1em}
\uventry{salono} salоn | saloon | Salon | залъ | salon.

\uventry{os} marque le futur | ending of future tense in verbs | bezeichnet das Futur | означаетъ будущее время | oznacza czas przyszły.

\end{ekzvocab}

\newpage % necesa

\ekzsec{§25.}

Mi estas amata. Mi estis amata. Mi estos amata. Mi estus amata. Estu amata. Esti amata. ― Vi estas lavita. Vi estis lavita. Vi estos lavita. Vi estus lavita. Estu lavita. Esti lavita. ― Li estas invitota. Li estis invitota. Li estos invitota. Li estus invitota. Estu invitota. Esti invitota. ― Tiu ĉi komercaĵo estas ĉiam volonte aĉetata de mi. ― La surtuto estas aĉetita de mi, sekve ĝi apartenas al mi. ― Kiam via domo estis konstruata, mia domo estis jam longe konstruita. ― Mi sciigas, ke de nun la ŝuldoj de mia filo ne estos pagataj de mi. ― Estu trankvila, mia tuta ŝuldo estos pagita al vi baldaŭ. ― Mia ora ringo ne estus nun tiel longe serĉata, se ĝi ne estus tiel lerte kaŝita de vi. ― Laŭ la projekto de la inĝenieroj tiu ĉi fervojo estas konstruota en la daŭro de du jaroj; sed mi pensas, ke ĝi estos konstruata pli ol tri jarojn. ― Honesta homo agas honeste. ― La pastro, kiu mortis antaŭ nelonge (aŭ antaŭ nelonga tempo), loĝis longe en nia urbo. ― Ĉu hodiaŭ estas varme aŭ malvarme? ― Sur la kameno inter du potoj staras fera kaldrono; el la kaldrono, en kiu sin trovas bolanta akvo, eliras vaporo; tra la fenestro, kiu sin trovas apud la pordo, la vaporo iras sur la korton.

\begin{ekzvocab}{1em}
\uventry{inviti} inviter | invite | einladen | приглашать | zapraszać.

\uventry{komerci} commercer | trade | handeln, Handel treiben | торговать | handlować.

\uventry{aĵ} quelque chose possédant une certaine qualité ou fait d’une certaine matière: ex. \uventry{mola} mou ― \uventry{molaĵo} partie molle d’une chose | made from or possessing the quality of; e.~g. \uventry{malnova} old ― \uventry{malnovaĵo} old thing; \uventry{frukto} fruit ― \uventry{fruktaĵo} something made from fruit | etwas von einer gewissen Eigenschaft, oder aus einem gewissen Stoffe; z.~B. \uventry{malnova} alt ― \uventry{malnovaĵo} altes Zeug; \uventry{frukto} Frucht ― \uventry{fruktaĵo} etwas aus Früchten bereitetes | нѣчто съ даннымъ качествомъ или изъ даннаго матеріала; напр. \uventry{mola} мягкій ― \uventry{molaĵo} мякишъ; \uventry{frukto} плодъ ― \uventry{fruktaĵo} нѣчто приготовленное изъ плодовъ | oznacza przedmiot posiadający pewną własność albo zrobiony z pewnego materjału; np. \uventry{malnova} stary ― \uventry{malnovaĵo} starzyzna; \uventry{frukto} owoc ― \uventry{fruktaĵo} coś zrobineego z owoców.

\uventry{sekvi} suivre | follow | folgen | слѣдовать | nastąpić.

\uventry{konstrui} construire | construct, build | bauen | строить | budować.

\uventry{ŝuldi} devoir (dette) | owe | schulden | быть должнымъ | być dłużnym.

\uventry{oro} or (métal) | gold | Gold | золото | złoto.

\uventry{ringo} anneau | ring (subst.) | Ring | кольцо | pierścień.

\uventry{lerta} adroit, habile | skilful | geschickt, gewandt | ловкій | zręczny.

\uventry{projekto} projet | project | Entwurf | проектъ | projekt.

\uventry{inĝeniero} ingénieur | engineer | Ingenieur | инженеръ | inżynier.

\uventry{fero} fer | iron | Eisen | желѣзо | żelazo.

\uventry{vojo} route, voie | way, road | Weg | дорога | droga.

\uventry{agi} agir | act | handeln, verfahren | поступать | postępować.

\uventry{pastro} prêtre, pasteur | priest, pastor | Priester | жрецъ, священникъ | kapłan.

\uventry{kameno} cheminée | fire-place | Kamin | каминъ | kominek.

\uventry{poto} pot | pot | Topf | горшокъ | garnek.

\uventry{kaldrono} chaudron | kettle | Kessel | котелъ | kocioł.

\uventry{boli} bouillir | boil | sieden | кипѣть | kipieć, wrzeć.

\uventry{vaporo} vapeur | steam | Dampf | паръ | para.

\uventry{pordo} porte | door | Thür | дверь | drzwi.

\uventry{korto} cour | yard, court | Hof | дворъ | podwórze.

\end{ekzvocab}


\ekzsec{§26.}

Kie vi estas? ― Mi estas en la ĝardeno. ― Kien vi iras? ― Mi iras en la ĝardenon. ― La birdo flugas en la ĉambro (= ĝi estas en la ĉambro kaj flugas en ĝi). ― La birdo flugas en la ĉambron (= ĝi estas ekster la ĉambro kaj flugas nun en ĝin). ― Mi vojaĝas en Hispanujo. ― Mi vojaĝas en Hispanujon. ― Mi sidas sur seĝo kaj tenas la piedojn sur benketo. ― Mi metis la manon sur la tablon. ― El sub la kanapo la muso kuris sub la liton, kaj nun ĝi kuras sub la lito. ― Super la tero sin trovas aero. ― Anstataŭ kafo li donis al mi teon kun sukero, sed sen kremo. ― Mi staras ekster la domo, kaj li estas interne. ― En la salono estis neniu krom li kaj lia fianĉino. ― La hirundo flugis trans la riveron, ĉar trans la rivero sin trovis aliaj hirundoj. ― Mi restas tie ĉi laŭ la ordono de mia estro. ― Kiam li estis ĉe mi, li staris tutan horon apud la fenestro. ― Li diras, ke mi estas atenta. ― Li petas, ke mi estu atenta. ― Kvankam vi estas riĉa, mi dubas, ĉu vi estas feliĉa. ― Se vi scius, kiu li estas, vi lin pli estimus. ― Se li jam venis, petu lin al mi. ― Ho, Dio! kion vi faras! ― Ha, kiel bele! ― For de tie ĉi! ― Fi, kiel abomene! ― Nu, iru pli rapide!

\begin{ekzvocab}{1em}
\uventry{ekster} hors, en dehors de | outside, besides | ausser, ausserhalb | внѣ | zewnątrz.

\uventry{vojaĝi} voyager | voyage | reisen | путешествовать | podróżować.

\uventry{piedo} pied | foot | Fuss, Bein | нога | noga.

\uventry{benko} banc | bench | Bank | скамья | ławka.

\uventry{et} marque diminution, décroissance; ex. \uventry{muro} mur ― \uventry{mureto} petit mur; \uventry{ridi} rire ― \uventry{rideti} sourire | denotes diminution of degree; e.~g. \uventry{ridi} laugh ― \uventry{rideti} smile | bezeichnet eine Verkleinerung oder Schwächung; z.~B. \uventry{muro} Wand ― \uventry{mureto} Wändchen; \uventry{ridi} lachen ― \uventry{rideti} lächeln | означаетъ уменьшеніе или ослабленіе степени; напр. \uventry{muro} стѣна ― \uventry{mureto} стѣнка; \uventry{ridi} смѣяться ― \uventry{rideti} улыбаться | oznacza zmniejszenie lub osłabienie stopnia; np. \uventry{muro} ściana ― \uventry{mureto} ścianka; \uventry{ridi} śmiać się ― \uventry{rideti} uśmiechać się.

\uventry{meti} mettre | put, place | hinthun | дѣть; класть | podziać.

\uventry{kanapo} canapé | sofa, lounge | Kanapee | диванъ | kanapa.

\uventry{muso} souris | mouse | Maus | мышь | mysź.

\uventry{lito} lit | bed | Bett | кровать | łóżko.

\uventry{super} au dessus de | over, above | über, oberhalb | надъ | nad.

\uventry{aero} air | air | Luft | воздухъ | powietrze.

\uventry{kafo} café | coffee | Kaffee | кофе | kawa.

\uventry{teo} thé | tea | Thee | чай | herbata.

\uventry{sukero} sucre | sugar | Zucker | сахаръ | cukier.

\uventry{kremo} crème | cream | Schmant, Sahne | сливки | śmietana.

\uventry{interne} à l’intérieur, dedans | within | innerhalb | внутри | wewnątrz.

\uventry{fianĉo} fiancé | betrothed person | Bräutigam | женихъ | narzeczony.

\uventry{hirundo} hirondelle | swallow | Schwalbe | ласточка | jaskółka.

\uventry{trans} au delà | across | jenseit | черезъ, надъ | przez.

\uventry{rivero} rivière, fleuve | river | Fluss | рѣка | rzeka.

\uventry{estro} chef | chief | Vorsteher | начальникъ | zwierzchnik.

\uventry{atenta} attentif | attentive | aufmerksam | внимательный | uważny.

\uventry{kvankam} quoique | although | obgleich | хотя | chociaź.

\uventry{dubi} douter | doubt | zweifeln | сомнѣваться | wątpić.

\uventry{estimi} estimer | esteem | schätzen, achten | уважать | szanować.

\uventry{fi} fi donc! | fie! | pfui! | фи, тьфу | fe!.

\uventry{abomeno} abomination | abomination | Abscheu | отвращеніе | odraza.

\uventry{rapida} rapide, vite | quick, rapid | schnell | быстрый | prędki.

\end{ekzvocab}

\newpage % necesa

\ekzsec{§27.}

La artikolo „la“ estas uzata tiam, kiam ni parolas pri personoj aŭ objektoj konataj. Ĝia uzado estas tia sama\unuakomon{} kiel en la aliaj lingvoj. La personoj, kiuj ne komprenas la uzadon de la artikolo (ekzemple rusoj aŭ poloj, kiuj ne scias alian lingvon krom sia propra), povas en la unua tempo tute ne uzi la artikolon, ĉar ĝi estas oportuna sed ne necesa. Anstataŭ „la“ oni povas ankaŭ diri „l’“ (sed nur post prepozicio, kiu finiĝas per vokalo). ― Vortoj kunmetitaj estas kreataj per simpla kunligado de vortoj; oni prenas ordinare la purajn radikojn, sed, se la bonsoneco aŭ la klareco postulas, oni povas ankaŭ preni la tutan vorton, t.~e. la radikon kune kun ĝia gramatika finiĝo. Ekzemploj: skribtablo aŭ skribotablo (= tablo, sur kiu oni skribas); internacia (= kiu estas inter diversaj nacioj); tutmonda (= de la tuta mondo); unutaga (= kiu daŭras unu tagon); unuataga (= kiu estas en la unua tago); vaporŝipo (= ŝipo, kiu sin movas per vaporo); matenmanĝi, tagmanĝi, vespermanĝi; abonpago (= pago por la abono).

\begin{ekzvocab}{1em}
\uventry{artikolo} article | article | Artikel | членъ, статья | artykuł, przedimek.

\uventry{tiam} alors | then | dann | тогда | wtedy.

\uventry{objekto} objet | object | Gegenstand | предметъ | przedmiot.

\uventry{tia} tel | such | solcher | такой | taki.

\uventry{kompreni} comprendre | understand | verstehen | понимать | rozumieć.

\uventry{ekzemplo} exemple | example | Beispiel | примѣръ | przykład.

\uventry{polo} Polonais | Pole | Pole | Полякъ | Polak.

\uventry{necesa} nécessaire | necessary | nothwendig | необходимый | niezbędny.

\uventry{prepozicio} préposition | preposition | Vorwort | предлогъ | przyimek.

\uventry{vokalo} voyelle | vowel | Vokal | гласная | samogłoska.

\uventry{kunmeti} composer | compound | zusammensetzen | слагать | składać.

\uventry{simpla} simple | simple | einfach | простой | prosty, zwyczajny.

\uventry{ligi} lier | bind, tie | binden | связывать | wiązać.

\uventry{radiko} racine | root | Wurzel | корень | korzeń.

\uventry{soni} sonner, rendre des sons | sound | tönen, lauten | звучать | brzmieć.

\uventry{klara} clair | clear | klar | ясный | jasny.

\uventry{postuli} exiger, requérir | require, claim | fordern | требовать | żądać.

\uventry{gramatiko} grammaire | grammar | Grammatik | грамматика | gramatyka.

\uventry{nacio} nation | nation | Nation | нація, народъ | naród, nacja.

\uventry{diversa} divers | various, diverse | verschieden | различный | różny.

\uventry{ŝipo} navire | ship | Schiff | корабль | okręt.

\uventry{matenmanĝi} déjeuner | breakfast | frühstücken | завтракать | śniadać.

\uventry{aboni} abonner | subscribe | abonniren | подписываться | prenumerować.

\end{ekzvocab}


\ekzsec{§28.}

Ĉiuj prepozicioj per si mem postulas ĉiam nur la nominativon. Se ni iam post prepozicio uzas la akuzativon, la akuzativo tie dependas ne de la prepozicio, sed de aliaj kaŭzoj. Ekzemple: por esprimi direkton, ni aldonas al la vorto la finon „n“; sekve: tie (= en tiu loko), tien (= al tiu loko); tiel same ni ankaŭ diras: „la birdo flugis en la ĝardenon, sur la tablon“, kaj la vortoj „ĝardenon“, „tablon“ staras tie ĉi en akuzativo ne ĉar la prepozicioj „en“ kaj „sur“ tion ĉi postulas, sed nur ĉar ni volis esprimi direkton, t.~e. montri, ke la birdo sin ne trovis antaŭe en la ĝardeno aŭ sur la tablo kaj tie flugis, sed ke ĝi de alia loko flugis al la ĝardeno, al la tablo (ni volas montri, ke la ĝardeno kaj tablo ne estis la loko de la flugado, sed nur la celo de la flugado); en tiaj okazoj ni uzus la finiĝon „n“ tute egale\unuakomon{} ĉu ia prepozicio starus aŭ ne. ― Morgaŭ mi veturos Parizon (aŭ en Parizon). ― Mi restos hodiaŭ dome. ― Jam estas tempo iri domen. ― Ni disiĝis kaj iris en diversajn flankojn: mi iris dekstren, kaj li iris maldekstren. ― Flanken, sinjoro! ― Mi konas neniun en tiu ĉi urbo. ― Mi neniel povas kompreni, kion vi parolas. ― Mi renkontis nek lin, nek lian fraton (aŭ mi ne renkontis lin, nek lian fraton).

\begin{ekzvocab}{1em}
\uventry{nominativo} nominatif | nominative | Nominativ | именительный падежъ | pierwszy przypadek.

\uventry{iam} jamais, un jour | at any time, ever | irgend wann | когда-нибудь | kiedyś.

\uventry{akuzativo} accusatif | accusative | Accusativ | винительный падежъ | czwarty przypadek.

\uventry{tie} là-bas, là, y | there | dort | тамъ | tam.

\uventry{dependi} dépendre | depend | abhängen | зависѣть | zależeć.

\uventry{kaŭzo} cause | cause | Ursache | причина | przyczyna.

\uventry{esprimi} exprimer | express | ausdrücken | выражать | wyrażać.

\uventry{direkti} diriger | direct | richten | направлять | kierować.

\uventry{celi} viser | aim | zielen | цѣлиться | celować.

\uventry{egala} égal | equal | gleich | одинаковый | jednakowy.

\uventry{ia} quelconque | of any kind | irgend welcher | какой-нибудь | jakiś.

\uventry{veturi} aller, partir | journey, travel | fahren | ѣхать | jechać.

\uventry{dis} marque division, dissémination; ex. \uventry{iri} aller ― \uventry{disiri} se séparer, aller chacun de son côté | has the same force as the English prefix dis; e.~g. \uventry{semi} sow ― \uventry{dissemi} disseminate; \uventry{ŝiri} tear ― \uventry{disŝiri} tear to pieces | zer-; z.~B. \uventry{ŝiri} reissen ― \uventry{disŝiri} zerreissen | раз-; напр. \uventry{ŝiri} рвать ― \uventry{disŝiri} разрывать | roz-; np. \uventry{ŝiri} rwać ― \uventry{disŝiri} rozrywać.

\uventry{flanko} côté | side | Seite | сторона | strona.

\uventry{dekstra} droit, droite | right-hand | recht | правый | prawy.

\uventry{neniel} nullement, en aucune façon | nohow | keineswegs, auf keine Weise | никакъ | w żaden sposób.

\uventry{nek} --- \uventry{nek} ni ― ni | neither ― nor | weder ― noch | ни ― ни | ani ― ani.

\end{ekzvocab}


\ekzsec{§29.}

Se ni bezonas uzi prepozicion kaj la senco ne montras al ni, kian prepozicion uzi, tiam ni povas uzi la komunan prepozicion „je“. Sed estas bone uzadi la vorton „je“ kiel eble pli malofte. Anstataŭ la vorto „je“ ni povas ankaŭ uzi akuzativon sen prepozicio. ― Mi ridas je lia naiveco (aŭ mi ridas pro lia naiveco, aŭ: mi ridas lian naivecon). ― Je la lasta fojo mi vidas lin ĉe vi (aŭ: la lastan fojon). ― Mi veturis du tagojn kaj unu nokton. ― Mi sopiras je mia perdita feliĉo (aŭ: mian perditan feliĉon). ― El la dirita regulo sekvas, ke se ni pri ia verbo ne scias, ĉu ĝi postulas post si la akuzativon (t. e. ĉu ĝi estas aktiva) aŭ ne, ni povas ĉiam uzi la akuzativon. Ekzemple, ni povas diri „obei al la patro“ kaj „obei la patron“ (anstataŭ „obei je la patro“). Sed ni ne uzas la akuzativon tiam, kiam la klareco de la senco tion ĉi malpermesas; ekzemple: ni povas diri „pardoni al la malamiko“ kaj „pardoni la malamikon“, sed ni devas diri ĉiam „pardoni al la malamiko lian kulpon“.

\begin{ekzvocab}{1em}
\uventry{senco} sens, acception | sense | Sinn | смыслъ | sens, znaczenie.

\uventry{komuna} commun | common | gemeinsam | общій | ogólny, wspólny.

\uventry{ebla} possible | able, possible | möglich | возможный | możliwy.

\uventry{ofte} souvent | often | oft | часто | często.

\uventry{ridi} rire | laugh | lachen | смѣяться | śmiać się.

\uventry{lasta} dernier | last, latest | letzt | послѣдній | ostatni.

\uventry{sopiri} soupirer | sigh, long for | sich sehnen | тосковать | tęsknić.

\uventry{regulo} règle | rule | Regel | правило | prawidło.

\uventry{verbo} verbe | verb | Zeitwort | глаголъ | czasownik.

\uventry{obei} obéir | obey | gehorchen | повиноваться | być posłusznym.

\uventry{permesi} permettre | permit, allow | erlauben | позволять | pozwalać.

\end{ekzvocab}


\ekzsec{§30.}

Ia, ial, iam, ie, iel, ies, io, iom, iu. ― La montritajn naŭ vortojn ni konsilas bone ellerni, ĉar el ili ĉiu povas jam fari al si grandan serion da aliaj pronomoj kaj adverboj. Se ni aldonas al ili la literon „k“, ni ricevas vortojn demandajn aŭ rilatajn: kia, kial, kiam, kie, kiel, kies, kio, kiom, kiu. Se ni aldonas la literon „t“, ni ricevas vortojn montrajn: tia, tial, tiam, tie, tiel, ties, tio, tiom, tiu. Aldonante la literon „ĉ“, ni ricevas vortojn komunajn: ĉia, ĉial, ĉiam, ĉie, ĉiel, ĉies, ĉio, ĉiom, ĉiu. Aldonante la prefikson „nen“, ni ricevas vortojn neajn: nenia, nenial, neniam, nenie, neniel, nenies, nenio, neniom, neniu. Aldonante al la vortoj montraj la vorton „ĉi“, ni ricevas montron pli proksiman; ekzemple: tiu (pli malproksima), tiu ĉi (aŭ ĉi tiu) (pli proksima); tie (malproksime), tie ĉi aŭ ĉi tie (proksime). Aldonante al la vortoj demandaj la vorton „ajn“, ni ricevas vortojn sendiferencajn: kia ajn, kial ajn, kiam ajn, kie ajn, kiel ajn, kies ajn, kio ajn, kiom ajn, kiu ajn. Ekster tio el la diritaj vortoj ni povas ankoraŭ fari aliajn vortojn, per helpo de gramatikaj finiĝoj kaj aliaj vortoj (sufiksoj); ekzemple: tiama, ĉiama, kioma, tiea, ĉi-tiea, tieulo, tiamulo k.~t.~p. (= kaj tiel plu).

\begin{ekzvocab}{1em}
\uventry{ia} quelconque, quelque | of any kind | irgend welcher | какой-нибудь | jakiś.

\uventry{ial} pour une raison quelconque | for any cause | irgend warum | почему-нибудь | dla jakiejś przyczyny.

\uventry{iam} jamais, un jour | at any time, ever | irgend wann, einst | когда-нибудь | kiedyś.

\uventry{ie} quelque part | any where | irgend wo | гдѣ-нибудь | gdzieś.

\uventry{iel} d’une manière quelconque | anyhow | irgend wie | какъ-нибудь | jakoś.

\uventry{ies} de quelqu’un | anyone’s | irgend jemandes | чей-нибудь | czyjś.

\uventry{io} quelque chose | anything | etwas | что-нибудъ | coś.

\uventry{iom} quelque peu | any quantity | ein wenig, irgend wie viel | сколько-нибудь | ilekolwiek.

\uventry{iu} quelqu’un | any one | jemand | кто-нибудь | ktoś.

\uventry{konsili} conseiller | advise, counsel | rathen | совѣтовать | radzić.

\uventry{serio} série | series | Reihe | рядъ, серія | serya.

\uventry{pronomo} pronom | pronoun | Fürwort | мѣстоименіе | zaimek.

\uventry{adverbo} adverbe | adverb | Nebenwort | нарѣчіе | przysłówek.

\uventry{litero} lettre (de l’alphabet) | letter (of the alphabet) | Buchstabe | буква | litera.

\uventry{rilati} concerner; avoir rapport à | be related to | sich beziehen | относиться | odnosić się.

\uventry{prefikso} préfixe | prefix | Präfix | приставка | przybranka.

\uventry{ajn} que ce soit | ever | auch nur | бы-ни | kolwiek, bądź.

\uventry{diferenci} différer (v. n.) | differ | sich unterscheiden | различаться | różnic się.

\uventry{helpi} aider | help | helfen | помогать | pomagać.

\uventry{sufikso} suffixe | suffix | Suffix | суффиксъ | przyrostek.

\end{ekzvocab}


\ekzsec{§31.}

Lia kolero longe daŭris. ― Li estas hodiaŭ en kolera humoro. ― Li koleras kaj insultas. ― Li fermis kolere la pordon. ― Lia filo mortis kaj estas nun malviva. ― La korpo estas morta, la animo estas senmorta. ― Li estas morte malsana, li ne vivos pli, ol unu tagon. ― Li parolas, kaj lia parolo fluas dolĉe kaj agrable. ― Ni faris la kontrakton ne skribe, sed parole. ― Li estas bona parolanto. ― Starante ekstere, li povis vidi nur la eksteran flankon de nia domo. ― Li loĝas ekster la urbo. ― La ekstero de tiu ĉi homo estas pli bona, ol lia interno. ― Li tuj faris, kion mi volis, kaj mi dankis lin por la tuja plenumo de mia deziro. ― Kia granda brulo! kio brulas? ― Ligno estas bona brula materialo. ― La fera bastono, kiu kuŝis en la forno, estas nun brule varmega. ― Ĉu li donis al vi jesan respondon aŭ nean? Li eliris el la dormoĉambro kaj eniris en la manĝoĉambron. ― La birdo ne forflugis: ĝi nur deflugis de la arbo, alflugis al la domo kaj surflugis sur la tegmenton. ― Por ĉiu aĉetita funto da teo tiu ĉi komercisto aldonas senpage funton da sukero. ― Lernolibron oni devas ne tralegi, sed tralerni. ― Li portas rozokoloran superveston kaj teleroforman ĉapelon. ― En mia skribotablo sin trovas kvar tirkestoj. ― Liaj lipharoj estas pli grizaj, ol liaj vangharoj.

\begin{ekzvocab}{1em}
\uventry{humoro} humeur | humor | Laune | расположеніе духа | humor.

\uventry{fermi} fermer | shut | schliessen, zumachen | запирать | zamykać.

\uventry{korpo} corps | body | Körper | тѣло | ciało.

\uventry{animo} âme | soul | Seele | душа | dusza.

\uventry{kontrakti} contracter | contract | einen Vertrag abschliessen | заключать договоръ | zawierać umowę.

\uventry{um} suffixe peu employé, et qui reçoit différents sens aisément suggérés par le contexte et la signification de la racine à laquelle il est joint | this syllable has no fixed meaning | Suffix von verschiedener Bedeutung | суффиксъ безъ постояннаго значенія | przyrostek nie mający stlałego znaczenia.

\uvsubentry{}\uventry{(plenumi} accomplir | fulfil, accomplish | erfüllen | исполнять | spełniać.)

\uventry{bruli} brûler (être en feu) | burn (v. n.) | brennen (v. n.) | горѣть | palić się.

\uventry{ligno} bois | wood (the substance) | Holz | дерево, дрова | drzewo, drwa.

\uventry{materialo} matière | material | Stoff | матеріялъ | materjał.

\uventry{bastono} bâton | stick | Stock | палка | kij, laska.

\uventry{tegmento} toit | roof | Dach | крыша | dach.

\uventry{funto} livre | pound | Pfund | фунтъ | funt.

\uventry{ist} marque la profession; ex. \uventry{boto} botte ― \uventry{botisto} bottier; \uventry{maro} mer ― \uventry{maristo} marin | person occupied with; e.~g. \uventry{boto} boot ― \uventry{botisto} boot-maker; \uventry{maro} sea ― \uventry{maristo} sailor | sich mit etwas beschäftigend; z.~B. \uventry{boto} Stiefel ― \uventry{botisto} Schuster; \uventry{maro} Meer ― \uventry{maristo} Seeman | занимающійся; напр. \uventry{boto} сапогъ ― \uventry{botisto} сапожникъ; \uventry{maro} море ― \uventry{maristo} морякъ | zajmujący się; np. \uventry{boto} but ― \uventry{botisto} szewc; \uventry{maro} morze ― \uventry{maristo} marynarz.

\uventry{koloro} couleur | color | Farbe | краска, цвѣтъ | kolor.

\uventry{supre}\footnoteE{La vortoj \emph{supre} kaj \emph{tero}, kvankam en ĉi tiu vortaro, ne koncernas ĉi tiun Ekzercon.} en haut | above, upper | oben | вверху | na górze.

\uventry{telero} assiette | plate | Teller | тарелка | talerz.

\uventry{tero}\footnoteE{Vidu la antaŭan piednoton.} terre | earth | Erde | земля | ziemia.

\uventry{kesto} caisse, coffre | chest, box | Kiste, Kasten, Lade | ящикъ | skrzynia.

\uventry{lipo} lèvre | lip | Lippe | губа | warga.

\uventry{haro} cheveu | hair | Haar | волосъ | włos.

\uventry{griza} gris | grey | grau | сѣрый, сѣдой | szary, siwy.

\uventry{vango} joue | check | Wange | щека | policzek.

\end{ekzvocab}


\ekzsec{§32.}

Teatramanto ofte vizitas la teatron kaj ricevas baldaŭ teatrajn manierojn. ― Kiu okupas sin je meĥaniko, estas meĥanikisto, kaj kiu okupas sin je ĥemio, estas ĥemiisto. ― Diplomatiiston oni povas ankaŭ nomi diplomato, sed fizikiston oni ne povas nomi fiziko, ĉar fiziko estas la nomo de la scienco mem. ― La fotografisto fotografis min, kaj mi sendis mian fotografaĵon al mia patro. ― Glaso de vino estas glaso, en kiu antaŭe sin trovis vino, aŭ kiun oni uzas por vino; glaso da vino estas glaso plena je vino. ― Alportu al mi metron da nigra drapo (Metro de drapo signifus metron, kiu kuŝis sur drapo, aŭ kiu estas uzata por drapo). ― Mi aĉetis dekon da ovoj. ― Tiu ĉi rivero havas ducent kilometrojn da longo. ― Sur la bordo de la maro staris amaso da homoj. ― Multaj birdoj flugas en la aŭtuno en pli varmajn landojn. ― Sur la arbo sin trovis multe (aŭ multo) da birdoj. ― Kelkaj homoj sentas sin la plej feliĉaj, kiam ili vidas la suferojn de siaj najbaroj. ― En la ĉambro sidis nur kelke da homoj. ― „Da“ post ia vorto montras, ke tiu ĉi vorto havas signifon de mezuro.

\begin{ekzvocab}{1em}
\uventry{teatro} theâtre | theatre | Theater | театръ | teatr.

\uventry{maniero} manière, façon | manner | Manier, Weise, Art | способъ, манера | sposób, manjera.

\uventry{okupi} occuper | occupy | einnehmen, beschäftigen | занимать | zajmować.

\uventry{meĥaniko} mécanique | mechanics | Mechanik | механика | mechanika.

\uventry{ĥemio} chimie | chemistry | Chemie | химія | chemia.

\uventry{diplomatio} diplomatie | diplomacy | Diplomatie | дипломатія | dyplomacja.

\uventry{fiziko} phyique | phyics | Phyik | физика | fizyka.

\uventry{scienco} science | science | Wissenschaft | наука | nauka.

\uventry{glaso} verre (à boire) | glass | Glas (Gefäss) | стаканъ | szklanka.

\uventry{nigra} noir | black | schwarz | черный | czarny.

\uventry{drapo} drap | woollen goods | Tuch (wollenes Gewebe) | сукно | sukno.

\uventry{signifi} signifier | signify, mean | bezeichnen, bedeuten | означать | oznaczać.

\uventry{ovo} œuf | egg | Ei | яйцо | jajko.

\uventry{bordo} bord, rivage | shore | Ufer | берегъ | brzeg.

\uventry{maro} mer | sea | Meer | море | morze.

\uventry{amaso} amas, foule | crowd, mass | Haufen, Menge | куча, толпа | kupa, tłum.

\uventry{aŭtuno} automne | autumn | Haufen, Menge | осень | jesień.

\uventry{lando} pays | land, country | Land | страна | kraj.

\uventry{suferi} souffrir, endurer | suffer | leiden | страдать | cierpieć.

\uventry{najbaro} voisin | neighbour | Nachbar | сосѣдъ | sąsiad.

\uventry{mezuri} mesurer | measure | messen | мѣрить | mierzyć.

\end{ekzvocab}


\ekzsec{§33.}

Mia frato ne estas granda, sed li ne estas ankaŭ malgranda: li estas de meza kresko. ― Li estas tiel dika, ke li ne povas trairi tra nia mallarĝa pordo. ― Haro estas tre maldika. ― La nokto estis tiel malluma, ke ni nenion povis vidi eĉ antaŭ nia nazo. ― Tiu ĉi malfreŝa pano estas malmola, kiel ŝtono. ― Malbonaj infanoj amas turmenti bestojn. ― Li sentis sin tiel malfeliĉa, ke li malbenis la tagon, en kiu li estis naskita. ― Ni\footnoteE{La fruaj eldonoj de la \emph{Ekzercaro} havis ``Mi''.} forte malestimas tiun ĉi malnoblan homon. ― La fenestro longe estis nefermita; mi ĝin fermis, sed mia frato tuj ĝin denove malfermis. ― Rekta vojo estas pli mallonga, ol kurba. ― La tablo staras malrekte kaj kredeble baldaŭ renversiĝos. ― Li staras supre sur la monto kaj rigardas malsupren sur la kampon. ― Malamiko venis en nian landon. ― Oni tiel malhelpis al mi, ke mi malbonigis mian tutan laboron. ― La edzino de mia patro estas mia patrino kaj la avino de miaj infanoj. ― Sur la korto staras koko kun tri kokinoj. ― Mia fratino estas tre bela knabino. ― Mia onklino estas bona virino. ― Mi vidis vian avinon kun ŝiaj kvar nepinoj kaj kun mia nevino. ― Lia duonpatrino estas mia bofratino. ― Mi havas bovon kaj bovinon. ― La juna vidvino fariĝis denove fianĉino.

\begin{ekzvocab}{1em}
\uventry{mezo} milieu | middle | Mitte | средина | środek.

\uventry{kreski} croître | grow, increase | wachsen | рости | rosnąć.

\uventry{dika} gros | thick, fat | dick | толстый | gruby.

\uventry{larĝa} large | broad | breit | широкій | szeroki.

\uventry{lumi} luire | light | leuchten | свѣтить | świecić.

\uventry{mola} mou | soft | weich | мягкій | miękki.

\uventry{turmenti} tourmenter | torment | quälen, martern | мучить | męczyć.

\uventry{senti} ressentir, éprouver | feel, perceive | fühlen | чувствовать | czuć.

\uventry{beni} bénir | bless | segnen | благословлять | błogosławić.

\uventry{nobla} noble | noble | edel | благородный | szlachetny.

\uventry{rekta} droit, direct | straight | gerade | прямой | prosty.

\uventry{kurba} courbe, tortueux | curved | krumm | кривой | krzywy.

\uventry{kredi} croire | believe | glauben | вѣрить | wierzyć.

\uventry{renversi} renverser | upset | umwerfen, umstürzen | опрокидывать | przewracać.

\uventry{monto} montagne | mountain | Berg | гора | góra.

\uventry{kampo} champ, campagne | field | Feld | поле | pole.

\uventry{koko} coq | cock | Hahn | пѣтухъ | kogut.

\uventry{nepo} petit-fils | grandson | Enkel | внукъ | wnuk.

\uventry{nevo} neveu | nephew | Neffe | племянникъ | siostrzeniec, bratanek.

\begin{minipage}{\textwidth}
\uventry{bo} marque la parenté résultant du mariage; ex. \uventry{patro} père ― \uventry{bopatro} beau-père | relation by marriage; e.~g. \uventry{patrino} mother ― \uventry{bopatrino} mother-in-law | durch Heirath erworben; z.~B. \uventry{patro} Vater ― \uventry{bopatro} Schwiegervater; \uventry{frato} Bruder ― \uventry{bofrato} Schwager | пріобрѣтенный бракомъ; напр. \uventry{patro} отецъ ― \uventry{bopatro} тесть, свекоръ; \uventry{frato} братъ ― \uventry{bofrato} шуринъ, зять, деверь, | nabyty przez małżeństwo; np. \uventry{patro} ojciec ― \uventry{bopatro} teść; \uventry{frato} brat ― \uventry{bofrato} szwagier.

\uventry{duonpatro} beau-père | step-father | Stiefvater | отчимъ | ojczym.

\uventry{bovo} bœuf | ox | Ochs | быкъ | byk.
\end{minipage}

\end{ekzvocab}

\begin{samepage}
\ekzsec{§34.}

La tranĉilo estis tiel malakra, ke mi ne povis tranĉi per ĝi la viandon kaj mi devis uzi mian poŝan tranĉilon. ― Ĉu vi havas korktirilon, por malŝtopi la botelon? ― Mi volis ŝlosi la pordon, sed mi perdis la ŝlosilon. ― Ŝi kombas al si la harojn per arĝenta kombilo. ― En somero ni veturas per diversaj veturiloj, kaj en vintro ni veturas per glitveturilo. ― Hodiaŭ estas bela frosta vetero, tial mi prenos miajn glitilojn kaj iros gliti. ― Per hakilo ni hakas, per segilo ni segas, per fosilo ni fosas, per kudrilo ni kudras, per tondilo ni tondas, per sonorilo ni sonoras, per fajfilo ni fajfas. ― Mia skribilaro konsistas el inkujo, sablujo, kelke da plumoj, krajono kaj inksorbilo. ― Oni metis antaŭ mi manĝilaron, kiu konsistis el telero, kulero, tranĉilo, forko, glaseto por brando, glaso por vino kaj telertuketo. ― En varmega tago mi amas promeni en arbaro. ― Nia lando venkos, ĉar nia militistaro estas granda kaj brava. ― Sur kruta ŝtuparo li levis sin al la tegmento de la domo. ― Mi ne scias la lingvon hispanan, sed per helpo de vortaro hispana-germana mi tamen komprenis iom vian leteron. ― Sur tiuj ĉi vastaj kaj herboriĉaj kampoj paŝtas sin grandaj brutaroj, precipe aroj da bellanaj ŝafoj.
\end{samepage}

\begin{ekzvocab}{1em}
\uventry{viando} viande | meat, flesch\eraro{flesh} | Fleisch | мясо | mięso.

\uventry{poŝo} poche | pocket | Tasche | карманъ | kieszeń.

\uventry{korko} bouchon | cork | Kork | пробка | korek.

\uventry{tiri} tirer | draw, pull, drag | ziehen | тянуть | ciągnąć.

\uventry{ŝtopi} boucher | stop, fasten down | stopfen | затыкать | zatykać.

\uventry{botelo} bouteille | bottle | Flasche | бутылка | butelka.

\uventry{ŝlosi} fermer à clef | lock, fasten | schliessen | запирать на ключъ | zamykać na klucz.

\uventry{kombi} peigner | comb | kämmen | чесать | czesać.

\uventry{somero} été | summer | Sommer | лѣто | lato.

\uventry{gliti} glisser | sakte | gleiten, glitschen | скользить, кататься | ślizgać się.

\uventry{frosto} gelée | frost | Frost | морозъ | mróz.

\uventry{vetero} temps (température) | weather | Wetter | погода | pogoda.

\uventry{haki} hacher, abattre | hew, chop | hauen, hacken | рубить | rąbać.

\uventry{segi} scier | saw | sägen | пилить | piłować.

\uventry{fosi} creuser | dig | graben | копать | kopać.

\uventry{kudri} coudre | sew | nähen | шить | szyć.

\uventry{tondi} tondre | clip, shear | scheeren | стричь | strzydz.

\uventry{sonori} tinter | give out a sound (as a bell) | klingen | звенѣть | brzęczeć, dzwonić.

\uventry{fajfi} siffler | whistle | pfeifen | свистать | świstać.

\uventry{inko} encre | ink | Tinte | чернила | atrament.

\uventry{sablo} sable | sand | Sand | песокъ | piasek.

\uventry{sorbi} humer | sip | schlürfen | хлебать | chlipać.

\uventry{brando} eau-de-vie | brandy | Branntwein | водка | wódka.

\uventry{tuko} mouchoir | cloth | Tuch (Hals-, Schnupf- etc.) | платокъ | chustka.

\uventry{militi} guerroyer | fight | Krieg führen | воевать | wojować.

\uventry{brava} brave, solide | valliant, brave | tüchtig | дѣльный, удалый | dzielny, chwacki.

\uventry{kruta} roide, escarpé | steep | steil | крутой | stromy.

\uventry{ŝtupo} marche, échelon | step | Stufe | ступень | stopień.

\uventry{Hispano} Espagnol | Spaniard | Spanier | Испанецъ | Hiszpan.

\uventry{Germano} Allemand | German | Deutscher | Нѣмецъ | Niemiec.

\uventry{tamen} pourtant, néanmoins | however, nevertheless | doch, jedoch | однако | jednak.

\uventry{vasta} vaste, étendu | wide, vast | weit, geräumig | обширный, просторный | obszerny.

\uventry{herbo} herbe | grass | Gras | трава | trawa.

\uventry{paŝti} paître | pasture, feed animals | weiden lassen | пасти | paść.

\uventry{bruto} brute, bétail | brute | Vieh | скотъ | bydło.

\uventry{precipe} principalement, surtout | particularly | besonders, vorzüglich | преимущественно | szczególnie.

\uventry{lano} laine | wool | Wolle | шерсть | wełna.

\uventry{ŝafo} mouton | sheep | Schaf | овца | owca.

\end{ekzvocab}


\ekzsec{§35.}

Vi parolas sensencaĵon, mia amiko. ― Mi trinkis teon kun kuko kaj konfitaĵo. ― Akvo estas fluidaĵo. ― Mi ne volis trinki la vinon, ĉar ĝi enhavis en si ian suspektan malklaraĵon. ― Sur la tablo staris diversaj sukeraĵoj. ― En tiuj ĉi boteletoj sin trovas diversaj acidoj: vinagro, sulfuracido, azotacido kaj aliaj. ― Via vino estas nur ia abomena acidaĵo. ― La acideco de tiu ĉi vinagro estas tre malforta. ― Mi manĝis bongustan ovaĵon. ― Tiu ĉi granda altaĵo ne estas natura monto. ― La alteco de tiu monto ne estas tre granda. ― Kiam mi ien veturas, mi neniam prenas kun mi multon da pakaĵo. ― Ĉemizojn, kolumojn, manumojn kaj ceterajn similajn objektojn ni nomas tolaĵo, kvankam ili ne ĉiam estas faritaj el tolo. ― Glaciaĵo estas dolĉa glaciigita frandaĵo. ― La riĉeco de tiu ĉi homo estas granda, sed lia malsaĝeco estas ankoraŭ pli granda. ― Li amas tiun ĉi knabinon pro ŝia beleco kaj boneco. ― Lia heroeco tre plaĉis al mi. ― La tuta supraĵo de la lago estis kovrita per naĝantaj folioj kaj diversaj aliaj kreskaĵoj. ― Mi vivas kun li en granda amikeco.

\begin{ekzvocab}{1em}
\uventry{kuko} gâteau | cаке | Kuchen | пирогъ | pieroźek.

\uventry{konfiti} confire | preserve with sugar | einmachen (mit Zucker) | варить въ сахарѣ | smażyć w cukrze.

\uventry{fluida} liquide | fluid | flüssig | жидкій | płynny.

\uventry{suspekti} suspecter, soupçonner | suspect | verdächtigen | подозрѣвать | podejrzewać.

\uventry{acida} aigre | sour | sauer | кислый | kwaśny.

\uventry{vinagro} vinaigre | vinegar | Essig | уксусъ | ocet.

\uventry{sulfuro} soufre | sulphur | Schwefel | сѣра | siara.

\uventry{azoto} azote | azotе | Stickstoff | азоть | azot.

\uventry{gusto} goût | taste | Geschmack | вкусъ | smak, gust.

\uventry{alta} haut | high | hoch | высокій | wysoki.

\uventry{naturo} nature | nature | Natur | природа | przyroda.

\uventry{paki} empaqueter, emballer | pack, put ut | packen, einpacken | укладывать, упаковывать | pakować.

\uventry{ĉemizo} chemise | shirt | Hemd | сорочка | koszula.

\uventry{kolo} cou | neck | Hals | шея | szyja.

\uventry{cetera} autre (le reste) | rest, remainder | übrig | прочій | pozostaly\eraro{pozostały}.

\uventry{tolo} toile | linen | Leinwand | полотно | płótno.

\uventry{glacio} glace | ice | Eis | ледъ | lód.

\uventry{frandi} goûter par friandise | dainty | naschen | лакомиться | złakomić się.

\uventry{heroo} héros | hero, champion | Held | герой | bohater.

\uventry{plaĉi} plaire | please | gefallen | нравиться | podobać się.

\uventry{lago} lac | lake | See (der) | озеро | jezioro.

\uventry{kovri} couvrir | cover | verdecken, verhüllen | закрывать | zakrywać.

\begin{minipage}{\textwidth}
\uventry{naĝi} nager | swim | schwimmen | плавать | pływać.

\uventry{folio} feuille | leaf | Blatt, Bogen | листъ | liść, arkusz.
\end{minipage}
\end{ekzvocab}

\newpage % necesa

\ekzsec{§36.}

Patro kaj patrino kune estas nomataj gepatroj. ― Petro, Anno kaj Elizabeto estas miaj gefratoj. ― Gesinjoroj N. hodiaŭ vespere venos al ni. ― Mi gratulis telegrafe la junajn geedzojn. ― La gefianĉoj staris apud la altaro. ― La patro de mia edzino estas mia bopatro, mi estas lia bofilo, kaj mia patro estas la bopatro de mia edzino. ― Ĉiuj parencoj de mia edzino estas miaj boparencoj, sekve ŝia frato estas mia bofrato, ŝia fratino estas mia bofratino; mia frato kaj fratino (gefratoj) estas la bogefratoj de mia edzino. ― La edzino de mia nevo kaj la nevino de mia edzino estas miaj bonevinoj. ― Virino, kiu kuracas, estas kuracistino; edzino de kuracisto estas kuracistedzino. ― La doktoredzino A. vizitis hodiaŭ la gedoktorojn P. ― Li ne estas lavisto, li estas lavistinedzo. ― La filoj, nepoj kaj pranepoj de reĝo estas reĝidoj. ― La hebreoj estas Izraelidoj, ĉar ili devenas de Izraelo. ― Ĉevalido estas nematura ĉevalo, kokido ― nematura koko, bovido ― nematura bovo, birdido ― nematura birdo.

\begin{ekzvocab}{1em}
\uventry{ge} les deux sexes réunis; ex. \uventry{patro} père ― \uventry{gepatroj} les parents (père et mère) | of both sexes; e.~g. \uventry{patro} father ― \uventry{gepatroj} parents | beiderlei Geschlechtes; z.~B. \uventry{patro} Vater ― \uventry{gepatroj} Eltern; \uventry{mastro} Wirth ― \uventry{gemastroj} Wirth und Wirthin | обоего пола, напр. \uventry{patro} отецъ ― \uventry{gepatroj} родители; \uventry{mastro} хозяинъ ― \uventry{gemastroj} хозяинъ съ хозяйкой | obojej płci, np. \uventry{patro} ojciec ― \uventry{gepatroj} rodzice; \uventry{mastro} gospodarz ― \uventry{gemastroj} gospodarstwo (gospodarz i gospodyni).

\uventry{gratuli} féliciter | congratulate | gratuliren | поздравлять | winszować.

\uventry{altaro} autel | altar | Altar | алтарь | ołtarz.

\uventry{kuraci} traiter (une maladie) | cure, heal | kuriren, heilen | лѣчить | leczyć.

\uventry{doktoro} docteur | doctor | Doctor | докторъ | doktór.

\uventry{pra} bis-, arrière- | great-, primordial | ur- | пра- | pra-.

\uventry{id} enfant, descendant; ex. \uventry{bovo} bœuf ― \uventry{bovido} veau; \uventry{Izraelo} Israël ― \uventry{Izraelido} Israëlite | descendant, young one; e.~g. \uventry{bovo} ox ― \uventry{bovido} calf | Kind, Nachkomme; z.~B. \uventry{bovo} Ochs ― \uventry{bovido} Kalb; \uventry{Izraelo} Israel ― \uventry{Izraelido} Israelit | дитя, потомокъ; напр. \uventry{bovo} быкъ ― \uventry{bovido} теленокъ; \uventry{Izraelo} Израиль ― \uventry{Izraelido} Израильтянинъ | dziecię, potomek; np. \uventry{bovo} byk ― \uventry{bovido} cielę; \uventry{Izraelo} Izrael ― \uventry{Izraelido} Izraelita.

\uventry{hebreo} juif | Jew | Jude | еврей | żyd.

\uventry{ĉevalo} cheval | horse | Pferd | конь | koń.

\end{ekzvocab}


\ekzsec{§37.}

La ŝipanoj devas obei la ŝipestron. ― Ĉiuj loĝantoj de regno estas regnanoj. ― Urbanoj estas ordinare pli ruzaj, ol vilaĝanoj. ― La regnestro de nia lando estas bona kaj saĝa reĝo. ― La Parizanoj estas gajaj homoj. ― Nia provincestro estas severa, sed justa. ― Nia urbo havas bonajn policanojn, sed ne sufiĉe energian policestron. ― Luteranoj kaj Kalvinanoj estas kristanoj. ― Germanoj kaj francoj, kiuj loĝas en Rusujo, estas Rusujanoj, kvankam ili ne estas rusoj. ― Li estas nelerta kaj naiva provincano. ― La loĝantoj de unu regno estas samregnanoj, la loĝantoj de unu urbo estas samurbanoj, la konfesantoj de unu religio estas samreligianoj. ― Nia regimentestro estas por siaj soldatoj kiel bona patro. ― La botisto faras botojn kaj ŝuojn. ― La lignisto vendas lignon, kaj la lignaĵisto faras tablojn, seĝojn kaj aliajn lignajn objektojn. - Ŝteliston neniu lasas en sian domon. ― La kuraĝa maristo dronis en la maro. ― Verkisto verkas librojn, kaj skribisto simple transskribas paperojn. ― Ni havas diversajn servantojn: kuiriston, ĉambristinon, infanistinon kaj veturigiston. ― La riĉulo havas multe da mono. ― Malsaĝulon ĉiu batas. ― Timulo timas eĉ sian propran ombron. ― Li estas mensogisto kaj malnoblulo. ― Preĝu al la Sankta Virgulino.

\begin{ekzvocab}{1em}
\uventry{an} membre, habitant, partisan; ex. \uventry{regno} l’état ― \uventry{regnano} citoyen | inhabitant, member; e.~g. \uventry{Nov-Jorko} New York ― \uventry{Nov-Jorkano} New Yorker | Mitglied, Einwohner, Anhänger; z.~B. \uventry{regno} Staat ― \uventry{regnano} Bürger; \uventry{Varsoviano} Warschauer | членъ, житель, приверженец; напр. \uventry{regno} государство ― \uventry{regnano} гражданинъ; \uventry{Varsoviano} Варшавянинъ | członek, mieszkaniec, zwolennik; np. \uventry{regno} państwo ― \uventry{regnano} obywatel; \uventry{Varsoviano} Warszawianin.

\uventry{regno} l’Etat | kingdom | Staat | государство | państwo.

\uventry{vilaĝano} paysan | caountryman | Bauer | крестьянинъ | wieśniak.

\uventry{provinco} province | province | Provinz | область, провинція | prowincya.

\uventry{severa} sévère | severe | streng | строгій | surowy, srogi, ostry.

\uventry{justo} juste | just, righteous | gerecht | справедливый | sprawiedliwy.

\uventry{polico} police | police | Polizei | полиція | policya.

\uventry{sufiĉe} suffisant | enough | genug | довольно, достаточно | dosyć, dostatecznie.

\uventry{Kristo} Christ | Christ | Christus | Христосъ | Chrystus.

\uventry{Franco} Français | Frenchman | Franzose | Французъ | Francuz.

\uventry{konfesi} avouer | confess | bekennen, gestehen | признавать, исповѣдывать | przyznawać.

\uventry{religio} religion | religion | Religion | вѣра, религія | religia.

\uventry{regimento} regiment | regiment | Regiment | полкъ | półk.

\uventry{boto} botte | boot | Stiefel | сапогъ | but.

\uventry{ŝuo} soulier | shoe | Schuh | башмакъ | trzewik.

\uventry{lasi} laisser, abandonner | leave, let alone | lassen | пускать, оставлять | puszczać, zostawiać.

\uventry{droni} se noyer | drown | ertrinken | тонуть | tonąć.

\uventry{verki} composer, faire des ouvrages (littér.) | work (literary) | verfassen | сочинять | tworzyć, pisać.

\uventry{ul} qui est caractérisé par telle ou telle qualité; ex. \uventry{bela} beau ― \uventry{belulo} bel homme | person noted for\ldots{}; e.~g. \uventry{avara} covetous ― \uventry{avarulo} miser, covetous person | Person, die sich durch\ldots{} unterscheidet; z.~B. \uventry{juna} jung ― \uventry{junulo} Jüngling | особа, отличающаяся даннымъ качествомъ; напр. \uventry{bela} красивый ― \uventry{belulo} красавецъ | człowiek, posiadający dany przymiot; np. \uventry{riĉa} bogaty ― \uventry{riĉulo} bogacz.

\uventry{eĉ} même, jusqu’à | even | sogar | даже | nawet.

\uventry{ombro} ombre | shadow | Schatten | тѣнь | cień.

\uventry{preĝi} prier (Dieu) | pray | beten | молиться | modlić się.

\uventry{virga} virginal | virginal | jungfräulich | дѣвственный | dziewiczy.

\end{ekzvocab}


\ekzsec{§38.}

Mi aĉetis por la infanoj tableton kaj kelke da seĝetoj. ― En nia lando sin ne trovas montoj, sed nur montetoj. ― Tuj post la hejto la forno estis varmega, post unu horo ĝi estis jam nur varma, post du horoj ĝi estis nur iom varmeta, kaj post tri horoj ĝi estis jam tute malvarma. ― En somero ni trovas malvarmeton en densaj arbaroj. ― Li sidas apud la tablo kaj dormetas. ― Mallarĝa vojeto kondukas tra tiu ĉi kampo al nia domo. ― Sur lia vizaĝo mi vidis ĝojan rideton. ― Kun bruo oni malfermis la pordegon, kaj la kaleŝo enveturis en la korton. Tio ĉi estis jam ne simpla pluvo, sed pluvego. ― Grandega hundo metis sur min sian antaŭan piedegon, kaj mi de teruro ne sciis, kion fari, ― Antaŭ nia militistaro staris granda serio da pafilegoj. ― Johanon, Nikolaon, Erneston, Vilhelmon, Marion, Klaron kaj Sofion iliaj gepatroj nomas Johanĉjo (aŭ Joĉjo), Nikolĉjo (aŭ Nikoĉjo aŭ Nikĉjo aŭ Niĉjo), Erneĉjo (aŭ Erĉjo), Vilhelĉjo (aŭ Vilheĉjo aŭ Vilĉjo aŭ Viĉjo), Manjo (aŭ Marinjo), Klanjo kaj Sonjo (aŭ Sofinjo).

\begin{ekzvocab}{1em}
\uventry{densa} épais, dense | dense | dicht | густой | gęsty.

\uventry{brui} faire du bruit | make a noise | lärmen, brausen | шумѣть | szumieć, hałasować.

\uventry{kaleŝo} carosse, calèche | carriage | Wagen | коляска | powóz.

\uventry{pluvo} pluie | rain | Regen | дождь | deszcz.

\uventry{pafi} tirer, faire feu | shoot | schiessen | стрѣлять | strzelać.
\vspace{2pt}

\begin{minipage}{\textwidth}
\setlength{\leftskip}{1em}
\setlength{\parindent}{0em}
\begin{wrapfigure}[2]{l}[1.65em]{1.5em}
\vspace{-3.6ex}
\begin{tabu} to 1.5em{l@{ }l}
\footnotesize \uventry{cj} & \footnotesize \bf \trbb \\
\footnotesize \uventry{nj} & \\
\end{tabu}%
\end{wrapfigure}
après les 1-6 premières lettres d’un prénom masculin (\uventry{nj} ― féminin) lui donne un caractère diminutif et caressant | affectionate diminutive of masculine (\uventry{nj} ― feminine) names | den ersten 1-6 Buchstaben eines männlichen (\uventry{nj} ― weiblichen) Eigennamens beigefügt, verwandelt diesen in ein Kosewort | приставленное къ первымъ 1-6 буквамъ имени собственнаго мужескаго (\uventry{nj} ― женскаго) пола, превращаетъ его въ ласкательное | dodane do pierwszych 1-6 liter imienia własnego męskiego (\uventry{nj} ― żenńskiego) rodzaju zmienia je w pieszczotliwe.
\end{minipage}

\end{ekzvocab}

\ekzsec{§39.}

En la kota vetero mia vesto forte malpuriĝis; tial mi prenis broson kaj purigis la veston. ― Li paliĝis de timo kaj poste li ruĝiĝis de honto. ― Li fianĉiĝis kun fraŭlino Berto; post tri monatoj estos la edziĝo; la edziĝa soleno estos en la nova preĝejo, kaj la edziĝa festo estos en la domo de liaj estontaj bogepatroj. ― Tiu ĉi maljunulo tute malsaĝiĝis kaj infaniĝis. ― Post infekta malsano oni ofte bruligas la vestojn de la malsanulo. ― Forigu vian fraton, ĉar li malhelpas al ni. ― Ŝi edziniĝis kun sia kuzo, kvankam ŝiaj gepatroj volis ŝin edzinigi kun alia persono. ― En la printempo la glacio kaj la neĝo fluidiĝas. ― Venigu la kuraciston, ĉar mi estas malsana. ― Li venigis al si el Berlino multajn librojn. ― Mia onklo ne mortis per natura morto, sed li tamen ne mortigis sin mem kaj ankaŭ estis mortigita de neniu; unu tagon, promenante apud la reloj de fervojo, li falis sub la radojn de veturanta vagonaro kaj mortiĝis. ― Mi ne pendigis mian ĉapon sur tiu ĉi arbeto; sed la vento forblovis de mia kapo la ĉapon, kaj ĝi, flugante, pendiĝis sur la branĉoj de la arbeto. ― Sidigu vin (aŭ sidiĝu), sinjoro! ― La junulo aliĝis al nia militistaro kaj kuraĝe batalis kune kun ni kontraŭ niaj malamikoj.

\begin{ekzvocab}{1em}
\uventry{koto} boue | dirt | Koth, Schmutz | грязь | błoto.

\uventry{broso} brosse | brush | Bürste | щетка | szczotka.

\uventry{ruĝa} rouge | red | roth | красный | czerwony.

\uventry{honti} avoir honte | be ashamed | sich schämen | стыдиться | wstydzić się.

\uventry{solena} solennel | solemn | feierlich | торжественный | uroczysty.

\uventry{infekti} infecter | infect | anstecken | заражать | zarażać.

\uventry{printempo} printemps | spring time | Frühling | весна | wiosna.

\uventry{relo} rail | rail | Schiene | рельса | szyna.

\uventry{rado} roue | wheel | Rad | колесо | koło (od woza i t. p.).

\uventry{pendi} pendre, être suspendu | hang | hängen (v. n.) | висѣть | wisieć.

\uventry{ĉapo} bonnet | bonnet | Mütze | шапка | czapka.

\uventry{vento} vent | wind | Wind | вѣтеръ | wiatr.

\uventry{blovi} souffler | blow | blasen, wehen | дуть | dąć, dmuchać.

\uventry{kapo} tête | head | Kopf | голова | głowa.

\uventry{branĉo} branche | branch | Zweig | вѣтвь | gałaź.

\end{ekzvocab}


\ekzsec{§40.}

En la daŭro de kelke da minutoj mi aŭdis du pafojn. ― La pafado daŭris tre longe. ― Mi eksaltis de surprizo. ― Mi saltas tre lerte. ― Mi saltadis la tutan tagon de loko al loko. ― Lia hieraŭa parolo estis tre bela, sed la tro multa parolado lacigas lin. ― Kiam vi ekparolis, ni atendis aŭdi ion novan, sed baldaŭ ni vidis, ke ni trompiĝis, ― Li kantas tre belan kanton. ― La kantado estas agrabla okupo. ― La diamanto havas belan brilon. ― Du ekbriloj de fulmo trakuris tra la malluma ĉielo. ― La domo, en kiu oni lernas, estas lernejo, kaj la domo, en kiu oni preĝas, estas preĝejo. ― La kuiristo sidas en la kuirejo. ― La kuracisto konsilis al mi iri en ŝvitbanejon. ― Magazeno, en kiu oni vendas cigarojn, aŭ ĉambro, en kiu oni tenas cigarojn, estas cigarejo; skatoleto aŭ alia objekto, en kiu oni tenas cigarojn, estas cigarujo; tubeto, en kiun oni metas cigaron, kiam oni ĝin fumas, estas cigaringo. ― Skatolo, en kiu oni tenas plumojn, estas plumujo, kaj bastoneto, sur kiu oni tenas plumon por skribado, estas plumingo. ― En la kandelingo sidis brulanta kandelo. ― En la poŝo de mia pantalono mi portas monujon, kaj en la poŝo de mia surtuto mi portas paperujon; pli grandan paperujon mi portas sub la brako. ― La rusoj loĝas en Rusujo kaj la germanoj en Germanujo.

\begin{ekzvocab}{1em}
\uventry{surprizi} surprendre | surprise | überraschen | дѣлать сюрпризъ | robić niespodzianki.

\uventry{laca} las, fatigué | weary | müde | усталый | zmęczony.

\uventry{trompi} tromper, duper | deceive, cheat | betrügen | обманывать | oszukiwać.

\uventry{fulmo} éclair | lightning | Blitz | молнія | błyskawica.

\uventry{ŝviti} suer | perspire | schwitzen | потѣть | pocić się.

\uventry{bani} baigner | bath | baden | купать | kąpać.

\uventry{magazeno} magazin\eraro{magasin} | store | Kaufladen | лавка, магазинъ | sklep, magazyn.

\uventry{vendi} vendre | sell | verkaufen | продавать | sprzedawać.

\uventry{cigaro} cigare | cigar | Cigarre | сигара | cygaro.

\uventry{tubo} tuyau | tube | Röhre | труба | rura.

\uventry{fumo} fumée | smoke | Rauch | дымъ | dym.

\uventry{ing} marque l’objet dans lequel se met, ou mieux s’introduit\ldots{}; ex. \uventry{kandelo} chandelle ― \uventry{kandelingo} chandelier | holder for; e.~g. \uventry{kandelo} candle ― \uventry{kandelingo} candle-stick | Gegenstand, in den etwas eingestellt, eingesetzt wird; z.~B. \uventry{kandelo} Kerze ― \uventry{kandelingo} Leuchter | вещь, въ которую вставляется, всаживается; напр. \uventry{kandelo} свѣча ― \uventry{kandelingo} подсвѣчникъ | przedmiot, w który się coś wsadza, wstawia; np. \uventry{kandelo} świeca ― \uventry{kandelingo} lichtarz.

\uventry{skatolo} boîte | small box, case | Büchse, Schachtel | коробка | pudełko.

\uventry{pantalono} pantalon | pantaloons, trowsers | Hosen | брюки | spodnie.

\uventry{surtuto} redingote | over-coat | Rock | сюртукъ | surdut.

\uventry{brako} bras | arm | Arm | рука, объятія | ramię.

\end{ekzvocab}


\ekzsec{§41.}

Ŝtalo estas fleksebla, sed fero ne estas fleksebla. ― Vitro estas rompebla kaj travidebla. ― Ne ĉiu kreskaĵo estas manĝebla. ― Via parolo estas tute nekomprenebla kaj viaj leteroj estas ĉiam skribitaj tute nelegeble. ― Rakontu al mi vian malfeliĉon, ĉar eble mi povos helpi al vi. ― Li rakontis al mi historion tute nekredeblan.\footnoteE{La fruaj eldonoj de la \emph{Ekzercaro} havis ``ne kredeblan''.} ― Ĉu vi amas vian patron? Kia demando! kompreneble, ke mi lin amas. ― Mi kredeble ne povos veni al vi hodiaŭ, ĉar mi pensas, ke mi mem havos hodiaŭ gastojn. ― Li estas homo ne kredinda. ― Via ago estas tre laŭdinda. ― Tiu ĉi grava tago restos por mi ĉiam memorinda. ― Lia edzino estas tre laborema kaj ŝparema, sed ŝi estas ankaŭ tre babilema kaj kriema. Li estas tre ekkolerema kaj ekscitiĝas ofte ĉe la plej malgranda bagatelo; tamen li estas tre pardonema, li ne portas longe la koleron kaj li tute ne estas venĝema. ― Li estas tre kredema: eĉ la plej nekredeblajn aferojn, kiujn rakontas al li la plej nekredindaj homoj, li tuj kredas. ― Centimo, pfenigo kaj kopeko estas moneroj. ― Sablero enfalis en mian okulon. ― Li estas tre purema, kaj eĉ unu polveron vi ne trovos sur lia vesto. ― Unu fajrero estas sufiĉa, por eksplodigi pulvon.

\begin{ekzvocab}{1em}
\uventry{ŝtalo} acier | steet | Stahl | сталь | stal.

\uventry{fleksi} fléchir, ployer | bend | biegen | гнуть | giąć.

\uventry{vitro} verre (matière) | glass (substance) | Glas | стекло | szkło.

\uventry{rompi} rompre, casser | break | brechen | ломать | łamać.

\uventry{laŭdi} louer, vanter | praise | loben | хвалить | chwalić.

\uventry{memori} se souvenir, se rappeler | remember | im Gedächtniss behalten, sich erinnern | помнить | pamiętać.

\uventry{ŝpari} ménager, épargner | be sparing | sparen | сберегать | oszczędzać.

\uventry{bagatelo} bagatelle | trifle, toy | Kleinigkeit | мелочь, бездѣлица | drobnostka.

\uventry{venĝi} se venger | revenge | rächen | мстить | mścić się.

\uventry{eksciti} exciter, émouvoir | excite | erregen | возбуждать | wzbudzać.

\uventry{er} marque l’unité; ex. \uventry{sablo} sable ― \uventry{sablero} un grain de sable | one of many objects of the same kind; e.~g. \uventry{sablo} sand ― \uventry{sablero} grain of sand | ein einziges; z.~B. \uventry{sablo} Sand ― \uventry{sablero} Sandkörnchen | отдѣльная единица; напр. \uventry{sablo} песокъ ― \uventry{sablero} песчинка | oddzielna jednostka; np. \uventry{sablo} piasek ― \uventry{sablero} ziarnko piasku.

\uventry{polvo} poussière | dust | Staub | пыль | kurz.

\uventry{fajro} feu | fire | Feuer | огонь | ogień.

\uventry{eksplodi} faire explosion | explode | explodiren | взрывать | wybuchać.

\uventry{pulvo} poudre à tirer | gunpowder | Pulver (Schiess-) | порохъ | proch.

\end{ekzvocab}


\ekzsec{§42.}

Ni ĉiuj kunvenis, por priparoli tre gravan aferon; sed ni ne povis atingi ian rezultaton, kaj ni disiris. ― Malfeliĉo ofte kunigas la homojn, kaj feliĉo ofte disigas ilin. ― Mi disŝiris la leteron kaj disĵetis ĝiajn pecetojn en ĉiujn angulojn de la ĉambro. ― Li donis al mi monon, sed mi ĝin tuj redonis al li. ― Mi foriras, sed atendu min, ĉar mi baldaŭ revenos. ― La suno rebrilas en la klara akvo de la rivero. ― Mi diris al la reĝo: via reĝa moŝto, pardonu min! ― El la tri leteroj unu estis adresita: al Lia Episkopa Moŝto, Sinjoro N.; la dua: al Lia Grafa Moŝto, Sinjoro P.; la tria: al Lia Moŝto, Sinjoro D. ― La sufikso «um» ne havas difinitan signifon, kaj tial la (tre malmultajn) vortojn kun «um» oni devas lerni, kiel simplajn vortojn. Ekzemple: plenumi, kolumo, manumo. ― Mi volonte plenumis lian deziron. ― En malbona vetero oni povas facile malvarmumi. ― Sano, sana, sane, sani, sanu, saniga, saneco, sanilo, sanigi, saniĝi, sanejo, sanisto, sanulo, malsano, malsana, malsane, malsani, malsanulo, malsaniga, malsaniĝi, malsaneta, malsanema, malsanulejo, malsanulisto, malsanero, malsaneraro, sanigebla, sanigisto, sanigilo, resanigi, resaniĝanto, sanigilejo, sanigejo, malsanemulo, sanilaro, malsanaro, malsanulido, nesana, malsanado, sanulaĵo, malsaneco, malsanemeco, saniginda, sanilujo, sanigilujo, remalsano, remalsaniĝo, malsanulino, sanigista, sanigilista, sanilista, malsanulista k.~t.~p.

\begin{ekzvocab}{1em}
\uventry{atingi} atteindre | attain, reach | erlangen, erreichen | достигать | dosięgać.

\uventry{rezultato} résultat | result | Ergebniss | результатъ | rezultat.

\uventry{ŝiri} déchirer | tear, rend | reissen | рвать | rwać.

\uventry{peco} morceau | piece | Stück | кусокъ | kawał.

\uventry{moŝto} titre commun | universal title | allgemeiner Titel | общій титулъ | Mość.

\begin{minipage}{\textwidth}
\uventry{episkopo} évêque | bishop | Bischof | епископъ | biskup.

\uventry{grafo} comte | earl, count | Graf | графъ | hrabia.

\uventry{difini} définir, déterminer | define | bestimmen | опредѣлять | wyznaczać, określać.
\end{minipage}
\end{ekzvocab}



