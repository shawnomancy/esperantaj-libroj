% Fundamento de Esperanto
%
% L. L. ZAMENHOF, 1905
%
% XeLaTeX-a eldono de Shawn C. KNIGHT, 2019
%
% This work is licensed under a Creative Commons 
% Attribution-NonCommercial-ShareAlike 4.0 International License.
% The original work by Dr. Zamenhof is in the public domain.
%
% Ĉi tiu verko estas permesita per Creative Commons 
% Attribution-NonCommercial-ShareAlike 4.0 Internacia Permesilo.
% La originala verko de D-ro Zamenhof estas senkopirajta.
%
\documentclass[11pt,twoside]{book}

% load the preamble
%
% Komuna Enkonduko por Esperantaj Libroj
%

%%%%%%%%%%%%%%%%%%%%%%%%%%%%%%%%%%%%%%%%%%%%%%%%%%%%%%%%%%%%

%
% Geometrio
%
\usepackage[a5paper,margin=2cm]{geometry}

%%%%%%%%%%%%%%%%%%%%%%%%%%%%%%%%%%%%%%%%%%%%%%%%%%%%%%%%%%%%

% 
% Citilojn kaj plu
%
\usepackage[english,french,polish,german,russian,esperanto]{babel}  

%%%%%%%%%%%%%%%%%%%%%%%%%%%%%%%%%%%%%%%%%%%%%%%%%%%%%%%%%%%%

%
% Verso
%
\usepackage{verse}

%%%%%%%%%%%%%%%%%%%%%%%%%%%%%%%%%%%%%%%%%%%%%%%%%%%%%%%%%%%%

%
% Tiparoj
%
\usepackage{fontspec}

% el Google Fonts
\setmainfont{Old Standard TT}
\newfontfamily\cowboyfont{Smokum}[LetterSpace=5]
\newfontfamily\arbfont{Arbutus Slab}
\newfontfamily\latin{Amethysta}
\newfontfamily\nicefont{Cardo}
\newfontfamily\fjallafont{FjallaOne}
\newfontfamily\grammarpartsfont{FjallaOne}[LetterSpace=20]
\newfontfamily\tuscan{Sancreek}

% en MacOS
\newfontfamily\didone{Didot}
\newfontfamily\copper{Copperplate Light}[LetterSpace=5]
\newfontfamily\chunk{Rockwell}

% el 1001 Fonts
\newfontfamily\curve{Pinyon Script}

% el dafont kaj Unifraktur Project
\newfontfamily\fr[BoldFont=Fette classic UNZ Fraktur]{UnifrakturMaguntia}

% el Liberation Fonts
\newfontfamily\sansfont{Liberation Sans}[LetterSpace=5]
\newfontfamily\sansfontclose{Liberation Sans}

% en TeXlive
\newfontfamily\csfont{TeX Gyre Schola}
\newfontfamily\bookman{TeX Gyre Bonum}
\newfontfamily\times{TeX Gyre Termes}

% sans-serif bolds in body text
\newcommand\inbold[1]{\scalebox{1}[0.8]{\fjallafont{#1}}}

% Ombroj por ornamoj tiparoj
\usepackage{shadowtext}
\shadowoffset{1pt}

%%%%%%%%%%%%%%%%%%%%%%%%%%%%%%%%%%%%%%%%%%%%%%%%%%%%%%%%%%%%

%
% Pliaj simboloj
%

% Unuoblaj anglaj citiloj
\usepackage{textcomp}

% Creative Commons ikonoj
\usepackage{ccicons}

% Por ornamoj kiel la "por angloj" etikedo; beletaj sekcilineoj
\usepackage{pgfornament} 

% la xelatex-simbolo en la kompostanta komento
\usepackage{dtk-logos}

% la klasikaj montraj fingroj, kiuj ekkrius "19a jarcento" se ili povus
\usepackage{dingbat}
\usepackage{bbding}

%%%%%%%%%%%%%%%%%%%%%%%%%%%%%%%%%%%%%%%%%%%%%%%%%%%%%%%%%%%%

%
% Trovu / anstataŭigu
%
\usepackage{regexpatch}

%%%%%%%%%%%%%%%%%%%%%%%%%%%%%%%%%%%%%%%%%%%%%%%%%%%%%%%%%%%%

%
% Tabloj
%

% granda longa tablo (la "mi ne scias ..." analizo)
%
\usepackage{longtable}

% tabu kaj kolonoj
%
\usepackage{tabu}
\usepackage{array,ragged2e}
\newcolumntype{Y}{>{\centering\arraybackslash}X}
\newcolumntype{+}{>{\global\let\currentrowstyle\relax}}
\newcolumntype{Z}{>{\currentrowstyle}}
\newcommand{\rowstyle}[1]{\gdef\currentrowstyle{#1}%
#1\ignorespaces
}

%%%%%%%%%%%%%%%%%%%%%%%%%%%%%%%%%%%%%%%%%%%%%%%%%%%%%%%%%%%%

%
% Aliaĵoj
%

%
% Substrekoj kaj litera spaco
% 
\usepackage{soul}
\sodef\spaceout{}{.2em}{0.6em}{0pt}
\sodef\spaceoutmed{}{.1em}{0.5em}{0pt}
\sodef\spaceoutless{}{.05em}{0.4em}{0pt}
\newcommand{\narrow}[1]{\scalebox{0.8}[1]{#1}}

% interlinea distanco
\usepackage{setspace} 

% star lines in poems
% 
\newcommand{\pstars}{%
\hspace*{\fill} * \hspace{3em} \raisebox{-1em}{*} \hspace{3em} * \hspace*{\fill}}

% ŝanĝas la grando de la intermorgema signeto por ĝia kunteksto
%
\usepackage{relsize} 

% la krommarĝeno por la unua lineo de ĉiu alineo devus esti kiel la aliaj
%
\usepackage{indentfirst} 

% this will let us make the nice diagonal 1/2 fraction in the prices page
%
\usepackage{units}

% la krommarĝeno por unuobla alineo
%
\usepackage{changepage}

% ifthenelse
\usepackage{ifthen}

% horitontala formato kaj pluraj kolonoj por la vortaro
%
\usepackage{pdflscape}
\usepackage{multicol}



% por la longaj krampoj en la "Mi ne scias" tablo, kaj la vortara
% lineoj de "la" kaj "l'"
%
\usepackage{multirow} 

% titlesec -- better section/chapter headings
%
\usepackage[compact,center]{titlesec}
\titlespacing*{\chapter}{0pt}{6em}{0pt}

% fancy page headers, that is, in imitation of the original
%
\usepackage{fancyhdr}
\setlength{\headheight}{15pt}
\pagestyle{fancy}

\fancyhf{}

% remove the rule at top of page
\renewcommand{\headrulewidth}{0pt}



\newcolumntype{R}[1]{>{\RaggedLeft\arraybackslash}m{#1}}

% this bit removes the page number from the chapter start page
%
\fancypagestyle{plain}{%
  \renewcommand{\headrulewidth}{0pt}%
  \fancyhf{}%
}

% pgfornament line to end a section
%
\newcommand{\sectionline}{
\begin{center}
\pgfornament[width=0.5\textwidth]{89}
\end{center}}

% la "por angloj" etikedo
%
\sodef\angloj{}{.2em}{0.6em}{0pt}
\newcommand{\poranglojbox}{%
\begin{tikzpicture}[every node/.style={inner sep=0pt}]
\node[align=center](Text){\scalebox{1}[1.2]{\didone{\bf\angloj{por Angloj}}}} ;
\node[shift={(5pt,2pt)},anchor=center](CNW) at (Text.north east) {};
\node[shift={(-5pt,2pt)},anchor=center](CNE) at (Text.north west) {};
\node[shift={(-5pt,-2pt)},anchor=center](CSW) at (Text.south west) {};
\node[shift={(5pt,-2pt)},anchor=center](CSE) at (Text.south east) {};
\draw(CNW) to (CNE);
\draw(CSW) to (CSE);
\pgftransformshift{\pgfpoint{-1cm}{-0.537cm}}
\pgftransformscale{0.05};
\pgftransformrotate{90};
\anglojcurlicue{};
\pgftransformreset;
\pgftransformshift{\pgfpoint{1cm}{0.537cm}}
\pgftransformscale{0.05};
\pgftransformrotate{-90};
\anglojcurlicue{};
\end{tikzpicture}}

% the command for the morpheme-separation stroke
%
\renewcommand{\,}{%
{\relsize{-2.5}\protect\raisebox{-1.35ex}{$'$}}}

% kapliteroj por la Vortaro
% 
\newcommand{\vorhead}[1]{\vspace{1.5ex}
{\hfil \scalebox{1.5}[1]{\sansfont{#1}}}
\vspace{1.5ex}}

\newenvironment{outdent}[1]
  {\setlength{\leftskip}{#1}%%
   \setlength{\parindent}{-#1}%%
  }
  {\par}

% Big delimiters in the demo sentence table
%
\newcommand{\tlba}{\scalebox{1}[1.25]\{}
\newcommand{\trba}{\scalebox{1}[1.25]\}}
\newcommand{\tlbb}{\multirow{2}{*}{\scalebox{1}[2.5]\{}}
\newcommand{\trbb}{\multirow{2}{*}{\scalebox{1}[2.5]\}}}
\newcommand{\tlbc}{\multirow{3}{*}{\scalebox{1}[4]\{}}
\newcommand{\trbc}{\multirow{3}{*}{\scalebox{1}[4]\}}}

% Heading for the colophon
%
\newcommand\kolofono{
\fancyhead[C]{}
\titleformat{\chapter}[display]{\centering\sansfont}{\chaptertitlename}{0pt}{\Large}
\chapter*{KOMENTO PRI KOMPOSTADO.}

\begin{center}
\rule[0.5ex]{0.5\textwidth}{0.4pt}

\emph{Jen versio \laversio{} de ĉi tiu} \XeLaTeX{} \emph{versio.}
\end{center}
}

% remove rule at bottom of page
%
\renewcommand{\footrulewidth}{0pt}

%
% Dr. Zamenhof's portrait
%
\newcommand\zamenhof{
\thispagestyle{empty}
\begin{center}
 \vspace*{\stretch{1}}
\begin{figure}[!ht]
\centering
\includegraphics[scale=0.15]{../graphics/Zamenhof}
\end{figure}
\vspace*{0.5cm}
\nicefont
{\LARGE Lazaro Ludoviko ZAMENHOF} \\[1ex]
{\large Aŭtoro de la lingvo «Esperanto»} \\[1ex]
{\small (la 15-a de Decembro 1859 — la 14-a de Aprilo 1917)}
 \vspace*{\stretch{1}}
\end{center}
\newpage
}

% Ligiloj, kaj ilia koloro 
%
\usepackage{color}
\definecolor{verda_ligilo}{rgb}{0,0.5,0}
\usepackage[colorlinks,linkcolor=verda_ligilo]{hyperref}
\usepackage{bookmark}

% lengths to save
%
\newlength{\savedparskip}
\newlength{\savedparindent}
\setlength{\savedparskip}{\parskip}
\setlength{\savedparindent}{\parindent}

%
% Enkonduko por la Fundamento
%

%%%%%%%%%%%%%%%%%%%%%%%%%%%%%%%%%%%%%%%%%%%%%%%%%%%%%%%%%%%%

%
% Tiparoj
%

% de Google Fonts
%
\newfontfamily\brevefont[FakeSlant=1,FakeBold=0.5]{Old Standard TT}
\newfontfamily\stdfont{Old Standard TT}

\defaultfontfeatures{Scale=MatchLowercase}

%%%%%%%%%%%%%%%%%%%%%%%%%%%%%%%%%%%%%%%%%%%%%%%%%%%%%%%%%%%%

%
% Editorial footnoting
%
%\usepackage[perpage]{footmisc}
\usepackage[perpage]{manyfoot}
\DeclareNewFootnote{E}[fnsymbol]
\newcommand\eraro[1]{\footnoteE{Ŝajnerare anstataŭ \emph{#1.}}}
\newcommand\unuakomon{\footnoteE{%
La unua eldono de la \emph{Ekzercaro} havis komon ĉi tie.}}

%%%%%%%%%%%%%%%%%%%%%%%%%%%%%%%%%%%%%%%%%%%%%%%%%%%%%%%%%%%%

%
% Speciale por la germana gramatiko
%

% Fraktur bold
%
\newcommand{\frb}[1]{%
{\relsize{0}{\textbf{#1}}}}

% Switch back to antiqua in the middle of fraktur for E-o examples
%
\newcommand{\std}[1]{%
{\relsize{0}{\stdfont{#1}}}}

%%%%%%%%%%%%%%%%%%%%%%%%%%%%%%%%%%%%%%%%%%%%%%%%%%%%%%%%%%%%

%
% Speciale por la rusa gramatiko
%

% Esperanto text and morpheme dividers in the Russian grammar
\newcommand{\es}[1]{{{\textbf{#1}}}}
\renewcommand{\|}{$|$}

% Stretch bold.  Use for Russian bold words, and for titles
% in Latin-alphabet grammar sections.
\newcommand{\br}[1]{\scalebox{1.2}[1]{\textbf{#1}}}

%%%%%%%%%%%%%%%%%%%%%%%%%%%%%%%%%%%%%%%%%%%%%%%%%%%%%%%%%%%%

%
% Speciale por la Ekzercaro
%

% Ekzercaro section headings
%
\newcommand{\myTempSubsect}{\ldots}
\newcommand{\ekzsec}[2][]{%
\begin{center}
\phantomsection
\narrow{\large #2}
{\ifthenelse{\equal{#1}{}}{}{\vskip 1ex
{#1}}}
\end{center}
{\ifthenelse{\equal{#1}{}}{%
\addcontentsline{toc}{section}{#2}}{%
\renewcommand{\myTempSubsect}{#1}
\xpatchcmd*{\myTempSubsect}{1ex}{}{}{}
\addcontentsline{toc}{section}{#2 \myTempSubsect}}
}}

% outdented paragraphs in ekzercaro
%
\newenvironment{ekzvocab}[1]
  {\setlength{\leftskip}{#1}%%
   \setlength{\parindent}{-#1}%%
   \vskip 1ex
     \footnotesize
  }
  {\par}

%
% Handcrafted accents for the script letters in the Ekzercaro.
%

% dotless j -- need to remove the dot before we drop the circumflex on it!
%
\usepackage{stackengine}
\newcommand{\dotlessj}{ȷ}

% breve -- this font lacks a breve to use.  We defined a breve font
% above using FakeSlant to give it the desired slant to match our
% script U/u.
%
\newcommand{\scriptbreve}{\Large\brevefont{\u{}}}

% Now the letters.  If you use a different script font, you'll probably
% need to play with all these numbers. Or maybe you'll have one that
% just has the accents built in already!
%
\newcommand{\scriptC}{\stackinset{c}{2ex}{t}{-1ex}{ˆ}{C}}
\newcommand{\scriptc}{\stackinset{c}{0.8ex}{b}{0.1ex}{ˆ}{c}}
\newcommand{\scriptG}{\stackinset{c}{2.1ex}{t}{-1.1ex}{ˆ}{G}}
\newcommand{\scriptg}{\stackinset{c}{0.9ex}{b}{2.1ex}{ˆ}{g}}
\newcommand{\scriptH}{\stackinset{c}{2.2ex}{t}{-1.1ex}{ˆ}{H}}
\newcommand{\scripth}{\stackinset{c}{2ex}{t}{-1.1ex}{ˆ}{h}}
\newcommand{\scriptJ}{\stackinset{c}{2.4ex}{t}{-1.2ex}{ˆ}{J}}
\newcommand{\scriptj}{\stackinset{c}{0.9ex}{b}{2.1ex}{ˆ}{\dotlessj}}
\newcommand{\scriptS}{\stackinset{c}{2.5ex}{t}{-1.3ex}{ˆ}{S}}
\newcommand{\scripts}{\stackinset{c}{0.9ex}{b}{0.5ex}{ˆ}{s}}
\newcommand{\scriptU}{\stackinset{c}{-0.3ex}{t}{-0.7ex}{\scriptbreve}{U}}
\newcommand{\scriptu}{\stackinset{l}{-1ex}{b}{0ex}{\scriptbreve}{u}}

%%%%%%%%%%%%%%%%%%%%%%%%%%%%%%%%%%%%%%%%%%%%%%%%%%%%%%%%%%%%

%
% Speciale por la Universala Vortaro
%

% Vortaro headletters
\newcommand{\uvlitero}[1]{%
\phantomsection
{\centering\vskip 1ex\LARGE\bookman{#1}\vskip 1ex\par}
\addcontentsline{toc}{section}{#1}
}

% Vortaro entries
\newcommand{\uventry}[1]{{\csfont{\textbf{#1}}}}

% Vortaro subentries
\newcommand{\uvsubentry}{\hspace{1em}}

%%%%%%%%%%%%%%%%%%%%%%%%%%%%%%%%%%%%%%%%%%%%%%%%%%%%%%%%%%%%

% fancy page headers, that is, in imitation of the original
%
\renewcommand\headrule{}
\renewcommand\footrule{}

% fake small caps
%
\makeatletter
\newlength\fake@f
\newlength\fake@c
\def\fakesc#1{%
  \begingroup%
  \xdef\fake@name{\csname\curr@fontshape/\f@size\endcsname}%
  \fontsize{\fontdimen8\fake@name}{\baselineskip}\selectfont%
  \uppercase{#1}%
  \endgroup%
}
\makeatother
\newcommand\fauxsc[1]{\fauxschelper#1 \relax\relax}
\def\fauxschelper#1 #2\relax{%
  \fauxschelphelp#1\relax\relax%
  \if\relax#2\relax\else\ \fauxschelper#2\relax\fi%
}
\def\Hscale{.83}\def\Vscale{.72}\def\Cscale{1.00}
\def\fauxschelphelp#1#2\relax{%
  \ifnum`#1>``\ifnum`#1<`\{\scalebox{\Hscale}[\Vscale]{\uppercase{#1}}\else%
    \scalebox{\Cscale}[1]{#1}\fi\else\scalebox{\Cscale}[1]{#1}\fi%
  \ifx\relax#2\relax\else\fauxschelphelp#2\relax\fi}


% Small rules
\newcommand{\smallrule}{%
\begin{center}%
\rule{13mm}{0.4pt}\end{center}}

\newcommand{\flatsmallrule}{{\centering\rule[0.9ex]{13mm}{0.4pt}\par}}

% Grammar subsection headings (for the three that use 
% Antiqua letters; also in ekzercaro)
%
\newcommand{\gramsec}[1]{{\chunk{\br{#1}}}}

% wrapfig for the cj/nj definition
%
\usepackage{wrapfig}

% titlesec -- better section/chapter headings
%
% memoru:
% titleformat{command}[shape]{format}{label}{sep}{before-code}[after-code]
%

\titleclass{\chapter}{straight}
\renewcommand{\thesection}{\arabic{section}}
\titleformat{\section}[display]{\centering\large\bf}{}{0pt}{\phantomsection\scalebox{2}[1]}
\titlespacing{\section}{0pt}{1em}{1em}

\titleformat{\chapter}[display]{\centering}{}{0pt}{\Large\bookman}
\titlespacing{\chapter}{0pt}{1em}{1em}

\newcommand\mychap[1]{%
\phantomsection
\chapter*{#1}
\addcontentsline{toc}{chapter}{#1}
}


\newcommand\mychapbig[1]{%
\phantomsection
\chapter*{\protect\scalebox{2}[1]{#1}}
\addcontentsline{toc}{chapter}{#1}
}

\newcommand\mysect[1]{%
\section*{#1}
\addcontentsline{toc}{section}{#1}
}

%
% End of preamble
%


% Version number 
%
\newcommand{\laversio}{0.9}

%
% End preamble and begin title page
% 
\begin{document}
\frontmatter
\sloppy

%

\begin{titlepage}
\vspace*{\fill}
\parbox{0.9\textwidth}{\nicefont{\fontsize{42}{48}\selectfont Fundamento\vskip 0.7ex\hfill de Esperanto}}
\vspace*{\fill}
\end{titlepage}

%
% adverts page
%
%
% adverts page
%
\thispagestyle{plain}
{\centering
{\footnotesize \so{LIBREJO HACHETTE KAJ K\textsuperscript{o}, PARIS}\par}
\rule[0.9ex]{\textwidth}{0.4pt}

{\timesfont{\Large \spaceoutmed{TUTMONDA}}}\\[1ex]
\parbox{0.7\textwidth}{\timesfont{\Huge \spaceoutmed{Jarlibro}\vskip 1ex\hfill \spaceoutmed{Esperantista}}}\\[1ex]
{\footnotesize\fauxsc{enhavanta la}}\par

{\csfont{\textbf{\spaceoutless{Adresaron de D\textsuperscript{ro} ZAMENHOF}}}\par}
}

{\footnotesize Tiu ĉi jarlibro eldoniĝas ĉiujare dum Marto.  Ĝi entenas : pli ol 20,000 adresojn de Esperantistoj de l’ tuta mondo, plenajn sciigojn pri la propagandaj Societoj, pri la libroj, la gazetoj, la ĉiuspecaj eldonaĵoj en Esperanto.}

\newlength{\tempindent}
\setlength{\tempindent}{\parindent}
\setlength{\parindent}{0pt}
{\footnotesize Unu volumo, in-16, broŝurita \dotfill 2 fr. 50}
\setlength{\parindent}{\tempindent}

{\centering
\rule[0.9ex]{\textwidth}{0.4pt}
{\timesfont{\Large Internacia}}

\vspace{0.5em}
{\timesfont{\LARGE \spaceoutmed{Scienca Revuo}}}
\vspace{0.5em}

{\small \bf Monata scienca revuo redaktita} \\
{\latin\scalebox{1.5}[1]{\spaceoutmed{EN ESPERANTO}}} \\
{Eldonata de la 1\textsuperscript{a} de Januaro 1904}

\rule[0.9ex]{13mm}{0.4pt}

{\fauxsc{patronaro} :\par}
}

{\footnotesize Franca Societo de Fiziko, Internacia Societo de Elektristoj, S\textsuperscript{oj} Adelsköld, Appel, d’Arsonval, Baudoin de Courtenay, Becquerel, Berthelot, Bouchard, Brouardel, Deslandres, G\textsuperscript{al} Sébert, anoj de diversaj akademioj.}

{\centering
\begin{tblr}{colspec={YY},stretch=0}
\footnotesize\fauxsc{redakcio} : & \footnotesize\fauxsc{administracio} : \\
\fjallafont{\spaceoutmed{P. FRUICTIER}} & \fjallafont{\spaceoutmed{HACHETTE \& K\textsuperscript{\nicefont\bf o}}} \\
\footnotesize 27, {\it boulevard Arago,} & \footnotesize 79, {\it boulevard Saint-Germain,} \\
\footnotesize\it PARIS & \footnotesize\it PARIS
\end{tblr}

\rule{13mm}{0.4pt}

{\footnotesize\fauxsc{jara abono}\par}

Francujo \Dotfill 6 fr. 50 | Ceteraj landoj \Dotfill 7 fr.

\footnotesize\fauxsc{unu numero} : 60 centimoj\par

\rule{\textwidth}{0.4pt}
}
\begin{tblr}{X@{}r}
\footnotesize D\textsuperscript{ro} \fauxsc{Helte.} — {\bf Pri la Teorio de l’Jonoj} \Dotfill & \footnotesize » 30\\
\footnotesize \fauxsc{Mendelejev.} — {\bf Provo de Kemia Kompreno de l’Monda Etero} \Dotfill & \footnotesize » 30
\end{tblr}



%
% fuller title page
%
%
% decorated title page
%
\begin{titlepage}

\begin{center}
\includegraphics[width=0.9\textwidth, angle=0]{../graphics/terglobo.png}\\[1em]

\nicefont{\Large{L.-L. ZAMENHOF}}
\smallrule{}

\parbox{0.9\textwidth}{\nicefont{\fontsize{36}{42}\selectfont Fundamento\vskip 0.7ex\hfill de Esperanto}}

\vspace*{\fill}

{\nicefont\footnotesize GRAMATIKO, EKZERCARO, UNIVERSALA VORTARO}

\vspace*{\fill}

\rule{0.4\textwidth}{0.4pt}

\vspace*{\fill}

\setstretch{1.2}

\csfont \footnotesize

\textsc{francujo.} — HACHETTE et C\textsuperscript{ie}, \textsc{\textit{paris.}}

\rule[0.9ex]{13mm}{0.4pt}

\textsc{anglujo.} – « REVIEW of REVIEWS », \textsc{\textit{london}}

\textsc{danujo.} — ANDR.-FRED. HÖST \& SÖN, \textsc{\textit{kjobenhavn.}}

\textsc{germanujo.} — MÖLLER \& BOREL, \textsc{\textit{berlin.}}

\textsc{hispanujo.} — J. ESPASA, \textsc{\textit{barcelona.}}

\textsc{italujo.} — RAFFAELO GIUSTI, \textsc{\textit{livorno.}}

\textsc{polujo.} — M. ARCT, \textsc{\textit{warszawa.}}

\textsc{svedujo.} — ESPERANTOFÖRENING, \textsc{\textit{stockholm.}}

\rule{5mm}{0.4pt}

\oldstylenums{1905}

\end{center}

\end{titlepage}


%
% Inside title page
%
\renewcommand{\footrulewidth}{0.4pt}
\setstretch{1.1}

\zamenhof

\fancyhead[C]{}
\titleformat{\chapter}[display]{\centering\sansfont}{\chaptertitlename}{0pt}{\Large}
\phantomsection
\chapter*{KOMENTO DE LA KOMPOSTANTO.}
\addcontentsline{toc}{chapter}{Komento de la kompostanto}

\begin{center}
\rule[0.5ex]{0.5\textwidth}{0.4pt}

\emph{Jen versio \laversio{} de ĉi tiu} \XeLaTeX{} \emph{versio.}
\end{center}

En miaj aliaj \XeLaTeX-aj versioj de la ĉefaj libroj de Esperanto (\emph{Unua Libro, Dua Libro, Fundamenta Krestomatio}), mia komento sekvas la reprodukton; sed la specialeco de la \emph{Fundamento} postulas, ke mia komento iru antaŭ la reprodukto, por zorge klarigi la aferon. Jen la plej grava punkto: Ĉi tiu dokumento \textbf{ne estas} la vera \emph{Fundamento de Esperanto} mem!

La \emph{Fundamento} estas netuŝebla; tial la Akademio de Esperanto diras (ĉe ilia retejo) «oni eldonas la Fundamenton normale nur en faksimila (fotokopiita) formo. Rekompostado ĉiam riskus falsi la tekston.» Jes ja, sed ankoraŭ, ĉar mi faris \XeLaTeX-ajn versiojn de la aliaj libroj, imitante la originalan tipografion, mi elektis fari ĉi tiun reprodukton ankaŭ de la \emph{Fundamento}, \textbf{kun la averto supre.}

Piednotoj en la \emph{Fundamenta Gramatiko}, kiuj estas montritaj per numeroj, estas en la originala. Piednotoj en la tuta libro, kiuj estas montritaj per nenumeraj simboloj (ekz. «\**» aŭ «†»), estas el mi.  Mi montras ŝajnajn erarojn en la originala, kaj diversajn lokojn, kie la \emph{Ekzercaro} en la \emph{Fundamento} diferencis el la fruaj eldonoj de la \emph{Ekzercaro} kiel memstaranta libro.  La pli novaj eldonoj de la \emph{Fundamento} montras la aferojn pli klare.

\vspace{1ex}

{\setlength{\parindent}{0em}
Shawn KNIGHT (angle elparolata \emph{ŝan najt})\\
\hodiau}
\titleformat{\chapter}[display]{\centering}{}{0pt}{\Large\bookman}

\cleardoublepage

% Antaŭparolo
\label{antau}
\fancyhead[LE,RO]{\footnotesize\thepage}
\fancyhead[CE]{\footnotesize\narrow{\leftmark}}
\fancyhead[CO]{\footnotesize\narrow{\rightmark}}
\markboth{FUNDAMENTO DE ESPERANTO}{ANTAŬPAROLO}
\thispagestyle{plain}

\vspace*{1.35em}

\begin{center}

\scalebox{0.6}[1]{\Huge\bf FUNDAMENTO DE ESPERANTO}

\vspace{1em}

\rule{0.9\textwidth}{0.4pt}

\vspace{3em}

\Large\bookman{Antaŭparolo}
\end{center}
\vspace{1ex}

Por ke lingvo internacia povu bone kaj regule progresadi kaj por ke ĝi havu plenan certecon, ke ĝi neniam disfalos kaj ia facilanima paŝo de ĝiaj amikoj estontaj ne detruos la laborojn de ĝiaj amikoj estintaj, ― estas plej necesa antaŭ ĉio unu kondiĉo: la ekzistado de klare difinita, neniam tuŝebla kaj neniam ŝanĝebla \textbf{Fundamento} de la lingvo. Kiam nia lingvo estos oficiale akceptita de la \emph{registaroj} de la plej ĉefaj regnoj kaj tiuj ĉi registaroj per speciala \emph{leĝo} garantios al Esperanto tute certan vivon kaj uzatecon kaj plenan sendanĝerecon kontraŭ ĉiuj personaj kapricoj aŭ disputoj, tiam aŭtoritata komitato, interkonsente elektita de tiuj registaroj, havos la rajton fari en la fundamento de la lingvo unu fojon por ĉiam ĉiujn deziritajn ŝanĝojn, \emph{se} tiaj ŝanĝoj montriĝos necesaj; sed \emph{ĝis tiu tempo} la fundamento de Esperanto devas plej severe resti absolute senŝanĝa, ĉar severa netuŝebleco de nia fundamento estas la plej grava kaŭzo de nia ĝisnuna progresado kaj la plej grava kondiĉo por nia regula kaj paca progresado estonta. \emph{Neniu persono kaj neniu societo devas havi la rajton arbitre fari en nia Fundamento iun eĉ plej malgrandan ŝanĝon!} Tiun ĉi tre gravan principon la esperantistoj volu ĉiam bone memori kaj kontraŭ la ektuŝo de tiu ĉi principo ili volu ĉiam energie batali, ĉar la momento, en kiu ni ektuŝus tiun principon, estus la komenco de nia morto.

Laŭ silenta interkonsento de ĉiuj esperantistoj jam de tre longa tempo la sekvantaj tri verkoj estas rigardataj kiel fundamento de Esperanto: 1.) La 16-regula \emph{gramatiko}; 2) la «~\emph{Universala Vortaro}~»; 3) la «~\emph{Ekzercaro}~». Tiujn ĉi tri verkojn la aŭtoro de Esperanto rigardadis ĉiam kiel \emph{leĝojn} por li, kaj malgraŭ oftaj tentoj kaj delogoj li neniam permesis al si (almenaŭ \emph{konscie}) eĉ la plej malgrandan pekon kontraŭ tiuj ĉi leĝoj; li esperas, ke pro la bono de nia afero ankaŭ ĉiuj aliaj esperantistoj ĉiam rigardados tiujn ĉi tri verkojn kiel la solan leĝan kaj netuŝeblan fundamenton de Esperanto.

Por ke ia regno estu forta kaj glora kaj povu sane disvolviĝadi, estas necese, ke ĉiu regnano sciu, ke li neniam dependos de la kapricoj de tiu aŭ alia persono, sed devas obei ĉiam nur klarajn, tute difinitajn fundamentajn \emph{leĝojn} de sia lando, kiuj estas egale devigaj por la regantoj kaj regatoj kaj en kiuj neniu havas la rajton fari arbitre laŭ persona bontrovo ian ŝanĝon aŭ aldonon. Tiel same por ke nia afero bone progresadu, estas necese, ke ĉiu esperantisto havu la plenan certecon, ke leĝodonanto por li ĉiam estos ne ia \emph{persono}, sed ia klare difinita \emph{verko}. Tial, por meti finon al ĉiuj malkompreniĝoj kaj disputoj, kaj por ke ĉiu esperantisto sciu tute klare, per kio li devas en ĉio sin gvidi, la aŭtoro de Esperanto decidis nun eldoni en formo de unu libro tiujn tri verkojn, kiuj laŭ silenta interkonsento de ĉiuj esperantistoj jam de longe fariĝis fundamento por Esperanto, kaj li petas, ke la okuloj de ĉiuj esperantistoj estu ĉiam turnataj ne al li, sed al \emph{tiu ĉi libro.} Ĝis la tempo, kiam ia por ĉiuj aŭtoritata kaj nedisputebla institucio decidos alie, ĉio, kio troviĝas en tiu ĉi libro, devas esti rigardata kiel deviga por ĉiuj; ĉio, kio estas kontraŭ tiu ĉi libro, devas esti rigardata kiel malbona, se ĝi eĉ apartenus al la plumo de la aŭtoro de Esperanto mem. Nur la supre nomitaj tri verkoj publikigitaj en la libro «~Fundamento de Esperanto~», devas esti rigardataj kiel oficialaj; ĉio alia, kion mi verkis aŭ verkos, konsilas, korektas, aprobas k.~t.~p., estas nur verkoj \emph{privataj}, kiujn la esperantistoj ― se ili trovas tion ĉi utila por la unueco de nia afero ― povas rigardadi kiel \emph{modela}, sed ne kiel \emph{deviga}.

Havante la karakteron de \emph{fundamento}, la tri verkoj represitaj en tiu ĉi libro devas antaŭ ĉio esti \emph{netuŝeblaj}. Tial la legantoj ne miru, ke ili trovos en la nacia traduko de diversaj vortoj en tiu ĉi libro (precipe en la angla parto) tute nekorektite tiujn samajn \emph{erarojn}, kiuj sin trovis en la unua eldono de la «~Universala Vortaro~». Mi permesis al mi nur korekti la \emph{preserarojn}; sed se ia vorto estis erare aŭ nelerte \emph{tradukita}, mi ĝin lasis en tiu ĉi libro tute senŝanĝe; ĉar se mi volus plibonigi, tio ĉi jam estus \emph{ŝanĝo}, kiu povus kaŭzi disputojn kaj kiu en verko fundamenta ne povas esti tolerata. \emph{La fundamento devas resti severe netuŝebla eĉ kune kun siaj eraroj.} La \emph{erareco} en la nacia traduko de tiu aŭ alia vorto ne prezentas grandan malfeliĉon, ĉar, komparante la kuntekstan tradukon en la aliaj lingvoj, oni facile trovos la veran sencon de ĉiu vorto; sed senkompare pli grandan danĝeron prezentus la \emph{ŝanĝado} de la traduko de ia vorto, ĉar, perdinte la severan netuŝeblecon, la verko perdus sian eksterordinare necesan karakteron de dogma fundamenteco, kaj, trovante en unu eldono alian tradukon ol en alia, la uzanto ne havus la certecon, ke mi morgaŭ ne faros ian alian ŝanĝon, kaj li perdus sian konfidon kaj apogon. Al ĉiu, kiu montros al mi ian nebonan esprimon en la Fundamenta libro, mi respondos trankvile: jes, ĝi estas eraro, sed ĝi devas resti netuŝebla, ĉar ĝi apartenas al la fundamenta dokumento, en kiu neniu havas la rajton fari ian ŝanĝon.

La «~Fundamento de Esperanto~» tute ne devas esti rigardata kiel la plej bona lernolibro kaj vortaro de Esperanto. Ho, ne! Kiu volas \emph{perfektiĝi} en Esperanto, al tiu mi rekomendas la diversajn lernolibrojn kaj vortarojn, multe \emph{pli bonajn} kaj \emph{pli vastajn}, kiuj estas eldonitaj de niaj plej kompetentaj amikoj por ĉiu nacio aparte kaj el kiuj la plej gravaj estas eldonitaj tre bone kaj zorgeme, sub mia persona kontrolo kaj kunhelpo. Sed la «~Fundamento de Esperanto~» devas troviĝi en la manoj de ĉiu bona esperantisto kiel konstanta \emph{gvida dokumento}, por ke li bone ellernu kaj per ofta enrigardado konstante memorigadu al si, kio en nia lingvo estas oficiala kaj netuŝebla, por ke li povu ĉiam bone distingi la vortojn kaj regulojn \emph{oficialajn}, kiuj devas troviĝi en ĉiuj lernoverkoj de Esperanto, de la vortoj kaj reguloj rekomendataj \emph{private}, kiuj eble ne al ĉiuj esperantistoj estas konataj aŭ eble ne de ĉiuj estas aprobataj. La «~Fundamento de Esperanto~» devas troviĝi en la manoj de ĉiu esperantisto kiel konstanta \emph{kontrolilo}, kiu gardos lin de deflankiĝado de la vojo de unueco.

Mi diris, ke la fundamento de nia lingvo devas esti absolute netuŝebla, se eĉ ŝajnus al ni, ke tiu aŭ alia punkto estas sendube erara. Tio ĉi povus naski la penson, ke nia lingvo restos ĉiam rigida kaj neniam disvolviĝos... Ho, ne! Malgraŭ la severa netuŝebleco de la fundamento, nia lingvo havos la plenan eblon ne sole konstante riĉiĝadi, sed eĉ konstante \emph{pliboniĝadi} kaj \emph{perfektiĝadi}; la netuŝebleco de la fundamento nur garantios al ni konstante, ke tiu perfektiĝado fariĝados ne per arbitra, interbatala kaj ruiniga \emph{rompado} kaj \emph{ŝanĝado}, ne per nuligado aŭ sentaŭgigado de nia ĝisnuna literaturo, sed per vojo \emph{natura}, senkonfuza kaj sendanĝera. Pli detale mi parolos pri tio ĉi en la Bulonja kongreso; nun mi diros pri tio ĉi nur kelkajn vortojn, por ke mia opinio ne ŝajnu tro paradoksa:

1) \emph{Riĉigadi} la lingvon per novaj vortoj oni povas jam \emph{nun}, per konsiliĝado kun tiuj personoj, kiuj estas rigardataj kiel la plej aŭtoritataj en nia lingvo, kaj zorgante pri tio, ke ĉiuj uzu tiujn vortojn en la sama formo; sed tiuj ĉi vortoj devas esti nur rekomendataj, ne altrudataj; oni devas ilin uzadi nur en la \emph{literaturo}; sed en korespondado kun personoj \emph{nekonataj} estas bone ĉiam peni uzadi nur vortojn el la «~Fundamento~» ĉar nur pri tiaj vortoj ni povas esti certaj, ke nia adresato ilin nepre trovos en sia vortaro. Nur iam poste, kiam la plej granda parto de la novaj vortoj estos jam tute matura, ia aŭtoritata institucio enkondukos ilin en la vortaron \emph{oficialan}, kiel «~Aldonon al la Fundamento~»

2) Se ia aŭtoritata centra institucio trovos, ke tiu aŭ alia vorto aŭ regulo en nia lingvo estas \emph{tro neoportuna}, ĝi ne devos \emph{forigi} aŭ \emph{ŝanĝi} la diritan formon, sed ĝi povos proponi formon novan, kiun ĝi rekomendos uzadi \emph{paralele} kun la formo malnova. Kun la tempo la formo nova iom post iom elpuŝos la formon malnovan, kiu fariĝos \emph{arĥaismo}, kiel ni tion ĉi vidas en ĉiu natura lingvo. Sed, prezentante parton de la \emph{fundamento}, tiuj ĉi arĥaismoj neniam estos elĵetitaj, sed ĉiam estos presataj en ĉiuj lernolibroj kaj vortaroj samtempe kun la formoj novaj, kaj tiamaniere ni havos la certecon, ke eĉ ĉe la plej granda perfektiĝado la unueco de Esperanto neniam estos rompata kaj neniu verko Esperanta eĉ el la plej frua tempo iam perdos sian valoron kaj kompreneblecon por la estontaj generacioj.

Mi montris en \emph{principo}, kiamaniere la severa netuŝebleco de la «~Fundamento~» gardos ĉiam la unuecon de nia lingvo, ne malhelpante tamen al la lingvo ne sole riĉiĝadi, sed eĉ konstante \emph{perfektiĝadi}. Sed en la \emph{praktiko} ni (pro kaŭzoj jam multajn fojojn priparolitaj) devas kompreneble esti \emph{tre singardaj} kun ĉia «~perfektigado~» de la lingvo: a) ni devas tion ĉi fari ne facilanime, sed nur en okazoj de efektiva \emph{neceseco}; b) fari tion ĉi (post matura prijuĝado) povas ne apartaj personoj, sed nur ia centra institucio, kiu havos nedisputeblan aŭtoritatecon por la tuta esperantistaro.

Mi finas do per la jenaj vortoj:

1. pro la unueco de nia afero ĉiu bona esperantisto devas antaŭ ĉio bone koni la \emph{fundamenton} de nia lingvo;

2. la fundamento de nia lingvo devas resti por ĉiam \emph{netuŝebla};

3. ĝis la tempo kiam aŭtoritata centra institucio decidos \emph{pligrandigi} (neniam \emph{ŝanĝi}!) la ĝisnunan fundamenton per oficialigo de novaj vortoj aŭ reguloj, ĉio bona, kio ne troviĝas en la «~Fundamento de Esperanto~», devas esti rigardata ne kiel deviga, sed nur kiel rekomendata.

La ideoj, kiujn mi supre esprimis pri la Fundamento de Esperanto, prezentas dume nur mian \emph{privatan} opinion. Leĝan sankcion ili ricevos nur en tia okazo, se ili estos akceptitaj de la unua internacia kongreso de esperantistoj, al kiu tiu ĉi verko kune kun sia antaŭparolo estos prezentita.

\begin{flushright}
\small \fauxsc{L. Zamenhof}.~~~~~~~~~~~~~
\end{flushright}

{\footnotesize \hspace{3em} Varsovio, Julio 1905.}

\cleardoublepage



% Gramatiko por francoj
%
% Esperanta gramatiko por francoj
%
\mainmatter
\label{gram:franca}
\markright{GRAMMAIRE}
\thispagestyle{plain}
\begin{center}
\phantomsection
\narrow{\huge\bf \spaceoutless{FUNDAMENTA GRAMATIKO}}
\addcontentsline{toc}{chapter}{Fundamenta Gramatiko}
\vspace{1em}

\narrow{\bf DE LA LINGVO ESPERANTO}
\vspace{1em}

\narrow{\LARGE {\spaceoutmed{\bf EN KVIN LINGVOJ}}}

\rule{0.9\textwidth}{0.4pt}
\vspace{2em}

\phantomsection
\narrow{\LARGE\bf \spaceoutless{GRAMMAIRE}}
\addcontentsline{toc}{section}{Grammaire (Gramatiko Franca)}
\selectlanguage{french}

\rule{13mm}{0.4pt}\\[1em]

{\large \gramsec{A) ALPHABET}}
\vspace{1em}

\setstretch{1}
\begin{tabu} to \textwidth{+Y@{}ZY@{}ZY@{}ZY@{}ZY@{}ZY@{}ZY}
\rowstyle{\Large\arbfont} Aa, & Bb, & Cc, & Ĉĉ, & Dd, & Ee, & Ff, \\
\rowstyle{\footnotesize} â & b & ts (tsar) & tch (tchèque) & d & é & f \\[1ex]
\rowstyle{\Large\arbfont} Gg, & Ĝĝ, & Hh, & Ĥĥ, & Ii, & Jj, & Ĵĵ, \\
\rowstyle{\footnotesize} g dur~(gant) & dj (adjudant) & h légère\-ment aspiré & h~forte\-ment aspiré & i & y (yeux) & j \\[1ex]
\rowstyle{\Large\arbfont} Kk, & Ll, & Mm, & Nn, & Oo, & Pp, & Rr, \\
\rowstyle{\footnotesize} k & l & m & n & ô & p & r \\[1ex]
\rowstyle{\Large\arbfont} Ss, & Ŝŝ, & Tt, & Uu, & Ŭŭ, & Vv, & Zz. \\
\rowstyle{\footnotesize} ss, ç & ch (chat) & t & ou & ou bref (dans~l‘alle\-mand~„laut„) & v & z 
\end{tabu}
\end{center}

{\footnotesize \br{Remarque.} ― Les typographies qui n’ont pas les caractères \emph{ĉ, ĝ, ĥ, ĵ, ŝ, ŭ,} peuvent les remplacer par \emph{ch, gh, hh, jh, sh, u.}}
\begin{center}
\large \gramsec{B) PARTIES DU DISCOURS}
\end{center}

\textbf{1.} L’Esperanto n’a qu’un \textbf{article défini} (\emph{la}), invariable pour tous les genres, nombres et cas. Il n’a pas d’article indéfini.

{\footnotesize \br{Remarque.} ― L’emploi de l’article est le même qu’en français ou en allemand. Mais les personnes auxquelles il présenterait quelque difficulté peuvent fort bien ne pas s’en servir.}

\textbf{2.} Le \textbf{substantif} finit toujours par \emph{o}. Pour former le pluriel on ajoute \emph{j} au singulier. La langue n’a que deux cas: le \emph{nominatif} et \emph{l’accusatif}. Ce dernier se forme du nominatif par l’addition d’un \emph{n}. Les autres cas sont marqués par des prépositions: le \emph{génitif} par \emph{de} (de), le \emph{datif} par \emph{al} (à), l’\emph{ablatif} par \emph{per} (par, au moyen de) ou par d’autres prépositions, selon le sens. Ex.: \emph{la patr$'$o} ― le père; \emph{al la patr$'$o} ― au père, \emph{de la patr$'$o} ― du père, \emph{la patr$'$o$'$n} ― le père (à l’accusatif, c.-à-d. complément direct), \emph{per la patr$'$o$'$j} ― par les pères ou au moyen des pères, \emph{la patr$'$o$'$j$'$n} ― les pères (accus. plur.), \emph{por la patr$'$o} ― pour le père, \emph{kun la patr$'$o} ― avec le père, etc.

\textbf{3. L’adjectif} finit toujours par \emph{a}. Ses cas et ses nombres se marquent de la même manière que ceux du substantif. Le \emph{comparatif} se forme à l’aide du mot \emph{pli} ― plus, et le \emph{superlatif} à l’aide du mot \emph{plej} ― le plus. Le „que“ du comparatif se traduit par \emph{„ol“} et le „de“ du superlatif par \emph{„el“} (d’entre). Ex.: \emph{pli blank$'$a ol neĝ$'$o} ― plus blanc que neige; \emph{mi hav$'$as la plej bel$'$a$'$n patr$'$in$'$o$'$n el ĉiu$'$j} ― j’ai la plus belle mère de toutes.

\textbf{4.} Les \emph{adjectifs} \textbf{numéraux} \emph{cardinaux} sont invariables: \emph{unu} (1), \emph{du} (2), \emph{tri} (3), \emph{kvar} (4), \emph{kvin} (5), \emph{ses} (6), \emph{sep} (7), \emph{ok} (8), \emph{naŭ} (9), \emph{dek} (10), \emph{cent} (100), \emph{mil} (1000). Le dizaines et les centaines se forment par la simple réunion des dix premiers nombres. Aux adjectifs numéraux cardinaux on ajoute: la terminaison (\emph{a}) de l’adjectif, pour les \emph{numéraux ordinaux;} \emph{obl}, pour les numéraux \emph{multiplicatifs}; \emph{on}, pour les numéraux \emph{fractionnaires}; \emph{op}, pour les numéraux \emph{collectifs}. On met \emph{po} avant ces nombres pour marquer les numéraux \emph{distributifs}. Enfin, dans la langue, les adjectifs numéraux peuvent s’employer substantivement ou adverbialement. Ex.: \emph{Kvin$'$cent tri$'$dek tri} ― 533; \emph{kvar$'$a} ― 4\textsuperscript{me}; \emph{tri$'$obl$'$a} ― triple; \emph{kvar$'$on$'$o} ― un quart; \emph{du$'$op$'$e} ― à deux; \emph{po kvin} ― au taux de cinq (chacun); \emph{unu$'$o} ― (l’) unité; \emph{sep$'$e} ― septièmement.

\textbf{5.} Les \textbf{pronoms} personnels sont: \emph{mi} (je, moi), \emph{vi} (vous, tu, toi), \emph{li} (il, lui), \emph{ŝi} (elle), \emph{ĝi} (il, elle, pour les animaux ou les choses), \emph{si} (soi), \emph{ni} (nous), \emph{ili} (ils, elles), \emph{oni} (on). Pour en faire des adjectifs ou des pronoms possessifs, on ajoute la terminaison (\emph{a}) de l’adjectif. Les pronoms se déclinent comme le substantif. Ex.: \emph{mi$'$n} ― moi, me (accus.), \emph{mi$'$a} ― mon, \emph{la vi$'$a$'$j} ― les vôtres.

\textbf{6.} Le \textbf{verbe} ne change ni pour les personnes, ni pour les nombres. Ex.: \emph{mi far$'$as} ― je fais, \emph{la patr$'$o far$'$as} ― le père fait, \emph{ili far$'$as} ― ils font.

\begin{center}
\bf Formes du verbe:
\end{center}

a) Le \emph{présent} est caractérisé par \emph{as}; ex.: \emph{mi far$'$as} ― je fais.

b) Le \emph{passé}, par \emph{is}: \emph{vi far$'$is} ― vous faisiez, vous avez fait.

c) Le \emph{futur}, par \emph{os}: \emph{ili far$'$os} ― ils feront.

ĉ) Le \emph{conditionnel}, par \emph{us}: \emph{ŝi far$'$us} ― elle ferait.

d) L’\emph{impératif}, par \emph{u}: \emph{far$'$u} ― fais, faites; \emph{ni far$'$u} ― faisons.

e) L’\emph{infinitif}, par \emph{i}: \emph{far$'$i} ― faire.

f) Le \emph{participe présent actif}, par \emph{ant}: \emph{far$'$ant$'$a} ― faisant, \emph{far$'$ant$'$e} ― en faisant.

g) Le \emph{participe passé actif}, par \emph{int}: \emph{far$'$int$'$a} ― ayant fait.

ĝ) Le \emph{participe futur actif}, par \emph{ont}: \emph{far$'$ont$'$a} ― devant faire, qui fera.

h) Le \emph{participe présent passif}, par \emph{at}: \emph{far$'$at$'$a} ― étant fait, qu’on fait.

ĥ) Le \emph{participe passé passif}, par \emph{it}: \emph{far$'$it$'$a} - ayant été fait, qu’on a fait.

i) Le \emph{participe futur passif}, par \emph{ot}: \emph{far$'$ot$'$a} ― devant être fait, qu’on fera.

La voix passive n’est que la combinaison du verbe \emph{est} (être) et du participe présent ou passé du verbe passif donné. Le „de“ ou le „par“ du complément indirect se rendent par \emph{de}. Ex.: \emph{ŝi est$'$as am$'$at$'$a de ĉiu$'$j} ― elle est aimée de tous (part. prés.: la chose se fait). \emph{La pord$'$o est$'$as ferm$'$it$'$a} ― la porte est fermée (part. pas.: la chose a été faite).

\textbf{7.} L’\textbf{adverbe} est caractérisé par \emph{e}. Ses degrés de comparaison se marquent de la même manière que ceux de l’adjectif. Ex.: \emph{mi$'$a frat$'$o pli bon$'$e kant$'$as ol mi} ― mon frère chante mieux que moi.

\textbf{8.} Toutes les \textbf{prépositions} veulent, par elles-mêmes, le nominatif.

\begin{center}
\large \gramsec{C) RÈGLES GÉNÉRALES}
\end{center}

\textbf{9.} Chaque mot se prononce absolument comme il est écrit.

\textbf{10.} L’accent tonique se place toujours sur l’avant-dernière syllabe.

\textbf{11.} Les mots composés s’obtiennent par la simple réunion des éléments qui les forment, écrits ensemble, mais séparés par de petits traits\footnote{Dans les lettres ou dans les ouvrages, qui s’adressent à des personnes connaissant déjà la langue, on peut omettre ces petits traits. Ils ont pour but de permettre à tous de trouver aisément, dans le dictionnaire, le sens précis de chacun des éléments du mot, et d’en obtenir ainsi la signification complète, sans aucune étude préalable de la grammaire.}). Le mot fondamental doit toujours être à la fin. Les terminaisons grammaticales sont considérées comme des mots. Ex.: \emph{vapor$'$ŝip$'$o} (bateau à vapeur) est formé de: \emph{vapor} ― vapeur, \emph{ŝip} ― bateau, \emph{o} ― terminaison caractéristique du substantif.

\textbf{12.} S’il y a dans la phrase un autre mot de sens négatif, l’adverbe „ne“ se supprime. Ex.: \emph{mi neniam vid$'$is} ― je n’ai jamais vu.

\textbf{13.} Si le mot marque le lieu où l’on va, il prend la terminaison de l’accusatif. Ex.: \emph{kie vi est$'$as?} ― où êtes-vous? \emph{kie$'$n vi ir$'$as?} ― où allez-vous? \emph{Mi ir$'$as Pariz$'$o$'$n} ― je vais à Paris.

\textbf{14.} Chaque préposition possède, en Esperanto, un sens immuable et bien déterminé, qui en fixe l’emploi. Cependant, si le choix de celle-ci plutôt que de celle-là ne s’impose pas clairement à l’esprit, on fait usage de la préposition \emph{je} qui n’a pas de signification propre. Ex.: \emph{ĝoj$'$i je tio} ― s’en réjouir, \emph{rid$'$i je tio} ― en rire, \emph{enu$'$o je la patr$'$uj$'$o} ― regret de la patrie.

La clarté de la langue n’en souffre aucunement, car, dans toutes, on emploie, en pareil cas, une préposition quelconque, pourvu qu’elle soit sanctionnée par l’usage. L’Esperanto adopte pour cet office la seule préposition \emph{je}.

A sa place on peut cependant employer aussi l’accusatif sans préposition, quand aucune amphibologie n’est à craindre.

\textbf{15.} Les mots „étrangers“ c.-à-d. ceux que la plupart des langues ont empruntés à la même source, ne changent pas en Esperanto. Ils prennent seulement l’orthographe et les terminaisons grammaticales de la langue. Mais quand, dans une catégorie, plusieurs mots différents dérivent de la même racine, il vaut mieux n’employer que le mot fondamental, sans altération, et former les autres d’après les règles de la langue internationale. Ex.: tragédie ― \emph{tragedi$'$o}, tragique ― \emph{tragedi$'$a}.

\textbf{16.} Les terminaisons des substantifs et de l’article peuvent se supprimer et se remplacer par une apostrophe. Ex.: \emph{Ŝiller’} (Schiller) au lieu de \emph{Ŝiller$'$o;} \emph{de l’ mond$'$o} au lieu de \emph{de la mond$'$o}.

\newpage


% Gramatiko por angloj
%
% Esperanta gramatiko por angloj
%
\label{gram:angla}
\markright{GRAMMAR}
\thispagestyle{plain}
\begin{center}
\phantomsection
\narrow{\LARGE\bf \spaceoutless{GRAMMAR}}
\addcontentsline{toc}{section}{Grammar (Gramatiko Angla)}
\selectlanguage{english}

\rule{13mm}{0.4pt}
\vspace{2em}

{\large\gramsec{A) THE ALPHABET}}
\vspace{1em}

\setstretch{1}
\begin{tabu} to \textwidth{+Y@{}ZY@{}ZY@{}ZY@{}ZY@{}ZY@{}ZY}
\rowstyle{\Large\arbfont} Aa, & Bb, & Cc, & Ĉĉ, & Dd, & Ee, & Ff, \\
\rowstyle{\footnotesize} \emph{a} as in „last“ & \emph{b} as in „be“ & \emph{ts} as in „wits“ & \emph{ch} as in „church“ & \emph{d} as in „do“ & \emph{e} as in „make“ & \emph{f} as in „fly“ \\[1ex]
\rowstyle{\Large\arbfont} Gg, & Ĝĝ, & Hh, & Ĥĥ, & Ii, & Jj, & Ĵĵ, \\
\rowstyle{\footnotesize} \emph{g} as in „gun“ & \emph{j} as in „join“ & \emph{h} as in „half“ & strongly aspirated h, \emph{ch} as in „loch“ (Scotch) & \emph{i} as in „marine“ & \emph{y} as in „yoke“ & \emph{z} as in „azure“  \\[1ex]
\rowstyle{\Large\arbfont} Kk, & Ll, & Mm, & Nn, & Oo, & Pp, & Rr, \\
\rowstyle{\footnotesize} \emph{k} as in „key“ & \emph{l} as in „line“ & \emph{m} as in „make“ & \emph{n} as in „now“ & \emph{o} as in „not“ & \emph{p} as in „pair“ & \emph{r} as in „rare“ \\[1ex]
\rowstyle{\Large\arbfont} Ss, & Ŝŝ, & Tt, & Uu, & Ŭŭ, & Vv, & Zz. \\
\rowstyle{\footnotesize}  \emph{s} as in „see“ & \emph{sh} as in „show“ & \emph{t} as in „tea“ & \emph{u} as in „bull“ & \emph{u} as in „mount“ (used in diphthongs) & \emph{v} as in „very“ & \emph{z} as in „zeal“ 
\end{tabu}
\end{center}

{\footnotesize {\bf Remark.} — If it be found impractical to print works with the diacritical signs (ˆ,˘), the letter \emph{h} may be substituted for the sign (ˆ), and the sign ( ˘ ) may be altogether omitted.}
\begin{center}
\large \gramsec{B) PARTS OF SPEECH}
\end{center}

\textbf{1.} There is no indefinite, and only one definite, article, \emph{la}, for all genders, numbers, and cases.

\textbf{2.} Substantives are formed by adding \emph{o} to the root. For the plural, the letter \emph{j} must be added to the singular. There are two cases: the nominative and the objective (accusative). The root with the added \emph{o} is the nominative, the objective adds an \emph{n} after the \emph{o}. Other cases are formed by prepositions; thus, the possessive (genitive) by \emph{de}, “of”; the dative by \emph{al}, “to”; the instrumental (ablative) by \emph{kun}, “with”, or other preposition as the sense demands. E.~g., root \emph{patr}, “father”; \emph{la patr$'$o}, “the father”; \emph{patr$'$o$'$n}, “father” (objective), \emph{de la patr$'$o}, “of the father”, \emph{al la patr$'$o}, “to the father”, \emph{kun la patr$'$o}, “with the father”; \emph{la patro$'$j}, “the fathers”; \emph{la patro$'$j$'$n}, “the fathers” (obj.), \emph{por la patr$'$o$'$j}, “for the fathers”.

\textbf{3.} Adjectives are formed by adding \emph{a} to the root. The numbers and cases are the same as in substantives. The comparative degree is formed by prefixing \emph{pli} (more); the superlative by \emph{plej} (most). The word “than” is rendered by \emph{ol}, e.~g., \emph{pli blank$'$a ol neĝ$'$o}, “whiter than snow”.

\textbf{4.} The cardinal numerals do not change their forms for the different cases. They are :
\emph{unu} (1), \emph{du} (2), \emph{tri} (3), \emph{kvar} (4), \emph{kvin} (5), \emph{ses} (6), \emph{sep} (7), \emph{ok} (8), \emph{naŭ} (9), \emph{dek} (10), \emph{cent} (100), \emph{mil} (1000).

The tens and hundreds are formed by simple junction of the numerals, e.~g., 583\eraro{533} = \emph{kvin$'$cent tri$'$dek tri}.

Ordinals are formed by adding the adjectival \emph{a} to the cardinals, e.~g., \emph{unu$'$a}, “first”; \emph{du$'$a}, “second”, etc.

Multiplicatives (as “threefold”, “fourfold”, etc.) add \emph{obl}, e.~g., \emph{tri$'$obl$'$a}, “threefold”.

Fractionals add \emph{on}, as \emph{du$'$on$'$o}, “a half”, \emph{kvar$'$on,o}, “a quarter”. Collective numerals add \emph{op}, as \emph{kvar$'$op$'$e}, “four together”.

Distributives prefix \emph{po}, e.~g., \emph{po kvin}, “five apiece”.

Adverbials take \emph{e}, e.~g., \emph{unu$'$e}, “firstly”, etc.

\textbf{5.} The Personal Pronouns are: \emph{mi}, I; \emph{vi}, thou, you; \emph{li}, he; \emph{ŝi}, she; \emph{ĝi}, it; \emph{si}, “self”; \emph{ni}, “we”; \emph{ili}, “they”; \emph{oni}, “one”, “people”, (French “on”).

Possessive pronouns are formed by suffixing to the required personal, the adjectival termination. The declension of the pronouns is identical with that of substantives. E.~g., \emph{mi}, “I”; \emph{mi$'$n}, “me” (obj.); \emph{mi$'$a}, “my”, “mine”.

\textbf{6.} The verb does not change its form for numbers or persons, e.~g., \emph{mi far$'$as}, “I do”; \emph{la patr$'$o far$'$as}, “the father does”; \emph{ili far$'$as}, “they do”.

\begin{center}
\bf Forms of the Verb:
\end{center}

a) The present tense ends in \emph{as}, e.~g., \emph{mi far$'$as}, “I do”.

b) The past tense ends in \emph{is}, e.~g., \emph{li far$'$is}, “he did”.

c) The future tense ends in \emph{os}, e.~g., \emph{ili far$'$os}, “they will do”.

ĉ) The subjunctive mood ends in \emph{us}, e.~g., \emph{ŝi far$'$us}, “the\eraro{``she''} may do”.

d) The imperative mood ends in \emph{u}, e.~g., \emph{ni far$'$u}, “let us do”.

e) The infinitive mood ends in \emph{i}, e.~g., \emph{far$'$i}, “to do”.

There are two forms of the participle in the international language, the changeable or adjectival, and the unchangeable or adverbial. 

f) The present participle active ends in \emph{ant}, e.~g., \emph{far$'$ant$'$a}, “he who is doing”; \emph{far$'$ant$'$e}, “doing”.

g) The past participle active ends in \emph{int}, e.~g., \emph{far$'$int$'$a}, “he who has done”; \emph{far$'$int$'$e}, “having done”.

ĝ) The future participle active ends in \emph{ont}, e.~g., \emph{far$'$ont$'$a}, “he who will do”; \emph{far$'$ont$'$e}, “about to do”.

h) The present participle passive ends in \emph{at}, e.~g., \emph{far$'$at$'$e}, “being done”.

ĥ) The past participle passive ends in \emph{it}, e.~g., \emph{far$'$it$'$a}, “that which has been done”; \emph{far$'$it$'$e}, “having been done”.

i) The future participle passive ends in \emph{ot}, e.~g., \emph{far$'$ot$'$a}, “that which will be done”; \emph{far$'$ot$'$e}, “about to be done”.

All forms of the passive are rendered by the respective forms of the verb \emph{est} (to be) and the present participle passive of the required verb; the preposition used is \emph{de}, “by”. E.~g., \emph{ŝi est$'$as am$'$at$'$a de ĉiu$'$j}, “she is loved by every one.”

\textbf{7.} Adverbs are formed by adding \emph{e} to the root. The degrees of comparison are the same as in adjectives, e.~g., \emph{mi$'$a frat$'$o kant$'$as pli bon$'$e ol mi}, “my brother sings better than I”.

\textbf{8.} All prepositions govern the nominative case.

\begin{center}
\large \gramsec{C) GENERAL RULES}
\end{center}

\textbf{9.} Every word is to be read exactly as written, there are no silent letters.

\textbf{10.} The accent falls on the last syllable but one, (penultimate).

\textbf{11.} Compound words are formed by the simple junction of roots, (the principal word standing last), which are written as a single word, but, in elementary works, separated by a small line ($'$). Grammatical terminations are considered as independent words, e.~g., \emph{vapor$'$ŝip$'$o}, “steamboat”, is composed of the roots \emph{vapor}, “steam”, and \emph{ŝip}, “a boat”, with the substantival termination \emph{o}.

\textbf{12.} If there be one negative in a clause, a second is not admissible.

\textbf{13.} In phrases answering the question “where?” (meaning direction), the words take the termination of the objective case; e.~g., \emph{kie$'$n vi ir$'$as?} “where are you going?” \emph{dom$'$o$'$n}, “home”; \emph{London$'$o$'$n}, “to London”; etc.

\textbf{14.} Every preposition in the international language has a definite fixed meaning. If it be necessary to employ some preposition, and it is not quite evident from the sense which it should be, the word \emph{je} is used, which has no definite meaning; for example, \emph{ĝoj$'$i je tio}, “to rejoice \emph{over} it”; \emph{rid$'$i je tio} “to laugh \emph{at} it”; \emph{enu$'$o je la patr$'$uj$'$o}, “a longing \emph{for} one’s fatherland”. In every language different prepositions, sanctioned by usage, are employed in these dubious cases, in the international language, one word, \emph{je}, suffices for all. Instead of \emph{je}, the objective without a preposition may be used, when no confusion is to be feared.

\textbf{15.} The so-called “foreign” words, i.~e., words which the greater number of languages have derived from the same source, undergo no change in the international language, beyond conforming to its system of orthography.---Such is the rule with regard to primary words, derivatives are better formed (from the primary word) according to the rules of the international grammar: e.~g., \emph{teatr$'$o}, “theater”, but \emph{teatr$'$a}, “theatrical”, (not \emph{teatrical$'$a}), etc.

\textbf{16.} The \emph{a} of the article, and the final \emph{o} of substantives, may be sometimes dropped euphoniae gratia, e.~g., \emph{de l’ mond$'$o} for \emph{de la mond$'$o}; \emph{Ŝiller’} for \emph{Ŝiller$'$o}; in such cases an apostrophe should be substituted for the discarded vowel. 

\newpage

% Gramatiko por germanoj
%
% Esperanta gramatiko por germanoj
%
\label{gram:germana}
\markboth{GRAMMATIK}{FUNDAMENTO DE ESPERANTO}
\thispagestyle{plain}
\begin{center}
\phantomsection
{\fr\huge Grammatik}
\addcontentsline{toc}{section}{Grammatik (Gramatiko Germana)}
\selectlanguage{german}
\vspace{1em}

\rule{13mm}{0.4pt}
\vspace{1em}

\large\gramsec{A)} {\Large\bf\fr{Das Alphabet.}}
\vspace{1ex}

\setstretch{1}
\begin{tabu} to \textwidth{+Y@{}ZY@{}ZY@{}ZY@{}ZY@{}ZY@{}ZY}
\rowstyle{\Large\arbfont} Aa, & Bb, & Cc, & Ĉĉ, & Dd, & Ee, & Ff, \\
\rowstyle{\small} a & b & c, z & tsch & d & e & f \\[1ex]
\rowstyle{\Large\arbfont} Gg, & Ĝĝ, & Hh, & Ĥĥ, & Ii, & Jj, & Ĵĵ, \\
\rowstyle{\small} g & dsch & h & ch & i & j  & sh \\[1ex]
\rowstyle{\Large\arbfont} Kk, & Ll, & Mm, & Nn, & Oo, & Pp, & Rr, \\
\rowstyle{\small} k  & l & m & n & o & p & r \\[1ex]
\rowstyle{\Large\arbfont} Ss, & Ŝŝ, & Tt, & Uu, & Ŭŭ, & Vv, & Zz. \\
\rowstyle{\small} ss & sch & t & u & kurzes u & w & s \\
\end{tabu}
{\raggedleft\scriptsize(wie in \glqq{}lesen\grqq{})\par}

\end{center}

{\fr
\small \spaceout{Anmerkung:} ~\std{ĝ} lautet wie das engliſche \std{\glqq{}g\grqq{}} in \std{\glqq{}gentleman\grqq{}}; \std{ĵ —}wie das franzöſiſche \std{\glqq{}j\grqq{}} in \std{\glqq{}journal\grqq{}}; \std{u —} wie das kurze \glqq{}u\grqq{} in \glqq{}glauben\grqq{} (wird nur nach einem Vokal gebraucht). Bei mangelnden Typen im Druck erſetzt man \std{ĉ, ĝ, ĥ, ĵ, ŝ, ŭ} durch \std{ch, gh, hh, jh, sh, u.}}
\begin{center}
\large\gramsec{B)} \Large\bf\fr{Redetheile.}
\end{center}

{\fr \large
    \textbf{1.} Der beſtimmte \textbf{Artikel} iſt \std{la}, für alle Geſchlechter und Fälle, für die Einzahl und Mehrzahl. Einen unbeſtimmten Artikel gibt es nicht.
    
    \textbf{2.} Das \textbf{Hauptwort} bekommt immer die Endung \std{o}. Der Plural bekommt die Endung \std{j}. Es gibt nur zwei Fälle: Nominativ und Akkuſativ; der letztere entſteht aus dem Nominativ, indem die Endung \std{n} hinzugefügt wird. Die übrigen Fälle werden vermittelſt der Präpoſitionen ausgedrückt: der Genitiv durch \std{de} (von), der Dativ durch \std{al} (zu), der Ablativ durch \std{kun} (mit), oder andere, dem Sinne entſprechende, Präpoſitionen. Z.~B. \std{la patr$'$o}, der Vater; \std{al la patr$'$o}, dem Vater; \std{la patr$'$o$'$n}, den Vater; \std{la patr$'$o$'$j$'$n}, die Väter (Akkuſativ).
    
    \textbf{3}. Das \textbf{Eigenſchaftswort} endet immer auf \std{a}. Deklinationen wie beim Subſtantiv. Der Komparativ wird mit Hülfe des Wortes \std{pli} (mehr), der Superlativ durch \std{plej} (am meiſten) gebildet. Das Wort \glqq{}als\grqq{} heißt \std{ol}. Z.~B.: \std{pli blank$'$a ol neĝ$'$o,} weißer als Schnee.
    
    \textbf{4.} Die \textbf{Grundzahlwörter} (undeklinirbar) ſind folgende: \std{unu (1), du (2), tri (3), kvar (4), kvin (5), ses (6), sep (7), ok (8), naŭ (9), dek (10), cent (100), mil (1000).} Zehner und Hunderte werden durch einfache Anreihung der Zahlwörter gebildet; z.~B.: \std{kvin$'$cent tri$'$dek tri = 533.} Ordnungszahlwörter entſtehen, indem ſie die Endung des Adjektivs annehmen; z.~B. \std{kvar$'$a}, vierter. Vervielfältigungszahlwörter \std{—} durch Einſchiebung des Suffixes \std{obl;} z.~B.: \std{tri$'$obl$'$a,} dreifach. Bruchzahlwörter \std{—} durch \std{on}; z.~B. \std{kvar$'$on$'$o,} ein Viertel. Sammelzahlwörter \std{—} durch \std{op}; z.~B. \std{du$'$op$'$e,} ſelbander. Diſtributive Zahlwörter \std{—} durch das Wort \std{po;} z.~B. \std{po kvin,} zu fünf. Außerdem gibt es Subſtantiv- und Adverbialzahlwörter; z.~B. \std{cent$'$o,} das Hundert, \std{du$'$e}, zweitens.
    
    \textbf{5.} Die persönlichen \textbf{Fürwörter} sind: \std{mi} (ich), \std{vi} (du, Ihr), \std{li} (er), \std{ŝi} (sie), \std{ĝi} (es; von Thieren oder Sachen), \std{si} (sich), \std{ni} (wir), \std{ili} (sie [Mehrzahl]), \std{oni} (man). Poſſeſſive Pronomina werden durch die Hinzufügung der Endung des Adjektivs gebildet. Die Pronomina werden gleich den Subſtantiven deklinirt. Z.~B.: \std{mi$'$a,} mein, \std{mi$'$n,} mich.
       
    \textbf{6.} Das \textbf{Zeitwort} hat weder Personen noch Mehrzahl; z.~B. \std{mi far$'$as,} ich mache; \std{la patr$'$o far$'$as,} der Vater macht; \std{ili far$'$as,} sie machen.

\begin{minipage}{\textwidth}
\begin{center}
\textbf{Formen des Zeitwortes :}
\end{center}

        \std{a)} Das Präſens endet auf \std{as}; z.~B. \std{mi far$'$as,} ich mache.
\end{minipage}
        
        \std{b)} Die vergangene Zeit \std{―} auf \std{is}; z.~B. \std{li far$'$is,} er hat gemacht.
        
        \std{c)} Das Futurum \std{―} auf \std{os}; z.~B. \std{ili far$'$os,} ſie werden machen.
        
        \std{ĉ)} Der Konditionalis \std{―} auf \std{us}; z.~B. \std{ŝi far$'$us,} ſie würde machen.
        
        \std{d)} Der Imperativ \std{―} auf \std{u}; z.~B. \std{far$'$u,} mache, macht; \std{ni far$'$u,} laſſet uns machen.
        
        \std{e)} Der Infinitiv \std{―} auf \std{i}; z.~B. \std{far$'$i,} machen.
        
        \std{f)} Partizipium präſentis aktivi \std{―} auf \std{ant}; z.~B. \std{far$'$ant$'$a}, machender; \std{far$'$ant$'$e,} machend.
        
        \std{g)} Partizipium perfekti aktivi \std{―} \std{int}; z.~B. \std{far$'$int$'$a,} der gemacht hat.
        
        \std{ĝ)} Partizipium futuri aktivi \std{―} \std{ont}; \std{far$'$ont$'$a,} der machen wird.
        
        \std{h)} Partizipium präſentis paſſivi \std{―} \std{at}; z.~B. \std{far$'$at$'$a,} der gemacht wird.
        
        \std{ĥ)} Partizipium perfekti paſſivi \std{―} \std{it}; z.~B. \std{far$'$it$'$a,} gemacht.
        
        \std{i)} Partizipium futuri paſſivi \std{―} \std{ot}; \std{far$'$ot$'$a,} der gemacht werden wird.

    Alle Formen des Paſſivs werden mit Hülfe der entſprechenden Form des Wortes \std{est} (ſein) und des Partizipium paſſivi des gegebenen Zeitwortes gebildet, wobei die Präpoſition \std{de} gebraucht wird; z.~B. \std{ŝi est$'$as am$'$at$'$a de ĉiu$'$j,} ſie wird von Allen geliebt.
    
    \textbf{7.} Das \textbf{Adverbium} endet auf \std{e}; Komparation wie beim Adjektiv. Z.~B: \std{mi$'$a frat$'$o pli bon$'$e kanta$'$as ol mi} = mein Bruder ſingt beſſer als ich.
    
    \textbf{8.} Alle \textbf{Präpoſitionen} regieren den Nominativ.
}

\begin{center}
\Large \bf C) \fr{Allgemeine Regeln.}
\end{center}

\enlargethispage{-\baselineskip}
{\fr \large
    \textbf{9.} Jedes Wort wird geleſen ſo wie es geſchrieben ſteht.
    
    \textbf{10.} Der Accent fällt immer auf die vorletzte Silbe.
    
    \textbf{11.} Zuſammengeſetzte Wörter entſtehen durch einfache Anreihung der Wörter, indem man ſie durch hochſtehende Striche trennt\footnote{\fr \small Im Briefwechſel mit ſolchen Perſonen, die der internationalen Sprache ſchon mächtig ſind, oder in Werken, die für eben ſolche Perſonen beſtimmt ſind; fallen die hochſtehenden Striche zwiſchen den verſchiedenen Theilen der Wörter weg.}. Das Grundwort kommt zuletzt. Grammatikaliſche Endungen werden als ſelbſtſtändige Wörter betrachtet. Z.~B. \std{vapor$'$ŝip$'$o} (Dampfschiff) beſteht aus \std{vapor,} Dampf, \std{ŝip,} Schiff, und \std{o}=Endung des Subſtantivs.

    \textbf{12.} Wenn im Satze ein Wort vorkommt, das von ſelbſt eine verneinende Bedeutung hat, ſo wird die Negation ne weggelaſſen; z.~B. \std{mi nenio$'$n vid$'$is,} ich habe Nichts geſehen.
    
    \textbf{13.} Auf die Frage \glqq{}wohin\grqq{} nehmen die Wörter die Endung des Akkuſativs an; z.~B. \std{tie,} da; \std{tie$'$n,} dahin; \std{Varsovi$'$o$'$n,} nach Warschau.
    
    \textbf{14.} Jede Präpoſition hat eine beſtimmte, feſte Bedeutung; iſt es aber aus dem Sinne des Satzes nicht erſichtlich, welche Präpoſition anzuwenden iſt, ſo wird die Präpoſition \std{je} gebraucht, welche keine ſelbſtſtändige Bedeutung hat; z.~B. \std{ĝoj$'$i je tio,} ſich darüber freuen; \std{rid$'$i je tio,} darüber lachen; \std{enu$'$o je la patr$'$uj$'$o,} Sehnſucht nach dem Vaterlande, \&c. Die Klarheit leidet keineswegs darunter, da doch dasſelbe in allen Sprachen geſchieht, nämlich, daß man in ſolchen Fällen eine beliebige Präpoſition begraucht, wenn ſie nur einmal angenommen iſt. In der internationalen Sprache wird in ſolchen Fällen immer nur die eine Präpoſition \std{je} angewendet. Statt der Präpoſition \std{je} kann man auch den Akkuſativ ohne Präpoſition gebrauchen, wo kein Doppelſinn zu befürchten iſt.
    
    \textbf{15.} Sogenannte Fremdwörter, d.~h. ſolche Wörter, welche die Mehrheit der Sprachen aus einer und derſelben fremden Quelle entlehnt hat, werden in der internationalen Sprache unverändert gebraucht, indem ſie nur die internationale Orthographie annehmen; aber bei verſchiedenen Wörtern, die eine gemeinſame Wurzel haben, ist es beſſer, nur das Grundwort unverändert zu gebrauchen, die abgeleiteten Wörter aber \std{―} nach den Regeln der internationalen Sprache zu bilden; z.~B. Theater, \std{teatr$'$o;} theatraliſch, \std{teatr$'$a.}
    
    \textbf{16.} Die Endung des Subſtantivs und des Artikels kann ausgelaſſen werden, indem man dieſelbe durch einen Apoſtroph erſetzt; z.~B. \std{Ŝiller’,} ſtatt \std{Ŝiller$'$o;} \std{de l’ mond$'$o,} ſtatt \std{de la mond$'$o.} 
}
\newpage


% Gramatiko por rusoj
%
% Esperanta gramatiko por rusoj
%
\label{gram:rusa}
\markboth{FUNDAMENTO DE ESPERANTO}{ГРАММАТИКА}
\thispagestyle{plain}
\begin{center}
\phantomsection
\narrow{\LARGE\textbf{ГРАММАТИКА}}
\addcontentsline{toc}{section}{Грамматика (Gramatiko Rusa)}
\selectlanguage{russian}

\rule{13mm}{0.4pt}
\vspace{1em}

{\large\gramsec{A) АЗБУКА}}
\vspace{1em}

\setstretch{1}
\begin{tabu} to \textwidth{+Y@{}ZY@{}ZY@{}ZY@{}ZY@{}ZY@{}ZY}
\rowstyle{\Large\arbfont} Aa, & Bb, & Cc, & Ĉĉ, & Dd, & Ee, & Ff, \\
\rowstyle{\small} а & б & ц & ч & д & э & ф \\[1ex]
\rowstyle{\Large\arbfont} Gg, & Ĝĝ, & Hh, & Ĥĥ, & Ii, & Jj, & Ĵĵ, \\
\rowstyle{\small} г & дж & (г, х) & х & и & й  & ж \\[1ex]
\rowstyle{\Large\arbfont} Kk, & Ll, & Mm, & Nn, & Oo, & Pp, & Rr, \\
\rowstyle{\small} к  & л & м & н & о & п & п \\[1ex]
\rowstyle{\Large\arbfont} Ss, & Ŝŝ, & Tt, & Uu, & Ŭŭ, & Vv, & Zz. \\
\rowstyle{\small} с & ш & т & у & у (краткое) & в & з \\
\end{tabu}
\end{center}

{\footnotesize \br{Примѣчаніе I.} Буква \es{h} произносится какъ \es{h} въ языкахъ нѣмецкомъ, латинскомъ и другихъ; буква \es{ŭ} (которая употребляется только послѣ гласной), произносится какъ корокое \es{u} въ нѣмецкомъ словѣ \es{kaufen} или въ латинскомъ \es{laudo.} Лица, не знающія никакой другой азбуки кромѣ русской, могутъ на первыхъ порахъ произносить \es{h} какъ русское \br{х}, а \es{ŭ} какъ русское \br{у}.

\br{Примѣчаніе II.} Типографіи, не имѣющія буквъ \es{ĉ, ĝ, ĥ, ĵ, ŝ, ŭ}, могутъ на первыхъ порахъ употреблять \es{ch, gh, hh, jh, sh, u.}}
\begin{center}
{\large\gramsec{B) ЧАСТИ РѢЧИ}}
\end{center}

\br{1. Члена} неопредѣленнаго нѣтъ; есть только опредѣленный (\es{la}), одинаковый для всѣхъ родовъ, падежей и чиселъ.

{\footnotesize \br{Примѣчаніе.} Употребленіе члена такое же, какъ въ языкахъ нѣмецкомъ, французскомъ и другихъ. Лица, для которыхъ употребленіе члена представляетъ трудности, могутъ совершенно его не употреблять.}
    
    \br{2. Существительное} всегда оканчивается на \es{o}. Для образованія множественнаго числа прибавляется окончаніе \es{j}. Падежей есть только два: именительный и винительный; послѣдний получается изъ именительнаго прибавленіемъ окончанія \es{n}. Остальные падежи выражаются помощью предлоговъ: для родительнаго ― \es{de} (отъ), для дательнаго ― \es{al} (къ), для творительнаго ― \es{per} (посредствомъ) или другіе предлоги соотвѣтственно смыслу. (\br{Примѣры:} \es{patr\|o} отецъ, \es{al patr\|o} отцу, \es{patr\|o\|n} отца (винит. пад.), \es{por patr\|o\|j} для отцовъ, \es{patr\|o\|j\|n} отцовъ, (винит. пад.).
    
    \br{3. Прилагательное} всегда оканчивается на a. Падежи и числа какъ у существительнаго. Сравнительная степень образуется помощью слова \es{pli} (болѣе), а превосходная ― \es{plej} (наиболѣе); слово „чѣмъ“ переводится ol. (\br{Прим.:} \es{pli blank\|a ol neĝ\|o} бѣлѣе снѣга; \es{mi hav\|as la plej bon\|a\|n patr\|in\|o\|n} я имѣю самую лучшую мать).
    
    \br{4. Числительныя} количественныя (не склоняются): \es{unu} (1), \es{du} (2), \es{tri} (3), \es{kvar} (4), \es{kvin} (5) \es{ses} (6), \es{sep} (7), \es{ok} (8), \es{naŭ} (9), \es{dek} (10), \es{cent} (100), \es{mil} (1000). Десятки и сотни образуются простымъ сліяніемъ числительныхъ. Для образованія порядковыхъ прибавляется окончаніе прилагательнаго; для множительныхъ ― вставка \es{obl}, для дробныхъ ― \es{on}, для собирательныхъ ― \es{op}, для раздѣлительныхъ ― слово \es{po}. Кромѣ того могутъ быть числительныя существительныя и нарѣчныя. (\br{Примѣры:} \es{Kvin\|cent tri\|dek tri} = 533; \es{kvar\|a} четвертый; \es{unu\|o} единица; \es{du\|e} во вторыхъ; \es{tri\|obl\|a} тройной, \es{kvar\|on\|o} четверть; \es{du\|op\|e} вдвоемъ; \es{po kvin} по пяти).
    
    \br{5. Мѣстоименія} личныя: \es{mi} (я), \es{vi} (вы, ты), \es{li} (онъ), \es{ŝi} (она), \es{ĝi} (оно; о вещи или о животномъ), \es{si} (себя), \es{ni} (мы), \es{ili} (они, онѣ), \es{oni} (безличное множественнаго числа); притяжательныя образуются прибавленіемъ окончанія прилагательнаго. Склоненіе какъ у существительныхъ (\br{Примѣры:} \es{mi\|n} меня (винит.); \es{mi\|a} мой).
    
    \br{6. Глаголъ} по лицамъ и числамъ не измѣняется (наприм.: \es{mi far\|as} я дѣлаю, \es{la patr\|o far\|as} отецъ дѣлаетъ, \es{ili far\|as} они дѣлаютъ). Формы глагола:
    
        a) Настоящее время принимаетъ окончаніе \es{as} (напримѣръ: \es{mi far\|as} я дѣлаю).
        
        b) Прошедшее ― \es{is} (\es{li far\|is} онъ дѣлатъ).
        
        c) Будущее ― \es{os} (\es{ili far\|os} они будутъ дѣлать).
        
        ĉ) Условное наклоненіе ― \es{us} (\es{ŝi far\|us} она бы дѣлала).
        
        d) Повелительное наклоненіе ― \es{u} (\es{far\|u} дѣлайте).
        
        e) Неопредѣленное наклоненіе ― \es{i} (\es{far\|i} дѣлать).
        
\begin{center}
Причастія (и дѣепричастія):
\end{center}
    
        f) Дѣйствит. залога настоящаго времени ― \es{ant} (\es{far\|ant\|a} дѣлающій, \es{far\|ant\|e} дѣлая).
        
        g) Дѣйствит. залога прошедш. времени ― \es{int} (\es{far\|int\|a} сдѣлавшій).
        
        ĝ) Дѣйствит. залога будущ. времени ― \es{ont} (\es{far\|ont\|a} который сдѣлаетъ).
        
        h) Страдат. залога настоящ. времени ― \es{at} (\es{far\|at\|a} дѣлаемый).
        
        ĥ) Страдат. залога прошедш. времени ― \es{it} (\es{far\|it\|a} сдѣланный).
        
        i) Страдат. залога будущ. времени ― \es{ot} (\es{far\|ot\|a} имѣющій быть сдѣланнымъ).

    Всѣ формы страдательнаго залога образуются помощью соотвѣтственной формы глагола \es{est} (быть) и причастія страдательнаго залога даннаго глагола; предлогъ при этомъ употребляется de (\br{Примѣръ:} \es{ŝi est\|as am\|at\|a de ĉiu\|j} она любима всѣми).
    
    \br{7. Нарѣчія} оканчиваются на \es{e}. Степени сравненія какъ у прилагательныхъ (\br{Примѣръ:} \es{mi\|a frat\|o pli bon\|e kant\|as} ol mi мой братъ лучше меня поетъ).
    
    \br{8. Предлоги} всѣ требуютъ именительнаго падежа.

\begin{samepage} 
\begin{center}
{\large\gramsec{C) ОБЩІЯ ПРАВИЛА.}}
\end{center}

    \br{9.} Каждое слово читается такъ, какъ оно написано.
\end{samepage}

    \br{10.} Удареніе всегда находится на предпослѣднемъ слогѣ.
    
    \br{11.} Сложныя слова образуются простымъ сліяниемъ словъ (главное на концѣ), которыя пишутся вмѣстѣ, но отдѣляются другъ отъ друга черточкой\footnote{Въ письмахъ и сочиненіяхъ, назначенныхъ для лицъ, владѣющихъ уже международнымъ языкомъ, черточки между частями словъ не употребляются.}.) Грамматическія окончанія разсматриваются также какъ самостоятельныя слова (\br{Примѣръ:} \es{vapor\|ŝip\|o,} пароходъ ― изъ \es{vapor} паръ, \es{ŝip} корабль, \es{o} окончаніе существительныхъ).
    
    \br{12.} При другомъ отрицательномъ словѣ отрицаніе ne опускается (\br{Примѣръ:} \es{mi neniam vid\|is} я никогда не видалъ).
    
    \br{13.} На вопросъ „куда“ слова принимаютъ окончаніе винительнаго падежа (\br{Примѣры:} \es{tie} тамъ ― \es{tie\|n} туда; \es{Varsovi\|o\|n} въ Варшаву).
    
    \br{14.} Каждый предлогъ имѣтъ опредѣленное постоянное значеніе; если же нужно употребить предлогъ, а прямой смыслъ не указываетъ, какой именно, то употребляется предлогъ je, который самостоятельнаго значенія не имѣетъ (\br{Примѣры:} \es{ĝoj\|i je tio} радоваться этому; \es{rid\|i je tio} смѣяться надъ этимъ; \es{enu\|o je la patr\|uj\|o} тоска по родинѣ и т. д.).
    
    Ясность отъ этого не страдаетъ, потому что во всѣхъ языкахъ въ этихъ случаяхъ употребляется какой угодно предлогъ, лишь бы обычай далъ ему санкцію; въ междунатодномъ же языкѣ санкція на всѣ подобные случаи дана \br{одному} предлогу \es{je}.
    
    Вмѣсто предлога \es{je} можно также употребить винительный падежъ.
    
    \br{15.} Такъ называемыя „иностранныя“ слова, т.е. такія, которыя большинствомъ языковъ взяты изъ одного чужого источника, употребляются въ международномъ языкѣ безъ измѣненія, принимая только орѳографію этого языка; но при различныхъ словахъ одного корня лучше употреблять безъ измѣненія только основное слово, а другія образовать по правиламъ международнаго языка (\br{Примѣръ:} театръ ― \es{teatr\|o,} но театральный ― \es{teatr\|a}).
    
    \br{16.} Окончанія существительнаго и члена могутъ быть опущены и замѣнены апострофомъ (\br{Примѣры:} \es{dom’} вм. \es{dom\|o; de l’mond\|o} вм. \es{de la mond\|o.}
    
\newpage


% Gramatiko por poloj
%
% Esperanta gramatiko por poloj
%
\label{gram:pola}
\markboth{GRAMATYKA}{FUNDAMENTO DE ESPERANTO}
\thispagestyle{plain}
\begin{center}
\phantomsection
\narrow{\LARGE\bf \spaceoutless{GRAMATYKA}}
\addcontentsline{toc}{section}{Gramatyka (Gramatiko Pola)}
\selectlanguage{polish}

\rule{13mm}{0.4pt}
\vspace{2em}

{\large\gramsec{A) ABECADŁO}}
\vspace{1em}

\setstretch{1}
\begin{tabu} to \textwidth{+Y@{}ZY@{}ZY@{}ZY@{}ZY@{}ZY@{}ZY}
\rowstyle{\Large\arbfont} Aa, & Bb, & Cc, & Ĉĉ, & Dd, & Ee, & Ff, \\
\rowstyle{\footnotesize} a & b & c & cz & d & e & f \\[1ex]
\rowstyle{\Large\arbfont} Gg, & Ĝĝ, & Hh, & Ĥĥ, & Ii, & Jj, & Ĵĵ, \\
\rowstyle{\footnotesize} g &  dż & h & ch & i & j & ż \\[1ex]
\rowstyle{\Large\arbfont} Kk, & Ll, & Mm, & Nn, & Oo, & Pp, & Rr, \\
\rowstyle{\footnotesize} k & l & m & n & o & p & r \\[1ex]
\rowstyle{\Large\arbfont} Ss, & Ŝŝ, & Tt, & Uu, & Ŭŭ, & Vv, & Zz. \\
\rowstyle{\footnotesize} s & sz & t & u & u (krótkie) & w & z \\
\end{tabu}
\end{center}

{\footnotesize {\so{\textbf{UWAGA.}}― Drukarnia, nie posiadająca czcionek ze znaczkami, może zamiast Ĉ, Ĝ, Ĥ, Ĵ, Ŝ, Ŭ, drukować ch, gh, hh, jh, sh, u.}}
\begin{center}
\large \gramsec{B) CZĘSCI MOWY}
\end{center}

\textbf{1. Przedimka} nieokreślnego niema; jest tylko określny \textbf{la}, wspólny dla wszystkich rodzajów, przypadków i liczb.

\textbf{2. Rzeczownik} kończy się zawsze na \textbf{o}. Dla utworzenia liczby mnogiéj dodaje się końcówka \textbf{j}. Przypadków jest dwa: mianownik (nominativus) i biernik (accusativus); ten ostatni powstaje z mianownika przez dodanie zakończenia \textbf{n}. Resztę przypadków oddaje się za pomocą przyimków: dla dopełniacza (genitivus) ― \textbf{de} (od), dla celownika (dativus) ― \textbf{al} (do), dla narzędnika (instrumentalis) ― \textbf{per} (przez), lub inne przyimki odpowiednio do znaczenia. \textbf{Przykłady:} \textbf{patr/o} ojciec, \textbf{al patr/o} ojcu, \textbf{patr/o/n} ojca (przypadek czwarty), \textbf{por patr/o/j} dla ojców, \textbf{patro/j/n} ojców (przyp. czwarty).

\textbf{3. Przymiotnik} zawsze kończy się na \textbf{a}. Przypadki i liczby też same co dla rzeczownika. Stopień wyższy tworzy się przez dodanie wyrazu \textbf{pli} (więcéj), a najwyższy przez dodanie \textbf{plej} (najwięcéj); wyraz „niż“ tłomaczy się przez \textbf{ol}. \textbf{Przykład: Pli blank/a ol neĝ/o} bielszy od śniegu.

\textbf{4. Liczebniki} główne nie odmieniają się: \textbf{unu} (1), \textbf{du} (2), \textbf{tri} (3), \textbf{kvar} (4), \textbf{kvin} (5), \textbf{ses} (6), \textbf{sep} (7), \textbf{ok} (8), \textbf{naŭ} (9), \textbf{dek} (10), \textbf{cent} (100), \textbf{mil} (1000). Dziesiątki i setki tworzą się przez proste połączenie liczebników. Dla utworzenia liczebników porządkowych dodaje się końcówka przymiotnika, dla wielorakich ― przyrostek \textbf{obl}, dla ułamkowych ― \textbf{on}, dla zbiorowych ― \textbf{op}, dla podziałowych ― wyraz \textbf{po}. Prócz tego mogą być liczebniki rzeczowne i przysłówkowe. \textbf{Przykłady: kvin/cent tri/dek tri} = 533; \textbf{kvar/a} czwarty; \textbf{unu/o} jednostka; \textbf{du/e} powtóre; \textbf{tri/obl/a} potrójny, trojaki; \textbf{kvar/on/o} czwarta część; \textbf{du/op/e} we dwoje; \textbf{po kvin} po pięć.

\textbf{5. Zaimki} osobiste: \textbf{mi} (ja), \textbf{vi} (wy, ty) \textbf{li} (on), \textbf{ŝi} (ona), \textbf{ĝi} (ono; o rzeczy lub zwierzęciu), \textbf{si} (siebie) \textbf{ni} (my), \textbf{ili} (oni, one), \textbf{oni} (zaimek nieosobisty liczby mnogiéj); dzierżawcze tworzą się przez dodanie końcówki przymiotnika. Zaimki odmieniają się jak rzeczowniki. \textbf{Przykłady:} \textbf{mi/n} mnie (przyp. czwarty); \textbf{mi/a} mój.

\textbf{6. Słowo} nie odmienia się przez osoby i liczby. Np. \textbf{mi far/as} ja czynię, \textbf{la patr/o far/as} ojciec czyni, \textbf{ili far/as} oni czynią. Formy słowa:

a) Czas teraźniejszy ma zakończenie \textbf{as}. (Przykład: \textbf{mi far/as} ja czynię).

b) Czas przeszły ― \textbf{is} (\textbf{li far/is} on czynił).

c) Czas przyszły ― \textbf{os} (\textbf{ili far/os} oni będą czynili).

ĉ) Tryb warunkowy ― \textbf{us} (\textbf{ŝi far/us} ona by czynila).

d) Tryb rozliazujący ― \textbf{u} (\textbf{far/u} czyń, czyńcie).

e) Tryb bezokoliczny ― \textbf{i} (\textbf{far/i} czynić).

Imiesłowy (odmienne i nieodmienne):

f) Imiesłów czynny czasu teraźniejszego ― \textbf{ant} (\textbf{far/ant/a} czyniący, \textbf{far/ant/e} czyniąc).

g) Imiesłów czynny czasu przeszłego ― \textbf{int} (\textbf{far/int/a} który uczynił).

ĝ) Imiesłów czynny czasu przyszłego ― \textbf{ont} (\textbf{far/ont/a} który uczyni).

h) Imiesłów bierny czasu teraźn. ― \textbf{at} (\textbf{far/at/a} czyniony).

ĥ) Imiesłów bierny czasu przeszłego ― \textbf{it} (\textbf{far/it/a} uczyniony).

i) Imiesłów bierny czasu przyszłego ― \textbf{ot} (\textbf{far/ot/a} mający być uczynionym).

Wszystkie formy strony biernéj tworzą się zapomocą odpowiedniéj formy słowa \textbf{est} być i imiesłow biernego danego słowa; używa się przytem przyimka \textbf{de} (np. ŝi est/as am/at/a de ĉiu/j ― ona kochana jest przez wszystkich).

\textbf{7. Przysłówki} mają zakończenie \textbf{e}. Stopniowanie podobnem jest do stopniowania przymiotników (np. \textbf{mi/a frat/o pli bon/e kant/as ol mi} ― brat mój lepiej śpiewa odemnie).

\textbf{8. Przyimki} rządzą wszystkie przypadkiem pierwszym.

\begin{center}
\large \gramsec{C) PRAWIDŁA OGOLNE}
\end{center}

\textbf{9.} Każdy wyraz tak się czyta, jak się pisze.

\textbf{10.} Akcent pada zawsze na przedostatnią zgłoskę.

\textbf{11.} Wyrazy złożone tworzą się przez proste połączenie wyrazów (główny na końcu). \textbf{Przykład: vapor/ŝip/o,} parostatek ― z \textbf{vapor} para, \textbf{ŝip} okręt, \textbf{o} ― końcówka rzeczownika.

\textbf{12.} Przy innym przeczącym wyrazie opuszcza się przysłówek przeczący \textbf{ne} (np. \textbf{mi neniam vid/is} nigdy nie widziałem).

\textbf{13.} Na pytanie „dokąd“ wyrazy przybierają końcówkę przypadku czwartego (np. \textbf{tie} tam (w tamtem miejscu) ― \textbf{tie/n} tam (do tamtego miejsca); \textbf{Varsovi/o/n} (do Warszawy).

\textbf{14.} Każdy przyimek ma określone, stałe znaczenie; jeżeli należy użyć przyimka w wypadkach, gdzie wybór jego nie wypływa z natury rzeczy, używany bywa przyimek \textbf{je}, który nie ma samoistnego znaczenia (np. \textbf{ĝoj/i je tio} cieszyć się \textbf{z} tego; \textbf{mal/san/a je la okul/o/j} chory \textbf{na} oczy; \textbf{enu/o je la patr/uj/o} tęsknota \textbf{za} ojczyzną i t. p. Jasność języka wcale wskutek tego nie szwankuje, albowiem w tym razie wszystkie języki używaja jakiegokolwiek przyimka, byle go tylko zwyczaj uświęcił; w języku zaś międzynarodowym sankcja we wszystkich podobnych wypadkach nadaną została \textbf{jednemu} tylko przyimkowi \textbf{je}). Zamiast przmyika\eraro{przyimka} \textbf{je} używać też można, przypadku czwartego bez przyimka tam, gdzie nie zachodzi obawa dwuznaczności.

\textbf{15.} Tak zwane wyrazy „cudzoziemskie“ t. j. takie, które większość języków przyjęła z jednego obcego źródła, nie ulegają w języku międzynarodowym żadnéj zmianie, lecz otrzymują tylko pisownię międzynarodową; przy rozmaitych wszakże wyrazach jednego źródłosłowu, lepiéj używać bez zmiany tylko wyrazu pierwotnego, a inne tworzyć według prawideł języka międzynarodowego (np. \textbf{teatr/o} ― teatr, lecz teatralny ― \textbf{teatr/a}).

\textbf{16.} Końcówkę rzeczownika i przedimka można opuścić i zastąpić apostrofem (np. \textbf{Ŝiller’} zam. \textbf{Ŝiller/o}; \textbf{de l’ mond/o} zamiast \textbf{de la mond/o}).

\cleardoublepage


% Ekzercaro
%
% Ekzercaro
%
\label{ekzercaro}

\thispagestyle{plain}
\vspace*{1em}
\begin{center}
\narrow{\huge\bf \spaceoutless{EKZERCARO}}
\vspace{1em}

\bookman{\Large de la lingvo internacia «~Esperanto~»}

\rule{0.9\textwidth}{0.4pt}
\vspace{2em}
\end{center}

\ekzsec{§1.}

\begin{center}
{\large \gramsec{ALFABETO}}

\vspace{1em}

\setstretch{1}
\begin{tabularx}{\textwidth}{+Y@{}ZY@{}ZY@{}ZY@{}ZY@{}ZY@{}ZY}
\rowstyle{\Large\arbfont} Aa, & Bb, & Cc, & Ĉĉ, & Dd, & Ee, & Ff, \\[1ex]
\rowstyle{\Large\arbfont} Gg, & Ĝĝ, & Hh, & Ĥĥ, & Ii, & Jj, & Ĵĵ, \\[1ex]
\rowstyle{\Large\arbfont} Kk, & Ll, & Mm, & Nn, & Oo, & Pp, & Rr, \\[1ex]
\rowstyle{\Large\arbfont} Ss, & Ŝŝ, & Tt, & Uu, & Ŭŭ, & Vv, & Zz. \\[1ex]

%\rowstyle{\huge\curve} A a, & B b, & C c, & Ĉ ĉ, & D d, & E e, & F f, \\[1ex]
%\rowstyle{\huge\curve} G g, & Ĝ ĝ, & H h, & Ĥ ĥ, & I i, & J j, & Ĵ ĵ, \\[1ex]
%\rowstyle{\huge\curve} K k, & L l, & M m, & N n, & O o, & P p, & R r, \\[1ex]
%\rowstyle{\huge\curve} S s, & Ŝ ŝ, & T t, & U u, & Ŭ ŭ, & V v, & Z z. \\[1ex]

\rowstyle{\huge\curve} A a, & B b, & C c, & \scriptC{} \scriptc{}, & D d, & E e, & F f, \\[1ex]
\rowstyle{\huge\curve} G g, & \scriptG{} \scriptg{}, & H h, & \scriptH{} \scripth{}, & I i, & J j, & \scriptJ{} \scriptj{}, \\[1ex]
\rowstyle{\huge\curve} K k, & L l, & M m, & N n, & O o, & P p, & R r, \\[1ex]
\rowstyle{\huge\curve} S s, & \scriptS{} \scripts{}, & T t, & U u, & \scriptU{} \scriptu{}, & V v, & Z z. \\[1ex]
\end{tabularx}
\end{center}

\textbf{Nomoj de la literoj:} a, bo, co, ĉo, do, e, fo, go, ĝo, ho, ĥo, i, jo, ĵo, ko, lo, mo, no, o, po, ro, so, ŝo, to, u, ŭo, vo, zo.

\begin{center}
\narrow{\large §2.}

{\bf Ekzerco de legado.}
\end{center}

Al. Bá-lo. Pát-ro. Nú-bo. Cé-lo. Ci-tró-no. Cén-to. Sén-to. Scé-no. Scí-o. Có-lo. Kó-lo. O-fi-cí-ro. Fa-cí-la. Lá-ca. Pa-cú-lo. Ĉar. Ĉe-mí-zo. Ĉi-ká-no. Ĉi-é-lo. Ĉu. Fe-lí-ĉa. Cí-a. Ĉí-a. Pro-cé-so. Sen-ĉé-sa Ec. Eĉ. Ek. Da. Lú-do. Dén-to. Plén-di. El. En. De. Té-ni. Sen. Vé-ro. Fá-li. Fi-dé-la. Trá-fi. Gá-lo. Grán-da. Gén-to. Gíp-so. Gús-to. Lé-gi. Pá-go. Pá-ĝo. Ĝis. Ĝús-ta. Ré-ĝi. Ĝar-dé-no. Lón-ga. Rég-no. Síg-ni. Gvar-dí-o. Lín-gvo. Ĝu-á-do. Há-ro. Hi-rún-do. Há-ki. Ne-hé-la. Pac-hó-ro. Ses-hó-ra Bat-hú-fo. Hó-ro. Ĥó-ro. Kó-ro. Ĥo-lé-ro. Ĥe-mí-o. I-mí-ti. Fí-lo. Bír-do. Tró-vi. Prin-tém-po. Min. Fo-í-ro. Fe-í-no. I-el. I-am. In. Jam. Ju. Jes. Ju-ris-to. Kra-jó-no. Ma-jés-ta. Tuj. Dó-moj. Ru-í-no. Prúj-no. Ba-lá-i. Pá-laj. De-í-no. Véj-no. Pe-ré-i. Mál-plej. Jús-ta. Ĵus. Ĵé-ti. Ĵa-lú-za. Ĵur-nálo. Má-jo. Bo-ná-ĵo. Ká-po. Ma-kú-lo. Kés-to. Su-ké-ro. Ak-vo. Ko-ké-to. Li-kvó-ro. Pac-ká-po.

\begin{center}
\narrow{\large §3.}

{\bf Ekzerco de legado.}
\end{center}

Lá-vi. Le-ví-lo. Pa-ró-li. Mem. Im-plí-ki. Em-ba-rá-so. Nó-mo. In-di-fe-rén-ta. In-ter-na-cí-a. Ol. He-ró-i. He-ro-í-no. Fój-no. Pí-a. Pál-pi. Ri-pé-ti. Ar-bá-ro. Sá-ma. Stá-ri. Si-gé-lo. Sis-té-mo. Pe-sí-lo. Pe-zí-lo. Sén-ti. So-fís-mo. Ci-pré-so. Ŝi. Pá-ŝo. Stá-lo. Ŝtá-lo. Vés-to. Véŝ-to. Dis-ŝí-ri. Ŝan-cé-li. Ta-pí-ŝo Te-o-rí-o. Pa-tén-to. U-tí-la. Un-go. Plú-mo. Tu-múl-to. Plu. Lú-i. Kí-u. Ba-lá-u. Tra-ú-lo. Pe-ré-u. Ne-ú-lo. Fráŭ-lo. Paŭ-lí-no. Láŭ-di. Eŭ-ró-po. Tro-ú-zi. Ho-dí-aŭ. Vá-na. Vér-so. Sól-vi. Zór-gi. Ze-ní-to. Zo-o-lo-gí-o. A-zé-no. Me-zú-ro. Ná-zo. Tre-zó-ro. Mez-nók-to. Zú-mo. Sú-mo. Zó-no. Só-no. Pé-zo. Pé-co. Pé-so. Ne-ní-o. A-dí-aŭ. Fi-zí-ko. Ge-o-gra-fí-o. Spi-rí-to. Lip-há-ro. In-díg-ni. Ne-ní-el. Spe-gú-lo. Ŝpí-no. Né-i. Ré-e. He-ró-o. Kon-scí-i. Tra-e-té-ra. He-ro-é-to. Lú-e. Mó-le. Pá-le. Tra-í-re. Pa-sí-e. Me-tí-o. In-ĝe-ni-é-ro. In-sék-to. Re-sér-vi. Re-zér-vi.

\begin{center}
\narrow{\large §4.\par}

\nopagebreak

{\bf Ekzerco de legado.}
\end{center}

Citrono. Cento. Sceno. Scio. Balau. Ŝanceli. Neniel. Embaraso. Zoologio. Reservi. Traire. Hodiaŭ. Disŝiri. Neulo. Majesta. Packapo. Heroino. Pezo. Internacia. Seshora. Cipreso. Stalo. Feino. Plu. Sukero. Gento. Indigni. Sigelo. Krajono. Ruino. Pesilo. Lipharo. Metio. Ĝardeno. Sono. Laŭdi. Pale. Facila. Insekto. Kiu. Zorgi. Cikano. Traetera. Sofismo. Domoj. Spino. Majo. Signi. Ec. Bonaĵo. Legi. Iel. Juristo. Ĉielo. Ĥemio.

\ekzsec{§5.}

Patro kaj frato. ― Leono estas besto. ― Rozo estas floro kaj kolombo estas birdo. ― La rozo apartenas al Teodoro. ― La suno brilas. ― La patro estas sana. ― La patro estas tajloro.

\begin{ekzvocab}{1em}
\uventry{patro} père | father | Vater | отецъ | ojciec.

\uventry{o} marque le substantif | ending of nouns (substantive) | bezeichnet das Substantiv | означаетъ существительное | oznacza rzeczownik.

\uventry{kaj} et | and | und | и | i.

\uventry{frato} frère | brother | Bruder | братъ | brat.

\uventry{leono} lion | lion | Löwe | левъ | lew.

\uventry{esti} être | be | sein | быть | być.

\uventry{as} marque le présent d’un verbe | ending of the present tense in verbs | bezeichnet das Präsens | означаетъ настоящее время глагола | oznacza czas teraźniejszy.

\uventry{besto} animal | beast | Thier | животное | zwierzę.

\uventry{rozo} rose | rose | Rose | роза | róża.

\uventry{floro} fleur | flouwer | Blume | цвѣтъ, цвѣтокъ | kwiat.

\uventry{kolombo} pigeon | dove | Taube | голубь | gołąb’.

\uventry{birdo} oiseau | bird | Vogel | птица | ptak.

\uventry{la} article défini (le, la, les) | the | bestimmter Artikel (der, die, das) | членъ опредѣленный (по русски не переводится) | przedimek określny (nie tłómaczy się).

\uventry{aparteni} appartenir | belong | gehören | принадлежать | należeć.

\uventry{al} à | to | zu (ersetzt zugleich den Dativ) | къ (замѣняетъ также дательный падежъ) | do (zastępuje też przypadek trzeci).

\uventry{suno} soleil | sun | Sonne | солнце | słońce.

\uventry{brili} briller | shine | glänzen | блистать | błyszczeć.

\uventry{sana} sain, en santé | well, healthy | gesund | здоровый | zdrowy.

\uventry{a} marque l’adjectif | termination of adjectives | bezeichnet das Adjektiv | означаетъ прилагательное | oznacza przymiotnik.

\uventry{tajloro} tailleur | tailor | Schneider | портной | krawiec.

\end{ekzvocab}


\ekzsec{§6.}

Infano ne estas matura homo. ― La infano jam ne ploras. ― La ĉielo estas blua. ― Kie estas la libro kaj la krajono? ― La libro estas sur la tablo, kaj la krajono kuŝas sur la fenestro. ― Sur la fenestro kuŝas krajono kaj plumo. ― Jen estas pomo. ― Jen estas la pomo, kiun mi trovis. ― Sur la tero kuŝas ŝtono.

\begin{ekzvocab}{1em}
\uventry{infano} enfant | child | Kind | дитя | dziecię.

\uventry{ne} non, ne, ne\ldots{} pas | no, not | nicht, nein | не, нѣтъ | nie.

\uventry{matura} mûr | mature, ripe | reif | зрѣлый | dojrzały.

\uventry{homo} homme | man | Mensch | человѣкъ | człowiek.

\uventry{jam} déjà | already | schon | уже | juź.

\uventry{plori} pleurer | mourn, weep | weinen | плакать | płakać.

\uventry{ĉielo} ciel | heaven | Himmel | небо | niebo.

\uventry{blua} bleu | blue | blau | синій | niebieski.

\uventry{kie} où | where | wo | гдѣ | gdzie.

\uventry{libro} livre | book | Buch | книга | księga, książka.

\uventry{krajono} crayon | pencil | Bleistift | карандашъ | ołówek.

\uventry{sur} sur | upon, on | auf | на | na.

\uventry{tablo} table | table | Tisch | столъ | stół.

\uventry{kuŝi} être couché | lie (down) | liegen | лежать | leżeć.

\uventry{fenestro} fenêtre | window | Fenster | окно | okno.

\uventry{plumo} plume | pen | Feder | перо | pióro.

\uventry{jen} voici, voilà | behold, lo | da, siehe | вотъ | otóż.

\uventry{pomo} pomme | apple | Apfel | яблоко | jabłko.

\uventry{kiu} qui lequel, laquelle | who, which | wer, welcher | кто, который | kto, który.

\uventry{n} marque l’accusatif ou complément direct | ending of the objective | bezeichnet den Accusativ | означаетъ винительный падежъ | oznacza przypadek czwarty.

\uventry{mi} je, moi | I | ich | я | ja.

\uventry{trovi} trouver | find | finden | находить | znajdować.

\uventry{is} marque le passé | ending of past tense in verbs | bezeichnet die vergangene Zeit | означаетъ прошедшее время | oznaczca czas przeszły.

\uventry{tero} terre | earth | Erde | земля | ziemia.

\uventry{ŝtono} pierre | stone | Stein | камень | kamień.

\end{ekzvocab}


\ekzsec{§7.}

Leono estas forta. ― La dentoj de leono estas akraj. ― Al leono ne donu la manon. ― Mi vidas leonon. ― Resti kun leono estas danĝere. ― Kiu kuraĝas rajdi sur leono? ― Mi parolas pri leono.

\begin{ekzvocab}{1em}
\uventry{forta} fort | strong | stark, kräftig | сильный | silny, mocny.

\uventry{dento} dent | tooth | Zahn | зубъ | ząb.

\uventry{j} marque le pluriel | sign of the plural | bezeichnet die Mehrzahl | означаетъ множественное число | oznacza liczbę mnogą.

\uventry{de} de | of, from | von; ersetzt auch den Genitiv | отъ; замѣняетъ также родительный падежъ | od; zastępuje też przypadek drugi.

\uventry{akra} aigu | sharp | scharf | острый | ostry.

\uventry{doni} donner | give | geben | давать | dawać.

\uventry{u} marque l’impératif | ending of the imperative in verbs | bezeichnet den Imperativ | означаетъ повелительное наклоненіе | oznacza tryb rozkazujący.

\uventry{mano} main | hand | Hand | рука | ręka.

\uventry{vidi} voir | see | sehen | видѣть | widzieć.

\uventry{resti} rester | remain | bleiben | оставаться | pozostawać.

\uventry{kun} avec | with | mit | съ | z.

\uventry{danĝero} danger | danger | Gefahr | опасность | niebezpieczeństwo.

\uventry{e} marque l’adverbe | ending of adverbs | Endung des Adverbs | окончаніе нарѣчія | zakończenie przysłówka.

\uventry{kuraĝa} courageux | courageous, daring | kühn, dreist | смѣлый | śmiały.

\uventry{rajdi} aller à cheval | ride | reiten | ѣздить верхомъ | jeździć konno.

\uventry{i} marque l’infinitif | termination of the infinitive in verbs | bezeichnet den Infinitiv | означаетъ неопредѣленное наклоненіе | oznacza tryb bezokoliczny słowa.

\uventry{paroli} parler | speak | sprechen | говорить | mówić.

\uventry{pri} sur, touchant, de | concerning, about | von, über | о, объ | o.

\end{ekzvocab}

\begin{samepage}
\ekzsec{§8.}

La patro estas bona. ― Jen kuŝas la ĉapelo de la patro. ― Diru al la patro, ke mi estas diligenta. ― Mi amas la patron. ― Venu kune kun la patro. ― La filo staras apud la patro. ― La mano de Johano estas pura. ― Mi konas Johanon. ― Ludoviko, donu al mi panon. ― Mi manĝas per la buŝo kaj flaras per la nazo. ― Antaŭ la domo staras arbo. ― La patro estas en la ĉambro.
\end{samepage}

\begin{ekzvocab}{1em}
\uventry{bona} bon | good | gut | добрый | dobry.

\uventry{ĉapelo} chapeau | hat | Hut | шляпа | kapelusz.

\uventry{diri} dire | say | sagen | сказать | powiadać.

\uventry{ke} que | that (conj.) | dass | что | że.

\uventry{diligenta} diligent, assidu | diligent | fleissig | прилежный | pilny.

\uventry{ami} aimer | love | lieben | любить | lubić, kochać.

\uventry{veni} venir | come | kommen | приходить | przychodzić.

\uventry{kune} ensemble | together | zusammen | вмѣстѣ | razem, wraz.

\uventry{filo} fils | son | Sohn | сынъ | syn.

\uventry{stari} être debout | stand | stehen | стоять | stać.

\uventry{apud} auprès de | near by | neben, an | при, возлѣ | przy, obok.

\uventry{pura} pur, propre | clean, pure | rein | чистый | czysty.

\uventry{koni} connaître | know, recognise | kennen | знать (быть знакомымъ) | znać.

\uventry{pano} pain | bread | Brot | хлѣбъ | chleb.

\uventry{manĝi} manger | eat | essen | ѣсть | jeść.

\uventry{per} par, au moyen de | through, by means of | mittelst, vermittelst, durch | посредствомъ | przez, za pomocą.

\uventry{buŝo} bouche | mouth | Mund | ротъ | usta.

\uventry{flari} flairer, sentir | smell | riechen, schnupfen | нюхать, обонять | wąchać.

\uventry{nazo} nez | nose | Nase | носъ | nos.

\uventry{antaŭ} devant | before | vor | предъ | przed.

\uventry{domo} maison | house | Haus | домъ | dom.

\uventry{arbo} arbre | tree | Baum | дерево | drzewo.

\uventry{ĉambro} chambre | room | Zimmer | комната | pokój.

\end{ekzvocab}


\ekzsec{§9.}

La birdoj flugas. ― La kanto de la birdoj estas agrabla. ― Donu al la birdoj akvon, ĉar ili volas trinki. ― La knabo forpelis la birdojn. ― Ni vidas per la okuloj kaj aŭdas per la oreloj. ― Bonaj infanoj lernas diligente. ― Aleksandro ne volas lerni, kaj tial mi batas Aleksandron. ― De la patro mi ricevis libron, kaj de la frato mi ricevis plumon. ― Mi venas de la avo, kaj mi iras nun al la onklo. ― Mi legas libron. ― La patro ne legas libron, sed li skribas leteron.

\begin{ekzvocab}{1em}
\uventry{flugi} voler (avec des ailes) | fly (vb.) | fliegen | летать | latać.

\uventry{kanti} chanter | sing | singen | пѣть | śpiewać.

\uventry{agrabla} agréable | agreeable | angenehm | пріятный | przyjemny.

\uventry{akvo} eau | water | Wasser | вода | woda.

\uventry{ĉar} car, parce que | for | weil, da, denn | ибо, такъ какъ | albowiem, ponieważ.

\uventry{ili} ils, elles | they | sie (Mehrzahl) | они, онѣ | oni, one.

\uventry{voli} vouloir | wish, will | wollen | хотѣть | chcieć.

\uventry{trinki} boire | drink | trinken | пить | pić.

\uventry{knabo} garçon | boy | Knabe | мальчикъ | chłopiec.

\uventry{for} loin, hors | forth, out | fort | прочь | precz.

\uventry{peli} chasser, renvoyer | pursue, chase out | jagen, treiben | гнать | gonić.

\uventry{ni} nous | we | wir | мы | my.

\uventry{okulo} œil | eye | Auge | глазъ | oko.

\uventry{aŭdi} entendre | hear | hören | слышать | słyszeć.

\uventry{orelo} oreille | ear | Ohr | ухо | ucho.

\uventry{lerni} apprendre | learn | lernen | учиться | uczyć się.

\uventry{tial} c’est pourquoi | therefore | darum, deshalb | потому | dla tego.

\uventry{bati} battre | beat | schlagen | бить | bić.

\uventry{ricevi} recevoir, obtenir | obtain, get, receive | bekommen, erhalten | получать | otrzymywać.

\uventry{avo} grand-père | grandfather | Grossvater | дѣдъ, дѣдушка | dziad, dziadek.

\uventry{iri} aller | go | gehen | идти | iść.

\uventry{nun} maintenant | now | jetzt | теперь | teraz.

\uventry{onklo} oncle | uncle | Onkel | дядя | wuj, stryj.

\uventry{legi} lire | read | lesen | читать | czytać.

\uventry{sed} mais | but | aber, sondern | но, а | lecz.

\uventry{li} il, lui | he | er | онъ | on.

\uventry{skribi} écrire | write | schreiben | писать | pisać.

\uventry{letero} lettre, épître | letter | Brief | письмо | list.

\end{ekzvocab}


\ekzsec{§10.}

Papero estas blanka. ― Blanka papero kuŝas sur la tablo. ― La blanka papero jam ne kuŝas sur la tablo. ― Jen estas la kajero de la juna fraŭlino. ― La patro donis al mi dolĉan pomon. ― Rakontu al mia juna amiko belan historion. ― Mi ne amas obstinajn homojn. ― Mi deziras al vi bonan tagon, sinjoro! ― Bonan matenon! ― Ĝojan feston! (mi deziras al vi). ― Kia ĝoja festo! (estas hodiaŭ). ― Sur la ĉielo staras la bela suno. ― En la tago ni vidas la helan sunon, kaj en la nokto ni vidas la palan lunon kaj la belajn stelojn. ― La papero estas tre blanka, sed la neĝo estas pli blanka. ― Lakto estas pli nutra, ol vino. ― Mi havas pli freŝan panon, ol vi. ― Ne, vi eraras, sinjoro: via pano estas malpli freŝa, ol mia. ― El ĉiuj miaj infanoj Ernesto estas la plej juna. ― Mi estas tiel forta, kiel vi. ― El ĉiuj siaj fratoj Antono estas la malplej saĝa.

\begin{ekzvocab}{1em}
\uventry{papero} papier | paper | Papier | бумага | papier.

\uventry{blanka} blanc | white | weiss | бѣлый | biały.

\uventry{kajero} cahier | copy book | Heft | тетрадь | kajet.

\uventry{juna} jeune | young | jung | молодой | młody.

\uventry{fraŭlo} homme non marié | bachelor | unverheiratheter Herr | холостой господинъ | kawaler.

\uventry{in} marque le féminin; ex.: \uventry{patro} père ― \uventry{patrino} mère | ending of feminine words; e.~g. \uventry{patro} father ― \uventry{patrino} mother | bezeichnet das weibliche Geschlecht; z.~B. \uventry{patro} Vater ― \uventry{patrino} Mutter; \uventry{fianĉo} Bräutigam ― \uventry{fianĉino} Braut | означаетъ женскій полъ; напр. \uventry{patro} отецъ ― patrino \uventry{мать}; \uventry{fianĉo} женихъ ― \uventry{fianĉino} невѣста | oznacza płeć żeńską; np. \uventry{patro} ojciec ― \uventry{patrino} matka; \uventry{koko} kogut ― \uventry{kokino} kura.

\uventry{(fraŭlino} demoiselle, mademoiselle | miss | Fräulein | барышня | panna.)

\uventry{dolĉa} doux | sweet | süss | сладкій | słodki.

\uventry{rakonti} raconter | tell, relate | erzählen | разсказывать | opowiadać.

\uventry{mia} mon | my | mein | мой | mój.

\uventry{amiko} ami | friend | Freund | другъ | przyjaciel.

\uventry{bela} beau | beautiful | schön, hübsch | красивый, прекрасный | piękny, ładny.

\uventry{historio} histoire | history, story | Geschichte | исторія | historja.

\uventry{obstina} entêté, obstiné | obstinate | eigensinnig | упрямый | uparty.

\uventry{deziri} désirer | desire | wünschen | желать | życzyć.

\uventry{vi} vous, toi, tu | you | Ihr, du, Sie | вы, ты | wy, ty.

\uventry{tago} jour | day | Tag | день | dzień.

\uventry{sinjoro} monsieur | Sir, Mr | Herr | господинъ | pan.

\uventry{mateno} matin | morning | Morgen | утро | poranek.

\uventry{ĝoji} se réjouir | rejoice | sich freuen | радоваться | cieszyć się.

\uventry{festi} fêter | feast | feiern | праздновать | świętować.

\uventry{kia} quel | of what kind, what a | was für ein, welcher | какой | jaki.

\uventry{hodiaŭ} aujourd’hui | to-day | heute | сегодня | dziś.

\uventry{en} en, dans | in | in, ein- | въ | w.

\uventry{hela} clair (qui n’est pas obscur) | clear, glaring | hell, grell | яркій | jasny, jaskrawy.

\uventry{nokto} nuit | night | Nacht | ночь | noc.

\uventry{pala} pâle | pale | bleich, blass | блѣдный | blady.

\uventry{luno} lune | moon | Mond | луна | księźyć.

\uventry{stelo} étoile | star | Stern | звѣзда | gwiazda.

\uventry{neĝo} neige | snow | Schnee | снѣгъ | śnieg.

\uventry{pli} plus | more | mehr | болѣе, больше | więcej.

\uventry{lakto} lait | milk | Milch | молоко | mleko.

\uventry{nutri} nourrir | nourish | nähren | питать | karmić, pożywiać.

\uventry{ol} que (dans une comparaison) | than | als | чѣмъ | niź.

\uventry{vino} vin | vine | Wein | вино | wino.

\uventry{havi} avoir | have | haben | имѣть | mieć.

\uventry{freŝa} frais, récent | fresh | frisch | свѣжій | świeźy.

\uventry{erari} errer | err, mistake | irren | ошибаться, блуждать | błądzić, mylić się.

\uventry{mal} marque les contraires; ex. \uventry{bona} bon ― \uventry{malbona} mauvais; \uventry{estimi} estimer ― \uventry{malestimi} mépriser | denotes opposites; e.~g. \uventry{bona} good ― \uventry{malbona} evil; \uventry{estimi} esteem ― \uventry{malestimi} despise | bezeichnet einen geraden Gegensatz; z.~B. \uventry{bona} gut ― \uventry{malbona} schlecht; \uventry{estimi} schätzen ― \uventry{malestimi} verachten | прямо противоположно; напр. \uventry{bona} хорошій ― \uventry{malbona} дурной; \uventry{estimi} уважать ― \uventry{malestimi} презирать | oznacza preciwieństwo; np. \uventry{bona} dobry ― \uventry{malbona} zły; \uventry{estimi} poważać ― \uventry{malestimi} gardzić.

\uventry{el} de, d’entre, é-, ex- | from, out from | aus | изъ | z.

\uventry{ĉiu} chacun | each, every one | jedermann | всякій, каждый | wszystek, każdy.

\uventry{(ĉiuj} tous | all | alle | всѣ | wszyscy.)

\uventry{plej} le plus | most | am meisten | наиболѣе | najwięcej.

\uventry{tiel} ainsi, de cette manière | thus, so | so | такъ | tak.

\uventry{kiel} comment | how, as | wie | какъ | jak.

\uventry{si} soi, se | one’s self | sich | себя | siebie.

\uventry{(sia} son, sa | one’s | sein | свой | swój.)

\uventry{saĝa} sage, sensé | wise | klug, vernünftig | умный | mądry.

\end{ekzvocab}

\begin{samepage}
\begin{center}
\narrow{\large §11.}
\vskip 1ex
\Large\bookman{La feino.}
\end{center}

Unu vidvino havis du filinojn. La pli maljuna estis tiel simila al la patrino per sia karaktero kaj vizaĝo, ke ĉiu, kiu ŝin vidis, povis pensi, ke li vidas la patrinon; ili ambaŭ estis tiel malagrablaj kaj tiel fieraj, ke oni ne povis vivi kun ili. La pli juna filino, kiu estis la plena portreto de sia patro laŭ sia boneco kaj honesteco, estis krom tio unu el la plej belaj knabinoj, kiujn oni povis trovi.
\end{samepage}

\begin{ekzvocab}{1em}
\uventry{feino} fée | fairy | Fee | фея | wieszczka.

\uventry{unu} un | one | ein, eins | одинъ | jeden.

\uventry{vidvo} veuf | widower | Wittwer | вдовецъ | wdowiec.

\uventry{du} deux | two | zwei | два | dwa.

\uventry{simila} semblable | like, similar | ähnlich | похожій | podobny.

\uventry{karaktero} caractère | character | Charakter | характеръ | charakter.

\uventry{vizaĝo} visage | face | Gesicht | лицо | twarz.

\uventry{povi} pouvoir | be able, can | können | мочь | módz.

\uventry{pensi} penser | think | denken | думать | myśleć.

\uventry{ambaŭ} l’un et l’autre | both | beide | оба | obaj.

\uventry{fiera} fier, orgueilleux | proud | stolz | гордый | dumny.

\uventry{oni} on | one, people, they | man | безличное мѣстоименіе множественнаго числа | zaimek nieosobisty liczby mnogiej.

\uventry{vivi} vivre | live | leben | жить | żyć.

\uventry{plena} plein | full, complete | voll | полный | pełny.

\uventry{portreto} portrait | portrait | Portrait | портретъ | portret.

\uventry{laŭ} selon, d’après | according to | nach, gemäss | по, согласно | według.

\uventry{ec} marque la qualité (abstraitement); ex. \uventry{bona} bon ― \uventry{boneco} bonté; \uventry{viro} homme ― \uventry{vireco} virilité | denotes qualities; e.~g. \uventry{bona} good ― \uventry{boneco} goodness; \uventry{viro} man ― \uventry{vireco} manliness; \uventry{virino} woman ― \uventry{virineco} womanliness | Eigenschaft; z.~B. \uventry{bona} gut ― \uventry{boneco} Güte; \uventry{virino} Weib ― \uventry{virineco} Weiblichkeit | качество или состояніе; напр. \uventry{bona} добрый ― \uventry{boneco} доброта; \uventry{virino} женщина ― \uventry{virineco} женственность | przymiot; np. \uventry{bona} dobry ― \uventry{boneco} dobroć; \uventry{infano} dziecię ― \uventry{infaneco} dziecinśtwo.

\uventry{honesta} honnête | honest | ehrlich | честный | uczciwy.

\uventry{krom} hors, hormis, excepté | besides, without, except | ausser | кромѣ | oprócz.

\uventry{tio} cela | that, that one | jenes, das | то, это | to, tamto.

\end{ekzvocab}


\ekzsec{§12.}

Du homoj povas pli multe fari ol unu. ― Mi havas nur unu buŝon, sed mi havas du orelojn. ― Li promenas kun tri hundoj. ― Li faris ĉion per la dek fingroj de siaj manoj. ― El ŝiaj multaj infanoj unuj estas bonaj kaj aliaj estas malbonaj. ― Kvin kaj sep faras dek du. ― Dek kaj dek faras dudek. ― Kvar kaj dek ok faras dudek du. ― Tridek kaj kvardek kvin faras sepdek kvin. ― Mil okcent naŭdek tri. ― Li havas dek unu infanojn. ― Sesdek minutoj faras unu horon, kaj unu minuto konsistas el sesdek sekundoj. ― Januaro estas la unua monato de la jaro, Aprilo estas la kvara, Novembro estas la dek-unua, Decembro estas la dek-dua. ― La dudeka (tago) de Februaro estas la kvindek-unua tago de la jaro. ― La sepan tagon de la semajno Dio elektis, ke ĝi estu pli sankta, ol la ses unuaj tagoj. ― Kion Dio kreis en la sesa tago? ― Kiun daton ni havas hodiaŭ? ― Hodiaŭ estas la dudek sepa (tago) de Marto. ― Georgo Vaŝington estis naskita la dudek duan de Februaro de la jaro mil sepcent tridek dua.

\begin{ekzvocab}{1em}
\uventry{multe} beaucoup, nombreux | much, many | viel | много | wiele.

\uventry{fari} faire | do | thun, machen | дѣлать | robić.

\uventry{nur} seulement, ne\ldots{} que | only (adv.) | nur | только | tylko.

\uventry{promeni} se promener | walk, promenade | spazieren | прогуливаться | spacerować.

\uventry{tri} trois | three | drei | три | trzy.

\uventry{hundo} chien | dog | Hund | песъ, собака | pies.

\uventry{ĉio} tout | everything | alles | все | wszystko.

\uventry{dek} dix | ten | zehn | десять | dziesięć.

\uventry{fingro} doigt | finger | Finger | палецъ | palec.

\uventry{alia} autre | other | ander | иной | inny.

\uventry{kvin} cinq | five | fünf | пять | pięć.

\uventry{sep} sept | seven | sieben | семь | siedm.

\uventry{kvar} quatre | four | vier | четыре | cztery.

\uventry{ok} huit | eight | acht | восемь | ośm.

\uventry{mil} mille (nombre) | thousand | tausend | тысяча | tysiąc.

\uventry{cent} cent | hundred | hundert | сто | sto.

\uventry{naŭ} neuf (9) | nine | neun | девять | dziewięć.

\uventry{ses} six | six | sechs | шесть | sześć.

\uventry{minuto} minute | minute | Minute | минута | minuta.

\uventry{horo} heure | hour | Stunde | часъ | godzina.

\uventry{konsisti} consister | consist | bestehen | состоять | składać się.

\uventry{sekundo} seconde | second | Sekunde | секунда | sekunda.

\uventry{Januaro} Janvier | January | Januar | Январь | Styczeń.

\uventry{monato} mois | month | Monat | мѣсяцъ | miesiąc.

\uventry{jaro} année | year | Jahr | годъ | rok.

\uventry{Aprilo} Avril | April | April | Апрѣль | Kwiecień.

\uventry{Novembro} Novembre | November | November | Ноябрь | Listopad.

\uventry{Decembro} Décembre | December | December | Декабрь | Grudzień.

\uventry{Februaro} Février | February | Februar | Февраль | Luty.

\uventry{semajno} semaine | week | Woche | недѣля | tydzień.

\uventry{Dio} Dieu | God | Gott | Богъ | Bóg.

\uventry{elekti} choisir | choose | wählen | выбирать | wybierać.

\uventry{ĝi} cela, il, elle | it | es, dieses | оно, это | ono, to.

\uventry{sankta} saint | holy | heilig | святой, священный | święty.

\uventry{krei} créer | create | schaffen, erschaffen | создавать | stwarzać.

\uventry{dato} date | date | Datum | число (мѣсяца) | data.

\uventry{Marto} Mars | March | März | Мартъ | Marzec.

\uventry{naski} enfanter, faire naître | bear, produce | gebären | рождать | rodzić.

\uventry{it} marque le participe passé passif | ending of past part. pass. in verbs | bezeichnet das Participium perfecti passivi | означаетъ причастіе прошедшаго времени страдат. залога | oznacza imiesłów bierny czasu przeszłego.

\end{ekzvocab}

\begin{center}
\narrow{\large §13.}
\vskip 1ex
{\Large\bookman{La feino}} {\large(Daŭrigo).}
\end{center}

Ĉar ĉiu amas ordinare personon, kiu estas simila al li, tial tiu ĉi patrino varmege amis sian pli maljunan filinon, kaj en tiu sama tempo ŝi havis teruran malamon kontraŭ la pli juna. Ŝi devigis ŝin manĝi en la kuirejo kaj laboradi senĉese. Inter aliaj aferoj tiu ĉi malfeliĉa infano devis du fojojn en ĉiu tago iri ĉerpi akvon en tre malproksima loko kaj alporti domen plenan grandan kruĉon.

\begin{ekzvocab}{1em}
\uventry{daŭri} durer | endure, last | dauern | продолжаться | trwać.

\uventry{ig} faire\ldots{}; ex. \uventry{pura} pur, propre ― \uventry{purigi} nettoyer; \uventry{morti} mourir ― \uventry{mortigi} tuer (faire mourir) | cause to be; e.~g. \uventry{pura} pure ― \uventry{purigi} purify; \uventry{sidi} sit ― \uventry{sidigi} seat | zu etwas machen, lassen; z.~B. \uventry{pura} rein ― \uventry{purigi} reinigen; \uventry{bruli} brennen (selbst) ― \uventry{bruligi} brennen (etwas) | дѣлать чѣмъ нибудь, заставить дѣлать; напр. \uventry{pura} чистый ― \uventry{purigi} чистить; \uventry{bruli} горѣть ― \uventry{bruligi} жечь | robić czemś; np. \uventry{pura} czysty ― \uventry{purigi} czyścić; \uventry{bruli} palić się ― \uventry{bruligi} palić.

\uventry{ordinara} ordinaire | ordinary | gewöhnlich | обыкновенный | zwyczajny.

\uventry{persono} personne | person | Person | особа, лицо | osoba.

\uventry{tiu} celui-là | that | jener | тотъ | tamten.

\uventry{ĉi} ce qui est le plus près; ex. \uventry{tiu} celui-là ― \uventry{tiu ĉi} celui-ci | denotes proximity; e.~g. \uventry{tiu} that ― \uventry{tiu ĉi} this; \uventry{tie} there ― \uventry{tie ĉi} here | die nächste Hinweisung; z.~B. \uventry{tiu} jener ― \uventry{tiu ĉi} dieser; \uventry{tie} dort ― \uventry{tie ĉi} hier | ближайшее указаніе; напр. \uventry{tiu} тотъ ― \uventry{tiu ĉi} этотъ; \uventry{tie} тамъ ― \uventry{tie ĉi} здѣсь | wskazanie najbliższe; np. \uventry{tiu} tamten ― \uventry{tiu ĉi} ten; \uventry{tie} tam ― \uventry{tie ĉi} tu.

\uventry{varma} chaud | warm | warm | теплый | ciepły.

\uventry{eg} marque augmentation, plus haut degré; ex. \uventry{pordo} porte ― \uventry{pordego} grande porte; \uventry{peti} prier ― \uventry{petegi} supplier | denotes increase of degree; e.~g. \uventry{varma} warm ― \uventry{varmega} hot | bezeichnet eine Vergösserung oder Steigerung; z.~B. \uventry{pordo} Thür ― \uventry{pordego} Thor; \uventry{varma} warm ― \uventry{varmega} heiss | означаетъ увеличеніе или усиленіе степени; напр. \uventry{mano} рука ― \uventry{manego} ручище; \uventry{varma} теплый ― \uventry{varmega} горячій | oznacza zwiększenie lub wzmocnienie stopnia; np. \uventry{mano} ręka ― \uventry{manego} łapa; \uventry{varma} ciepły ― \uventry{varmega} gorący.

\uventry{sama} même (qui n’est pas autre) | same | selb, selbst (z.~B. derselbe, daselbst) | же, самый (напр. тамъ же, тотъ самый) | źe, sam (np. tam że, ten sam).

\uventry{tempo} temps (durée) | time | Zeit | время | czas.

\uventry{teruro} terreur, effroi | terror | Schrecken | ужасъ | przerażenie.

\uventry{kontraŭ} contre | against | gegen | противъ | przeciw.

\uventry{devi} devoir | ought, must | müssen | долженствовать | musieć.

\uventry{kuiri} faire cuire | cook | kochen | варить | gotować.

\uventry{ej} marque le lieu spécialement affecté à\ldots{} ex. \uventry{preĝi} prier ― \uventry{preĝejo} église; \uventry{kuiri} faire cuire ― \uventry{kuirejo} cuisine | place of an action; e.~g. \uventry{kuiri} cook ― \uventry{kuirejo} kitchen | Ort für\ldots{}; z.~B. \uventry{kuiri} kochen ― \uventry{kuirejo} Küche; \uventry{preĝi} beten ― \uventry{preĝejo} Kirche | мѣсто для\ldots{}; напр. \uventry{kuiri} варить ― \uventry{kuirejo} кухня; \uventry{preĝi} молиться ― \uventry{preĝejo} церковь | miejsce dla\ldots{}; np. \uventry{kuiri} gotować ― \uventry{kuirejo} kuchnia; \uventry{preĝi} modlić się ― \uventry{preĝejo} kościół.

\uventry{labori} travailler | labor, work | arbeiten | работать | pracować.

\uventry{ad} marque durée dans l’action; ex. \uventry{pafo} coup de fusil ― \uventry{pafado} fusillade | denotes duration of action; e.~g. \uventry{danco} dance ― \uventry{dancado} dancing | bezeichnet die Dauer der Thätigkeit; z.~B. \uventry{danco} der Tanz ― \uventry{dancado} das Tanzen | означаетъ продолжительность дѣйствія: напр. \uventry{iri} идти ― \uventry{iradi} ходить, хаживать | oznacza trwanie czynności; np. \uventry{iri} iść -- \uventry{iradi} chodzić.

\uventry{sen} sans | without | ohne | безъ | bez.

\uventry{ĉesi} cesser | cease, desist | aufhören | переставать | przestawać.

\uventry{inter} entre, parmi | between, among | zwischen | между | między.

\uventry{afero} affaire | affair | Sache, Angelegenheit | дѣло | sprawa.

\uventry{feliĉa} heureux | happy | glücklich | счастливый | szczęśliwy.

\uventry{fojo} fois | time (e.~g. three times etc.) | Mal | разъ | raz.

\uventry{ĉerpi} puiser | draw | schöpfen (z.~B. Wasser) | черпать | czerpać.

\uventry{tre} très | very | sehr | очень | bardzo.

\uventry{proksima} proche, près de | near | nahe | близкій | blizki.

\uventry{loko} place, lieu | place | Ort | мѣсто | miejsce.

\uventry{porti} porter | pack, carry | tragen | носить | nosić.

\uventry{n} marque l’accusatif et le lieu ou où l’on va | ending of the objective, also marks direction towards | bezeichnet den Accusativ, auch die Richtung | означаетъ винит. падежъ, а также направленіе | oznacza przypadek czwarty, również kierunek.

\uventry{kruĉo} cruche | pitcher | Krug | кувшинъ | dzban.

\end{ekzvocab}


\ekzsec{§14.}

Mi havas cent pomojn. ― Mi havas centon da pomoj. ― Tiu ĉi urbo havas milionon da loĝantoj. ― Mi aĉetis dekduon (aŭ dek-duon) da kuleroj kaj du dekduojn da forkoj. ― Mil jaroj (aŭ milo da jaroj) faras miljaron. ― Unue mi redonas al vi la monon, kiun vi pruntis al mi; due mi dankas vin por la prunto; trie mi petas vin ankaŭ poste prunti al mi, kiam mi bezonos monon. ― Por ĉiu tago mi ricevas kvin frankojn, sed por la hodiaŭa tago mi ricevis duoblan pagon, t. e. (= tio estas) dek frankojn. ― Kvinoble sep estas tridek kvin. ― Tri estas duono de ses. ― Ok estas kvar kvinonoj de dek. ― Kvar metroj da tiu ĉi ŝtofo kostas naŭ frankojn; tial du metroj kostas kvar kaj duonon frankojn (aŭ da frankoj). ― Unu tago estas tricent-sesdek-kvinono aŭ tricent-sesdek-sesono de jaro. ― Tiuj ĉi du amikoj promenas ĉiam duope. ― Kvinope ili sin ĵetis sur min, sed mi venkis ĉiujn kvin atakantojn. ― Por miaj kvar infanoj mi aĉetis dek du pomojn, kaj al ĉiu el la infanoj mi donis po tri pomoj. ― Tiu ĉi libro havas sesdek paĝojn; tial, se mi legos en ĉiu tago po dek kvin paĝoj, mi finos la tutan libron en kvar tagoj.

\begin{ekzvocab}{1em}
\uventry{on} marque les nombres fractionnaires; ex. \uventry{kvar} quatre ― \uventry{kvarono} le quart | marks fractions; e.~g. \uventry{kvar} four ― \uventry{kvarono} a fourth, quarter | Bruchzahlwort; z.~B. \uventry{kvar} vier ― \uventry{kvarono} Viertel | означаетъ числительное дробное; напр. \uventry{kvar} четыре ― \uventry{kvarono} четверть | liczebnik ułamkowy; np. \uventry{kvar} cztery ― \uventry{kvarono} ćwierć.

\uventry{da} de (après les mots marquant mesure, poids, nombre) | is used instead of de after words expressing weight or measure | ersetzt den Genitiv nach Mass, Gewicht u. drgl bezeichnenden Wörtern | замѣняетъ родительный падежъ послѣ словъ, означающихъ мѣру, вѣсъ и т. п. | zastępuje przypadek drugi po słowach oznaczających miarę, wagę i. t. p.

\uventry{urbo} ville | town | Stadt | городъ | miasto.

\uventry{loĝi} habiter, loger | lodge | wohnen | жить, квартировать | mieszkać.

\uventry{ant} marque le participe actif | ending of pres. part. act. in verbs | bezeichnet das Participium praes. act. | означаетъ причасіе настоящаго времени дѣйств. залога | oznacza imieslów czynny czasu teraźniejsz.

\uventry{aĉeti} acheter | buy | kaufen | покупать | kupować.

\uventry{aŭ} ou | or | oder | или | albo, lub.

\uventry{kulero} cuillère | spoon | Löffel | ложка | łyżka.

\uventry{forko} fourchette | fork | Gabel | вилы, вилка | widły, widelec.

\uventry{re} de nouveau, de retour | again, back | wieder, zurück | снова, назадъ | znowu, napowrót.

\uventry{mono} argent (monnaie) | money | Geld | деньги | pieniądze.

\uventry{prunti} prêter | lend, borrow | leihen, borgen | взаймы давать или брать | pożyczać.

\uventry{danki} remercier | thank | danken | благодарить | dziękować.

\uventry{por} pour | for | für | для, за | dla, za.

\uventry{peti} prier | request, beg | bitten | просить | prosić.

\uventry{ankaŭ} aussi | also | auch | также | także.

\uventry{post} après | after, behind | nach, hinter | послѣ, за | po, za, potem.

\uventry{kiam} quand, lorsque | when | wann | когда | kiedy.

\uventry{bezoni} avoir besoin de | need, want | brauchen | нуждаться | potrzebować.

\uventry{obl} marque l’adjectif numéral multiplicatif; ex. \uventry{du} deux ― \uventry{duobla} double | \ldots{}fold; e.~g. \uventry{du} two ― \uventry{duobla} twofold, duplex | bezeichnet das Vervielfachungszahlwort; z.~B. \uventry{du} zwei ― \uventry{duobla} zweifach | означаетъ числительное множительное; напр. \uventry{du} два ― \uventry{duobla} двойной | oznacza liczebnik wieloraki; np. \uventry{du} dwa ― \uventry{duobla} podwójny.

\uventry{pagi} payer | pay | zahlen | платить | płacić.

\uventry{ŝtofo} étoffe | stuff, matter, goods | Stoff | вещество, матерія | materja, materjał.

\uventry{kosti} coûter | cost | kosten | стоить | kosztować.

\uventry{ĉiam} toujours | always | immer | всегда | zawsze.

\uventry{op} marque ládjectif numéral collectif; ex. \uventry{du} deux ― \uventry{duope} à deux | marks collective numerals; e.~g. \uventry{tri} three ― \uventry{triope} three together | Sammelzahlwort; z.~B. \uventry{du} zwei ― \uventry{duope} selbander, zwei zusammen | означаетъ числительное собирательное; напр. \uventry{du} два ― \uventry{duope} вдвоемъ | oznacza liczebnik zbiorowy; np. \uventry{du} dwa ― \uventry{duope} we dwoje.

\uventry{ĵeti} jeter | throw | werfen | бросать | rzucać.

\uventry{venki} vaincre | conquer | siegen | побѣждать | zwyciężać.

\uventry{ataki} attaquer | attack | angreifen | нападать | atakować.

\uventry{paĝo} page (d’un livre) | page | Seite (Buch-) | страница | stronica.

\uventry{se} si | if | wenn | если | jeżeli.

\uventry{fini} finir | end, finish | enden, beendigen | кончать | kończyć.

\uventry{tuta} entier, total | whole | ganz | цѣлый, весь | cały.

\end{ekzvocab}


\begin{center}
\narrow{\large §15.}
\vskip 1ex
{\Large\bookman{La feino}} {\large(Daŭrigo).}
\end{center}

En unu tago, kiam ŝi estis apud tiu fonto, venis al ŝi malriĉa virino, kiu petis ŝin, ke ŝi donu al ŝi trinki. “Tre volonte, mia bona,”; diris la bela knabino. Kaj ŝi tuj lavis sian kruĉon kaj ĉerpis akvon en la plej pura loko de la fonto kaj alportis al la virino, ĉiam subtenante la kruĉon, por ke la virino povu trinki pli oportune. Kiam la bona virino trankviligis sian soifon, ŝi diris al la knabino: “Vi estas tiel bela, tiel bona kaj tiel honesta, ke mi devas fari al vi donacon” (ĉar tio ĉi estis feino, kiu prenis sur sin la formon de malriĉa vilaĝa virino, por vidi, kiel granda estos la ĝentileco de tiu ĉi juna knabino). “Mi faras al vi donacon,” daŭrigis la feino, “ke ĉe ĉiu vorto, kiun vi diros, el via buŝo eliros aŭ floro aŭ multekosta ŝtono.”

\begin{ekzvocab}{1em}
\uventry{fonto} source | fountain | Quelle | источникъ | żródło.

\uventry{riĉa} riche | rich | reich | богатый | bogaty.

\uventry{viro} homme (sexe) | man | Mann | мужчина, мужъ | mężczyzna, mąż.

\uventry{volonte} volontiers | willingly | gern | охотно | chętnie.

\uventry{tuj} tout de suite, aussitôt | immediate | bald, sogleich | сейчасъ | natychmiast.

\uventry{lavi} laver | wash | waschen | мыть | myć.

\uventry{sub} sous | under, beneath, below | unter | подъ | pod.

\uventry{teni} tenir | hold, grasp | halten | держать | trzymać.

\uventry{oportuna} commode, qui est à propos | opportune, suitable | bequem | удобный | wygodny.

\uventry{trankvila} tranquille | quiet | ruhig | спокойный | spokojny.

\uventry{soifi} avoir soif | thirst | dursten | жаждать | pragnąć.

\uventry{donaci} faire cadeau | make a present | schenken | дарить | darować.

\uventry{preni} prendre | take | nehmen | брать | brać.

\uventry{formo} forme | form | Form | форма | forma, kształt.

\uventry{vilaĝo} village | village | Dorf | деревня | wieś.

\uventry{ĝentila} gentil, poli | polite, gentle | höflich | вѣжливый | grzeczny.

\uventry{ĉe} chez | at | bei | у, при | u, przy.

\end{ekzvocab}


\ekzsec{§16.}

Mi legas. ― Ci skribas (anstataŭ ``ci" oni uzas ordinare ``vi"). ― Li estas knabo, kaj ŝi estas knabino. ― La tranĉilo tranĉas bone, ĉar ĝi estas akra. ― Ni estas homoj. ― Vi estas infanoj. ― Ili estas rusoj. ― Kie estas la knaboj? ― Ili estas en la ĝardeno. ― Kie estas la knabinoj? ― Ili ankaŭ estas en la ĝardeno. ― Kie estas la tranĉiloj? ― Ili kuŝas sur la tablo. ― Mi vokas la knabon, kaj li venas. ― Mi vokas la knabinon, kaj ŝi venas. ― La infano ploras, ĉar ĝi volas manĝi. ― La infanoj ploras, ĉar ili volas manĝi. ― Knabo, vi estas neĝentila. ― Sinjoro, vi estas neĝentila. ― Sinjoroj, vi estas neĝentilaj. ― Mia hundo, vi estas tre fidela. ― Oni diras, ke la vero ĉiam venkas. ― En la vintro oni hejtas la fornojn. ― Kiam oni estas riĉa (aŭ riĉaj), oni havas multajn amikojn.

\begin{ekzvocab}{1em}
\uventry{ci} tu, toi, | thou | du | ты | ty.

\uventry{anstataŭ} au lieu de | instead | anstatt, statt | вмѣсто | zamiast.

\uventry{uzi} employer | use | gebrauchen | употреблять | używać.

\uventry{tranĉi} trancher, couper | cut | schneiden | рѣзать | rżnąć.

\uventry{il} instrument; ex. \uventry{tondi} tondre ― \uventry{tondilo} ciseaux; \uventry{pafi} tirer (coup de feu) ― \uventry{pafilo} fusil | instrument; e.~g. \uventry{tondi} shear ― \uventry{tondilo} scissors | Werkzeug; z.~B. \uventry{tondi} scheeren ― \uventry{tondilo} Scheere; \uventry{pafi} schiessen ― \uventry{pafilo} Flinte | орудіе; напр. \uventry{tondi} стричь ― \uventry{tondilo} ножницы; \uventry{pafi} стрѣлять ― \uventry{pafilo} ружье | narzędzie; np. \uventry{tondi} strzydz ― \uventry{tondilo} nożyce; \uventry{pafi} strzelać ― \uventry{pafilo} fuzya.

\uventry{ruso} russe | Russian | Russe | русскій | rossjanin.

\uventry{ĝardeno} jardin | garden | Garten | садъ | ogród.

\uventry{voki} appeler | call | rufen | звать | wołać.

\uventry{voli} vouloir | wish, will | wollen | хотѣть | chcieć.

\uventry{fidela} fidèle | faithful | treu | вѣрный | wierny.

\uventry{vero} vérité | true | Wahrheit | истина | prawda.

\uventry{vintro} hiver | winter | Winter | зима | zima.

\uventry{hejti} chauffer, faire du feu | heat (vb.) | heizen | топить (печку) | palić (w piecu).

\uventry{forno} fourneau, poële, four | stove | Ofen | печь, печка | piec.

\end{ekzvocab}


\begin{center}
\narrow{\large §17.}
\vskip 1ex
{\Large\bookman{La feino}} {\large(Daŭrigo).}
\end{center}

Kiam tiu ĉi bela knabino venis domen, ŝia patrino insultis ŝin, kial ŝi revenis tiel malfrue de la fonto. “Pardonu al mi, patrino,” diris la malfeliĉa knabino, “ke mi restis tiel longe”. Kaj kiam ŝi parolis tiujn ĉi vortojn, elsaltis el ŝia buŝo tri rozoj, tri perloj kaj tri grandaj diamantoj. “Kion mi vidas!” diris ŝia patrino kun grandega miro. “Ŝajnas al mi, ke el ŝia buŝo elsaltas perloj kaj diamantoj! De kio tio ĉi venas, mia filino?” (Tio ĉi estis la unua fojo, ke ŝi nomis ŝin sia filino). La malfeliĉa infano rakontis al ŝi naive ĉion, kio okazis al ŝi, kaj, dum ŝi parolis, elfalis el ŝia buŝo multego da diamantoj. “Se estas tiel,” diris la patrino, “mi devas tien sendi mian filinon. Marinjo, rigardu, kio eliras el la buŝo de via fratino, kiam ŝi parolas; ĉu ne estus al vi agrable havi tian saman kapablon? Vi devas nur iri al la fonto ĉerpi akvon; kaj kiam malriĉa virino petos de vi trinki, vi donos ĝin al ŝi ĝentile.”

\begin{ekzvocab}{1em}
\uventry{insulti} injurier | insult | schelten, schimpfen | ругать | besztać, łajać.

\uventry{kial} pourquoi | because, wherefore | warum | почему | dlaczego.

\uventry{frue} de bonne heure | early | früh | рано | rano, wcześnie.

\uventry{pardoni} pardonner | forgive | verzeihen | прощать | przebaczać.

\uventry{longa} long | long | lang | долгій, длинный | długi.

\uventry{salti} sauter, bondir | leap, jump | springen | прыгать | skakać.

\uventry{perlo} perle | pearl | Perle | жемчугъ | perła.

\uventry{granda} grand | great, tall | gross | большой, великій | wielki, duźy.

\uventry{diamanto} diamant | diamond | Diamant | алмазъ | djament.

\uventry{miri} s’étonner, admirer | wonder | sich wundern | удивляться | dziwić się.

\uventry{ŝajni} sembler | seem | scheinen | казаться | wydawać się.

\uventry{nomi} nommer, appeler | name, nominate | nennen | называть | naźywać.

\uventry{naiva} naïf | naïve | naiv | наивный | naiwny.

\uventry{okazi} avoir lieu, arriver | happen | vorfallen | случаться | zdarzać się.

\uventry{dum} pendant, tandis que | while | während | пока, между тѣмъ какъ | póki.

\uventry{sendi} envoyer | send | senden, schicken | посылать | posyłać.

\uventry{kapabla} capable, apte | capable | fähig | способный | zdolny.

\end{ekzvocab}


\ekzsec{§18.}

Li amas min, sed mi lin ne amas. ― Mi volis lin bati, sed li forkuris de mi. ― Diru al mi vian nomon. ― Ne skribu al mi tiajn longajn leterojn. ― Venu al mi hodiaŭ vespere. ― Mi rakontos al vi historion. ― Ĉu vi diros al mi la veron? ― La domo apartenas al li. ― Li estas mia onklo, ĉar mia patro estas lia frato. ― Sinjoro Petro kaj lia edzino tre amas miajn infanojn; mi ankaŭ tre amas iliajn (infanojn). ― Montru al ili vian novan veston. ― Mi amas min mem, vi amas vin mem, li amas sin mem, kaj ĉiu homo amas sin mem. ― Mia frato diris al Stefano, ke li amas lin pli, ol sin mem. ― Mi zorgas pri ŝi tiel, kiel mi zorgas pri mi mem; sed ŝi mem tute ne zorgas pri si kaj tute sin ne gardas. ― Miaj fratoj havis hodiaŭ gastojn; post la vespermanĝo niaj fratoj eliris kun la gastoj el sia domo kaj akompanis ilin ĝis ilia domo. ― Mi jam havas mian ĉapelon; nun serĉu vi vian. ― Mi lavis min en mia ĉambro, kaj ŝi lavis sin en sia ĉambro. ― La infano serĉis sian pupon; mi montris al la infano, kie kuŝas ĝia pupo. ― Oni ne forgesas facile sian unuan amon.

\begin{ekzvocab}{1em}
\uventry{kuri} courir | run | laufen | бѣгать | biegać, lecieć.

\uventry{vespero} soir | evening | Abend | вечеръ | wieczór.

\uventry{ĉu} est-ce que | whether | ob | ли, развѣ | czy.

\uventry{edzo} mari, époux | married person, husband | Gemahl | супругъ | małżonek.

\uventry{montri} montrer | show | zeigen | показывать | pokazywać.

\uventry{nova} nouveau | new | neu | новый | nowy.

\uventry{vesti} vêtir, habiller | clothe | ankleiden | одѣвать | odziewać, ubierać.

\uventry{mem} même (moi-, toi-, etc.) | self | selbst | самъ | sam.

\uventry{zorgi} avoir soin | care, be anxious | sorgen | заботиться | troszczyć się.

\uventry{gardi} garder | guard | hüten | стеречь, беречь | strzedz.

\uventry{gasto} hôte | guest | Gast | гость | gość.

\uventry{akompani} accompagner | accompany | begleiten | сопровождать | towarzyszyć.

\uventry{ĝis} jusqu’à | up to, until | bis | до | do, aż.

\uventry{serĉi} chercher | search | suchen | искать | szukać.

\uventry{pupo} poupée | doll | Puppe | кукла | lalka.

\uventry{forgesi} oublier | forget | vergessen | забывать | zapominać.

\uventry{facila} facile | easy | leicht | легкій | łatwy, lekki.

\end{ekzvocab}

\begin{center}
\narrow{\large §19.}
\vskip 1ex
{\Large\bookman{La feino}} {\large(Daŭrigo).}
\end{center}

“Estus tre bele,” respondis la filino malĝentile, “ke mi iru al la fonto!” ― “Mi volas ke vi tien iru,” diris la patrino, “kaj iru tuj!” La filino iris, sed ĉiam murmurante. Ŝi prenis la plej belan arĝentan vazon, kiu estis en la loĝejo. Apenaŭ ŝi venis al la fonto, ŝi vidis unu sinjorinon, tre riĉe vestitan, kiu eliris el la arbaro kaj petis de ŝi trinki (tio ĉi estis tiu sama feino, kiu prenis sur sin la formon kaj la vestojn de princino, por vidi, kiel granda estos la malboneco de tiu ĉi knabino). “Ĉu mi venis tien ĉi,” diris al ŝi la malĝentila kaj fiera knabino, “por doni al vi trinki? Certe, mi alportis arĝentan vazon speciale por tio, por doni trinki al tiu ĉi sinjorino! Mia opinio estas: prenu mem akvon, se vi volas trinki.” ― “Vi tute ne estas ĝentila,” diris la feino sen kolero. “Bone, ĉar vi estas tiel servema, mi faras al vi donacon, ke ĉe ĉiu vorto, kiun vi parolos, eliros el via buŝo aŭ serpento aŭ rano.”

\begin{ekzvocab}{1em}
\uventry{us} marque le conditionnel (ou le subjonctif) | ending of conditional in verbs | bezeichnet den Konditionalis (oder Konjunktiv) | означаетъ условное наклоненіе (или сослагательное) | oznacza tryb warunkowy.

\uventry{murmuri} murmurer, grommeler | murmur | murren, brummen | ворчать | mruczеć.

\uventry{vazo} vase | vase | Gefäss | сосудъ | naczynie.

\uventry{arĝento} argent (métal) | silver | Silber | серебро | srebro.

\uventry{apenaŭ} à peine | scarcely | kaum | едва | ledwie.

\uventry{ar} une réunion de certains objets; ex. \uventry{arbo} arbre ― \uventry{arbaro} forêt | a collection of objects; e.~g. \uventry{arbo} tree ― \uventry{arbaro} forest; \uventry{ŝtupo} step ― \uventry{ŝtuparo} stairs | Sammlung gewisser Gegenstände; z.~B. \uventry{arbo} Baum ― \uventry{arbaro} Wald; \uventry{ŝtupo} Stufe ― \uventry{ŝtuparo} Treppe, Leiter | собраніе данныхъ предметовъ; напр. \uventry{arbo} дерево ― \uventry{arbaro} лѣсъ; \uventry{ŝtupo} ступень ― \uventry{ŝtuparo} лѣстница | zbiór danych przedmiotów; np. \uventry{arbo} drzewo ― \uventry{arbaro} las; \uventry{ŝtupo} szczebel ― \uventry{ŝtuparo} drabina.

\uventry{princo} prince | prince | Fürst, Prinz | принцъ, князь | książe.

\uventry{certa} certain | certain, sure | sicher, gewiss | вѣрный, извѣстный | pewny.

\uventry{speciala} spécial | special | speciell | спеціальный | specjalny.

\uventry{opinio} opinion | opinion | Meinung | мнѣніе | opinja.

\uventry{koleri} se fâcher | be angry | zürnen | сердиться | gniewać się.

\uventry{servi} servir | serve | dienen | служить | słuźyć.

\uventry{em} qui a le penchant, l’habitude; ex. \uventry{babili} babiller ― \uventry{babilema} babillard | inclined to; e.~g. \uventry{babili} chatter ― \uventry{babilema} talkative | geneigt, gewohnt; z.~B. \uventry{babili} plaudern ― \uventry{babilema} geschwätzig | склонный, имѣющій привычку; напр. \uventry{babili} болтать ― \uventry{babilema} болтливый | skłonny, przyzwyczajony; np. \uventry{babili} paplać ― \uventry{babilema} gadula.

\uventry{serpento} serpent | serpent | Schlange | змѣя | wąź.

\uventry{rano} grenouille | frog | Frosch | лягушка | żaba.

\end{ekzvocab}


\ekzsec{§20.}

Nun mi legas, vi legas kaj li legas; ni ĉiuj legas. ― Vi skribas, kaj la infanoj skribas; ili ĉiuj sidas silente kaj skribas. ― Hieraŭ mi renkontis vian filon, kaj li ĝentile salutis min. ― Hodiaŭ estas sabato, kaj morgaŭ estos dimanĉo. ― Hieraŭ estis vendredo, kaj postmorgaŭ estos lundo. ― Antaŭ tri tagoj mi vizitis vian kuzon kaj mia vizito faris al li plezuron. ― Ĉu vi jam trovis vian horloĝon? ― Mi ĝin ankoraŭ ne serĉis; kiam mi finos mian laboron, mi serĉos mian horloĝon, sed mi timas, ke mi ĝin jam ne trovos. ― Kiam mi venis al li, li dormis; sed mi lin vekis. ― Se mi estus sana, mi estus feliĉa. ― Se li scius, ke mi estas tie ĉi, li tuj venus al mi. ― Se la lernanto scius bone sian lecionon, la instruanto lin ne punus. ― Kial vi ne respondas al mi? ― Ĉu vi estas surda aŭ muta? ― Iru for! ― Infano, ne tuŝu la spegulon! ― Karaj infanoj, estu ĉiam honestaj! ― Li venu, kaj mi pardonos al li. Ordonu al li, ke li ne babilu. ― Petu ŝin, ke ŝi sendu al mi kandelon. ― Ni estu gajaj, ni uzu bone la vivon, ĉar la vivo ne estas longa. ― Ŝi volas danci. ― Morti pro la patrujo estas agrable. ― La infano ne ĉesas petoli.

\begin{ekzvocab}{1em}
\uventry{sidi} être assis | sit | sitzen | сидѣть | siedzieć.

\uventry{silenti} se taire | be silent | schweigen | молчать | milczeć.

\uventry{hieraŭ} hier | yesterday | gestern | вчера | wczoraj.

\uventry{renkonti} rencontrer | meet | begegnen | встрѣчать | spotykać.

\uventry{saluti} saluer | salute, greet | grüssen | кланяться | kłaniać się.

\uventry{sabato} samedi | Saturday | Sonnabend | суббота | sobota.

\uventry{morgaŭ} demain | to-morrow | morgen | завтра | jutro.

\uventry{dimanĉo} dimanche | Sunday | Sonntag | воскресенье | niedziela.

\uventry{vendredo} vendredi | Friday | Freitag | пятница | piątek.

\uventry{lundo} lundi | Monday | Montag | понедѣльникъ | poniedziałek.

\uventry{viziti} visiter | visit | besuchen | посѣщать | odwiedzać.

\uventry{kuzo} cousin | cousin | Vetter, Cousin | двоюродный братъ | kuzyn.

\uventry{plezuro} plaisir | pleasure | Vergnügen | удовольствіе | przyjemność.

\uventry{horloĝo} horloge, montre | clock | Uhr | часы | zegar.

\uventry{timi} craindre | fear | fürchten | бояться | obawiać się.

\uventry{dormi} dormir | sleep | schlafen | спать | spać.

\uventry{veki} réveiller, éveiller | wake, arouse | wecken | будить | budzić.

\uventry{scii} savoir | know | wissen | знать | wiedzieć.

\uventry{leciono} leçon | lesson | Lektion | урокъ | lekcya.

\uventry{instrui} instruire, enseigner | instruct, teach | lehren | учить | uczyć.

\uventry{puni} punir | punish | strafen | наказывать | karać.

\uventry{surda} sourd | deaf | taub | глухой | głuchy.

\uventry{muta} muet | dumb | stumm | нѣмой | niemy.

\uventry{tuŝi} toucher | touch | rühren | трогать | ruszać, dotykać.

\uventry{spegulo} miroir | looking-glass | Spiegel | зеркало | zwierciadło.

\uventry{kara} cher | dear | theuer | дорогой | drogi.

\uventry{ordoni} ordonner | order, command | befehlen | приказывать | rozkazywać.

\uventry{babili} babiller | chatter | schwatzen, plaudern | болтать | paplać.

\uventry{kandelo} chandelle | candle | Licht, Kerze | свѣча | świeca.

\uventry{gaja} gai | gay, glad | lustig | веселый | wesoły.

\uventry{danci} danser | dance | tanzen | танцовать | tańczyć.

\uventry{morti} mourir | die | sterben | умирать | umierać.

\uventry{petoli} faire le polisson, faire des bêtises | be mischievous | muthwillig sein | шалить | swawolić.

\uventry{uj} qui porte, qui contient, qui est peuplé de; ex. \uventry{pomo} pomme ― \uventry{pomujo} pommier; \uventry{cigaro} cigare ― \uventry{cigarujo} porte-cigares; \uventry{Turko} Turc ― \uventry{Turkujo} Turquie | containing, filled with; e.~g. \uventry{cigaro} cigar ― \uventry{cigarujo} cigar-case; \uventry{pomo} apple ― \uventry{pomujo} apple-tree; \uventry{Turko} Turk ― \uventry{Turkujo} Turkey | Behälter, Träger (d. h. Gegenstand worin\ldots{} aufbewahrt wird,\ldots{} Früchte tragende Pflanze, von\ldots{} bevölkertes Land); z.~B. \uventry{cigaro} Cigarre ― \uventry{cigarujo} Cigarrenbüchse; \uventry{pomo} Apfel ― \uventry{pomujo} Apfelbaum; \uventry{Turko} Türke ― \uventry{Turkujo} Türkei | вмѣститель, носитель (т.~е. вещь, въ которой храниться\ldots{}; растеніе несущее\ldots{} или страна, заселенная\ldots{}); напр. \uventry{cigaro} сигара ― \uventry{cigarujo} портъ-сигаръ; \uventry{pomo} яблоко ― \uventry{pomujo} яблоня; \uventry{Turko} Турокъ ― Turkujo \uventry{Турція} | zawierający, noszący (t.~j. przedmiot, w którym się coś przechowuje, roślina, która wydaje owoc, lub kraj, względem zaludniających go mieszkańców; np. \uventry{cigaro} cygaro ― \uventry{cigarujo} cygarnica; \uventry{pomo} jabłko ― \uventry{pomujo} jabłoń; \uventry{Turko} turek ― \uventry{Turkujo} Turcya.

\end{ekzvocab}

\begin{center}
\narrow{\large §21.}
\vskip 1ex
{\Large\bookman{La feino}} {\large(Daŭrigo).}
\end{center}

Apenaŭ ŝia patrino ŝin rimarkis, ŝi kriis al ŝi: «Nu, mia filino?» ― «Jes, patrino», respondis al ŝi la malĝentilulino, elĵetante unu serpenton kaj unu ranon. ― «Ho, ĉielo!» ekkriis la patrino, «kion mi vidas? Ŝia fratino en ĉio estas kulpa; mi pagos al ŝi por tio ĉi!» Kaj ŝi tuj kuris bati ŝin. La malfeliĉa infano forkuris kaj kaŝis sin en la plej proksima arbaro. La filo de la reĝo, kiu revenis de ĉaso, ŝin renkontis; kaj, vidante, ke ŝi estas tiel bela, li demandis ŝin, kion ŝi faras tie ĉi tute sola kaj pro kio ŝi ploras. ― «Ho ve, sinjoro, mia patrino forpelis min el la domo».

\begin{ekzvocab}{1em}
\uventry{rimarki} remarquer | remark | merken, bemerken | замѣчать | postrzegać, zauwaźać.

\uventry{krii} crier | cry | schreien | кричать | krzyczeć.

\uventry{nu} eh bien! | well! | nu! | nun | ну! | no!

\uventry{jes} oui | yes | ja | да | tak.

\uventry{ek} indique une action qui commence ou qui est momentanée; ex. \uventry{kanti} chanter ― \uventry{ekkanti} commencer à chanter; \uventry{krii} crier ― \uventry{ekkrii} s’écrier | denotes sudden or momentary action; e.~g. \uventry{krii} cry ― \uventry{ekkrii} cry out | bezeichnet eine anfangende oder momentane Handlung; z.~B. \uventry{kanti} singen ― \uventry{ekkanti} einen Gesang anstimmen; \uventry{krii} schreien ― \uventry{ekkrii} aufschreien | начало или мгновенность; напр. \uventry{kanti} пѣть ― \uventry{ekkanti} запѣть; \uventry{krii} кричать ― \uventry{ekkrii} вскрикнуть | oznacza początek lub chwilowość; np. \uventry{kanti} śpiewać ― \uventry{ekkanti} zaśpiewać; \uventry{krii} krzyczeć ― \uventry{ekkrii} krzyknąć.

\uventry{kulpa} coupable | blameable | schuldig | виноватый | winny.

\uventry{kaŝi} cacher | hide (vb.) | verbergen | прятать | chować.

\uventry{reĝo} roi | king | König | король, царь | król.

\uventry{ĉasi} chasser (vénerie) | hunt | jagen, Jagd machen | охотиться | polować.

\uventry{demandi} demander, questionner | demand, ask | fragen | спрашивать | pytać.

\uventry{sola} seul | only, alone | einzig, allein | единственный | jedyny.

\uventry{pro} à cause de, pour | for the sake of | wegen | ради | dla.

\uventry{ho} oh! | oh! | o! och! | о! охъ! | o! och!.

\uventry{ve} malheur! | woe! | wehe! | увы! | biada! nestety!.

\end{ekzvocab}

\ekzsec{§22.}

Fluanta akvo estas pli pura, ol akvo staranta senmove. ― Promenante sur la strato, mi falis. ― Kiam Nikodemo batas Jozefon, tiam Nikodemo estas la batanto kaj Jozefo estas la batato. ― Al homo, pekinta senintence, Dio facile pardonas. ― Trovinte pomon, mi ĝin manĝis. ― La falinta homo ne povis sin levi. Ne riproĉu vian amikon, ĉar vi mem plimulte meritas riproĉon; li estas nur unufoja mensoginto dum vi estas ankoraŭ nun ĉiam mensoganto. ― La tempo pasinta jam neniam revenos; la tempon venontan neniu ankoraŭ konas. ― Venu, ni atendas vin, Savonto de la mondo. ― En la lingvo «Esperanto» ni vidas la estontan lingvon de la tuta mondo. ― Aŭgusto estas mia plej amata filo. ― Mono havata estas pli grava ol havita. ― Pasero kaptita estas pli bona, ol aglo kaptota. ― La soldatoj kondukis la arestitojn tra la stratoj. ― Li venis al mi tute ne atendite. ― Homo, kiun oni devas juĝi, estas juĝoto.

\begin{ekzvocab}{1em}
\uventry{flui} couler | flow | fliessen | течь | płynąć, cieknąć.

\uventry{movi} mouvoir | move | bewegen | двигать | ruszać.

\uventry{strato} rue | street | Strasse | улица | ulica.

\uventry{fali} tomber | fall | fallen | падать | padać.

\uventry{at} marque le participe présent passif | ending of pres. part. pass. in verbs | bezeichnet das Participium praes. passivi | означаеть причастіе настоящаго времени страд. залога | oznacza imiesłów bierny czasu teraźniejszego.

\uventry{peki} pécher | sin | sündigen | грѣшить | grzeszyć.

\uventry{int} marque le participe passé du verbe actif | ending of past part. act. in verbs | bezeichnet das Participium perfecti activi | означаетъ причастіе прошедшаго времени дѣйствит. залога | oznacza imiesłów czynny czasu przeszłego.

\uventry{intenci} se proposer de | intend | beabsichtigen | намѣреваться | zamierzać.

\uventry{levi} lever | lift, raise | aufheben | поднимать | podnosić.

\uventry{riproĉi} reprocher | reproach | vorwerfen | упрекать | zarzucać.

\uventry{meriti} mériter | merit | verdienen | заслуживать | zasługiwać.

\uventry{mensogi} mentir | tell a lie | lügen | лгать | kłamać.

\uventry{pasi} passer | pass | vergehen | проходить | przechodzić.

\uventry{neniam} ne\ldots{} jamais | never | niemals | никогда | nigdy.

\uventry{ont} marque le participe futur d’un verbe actif | ending of fut. part. act. in verbs | bezeichnet das Participium fut. act. | означаетъ причастіе будущаго времени дѣйствит. залога | oznacza imiesłów czynny czasu przyszłego.

\uventry{neniu} personne | nobody | Niemand | никто | nikt.

\uventry{atendi} attendre | wait, expect | warten, erwarten | ждать, ожидать | czekać.

\uventry{savi} sauver | save | retten | спасать | ratować.

\uventry{mondo} monde | world | Welt | міръ | świat.

\uventry{lingvo} langue, langage | language | Sprache | языкъ, рѣчь | język, mowa.

\uventry{grava} grave, important | important | wichtig | важный | ważny.

\uventry{pasero} passereau | sparrow | Sperling | воробей | wróbel.

\uventry{kapti} attraper | catch | fangen | ловить | chwytać.

\uventry{aglo} aigle | eagle | Adler | орелъ | orzeł.

\uventry{ot} marque le participe futur d’un verbe passif | ending of fut. part. pass. in verbs | bezeichnet das Participium fut. pass. | означаетъ причастіе будущ. времени страд. залога | oznacza imiesłów bierny czasu przyszłego.

\uventry{soldato} soldat | soldier | Soldat | солдатъ | żolnierz.

\uventry{konduki} conduire | conduct | führen | вести | prowadzić.

\uventry{aresti} arrêter | arrest | verhaften | арестовать | aresztować.

\uventry{tra} à travers | through | durch | черезъ, сквозь | przez (wskroś).

\uventry{juĝi} juger | judge | richten, urtheilen | судить | sądzić.

\end{ekzvocab}

\begin{samepage}
\begin{center}
\narrow{\large §23.}\ldots{}
\vskip 1ex
{\Large\bookman{La feino}} {\large(Fino)}
\end{center}

La reĝido, kiu vidis, ke el ŝia buŝo eliris kelke da perloj kaj kelke da diamantoj, petis ŝin, ke ŝi diru al li, de kie tio ĉi venas. Ŝi rakontis al li sian tutan aventuron. La reĝido konsideris, ke tia kapablo havas pli grandan indon, ol ĉio, kion oni povus doni dote al alia fraŭlino, forkondukis ŝin al la palaco de sia patro, la reĝo, kie li edziĝis je ŝi. Sed pri ŝia fratino ni povas diri, ke ŝi fariĝis tiel malaminda, ke ŝia propra patrino ŝin forpelis de si; kaj la malfeliĉa knabino, multe kurinte kaj trovinte neniun, kiu volus ŝin akcepti, baldaŭ mortis en angulo de arbaro.
\end{samepage}

\begin{ekzvocab}{1em}
\uventry{kelke} quelque | some | mancher, einige | нѣкоторый, нѣсколько | niektóry, kilka.

\uventry{aventuro} aventure | adventure | Abenteuer | приключеніе | przygoda.

\uventry{konsideri} considérer | consider | betrachten, erwägen | соображать | zastanawiać się.

\uventry{inda} mérite, qui mérite, est digne | worthy, valuable | würdig, werth | достойный | godny, wart.

\uventry{doto} dot | dowry | Mitgift | приданое | posag.

\uventry{palaco} palais | palace | Schloss (Gebäude) | дворецъ | pałac.

\uventry{iĝ} se faire, devenir\ldots{}; ex. \uventry{pala} pâle ― \uventry{paliĝi} pâlir; \uventry{sidi} être assis ― \uventry{sidiĝi} s’asseoir | to become; e.~g. \uventry{pala} pale ― \uventry{paliĝi} turn pale; \uventry{sidi} sit ― \uventry{sidiĝi} become seated | zu etwas werden, sich zu etwas veranlassen; z.~B. \uventry{pala} blass ― \uventry{paliĝi} erblassen; \uventry{sidi} sitzen ― \uventry{sidiĝi} sich setzen | дѣлаться чѣмъ нибудь, заставить себя; напр. \uventry{pala} блѣдный ― \uventry{paliĝi} блѣднѣть; \uventry{sidi} сидѣть ― \uventry{sidiĝi} сѣсть | stawać się czemś; np. \uventry{pala} blady ― \uventry{paliĝi} blednąć; \uventry{sidi} siedzieć ― \uventry{sidiĝi} usiąść.

\uventry{je} se traduit par différentes prépositions | can be rendered by various prepositions | kann durch verschiedene Präpositionen übersetzt werden | можетъ быть переведено различным предлогами | może być przetłomaczone za pomocą różnych przyimków.

\uventry{propra} propre (à soi) | own (one’s own) | eigen | собственный | własny.

\uventry{akcepti} accepter | accept | annehmen | принимать | przyjmować.

\uventry{baldaŭ} bientôt | soon | bald | сейчасъ, скоро | zaraz.

\uventry{angulo} coin, angle | corner, angle | Winkel | уголъ | kąt.

\end{ekzvocab}


\ekzsec{§24.}

Nun li diras al mi la veron. ― Hieraŭ li diris al mi la veron. ― Li ĉiam diradis al mi la veron. ― Kiam vi vidis nin en la salono, li jam antaŭe diris al mi la veron (aŭ li estis dirinta al mi la veron). ― Li diros al mi la veron. ― Kiam vi venos al mi, li jam antaŭe diros al mi la veron (aŭ li estos dirinta al mi la veron; aŭ antaŭ ol vi venos al mi, li diros al mi la veron). ― Se mi petus lin, li dirus al mi la veron. ― Mi ne farus la eraron, se li antaŭe dirus al mi la veron (aŭ se li estus dirinta al mi la veron). ― Kiam mi venos, diru al mi la veron. ― Kiam mia patro venos, diru al mi antaŭe la veron (aŭ estu dirinta al mi la veron). ― Mi volas diri al vi la veron. ― Mi volas, ke tio, kion mi diris, estu vera (aŭ mi volas esti dirinta la veron).

\begin{ekzvocab}{1em}
\uventry{salono} salоn | saloon | Salon | залъ | salon.

\uventry{os} marque le futur | ending of future tense in verbs | bezeichnet das Futur | означаетъ будущее время | oznacza czas przyszły.

\end{ekzvocab}


\ekzsec{§25.}

Mi estas amata. Mi estis amata. Mi estos amata. Mi estus amata. Estu amata. Esti amata. ― Vi estas lavita. Vi estis lavita. Vi estos lavita. Vi estus lavita. Estu lavita. Esti lavita. ― Li estas invitota. Li estis invitota. Li estos invitota. Li estus invitota. Estu invitota. Esti invitota. ― Tiu ĉi komercaĵo estas ĉiam volonte aĉetata de mi. ― La surtuto estas aĉetita de mi, sekve ĝi apartenas al mi. ― Kiam via domo estis konstruata, mia domo estis jam longe konstruita. ― Mi sciigas, ke de nun la ŝuldoj de mia filo ne estos pagataj de mi. ― Estu trankvila, mia tuta ŝuldo estos pagita al vi baldaŭ. ― Mia ora ringo ne estus nun tiel longe serĉata, se ĝi ne estus tiel lerte kaŝita de vi. ― Laŭ la projekto de la inĝenieroj tiu ĉi fervojo estas konstruota en la daŭro de du jaroj; sed mi pensas, ke ĝi estos konstruata pli ol tri jarojn. ― Honesta homo agas honeste. ― La pastro, kiu mortis antaŭ nelonge (aŭ antaŭ nelonga tempo), loĝis longe en nia urbo. ― Ĉu hodiaŭ estas varme aŭ malvarme? ― Sur la kameno inter du potoj staras fera kaldrono; el la kaldrono, en kiu sin trovas bolanta akvo, eliras vaporo; tra la fenestro, kiu sin trovas apud la pordo, la vaporo iras sur la korton.

\begin{ekzvocab}{1em}
\uventry{inviti} inviter | invite | einladen | приглашать | zapraszać.

\uventry{komerci} commercer | trade | handeln, Handel treiben | торговать | handlować.

\uventry{aĵ} quelque chose possédant une certaine qualité ou fait d’une certaine matière: ex. \uventry{mola} mou ― \uventry{molaĵo} partie molle d’une chose | made from or possessing the quality of; e.~g. \uventry{malnova} old ― \uventry{malnovaĵo} old thing; \uventry{frukto} fruit ― \uventry{fruktaĵo} something made from fruit | etwas von einer gewissen Eigenschaft, oder aus einem gewissen Stoffe; z.~B. \uventry{malnova} alt ― \uventry{malnovaĵo} altes Zeug; \uventry{frukto} Frucht ― \uventry{fruktaĵo} etwas aus Früchten bereitetes | нѣчто съ даннымъ качествомъ или изъ даннаго матеріала; напр. \uventry{mola} мягкій ― \uventry{molaĵo} мякишъ; \uventry{frukto} плодъ ― \uventry{fruktaĵo} нѣчто приготовленное изъ плодовъ | oznacza przedmiot posiadający pewną własność albo zrobiony z pewnego materjału; np. \uventry{malnova} stary ― \uventry{malnovaĵo} starzyzna; \uventry{frukto} owoc ― \uventry{fruktaĵo} coś zrobineego z owoców.

\uventry{sekvi} suivre | follow | folgen | слѣдовать | nastąpić.

\uventry{konstrui} construire | construct, build | bauen | строить | budować.

\uventry{ŝuldi} devoir (dette) | owe | schulden | быть должнымъ | być dłużnym.

\uventry{oro} or (métal) | gold | Gold | золото | złoto.

\uventry{ringo} anneau | ring (subst.) | Ring | кольцо | pierścień.

\uventry{lerta} adroit, habile | skilful | geschickt, gewandt | ловкій | zręczny.

\uventry{projekto} projet | project | Entwurf | проектъ | projekt.

\uventry{inĝeniero} ingénieur | engineer | Ingenieur | инженеръ | inżynier.

\uventry{fero} fer | iron | Eisen | желѣзо | żelazo.

\uventry{vojo} route, voie | way, road | Weg | дорога | droga.

\uventry{agi} agir | act | handeln, verfahren | поступать | postępować.

\uventry{pastro} prêtre, pasteur | priest, pastor | Priester | жрецъ, священникъ | kapłan.

\uventry{kameno} cheminée | fire-place | Kamin | каминъ | kominek.

\uventry{poto} pot | pot | Topf | горшокъ | garnek.

\uventry{kaldrono} chaudron | kettle | Kessel | котелъ | kocioł.

\uventry{boli} bouillir | boil | sieden | кипѣть | kipieć, wrzeć.

\uventry{vaporo} vapeur | steam | Dampf | паръ | para.

\uventry{pordo} porte | door | Thür | дверь | drzwi.

\uventry{korto} cour | yard, court | Hof | дворъ | podwórze.

\end{ekzvocab}


\ekzsec{§26.}

Kie vi estas? ― Mi estas en la ĝardeno. ― Kien vi iras? ― Mi iras en la ĝardenon. ― La birdo flugas en la ĉambro (= ĝi estas en la ĉambro kaj flugas en ĝi). ― La birdo flugas en la ĉambron (= ĝi estas ekster la ĉambro kaj flugas nun en ĝin). ― Mi vojaĝas en Hispanujo. ― Mi vojaĝas en Hispanujon. ― Mi sidas sur seĝo kaj tenas la piedojn sur benketo. ― Mi metis la manon sur la tablon. ― El sub la kanapo la muso kuris sub la liton, kaj nun ĝi kuras sub la lito. ― Super la tero sin trovas aero. ― Anstataŭ kafo li donis al mi teon kun sukero, sed sen kremo. ― Mi staras ekster la domo, kaj li estas interne. ― En la salono estis neniu krom li kaj lia fianĉino. ― La hirundo flugis trans la riveron, ĉar trans la rivero sin trovis aliaj hirundoj. ― Mi restas tie ĉi laŭ la ordono de mia estro. ― Kiam li estis ĉe mi, li staris tutan horon apud la fenestro. ― Li diras, ke mi estas atenta. ― Li petas, ke mi estu atenta. ― Kvankam vi estas riĉa, mi dubas, ĉu vi estas feliĉa. ― Se vi scius, kiu li estas, vi lin pli estimus. ― Se li jam venis, petu lin al mi. ― Ho, Dio! kion vi faras! ― Ha, kiel bele! ― For de tie ĉi! ― Fi, kiel abomene! ― Nu, iru pli rapide!

\begin{ekzvocab}{1em}
\uventry{ekster} hors, en dehors de | outside, besides | ausser, ausserhalb | внѣ | zewnątrz.

\uventry{vojaĝi} voyager | voyage | reisen | путешествовать | podróżować.

\uventry{piedo} pied | foot | Fuss, Bein | нога | noga.

\uventry{benko} banc | bench | Bank | скамья | ławka.

\uventry{et} marque diminution, décroissance; ex. \uventry{muro} mur ― \uventry{mureto} petit mur; \uventry{ridi} rire ― \uventry{rideti} sourire | denotes diminution of degree; e.~g. \uventry{ridi} laugh ― \uventry{rideti} smile | bezeichnet eine Verkleinerung oder Schwächung; z.~B. \uventry{muro} Wand ― \uventry{mureto} Wändchen; \uventry{ridi} lachen ― \uventry{rideti} lächeln | означаетъ уменьшеніе или ослабленіе степени; напр. \uventry{muro} стѣна ― \uventry{mureto} стѣнка; \uventry{ridi} смѣяться ― \uventry{rideti} улыбаться | oznacza zmniejszenie lub osłabienie stopnia; np. \uventry{muro} ściana ― \uventry{mureto} ścianka; \uventry{ridi} śmiać się ― \uventry{rideti} uśmiechać się.

\uventry{meti} mettre | put, place | hinthun | дѣть; класть | podziać.

\uventry{kanapo} canapé | sofa, lounge | Kanapee | диванъ | kanapa.

\uventry{muso} souris | mouse | Maus | мышь | mysź.

\uventry{lito} lit | bed | Bett | кровать | łóżko.

\uventry{super} au dessus de | over, above | über, oberhalb | надъ | nad.

\uventry{aero} air | air | Luft | воздухъ | powietrze.

\uventry{kafo} café | coffee | Kaffee | кофе | kawa.

\uventry{teo} thé | tea | Thee | чай | herbata.

\uventry{sukero} sucre | sugar | Zucker | сахаръ | cukier.

\uventry{kremo} crème | cream | Schmant, Sahne | сливки | śmietana.

\uventry{interne} à l’intérieur, dedans | within | innerhalb | внутри | wewnątrz.

\uventry{fianĉo} fiancé | betrothed person | Bräutigam | женихъ | narzeczony.

\uventry{hirundo} hirondelle | swallow | Schwalbe | ласточка | jaskółka.

\uventry{trans} au delà | across | jenseit | черезъ, надъ | przez.

\uventry{rivero} rivière, fleuve | river | Fluss | рѣка | rzeka.

\uventry{estro} chef | chief | Vorsteher | начальникъ | zwierzchnik.

\uventry{atenta} attentif | attentive | aufmerksam | внимательный | uważny.

\uventry{kvankam} quoique | although | obgleich | хотя | chociaź.

\uventry{dubi} douter | doubt | zweifeln | сомнѣваться | wątpić.

\uventry{estimi} estimer | esteem | schätzen, achten | уважать | szanować.

\uventry{fi} fi donc! | fie! | pfui! | фи, тьфу | fe!.

\uventry{abomeno} abomination | abomination | Abscheu | отвращеніе | odraza.

\uventry{rapida} rapide, vite | quick, rapid | schnell | быстрый | prędki.

\end{ekzvocab}


\ekzsec{§27.}

La artikolo „la“ estas uzata tiam, kiam ni parolas pri personoj aŭ objektoj konataj. Ĝia uzado estas tia sama kiel en la aliaj lingvoj. La personoj, kiuj ne komprenas la uzadon de la artikolo (ekzemple rusoj aŭ poloj, kiuj ne scias alian lingvon krom sia propra), povas en la unua tempo tute ne uzi la artikolon, ĉar ĝi estas oportuna sed ne necesa. Anstataŭ „la“ oni povas ankaŭ diri „l’“ (sed nur post prepozicio, kiu finiĝas per vokalo). ― Vortoj kunmetitaj estas kreataj per simpla kunligado de vortoj; oni prenas ordinare la purajn radikojn, sed, se la bonsoneco aŭ la klareco postulas, oni povas ankaŭ preni la tutan vorton, t.~e. la radikon kune kun ĝia gramatika finiĝo. Ekzemploj: skribtablo aŭ skribotablo (= tablo, sur kiu oni skribas); internacia (= kiu estas inter diversaj nacioj); tutmonda (= de la tuta mondo); unutaga (= kiu daŭras unu tagon); unuataga (= kiu estas en la unua tago); vaporŝipo (= ŝipo, kiu sin movas per vaporo); matenmanĝi, tagmanĝi, vespermanĝi; abonpago (= pago por la abono).

\begin{ekzvocab}{1em}
\uventry{artikolo} article | article | Artikel | членъ, статья | artykuł, przedimek.

\uventry{tiam} alors | then | dann | тогда | wtedy.

\uventry{objekto} objet | object | Gegenstand | предметъ | przedmiot.

\uventry{tia} tel | such | solcher | такой | taki.

\uventry{kompreni} comprendre | understand | verstehen | понимать | rozumieć.

\uventry{ekzemplo} exemple | example | Beispiel | примѣръ | przykład.

\uventry{polo} Polonais | Pole | Pole | Полякъ | Polak.

\uventry{necesa} nécessaire | necessary | nothwendig | необходимый | niezbędny.

\uventry{prepozicio} préposition | preposition | Vorwort | предлогъ | przyimek.

\uventry{vokalo} voyelle | vowel | Vokal | гласная | samogłoska.

\uventry{kunmeti} composer | compound | zusammensetzen | слагать | składać.

\uventry{simpla} simple | simple | einfach | простой | prosty, zwyczajny.

\uventry{ligi} lier | bind, tie | binden | связывать | wiązać.

\uventry{radiko} racine | root | Wurzel | корень | korzeń.

\uventry{soni} sonner, rendre des sons | sound | tönen, lauten | звучать | brzmieć.

\uventry{klara} clair | clear | klar | ясный | jasny.

\uventry{postuli} exiger, requérir | require, claim | fordern | требовать | żądać.

\uventry{gramatiko} grammaire | grammar | Grammatik | грамматика | gramatyka.

\uventry{nacio} nation | nation | Nation | нація, народъ | naród, nacja.

\uventry{diversa} divers | various, diverse | verschieden | различный | różny.

\uventry{ŝipo} navire | ship | Schiff | корабль | okręt.

\uventry{matenmanĝi} déjeuner | breakfast | frühstücken | завтракать | śniadać.

\uventry{aboni} abonner | subscribe | abonniren | подписываться | prenumerować.

\end{ekzvocab}


\ekzsec{§28.}

Ĉiuj prepozicioj per si mem postulas ĉiam nur la nominativon. Se ni iam post prepozicio uzas la akuzativon, la akuzativo tie dependas ne de la prepozicio, sed de aliaj kaŭzoj. Ekzemple: por esprimi direkton, ni aldonas al la vorto la finon „n“; sekve: tie (= en tiu loko), tien (= al tiu loko); tiel same ni ankaŭ diras: „la birdo flugis en la ĝardenon, sur la tablon“, kaj la vortoj „ĝardenon“, „tablon“ staras tie ĉi en akuzativo ne ĉar la prepozicioj „en“ kaj „sur“ tion ĉi postulas, sed nur ĉar ni volis esprimi direkton, t.~e. montri, ke la birdo sin ne trovis antaŭe en la ĝardeno aŭ sur la tablo kaj tie flugis, sed ke ĝi de alia loko flugis al la ĝardeno, al la tablo (ni volas montri, ke la ĝardeno kaj tablo ne estis la loko de la flugado, sed nur la celo de la flugado); en tiaj okazoj ni uzus la finiĝon „n“ tute egale ĉu ia prepozicio starus aŭ ne. ― Morgaŭ mi veturos Parizon (aŭ en Parizon). ― Mi restos hodiaŭ dome. ― Jam estas tempo iri domen. ― Ni disiĝis kaj iris en diversajn flankojn: mi iris dekstren, kaj li iris maldekstren. ― Flanken, sinjoro! ― Mi konas neniun en tiu ĉi urbo. ― Mi neniel povas kompreni, kion vi parolas. ― Mi renkontis nek lin, nek lian fraton (aŭ mi ne renkontis lin, nek lian fraton).

\begin{ekzvocab}{1em}
\uventry{nominativo} nominatif | nominative | Nominativ | именительный падежъ | pierwszy przypadek.

\uventry{iam} jamais, un jour | at any time, ever | irgend wann | когда-нибудь | kiedyś.

\uventry{akuzativo} accusatif | accusative | Accusativ | винительный падежъ | czwarty przypadek.

\uventry{tie} là-bas, là, y | there | dort | тамъ | tam.

\uventry{dependi} dépendre | depend | abhängen | зависѣть | zależeć.

\uventry{kaŭzo} cause | cause | Ursache | причина | przyczyna.

\uventry{esprimi} exprimer | express | ausdrücken | выражать | wyrażać.

\uventry{direkti} diriger | direct | richten | направлять | kierować.

\uventry{celi} viser | aim | zielen | цѣлиться | celować.

\uventry{egala} égal | equal | gleich | одинаковый | jednakowy.

\uventry{ia} quelconque | of any kind | irgend welcher | какой-нибудь | jakiś.

\uventry{veturi} aller, partir | journey, travel | fahren | ѣхать | jechać.

\uventry{dis} marque division, dissémination; ex. \uventry{iri} aller ― \uventry{disiri} se séparer, aller chacun de son côté | has the same force as the English prefix dis; e.~g. \uventry{semi} sow ― \uventry{dissemi} disseminate; \uventry{ŝiri} tear ― \uventry{disŝiri} tear to pieces | zer-; z.~B. \uventry{ŝiri} reissen ― \uventry{disŝiri} zerreissen | раз-; напр. \uventry{ŝiri} рвать ― \uventry{disŝiri} разрывать | roz-; np. \uventry{ŝiri} rwać ― \uventry{disŝiri} rozrywać.

\uventry{flanko} côté | side | Seite | сторона | strona.

\uventry{dekstra} droit, droite | right-hand | recht | правый | prawy.

\uventry{neniel} nullement, en aucune façon | nohow | keineswegs, auf keine Weise | никакъ | w żaden sposób.

\uventry{nek} --- \uventry{nek} ni ― ni | neither ― nor | weder ― noch | ни ― ни | ani ― ani.

\end{ekzvocab}


\ekzsec{§29.}

Se ni bezonas uzi prepozicion kaj la senco ne montras al ni, kian prepozicion uzi, tiam ni povas uzi la komunan prepozicion „je“. Sed estas bone uzadi la vorton „je“ kiel eble pli malofte. Anstataŭ la vorto „je“ ni povas ankaŭ uzi akuzativon sen prepozicio. ― Mi ridas je lia naiveco (aŭ mi ridas pro lia naiveco, aŭ: mi ridas lian naivecon). ― Je la lasta fojo mi vidas lin ĉe vi (aŭ: la lastan fojon). ― Mi veturis du tagojn kaj unu nokton. ― Mi sopiras je mia perdita feliĉo (aŭ: mian perditan feliĉon). ― El la dirita regulo sekvas, ke se ni pri ia verbo ne scias, ĉu ĝi postulas post si la akuzativon (t. e. ĉu ĝi estas aktiva) aŭ ne, ni povas ĉiam uzi la akuzativon. Ekzemple, ni povas diri „obei al la patro“ kaj „obei la patron“ (anstataŭ „obei je la patro“). Sed ni ne uzas la akuzativon tiam, kiam la klareco de la senco tion ĉi malpermesas; ekzemple: ni povas diri „pardoni al la malamiko“ kaj „pardoni la malamikon“, sed ni devas diri ĉiam „pardoni al la malamiko lian kulpon“.

\begin{ekzvocab}{1em}
\uventry{senco} sens, acception | sense | Sinn | смыслъ | sens, znaczenie.

\uventry{komuna} commun | common | gemeinsam | общій | ogólny, wspólny.

\uventry{ebla} possible | able, possible | möglich | возможный | możliwy.

\uventry{ofte} souvent | often | oft | часто | często.

\uventry{ridi} rire | laugh | lachen | смѣяться | śmiać się.

\uventry{lasta} dernier | last, latest | letzt | послѣдній | ostatni.

\uventry{sopiri} soupirer | sigh, long for | sich sehnen | тосковать | tęsknić.

\uventry{regulo} règle | rule | Regel | правило | prawidło.

\uventry{verbo} verbe | verb | Zeitwort | глаголъ | czasownik.

\uventry{obei} obéir | obey | gehorchen | повиноваться | być posłusznym.

\uventry{permesi} permettre | permit, allow | erlauben | позволять | pozwalać.

\end{ekzvocab}


\ekzsec{§30.}

Ia, ial, iam, ie, iel, ies, io, iom, iu. ― La montritajn naŭ vortojn ni konsilas bone ellerni, ĉar el ili ĉiu povas jam fari al si grandan serion da aliaj pronomoj kaj adverboj. Se ni aldonas al ili la literon „k“, ni ricevas vortojn demandajn aŭ rilatajn: kia, kial, kiam, kie, kiel, kies, kio, kiom, kiu. Se ni aldonas la literon „t“, ni ricevas vortojn montrajn: tia, tial, tiam, tie, tiel, ties, tio, tiom, tiu. Aldonante la literon „ĉ“, ni ricevas vortojn komunajn: ĉia, ĉial, ĉiam, ĉie, ĉiel, ĉies, ĉio, ĉiom, ĉiu. Aldonante la prefikson „nen“, ni ricevas vortojn neajn: nenia, nenial, neniam, nenie, neniel, nenies, nenio, neniom, neniu. Aldonante al la vortoj montraj la vorton „ĉi“, ni ricevas montron pli proksiman; ekzemple: tiu (pli malproksima), tiu ĉi (aŭ ĉi tiu) (pli proksima); tie (malproksime), tie ĉi aŭ ĉi tie (proksime). Aldonante al la vortoj demandaj la vorton „ajn“, ni ricevas vortojn sendiferencajn: kia ajn, kial ajn, kiam ajn, kie ajn, kiel ajn, kies ajn, kio ajn, kiom ajn, kiu ajn. Ekster tio el la diritaj vortoj ni povas ankoraŭ fari aliajn vortojn, per helpo de gramatikaj finiĝoj kaj aliaj vortoj (sufiksoj); ekzemple: tiama, ĉiama, kioma, tiea, ĉi-tiea, tieulo, tiamulo k.~t.~p. (= kaj tiel plu).

\begin{ekzvocab}{1em}
\uventry{ia} quelconque, quelque | of any kind | irgend welcher | какой-нибудь | jakiś.

\uventry{ial} pour une raison quelconque | for any cause | irgend warum | почему-нибудь | dla jakiejś przyczyny.

\uventry{iam} jamais, un jour | at any time, ever | irgend wann, einst | когда-нибудь | kiedyś.

\uventry{ie} quelque part | any where | irgend wo | гдѣ-нибудь | gdzieś.

\uventry{iel} d’une manière quelconque | anyhow | irgend wie | какъ-нибудь | jakoś.

\uventry{ies} de quelqu’un | anyone’s | irgend jemandes | чей-нибудь | czyjś.

\uventry{io} quelque chose | anything | etwas | что-нибудъ | coś.

\uventry{iom} quelque peu | any quantity | ein wenig, irgend wie viel | сколько-нибудь | ilekolwiek.

\uventry{iu} quelqu’un | any one | jemand | кто-нибудь | ktoś.

\uventry{konsili} conseiller | advise, counsel | rathen | совѣтовать | radzić.

\uventry{serio} série | series | Reihe | рядъ, серія | serya.

\uventry{pronomo} pronom | pronoun | Fürwort | мѣстоименіе | zaimek.

\uventry{adverbo} adverbe | adverb | Nebenwort | нарѣчіе | przysłówek.

\uventry{litero} lettre (de l’alphabet) | letter (of the alphabet) | Buchstabe | буква | litera.

\uventry{rilati} concerner; avoir rapport à | be related to | sich beziehen | относиться | odnosić się.

\uventry{prefikso} préfixe | prefix | Präfix | приставка | przybranka.

\uventry{ajn} que ce soit | ever | auch nur | бы-ни | kolwiek, bądź.

\uventry{diferenci} différer (v. n.) | differ | sich unterscheiden | различаться | różnic się.

\uventry{helpi} aider | help | helfen | помогать | pomagać.

\uventry{sufikso} suffixe | suffix | Suffix | суффиксъ | przyrostek.

\end{ekzvocab}


\ekzsec{§31.}

Lia kolero longe daŭris. ― Li estas hodiaŭ en kolera humoro. ― Li koleras kaj insultas. ― Li fermis kolere la pordon. ― Lia filo mortis kaj estas nun malviva. ― La korpo estas morta, la animo estas senmorta. ― Li estas morte malsana, li ne vivos pli, ol unu tagon. ― Li parolas, kaj lia parolo fluas dolĉe kaj agrable. ― Ni faris la kontrakton ne skribe, sed parole. ― Li estas bona parolanto. ― Starante ekstere, li povis vidi nur la eksteran flankon de nia domo. ― Li loĝas ekster la urbo. ― La ekstero de tiu ĉi homo estas pli bona, ol lia interno. ― Li tuj faris, kion mi volis, kaj mi dankis lin por la tuja plenumo de mia deziro. ― Kia granda brulo! kio brulas? ― Ligno estas bona brula materialo. ― La fera bastono, kiu kuŝis en la forno, estas nun brule varmega. ― Ĉu li donis al vi jesan respondon aŭ nean? Li eliris el la dormoĉambro kaj eniris en la manĝoĉambron. ― La birdo ne forflugis: ĝi nur deflugis de la arbo, alflugis al la domo kaj surflugis sur la tegmenton. ― Por ĉiu aĉetita funto da teo tiu ĉi komercisto aldonas senpage funton da sukero. ― Lernolibron oni devas ne tralegi, sed tralerni. ― Li portas rozokoloran superveston kaj teleroforman ĉapelon. ― En mia skribotablo sin trovas kvar tirkestoj. ― Liaj lipharoj estas pli grizaj, ol liaj vangharoj.

\begin{ekzvocab}{1em}
\uventry{humoro} humeur | humor | Laune | расположеніе духа | humor.

\uventry{fermi} fermer | shut | schliessen, zumachen | запирать | zamykać.

\uventry{korpo} corps | body | Körper | тѣло | ciało.

\uventry{animo} âme | soul | Seele | душа | dusza.

\uventry{kontrakti} contracter | contract | einen Vertrag abschliessen | заключать договоръ | zawierać umowę.

\uventry{um} suffixe peu employé, et qui reçoit différents sens aisément suggérés par le contexte et la signification de la racine à laquelle il est joint | this syllable has no fixed meaning | Suffix von verschiedener Bedeutung | суффиксъ безъ постояннаго значенія | przyrostek nie mający stlałego znaczenia.

\uvsubentry{}\uventry{(plenumi} accomplir | fulfil, accomplish | erfüllen | исполнять | spełniać.)

\uventry{bruli} brûler (être en feu) | burn (v. n.) | brennen (v. n.) | горѣть | palić się.

\uventry{ligno} bois | wood (the substance) | Holz | дерево, дрова | drzewo, drwa.

\uventry{materialo} matière | material | Stoff | матеріялъ | materjał.

\uventry{bastono} bâton | stick | Stock | палка | kij, laska.

\uventry{tegmento} toit | roof | Dach | крыша | dach.

\uventry{funto} livre | pound | Pfund | фунтъ | funt.

\uventry{ist} marque la profession; ex. \uventry{boto} botte ― \uventry{botisto} bottier; \uventry{maro} mer ― \uventry{maristo} marin | person occupied with; e.~g. \uventry{boto} boot ― \uventry{botisto} boot-maker; \uventry{maro} sea ― \uventry{maristo} sailor | sich mit etwas beschäftigend; z.~B. \uventry{boto} Stiefel ― \uventry{botisto} Schuster; \uventry{maro} Meer ― \uventry{maristo} Seeman | занимающійся; напр. \uventry{boto} сапогъ ― \uventry{botisto} сапожникъ; \uventry{maro} море ― \uventry{maristo} морякъ | zajmujący się; np. \uventry{boto} but ― \uventry{botisto} szewc; \uventry{maro} morze ― \uventry{maristo} marynarz.

\uventry{koloro} couleur | color | Farbe | краска, цвѣтъ | kolor.

\uventry{supre} en haut | above, upper | oben | вверху | na górze.

\uventry{telero} assiette | plate | Teller | тарелка | talerz.

\uventry{tero} terre | earth | Erde | земля | ziemia.

\uventry{kesto} caisse, coffre | chest, box | Kiste, Kasten, Lade | ящикъ | skrzynia.

\uventry{lipo} lèvre | lip | Lippe | губа | warga.

\uventry{haro} cheveu | hair | Haar | волосъ | włos.

\uventry{griza} gris | grey | grau | сѣрый, сѣдой | szary, siwy.

\uventry{vango} joue | check | Wange | щека | policzek.

\end{ekzvocab}


\ekzsec{§32.}

Teatramanto ofte vizitas la teatron kaj ricevas baldaŭ teatrajn manierojn. ― Kiu okupas sin je meĥaniko, estas meĥanikisto, kaj kiu okupas sin je ĥemio, estas ĥemiisto. ― Diplomatiiston oni povas ankaŭ nomi diplomato, sed fizikiston oni ne povas nomi fiziko, ĉar fiziko estas la nomo de la scienco mem. ― La fotografisto fotografis min, kaj mi sendis mian fotografaĵon al mia patro. ― Glaso de vino estas glaso, en kiu antaŭe sin trovis vino, aŭ kiun oni uzas por vino; glaso da vino estas glaso plena je vino. ― Alportu al mi metron da nigra drapo (Metro de drapo signifus metron, kiu kuŝis sur drapo, aŭ kiu estas uzata por drapo). ― Mi aĉetis dekon da ovoj. ― Tiu ĉi rivero havas ducent kilometrojn da longo. ― Sur la bordo de la maro staris amaso da homoj. ― Multaj birdoj flugas en la aŭtuno en pli varmajn landojn. ― Sur la arbo sin trovis multe (aŭ multo) da birdoj. ― Kelkaj homoj sentas sin la plej feliĉaj, kiam ili vidas la suferojn de siaj najbaroj. ― En la ĉambro sidis nur kelke da homoj. ― „Da“ post ia vorto montras, ke tiu ĉi vorto havas signifon de mezuro.

\begin{ekzvocab}{1em}
\uventry{teatro} theâtre | theatre | Theater | театръ | teatr.

\uventry{maniero} manière, façon | manner | Manier, Weise, Art | способъ, манера | sposób, manjera.

\uventry{okupi} occuper | occupy | einnehmen, beschäftigen | занимать | zajmować.

\uventry{meĥaniko} mécanique | mechanics | Mechanik | механика | mechanika.

\uventry{ĥemio} chimie | chemistry | Chemie | химія | chemia.

\uventry{diplomatio} diplomatie | diplomacy | Diplomatie | дипломатія | dyplomacja.

\uventry{fiziko} phyique | phyics | Phyik | физика | fizyka.

\uventry{scienco} science | science | Wissenschaft | наука | nauka.

\uventry{glaso} verre (à boire) | glass | Glas (Gefäss) | стаканъ | szklanka.

\uventry{nigra} noir | black | schwarz | черный | czarny.

\uventry{drapo} drap | woollen goods | Tuch (wollenes Gewebe) | сукно | sukno.

\uventry{signifi} signifier | signify, mean | bezeichnen, bedeuten | означать | oznaczać.

\uventry{ovo} œuf | egg | Ei | яйцо | jajko.

\uventry{bordo} bord, rivage | shore | Ufer | берегъ | brzeg.

\uventry{maro} mer | sea | Meer | море | morze.

\uventry{amaso} amas, foule | crowd, mass | Haufen, Menge | куча, толпа | kupa, tłum.

\uventry{aŭtuno} automne | autumn | Haufen, Menge | осень | jesień.

\uventry{lando} pays | land, country | Land | страна | kraj.

\uventry{suferi} souffrir, endurer | suffer | leiden | страдать | cierpieć.

\uventry{najbaro} voisin | neighbour | Nachbar | сосѣдъ | sąsiad.

\uventry{mezuri} mesurer | measure | messen | мѣрить | mierzyć.

\end{ekzvocab}


\ekzsec{§33.}

Mia frato ne estas granda, sed li ne estas ankaŭ malgranda: li estas de meza kresko. ― Li estas tiel dika, ke li ne povas trairi tra nia mallarĝa pordo. ― Haro estas tre maldika. ― La nokto estis tiel malluma, ke ni nenion povis vidi eĉ antaŭ nia nazo. ― Tiu ĉi malfreŝa pano estas malmola, kiel ŝtono. ― Malbonaj infanoj amas turmenti bestojn. ― Li sentis sin tiel malfeliĉa, ke li malbenis la tagon, en kiu li estis naskita. ― Ni forte malestimas tiun ĉi malnoblan homon. ― La fenestro longe estis nefermita; mi ĝin fermis, sed mia frato tuj ĝin denove malfermis. ― Rekta vojo estas pli mallonga, ol kurba. ― La tablo staras malrekte kaj kredeble baldaŭ renversiĝos. ― Li staras supre sur la monto kaj rigardas malsupren sur la kampon. ― Malamiko venis en nian landon. ― Oni tiel malhelpis al mi, ke mi malbonigis mian tutan laboron. ― La edzino de mia patro estas mia patrino kaj la avino de miaj infanoj. ― Sur la korto staras koko kun tri kokinoj. ― Mia fratino estas tre bela knabino. ― Mia onklino estas bona virino. ― Mi vidis vian avinon kun ŝiaj kvar nepinoj kaj kun mia nevino. ― Lia duonpatrino estas mia bofratino. ― Mi havas bovon kaj bovinon. ― La juna vidvino fariĝis denove fianĉino.

\begin{ekzvocab}{1em}
\uventry{mezo} milieu | middle | Mitte | средина | środek.

\uventry{kreski} croître | grow, increase | wachsen | рости | rosnąć.

\uventry{dika} gros | thick, fat | dick | толстый | gruby.

\uventry{larĝa} large | broad | breit | широкій | szeroki.

\uventry{lumi} luire | light | leuchten | свѣтить | świecić.

\uventry{mola} mou | soft | weich | мягкій | miękki.

\uventry{turmenti} tourmenter | torment | quälen, martern | мучить | męczyć.

\uventry{senti} ressentir, éprouver | feel, perceive | fühlen | чувствовать | czuć.

\uventry{beni} bénir | bless | segnen | благословлять | błogosławić.

\uventry{nobla} noble | noble | edel | благородный | szlachetny.

\uventry{rekta} droit, direct | straight | gerade | прямой | prosty.

\uventry{kurba} courbe, tortueux | curved | krumm | кривой | krzywy.

\uventry{kredi} croire | believe | glauben | вѣрить | wierzyć.

\uventry{renversi} renverser | upset | umwerfen, umstürzen | опрокидывать | przewracać.

\uventry{monto} montagne | mountain | Berg | гора | góra.

\uventry{kampo} champ, campagne | field | Feld | поле | pole.

\uventry{koko} coq | cock | Hahn | пѣтухъ | kogut.

\uventry{nepo} petit-fils | grandson | Enkel | внукъ | wnuk.

\uventry{nevo} neveu | nephew | Neffe | племянникъ | siostrzeniec, bratanek.

\begin{minipage}{\textwidth}
\uventry{bo} marque la parenté résultant du mariage; ex. \uventry{patro} père ― \uventry{bopatro} beau-père | relation by marriage; e.~g. \uventry{patrino} mother ― \uventry{bopatrino} mother-in-law | durch Heirath erworben; z.~B. \uventry{patro} Vater ― \uventry{bopatro} Schwiegervater; \uventry{frato} Bruder ― \uventry{bofrato} Schwager | пріобрѣтенный бракомъ; напр. \uventry{patro} отецъ ― \uventry{bopatro} тесть, свекоръ; \uventry{frato} братъ ― \uventry{bofrato} шуринъ, зять, деверь, | nabyty przez małżeństwo; np. \uventry{patro} ojciec ― \uventry{bopatro} teść; \uventry{frato} brat ― \uventry{bofrato} szwagier.

\uventry{duonpatro} beau-père | step-father | Stiefvater | отчимъ | ojczym.

\uventry{bovo} bœuf | ox | Ochs | быкъ | byk.
\end{minipage}

\end{ekzvocab}

\begin{samepage}
\ekzsec{§34.}

La tranĉilo estis tiel malakra, ke mi ne povis tranĉi per ĝi la viandon kaj mi devis uzi mian poŝan tranĉilon. ― Ĉu vi havas korktirilon, por malŝtopi la botelon? ― Mi volis ŝlosi la pordon, sed mi perdis la ŝlosilon. ― Ŝi kombas al si la harojn per arĝenta kombilo. ― En somero ni veturas per diversaj veturiloj, kaj en vintro ni veturas per glitveturilo. ― Hodiaŭ estas bela frosta vetero, tial mi prenos miajn glitilojn kaj iros gliti. ― Per hakilo ni hakas, per segilo ni segas, per fosilo ni fosas, per kudrilo ni kudras, per tondilo ni tondas, per sonorilo ni sonoras, per fajfilo ni fajfas. ― Mia skribilaro konsistas el inkujo, sablujo, kelke da plumoj, krajono kaj inksorbilo. ― Oni metis antaŭ mi manĝilaron, kiu konsistis el telero, kulero, tranĉilo, forko, glaseto por brando, glaso por vino kaj telertuketo. ― En varmega tago mi amas promeni en arbaro. ― Nia lando venkos, ĉar nia militistaro estas granda kaj brava. ― Sur kruta ŝtuparo li levis sin al la tegmento de la domo. ― Mi ne scias la lingvon hispanan, sed per helpo de vortaro hispana-germana mi tamen komprenis iom vian leteron. ― Sur tiuj ĉi vastaj kaj herboriĉaj kampoj paŝtas sin grandaj brutaroj, precipe aroj da bellanaj ŝafoj.
\end{samepage}

\begin{ekzvocab}{1em}
\uventry{viando} viande | meat, flesh | Fleisch | мясо | mięso.

\uventry{poŝo} poche | pocket | Tasche | карманъ | kieszeń.

\uventry{korko} bouchon | cork | Kork | пробка | korek.

\uventry{tiri} tirer | draw, pull, drag | ziehen | тянуть | ciągnąć.

\uventry{ŝtopi} boucher | stop, fasten down | stopfen | затыкать | zatykać.

\uventry{botelo} bouteille | bottle | Flasche | бутылка | butelka.

\uventry{ŝlosi} fermer à clef | lock, fasten | schliessen | запирать на ключъ | zamykać na klucz.

\uventry{kombi} peigner | comb | kämmen | чесать | czesać.

\uventry{somero} été | summer | Sommer | лѣто | lato.

\uventry{gliti} glisser | sakte | gleiten, glitschen | скользить, кататься | ślizgać się.

\uventry{frosto} gelée | frost | Frost | морозъ | mróz.

\uventry{vetero} temps (température) | weather | Wetter | погода | pogoda.

\uventry{haki} hacher, abattre | hew, chop | hauen, hacken | рубить | rąbać.

\uventry{segi} scier | saw | sägen | пилить | piłować.

\uventry{fosi} creuser | dig | graben | копать | kopać.

\uventry{kudri} coudre | sew | nähen | шить | szyć.

\uventry{tondi} tondre | clip, shear | scheeren | стричь | strzydz.

\uventry{sonori} tinter | give out a sound (as a bell) | klingen | звенѣть | brzęczeć, dzwonić.

\uventry{fajfi} siffler | whistle | pfeifen | свистать | świstać.

\uventry{inko} encre | ink | Tinte | чернила | atrament.

\uventry{sablo} sable | sand | Sand | песокъ | piasek.

\uventry{sorbi} humer | sip | schlürfen | хлебать | chlipać.

\uventry{brando} eau-de-vie | brandy | Branntwein | водка | wódka.

\uventry{tuko} mouchoir | cloth | Tuch (Hals-, Schnupf- etc.) | платокъ | chustka.

\uventry{militi} guerroyer | fight | Krieg führen | воевать | wojować.

\uventry{brava} brave, solide | valliant, brave | tüchtig | дѣльный, удалый | dzielny, chwacki.

\uventry{kruta} roide, escarpé | steep | steil | крутой | stromy.

\uventry{ŝtupo} marche, échelon | step | Stufe | ступень | stopień.

\uventry{Hispano} Espagnol | Spaniard | Spanier | Испанецъ | Hiszpan.

\uventry{Germano} Allemand | German | Deutscher | Нѣмецъ | Niemiec.

\uventry{tamen} pourtant, néanmoins | however, nevertheless | doch, jedoch | однако | jednak.

\uventry{vasta} vaste, étendu | wide, vast | weit, geräumig | обширный, просторный | obszerny.

\uventry{herbo} herbe | grass | Gras | трава | trawa.

\uventry{paŝti} paître | pasture, feed animals | weiden lassen | пасти | paść.

\uventry{bruto} brute, bétail | brute | Vieh | скотъ | bydło.

\uventry{precipe} principalement, surtout | particularly | besonders, vorzüglich | преимущественно | szczególnie.

\uventry{lano} laine | wool | Wolle | шерсть | wełna.

\uventry{ŝafo} mouton | sheep | Schaf | овца | owca.

\end{ekzvocab}


\ekzsec{§35.}

Vi parolas sensencaĵon, mia amiko. ― Mi trinkis teon kun kuko kaj konfitaĵo. ― Akvo estas fluidaĵo. ― Mi ne volis trinki la vinon, ĉar ĝi enhavis en si ian suspektan malklaraĵon. ― Sur la tablo staris diversaj sukeraĵoj. ― En tiuj ĉi boteletoj sin trovas diversaj acidoj: vinagro, sulfuracido, azotacido kaj aliaj. ― Via vino estas nur ia abomena acidaĵo. ― La acideco de tiu ĉi vinagro estas tre malforta. ― Mi manĝis bongustan ovaĵon. ― Tiu ĉi granda altaĵo ne estas natura monto. ― La alteco de tiu monto ne estas tre granda. ― Kiam mi ien veturas, mi neniam prenas kun mi multon da pakaĵo. ― Ĉemizojn, kolumojn, manumojn kaj ceterajn similajn objektojn ni nomas tolaĵo, kvankam ili ne ĉiam estas faritaj el tolo. ― Glaciaĵo estas dolĉa glaciigita frandaĵo. ― La riĉeco de tiu ĉi homo estas granda, sed lia malsaĝeco estas ankoraŭ pli granda. ― Li amas tiun ĉi knabinon pro ŝia beleco kaj boneco. ― Lia heroeco tre plaĉis al mi. ― La tuta supraĵo de la lago estis kovrita per naĝantaj folioj kaj diversaj aliaj kreskaĵoj. ― Mi vivas kun li en granda amikeco.

\begin{ekzvocab}{1em}
\uventry{kuko} gâteau | cаке | Kuchen | пирогъ | pieroźek.

\uventry{konfiti} confire | preserve with sugar | einmachen (mit Zucker) | варить въ сахарѣ | smażyć w cukrze.

\uventry{fluida} liquide | fluid | flüssig | жидкій | płynny.

\uventry{suspekti} suspecter, soupçonner | suspect | verdächtigen | подозрѣвать | podejrzewać.

\uventry{acida} aigre | sour | sauer | кислый | kwaśny.

\uventry{vinagro} vinaigre | vinegar | Essig | уксусъ | ocet.

\uventry{sulfuro} soufre | sulphur | Schwefel | сѣра | siara.

\uventry{azoto} azote | azotе | Stickstoff | азоть | azot.

\uventry{gusto} goût | taste | Geschmack | вкусъ | smak, gust.

\uventry{alta} haut | high | hoch | высокій | wysoki.

\uventry{naturo} nature | nature | Natur | природа | przyroda.

\uventry{paki} empaqueter, emballer | pack, put ut | packen, einpacken | укладывать, упаковывать | pakować.

\uventry{ĉemizo} chemise | shirt | Hemd | сорочка | koszula.

\uventry{kolo} cou | neck | Hals | шея | szyja.

\uventry{cetera} autre (le reste) | rest, remainder | übrig | прочій | pozostały.

\uventry{tolo} toile | linen | Leinwand | полотно | płótno.

\uventry{glacio} glace | ice | Eis | ледъ | lód.

\uventry{frandi} goûter par friandise | dainty | naschen | лакомиться | złakomić się.

\uventry{heroo} héros | hero, champion | Held | герой | bohater.

\uventry{plaĉi} plaire | please | gefallen | нравиться | podobać się.

\uventry{lago} lac | lake | See (der) | озеро | jezioro.

\uventry{kovri} couvrir | cover | verdecken, verhüllen | закрывать | zakrywać.

\begin{minipage}{\textwidth}
\uventry{naĝi} nager | swim | schwimmen | плавать | pływać.

\uventry{folio} feuille | leaf | Blatt, Bogen | листъ | liść, arkusz.
\end{minipage}
\end{ekzvocab}

\begin{samepage}
\ekzsec{§36.}

Patro kaj patrino kune estas nomataj gepatroj. ― Petro, Anno kaj Elizabeto estas miaj gefratoj. ― Gesinjoroj N. hodiaŭ vespere venos al ni. ― Mi gratulis telegrafe la junajn geedzojn. ― La gefianĉoj staris apud la altaro. ― La patro de mia edzino estas mia bopatro, mi estas lia bofilo, kaj mia patro estas la bopatro de mia edzino. ― Ĉiuj parencoj de mia edzino estas miaj boparencoj, sekve ŝia frato estas mia bofrato, ŝia fratino estas mia bofratino; mia frato kaj fratino (gefratoj) estas la bogefratoj de mia edzino. ― La edzino de mia nevo kaj la nevino de mia edzino estas miaj bonevinoj. ― Virino, kiu kuracas, estas kuracistino; edzino de kuracisto estas kuracistedzino. ― La doktoredzino A. vizitis hodiaŭ la gedoktorojn P. ― Li ne estas lavisto, li estas lavistinedzo. ― La filoj, nepoj kaj pranepoj de reĝo estas reĝidoj. ― La hebreoj estas Izraelidoj, ĉar ili devenas de Izraelo. ― Ĉevalido estas nematura ĉevalo, kokido ― nematura koko, bovido ― nematura bovo, birdido ― nematura birdo.
\end{samepage}

\begin{ekzvocab}{1em}
\uventry{ge} les deux sexes réunis; ex. \uventry{patro} père ― \uventry{gepatroj} les parents (père et mère) | of both sexes; e.~g. \uventry{patro} father ― \uventry{gepatroj} parents | beiderlei Geschlechtes; z.~B. \uventry{patro} Vater ― \uventry{gepatroj} Eltern; \uventry{mastro} Wirth ― \uventry{gemastroj} Wirth und Wirthin | обоего пола, напр. \uventry{patro} отецъ ― \uventry{gepatroj} родители; \uventry{mastro} хозяинъ ― \uventry{gemastroj} хозяинъ съ хозяйкой | obojej płci, np. \uventry{patro} ojciec ― \uventry{gepatroj} rodzice; \uventry{mastro} gospodarz ― \uventry{gemastroj} gospodarstwo (gospodarz i gospodyni).

\uventry{gratuli} féliciter | congratulate | gratuliren | поздравлять | winszować.

\uventry{altaro} autel | altar | Altar | алтарь | ołtarz.

\uventry{kuraci} traiter (une maladie) | cure, heal | kuriren, heilen | лѣчить | leczyć.

\uventry{doktoro} docteur | doctor | Doctor | докторъ | doktór.

\uventry{pra} bis-, arrière- | great-, primordial | ur- | пра- | pra-.

\uventry{id} enfant, descendant; ex. \uventry{bovo} bœuf ― \uventry{bovido} veau; \uventry{Izraelo} Israël ― \uventry{Izraelido} Israëlite | descendant, young one; e.~g. \uventry{bovo} ox ― \uventry{bovido} calf | Kind, Nachkomme; z.~B. \uventry{bovo} Ochs ― \uventry{bovido} Kalb; \uventry{Izraelo} Israel ― \uventry{Izraelido} Israelit | дитя, потомокъ; напр. \uventry{bovo} быкъ ― \uventry{bovido} теленокъ; \uventry{Izraelo} Израиль ― \uventry{Izraelido} Израильтянинъ | dziecię, potomek; np. \uventry{bovo} byk ― \uventry{bovido} cielę; \uventry{Izraelo} Izrael ― \uventry{Izraelido} Izraelita.

\uventry{hebreo} juif | Jew | Jude | еврей | żyd.

\uventry{ĉevalo} cheval | horse | Pferd | конь | koń.

\end{ekzvocab}


\ekzsec{§37.}

La ŝipanoj devas obei la ŝipestron. ― Ĉiuj loĝantoj de regno estas regnanoj. ― Urbanoj estas ordinare pli ruzaj, ol vilaĝanoj. ― La regnestro de nia lando estas bona kaj saĝa reĝo. ― La Parizanoj estas gajaj homoj. ― Nia provincestro estas severa, sed justa. ― Nia urbo havas bonajn policanojn, sed ne sufiĉe energian policestron. ― Luteranoj kaj Kalvinanoj estas kristanoj. ― Germanoj kaj francoj, kiuj loĝas en Rusujo, estas Rusujanoj, kvankam ili ne estas rusoj. ― Li estas nelerta kaj naiva provincano. ― La loĝantoj de unu regno estas samregnanoj, la loĝantoj de unu urbo estas samurbanoj, la konfesantoj de unu religio estas samreligianoj. ― Nia regimentestro estas por siaj soldatoj kiel bona patro. ― La botisto faras botojn kaj ŝuojn. ― La lignisto vendas lignon, kaj la lignaĵisto faras tablojn, seĝojn kaj aliajn lignajn objektojn. - Ŝteliston neniu lasas en sian domon. ― La kuraĝa maristo dronis en la maro. ― Verkisto verkas librojn, kaj skribisto simple transskribas paperojn. ― Ni havas diversajn servantojn: kuiriston, ĉambristinon, infanistinon kaj veturigiston. ― La riĉulo havas multe da mono. ― Malsaĝulon ĉiu batas. ― Timulo timas eĉ sian propran ombron. ― Li estas mensogisto kaj malnoblulo. ― Preĝu al la Sankta Virgulino.

\begin{ekzvocab}{1em}
\uventry{an} membre, habitant, partisan; ex. \uventry{regno} l’état ― \uventry{regnano} citoyen | inhabitant, member; e.~g. \uventry{Nov-Jorko} New York ― \uventry{Nov-Jorkano} New Yorker | Mitglied, Einwohner, Anhänger; z.~B. \uventry{regno} Staat ― \uventry{regnano} Bürger; \uventry{Varsoviano} Warschauer | членъ, житель, приверженец; напр. \uventry{regno} государство ― \uventry{regnano} гражданинъ; \uventry{Varsoviano} Варшавянинъ | członek, mieszkaniec, zwolennik; np. \uventry{regno} państwo ― \uventry{regnano} obywatel; \uventry{Varsoviano} Warszawianin.

\uventry{regno} l’Etat | kingdom | Staat | государство | państwo.

\uventry{vilaĝano} paysan | caountryman | Bauer | крестьянинъ | wieśniak.

\uventry{provinco} province | province | Provinz | область, провинція | prowincya.

\uventry{severa} sévère | severe | streng | строгій | surowy, srogi, ostry.

\uventry{justo} juste | just, righteous | gerecht | справедливый | sprawiedliwy.

\uventry{polico} police | police | Polizei | полиція | policya.

\uventry{sufiĉe} suffisant | enough | genug | довольно, достаточно | dosyć, dostatecznie.

\uventry{Kristo} Christ | Christ | Christus | Христосъ | Chrystus.

\uventry{Franco} Français | Frenchman | Franzose | Французъ | Francuz.

\uventry{konfesi} avouer | confess | bekennen, gestehen | признавать, исповѣдывать | przyznawać.

\uventry{religio} religion | religion | Religion | вѣра, религія | religia.

\uventry{regimento} regiment | regiment | Regiment | полкъ | półk.

\uventry{boto} botte | boot | Stiefel | сапогъ | but.

\uventry{ŝuo} soulier | shoe | Schuh | башмакъ | trzewik.

\uventry{lasi} laisser, abandonner | leave, let alone | lassen | пускать, оставлять | puszczać, zostawiać.

\uventry{droni} se noyer | drown | ertrinken | тонуть | tonąć.

\uventry{verki} composer, faire des ouvrages (littér.) | work (literary) | verfassen | сочинять | tworzyć, pisać.

\uventry{ul} qui est caractérisé par telle ou telle qualité; ex. \uventry{bela} beau ― \uventry{belulo} bel homme | person noted for\ldots{}; e.~g. \uventry{avara} covetous ― \uventry{avarulo} miser, covetous person | Person, die sich durch\ldots{} unterscheidet; z.~B. \uventry{juna} jung ― \uventry{junulo} Jüngling | особа, отличающаяся даннымъ качествомъ; напр. \uventry{bela} красивый ― \uventry{belulo} красавецъ | człowiek, posiadający dany przymiot; np. \uventry{riĉa} bogaty ― \uventry{riĉulo} bogacz.

\uventry{eĉ} même, jusqu’à | even | sogar | даже | nawet.

\uventry{ombro} ombre | shadow | Schatten | тѣнь | cień.

\uventry{preĝi} prier (Dieu) | pray | beten | молиться | modlić się.

\uventry{virga} virginal | virginal | jungfräulich | дѣвственный | dziewiczy.

\end{ekzvocab}


\ekzsec{§38.}

Mi aĉetis por la infanoj tableton kaj kelke da seĝetoj. ― En nia lando sin ne trovas montoj, sed nur montetoj. ― Tuj post la hejto la forno estis varmega, post unu horo ĝi estis jam nur varma, post du horoj ĝi estis nur iom varmeta, kaj post tri horoj ĝi estis jam tute malvarma. ― En somero ni trovas malvarmeton en densaj arbaroj. ― Li sidas apud la tablo kaj dormetas. ― Mallarĝa vojeto kondukas tra tiu ĉi kampo al nia domo. ― Sur lia vizaĝo mi vidis ĝojan rideton. ― Kun bruo oni malfermis la pordegon, kaj la kaleŝo enveturis en la korton. Tio ĉi estis jam ne simpla pluvo, sed pluvego. ― Grandega hundo metis sur min sian antaŭan piedegon, kaj mi de teruro ne sciis, kion fari, ― Antaŭ nia militistaro staris granda serio da pafilegoj. ― Johanon, Nikolaon, Erneston, Vilhelmon, Marion, Klaron kaj Sofion iliaj gepatroj nomas Johanĉjo (aŭ Joĉjo), Nikolĉjo (aŭ Nikoĉjo aŭ Nikĉjo aŭ Niĉjo), Erneĉjo (aŭ Erĉjo), Vilhelĉjo (aŭ Vilheĉjo aŭ Vilĉjo aŭ Viĉjo), Manjo (aŭ Marinjo), Klanjo kaj Sonjo (aŭ Sofinjo).

\begin{ekzvocab}{1em}
\uventry{densa} épais, dense | dense | dicht | густой | gęsty.

\uventry{brui} faire du bruit | make a noise | lärmen, brausen | шумѣть | szumieć, hałasować.

\uventry{kaleŝo} carosse, calèche | carriage | Wagen | коляска | powóz.

\uventry{pluvo} pluie | rain | Regen | дождь | deszcz.

\uventry{pafi} tirer, faire feu | shoot | schiessen | стрѣлять | strzelać.
\vspace{2pt}

\begin{minipage}{\textwidth}
\begin{wrapfigure}[2]{l}[1.5ex]{1.5em}
\vspace{-3.6ex}
\begin{tabularx}{1.5em}{l@{ }l}
\footnotesize \uventry{cj} & \footnotesize \bf \trbb \\
\footnotesize \uventry{nj} & \\
\end{tabularx}%
\end{wrapfigure}
après les 1-6 premières lettres d’un prénom masculin (\uventry{nj} ― féminin) lui donne un caractère diminutif et caressant | affectionate diminutive of masculine (\uventry{nj} ― feminine) names | den ersten 1-6 Buchstaben eines männlichen (\uventry{nj} ― weiblichen) Eigennamens beigefügt, verwandelt diesen in ein Kosewort | приставленное къ первымъ 1-6 буквамъ имени собственнаго мужескаго (\uventry{nj} ― женскаго) пола, превращаетъ его въ ласкательное | dodane do pierwszych 1-6 liter imienia własnego męskiego (\uventry{nj} ― żenńskiego) rodzaju zmienia je w pieszczotliwe.
\end{minipage}

\end{ekzvocab}

\ekzsec{§39.}

En la kota vetero mia vesto forte malpuriĝis; tial mi prenis broson kaj purigis la veston. ― Li paliĝis de timo kaj poste li ruĝiĝis de honto. ― Li fianĉiĝis kun fraŭlino Berto; post tri monatoj estos la edziĝo; la edziĝa soleno estos en la nova preĝejo, kaj la edziĝa festo estos en la domo de liaj estontaj bogepatroj. ― Tiu ĉi maljunulo tute malsaĝiĝis kaj infaniĝis. ― Post infekta malsano oni ofte bruligas la vestojn de la malsanulo. ― Forigu vian fraton, ĉar li malhelpas al ni. ― Ŝi edziniĝis kun sia kuzo, kvankam ŝiaj gepatroj volis ŝin edzinigi kun alia persono. ― En la printempo la glacio kaj la neĝo fluidiĝas. ― Venigu la kuraciston, ĉar mi estas malsana. ― Li venigis al si el Berlino multajn librojn. ― Mia onklo ne mortis per natura morto, sed li tamen ne mortigis sin mem kaj ankaŭ estis mortigita de neniu; unu tagon, promenante apud la reloj de fervojo, li falis sub la radojn de veturanta vagonaro kaj mortiĝis. ― Mi ne pendigis mian ĉapon sur tiu ĉi arbeto; sed la vento forblovis de mia kapo la ĉapon, kaj ĝi, flugante, pendiĝis sur la branĉoj de la arbeto. ― Sidigu vin (aŭ sidiĝu), sinjoro! ― La junulo aliĝis al nia militistaro kaj kuraĝe batalis kune kun ni kontraŭ niaj malamikoj.

\begin{ekzvocab}{1em}
\uventry{koto} boue | dirt | Koth, Schmutz | грязь | błoto.

\uventry{broso} brosse | brush | Bürste | щетка | szczotka.

\uventry{ruĝa} rouge | red | roth | красный | czerwony.

\uventry{honti} avoir honte | be ashamed | sich schämen | стыдиться | wstydzić się.

\uventry{solena} solennel | solemn | feierlich | торжественный | uroczysty.

\uventry{infekti} infecter | infect | anstecken | заражать | zarażać.

\uventry{printempo} printemps | spring time | Frühling | весна | wiosna.

\uventry{relo} rail | rail | Schiene | рельса | szyna.

\uventry{rado} roue | wheel | Rad | колесо | koło (od woza i t. p.).

\uventry{pendi} pendre, être suspendu | hang | hängen (v. n.) | висѣть | wisieć.

\uventry{ĉapo} bonnet | bonnet | Mütze | шапка | czapka.

\uventry{vento} vent | wind | Wind | вѣтеръ | wiatr.

\uventry{blovi} souffler | blow | blasen, wehen | дуть | dąć, dmuchać.

\uventry{kapo} tête | head | Kopf | голова | głowa.

\uventry{branĉo} branche | branch | Zweig | вѣтвь | gałaź.

\end{ekzvocab}


\ekzsec{§40.}

En la daŭro de kelke da minutoj mi aŭdis du pafojn. ― La pafado daŭris tre longe. ― Mi eksaltis de surprizo. ― Mi saltas tre lerte. ― Mi saltadis la tutan tagon de loko al loko. ― Lia hieraŭa parolo estis tre bela, sed la tro multa parolado lacigas lin. ― Kiam vi ekparolis, ni atendis aŭdi ion novan, sed baldaŭ ni vidis, ke ni trompiĝis, ― Li kantas tre belan kanton. ― La kantado estas agrabla okupo. ― La diamanto havas belan brilon. ― Du ekbriloj de fulmo trakuris tra la malluma ĉielo. ― La domo, en kiu oni lernas, estas lernejo, kaj la domo, en kiu oni preĝas, estas preĝejo. ― La kuiristo sidas en la kuirejo. ― La kuracisto konsilis al mi iri en ŝvitbanejon. ― Magazeno, en kiu oni vendas cigarojn, aŭ ĉambro, en kiu oni tenas cigarojn, estas cigarejo; skatoleto aŭ alia objekto, en kiu oni tenas cigarojn, estas cigarujo; tubeto, en kiun oni metas cigaron, kiam oni ĝin fumas, estas cigaringo. ― Skatolo, en kiu oni tenas plumojn, estas plumujo, kaj bastoneto, sur kiu oni tenas plumon por skribado, estas plumingo. ― En la kandelingo sidis brulanta kandelo. ― En la poŝo de mia pantalono mi portas monujon, kaj en la poŝo de mia surtuto mi portas paperujon; pli grandan paperujon mi portas sub la brako. ― La rusoj loĝas en Rusujo kaj la germanoj en Germanujo.

\begin{ekzvocab}{1em}
\uventry{surprizi} surprendre | surprise | überraschen | дѣлать сюрпризъ | robić niespodzianki.

\uventry{laca} las, fatigué | weary | müde | усталый | zmęczony.

\uventry{trompi} tromper, duper | deceive, cheat | betrügen | обманывать | oszukiwać.

\uventry{fulmo} éclair | lightning | Blitz | молнія | błyskawica.

\uventry{ŝviti} suer | perspire | schwitzen | потѣть | pocić się.

\uventry{bani} baigner | bath | baden | купать | kąpać.

\uventry{magazeno} magazin | store | Kaufladen | лавка, магазинъ | sklep, magazyn.

\uventry{vendi} vendre | sell | verkaufen | продавать | sprzedawać.

\uventry{cigaro} cigare | cigar | Cigarre | сигара | cygaro.

\uventry{tubo} tuyau | tube | Röhre | труба | rura.

\uventry{fumo} fumée | smoke | Rauch | дымъ | dym.

\uventry{ing} marque l’objet dans lequel se met, ou mieux s’introduit\ldots{}; ex. \uventry{kandelo} chandelle ― \uventry{kandelingo} chandelier | holder for; e.~g. \uventry{kandelo} candle ― \uventry{kandelingo} candle-stick | Gegenstand, in den etwas eingestellt, eingesetzt wird; z.~B. \uventry{kandelo} Kerze ― \uventry{kandelingo} Leuchter | вещь, въ которую вставляется, всаживается; напр. \uventry{kandelo} свѣча ― \uventry{kandelingo} подсвѣчникъ | przedmiot, w który się coś wsadza, wstawia; np. \uventry{kandelo} świeca ― \uventry{kandelingo} lichtarz.

\uventry{skatolo} boîte | small box, case | Büchse, Schachtel | коробка | pudełko.

\uventry{pantalono} pantalon | pantaloons, trowsers | Hosen | брюки | spodnie.

\uventry{surtuto} redingote | over-coat | Rock | сюртукъ | surdut.

\uventry{brako} bras | arm | Arm | рука, объятія | ramię.

\end{ekzvocab}


\ekzsec{§41.}

Ŝtalo estas fleksebla, sed fero ne estas fleksebla. ― Vitro estas rompebla kaj travidebla. ― Ne ĉiu kreskaĵo estas manĝebla. ― Via parolo estas tute nekomprenebla kaj viaj leteroj estas ĉiam skribitaj tute nelegeble. ― Rakontu al mi vian malfeliĉon, ĉar eble mi povos helpi al vi. ― Li rakontis al mi historion tute nekredeblan. ― Ĉu vi amas vian patron? Kia demando! kompreneble, ke mi lin amas. ― Mi kredeble ne povos veni al vi hodiaŭ, ĉar mi pensas, ke mi mem havos hodiaŭ gastojn. ― Li estas homo ne kredinda. ― Via ago estas tre laŭdinda. ― Tiu ĉi grava tago restos por mi ĉiam memorinda. ― Lia edzino estas tre laborema kaj ŝparema, sed ŝi estas ankaŭ tre babilema kaj kriema. Li estas tre ekkolerema kaj ekscitiĝas ofte ĉe la plej malgranda bagatelo; tamen li estas tre pardonema, li ne portas longe la koleron kaj li tute ne estas venĝema. ― Li estas tre kredema: eĉ la plej nekredeblajn aferojn, kiujn rakontas al li la plej nekredindaj homoj, li tuj kredas. ― Centimo, pfenigo kaj kopeko estas moneroj. ― Sablero enfalis en mian okulon. ― Li estas tre purema, kaj eĉ unu polveron vi ne trovos sur lia vesto. ― Unu fajrero estas sufiĉa, por eksplodigi pulvon.

\begin{ekzvocab}{1em}
\uventry{ŝtalo} acier | steet | Stahl | сталь | stal.

\uventry{fleksi} fléchir, ployer | bend | biegen | гнуть | giąć.

\uventry{vitro} verre (matière) | glass (substance) | Glas | стекло | szkło.

\uventry{rompi} rompre, casser | break | brechen | ломать | łamać.

\uventry{laŭdi} louer, vanter | praise | loben | хвалить | chwalić.

\uventry{memori} se souvenir, se rappeler | remember | im Gedächtniss behalten, sich erinnern | помнить | pamiętać.

\uventry{ŝpari} ménager, épargner | be sparing | sparen | сберегать | oszczędzać.

\uventry{bagatelo} bagatelle | trifle, toy | Kleinigkeit | мелочь, бездѣлица | drobnostka.

\uventry{venĝi} se venger | revenge | rächen | мстить | mścić się.

\uventry{eksciti} exciter, émouvoir | excite | erregen | возбуждать | wzbudzać.

\uventry{er} marque l’unité; ex. \uventry{sablo} sable ― \uventry{sablero} un grain de sable | one of many objects of the same kind; e.~g. \uventry{sablo} sand ― \uventry{sablero} grain of sand | ein einziges; z.~B. \uventry{sablo} Sand ― \uventry{sablero} Sandkörnchen | отдѣльная единица; напр. \uventry{sablo} песокъ ― \uventry{sablero} песчинка | oddzielna jednostka; np. \uventry{sablo} piasek ― \uventry{sablero} ziarnko piasku.

\uventry{polvo} poussière | dust | Staub | пыль | kurz.

\uventry{fajro} feu | fire | Feuer | огонь | ogień.

\uventry{eksplodi} faire explosion | explode | explodiren | взрывать | wybuchać.

\uventry{pulvo} poudre à tirer | gunpowder | Pulver (Schiess-) | порохъ | proch.

\end{ekzvocab}


\ekzsec{§42.}

Ni ĉiuj kunvenis, por priparoli tre gravan aferon; sed ni ne povis atingi ian rezultaton, kaj ni disiris. ― Malfeliĉo ofte kunigas la homojn, kaj feliĉo ofte disigas ilin. ― Mi disŝiris la leteron kaj disĵetis ĝiajn pecetojn en ĉiujn angulojn de la ĉambro. ― Li donis al mi monon, sed mi ĝin tuj redonis al li. ― Mi foriras, sed atendu min, ĉar mi baldaŭ revenos. ― La suno rebrilas en la klara akvo de la rivero. ― Mi diris al la reĝo: via reĝa moŝto, pardonu min! ― El la tri leteroj unu estis adresita: al Lia Episkopa Moŝto, Sinjoro N.; la dua: al Lia Grafa Moŝto, Sinjoro P.; la tria: al Lia Moŝto, Sinjoro D. ― La sufikso «um» ne havas difinitan signifon, kaj tial la (tre malmultajn) vortojn kun «um» oni devas lerni, kiel simplajn vortojn. Ekzemple: plenumi, kolumo, manumo. ― Mi volonte plenumis lian deziron. ― En malbona vetero oni povas facile malvarmumi. ― Sano, sana, sane, sani, sanu, saniga, saneco, sanilo, sanigi, saniĝi, sanejo, sanisto, sanulo, malsano, malsana, malsane, malsani, malsanulo, malsaniga, malsaniĝi, malsaneta, malsanema, malsanulejo, malsanulisto, malsanero, malsaneraro, sanigebla, sanigisto, sanigilo, resanigi, resaniĝanto, sanigilejo, sanigejo, malsanemulo, sanilaro, malsanaro, malsanulido, nesana, malsanado, sanulaĵo, malsaneco, malsanemeco, saniginda, sanilujo, sanigilujo, remalsano, remalsaniĝo, malsanulino, sanigista, sanigilista, sanilista, malsanulista k.~t.~p.

\begin{ekzvocab}{1em}
\uventry{atingi} atteindre | attain, reach | erlangen, erreichen | достигать | dosięgać.

\uventry{rezultato} résultat | result | Ergebniss | результатъ | rezultat.

\uventry{ŝiri} déchirer | tear, rend | reissen | рвать | rwać.

\uventry{peco} morceau | piece | Stück | кусокъ | kawał.

\uventry{moŝto} titre commun | universal title | allgemeiner Titel | общій титулъ | Mość.

\begin{minipage}{\textwidth}
\uventry{episkopo} évêque | bishop | Bischof | епископъ | biskup.

\uventry{grafo} comte | earl, count | Graf | графъ | hrabia.

\uventry{difini} définir, déterminer | define | bestimmen | опредѣлять | wyznaczać, określać.
\end{minipage}
\end{ekzvocab}




\vspace{1ex}

\leftpointright{} La plej grava libro por ĉiu, kiu deziras perfectiĝi en la lingvo Esperanto, estas la verko
\begin{center}
\gramsec{Fundamenta Krestomatio,}
\end{center}

{\setlength{\parindent}{0pt} kiu konsistas el 458 paĝoj kaj enhavas tre multe da plej diversa materialo por legado \textit{en la plej modela Esperanta stilo.} Kosto fr. 3,50. Ricevebla ĉe Hachette kaj K\textsuperscript{io} en Parizo kaj en ĉiu esperantista librovendejo.}

{\centering\rule{13mm}{0.4pt}\par}

\cleardoublepage

% Universala Vortaro
%
% Universala Vortaro
%
\label{vortaro}
\markboth{FUNDAMENTO DE ESPERANTO}{UNIVERSALA VORTARO}
\thispagestyle{plain}
\vspace*{\fill}
\begin{center}

\narrow{\Large\spaceoutless{UNIVERSALA VORTARO}}
\vspace{1em}

{\small DE LA}
\vspace{1em}

\narrow{\large\spaceoutless{LINGVO INTERNACIA}}
\vspace{2em}

{\fontsize{30}{36}\selectfont\bookman{\spaceout{ESPERANTO}}}
\end{center}
\vspace*{\fill}
\newpage

\footnotesize

Ĉion, kio estas skribita en la lingvo internacia Esperanto, oni povas kompreni kun helpo de tiu ĉi vortaro. Vortoj, kiuj formas kune unu ideon, estas skribataj kune, sed dividataj unu de la alia per streketo, tiel ekzemple la vorto «~\uventry{frat$'$in$'$o}~», prezentante unu ideon, estas kunmetita el tri vortoj, el kiuj ĉiun oni devas serĉi aparte.

\flatsmallrule{}

Tout ce qui est écrit en langue internationale Esperanto peut se comprendre à l’aide de ce dictionnaire. Les mots qui forment ensemble une seule idée s’écrivent ensemble mais se séparent les uns des autres par de petits traits. Ainsi, par exemple, le mot «~\uventry{frat$'$in$'$o}~», qui n’exprime qu’une idée est formé de trois mots, et chacun d’eux se cherche à part.

\flatsmallrule{}

Everything written in the international language Esperanto can be translated by means of this vocabulary. If several words are required to express one idea, they must be written in one but, separated by commas; e.~g. «~\uventry{frat$'$in$'$o}~» though one idea, is yet composed of three words, which must be looked for separately in the vocabulary.

\flatsmallrule{}

Alles, was in der internationalen Sprache Esperanto geschrieben ist, kann man mit Hülfe dieses Wörterbuches verstehen. Wörter, welche zusammen einen Begriff bilden, werden zusammen geschrieben, aber von einander, durch einen senkrechten Strich getrennt; so ist z.~B. das Wort «~\uventry{frat$'$in$'$o}~», welches einen Begriff bildet, aus drei Wörtern zusammengesetzt, deren jedes besonders zu suchen ist.

\flatsmallrule{}

Все, что написано на международномъ языкѣ Эсперанто, можно понимать съ помощью этого словаря. Слова составляющія вмѣстѣ одно понятіе, пишутся вмѣстѣ, но отдѣляются другъ отъ друга черточкой; такъ напримѣръ слово «~\uventry{frat$'$in$'$o}~», составляя одно понятіе, сложено изъ трехъ словъ, изъ которыхъ каждое надо искать отдѣльно.

\flatsmallrule{}

Wszystko co napisano w języku międzynarodowyn Esperanto, można zrozumieć przy pomocy tego słownika. Wyrazy, stanowiące razem jedno pojęcie, pisze się razem, lecz oddziela się kréską pionową; tak naprzykład wyraz «~\uventry{frat$'$in$'$o}~» stanowiący jedno pojęcie, złożony jest z trzech wyrazów, z których każdego należy szukać oddzielnie.

\newpage

\begin{outdent}{1em}
\uvlitero{A}

\uventry{a}
marque l’adjectif; ex. \uventry{hom$'$} homme ― \uventry{hom$'$a}
humain | termination of adjectives; e. g. \uventry{hom$'$} man ―
\uventry{hom$'$a} human | bezeichnet das Adjektiv; z. B. \uventry{hom$'$} Mensch ―
\uventry{hom$'$a} menschlich | означаетъ прилагательное; напр. \uventry{hom$'$} человѣкъ ―
\uventry{hom$'$a} человѣческій | oznacza przymiotnik; np. \uventry{hom$'$} człowiek ― \uventry{hom$'$a} ludzki.

\uventry{abat$'$}
abbé | abbot | Abt | аббатъ | opat.

\uventry{abel$'$}
abeille | bee | Biene | пчела | pszczoła.

\uventry{abi$'$}
sapin | fir | Tanne | ель | jodła.

\uventry{abomen$'$}
abomination | abomination | Abscheu | отвращеніе | odraza.

\uventry{abon$'$}
abonner | subscribe | abonniren | подписываться | prenumerować.

\uventry{ablativ$'$}
ablatif | ablative | Ablativ | творительный падежъ | narzędnik.

\uventry{abrikot$'$}
abricot | apricot | Aprikose | абрикосъ | morela.

\uventry{absces$'$}
abcès | abscess | Geschwür, Eiterbeule | нарывъ | wrzód.

\uventry{absint$'$}
absinthe | absinthium | Wermuth | полынь | piołunkówka.

\uventry{acer$'$}
érable | maple | Ahorn | кленъ | klon.

\uventry{aĉet$'$}
acheter | buy | kaufen | покупать | kupować.

\uvsubentry{}\uventry{sub$'$aĉet$'$}
corrompre | corrupt | bestechen | подкупать | przekupyvać.

\uventry{acid$'$}
aigre | sour | sauer | кислый | kwaśny.

\uventry{ad$'$}
marque durée dans l’action; ex. \uventry{paf$'$} coup de fusil 
― \uventry{paf$'$ad$'$} fusillade | denotes duration of action; e. g. \uventry{danc$'$}
dance ― \uventry{danc$'$ad$'$} dancing | bezeichnet die Dauer der Thätigkeit;
z. B. \uventry{danc$'$} der Tanz ― \uventry{danc$'$ad$'$} das Tanzen | означаетъ 
продолжительность дѣйствія; напр. \uventry{ir$'$} идти ― \uventry{ir$'$ad$'$} ходить, 
хаживать | oznacza trwanie czynności; np. \uventry{ir$'$} iść ― \uventry{ir$'$ad$'$}
chodzić.

\uventry{adiaŭ}
adieu | good-by | lebe wohl | прощай | bądź zdrów.

\uventry{adjektiv$'$}
adjectif | adjective | Eigenschaftswort | имя прилагательное | przymiotnik.

\uventry{administr$'$}
admininistrer | administer | verwalten | управлять | zarządzać.

\uventry{admir$'$}
admirer | admire | bewundern | дивиться | podziwiać.

\uventry{admon$'$}
exhorter | exhort | ermahnen | увѣщевать | upominać.

\uventry{ador$'$}
adorer | adore | anbeten | обожать | uwielbiać.

\uventry{adult$'$}
adultérer | adulterate | ehebrechen | прелюбодѣйствовать | cudzołożyć.

\uventry{adverb$'$}
adverbe | adverb | Nebenwort | нарѣчіе | przysłówek.

\uventry{aer$'$}
air | air | Luft | воздухъ | powietrze.

\uvsubentry{}\uventry{aer$'$um$'$}
aérer | expose to the air | lüften | провѣтривать | przewietrzać.

\uventry{afabl$'$}
affable | affable | freundlich | ласковый | uprzejmy.

\uvsubentry{}\uventry{mal$'$afabl$'$}
grogneur | surly | mürrisch | угрюмый | mrukliwy.

\uventry{afekt$'$}
affectionner | affect | affectiren | жеманиться | afektować.

\uventry{afer$'$}
affaire | affair | Sache, Angelegenheit | дѣло | sprawa.

\uventry{ag$'$}
agir | act | handeln, verfahren | поступать | postępować.

\uventry{aĝ$'$}
âge | age | Alter | вѣкъ, возрастъ | wiek.

\uventry{agac$'$}
agacement | setting on edge | Stumpfwerden der Zähne | оскомина | drętwość.

\uventry{agl$'$}
aigle | eagle | Adler | орелъ | orzeł.

\uventry{agord$'$}
accorder | tune | stimmen | настраивать | nastrajać.

\uventry{agrabl$'$}
agreable | agreeable | angenehm | пріятный | przyjemny.

\uventry{aĵ$'$}
quelque chose possédant une certaine qualité ou fait d’une 
certaine matière: ex. \uventry{mol$'$} mou ― \uventry{mol$'$aĵ$'$} partie molle d’une
chose | made from or possessing the quality of; e. g. \uventry{sek$'$} dry ―
\uventry{sek$'$aĵ$'$} dry goods | etwas von einer gewissen Eigenschaft, oder aus
einem gewissen Stoffe; z. B. \uventry{mal$'$nov$'$} alt ― \uventry{mal$'$nov$'$aĵ$'$} altes
Zeug, \uventry{frukt$'$} Frucht ― \uventry{frukt$'$aĵ$'$} etwas aus Früchten bereitetes | нѣчто
съ данымъ качествомъ или изъ даннаго матеріала; нпр. \uventry{mol$'$}
мягкій ― \uventry{mol$'$aĵ$'$} мякишъ; \uventry{frukt$'$} плодъ ― \uventry{frukt$'$aĵ$'$} нѣчто
приготовленное изъ плодовъ | oznacza przedmiot posiadajacy pewną
własność albo zrobiony z pewnego materjału; np. \uventry{mal$'$nov$'$} stary ―
\uventry{mal$'$nov$'$aĵ$'$} starzyzna; \uventry{frukt$'$} owoc ― \uventry{frukt$'$aĵ$'$} coś zrobionego z owoców.

\uventry{ajl$'$}
ail | onion, garlic | Knoblauch | чеснокъ | czosnek.

\uventry{ajn}
que ce soit; ex. \uventry{kiu} qui ― \uventry{kiu ajn} qui que ce soit | ever; e.g.
\uventry{kiu} who ― \uventry{kiu ajn} whoever | auch nur; z. B. \uventry{kiu} wer ― \uventry{kiu
ajn} wer auch nur | бы ни; напр. \uventry{kiu} кто ― \uventry{kiu ajn} кто бы ни | kolwiek, bądź; np. \uventry{kiu} kto ― \uventry{kiu ajn} ktokolwiek, ktobądź.

\uventry{akar$'$}
mite | mite | Milbe | клещъ, червь | kleszcz, ślepak.

\uventry{akcel$'$}
dépêcher | accelerate | fördern | споспѣшествовать | popierać, przyspieszać.

\uventry{akcent$'$}
accent | accent | Accent | удареніе | akcent.

\uventry{akcept$'$}
accepter | accept | annehmen | принимать | przyjmować.

\uventry{akcipitr$'$}
autour | hawk | Habicht | ястребъ | jastrząb.

\uventry{akir$'$}
acquérir | acquire | erwerben | пріобрѣтать | uzyskać.

\uventry{akn$'$}
bouton, grains de ladrerie | pimple | Finne | угорь (сыпь) | węgry, krosty.

\uventry{akompan$'$}
accompagner | accompany | begleiten | сопровождать | towarzyszyć.

\uventry{akr$'$}
aigu | sharp | scharf | острый | ostry.

\uventry{akrid$'$}
sauterelle | grass-hopper | Heuschrecke | саранча | szarańcza.

\uventry{aks$'$}
axe | axle | Achse | ось | oś.

\uventry{akuŝ$'$}
accoucher | lie in | niederkommen, entbunden werden | разрѣшится отъ бремени | powić.

\uvsubentry{}\uventry{akuŝ$'$ist$'$in$'$}
sage-femme | midwife | Hebamme | акушерка | akuszerka.

\uventry{akuzativ$'$}
accusatif | accusative | Accusativ | винительный падежъ | biernik.

\uventry{akv$'$}
eau | water | Wasser | вода | woda.

\uventry{al}
à | to | zu (ersetzt zugleich den Dativ) | къ (замѣняетъ также
дательный падежъ) | do (zastępuje też celownik).

\uventry{alaŭd$'$}
alouette | lark | Lerche | жаворонокъ | skowronek.

\uventry{alcion$'$}
alcyon | halcyon | Eisvogel | зимородокъ | zimorodek.

\uventry{ali$'$}
autre | other | ander | иной | inny.

\uventry{alk$'$}
élan | elk | Elennthier | лось | łoś.

\uventry{almenaŭ}
au moins | at least | wenigstens | по крайней мѣрѣ | przynajmniej.

\uventry{almoz$'$}
aumône | alms | Almosen | милостыня | jałmużna.

\uventry{aln$'$}
aune | alder | Erle | ольха | olsza.

\uventry{alt$'$}
haut | high | hoch | высокій | wysoki.

\uventry{altar$'$}
autel | altar | Altar | алтарь | ołtarz.

\uventry{alte$'$}
althée | althee | Eibisch | проскурнякъ | ślaz.

\uventry{altern$'$}
alterner | alternate | untereinander abwechseln | чередоваться | zmianiać się kolejno.

\uventry{alud$'$}
faire allusion | allude | anspielen | намекать | dawać
przytyk.

\uventry{alumet$'$}
allumette | match | Zündhölzchen | спичка | zapałka.

\uventry{alun$'$}
alun | alum | Alaun | квасцы | ałun.

\uventry{am$'$}
aimer | love | lieben | любить | kochać, lubić.

\uventry{amas$'$}
amas, foule | crowd, mass | Haufen, Menge | куча, толпа | kupa, tłum.

\uventry{ambaŭ}
l’un et l’autre | both | beide | оба | obaj.

\uventry{ambos$'$}
enclume | anvil | Amboss | наковальня | kowadło.

\uventry{amel$'$}
amidon | starch | Stärkemehl | крахмалъ | krochmal.

\uventry{amfibi$'$}
amphibie | amphibium | Amphibie | земноводное животное | płaz.

\uventry{amik$'$}
ami | friend | Freund | друг | przyjaciel.

\uventry{am$'$ind$'$um$'$}
courtiser | court, make love | den Hof machen | любезничать | umizgać się.

\uventry{amoniak$'$}
ammoniac | ammoniac | Ammoniak, Salmiak | нашатырный спиртъ | amoniak.

\uventry{ampleks$'$}
extension | extension | Umfang | объемъ | objętość.

\uventry{amuz$'$}
amuser | amuse | belustigen | забавлять | bawić.

\uventry{an$'$}
membre, habitant, partisan; ex. \uventry{regn$'$} l’état ― \uventry{regn$'$an$'$}
citoyen | inhabitant, member; e. g. \uventry{Nov-Jork} New York ― \uventry{Nov-Jork$'$an$'$}
New Yorker | Mitglied, Einwohner, Anhänger; z. B. \uventry{regn$'$}
Staat ― \uventry{regn$'$an$'$} Bürger; \uventry{Varsovi$'$an$'$} Warschauer | членъ, житель,
приверженец; напр. \uventry{regn$'$} государство ― \uventry{regn$'$an$'$} гражданинъ;
\uventry{Varsovi$'$an$'$} Варшавянинъ | członek, mieszkaniec, zwolennik; np. \uventry{regn$'$}
państwo ― \uventry{regn$'$an$'$} obywatel; \uventry{Varsovi$'$an$'$} Warszawianin.

\uvsubentry{}\uventry{an$'$ar$'$}
troupe | troop | Trupp, Truppe | труппа | gromada, trupa.

\uventry{ananas$'$}
ananas | ananas | Ananas | ананасъ | ananas.

\uventry{anas$'$}
canard | duck | Ente | утка | kaczka.

\uvsubentry{}\uventry{mol$'$anas$'$}
canard à duvet | eider-duck | Eider-ente | гага | miękopiór.

\uventry{anĝel$'$}
ange | angel | Engel | ангелъ | anioł.

\uventry{angil$'$}
anguille | eel | Aal | угорь (животное) | węgorz.

\uventry{angul$'$}
angle | corner, angle | Winkel | уголъ | kąt.

\uventry{anim$'$}
âme | soul | Seele | душа | dusza.

\uventry{aniz$'$}
anis | anise | Anis | анисъ | anyż.

\uventry{ankaŭ}
aussi | also | auch | также | także.

\uventry{ankoraŭ}
encore | yet, still | noch | еще | jeszcze.

\uventry{ankr$'$}
ancre | anchor | Anker | якорь | kotwica.

\uventry{anonc$'$}
annoncer | annonce, advert | annonciren | объявлять | ogłaszać.

\uventry{anser$'$}
oie | goose | Gans | гусь | gęś.

\uventry{anstataŭ}
au lieu de | instead | anstatt, statt | вмѣсто | zamiast.

\uvsubentry{}\uventry{anstataŭ$'$i}
remplacer | replace | ersetzen | замѣнять | zastępować.

\uventry{ant$'$}
marque le participe présent actif | ending of
pres. part. act. in verbs | bezeichnet das Participium pres. act. | означаетъ причастіе настоящаго времени дѣйст. залога | oznacza
imiesłów czynny czasu teraźniejszego.

\uventry{antaŭ}
devant | before | vor | предъ | przed.

\uvsubentry{}\uventry{antaŭ$'$tuk$'$}
tablier, devantier | apron | Schürze | передникъ | fartuch.

\uventry{antikv$'$}
antique | antique | alt, altethümlich | древній | starożytny.

\uventry{antimon$'$}
antimoine | antimony | Antimon | сурьма | antymon.

\uventry{Anunciaci$'$}
Annonciation de la Vierge | Annunciation | Verkündigung
Mariä | Благовѣщеніе | Zwiastowanie.

\uventry{apart$'$}
qui est à part, separé | separate | besonder, abgesondert | особый | osobny.

\uventry{aparten$'$}
appartenir | belong | gehören | принадлежать | należeć.

\uventry{apenaŭ}
à peine | scarcely | kaum | едва | ledwie.

\uventry{aper$'$}
paraître, apparaître | appear | erscheinen | являться | zjawiać się.

\uvsubentry{}\uventry{mal$'$aper$'$}
disparaître | dissappear | verschwinden | исчезать | znikać.

\uventry{aplaŭd$'$}
applaudir | applaud | applaudiren | аплодировать | oklaskiwać.

\uventry{apog$'$}
appuyer | lean | anlehnen | опирать | opierać.

\uventry{apr$'$}
sanglier | wild boar | Eber | боровъ, вепрь, кабанъ | wieprz.

\uventry{April$'$}
Avril | April | April | Апрѣль | Kwiecień.

\uventry{aprob$'$}
approuver | approve | billigen, gut heissen | одобрять | aprobować.

\uventry{apud}
auprès de | near by | neben, an | возлѣ, при | przy, obok.

\uventry{ar$'$}
une réunion de certains objets; ex. \uventry{arb$'$} arbre ― \uventry{arb$'$ar$'$}
forêt | a collection of objects; e. g. \uventry{vort$'$} word ― \uventry{vort$'$ar$'$}
dictionary | Sammlung gewisser Gegenstände; z. B. \uventry{arb$'$} Baum ―
\uventry{arb$'$ar$'$} Wald; \uventry{ŝtup$'$} Stufe ― \uventry{ŝtup$'$ar$'$} Treppe, Leiter | собраніе данныхь предметовъ; напр. \uventry{arb$'$} дерево ― \uventry{arb$'$ar$'$} лѣсъ,
\uventry{ŝtup$'$} ступень ― \uventry{ŝtup$'$ar$'$} лѣстница | oznacza zbiór danych przedmiotów; np. \uventry{arb$'$} drzewo ― \uventry{arb$'$ar$'$} las;
\uventry{ŝtup$'$} szczebel, stopień ― \uventry{ŝtup$'$ar$'$} drabina, schody.

\uventry{arane$'$}
araignée | spider | Spinne | паукъ | pająk.

\uventry{aranĝ$'$}
arranger | arrange | einrichten | устроивать | urządzać.

\uventry{arb$'$}
arbre | tree | Baum | дерево | drzewo.

\uvsubentry{}\uventry{arb$'$et$'$aĵ$'$}
arbrisseau, buisson | shrub, bush | Strauch | кустъ | krzak.

\uventry{arbitr$'$}
arbitraire | arbitrary | willkürlich | произвольный | samowolny.

\uventry{arĉ$'$}
archet | bow (fiddle) | Violinbogen | смычекъ | smyczek.

\uventry{arde$'$}
héron | heron | Reiher | цапля | czapla.

\uventry{ardez$'$}
ardoise | slate | Schiefer | аспидъ (минералъ) | łupek.

\uventry{aren$'$}
arène | wrestling-place | Arena, Rennbahn | арена, ристалище | arena.

\uventry{arest$'$}
arrêter | arrest | verhaften | арестовать | aresztować.

\uventry{arĝent$'$}
argent (métal) | silver | Silber | серебро | srebro.

\uventry{argil$'$}
argile | clay | Thon | глина | glina.

\uventry{argument$'$}
argumenter | argue | beweisen | доказывать | dowodzić.

\uventry{arĥitektur$'$}
architecture | architecture | Architectur | архитектура | architektura.

\uventry{ark$'$}
arc | arch, bow | Bogen | дуга | łuk.

\uvsubentry{}\uventry{ark$'$aĵ$'$}
voûte | vault | Gewölbe | сводъ | sklepiente.

\uventry{arleken$'$}
arlequin | harlequin | Hanswurst | арлекинъ, шутъ | błazen,
arlekin.

\uventry{arm$'$}
armer | arm | rüsten | снаряжать, вооружать | uzbrajać.

\uventry{arogant$'$}
arrogant | arrogant | anmassend, hochmüthig | наглый,
высокомѣрный | zarozumiały.

\uventry{arsenik$'$}
arsenic | arsenic | Arsenik | мышьякъ | arszenik.

\uventry{art$'$}
art | art | Kunst | искусство | sztuka, kunszt.

\uventry{artifik$'$}
artifice | artifice | Kniff | уловка | fortel.

\uventry{artik$'$}
articulation | joint | Gelenk | суставъ | staw.

\uventry{artikol$'$}
article | article | Artikel | статья, членъ | artykuł.

\uventry{artiŝok$'$}
artichaut | artichoke | Artischoke | артишокъ | karczoch.

\uventry{artrit$'$}
goutte | gout | Gicht | ломота въ суставахъ | artretyzm.

\uventry{as}
marque le présent d’un verbe | ending of the present tense in
verbs | bezeichnet das Präsens | означаетъ настоящее время глагола | oznacza
czas teraźniejszy.

\uventry{as$'$}
as | ace | Ass (Kartsp.) | тузъ | as.

\uventry{asekur$'$}
assurer | insure | assecuriren | страховать | asekurować.

\uventry{asign$'$}
assigner | assign | anweissen | ассигновать | asygnować,
przekazać.

\uventry{asparag$'$}
asperge | asparagus | Spargel | спаржа | szparag.

\uventry{aspid$'$}
aspic | asp, adder | Natter | аспидъ (змѣй) | żmija.

\uventry{at$'$}
marque le participe présent passif | ending of
pres. part. pass. in verbs | bezeichnet das Participium praes. passivi | означаеть причастіе настоящаго времени страдат. залога | oznacza
imiesłów bierny czasu teraźniejszego.

\uventry{atak$'$}
attaquer | attack | angreifen | нападать | napadać, atakować.

\uventry{atenc$'$}
attenter | attempt | einen Anschlag (Attentat) machen | покушаться | robić zamach.

\uventry{atend$'$}
attendre | wait, expect | warten, erwarten | ждать, ожидать | czekać.

\uventry{atent$'$}
attentif | attentive | aufmerksam | внимательный | uważny.

\uventry{atest$'$}
témoigner | attest, affirm | zeugen, bezeugen, bescheinigen | свидѣтельствовать | świadczyć.

\uventry{ating$'$}
atteindre | attain, reach | erlangen, erreichen | достигать | dosięgać.

\uventry{atripl$'$}
patte d’oie | chenopodium | Melde | лебеда | łoboda.

\uventry{atut$'$}
atout | trump | Trumpf | козырь | kozera.

\uventry{aŭ}
ou | or | oder | или | albo, lub.

\uventry{aŭd$'$}
entendre | hear | hören | слышать | słyszeć.

\uventry{Aŭgust$'$}
Août | August | August | Августъ | Sierpień.

\uventry{aŭskult$'$}
écouter | listen | anhören, horchen | слушать | słuchać.

\uventry{aŭtun$'$}
automne | fall (of the year) | Herbst | осень | jesień.

\uventry{av$'$}
grand-père | grandfather | Grossvater | дѣдъ, дѣдушка | dziad.

\uventry{avar$'$}
avare | covetous | geizig | скупой | skąpy.

\uventry{avel$'$}
noisette, aveline | hazel-nut | Haselnuss | обыкновенный орѣхъ | orzech laskowy.

\uventry{aven$'$}
avoine | oats | Hafer | овесъ | owies.

\uventry{aventur$'$}
aventure | adventure | Abenteuer | приключеніе | przygoda.

\uventry{avert$'$}
avertir, prévenir | warn | warnen | предостерегать | przestrzegać.

\uventry{avid$'$}
avide, convoiteux | eager, covetous | gierig | жадный | chciwy.

\uventry{aviz$'$}
avis | advice | Avis | увѣдомленіе | awizacya.

\uventry{azen$'$}
âne | ass | Esel | оселъ | osioł.

\uventry{azot$'$}
azote | azot | Stickstoff | азоть | azot.

\uvlitero{B}

\uventry{babil$'$}
babiller | chatter | schwatzen, plaudern | болтать | paplać.

\uventry{bagatel$'$}
bagatelle | trifle, toy | Kleinigkeit | мелочь, бездѣлица | drobnostka.

\uventry{bajonet$'$}
baïonnette | bayonet | Bajonett | штыкъ | bagnet.

\uventry{bak$'$}
cuire | bake | backen | печь, испекать | piec, wypiekać.

\uventry{bala$'$}
balayer | sweep | fegen | мести, заметать | zamiatać.

\uventry{balanc$'$}
balancer | balance, swing | schaukeln | качать | huśtać, kołysać.

\uventry{balbut$'$}
bégayer, balbutier | stammer | stottern, stammeln | заикаться | jąkać się.

\uventry{baldaken$'$}
baldaquin | canopy | Baldachin | балдахинъ | baldachin.

\uventry{baldaŭ}
bientôt | soon | bald | сейчасъ, скоро | zaraz.

\uventry{balen$'$}
baleine | whale | Wallfisch | китъ | wieloryb.

\uvsubentry{}\uventry{balen$'$ost$'$}
baleine | whale-bone | Fischbein | китовый усъ | fiszbin.

\uventry{balustrad$'$}
garde-fou, balustrade | balustrade | Geländar | перила | poręcz.

\uventry{bambu$'$}
bambou | bamboo | Bambus | бамбукъ | bambus.

\uventry{ban$'$}
baigner | bath | baden | купать | kąpać.

\uventry{band$'$}
bande, troupe | band | Bande, Rotte | банда, шайка | banda, zgraja.

\uventry{bant$'$}
nœud | loop | Schlefe | бантъ | pętlica.

\uventry{bapt$'$}
baptiser | baptise | taufen | крестить | chrzcić.

\uvsubentry{}\uventry{bapt$'$an$'$}
compère | god-father | Gevatter | кумъ | kum.


\uvsubentry{}\uventry{bapt$'$o$'$fil$'$}
filleul | god-son | Taufsohn | крестный сынъ, крестникъ | chrześniak.


\uvsubentry{}\uventry{bapt$'$o$'$patr$'$}
parrain | god-father | Pathe, Taufvater | крестный отецъ | chrestny ojciec.

\uventry{bar$'$}
barrer | bar, obstruct | versperren | заграждать | zagradzać.

\uvsubentry{}\uventry{bar$'$il$'$}
haie | hedge | Zaum | заборъ | parkan.

\uventry{barakt$'$}
gigotler; piétiner | sprawl, trip | zappeln | барахтаться | trzepotać.

\uventry{barb$'$}
barbe | beard | Bart | борода | broda.

\uventry{barbir$'$}
barbier | barber | Barbier | цырульникъ, фельдшеръ | felczer.

\uventry{barĉ$'$}
soupe de betteraves | red-beet-soup | Beetensuppe | борщъ | barszcz.

\uventry{barel$'$}
tonneau | keg, barrel | Fass, Tonne | бочка | beczka.

\uventry{bark$'$}
barque | bark | Barke | барка | barka.

\uventry{bask$'$}
basque | lappet | Schoss, Rockschoss | пола | poła.

\uventry{bast$'$}
écorce d’arbre | bark (of a tree) | Bast | лубъ, лубокъ | łyko.

\uventry{bastion$'$}
bastion | bulwark | Bollwerk | бастіонъ | baszta, warownia.

\uventry{baston$'$}
bâton | stick | Stock | палка | kij, laska.

\uventry{bat$'$}
battre | beat | schlagen | бить | bić.

\uventry{batal$'$}
combattre | fight | kämpfen | бороться | walczyć.

\uvsubentry{}\uventry{batal$'$il$'$}
arme | weapon | Gewehr | оружіе | broń.

\uventry{bazar$'$}
marché foire | market fair | Markt | базаръ | targ, rynek.

\uventry{bed$'$}
couche | bed (garden) | Beet | гряда | grzęda, zagon.

\uventry{bedaŭr$'$}
regretter | pity, regret | bedauern | жалѣть | żałować.

\uventry{bek$'$}
bec | beak | Schnabel | клювъ | dziób.

\uventry{bel$'$}
beau | beautiful | schön, hübsch | красивый | piękny.

\uventry{beladon$'$}
belladonne | belladonna | Tollkirsche | сонная одурь | pokrzyk, wilcza jagoda.

\uventry{ben$'$}
bénir | bless | segnen | благословлять | błogosławić.

\uventry{benk$'$}
banc | bench | Bank (zum Sitzen) | скамья | ławka.

\uventry{ber$'$}
baie | berry | Beere | ягода | jagoda.

\uventry{best$'$}
animal | beast | Thier | животное | zwierzę.

\uventry{bet$'$}
betterave | red beet | Runkelrübe | свекловица | ćwikła,
burak.

\uventry{betul$'$}
bouleau | birch (tree) | Birke | береза | brzoza.

\uventry{bezon$'$}
avoir besoin de | need, want | brauchen | нуждаться | potrzebować.

\uventry{bien$'$}
bien | goods, estate | Gut, Landgut | имѣніе | posiadłość ziemska, majątek.

\uventry{bier$'$}
bière | beer | Bier | пиво | piwo.

\uventry{bind$'$}
relier | bind (books) | binden (Bücher) | переплетать | oprawiać.

\uventry{bird$'$}
oiseau | bird | Vogel | птица | ptak.

\uventry{biskvit$'$}
biscuit | biscuit | Zwieback | бисквитъ | sucharek.

\uventry{bismut$'$}
bismuth | bismuth | Bismuth | висмутъ | bismut.

\uventry{blank$'$}
blanc | white | weiss | бѣлый | biały.

\uventry{blat$'$}
blatte (ins) | scab | Schabe | тараканъ | karaczan.

\uventry{blek$'$}
beugler, hennir etc. | cry (of beasts) | blöken, wiehern etc. | мычать, блеять, ржать и т. п. | beczeć, rżeć.

\uventry{blind$'$}
aveugle | blind | blind | слѣпой | ślepy.

\uventry{blond$'$}
blond | fair | blond | русый, бѣлокурый | blondyn, płowy.

\uventry{blov$'$}
souffler | blow | blasen, wehen | дуть | dąć, dmuchać.

\uventry{blu$'$}
bleu | blue | blau | синій | niebieski.

\uventry{bo$'$}
marque la parenté resultant du mariage; ex. \uventry{patr$'$} père ―
\uventry{bo$'$patr$'$} beau-père | relation by marriage; e. g. \uventry{patr$'$in$'$} mother
― \uventry{bo$'$patr$'$in$'$} mother-in-law | durch Heirath (eigene oder fremde)
erworben; z. B. \uventry{patr$'$} Vater ― \uventry{bo$'$patr$'$} Schwiegervater; \uventry{frat$'$}
Bruder ― \uventry{bo$'$frat$'$} Schwager | пріобрѣтенный бракомъ (своимъ или
чужимъ); напр. \uventry{patr$'$} отецъ ― \uventry{bo$'$patr$'$} тесть, свекоръ; \uventry{frat$'$}
братъ ― \uventry{bo$'$frat$'$} шуринъ, зять, деверь | nabyty przez małżeństwo
(własne lub obce); np. \uventry{patr$'$} ojciec ― \uventry{bo$'$patr$'$} teść; \uventry{frat$'$}
brat ― \uventry{bo$'$frat$'$} szwagier.

\uventry{boa$'$}
boa | boa | Riesenschlange | боа, удавъ | boa.

\uventry{boat$'$}
bateau, canot | boat, bark | Boot | ботъ | bat, łódź.

\uventry{boben$'$}
bobine | spool, bobin | Spule | катушка | cewka, szpulka.

\uventry{boj$'$}
aboyer | bark (dog’s) | bellen | лаять | szczekać.

\uventry{bol$'$}
bouillir | boil (vb.) | sieden | кипѣть | kipić, wrzeć.

\uventry{bombon$'$}
bonbon | dainties | Bonbon | конфектъ | cukierek.

\uventry{bon$'$}
bon | good | gut | хорошій, добрый | dobry.

\uventry{bor$'$}
percer | bore (vb.) | bohren | буравить | wiercić, świdrować.

\uventry{boraks$'$}
borax | borax | Borax | бура | boraks.

\uventry{bord$'$}
bord, rivage | shore | Ufer | берегъ | brzeg.

\uventry{border$'$}
border | border | säumen, besäumen | обрубать (кайма) | obrębiać, bramować.

\uventry{bors$'$}
bourse | bourse, exchange | Börse (der Kaufleute) | биржа | giełda.

\uventry{bot$'$}
botte | boot | Stiefel | сапогъ | but.

\uventry{botel$'$}
bouteille | bottle | Flasche | бутылка | butelka.

\uventry{bov$'$}
boeuf | ox | Ochs, Stier | быкъ | byk.

\uventry{brak$'$}
bras | arm | Arm | рука, объятія | ramię.

\uventry{bram$'$}
brême | bream | Brassen | лещъ | leszcz.

\uventry{bran$'$}
son | bran | Kleie | отруби | otręby.

\uventry{branĉ$'$}
branche | branch | Zweig | вѣтвь | gałaź.

\uventry{brand$'$}
eau-de-vie | brandy | Branntwein | водка | wódka.

\uventry{brank$'$}
branchies, ouïes | gill, fish-ear | Kieme | жабра | dychawka.

\uventry{brasik$'$}
chou | cabbage | Kohl | капуста | kapusta.

\uventry{brav$'$}
brave, solide | valiant, brave | tüchtig | дѣльный, удалый | dzielny, chwacki.

\uventry{bret$'$}
tablette | tablet, shelf | Wandbrett, Regal | полка | półka.

\uventry{brid$'$}
bride | bridle | Zaum | узда | uzda.

\uventry{brik$'$}
brique | brick | Ziegel | кирпичъ | cegła.

\uventry{bril$'$}
briller | shine | glänzen | блистать | błyszczeć.

\uventry{briliant$'$}
brillant | brilliant | Brillantstein | бриліантъ | brylant.

\uventry{brod$'$}
broder | stitch, embroider | sticken | вышивать | haftować.

\uventry{brog$'$}
échauder | scald | brühen | обваривать кипяткомъ | parzyć.

\uventry{bronz$'$}
bronze | bronze | Bronze | бронза | bronz.

\uventry{bros$'$}
brosse | brush | Bürste | щетка | szczotka.

\uventry{brov$'$}
sourcil | eye-brow | Augenbraue | бровь | brew.

\uventry{bru$'$}
faire du bruit | noise | lärmen, brausen | шумѣть | szumieć,
hałasować.

\uventry{brul$'$}
brûler (être en feu) | burn (v. n.) | brennen (v. n.) | горѣть | palić się.

\uvsubentry{}\uventry{brul$'$um$'$}
inflammation | inflammation | Entzündung | воспаленіе | zapalenie.

\uventry{brun$'$}
brun | brown | braun | бурый | brunatny.

\uventry{brust$'$}
poitrine | breast | Brust | грудь | pierś.

\uventry{brut$'$}
brute, bétail | brute | Vieh | скотъ | bydło.

\uventry{bub$'$}
polisson | wicked boy | Bube | мальчишка | ulicznik.

\uventry{bubal$'$}
buffle | buff, buffle | Büffel | буйволъ | bawół.

\uventry{buĉ$'$}
tuer, assommer | slaughter, butcher | schlachten | заклать | zabijać, rzezać.

\uventry{bud$'$}
boutique | booth, shop | Bude | балаганъ | buda.

\uventry{buf$'$}
crapaud | toad | Kröte | жаба | ropucha.

\uventry{bufed$'$}
buffet | buffet | Buffet | буфетъ | bufet, kredens.

\uventry{buk$'$}
boucle | latch, buckle | Schnalle | пряжка | sprzączka.

\uventry{buked$'$}
bouquet | nosegay | Strauss (Blumen) | букетъ | bukiet.

\uventry{bukl$'$}
boucle | buckle, lock | Locke | локонъ | lok, pukiel.

\uventry{bul$'$}
boule, motte | clod | Kloss | комъ, клецка | bryla, kluska.

\uventry{bulb$'$}
oignon | bulb | Zwiebel | луковица | cebula.

\uventry{buljon$'$}
bouillon | broth | Bouillon | бульонъ | buljon.

\uventry{bulk$'$}
pain blanc | manchet loaf | Semmel | булка | bułka.

\uventry{burd$'$}
bourdon | humblebee | Hummel | шмель | trzmiel.

\uventry{burĝ$'$}
bourgeois | burgher | Bürger (nicht adeliger) | мѣщанинъ | mieszczanin.

\uventry{burĝon$'$}
bourgeon | bud | Knospe | почка (растеній) | pączek.

\uventry{buŝ$'$}
bouche | mouth | Mund | ротъ | usta.

\uvsubentry{}\uventry{buŝ$'$um$'$}
muselière | mouth-basket | Maulkorb | намордникъ | kaganiec.

\uventry{buŝel$'$}
boisseau | bushel | Scheffel | четверикъ | korzec.

\uventry{buter$'$}
beurre | butter | Butter | масло (коровье) | masło.

\uventry{butik$'$}
boutique | shop, hall | Laden, Krambude | лавка (торговая) | sklep.

\uventry{buton$'$}
bouton | button | Knopf | пуговица | guzik.

\uvsubentry{}\uventry{buton$'$um$'$}
boutonner | button | zuknöpfen | застегивать | zapinać.

\uvlitero{C, Ĉ}

\uventry{ĉagren$'$}
chagriner | disappoint | verdriessen | причинять досаду | martwić.

\uventry{ĉam$'$}
chamois | wild goat | Gemse | серна | dzika koza, giemza.

\uventry{ĉambelan$'$}
chambellan | chamberlain | Kammerherr | камергеръ | szambelan.

\uventry{ĉambr$'$}
chambre | chamber | Zimmer | комната | pokój.

\uventry{ĉampan$'$}
champagne (vin de) | champagne | Champagner | шампанское | szampan.

\uventry{ĉan$'$}
chien de fusil | cock | Hahn (am Schiessgewehre) | курокъ | kurek.

\uventry{ĉap$'$}
bonnet | bonnet | Mütze | шапка | czapka.

\uventry{ĉapel$'$}
chapeau | hat | Hut | шляпа | kapelusz.

\uventry{ĉapitr$'$}
chapitre | chapter | Kapitel | глава (книги) | rozdział.

\uventry{ĉar}
car, parce que | for | weil, denn, da | ибо, такъ какъ | albowiem, ponieważ.

\uventry{ĉarlatan$'$}
charlatan | charlatan | Charlatan | шарлатанъ | szalbierz.

\uventry{ĉarm$'$}
charmant | charm | anmuthig | милый | nadobny.

\uventry{ĉarnir$'$}
charnier | hing | Charnier | шарниръ | nit, wycięcie.

\uventry{ĉarpent$'$}
charpenter | do carpenter’s work | zimmern | плотничать | ciosać.

\uvsubentry{}\uventry{ĉarpent$'$ist$'$}
charpentier | carpenter | Zimmermann | плотникъ | cieśla.

\uventry{ĉarpi$'$}
charpie | lint for a wound | Charpie | корпія | skubanka.

\uventry{ĉas$'$}
chasser (vénerie) | hunt | jagen, Jagd machen | охотиться | polować.

\uvsubentry{}\uventry{ĉas$'$aĵ$'$}
gibier | game, venison | Wild | дичь | zwierzyna.

\uventry{ĉast$'$}
chaste | chaste | züchtig, keusch | цѣломудренный | niepokalany, czysty.

\uventry{ĉe}
chez | at | bei | у, при | u, przy.

\uventry{ced$'$}
céder | yield, give up | abtreten, weichen | уступать | ustępować.

\uventry{cedr$'$}
cèdre | cedar | Ceder | кедръ | cedr.

\uventry{ĉef$'$}
principal | chief | Haupt, Chef | глава, главный | szef,
główny.

\uventry{cejan$'$}
aubifoin, bluet | corn-flower | Kornblume | василекъ | bławatek.

\uventry{cel$'$}
viser | aim | zielen | цѣлиться | celować.

\uventry{ĉel$'$}
ceilule | cell | Zelle | ячея, ячейка, келья | cela, komórka.

\uventry{cement$'$}
ciment, lut | cement, lute | Cement, Kitt | цементъ | cement, kit.

\uventry{ĉemiz$'$}
chemise | shirt | Hemd | сорочка | koszula.

\uventry{ĉen$'$}
chaine | chain | Kette | цѣпь | łańcuch.

\uventry{cent}
cent | hundred | hundert | сто | sto.

\uventry{cerb$'$}
cerveau | brain | Gehirn | мозгъ | mózg.

\uventry{ĉeriz$'$}
cerise | cherry | Kirsche | вишня | wiśnia.

\uventry{ĉerk$'$}
cercueil | coffin | Sarg | гробъ | trumna.

\uventry{ĉerp$'$}
puiser | draw | schöpfen (z. B. Wasser) | черпать | czerpać.

\uventry{cert$'$}
certain | certain, sure | sicher, bekannt, gewiss | вѣрный,
извѣстный | pewny, znany.

\uvsubentry{}\uventry{cert$'$ig$'$}
assurer | assure | versichern | увѣрять | zapewniać.

\uventry{cerv$'$}
cerf | deer | Hirsch | олень | jeleń.

\uvsubentry{}\uventry{nord$'$a cerv$'$}
renne | reindeer | Rennthier | сѣверный олень | renifer, ren.

\uventry{ĉes$'$}
cesser | cease, desist | aufhören | переставать | przestawać.

\uventry{ceter$'$}
autre (le reste) | rest, remainder | übrig | прочій | pozostały.

\uventry{ĉeval$'$}
cheval | horse | Pferd | конь | koń.

\uventry{ci}
tu, toi, | thou | du | ты | ty.

\uvsubentry{}\uventry{ci$'$a}
ton, ta | thy, thine | dein | твой | twój.

\uventry{ĉi}
ce qui est le plus près; ex. \uventry{tiu} celui-là ― \uventry{tiu ĉi} celui-ci | denotes proximity; e. g. \uventry{tiu} that ― \uventry{tiu ĉi} this; \uventry{tie} there
― \uventry{tie ĉi} here | die nächste Hinweisung; z. B. \uventry{tiu} jener ― \uventry{tiu
ĉi} dieser; \uventry{tie} dort ― \uventry{tie ĉi} hier | ближайшее указаніе;
напр. \uventry{tiu} тотъ― \uventry{tiu ĉi} этотъ; \uventry{tie} тамъ ― \uventry{tie ĉi} здѣсь | określenie najbliższe; np. \uventry{tiu} tamten ― \uventry{tiu ĉi} ten; \uventry{tie} tam
― \uventry{tie ĉi} tu.

\uventry{ĉia}
chaque | every | jedweder, jeglicher | всякій, всяческій | wszelaki.

\uventry{ĉiam}
toujours | always | immer | всегда | zawsze.

\uventry{ĉie}
partout | everywhere | überall | повсюду | wszędzie.

\uventry{ĉiel}
de chaque (toute) manière | in every manner | auf jede Weise | всячески | wszelkim sposobem.

\uventry{ĉiel$'$}
ciel | heaven | Himmel | небо | niebo.

\uvsubentry{}\uventry{ĉiel$'$ark$'$}
arc-en-ciel | rain-bow | Regenbogen | радуга | tęcza.


\uvsubentry{}\uventry{ĉiel$'$ir$'$}
Ascension | Ascension | Himmelfahrt Christi | Вознесеніе Господне | Wniebowstąpienie.

\uventry{ĉif$'$}
froisser, chiffonner | crumple, muss | zerknittern, zerknüllen | мять | miętosić.

\uventry{cifer$'$}
chiffre | cipher | Ziffer | цифра | cyfra.

\uvsubentry{}\uventry{cifer$'$plat$'$}
cadran | dial | Zifferblatt | циферблатъ | tarcza zegarowa.

\uventry{ĉifon$'$}
chiffon | rag | Lappen, Lumpen | лоскутъ | szmata.

\uventry{cigar$'$}
cigare | cigar | Cigarre | сигара | cygaro.

\uventry{cigared$'$}
cigarette | cigarette | Cigarette | папироса | papieros.

\uventry{cign$'$}
cygne | swan | Schwan | лебедь | łabędź.

\uventry{ĉikan$'$}
cancaner | chicane | Klatschereien machen | сплетничать | rozsiewac plotki.

\uventry{cikatr$'$}
couture, cicatrice | scar, cicatrice | Narbe | рубецъ | blizna.

\uventry{cikoni$'$}
cigogne | stork | Storch | аистъ | bocian.

\uventry{cikori$'$}
chicorée | chicory | Cichorie | цикорія | cykorya.

\uventry{cim$'$}
punaise | bug | Wanze | клопъ | pluskwa.

\uventry{cimbal$'$}
cymbale | cymbal | Cymbel | кимвалы | cymbały.

\uventry{cinabr$'$}
cinabre | cinnabar | Zinnober | киноварь | cynober.

\uventry{cinam$'$}
cannelle | cinnamon | Zimmt | корица | cynamon.

\uventry{cindr$'$}
cendre | ash, cinder | Asche | пепелъ | popiół.

\uventry{ĉio}
tout | everything | alles | все | wszystko.

\uventry{cipres$'$}
cyprès | cypress | Cypresse | кипарисъ | cyprys.

\uventry{cir$'$}
cirage | shoe-polish | Wichse | вакса | szuwaks.

\uventry{ĉirkaŭ}
autour de, environ | about, around | um, herum, gegen | около, кругомъ | około, dokoła.

\uvsubentry{}\uventry{ĉirkaŭ$'$aĵ$'$}
alentours | environs | Umgegend | окрестность | okolica.


\uvsubentry{}\uventry{ĉirkaŭ$'$i}
entourer | encircle, environ | umgeben | окружать | otaczać.


\uvsubentry{}\uventry{ĉirkaŭ$'$o}
circuit, enceinte | circumference | Umfang, Umkreis | окружность | obwód.


\uvsubentry{}\uventry{ĉirkaŭ$'$man$'$}
bracelet | bracelet | Armband | браслетъ | bransoletka.


\uvsubentry{}\uventry{ĉirkaŭ$'$pren$'$}
embrasser | embrace | umarmen | обнимать | obejmować, ściskać.


\uvsubentry{}\uventry{ĉirkaŭ$'$skrib$'$}
circonscrire | circumscribe | umschreibcn | описывать | opisywać.

\uventry{cirkel$'$}
cercle, compas | circle, compass | Zirkel (Instrument) | циркуль | cyrkiel.

\uventry{cirkonstanc$'$}
cirkonstance | circumstance | Umstand | обстоятельство | okoliczność.

\uventry{cirkuler$'$}
circulaire | circular | Cirkular | циркуляръ | okólnik.

\uventry{cit$'$}
citer | cite | citiren | цитировать | przytaczać, cytować.

\uventry{citr$'$}
cithare | guitar | Zither | цитра | cytra.

\uventry{citron$'$}
citron | lemon | Citrone | лимонъ | cytryna.

\uventry{ĉiu}
chacun | each, every one | jedermann | всякій, каждый | wszystek, każdy.

\uvsubentry{}\uventry{ĉiu$'$j}
tous | all | alle | всѣ | wszyscy.

\uventry{civiliz$'$}
civiliser, éclaircir | civilize, clear up | aufklären | просвѣщать | oświecać.

\uventry{ĉiz$'$}
creuser avec le ciseau | chisel | meisseln | долбить | dłutować.

\uventry{ĉj$'$}
après les 1-5 premières lettres d’un prénom masculin lui donne
un caractère diminutif et caressant; ex. \uventry{Miĥael$'$} ― \uventry{Mi$'$ĉj$'$}
| affectionate diminutive of masculine names; e.g. \uventry{Johan$'$} John ―
\uventry{Jo$'$ĉj$'$} Johnnie | den 1-5 Buchstaben eines männlichen Eigennamens
beigefügt verwandelt diesen in ein Kosewort; z. B. \uventry{Miĥael$'$} ―
\uventry{Mi$'$ĉj$'$}; \uventry{Aleksandr$'$} ― \uventry{Ale$'$ĉj$'$}
| приставленное къ первымъ 1-5
буквамъ имени собственнаго мужскаго пола, превращаетъ его въ
ласкательное; напр. \uventry{Miĥael$'$} ― \uventry{Mi$'$ĉj$'$}; \uventry{Aleksandr$'$} ― \uventry{Ale$'$ĉj$'$}
| dodane do pierwszych 1-5 liter imenia własnego męzkiego rodzaju
zmienia je w pieszczotliwe; np. \uventry{Miĥael$'$} ― \uventry{Mi$'$ĉj$'$}; \uventry{Aleksandr$'$}
― \uventry{Ale$'$ĉj$'$}
.

\uventry{ĉokolad$'$}
chocolat | chocolate | Chocolade | шоколадъ | czekolada.

\uventry{col$'$}
pouce | inch | Zoll (Mass) | дюймъ | cal.

\uventry{ĉu}
est-ce que | whether | ob | ли, развѣ | czy.

\uvlitero{D}

\uventry{da}
de (après les mots marquant mesure, poids, nombre) | is used
instead of \uventry{de} after words expressing weight or measure;
e. g. \uventry{funt$'$o da viand$'$o} a pound of meat, \uventry{glas$'$o da te$'$o} a cup of
tea | ersetzt den Genitiv (nach Mass, Gewicht u. drgl. bezeichnenden
Wörtern); z. B. \uventry{kilogram$'$o da viand$'$o} ein Kilogramm Fleisch; \uventry{glas$'$o
da te$'$o} ein Glass Thee | (послѣ словъ, означающихъ мѣру, вѣсъ и т. д.)
замѣняетъ родительный падежъ; напр. \uventry{kilogram$'$o da viand$'$o} килограмъ
мяса; \uventry{glas$'$o da te$'$o} стаканъ чаю | zastępuje przypadek drugi (po słowach oznaczających miarę, wagę i. t. p.); np. \uventry{kilogram$'$o da viand$'$o} kilogram mięsa; \uventry{glas$'$o da te$'$o} szklanka herbaty.

\uventry{daktil$'$}
datte | date (fruit) | Dattel | финикъ | daktyl.

\uventry{dam$'$o$'$j}
jeu de dames | draughts | Damenspiel | шашки (игра) | warcaby.

\uventry{damask$'$}
damas | damask | Damast | камча | adamaszek.

\uventry{danc$'$}
danser | dance | tanzen | танцовать | tańczyć.

\uventry{dand$'$}
petit-maître | dandy | Stutzer | франтъ | fircyk, elegant.

\uventry{danĝer$'$}
danger | danger | Gefahr | опасность | niebezpieczeństwo.

\uventry{dank$'$}
remercier | thank | danken | благодарить | dziękować.

\uventry{dat$'$}
date | date | Datum | число (мѣсяца) | data.

\uventry{dativ$'$}
datif | dative | Dativ | дательный падежъ | celownik.

\uventry{datur$'$}
datura | datura | Stechapfel | дурманъ | bieluń, dziędzierzava.

\uventry{daŭr$'$}
durer | endure, last | dauern | продолжаться | trwać.

\uvsubentry{}\uventry{daŭr$'$ig$'$}
continuer | continue | fortsetzen | продолжать | dalej ciągnąć.

\uventry{de}
de | of, from | von; ersetzt auch den Genitiv | отъ; замѣняетъ
также родительный падежъ | od; zastępuje też dopełniacz.

\uventry{dec$'$}
être dû, être convenable | be due, become | gebühren | надлежать | należeć się.

\uventry{decid$'$}
décider | decide | entscheiden, beschliessen | рѣшать | rozstrzygać.

\uventry{Decembr$'$}
Décembre | December | December | Декабрь | Grudzień.

\uventry{deĉifr$'$}
déchiffrer | decipher | entziffern | дешифрировать | wyczytać, rozwiązać.

\uventry{dediĉ$'$}
vouer, dédier | dedicate | widmen | посвящать | poświęcać, dedykować.

\uventry{defend$'$}
défendre | defend | vertheidigen | защищать | bronić.

\uventry{degel$'$}
dégeler (se) | thaw | aufthauen | таять | odwilgnąć.

\uventry{degener$'$}
dégénérer | degenerate | ausarten | вырождаться | wyradzać się.

\uventry{degrad$'$}
dégrader | degrade | degradiren | разжаловать | degradowac.

\uventry{deĵor$'$}
être de service | be waiter | den Dienst haben | дежурить | deżurować.

\uventry{dek}
dix | ten | zehn | десять | dziesięć.

\uventry{deklinaci$'$}
décliner | decline | decliniren | склонять (граммат.) | przypadkować.

\uventry{dekliv$'$}
déclivité | declivity | Abhang | покатость | pochyłość.

\uventry{dekstr$'$}
droit, droite | right-hand | recht | правый | prawy.

\uvsubentry{}\uventry{mal$'$dekstr$'$}
gauche | left | link | лѣвый | lewy.

\uventry{delfen$'$}
dauphin | dolphin | Delphin | дельфинъ | delfin.

\uventry{delikat$'$}
délicat | delicate, fine | fein, zart | нѣжный | delikatny.

\uventry{delir$'$}
être en délire, rêver | be light-headed, rave | irre reden | бредить | majaczyć.

\uventry{demand$'$}
demander, questionner | demand, ask | fragen | спрашивать | pytać.

\uventry{demon$'$}
démon | demon | Dämon | демонъ | demon.

\uventry{denar$'$}
denier | denier, penny | Denar | динарій | denar.

\uventry{dens$'$}
épais, dense | dense | dicht | густой | gęsty.

\uventry{dent$'$}
dent | tooth | Zahn | зубъ | ząb.

\uventry{denunc$'$}
denoncer | denounce | denunciren | доносить | denuncyować.

\uventry{des}
d’autant plus | the (the... the ― \uventry{ju... des}) | desto, um so | тѣмъ | tem.

\uventry{deput$'$}
députer | depute | abordnen | отряжать, отправлять | wyprawiać.

\uventry{desegn$'$}
dessiner | design, purpose | zeichnen | чертить | kreślić, rysować.

\uventry{detal$'$}
détaillé | detail | ausführlich | подробный | szczegółowy.

\uventry{detru$'$}
détruire | destroy | zerstören | разрушать | burzyd,
niszczyć.

\uventry{dev$'$}
devoir | ought, must | sollen | долженствовать | musieć.

\uvsubentry{}\uventry{dev$'$ig$'$}
nécessiter | necessitate, compel | nöthigen, zwingen | принуждать | pzymuszać.

\uventry{deviz$'$}
devise | device | Devise | девизъ | dewiza.

\uventry{dezert$'$}
désert | desert | Wüste | пустыня | pustynia.

\uventry{dezir$'$}
désirer | desire | wünschen | желать | życzyć.

\uventry{Di$'$}
Dieu | God | Gott | Богъ | Bóg.

\uventry{diabl$'$}
diable | devil | Teufel | дьяволъ, чертъ | djabeł.

\uventry{diamant$'$}
diamant | diamond | Diamant | алмазъ | dyament.

\uventry{diboĉ$'$}
débaucher, crapuler | debauch, revel, riot | schwelgen | кутить | hulać.

\uventry{didelf$'$}
didelphe, philandre | opossum | Beutelthier | двуутробка | dydelf, torebnik.

\uventry{difekt$'$}
endommager, détériorer | damage, injure | beschädigen,
verletzen | повреждать | uszkadzać.

\uventry{diferenc$'$}
différer (v. n.) | differ | sich unterscheiden | различаться | różnic się.

\uventry{difin$'$}
définir, déterminer | define | bestimmen | опредѣлять | wyznaczać, określać.

\uventry{dig$'$}
digue | embank | Damm | плотина | tama.

\uventry{digest$'$}
digérer | digest | verdauen | переваривать (о желудкѣ) | trawić.

\uventry{dik$'$}
gros | thick | dick | толстый | gruby.

\uventry{dikt$'$}
dicter | dictate | dictiren | диктовать | dyktować.

\uventry{diligent$'$}
diligent, assidu | diligent | fleissig | прилежный | pilny.

\uventry{dimanĉ$'$}
dimanche | Sunday | Sonntag | воскресенье | niedziela.

\uventry{dir$'$}
dire | say | sagen | сказать | powiadać.

\uventry{direkt$'$}
diriger | direct | richten | направлять | kierować.

\uvsubentry{}\uventry{direkt$'$o}
direction | direction | Richtung | направленіе | kierunek.


\uvsubentry{}\uventry{direkt$'$il$'$o (de ŝip$'$o)}
gouvernail | helm | Steuerruder | руль | ster.

\uventry{dis$'$}
marque division, dissémination; ex. \uventry{ir$'$} aller ― \uventry{dis$'$ir$'$}
se séparer, aller chacun de son côté | has the same force as the
English prefix \uventry{dis}; e. g. \uventry{sem$'$} sow ―
\uventry{dis$'$sem$'$} disseminate; \uventry{ŝir$'$} tear ― \uventry{dis$'$ŝir$'$} tear to pieces | zer; z. B. \uventry{ŝir$'$} reissen ― \uventry{dis$'$ŝir$'$} zerreissen | раз-;
напр. \uventry{ŝir$'$} рвать ― \uventry{dis$'$ŝir$'$} разрывать | roz-; np. \uventry{ŝir$'$} rwać ―
\uventry{dis$'$ŝir$'$} rozrywać; \uventry{kur$'$} biegać ― \uventry{dis$'$kur$'$} rozbiegać się.

\uvsubentry{}\uventry{dis$'$ig$'$}
séparer, désunir | separate, disunite | trennen | разобщать | rozłączać.

\uventry{diskont$'$}
escompter | discount | discontiren | дисконтировать,
учитывать | dyskontować.

\uventry{dispon$'$}
disposer | dispose | verfügen, disponiren | располагать | rozporządzać.

\uventry{disput$'$}
disputer | dispute | streiten, disputiren | спорить | sprzeczać się.

\uventry{distil$'$}
distiller | distill | destilliren | дистиллировать | destylować.

\uventry{disting$'$}
distinguer | distinguish | auszeichnen | отличать | odznaczać.

\uventry{distr$'$}
distraire | distract | zerstreuen | разсѣивать | rozpraszać,
roztargać.

\uventry{distrikt$'$}
district | district | Bezirk | округь | okrąg.

\uventry{diven$'$}
deviner | divine, guess | errathen | угадывать | odgadywać.

\uventry{divers$'$}
divers | various, diverse | verschieden | различный | różny.

\uventry{divid$'$}
diviser, partager | divide | theilen | дѣлить | dzielić.

\uventry{do}
donc | then, indeed, however | doch, also | же | więc.

\uventry{dolĉ$'$}
doux | sweet | süss | сладкій | słodki.

\uventry{dolor$'$}
faire mal, causer de la douleur | sorrow | schmerzen | болѣть
(причинять боль) | boleć, dolegać.

\uventry{dom$'$}
maison | house | Haus | домъ | dom.

\uventry{domaĝ$'$}
dommage | pity (it is a pity) | Schade (es ist) | жаль | szkoda.

\uventry{domen$'$}
domino | domino | Domino | домино | domino.

\uventry{don$'$}
donner | give | geben | давать | dawać.

\uvsubentry{}\uventry{al$'$don$'$}
ajouter | add to | zugeben, beilegen | прибавлять | dodawać.

\uventry{donac$'$}
faire cadeau, donner en présent | make a present | schenken | дарить | darować.

\uventry{dorlot$'$}
gâter, dorloter | spoil | hätscheln | баловать | pieścić.

\uventry{dorm$'$}
dormir | sleep | schlafen | спать | spać.

\uventry{dorn$'$}
épine | thorn | Dorn | шипъ | cierń.

\uventry{dors$'$}
dos | back | Rücken | спина | grzbiet.

\uventry{dot$'$}
doter | endow | ausstatten | надѣлять (приданымъ) | wyposażyć.

\uvsubentry{}\uventry{dot$'$o}
dot | dowry | Mitgift | приданое | posag.

\uventry{drak$'$}
dragon | dragon, drake | Drache | драконъ | smok.

\uventry{drap$'$}
drap | woollen goods | Tuch (wollenes Gewebe) | сукно | sukno.

\uventry{drapir$'$}
draper | cover with cloth | drapiren | драпировать | drapować.

\uventry{draŝ$'$}
battre (le blé) | thrash | dreschen | молотить | młócić.

\uventry{dres$'$}
dresser | dress, make straight | dressiren, abrichten | дрессировать | tresować.

\uventry{drink$'$}
boire, ivrogner | drink, tipple | saufen, zechen | пить
(спиртные напитки) | pić (wódkę i t. p.).

\uvsubentry{}\uventry{drink$'$ej$'$}
cabaret, taverne | inn, tavern | Schenke | шинокъ | szynkownia.

\uventry{drog$'$}
drogue | drug | Drogue | москатильный товаръ | towary aptekarskie.

\uventry{dron$'$}
se noyer | drown | ertrinken | утопать | tonąć.

\uventry{du}
deux | two | zwei | два | dwa.

\uventry{dub$'$}
douter | doubt | zweifeln | сомнѣваться | wątpić.

\uventry{duk$'$}
duc | duke | Herzog | герцогъ | książe.

\uventry{dum}
pendant, tandis que | while | während | пока, между тѣмъ какъ | póki, podczas gdy.

\uvsubentry{}\uventry{dum$'$e}
cependant | meanwhile, during | unterdessen | между тѣмъ | tymczasem.

\uventry{dung$'$}
louer, embaucher | hire | dingen | нанимать | najmować.

\uventry{du$'$on$'$patr$'$}
beau-père | step-father | Stiefvater | отчимъ | ojczym.

\uvlitero{E}

\uventry{e}
marque l’adverbe; ex. \uventry{bon$'$e} bien | ending of adverbs;
e. g. \uventry{bon$'$} good ― \uventry{bon$'$e} well | Endung des Adverbs; z. B. \uventry{bon$'$e}
gut | окончаніе нарѣчія; напр. \uventry{bon$'$e} хорошо | zakończenie
przysłówka; np. \uventry{bon$'$e} dobrze.

\uventry{eben$'$}
égal (de même plan) | even, smooth | eben, glatt | ровный | równy.

\uventry{ebl$'$}
possible; ex. \uventry{kompren$'$} comprendre ― \uventry{kompren$'$ebl$'$}
compréhensible | able, possible | möglich | возможный | możliwy.

\uvsubentry{}\uventry{ebl$'$e}
peut-être | perhaps, may-be | vielleicht | можетъ быть | może.

\uventry{ebon$'$}
ébène | ebony | Ebenholz | черное дерево | heban.

\uventry{ec$'$}
marque la qualité (abstraitement); ex. \uventry{bon$'$} bon ― \uventry{bon$'$ec$'$}
bonté; \uventry{vir$'$} homme ― \uventry{vir$'$ec$'$} virilité | denotes qualites;
e. g. \uventry{bon$'$} good ― \uventry{bon$'$ec$'$} goodness | Eigenschaft; z. B. \uventry{bon$'$}
gut ― \uventry{bon$'$ec$'$} Güte; \uventry{vir$'$in$'$} Weib ― \uventry{vir$'$in$'$ec$'$} Weiblichkeit | качество или состояніе; напр. \uventry{bon$'$} добрый ― \uventry{bon$'$ec$'$} доброта;
\uventry{vir$'$in$'$} женщина ― \uventry{vir$'$in$'$ec$'$} женственность | przymiot jako
oddzielne pojęcie; np. \uventry{bon$'$} dobry ― \uventry{bon$'$ec$'$} dobroć; \uventry{infan$'$}
dziecię ― \uventry{infan$'$ec$'$} dzieciństwo.

\uvsubentry{}\uventry{ec$'$a}
qualitatif | qualitative | qualitativ | качественный | jakościowy.

\uventry{eĉ}
même (adv.) jusqu’à | even (adv.) | sogar | даже | nawet.

\uventry{edif$'$}
édifier | edify | erbauen (z. B. durch eine Predigt) | назидать | budować.

\uventry{eduk$'$}
élever (éducation) | educate | erziehen | воспитывать | wychowywać.

\uventry{edz$'$}
mari, époux | married person, husband |   | Gemahl | супругъ | małżonek.

\uvsubentry{}\uventry{edz$'$iĝ$'$}
se marier | marry | heirathen | жениться | żenić się.


\uvsubentry{}\uventry{edz$'$ec$'$}
mariage | marriage, matrimony | Ehe | бракъ | małżeństwo.

\uventry{efektiv$'$}
effectif, réel | real, actual | wirklich | дѣйствительный | rzeczywisty.

\uventry{efik$'$}
opérer, agir | effect | wirken | дѣйствовать | działać,
skutkonwać.

\uventry{eg$'$}
marque augmentation, plus haut degré; ex. \uventry{pord$'$} porte ―
\uventry{pord$'$eg$'$} grande porte; \uventry{pet$'$} prier ― \uventry{pet$'$eg$'$} supplier | denotes
increase of degree; e. g, \uventry{varm$'$} warm ― \uventry{varm$'$eg$'$} hot | bezeichnet
eine Vergrösserung oder Steigerung; z. B. \uventry{pord$'$} Thür ― \uventry{pord$'$eg$'$}
Thor; \uventry{varm$'$} warm ― \uventry{varm$'$eg$'$} heiss | означаетъ увеличеніе или
усиленіе степени; напр. \uventry{man$'$} рука ― \uventry{man$'$eg$'$} ручище; \uventry{varm$'$}
теплый ― \uventry{varm$'$eg$'$} горячій | oznacza zwiększenie lub wzmocnienie
stopnia; np. \uventry{man$'$} ręka ― \uventry{man$'$eg$'$} łapa; \uventry{varm$'$} ciepły ―
\uventry{varm$'$eg$'$} gorący.

\uventry{egal$'$}
égal, qui ne diffère pas | equal | gleich | одинаковый | jednakowy.

\uventry{eĥ$'$}
écho | echo | Echo | эхо | odgłos.

\uventry{ej$'$}
marque le lieu spécialement affecté à...; ex. \uventry{preĝ$'$} prier ―
\uventry{preĝ$'$ej$'$} église; \uventry{kuir$'$} faire cuire ― \uventry{kuir$'$ej$'$} cuisine | place
where an action occurs; e. g. \uventry{kuir$'$} cook ― \uventry{kuir$'$ej$'$} kitchen | Ort für...; z. B. \uventry{kuir$'$} kochen ― \uventry{kuir$'$ej$'$} Küche; \uventry{preĝ$'$} beten
― \uventry{preĝ$'$ej$'$} Kirche | мѣсто для...; напр. \uventry{kuir$'$} варить ―
\uventry{kuir$'$ej$'$} кухня; \uventry{preĝ$'$} молиться ― \uventry{preĝ$'$ej$'$} церковь | miejsce
dla...; np. \uventry{kuir$'$} gotować ― \uventry{kuir$'$ej$'$} kuchnia; \uventry{preĝ$'$} modlić się
― \uventry{preĝ$'$ej$'$} kościoł.

\uventry{ek$'$}
indique une action qui commence ou qui est momentanée;
ex. \uventry{kant$'$} chanter ― \uventry{ek$'$kant$'$} commencer à chanter; \uventry{kri$'$} crier
― \uventry{ek$'$kri$'$} s’écrier | denotes sudden or momentary action;
e. g. \uventry{kri$'$} cry ― \uventry{ek$'$kri$'$} cry out | bezeichnet eine anfangende
oder momentane Handlung; z. B. \uventry{kant$'$} singen ― \uventry{ek$'$kant$'$} einen
Gesang anstimmen; \uventry{kri$'$} schreien ― \uventry{ek$'$kri$'$} aufschreien | начало
или мгновенность; напр. \uventry{kant$'$} пѣть ― \uventry{ek$'$kant$'$} запѣвать; \uventry{kri$'$}
кричать ― \uventry{ek$'$kri$'$} вскрикнуть | oznacza początek lub chwilownść;
np. \uventry{kant$'$} śpiewać ― \uventry{ek$'$kant$'$} zaśpiewać; \uventry{kri$'$} krzyczeć ―
\uventry{ek$'$kri$'$} krzyknąć.

\uventry{eks$'$}
qui fut, ex- | ex-, late | ehemalig, verabschiedet, abgedankt | бывшій, отставной | były, dymisyonowany.

\uventry{ekscelenc$'$}
excellence | excellence | Excellenz | Превосходительство | Ekscelencya.

\uventry{ekscit$'$}
exciter, émouvoir | excite | erregen | возбуждать | wzbudzać.

\uventry{ekskurs$'$}
excursion | excursion | Ausflug | экскурсія | wycieczka.

\uventry{eksped$'$}
expédier | expedité | expediren, versenden | экспедировать | ekspedyować.

\uventry{eksplod$'$}
faire explosion | explode | explodiren | взрывать,
эксплодировать | wybuchać.

\uventry{ekspozici$'$}
exposition | exposition | Ausstellung | выставка | wystawa.

\uventry{ekster}
hors, en dehors de | outside, besides | ausser, ausserhalb | внѣ | zewnątrz.

\uventry{eksterm$'$}
exterminer | exterminate | ausrotten, vertilgen | истреблять | wytępiać.

\uventry{ekstr$'$}
extraordinaire, extra- | extraordinary, extra- | extra,
ausserordentlich | экстренный | nadzwyczajny.

\uventry{ekstrem$'$}
extrême | extreme | äusserst | крайній | skrajny, ostateczny.

\uventry{ekzamen$'$}
éprouver, examiner | try, examine | prüfen | испытывать | egzaminować.

\uventry{ekzekut$'$}
exécuter | execute | hinrichten | казнить | tracić,
wykonywać wyrok.

\uventry{ekzempl$'$}
exemple | example | Beispiel | примѣръ | przykład.

\uventry{ekzempler$'$}
exemplaire | exemplar | Exemplar | экземпляръ | egzemplarz.

\uventry{ekzerc$'$}
exercer | exercise | üben | упражнять | ćwiczyć.

\uventry{ekzil$'$}
bannir, exiler | banish, exile | verbannen | ссылать
(въ ссылку) | wyganiać.

\uventry{ekzist$'$}
exister | exist | bestehen, da sein | существовать | istnieć.

\uventry{el}
de, d’entre, é-, ex- | from, out from | aus | изъ | z.

\uventry{elast$'$}
élastique | elastic, elastical | elastisch | эластичный | sprężysty, elastyczny.

\uventry{elefant$'$}
éléphant | elephant | Elephant | слонъ | słoń.

\uventry{elekt$'$}
choisir | choose | wählen | выбирать | wybierać.

\uventry{elokvent$'$}
éloquent | eloquent | beredt | краснорѣчивый | krasomówny.

\uventry{em$'$}
qui a le penchant, l’habitude; ex. \uventry{babil$'$} babiller ―
\uventry{babil$'$em$'$} babillard | inclined to; e. g. \uventry{babil$'$} chatter ―
\uventry{babil$'$em$'$} talkative | geneigt, gewohnt | склонный, имѣющій привычку;
напр. \uventry{babil$'$} болтать ― \uventry{babil$'$em$'$} болтливый | skłonny,
przyzwyczajony; np. \uventry{babil$'$} paplać ― \uventry{babil$'$em$'$} gaduła.

\uventry{emajl$'$}
email | enamel | Email | эмаль | szmelc, amalia.

\uventry{embaras$'$}
embarras | embarrassment, puzzle | Verlegenheit | затрудненіе | ambaras.

\uventry{embri$'$}
embryon | embryo | Keim | зародышъ | zaród, zarodek.

\uventry{embusk$'$}
embuscade | ambush | Hinterhalt | засада | zasadzka.

\uventry{eminent$'$}
éminent | eminent | vornehm, hervorragend | знатный,
выдающійся | znakomity, wydatny.

\uventry{en}
en, dans | in (when followed by the accusative ― into) | in,
ein- | въ | w.

\uventry{enigm$'$}
énigme | puzzle | Räthsel | загадка | zagadka.

\uventry{entrepren$'$}
entreprendre | undertake | unternehmen | предпринимать | przedsiębrać.

\uventry{entuziasm$'$}
enthousiasme | enthusiasm | Begeisterung | воодушевленіе | zapał.

\uventry{enu$'$}
s’ennuyer | annoy, weary | sich langweilen | скучать | nudzić
się.

\uventry{envi$'$}
envier | envy | beneiden | завидовать | zazdrościć.

\uventry{episkop$'$}
évêque | bishop | Bischof | епископъ, архіерей | biskup.

\uvsubentry{}\uventry{ĉef$'$episkop$'$}
archevêque | archbishop | Erzbischof | архіепископъ | arcybiskup.

\uventry{epok$'$}
époque | epoch | Epoche | эпоха | epoka.

\uventry{epolet$'$}
épaulette | epaulet, shoulder-strap | Achselband | эполетъ | naramiennik.

\uventry{er$'$}
marque l’unité; ex. \uventry{sabl$'$} sable ― \uventry{sabl$'$er$'$} un grain de
sable | one of many objects of the same kind; e. g. \uventry{sabl$'$} sand ―
\uventry{sabl$'$er$'$} grain of sand | ein einziges; z. B. \uventry{sabl$'$} Sand ―
\uventry{sabl$'$er$'$} Sandkörnchen | отдѣльная единица; напр. \uventry{sabl$'$} песокъ ―
\uventry{sabl$'$er$'$} песчинка | oddzielna jednostka; np. \uventry{sabl$'$} piasek ―
\uventry{sabl$'$er$'$} ziarnko piasku.

\uventry{erar$'$}
errer | err, mistake | irren | ошибаться, блуждать | błądzić,
mylić się.

\uventry{erinac$'$}
hérisson | hedgehog | Igel | ежъ | jeż.

\uventry{ermen$'$}
hermine | ermine | Hermelin | горностай | gronostaj.

\uventry{ermit$'$}
ermite, solitaire | hermit, solitary | Einsiedler | отшельникъ | pustelnik.

\uventry{erp$'$}
herser | harrow | eggen | боронить | bronować.

\uventry{escept$'$}
excepter | except | ausschliessen, ausnehmen | исключать | wykluczać.

\uventry{esenc$'$}
être, essence | essence | Wesen | сущность | islota, treść.

\uventry{eskadr$'$}
escadre | squadron | Geschwader | эскадра | eskadra.

\uventry{esper$'$}
espérer | hope | hoffen | надѣяться | spodziewać się.

\uvsubentry{}\uventry{mal$'$esper$'$}
désespérer | despair | verzweifeln | отчаяваться | rozpaczać.

\uventry{esplor$'$}
explorer, rechercher | explore | forschen, untersuchen | изслѣдовать | badać.

\uventry{esprim$'$}
exprimer | express (vb.) | ausdrücken | выражать | wyrażać.

\uventry{est$'$}
être (verbe) | be | sein | быть | być.

\uventry{estim$'$}
estimer | esteem | schätzen | уважать | poważać.

\uventry{esting$'$}
éteindre | extinguish | löschen | гасить | gasić.

\uventry{estr$'$}
chef; ex. \uventry{ŝip$'$} navire ― \uventry{ŝip$'$estr$'$} capitaine | chief,
boss; e.g. \uventry{ŝip$'$} ship ― \uventry{ŝip$'$estr$'$} captain | Vorsteher | начальникъ; напр. \uventry{ŝip$'$} корабль ― \uventry{ŝip$'$estr$'$} капитанъ | wódz,
zwierzchnik.

\uvsubentry{}\uventry{estr$'$ar$'$}
gouvernement | government | Obrigkeit | начальство | zwierzchność, władza.

\uventry{eŝafod$'$}
échafaud | scaffold | Schaffot | эшафотъ | rusztowanie.

\uventry{et$'$}
marque diminution, décroissance; ex. \uventry{mur$'$} mur ― \uventry{mur$'$et$'$}
petitmur; \uventry{rid$'$} rire ― \uventry{rid$'$et$'$} sourire | denotes diminution of
degree; e. g. \uventry{rid$'$} laugh ― \uventry{rid$'$et$'$} smile | bezeichnet eine Verkleinerung 
oder Schwächung; z. B. \uventry{mur$'$} Wand ― \uventry{mur$'$et$'$} Wändchen; \uventry{rid$'$}
lachen ― \uventry{rid$'$et$'$} lächeln | означаетъ уменьшеніе или ослабленіе
степени; напр. \uventry{mur$'$} стѣна ― \uventry{mur$'$et$'$} стѣнка; \uventry{rid$'$} смѣяться ―
\uventry{rid$'$et$'$} улыбаться | oznacza zmniejszenie, zdrobnienie lub osłabienie
stopnia; np. \uventry{mur$'$} ściana ― \uventry{mur$'$et$'$} ścianka; \uventry{rid$'$} śmiać się ―
\uventry{rid$'$et$'$} uśmiechać się.

\uventry{etaĝ$'$}
étage | stage, story (of a house) | Stockwerk, Etage | этажъ | piętro.

\uventry{etat$'$}
état | condition | Etat | штатъ | etat.

\uventry{etend$'$}
étendre | extend | dehnen, strecken, ausbreiten | простирать | rospościerać.

\uventry{eter$'$}
éther | ether | Aether | эфиръ | eter.

\uventry{etern$'$}
éternel | eternal | ewig | вѣчный | wieczny.

\uventry{evit$'$}
éviter | avoid | meiden, ausweichen | избѣгать | unikać.

\uventry{ezok$'$}
brochet | pike (fish) | Hecht | щука | szczupak.

\uvlitero{F}

\uventry{fab$'$}
fève | bean | Bohne | бобъ | bób, fasola.

\uventry{fabel$'$}
conte | tale, story | Mährchen | сказка | bajka.

\uventry{fabl$'$}
fable | fable | Fabel | басня | baśń.

\uventry{fabrik$'$}
fabrique | fabric | Fabrik | фабрика | fabryka.

\uventry{facet$'$}
facette | facet | Facette | грань | grań.

\uventry{facil$'$}
facile | easy | leicht | легкій | łatwy, lekki.

\uventry{faden$'$}
fil | thread | Faden (zum Nähen etc.) | нить | nić.

\uvsubentry{}\uventry{metal$'$faden$'$}
fil de métal | thread, wire | Draht | проволока | drut.

\uventry{fag$'$}
hêtre | beech-tree | Buche | букъ | buk.

\uventry{fajenc$'$}
faïence | delft ware | Fayence | фаянсъ | fajans.

\uventry{fajf$'$}
siffler | whistle | pfeifen | свистать | świstać.

\uventry{fajl$'$}
limer | file | feilen | пилить | piłować.

\uventry{fajr$'$}
feu | fire | Feuer | огонь | ogień.

\uventry{fak$'$}
compartiment, branche | drawer, department | Fach | разгородка,
отдѣленіе | przegródka.

\uventry{faktur$'$}
facture | bill of lading | Factur, Frachtbrief | накладная | faktura.

\uventry{fal$'$}
tomber | fall | fallen | падать | padać.

\uvsubentry{}\uventry{fal$'$et$'$}
broncher | stumble | stolpern | спотыкаться | potknąć się.

\uventry{falbal$'$}
falbala | furbelow | Falbel | фалбала | falbana.

\uventry{falĉ$'$}
faucher | mow, cut grass | mähen | косить | kosić.

\uvsubentry{}\uventry{falĉ$'$il$'$}
faux | scythe | Sense | коса (для травы) | kosa.

\uventry{fald$'$}
plier | fold | falten | складывать (въ складки) | fałdować.

\uventry{falk$'$}
faucon | falcon | Falke | соколъ | sokół.

\uventry{fals$'$}
falsifier | falsify | fälschen | фальшивить, поддѣлывать | fałszować.

\uventry{fam$'$}
bruit | fame, rumour | Gerücht | молва | pogłoska.

\uventry{famili$'$}
famille | family | Familie | семейство | rodzina.

\uventry{fand$'$}
fondre | melt, cast | giessen, schmelzen | топить, растоплять | roztapiać.

\uventry{fanfaron$'$}
se vanter, faire le glorieur | boast, brag | prahlen | хвастать | przechwalać się.

\uventry{fantom$'$}
spectre, fantôme | vision, ghost | Gespenst | привидѣніе | widmo, upiór.

\uventry{far$'$}
faire | do | thun, machen | дѣлать | robić, czynić.

\uvsubentry{}\uventry{far$'$iĝ$'$}
devenir (se faire) | become | werden | дѣлаться | stawać się.

\uventry{faring$'$}
pharynx | throat | Schlund | глотка | gardziel.

\uventry{farm$'$}
affermer | farm | pachten | арендовать | dzierżawić.

\uventry{fart$'$}
se porter (santé) | be (well or unwell) | sich (wohl oder
nicht wohl) befinden | поживать | mieć się.

\uvsubentry{}\uventry{fart$'$o}
état | state | Zustand | состояніе | stan.

\uventry{farun$'$}
farine | meal, farina | Mehl | мука | mąka.

\uventry{fask$'$}
touffe, faisceau | bundle | Büschel, Bündel | пукъ, пучекъ | wiązka, pęk.

\uventry{fason$'$}
façon | fashion, cut | Façon | фасонъ | fason.

\uventry{fast$'$}
jeûner | fast (vb.) | fasten | поститься | pościć.

\uventry{faŭk$'$}
gueule | jaw | Rachen | зѣвъ | paszcza.

\uventry{fav$'$}
teigne | scurf | Grind, Räude | парша | parch.

\uventry{favor$'$}
favorable | favorable | günstig | благосклонный | przychylny.

\uventry{fazan$'$}
faisan | pheasant | Fasan | фазанъ | baźant.

\uventry{fe$'$in$'$}
fée | fairy | Fee | фея | wieszczka.

\uventry{febr$'$}
fièvre | fever | Fieber | лихорадка | febra.

\uventry{Februar$'$}
Février | February | Februar | Февраль | Luty.

\uventry{feĉ$'$}
lie | yeast | Hefen | дрожжи | droźdźe.

\uventry{fel$'$}
peau, fourrure | hide, fleece | Fell | шкура, мѣхъ | skora.

\uventry{feliĉ$'$}
heureux | happy | glücklich | счастливый | szczęśliwy.

\uventry{felt$'$}
feutre | felt | Filz | войлокъ | pilśń.

\uventry{femur$'$}
haut de la cuisse | thigh | Schenkel (Ober-) | бедро, ляшка | biodro.

\uventry{fend$'$}
fendre | split | spalten | раскалывать | łupać, rozpłatać.

\uventry{fenestr$'$}
fenêtre | window | Fenster | окно | okno.

\uventry{fenkol$'$}
fenouil | fennel | Fenchel | укропъ | kopr włoski.

\uventry{fer$'$}
fer | iron | Eisen | желѣзо | źelazo.

\uvsubentry{}\uventry{fer$'$voj$'$}
chemin de fer | railway | Eisenbahn | желѣзная дорога | droga źelazna, kolej.

\uventry{ferdek$'$}
pont, tillac | deck | Verdeck | палуба | pokład.

\uventry{ferm$'$}
fermer | shut | schliessen, zumachen | запирать | zamykać.

\uvsubentry{}\uventry{mal$'$ferm$'$}
ouvrir | open | öffnen | отворять | otwierać.

\uventry{ferment$'$}
fermenter | ferment | gähren | бродить, приходить въ
броженіе | fermentować.

\uventry{fervor$'$}
zèle, ferveur | zeal, ardour | Eifer | усердіе | gorliwość.

\uventry{fest$'$}
fêter | feast | feiern | праздновать | świętować.

\uventry{festen$'$}
banqueter | feast, banquet | schmausen | пировать | ucztować.

\uventry{fianĉ$'$}
fiancé | betrothed person | Bräutigam | женихъ | narzeczony.

\uventry{fibr$'$}
fibre | fibre | Faser | волокно | włókno.

\uventry{fid$'$}
se fier | rely upon, trust on | sich verlassen | полагаться на
к. н. | polegać, spuszczać się.

\uventry{fidel$'$}
fidèle | faithful | treu | вѣрный | wierny.

\uventry{fier$'$}
fier, orgueilleux | proud | stolz | гордый | dumny.

\uventry{fig$'$}
figue | fig | Feige | фига | figa.

\uventry{figur$'$}
figurer | figure, represent | abbilden | изображать | odmalować, wyobraźać.

\uventry{fil$'$}
fils | son | Sohn | сынъ | syn.

\uventry{filik$'$}
fougère | fern | Farrnkraut | папоротникъ | paproć.

\uventry{filtr$'$}
couler, filtrer | strain, filter | seihen | цѣдить | cedzić.

\uventry{fin$'$}
finir | end | enden, beendigen | кончать | kończyć.

\uventry{fingr$'$}
doigt | finger | Finger | палецъ | palec.

\uventry{firm$'$}
ferme, compacte | firm | fest | плотный | staly, mocny.

\uvsubentry{}\uventry{firm$'$o}
raison (de commerce), enseigne | firm | Firma | фирма | firma.

\uventry{fiŝ$'$}
poisson | fish | Fisch | рыба | ryba.

\uvsubentry{}\uventry{fiŝ$'$ole$'$}
huile de poisson | fish-oil | Thran | рыбій жиръ | tran.

\uventry{fistul$'$}
fistule | fistula | Fistel | фистула | fistuła.

\uventry{flag$'$}
bannière | flag | Flagge | флагъ | bandera, flaga.

\uventry{flam$'$}
flamme | flame | Flamme | пламя | płomień.

\uventry{flan$'$}
flan | flawn | Fladen | блинъ | placek.

\uventry{flanel$'$}
flanelle | flannel | Flanell | фланель | flanela.

\uventry{flank$'$}
côté | side | Seite | сторона | strona.

\uventry{flar$'$}
flairer, sentir | smell (vb.) | riechen, schnupfen | нюхать | wąchać.

\uventry{flat$'$}
flatter | flatter | schmeicheln | льстить | pochlebiać.

\uventry{flav$'$}
jaune | yellow | gelb | желтый | źółty.

\uventry{fleg$'$}
soigner | nourish | warten, pflegen | ухаживать | pielęgnować.

\uventry{flegm$'$}
flegme | phlegm | Phlegma | флегма | flegma.

\uventry{fleks$'$}
fléchir, ployer | bend | biegen | гнуть | giąć.

\uventry{flik$'$}
rapiécer, refaire | patch, repair | ausflicken, einen Flick auflegen | починять | łatać.

\uventry{flirt$'$}
voltiger, voleter | flirt, flutter | flattern | порхать | trzepotać.

\uventry{flok$'$}
flocon | flake, fluck | Flocke | клокъ, хлопокъ | kosmyk,
kłaczet.

\uventry{flor$'$}
fleurir | flourish | blühen | цвѣсти | kwitnąć.

\uvsubentry{}\uventry{flor$'$o}
fleur | flower, bloom | Blume | цвѣтокъ | kwiat.

\uventry{flos$'$}
radeau, flottage | float, raft | Floss | плотъ | tratwa, flis.

\uventry{flu$'$}
couler | flow | fliessen | течь | plynąć, cieknąć.

\uvsubentry{}\uventry{de$'$flu$'$il$'$}
rigole, égout, cannelure | gutter, channel | Rinne | водостокъ | rynna.

\uventry{flug$'$}
voler (avec des ailes) | fly (vb.) | fliegen | летать | latać.

\uvsubentry{}\uventry{flug$'$il$'$}
aile | wing | Flügel | крыло | skrzydło.

\uventry{fluid$'$}
liquide | fluid | flüssig | жидкій | płynny.

\uventry{flut$'$}
flûte | flute | Flöte | флейта | flet.

\uventry{foir$'$}
foire | fair (subst.) | Jahrmarkt, Messe | ярмарка | jarmark.

\uventry{foj$'$}
fois | time (e. g. three times etc.) | Mal, einmal | разъ | raz.

\uventry{fojn$'$}
foin | hay | Heu | сѣно | siano.

\uventry{fok$'$}
chien de mer, phoque | seal (animal) | Seehund | тюлень | foka,
pies morski.

\uventry{fokus$'$}
foyer | focus | Fokus | фокусъ | ognisko.

\uventry{foli$'$}
feuille | leaf | Blatt, Bogen (Papier) | листъ | liść,
arkusz.

\uventry{fond$'$}
fonder | found | gründen | основывать | zakładać.

\uventry{font$'$}
source | fountain | Quelle | источникъ | żródło.

\uventry{fontan$'$}
fontaine (jaillissante) | fountain, wellspring | Springbrunnen | фонтанъ | wodotrysk.

\uventry{for}
loin, hors | forth, out | fort | прочь | precz.

\uventry{forĝ$'$}
forger | forge | schmieden | ковать | kuć.

\uventry{forges$'$}
oublier | forget | vergessen | забывать | zapominać.

\uventry{fork$'$}
fourchette | fork | Gabel | вилы, вилка | widły, widelec.

\uventry{form$'$}
forme | form | Form | форма | forma, kształt.

\uventry{formik$'$}
fourmi | ant | Ameise | муравей | mrówka.

\uventry{forn$'$}
fourneau, poële, four | stove | Ofen | печь, печка | piec.

\uventry{for$'$permes$'$}
donner congé | give furlough | beurlauben | увольнять въ
отпускъ | dać urlop.

\uventry{fort$'$}
fort | strong | stark, kräftig | сильный | silny.

\uventry{fortepian$'$}
clavecin | piano-forte | Clavier | рояль | fortepian.

\uventry{fortik$'$}
solide, robuste | solid, durable | fest, haltbar | прочный,
крѣпкій | mocny, trwały.

\uvsubentry{}\uventry{fortik$'$aĵ$'$}
forteresse | fortress | Festung | крѣпость | twierdza.

\uventry{fos$'$}
creuser | dig | graben | копать | kopać.

\uvsubentry{}\uventry{fos$'$il$'$}
bêche | spade | Spaten | заступъ | rydel.

\uventry{fosfor$'$}
phosphore | phosphorus | Phosphor | фосфоръ | fosfor.

\uventry{fost$'$}
poteau | post, stake | Pfosten | косякъ | przywój.

\uventry{frag$'$}
fraise | strawberry | Erdbeere | земляника | poziomka.

\uventry{fragment$'$}
fragment | fragment | Bruchstück | отрывокъ | urywek.

\uventry{fraj$'$}
frai (des poissons) | spawn | Laich | икра | ikra.

\uventry{frak$'$}
frac | dress-coat | Frack | фракъ | frak.

\uventry{frakas$'$}
broyer, écraser | bruise, triturate | zermalmen | разможжать | druzgotać.

\uventry{fraksen$'$}
frêne | ash | Esche | ясень | jesion.

\uventry{framason$'$}
franc-maçon | freemason | Freimaurer | масонъ | mason,
wolny mularz.

\uventry{framb$'$}
framboise | raspberry | Himbeere | малина | malina.

\uventry{frand$'$}
goûter par friandise | junket | naschen | лакомиться | złakomić się.

\uventry{franĝ$'$}
frange | fringe | Franse | бахрома | frędzla.

\uventry{frangol$'$}
bourdaine | alder | Faulbaum | черемуха | wilczyna.

\uventry{frap$'$}
frapper | hit | klopfen | стучать, ударять | stukać, uderzać.

\uventry{frat$'$}
frère | brother | Bruder | братъ | brat.

\uventry{fraŭl$'$}
homme non marié | bachelor | unverheiratheter Herr | холостой
господинъ | kawaler.

\uvsubentry{}\uventry{fraŭl$'$in$'$}
demoiselle, mademoiselle | miss | Fräulein | барышня | panna.

\uventry{fremd$'$}
étranger | strange, foreign | fremd | чужой | obcy.

\uventry{frenez$'$}
fou | crazy | wahnsinnig | сумашедшій | obłąkany.

\uventry{freŝ$'$}
frais, récent | fresh | frisch | свѣжій | świeźy.

\uventry{fring$'$}
pinson | finch | Finke | зябликъ | zięba.

\uventry{fringel$'$}
serin (oiseau) | siskin | Zeisig | чижъ | czyźyk.

\uventry{fripon$'$}
fripon, coquin | rogue, knave | Spitzbube, Schelm | мошенникъ | szelma.

\uventry{friz$'$}
friser | friz, frizzle | frisiren | причесывать | uczesać,
trefić.

\uventry{fromaĝ$'$}
fromage | cheese | Käse | сыръ | ser.

\uventry{front$'$}
front | front | Fronte | фронтъ | front.

\uventry{frost$'$}
gelée | frost | Frost | морозъ | mróz.

\uventry{frot$'$}
frotter | rub | reiben | тереть | trzeć.

\uventry{fru$'$}
de bonne heure | early | früh | рано | rano, wcześnie.

\uventry{frugileg$'$}
freux, grolle | rook | Saatkrähe | грачъ | siewka.

\uventry{frukt$'$}
fruit | fruit | Frucht | плодъ | owoc.

\uventry{frunt$'$}
front | forehead | Stirn | лобъ | czoło.

\uventry{ftiz$'$}
phtisie | phthisis, consumption | Schwindsucht | чахотка | suchoty.

\uventry{fulg$'$}
suie | soot | Russ | сажа | sadza.

\uventry{fulm$'$}
éclair | lightning | Blitz | молнія | błyskawica.

\uventry{fum$'$}
fumée | smoke | Rauch | дымъ | dym.

\uvsubentry{}\uventry{fum$'$i}
fumer | smoke, fume | rauchen | курить | palić.

\uventry{fund$'$}
fond | bottom | Boden, Grund | дно | dno.

\uventry{fundament$'$}
fondement | foundation | Fundament | основаніе | fundament.

\uventry{funebr$'$}
deuil | funeral | Trauer | трауръ | źałoba.

\uvsubentry{}\uventry{funebr$'$a}
funèbre | funeral | Trauer-, Leichen- | траурный | źałobny.

\uventry{funel$'$}
entonnoir | funnel, mill-hopper | Trichter | воронка | lejek.

\uventry{fung$'$}
champignon | mushroom | Pilz | грибъ | grzyb.

\uventry{funt$'$}
livre | pound | Pfund | фунтъ | funt.

\uventry{furaĝ$'$}
fourrage | forage | Fourrage, Futter | фуражъ | furaź.

\uventry{furioz$'$}
furieux | furious, raging | toll, wüthend | бѣшенный | wściekły.

\uventry{furunk$'$}
furoncle | furuncle | Furunkel | чирей | czyriak.

\uventry{fuŝ$'$}
bousiller | bungle, spoil trade | pfuschen | кропать, плохо
работать | partaczyć.

\uventry{fusten$'$}
futaine | fustian | Barchent | бумазея | barchan.

\uventry{fut$'$}
pied (mesure) | foot | Fuss (Mass) | футъ | stopa.

\uvlitero{G, Ĝ}

\uventry{gad$'$}
merluche | stock-fish | Stockfisch | треска | sztokfisz.

\uventry{gaj$'$}
gai | gay, glad | lustig, fröhlich | веселый | wesoły.

\uventry{gajl$'$}
noix de galle | oak-apple | Gallapfel | чернильный орѣхъ | galas, dębianka.

\uventry{gajn$'$}
gagner | gain | gewinnen | выигрывать | wygrywać.

\uventry{gal$'$}
bile | gall | Galle | желчь | źółć.

\uventry{galanteri$'$}
nippes | millinery | Galanterie-Waare | галантерейный
товаръ | towar galanteryjny.

\uventry{galeri$'$}
galerie | gallery | Gallerie | галлерея | galerya.

\uventry{galon$'$}
galon | galloon | Galone | галунъ | galon.

\uventry{galoŝ$'$}
galoche | rubber-shoe | Galosche | калоша | kalosz.

\uventry{gam$'$}
gamme | gammut | Gamme | гамма | gama.

\uventry{gamaŝ$'$}
guêtre | gaiter | Gamasche | штиблетъ | kamasz.

\uventry{gant$'$}
gant | glove | Handschuh | перчатка | rękawiczka.

\uventry{garanti$'$}
garantir | warrant | bürgen | ручаться | ręczyć.

\uvsubentry{}\uventry{garanti$'$aĵ$'$}
gage | pawn, pledge | Pfand | залогъ | zastaw.


\uvsubentry{}\uventry{garanti$'$ul$'$}
otage | hostage | Geissel | заложникъ | zakładnik.

\uventry{garb$'$}
gerbe | sheaf, shock | Garbe | снопъ | snop.

\uventry{gard$'$}
garder (prendre soin) | guard | hüten | стеречь, беречь | strzedz.

\uventry{ĝarden$'$}
jardin | garden | Garten | садъ | ogród.

\uventry{gargar$'$}
rincer | rinse | spülen | полоскать | płókać.

\uventry{gas$'$}
gaz | gas | Gas | газъ | gaz.

\uventry{gast$'$}
hôte | guest | Gast | гость | gość.

\uventry{gazel$'$}
gazelle | gazel | Gazelle | газель | gazela.

\uventry{gazet$'$}
gazette | gazette, news-paper | Zeitung | газета | gazeta.

\uventry{ge$'$}
les deux sexes réunis; ex. \uventry{patr$'$} père ― \uventry{ge$'$patr$'$o$'$j} les
parents (père et mère) | of both sexes; e. g. \uventry{patr$'$} father ―
\uventry{ge$'$patr$'$o$'$j} parents | beiderlei Geschlechtes; z. B. \uventry{patr$'$} Vater
― \uventry{ge$'$patr$'$o$'$j} Eltern; \uventry{mastr$'$} Wirth ― \uventry{ge$'$mastr$'$o$'$j} Wirth und
Wirthin | обоего пола, напр. \uventry{patr$'$} отецъ ― \uventry{ge$'$patr$'$o$'$j} родители;
\uventry{mastr$'$} хозяинъ ― \uventry{ge$'$mastr$'$o$'$j} хозяинъ съ хозяйкой | obojej
płci, np. \uventry{patr$'$} ojciec ― \uventry{ge$'$patr$'$o$'$j} rodzice; \uventry{mastr$'$} gospodarz
― \uventry{ge$'$mastr$'$o$'$j} gospodarstwo (gospodarz i gospodyni).

\uventry{gelaten$'$}
gélatine | jelly | Gallerte | студень, желе | galareta.

\uventry{ĝem$'$}
gémir | groan | stöhnen | стонать | stękać.

\uventry{ĝen$'$}
gêner, serrer | constrain, embarass | genireu | стѣснять | źenować.

\uventry{generaci$'$}
génération | generation | Geschlecht, Generation | поколѣніе | pokolenie.

\uventry{genitiv$'$}
génitif | genitive | Genitiv | родительный падежъ | dopełniacz.

\uventry{genot$'$}
genet, genette | genet | Genettkatze | енотъ | junat.

\uventry{gent$'$}
race | race, kind, genus | Geschlecht, Stamm | племя | plemię.

\uventry{ĝentil$'$}
gentil, poli | gentle | höflich | вѣжливый | grzeczny.

\uventry{genu$'$}
genou | knee | Knie | колѣно | kolano.

\uventry{ĝerm$'$}
germe | bud, sprig | Keim | ростокъ | kiełek.

\uventry{gest$'$}
geste | gesture | Geberde | жестъ, тѣлодвиженіе | giest, ruch
ciała.

\uventry{ĝi}
cela, il, elle | it | es, dieses | оно, это | ono, to.

\uventry{ĝib$'$}
bosse | hump | Buckel, Höcker | горбъ | garb.

\uventry{gips$'$}
plâtre | gypsum | Gips | гипсъ | gips.

\uventry{ĝiraf$'$}
giraffe | girafe | Giraffe | жираффъ | źyrafa.

\uventry{ĝis}
jusqu’à, jusqu’à ce que | up to, until | bis | до | do, aż.

\uventry{gitar$'$}
guitare | guitar | Guitarre | гитара | gitara.

\uventry{glaci$'$}
glace | ice | Eis | ледъ | lód.

\uvsubentry{}\uventry{glaci$'$aĵ$'$}
glaces | ice | Gefrornes | мороженное | lody.

\uventry{glad$'$}
repasser (du linge) | smoothe | plätten | гладить (бѣлье) | prasować.

\uventry{glan$'$}
gland | acorn | Eichel | желудь | żołądź.

\uventry{gland$'$}
glande, glandule | gland, glandule | Drüse | железа | gruczoł.

\uventry{glas$'$}
verre (à boire) | glass, vase | Glas (Gefäss) | стаканъ | szklanka.

\uventry{glat$'$}
uni, lisse | slippery | glatt | гладкій | gładki.

\uventry{glav$'$}
glaive, épée | sword | Schwert | мечъ | miecz.

\uventry{glim$'$}
mica | glimmer | Glimmer | слюда | łyszczak.

\uventry{glit$'$}
glisser | glide | gleiten, glitschen | скользить | ślizgać
się.

\uvsubentry{}\uventry{glit$'$il$'$}
patin | skate | Schlittschuh | коньки | łyżwa.


\uvsubentry{}\uventry{glit$'$vetur$'$il$'$}
traîneau | sled | Schlitten | сани | sanie.

\uventry{glob$'$}
boule, globe | globe | Kugel | шаръ | kula, gałka.

\uventry{glor$'$}
glorifier | glory | rühmen, preisen | славить | wysławiać.

\uventry{glu$'$}
coller | glue | leimen | клеить | kleić.

\uventry{glut$'$}
avaler, engloutir | swallow (vb.) | schlingen, schlucken | глотать | połykać.

\uventry{gobi$'$}
goujon | gudgeon | Gründling | пискарь | kiełb.

\uventry{ĝoj$'$}
se réjouir | joy | sich freuen | радоваться | cieszyć się.

\uventry{golf$'$}
baie | bay | Bucht, Meerbusen | бухта, заливъ | zatoka.

\uventry{gorĝ$'$}
gorge, gosier | throat | Kehle, Gurgel, Hals | горло | gardło.

\uventry{graci$'$}
délié | slender | schlank | стройный | wysmukły, hoży.

\uventry{grad$'$}
degré | degree | Grad, Stufe | градусъ, степень | stopień.

\uventry{graf$'$}
comte | earl, count | Graf | графъ | hrabia.

\uventry{grajn$'$}
grain, pépin | a grain | Korn, Körnchen | зерно | ziarno.

\uventry{gramatik$'$}
grammaire | grammar | Grammatik | грамматика | gramatyka.

\uventry{granat$'$}
grenade | pomegranate | Granatapfel | гранатное яблоко | granatowe jabłko.

\uventry{grand$'$}
grand | great, tall | gross | большой, великій | wielki,
duźy.

\uvsubentry{}\uventry{grand$'$anim$'$}
magnanime | magnanimous | grossmüthig | великодушный | wspaniałomyślny.

\uventry{granit$'$}
granit | granite | Granit | гранитъ | granit.

\uventry{gras$'$}
graisse | fat | Fett | жиръ | tłuszcz.

\uventry{grat$'$}
gratter | scratch | kratzen, ritzen | царапать | drapać.

\uventry{gratul$'$}
féliciter | congratulate | gratuliren | поздравлять | winszować.

\uventry{grav$'$}
grave | important | wichtig | важный | waźny.

\uventry{graved$'$}
enceinte, grosse | pregnant | schwanger | беременная | cięźarna.

\uventry{gravur$'$}
graver | grave, engrave | graviren | гравировать | rytować.

\uventry{gren$'$}
blé | grain | Korn, Getreide | хлѣбъ, жито | zboże.

\uvsubentry{}\uventry{gren$'$ej$'$}
grenier | granary, ware-house | Speicher | амбаръ | spichrz.

\uventry{grenad$'$}
grenade | grenade | Granate | граната | granata.

\uventry{gri$'$}
gruau | groats | Grütze | крупа | kasza, krupa.

\uventry{grifel$'$}
burin, style | pin, pencil, style | Griffel | грифель | gryfel.

\uventry{gril$'$}
grillon | cricket (insect) | Grille | сверчокъ | świerk.

\uventry{grimac$'$}
grimace | grimace | Grimasse | гримасса, ужимка | grymas.

\uventry{grinc$'$}
grincer | grate, bruise | knirschen | скрежетать | zgrzytać.

\uventry{griz$'$}
gris | grey | grau | сѣрый, сѣдой | szary, siwy.

\uventry{gros$'$}
groseille à maquereau | gooseberry | Stachelbeere | крыжовникъ | agrest.

\uventry{groŝ$'$}
gros | groat | Groschen | грошъ | grosz.

\uventry{grot$'$}
grotte | grot | Grotte | гротъ | grota.

\uventry{gru$'$}
grue (oiseau) | crane (bird) | Kranich | журавль | źóraw.

\uventry{grup$'$}
groupe | group | Gruppe | группа | grupa.

\uventry{ĝu$'$}
jouir, prendre | enjoy, have the use of | geniessen, sich
erquicken | наслаждаться | uźywać, doznawać, cieszyć się.

\uventry{gudr$'$}
goudron | tar | Theer | деготь | dziegieć.

\uventry{guf$'$}
grand-duc | owl | Uhu | филинъ | puchacz.

\uventry{gum$'$}
gomme | gum, mucilage | Gummi | гумми, камедь | guma.

\uventry{gurd$'$}
orgue de Barbarie | german organ | Leierkasten | шарманка | katarynka.

\uventry{gust$'$}
goût | taste | Geschmack | вкусъ | smak, gust.

\uvsubentry{}\uventry{gust$'$um$'$}
goûter, essayer | taste | kosten, schmecken | отвѣдывать | kosztować, próbować.

\uventry{ĝust$'$}
juste, correct | straight, just | recht, richtig | какъ разъ,
вѣрно | właściwy.

\uventry{gut$'$}
dégoutter | drop | tropfen, triefen | капать | kapać.

\uvsubentry{}\uventry{gut$'$o}
goutte | drop | Tropfen | капля | kropla.

\uventry{guvern$'$}
gouverner | govern, rule | lenken, erziehen | наставлять | kierować, wychowywać.

\uventry{gvardi$'$}
garde | guard | Garde | гвардія | gwardya.

\uventry{gvid$'$}
guider | guide | leiten, anleiten | руководствовать | być
przewodnikiem.

\uvlitero{H, Ĥ}

\uventry{ha!}
ah! | ah, alas | a! ach! | а! ахъ! | a! ach!.

\uventry{hajl$'$}
grêle | hail | Hagel | градъ | grad.

\uventry{hak$'$}
hacher, abattre | hew, chop | hauen, hacken | рубить | rąbać.

\uvsubentry{}\uventry{hak$'$il$'$}
hache | hatchet, axe | Beil, Axt | топоръ | siekiera.

\uventry{hal$'$}
halle | hall | Halle | зала (базарная) | halla.

\uventry{haladz$'$}
exhalaison mauvaise | exhalation | Dunst | угаръ | swąd, czad.

\uventry{halt$'$}
s’arrêter | come to a stop | anhalten, Halt machen, stocken | останавливиться | stawać, zatrzymywać się.

\uventry{hamstr$'$}
hamster | hamster | Hamster | хомякъ | chomik.

\uventry{ĥaos$'$}
chaos | chaos | Chaos | хаосъ | zamęt, chaos.

\uventry{har$'$}
cheveu | hair | Haar | волосъ | włos.

\uvsubentry{}\uventry{har$'$ar$'$}
perruque | periwig | Perücke | парикъ | peruka.


\uvsubentry{}\uventry{har$'$eg$'$}
soie de cochon | bristle | Borste | щетина | szczecina.


\uvsubentry{}\uventry{har$'$lig$'$}
tresse de cheveux | weft of hair | Zopf | коса (волосъ) | warkocz, kosa.

\uventry{hard$'$}
endurcir | harden | abhärten | закалять | hartować.

\uventry{haring$'$}
hareng | herring | Häring | селедка | śledź.

\uventry{harp$'$}
harpe | harp | Harfe | арфа | arfa.

\uventry{haŭt$'$}
peau | skin | Haut | кожа | skóra.

\uventry{hav$'$}
avoir | have | haben | имѣть | mieć.

\uventry{haven$'$}
port, hâvre | port, harbour | Hafen | гавань | przystań,
port.

\uventry{heder$'$}
lierre | ivy | Epheu | плющъ | bluszcz.

\uventry{hejm$'$}
maison, patrie | home | daheim, Heimat | дома | dom, ojczyzna.

\uventry{hejt$'$}
chauffer, faire du feu | heat (vb.) | heizen | топить (печку) | palić (w piecu).

\uventry{hel$'$}
clair (qui n’est pas obscur) | clear, glaring | hell, grell | яркій | jasny, jaskrawy.

\uventry{help$'$}
aider | help | helfen | помогать | pomagać.

\uvsubentry{}\uventry{mal$'$help$'$}
déranger, empêcher | hinder | stören, hindern | мѣшать, препятствовать | przeszkadzać.

\uventry{ĥemi$'$}
chimie | chemistry | Chemie | химія | chemia.

\uventry{hepat$'$}
foie | liver | Leber | печень | wątroba.

\uventry{herb$'$}
herbe | grass | Gras | трава | trawa.

\uventry{herb$'$ej$'$}
pré, prairie | meadow, green field | Wiese | лугъ | łąka.

\uventry{hered$'$}
hériter | inherit | erben | наслѣдовать | dziedziczyć.

\uventry{herez$'$}
hérésie | heresy | Ketzerei | ересь | kacerstwo, herezja.

\uventry{herni$'$}
hernie | hernia | Bruch (Heilk.) | грыжа | ruptura, pzepuklina.

\uventry{hero$'$}
héros | hero, champion | Held | герой | bohater.

\uventry{hidrarg$'$}
vif-argent, mercure | quicksilver | Quecksilber | ртуть | rtęć.

\uventry{hidrogen$'$}
hydrogène | hydrogen | Wasserstoff | водородъ | wodór.

\uventry{hieraŭ}
hier | yesterday | gestern | вчера | wczoraj.

\uventry{ĥimer$'$}
chimère | chimera | Chimäre | химера | chimera.

\uventry{hipokrit$'$}
faire l’hypoćrite | feign, play the hypocrite | heucheln | лицемѣрить | być obłudnikiem.

\uventry{hirud$'$}
sangsue | leech | Blutegel | піявка | pijawka.

\uventry{hirund$'$}
hirondelle | swallow (bird) | Schwalbe | ласточка | jaskółka.

\uventry{hiskiam$'$}
jusquiame | henbane | Bilsenkraut | бѣлена | szaleń.

\uventry{histori$'$}
histoire | history, story | Geschichte | исторія | historya.

\uventry{histrik$'$}
porc-épic, hérisson | porcupine | Stachelschwein | дикобразъ | jeż cudzoziemski.

\uventry{ho!}
oh! | oh! | oh! och! | о! охъ! | o! och!.

\uventry{hodiaŭ}
aujourd’hui | to-day | heute | сегодня | dziś.

\uventry{hok$'$}
croc, crochet | hook | Haken, Angel | крюкъ | hak.

\uvsubentry{}\uventry{fiŝ$'$hok$'$}
hameçon | fishing-hook | Fischangel | уда, удочка | wędka.


\uvsubentry{}\uventry{pord$'$hok$'$}
gond (d’une porte) | hinge (of a door) | Thürangel | дверной крюкъ | zawiasa.

\uventry{ĥoler$'$}
choléra | cholera | Cholera | холера | cholera.

\uventry{hom$'$}
homme (l’espèce) | man | Mensch | человѣкъ | człowiek.

\uventry{honest$'$}
honnête | honest | ehrlich | честный | uczciwy.

\uventry{honor$'$}
honorer | honor | ehren | чтить | czcić.

\uvsubentry{}\uventry{honor$'$o}
honneur | honour | ehren | честь | cześć, zaszczyt.

\uventry{hont$'$}
avoir honte | shame | sich schämen | стыдиться | wstydzić
się.

\uventry{hor$'$}
heure | hour | Stunde | часъ | godzina.

\uventry{ĥor$'$}
chœur | chorus, choir | Chor | хоръ | chór.

\uventry{horde$'$}
orge | barley | Gerste | ячмень | jęczmień.

\uventry{horizontal$'$}
horizontal | horizontal | wagerecht | горизонтальный | poziomy.

\uventry{horloĝ$'$}
horloge, montre | clock | Uhr | часы | zegar.

\uventry{hortulan$'$}
ortolan | ortolan | Gartenammer | овсянка | poświerka.

\uventry{hosti$'$}
hostie | host | Weihbrod | просвора | hostya.

\uventry{hotel$'$}
hôtel | hotel | Herberge, Gasthaus | гостинница | hotel,
zajazd.

\uventry{huf$'$}
sabot, corne | hoof | Huf | копыто | kopyto.

\uventry{humil$'$}
humble | humble | demüthig | покорный | pokorny.

\uventry{humor$'$}
humeur (caractère) | humor | Laune | расположеніе духа | humor.

\uventry{hund$'$}
chien | dog | Hund | песъ, собака | pies.

\uventry{husar$'$}
houssard | hussar | Husar | гусаръ | huzar.

\uventry{huz$'$}
grand esturgeon | huso, sturgeon | Hausen | бѣлуга | wyz.

\uvlitero{I}

\uventry{i}
marque l’infinitif; ex. \uventry{laŭd$'$i} louer | termination of the
intinitive in verbs; e. g. \uventry{laŭd$'$i} to praise | bezeichnet den
Intinitiv; z. B. \uventry{laŭd$'$i} loben | означаетъ неопредѣленное наклоненіе;
напр. \uventry{laŭd$'$i} хвалить | oznacza tryb bezokoliczny słowa; np. \uventry{laŭd$'$i}
chwalić.

\uventry{ia}
quelconque, quelque | of any kind | irgend welcher | какой-нибудь | jakiś.

\uventry{ial}
pour une raison quelconque | for any cause | irgend warum | почему-нибудь | dla jakiejś przyczyny.

\uventry{iam}
jamais, un jour | at any time, ever | irgend wann, einst | когда-нибудь | kiedyś.

\uventry{ibis$'$}
ibis | ibis | Ibis | ибисъ | łukodziób.

\uventry{id$'$}
enfant, descendant; ex. \uventry{bov$'$} bœuf ― \uventry{bov$'$id$'$} veau;
\uventry{Izrael$'$} Israël ― \uventry{Izrael$'$id$'$} Israëlite | descendant, young one;
e. g. \uventry{bov$'$} ox ― \uventry{bov$'$id$'$} calf | Kind, Nachkomme; z. B. \uventry{bov$'$}
Ochs ― \uventry{bov$'$id$'$} Kalb; \uventry{Izrael$'$} Israel ― \uventry{Izrael$'$id$'$} Israelit | дитя, потомокъ; напр. \uventry{bov$'$} быкъ ― \uventry{bov$'$id$'$} теленокъ; \uventry{Izrael$'$}
Израиль ― \uventry{Izrael$'$id$'$} израильтянинъ | dziecię, potomek; np. \uventry{bov$'$}
byk ― \uventry{bov$'$id$'$} cielę; \uventry{Iszrael$'$} Izrael ― \uventry{Izrael$'$id$'$} Izraelita.

\uventry{idili$'$}
idylle | idyl | Idylle | идиллія | sielanka.

\uventry{idol$'$}
idole | idol | Abgott | идолъ | boźek, bałwan.

\uventry{ie}
quelque part | anywhere | irgend wo | гдѣ-нибудь | gdzieś.

\uventry{iel}
d’une manière quelconque | anyhow | irgend wie | какъ-нибудь | jakoś.

\uventry{ies}
de quelqu’un | anyone’s | irgend jemandes | чей-нибудь | czyjś.

\uventry{ig$'$}
faire...; ex. \uventry{pur$'$} pur, propre ― \uventry{pur$'$ig$'$} nettoyer; \uventry{mort$'$}
mourir ― \uventry{mort$'$ig$'$} tuer (faire mourir) | to cause to be;
e. g. \uventry{pur$'$} pure ― \uventry{pur$'$ig$'$} purify | zu etwas machen, lassen;
z. B. \uventry{pur$'$} rein ― \uventry{pur$'$ig$'$} reinigen; \uventry{brul$'$} brennen (selbst) ―
\uventry{brul$'$ig$'$} brennen (etwas) | дѣлать чѣмъ-нибудь, заставить дѣлать;
напр. \uventry{pur$'$} чистый ― \uventry{pur$'$ig$'$} чистить; \uventry{brul$'$} горѣть ―
\uventry{brul$'$ig$'$} жечь | robić czemś; np. \uventry{pur$'$} czysty ― \uventry{pur$'$ig$'$} czyścić;
\uventry{brul$'$} palić się ― \uventry{brul$'$ig$'$} palić.

\uventry{iĝ$'$}
se faire, devenir...; ex. \uventry{pal$'$} pâle ― \uventry{pal$'$iĝ$'$} pâlir;
\uventry{sid$'$} être assis ― \uventry{sid$'$iĝ$'$} s’asseoir | to become; e. g. \uventry{ruĝ$'$}
red ― \uventry{ruĝ$'$iĝ$'$} blush | zu etwas werden, sich zu etwas veranlassen;
z. B. \uventry{pal$'$} blass ― \uventry{pal$'$iĝ$'$} erblassen; \uventry{sid$'$} sitzen ―
\uventry{sid$'$iĝ$'$} sich setzen | дѣлаться чѣмъ нибудь, заставить себя...;
напр. \uventry{pal$'$} блѣдный ― \uventry{pal$'$iĝ$'$} блѣднѣть; \uventry{sid$'$} сидѣть ―
\uventry{sid$'$iĝ$'$} сѣсть | stawać się czemś; np. \uventry{pal$'$} blady ― \uventry{pal$'$iĝ$'$}
blednąć; \uventry{sid$'$} siedzieć ― \uventry{sid$'$iĝ$'$} usiąść.

\uventry{iĥtiokol$'$}
colle de poisson | isinglas | Hausenblase | рыбій клей | karuk.

\uventry{il$'$}
instrument; ex. \uventry{tond$'$} tondre ― \uventry{tond$'$il$'$} ciseaux; \uventry{paf$'$}
tirer (coup de feu) ― \uventry{paf$'$il$'$} fusil | instrument; e. g. \uventry{tond$'$}
shear ― \uventry{tond$'$il$'$} scissors | Werkzeug; z. B. \uventry{tond$'$} scheeren ―
\uventry{tond$'$il$'$} Scheere; \uventry{paf$'$} schiessen ― \uventry{paf$'$il$'$} Flinte | орудіе
для...; напр. \uventry{tond$'$} стричь ― \uventry{tond$'$il$'$} ножницы; \uventry{paf$'$} стрѣлять
― \uventry{paf$'$il$'$} ружье | narzędzie; np. \uventry{tond$'$} strzydz ― \uventry{tond$'$il$'$}
noźyce; \uventry{paf$'$} strzelać ― \uventry{paf$'$il$'$} fuzya.

\uventry{ili}
ils, elles | they | sie (Mehrzahl) | они, онѣ | oni.

\uvsubentry{}\uventry{ili$'$a}
leur | their | ihr | ихъ | ich.

\uventry{ilumin$'$}
illuminer | illuminate | illuminiren | иллюминовать | iluminować.

\uventry{imag$'$}
imaginer | imagine | einbilden | воображать | imaginować.

\uventry{imit$'$}
imiter | imitate | nachahmen | подражать | naśladować.

\uventry{imperi$'$}
empire | empire | Kaiserreich | имперія | cesarstwo.

\uventry{impres$'$}
impression | impression | Eindruck | впечатлѣніе | wraźenie.

\uventry{implik$'$}
impliquer, empêtrer | implicate | verwickeln | запутывать,
осложнять | zawikłać.

\uventry{in$'$}
marque le féminin; ex. \uventry{patr$'$} père ― \uventry{patr$'$in$'$} mère | ending
of feminine words; e. g. \uventry{bov$'$} ox ― \uventry{bov$'$in$'$} cow | bezeichnet das
weibliche Geschlecht; z. B. \uventry{patr$'$} Vater ― \uventry{patr$'$in$'$} Mutter;
\uventry{fianĉ$'$} Bräutigam ― \uventry{fianĉ$'$in$'$} Braut | женскій полъ; напр. \uventry{patr$'$}
отецъ ― \uventry{patr$'$in$'$} мать; \uventry{fianĉ$'$} женихъ ― \uventry{fianĉ$'$in$'$} невѣста | oznacza płeć źeńską; np. \uventry{patr$'$} ojciec ― \uventry{patr$'$in$'$} matka; \uventry{kok$'$} kogut
― \uventry{kok$'$in$'$} kura.

\uventry{incit$'$}
agacer, irriter | provoke, incite | reizen | раздражать | draźnić.

\uventry{ind$'$}
mérite, qui mérite..., qui est digne de...; ex. \uventry{laŭd$'$} louange \uventry{laŭd$'$ind$'$} digne de louange | worth | würdig, werth | достойный | godny, wart.

\uventry{indiferent$'$}
indifférent | indifferent | gleichgültig | равнодушный | obojętny.

\uventry{indign$'$}
s’indigner | be angry | entrüstet sein | негодовать | oburzać się.

\uventry{indulg$'$}
épargner | save | schonen, verschonen | щадить | oszczędzać,
przepuszczać.

\uventry{industri$'$}
industrie | industry | Industrie | промышленность | przemysł.

\uventry{infan$'$}
enfant | child | Kind | дитя | dziecię.

\uventry{infekt$'$}
infecter | infect | anstecken | заражать | zaraźać.

\uventry{infer$'$}
enfer | hell | Hölle | адъ | piekło.

\uventry{influ$'$}
influer | influence | Einfluss haben | вліять | wyvierać
wpływ.

\uventry{infuz$'$}
infuser | infuse | ziehen lassen, infundiren | настаивать,
настойка | wymaczać.

\uventry{ing$'$}
marque l’objet dans lequel se met, ou mieux s’introduit...;
ex. \uventry{kandel$'$} chandelle ― \uventry{kandel$'$ing$'$} chandelier; \uventry{plum$'$} plume
― \uventry{plum$'$ing$'$} porte-plume | holder for; e. g. \uventry{kandel$'$} candle ―
\uventry{kandel$'$ing$'$} candlestick | Gegenstand, in den etwas eingestellt,
eingesetzt wird; z. B. \uventry{kandel$'$} Kerze ― \uventry{kandel$'$ing$'$} Leuchter;
\uventry{plum$'$} Feder ― \uventry{plum$'$ing$'$} Federhalter | вещь, въ которую
вставляется, всаживается; напр. \uventry{kandel$'$} свѣча ― \uventry{kandel$'$ing$'$}
подсвѣчникъ; \uventry{plum$'$} перо ― \uventry{plum$'$ing$'$} ручка для перьевъ | przedmiot, w który się coś wsadza, wstawia; np. \uventry{kandel$'$} świeca ― \uventry{kandel$'$ing$'$} lichtarz; \uventry{plum$'$} pióro ― \uventry{plum$'$ing$'$} obsadka do pióra.

\uventry{ingven$'$}
aine | groin | Leisten-, Weichengegend | пахъ | pachwina.

\uventry{inĝenier$'$}
ingénieur | engineer | Ingenieur | инженеръ | inźynier.

\uventry{iniciat$'$}
causer, engager | cause, engage | anstiften, veranlassen | зачинать, починъ | zapoczątkować, dać inicyatywę.

\uventry{ink$'$}
encre | ink | Dinte | чернила | atrament.

\uventry{inklin$'$}
enclin | inclined | geneigt, bereit | склонный | skłonny.

\uventry{inokul$'$}
inoculer | imp, inoculate | impfen | прививать | szczepić.

\uventry{insekt$'$}
insecte | insect | Insekt | насѣкомое | owad.

\uventry{insid$'$}
tendre des pièges | lay snares | nachstellen | подстерегать | zasadzać się, prześladować.

\uventry{insign$'$}
armes, armoiries | arms | Wappen | гербъ | herb.

\uventry{inspir$'$}
inspirer | inspire | einflössen | вдохновлять | wpajać, natchnąć.

\uventry{instig$'$}
instiguer | instigate | antreiben, anspornen, hetzen | подстрекать | odniecać.

\uventry{institut$'$}
institut | institute | Anstalt | учрежденіе | zakład.

\uventry{instru$'$}
instruire, enseigner | instruct, teach | lehren | учить | uczyć.

\uventry{instrukci$'$}
instruction | instruction | Instruction | инструкція | polecenie.

\uventry{insul$'$}
île | island | Insel | островъ | wyspa.

\uventry{insult$'$}
injurier | insult | schelten, schimpfen | ругать | besztać, łajać, szkalować.

\uventry{int$'$}
marque le participe passé du verbe actif; ex. \uventry{far$'$} faire ―
\uventry{far$'$int$'$} ayant fait | ending of past part. act. in verbs;
e. g. \uventry{am$'$int$'$} having loved | bezeichnet das Particip. perfecti
act. | означаетъ причастіе прошедшаго времени дѣйств. залога | oznacza
imiesłów czynny czasu przeszłego.

\uventry{intenc$'$}
se proposer de | intend | beabsichtigen | намѣреваться | zamierzać.

\uventry{inter}
entre, parmi | between, among | zwischen | между | między.

\uventry{interes$'$}
intéresser | interest | interessiren | интересовать | interesować.

\uventry{interjekci$'$}
interjection | interjection | Interjection | междометіе | wykrzyknik.

\uventry{intern$'$}
intérieur, dedans | inner | innerhalb, im Innern | внутри | wewnątrz.

\uventry{interpunkci$'$}
ponctuation | punctuation | Interpunctionszeichen | знакъ препинанія | znaki pisarskie.

\uventry{intest$'$}
intestin | intestine | Darm | кишка | kiszka.

\uventry{intim$'$}
intime | intimate | intim | интимный | serdeczny, zaźyły.

\uventry{intrig$'$}
intriguer | intrigue | Ränkeschmieden | интриговать | intrygować.

\uventry{invit$'$}
inviter | invite | einladen | приглашать | zapraszać.

\uventry{intermit$'$}
omettre, interrompre | intermit | intermittiren | перемежаться | przerywać się, folgować.

\uventry{io}
quelque chose | anything | etwas | что-нибудь | coś.

\uventry{iom}
un peu, quelque peu de | any quantity | ein wenig | сколько-нибудь | ilekolwiek.

\uventry{ir$'$}
aller | go | gehen | идти | iść.

\uvsubentry{}\uventry{ir$'$il$'$}
échasse | stilt, scatch | Stelze | ходули | szczudło.

\uventry{is}
marque le passé; ex. \uventry{far$'$} faire ― \uventry{mi far$'$is} je faisais,
j’ai fait etc. | ending of past tense in verbs; e. g. \uventry{am$'$is} loved | bezeichnet die vergangene Zeit | означаетъ прошедшее время | oznaczca
czas przeszły.

\uventry{ist$'$}
marque la profession; ex. \uventry{bot$'$} botte ― \uventry{bot$'$ist$'$} bottier;
\uventry{mar$'$} mer ― \uventry{mar$'$ist$'$} marin | person occupied with; e. g. \uventry{mar$'$} sea
― \uventry{mar$'$ist$'$} sailor | sich mit etwas beschäftigend; z. B. \uventry{bot$'$}
Stiefel ― \uventry{bot$'$ist$'$} Schuster; \uventry{mar$'$} Meer ― \uventry{mar$'$ist$'$} Seeman | занимающійся; напр. \uventry{bot$'$} сапогъ ― \uventry{bot$'$ist$'$} сапожникъ; \uventry{mar$'$}
море ― \uventry{mar$'$ist$'$} морякъ | zajmujący się; np. \uventry{bot$'$} but ―
\uventry{bot$'$ist$'$} szewc; \uventry{mar$'$} morze ― \uventry{mar$'$ist$'$} marynarz.

\uventry{it$'$}
marque le participe passé passif; ex. \uventry{far$'$} faire ― \uventry{far$'$it$'$}
fait (qu’on a fait), ayant été fait | ending of past part. pass. in
verbs; e. g. \uventry{am$'$it$'$} having been loved | bezeichnet das
Particip. perfecti passivi | означаетъ причастіе прошедшаго времени
страдательнаго залога | oznacza imiesłów bierny czasu przeszłego.

\uventry{iu}
quelqu’un | any one | jemand | кто-нибудь | ktoś.

\uventry{izol$'$}
isoler | isolate | isoliren | уединять | odosabniać.

\uvlitero{J, Ĵ}

\uventry{j}
marque le pluriel; ex. \uventry{hom$'$o} homme ― \uventry{hom$'$o$'$j} hommes | sign
of the plural; e. g. \uventry{patr$'$o} father ― \uventry{patr$'$o$'$j} fathers | bezeichnet den Plural | означаетъ множественное число | oznacza liczbę
mnogą.

\uventry{ja}
en effet, de fait, donc, n’est-ce pas | indeed | ja, doch | вѣдь | wszakźe.

\uventry{jak$'$}
veste | jacket | Jacke | куртка | kurtka, kaftanik.

\uventry{ĵaluz$'$}
jaloux | jealous | eifersüchtig | ревнивый | zazdrosny.

\uventry{jam}
déjà | already | schon | уже | juź.

\uventry{Januar$'$}
Janvier | January | Januar | Январь | Styczeń.

\uventry{jar$'$}
année | year | Jahr | годъ | rok.

\uventry{jasmen$'$}
jasmin | jasmine | Jasmin | жасминъ | jaźmin.

\uventry{ĵaŭd$'$}
jeudi | Thursday | Donnerstag | четвергъ | czwartek.

\uventry{je}
se traduit par différentes prépositions; sa signification est
toujours aisément suggérée par le sens de la phrase | can be rendered
by various English prepositions | kann durch verschiedene
Präpositionen übersetzt werden | можетъ быть переведено различными
предлогами | moźe być przetłomaczone za pomocą róźnych przyimków.

\uventry{jen}
voilà, voici | behold, lo | da! siehe! | вотъ | otóź.

\uvsubentry{}\uventry{jen ― jen}
tantôt ― tantôt | sometimes ― sometimes | bald ― bald | то― то | to ― to.

\uventry{jes}
oui | yes | ja | да | tak.

\uvsubentry{}\uventry{jes$'$ig$'$}
confirmer | confirm | bestätigen | подтверждать | potwierdzać.

\uventry{ĵet$'$}
jeter | throw | werfen | бросать | rzucać.

\uventry{ĵongl$'$}
bouffonner | juggle | gaukeln | фокусничать | kuglować.

\uventry{ju ― des}
plus ― plus | the ― the | je ― desto | чѣмъ ―
тѣмъ | im ― tem.

\uventry{jug$'$}
joug | yoke | Joch | иго | jarzmo.

\uventry{juĝ$'$}
juger | judge | richten, urtheilen | судить | sądzić.

\uventry{jugland$'$}
noix | walnut | Wallnuss | грецкій орѣхъ | orzech włoski.

\uventry{juk$'$}
démanger | itch | jucken | зудѣть | swędzić.

\uventry{Juli$'$}
Juillet | July | Juli | Іюль | Lipiec.

\uventry{jun$'$}
jeune | young | jung | молодой | młody.

\uventry{jung$'$}
atteler | couple, harness (vb.) | spannen (z. B. Pferde) | запрягать | zaprzęgać.

\uventry{Juni$'$}
Juin | June | Juni | Іюнь | Czerwiec.

\uventry{juniper$'$}
genevrier, genièvre | juniper | Wachholder | можжевельникъ | jałowiec.

\uventry{jup$'$}
jupe, jupon | petticoat | Frauenrock, Unterrock | юбка | spódnica.

\uventry{ĵur$'$}
jurer | swear | schwören | клясться, божится | przysięgać.

\uventry{ĵus}
justement, à l’instant | just, exactly | soeben | только что | wlaśnie, tylko co.

\uventry{just$'$}
juste | just, righteous | gerecht | справедливый | sprawiedliwy.

\uventry{juvel$'$}
bijou | jewel | Edelstein | драгоцѣнный камень | kamień drogi.

\uvlitero{K}

\uventry{kaĉ$'$}
gâchis | pap | Brei | каша | kasza.

\uventry{kadr$'$}
cadre | frame | Rahmen | рама | rama.

\uventry{kaduk$'$}
caduc, périssable | falling, perishable | hinfällig | дряхлый | wątły, zgrzybiały.

\uventry{kaf$'$}
café | coffee | Kaffee | кофе | kawa.

\uventry{kaĝ$'$}
cage | cage | Käfig | клѣтка | klatka.

\uventry{kahel$'$}
carreau | earthen pane | Kachel | кафля | kafel, kafla.

\uventry{kaj}
et | and | und | и | i, a.

\uventry{kajer$'$}
cahier | paper covered book, copy book | Heft | тетрадь | kajet.

\uventry{kajut$'$}
cajute | cabin | Kajüte | каюта | kajuta.

\uventry{kal$'$}
cor (aux pieds) | corn (on the foot) | Hühnerauge | мозоль | nagniotek, odcisk.

\uventry{kaldron$'$}
chaudron | kettle | Kessel | котелъ | kocioł.

\uventry{kaleŝ$'$}
carosse calèche | carriage | Wagen | коляска | powóz.

\uventry{kalfatr$'$}
calfater | calk | kalfatern | конопатить | zalepiać, konopaczyć.

\uventry{kalik$'$}
coupe, calice | bowl | Kelch | чаша | kielich.

\uventry{kalk$'$}
chaux | lime | Kalk | известь | wapno.

\uventry{kalikot$'$}
calicot | calico | Calico | коленкоръ | perkal.

\uventry{kalkan$'$}
talon (du pied) | heel | Ferse | пятка | pięta.

\uventry{kalkul$'$}
compter | calculate | rechnen | считать | rachować, liczyć.

\uventry{kalson$'$}
caleçons | drawers | Unterhosen | подштанники | gacie,
kalesony.

\uventry{kalumni$'$}
calomnier | calumniate | verläumden | клеветать | obgadywać, potwarzać.

\uventry{kambi$'$}
lettre de change | exchange | Wechsel (Kaufm.) | вексель | weksel.

\uventry{kamel$'$}
chameau | camel | Kameel | верблюдъ | wielbłąd.

\uventry{kamen$'$}
cheminée | fire-place | Kamin | каминъ | kominek.

\uvsubentry{}\uventry{kamen$'$tub$'$}
cheminée | chimney | Schornstein | дымовая труба | komin.

\uventry{kamer$'$}
chambre | chamber | Kammer | камера | komora, komórka.

\uventry{kamfor$'$}
camphre | camphire, camphor | Kampher | камфора | kamfora.

\uventry{kamizol$'$}
camisole | waistcoat | Brustwamms | фуфайка | kaftanik.

\uventry{kamlot$'$}
camelot | camlet | Camelot | камлотъ | kamlot.

\uventry{kamomil$'$}
camomille | camomile | Kamille | ромашка | rumianek.

\uventry{kamp$'$}
champ, campagne | field | Feld | поле | pole.

\uventry{kan$'$}
roseau, canne | cane | Rohr | трость | trzcina.

\uventry{kanab$'$}
chanvre | hemp | Hanf | конопля | pieńka, konopie.

\uventry{kanajl$'$}
canaille | mob, canaille | Canaille | каналья | łajdak.

\uventry{kanap$'$}
canapé | sofa, lounge | Kanapee | диванъ | kanapa.

\uventry{kanari$'$}
canari, serin | canary | Kanarienvogel | канарейка | kanarek.

\uventry{kancelari$'$}
chancellerie | chancery | Kanzlei | канцелярія | kancelarya.

\uventry{kancelier$'$}
chancelier | chancellor | Kanzler | канцлеръ | kanclerz.

\uventry{kand$'$}
candi (sucre) | sugar-candy | Candelzucker | леденецъ | cukier
lodowaty.

\uventry{kandel$'$}
chandelle | candle | Licht, Kerze | свѣча | świeca.

\uventry{kankr$'$}
écrevisse | crab | Krebs | ракъ | rak.

\uventry{kant$'$}
chanter | sing | singen | пѣть | śpiewać.

\uventry{kantarid$'$}
cantharide | cantharide | spanische Fliege | шпанская муха | mucha hiszpańska.

\uventry{kantor$'$}
chantre | chanter | Cantor | канторъ | kantor, śpiewak kościelny.

\uventry{kanvas$'$}
canevas | canvass, draught | Canevas | канва | kanwa.

\uventry{kap$'$}
tête | head | Kopf | голова | głowa.

\uventry{kapabl$'$}
capable, apte | capable | fähig | способный | zdolny.

\uventry{kapel$'$}
chapelle | chapel | Kapelle | капелла | kapela.

\uventry{kapitan$'$}
capitaine | captain | Hauptmann | капитанъ | kapitan.

\uventry{kapitel$'$}
chapiteau | chapiter | Säulenknauf | капитель | słupogłów.

\uventry{kapitulac$'$}
capituler | capitulate | capituliren | капитулировать | kapitulować.

\uventry{kapon$'$}
chapon | capon | Kapaun | каплунъ | kapłón.

\uventry{kapot$'$}
capote | great coat with a cape | Capot | капотъ | kapota.

\uventry{kapor$'$}
câpre | caper | Kaper | каперсъ | kaparki.

\uventry{kapr$'$}
bouc | goat | Bock | козелъ | kozioł.

\uventry{kapreol$'$}
chevreuil | roe, roe-buck | Reh | козуля | sarna.

\uventry{kapric$'$}
caprice | caprice, whim | Caprice, Laune | капризъ | kaprys.

\uventry{kapsul$'$}
capsule | capsule | Kapsel | капсуля | kapsułka.

\uventry{kapt$'$}
attraper | catch | fangen | ловить | chwytać.

\uvsubentry{}\uventry{kapt$'$il$'$}
piège | trap, pit-fall | Schlinge, Falle | силокъ | sidło.

\uventry{kapucen$'$}
capucin | capuchin friar | Kapuziner | капуцинъ | kapcyn.

\uventry{kapuĉ$'$}
capuce, capuchon | capuchin, cowl | Capuchon | капишонъ | kaptur.

\uventry{kar$'$}
cher | dear | theuer | дорогой | drogi.

\uventry{karaben$'$}
carabine | carabine | Karabiner | карабинъ | karabin.

\uventry{karaf$'$}
carafe | caraffe, decanter | Caraffe | графинъ | karafka.

\uventry{karakter$'$}
caractère | character | Charakter | характеръ | charakter.

\uventry{karas$'$}
carassin (poisson) | crucian | Karausche | карась | karaś.

\uventry{karb$'$}
charbon | coal | Kohle | уголь | węgiel.

\uventry{kard$'$}
chardon | thistle | Distel | чертополохъ | oset.

\uventry{kardel$'$}
chardonneret | thistle-finch | Stieglitz, Distelfink | щеголъ, щегленокъ | szczygieł.

\uventry{kares$'$}
caresser | caress | liebkosen | ласкать | piescić.

\uventry{kariofil$'$}
girofle (clou de) | clove | Nelke | гвоздика | goździk.

\uventry{karmin$'$}
carmin | carmine | Carmin | карминъ | karmin.

\uventry{karnaval$'$}
carnaval | carnaval | Fasching | карнавалъ, масляница | karnawał.

\uventry{karo$'$}
carreau (cartes) | diamond | Carreau (in Karten) | бубны | karo.

\uventry{karob$'$}
caroube | carob bean | Johannisbrod | сладкій рожокъ | chleb św. Jana.

\uventry{karot$'$}
carotte | carrot | Möhre, Mohrrübe | морковь | marchew.

\uventry{karp$'$}
carpe (poisson) | carp (fish) | Karpfen | карпъ | karp.

\uventry{karpen$'$}
charme | yoke-elm | Hagebuche, Hornbaum | грабъ, грабина | grab.

\uventry{kart$'$}
carte | card | Karte | карта | karta; mapa.

\uventry{kartav$'$}
grasseyer | speak thick | schnarren (beim Sprechen) | картавить | nieczysto wymawiać.

\uventry{kartilag$'$}
cartilage | cartilage | Knorpel | хрящъ | chrząstka.

\uventry{kartoĉ$'$}
cartouche | cartouch | Kartätsche | картечь | kartacz.

\uventry{karton$'$}
carton | pap | Pappe | картонъ, папка | tektura.

\uventry{karusel$'$}
carrousel | carousal | Carroussel | карусель | karuzela.

\uventry{kas$'$}
caisse | chest, money-box | Kasse | касса | kassa.

\uventry{kaŝ$'$}
cacher | hide (vb.) | verbergen, verhehlen | прятать | chować.

\uventry{kaserol$'$}
casserole | stewpan | Casserolle | кострюля | rondel.

\uventry{kask$'$}
casque | helmet | Helm | шлемъ | hełm, szyszak.

\uventry{kaŝtan$'$}
châtaigne | chestnut | Kastanie | каштанъ | kasztan.

\uventry{kastel$'$}
château | castle | Schloss, Kastell | замокъ | zamek.

\uventry{kastor$'$}
castor | beaver | Biber | бобръ | bóbr.

\uventry{kastr$'$}
châtrer | cut, curtail | castriren | кастрировать | rzezać, wałaszyć.

\uventry{kat$'$}
chat | cat | Katze | котъ | kot.

\uventry{kataplasm$'$}
cataplasme | poultice | Kataplasma | припарка | kataplazm.

\uventry{katar$'$}
catarrhe | catarrh | Schnupfen, Katarrh | насморкъ, катаръ | katar.

\uventry{katarakt$'$}
cataracte (yeux) | cataract | Staar (Augenkrankheit) | катаракта | katarakta.

\uventry{katen$'$}
chaîne | fetter | Fessel | оковы, кандалы | kajdany.

\uventry{katun$'$}
toile de coton | cotton, calico | Kattun | ситецъ | kreton.

\uventry{kaŭteriz$'$}
cautériser | cauterise | ätzen | прижигать | wypalać.

\uventry{kaŭz$'$}
causer | cause | verursachen | причинять | powodować, sprawiać.

\uvsubentry{}\uventry{kaŭz$'$o}
cause | cause | Ursache | причина | przyczyna.

\uventry{kav$'$}
fosse, creux | cave | Grube | яма | dół, loch.

\uventry{kavalir$'$}
chevalier | cavalier, knight | Ritter | рыцарь | rycerz.

\uventry{kavern$'$}
caverne | cavern | Höhle | пещера | jaskinia, pieczara.

\uventry{kaviar$'$}
caviar | caviare | Kaviar | икра | kawior.

\uventry{kaz$'$}
cas | case | Kasus | падежъ | przypadek.

\uventry{kaze$'$}
fromage à la pie | whey-cheese | Quark | творогъ | twaróg.

\uventry{ke}
que | that (conj.) | dass, damit | что, чтобы | że, żeby.

\uventry{kegl$'$}
quille | keel | Kegel | кегель, кегля | kręgiel.

\uventry{kel$'$}
cave (la) | cellar | Keller | погребъ | piwnica.

\uventry{kelk$'$}
quelque | some | mancher | нѣкоторый | niektóry.

\uventry{kelner$'$}
garçon | boy | Kellner | половой, кельнеръ | kelner.

\uventry{ken$'$}
bois résineux | resinous wood | Kienholz | лучина | łuczywo.

\uventry{ker$'$}
cœur | heart | Herz | черви (въ картахъ) | czerwień.

\uventry{kern$'$}
noyau | kernel | Kern | ядро | jądro.

\uventry{kerub$'$}
chérubin | cherub | Cherub | херувимъ | cherubin.

\uventry{kest$'$}
caisse, coffre | chest, box | Kiste, Kasten, Lade | ящикъ | skrzynia.

\uvsubentry{}\uventry{tir$'$kest$'$}
tiroir | drawer | Schublade | выдвижной ящикъ | szuflada.

\uventry{kia}
quel | of what kind | was für ein, welcher | какой | jaki.

\uventry{kial}
pourquoi | why, wherefore | warum | почему | dlaczego.

\uventry{kiam}
quand, lorsque | when | wann | когда | kiedy.

\uventry{kie}
où | where | wo | гдѣ | gdzie.

\uventry{kiel}
comment | how | wie | какъ | jak.

\uventry{kies}
à qui? dont, duquel | whose | wessen | чей | czyj.

\uventry{kil$'$}
quille | keel, careen | Kiel (eines Schiffes) | киль | tram,
stępka.

\uventry{kio}
quoi | what | was | что | co.

\uventry{kiom}
combien | how much | wie viel | сколько | ile.

\uventry{kiras$'$}
cuirasse | cuirass | Kürass | панцырь | kirys.

\uventry{kis$'$}
baiser, embrasser | kiss | küssen | цѣловать | całować.

\uventry{kitel$'$}
souquenille | frock | Kittel | балахонъ | kieca.

\uventry{kiu}
qui lequel, laquelle | who, which | wer, welcher | кто, который | kto, który.

\uventry{klaft$'$}
toise (russe) | fathom (measure) | Faden, Klafter | сажень | sążeń.

\uventry{klap$'$}
clapet | flap | Klappe | клапанъ | klapka, zasuwka, zastawka.

\uventry{klar$'$}
clair (qui n’est pas trouble) | clear | klar | ясный | jasny.

\uventry{klarnet$'$}
clarinette | clarinet | Clarinette | кларнетъ | klarnet.

\uventry{klas$'$}
classe | class | Classe | классъ | klassa.

\uventry{klav$'$}
touche | cliff | Klaviertaste | клавишъ | klawisz.

\uventry{kler$'$}
bien élevé | educated | gebildet | образованный | wykształcony.

\uventry{klimat$'$}
climat | climate, clime | Klima | климатъ | klimat.

\uventry{klin$'$}
incliner, pencher | bend, incline | neigen | наклонять | chylić.

\uventry{klister$'$}
clystère | clyster | Klystier | клистиръ | enema, lewatywa.

\uventry{klopod$'$}
se donner de la peine | endeavour | sich Mühe geben | хлопотать | kłopotać się.

\uventry{kloŝ$'$}
cloche | bell | Kappe, Glocke (z. B. über einer Uhr) | колпакъ
(напр. лампы и т. п.) | klosz.

\uventry{klub$'$}
société | club | Club | клубъ | klub.

\uventry{kluz$'$}
écluse | sluice | Schleuse | шлюзъ | śluza.

\uventry{knab$'$}
garçon | boy | Knabe | мальчикъ | chłopiec.

\uventry{kned$'$}
pétrir | knead | kneten | мѣсить | ugniatać, mięsić.

\uventry{koaks$'$}
coke | coak, coke | Koaks | коксъ | koks.

\uventry{kobalt$'$}
cobalt | cobalt | Kobalt | кобальтъ | kobalt.

\uventry{kobold$'$}
farfadet | gnome | Kobold | домовой (духъ) | poczwara, dyabełek.

\uventry{koĉenil$'$}
cochenille | cochineal | Cochenille | кошениль | koszenila.

\uventry{kojn$'$}
coin (instrument) | wedge | Keil | клинъ | klin.

\uventry{kok$'$}
coq | rooster | Hahn | пѣтухъ | kogut.

\uventry{kokcinel$'$}
coccinelle | rose-chafer | Marienkäfer | козявка, Божья
коровка | biedrunka.

\uventry{koket$'$}
coquet | coquet | coquett | кокетливый | zalotny.

\uventry{kokluŝ$'$}
coqueluche | chin-cough, hooping-cough | Keuchhusten | коклюшъ | koklusz.

\uventry{kokos$'$}
coco | cocoa | Kokos | кокосъ | orzech kokosowy.

\uventry{koks$'$}
hanche | hip, hanch | Hüfte | тазобедренное сочлененіе | lędźwie.

\uventry{kol$'$}
cou | neck | Hals | шея | szyja.

\uvsubentry{}\uventry{kol$'$um$'$}
faux-col | collar | Kragen | воротникъ | kołnierz.


\uvsubentry{}\uventry{kol$'$har$'$o$'$j}
crinière | mane | Mähne | грива | grzywa.

\uventry{kolbas$'$}
andouille, boudin | sausage, saucisson | Wurst | колбаса | kiełbasa.

\uventry{koleg$'$}
camarade, collègue | colleague | Kamerad | товарищъ | towarzysz, kolega.

\uventry{kolekt$'$}
amasser, collectionner | collect | sammeln | собирать | zbierać.

\uventry{koler$'$}
se fâcher | mad be angry | zürnen | сердиться | gniewać się.

\uventry{kolibr$'$}
colibri | colibri | Kolibri | колибри | koliber.

\uventry{kolimb$'$}
plongeon (oiseau) | plungeon, diver | Taucher (Vogel) | гагара | nur.

\uventry{kolofon$'$}
colophane | rosin | Colophonium | канифоль | kolofonia.

\uventry{kolomb$'$}
pigeon, colombe | dove | Taube | голубь | gołąb$'$.

\uventry{kolon$'$}
colonne | column | Säule | столбъ | słup.

\uventry{kolor$'$}
couleur | color | Farbe | цвѣтъ, краска | kolor.

\uventry{kolport$'$}
colporter | hawk | hausiren | разносить (товары) | kolportować, roznosić.

\uventry{kolubr$'$}
couleuvre | adder, snake | Hausschlange | ужъ | wąż.

\uventry{kom$'$}
virgule | comma | Komma | запятая | przecinek.

\uventry{komand$'$}
commander | command | commandiren | командовать | komendować.

\uventry{komb$'$}
peigner | comb | kämmen | чесать | czesać.

\uventry{kombin$'$}
combiner | combine | combiniren | комбинировать | kombinować.

\uventry{komenc$'$}
commencer | commence | anfangen | начинать | zaczynać.

\uventry{komentari$'$}
commenter | comment | erläutern, commentiren | комментировать | komentować.

\uventry{komerc$'$}
commercer | trade | handeln, Handel treiben | торговать | handlować.

\uvsubentry{}\uventry{komerc$'$aĵ$'$}
marchandise | ware, merchandise | Waare | товаръ | towar.

\uventry{komfort$'$}
aise, agrément | comfort | Komfort | комфортъ | komfort,
przepych.

\uventry{komisi$'$}
commissionner, charger (quelqu’un de...) | commission | auftragen, beauftragen | поручать | zlecać.

\uventry{komitat$'$}
comité | committee | Comité, Ausschuss | комитетъ | komitet.

\uventry{komiz$'$}
commis (un) | clerk | Commis | прикащикъ | subjekt.

\uventry{komod$'$}
commode (meuble) | chest of drawers | Commode | коммодъ | komoda.

\uventry{kompar$'$}
comparer | compare | vergleichen | сравнивать | porównać.

\uventry{kompat$'$}
avoir compassion | compassionate | Mitleid haben | сострадать | współczuwać.

\uventry{komplez$'$}
complaisance | favor, liking | Gefallen | услуга, угожденіе | przysługa, usługa, dogadzanie.

\uventry{kompost$'$}
composer (typogr.) | set (type) | setzen (Buchdruck.) | набирать (въ типографіи) | składać (w druku).

\uventry{kompren$'$}
comprendre | understand | verstehen | понимать | rozumieć.

\uventry{kompres$'$}
compresse | compress | Compresse | компрессъ | kompres, okład.

\uventry{komun$'$}
commun | common | gemeinsam | общій | ogólny, wspólny.

\uvsubentry{}\uventry{komun$'$um$'$}
commune, paroisse | community, parish | Gemeinde | община | gmina.

\uventry{komuni$'$}
donner le Saint Sacrement | administer the sacrament | das heilige Abendmahl reichen | пріобщать (Св. Таинъ) | udzielać Komunję.

\uventry{komunik$'$}
communiquer | communicate | communiciren, verbinden | сообщать | komunikować.

\uventry{kon$'$}
connaître | know (by experience or study), recognise | kennen | знать (быть знакомымъ) | znać.

\uventry{koncern$'$}
concerner | concern | betreffen, angehen | касаться,
относиться | tyczyć się.

\uventry{kondamn$'$}
condamner | condemn | verurtheilen | осуждать | osądzać.

\uventry{kondiĉ$'$}
condition | condition | Bedingung | условіе | warunek.

\uventry{kondolenc$'$}
condouloir | condole | Beileid bezeigen | соболѣзновать | pocieszać, kondolować, okazać współczucie.

\uventry{konduk$'$}
conduire | conduct | führen | вести | prowadzić.

\uvsubentry{}\uventry{konduk$'$il$'$}
rêne | rein | Zügel | поводъ (у лошади) | leice.

\uventry{kondut$'$}
se conduire (bien ou mal) | conduct | sich aufführen, sich
benehmen | вести себя | prowadzić się, sprawować się.

\uventry{konfes$'$}
avouer | confess | bekennen, gestehen | признавать,
исповѣдывать | przyznawać.

\uventry{konfid$'$}
se fier, se confier | confide, trust | trauen, vertrauen | довѣрять | dowierzać.

\uventry{konfirm$'$}
confirmer, ratifier | confirm, ratify | bestätigen | утверждать | utrzymywać.

\uventry{konfit$'$}
confire | preserve with sugar | einmachen (mit Zucker) | варить въ сахарѣ | smażyć w cukrze.

\uventry{konform$'$}
conformément | conformably | gemäss, entsprechend | сообразный | odpowiedni.

\uventry{konfuz$'$}
confondre, embrouiller | confuse | verwirren | смущать | konfundować, zmięszać.

\uventry{konjekt$'$}
conjecturer | conject, conjecture | vermuthen | догадываться | domyślać się, przypuszczać.

\uventry{konjugaci$'$}
conjuguer | conjugate | conjugiren | спрягать | konjugować.

\uventry{konjunkci$'$}
conjonction | conjunction | Bindewort | союзъ | spójnik.

\uventry{konk$'$}
coquille, coquillage | shell | Muschel | раковина | muszla.

\uventry{konklud$'$}
conclure | conclude | schliessen, folgern | выводить
заключеніе | wnioskować.

\uventry{konkur$'$}
rivaliser | rival | concurriren | конкурировать | konkurować.

\uventry{konkurs$'$}
concours | concourse | Concurs | конкурсъ | konkurs.

\uventry{konsci$'$}
avoir conscience | be conscious of | sich bewusst sein | сознавать | mieć świadomość.

\uventry{konscienc$'$}
consciénce | conscience | Gewissen | совѣсть | sumienie.

\uventry{konsekvenc$'$}
conséquent | consequent | consequent | послѣдовательный | konsekwentny.

\uventry{konsent$'$}
consentir | consent | übereinstimmen, einwilligen | соглашаться | zgadzać się.

\uventry{konserv$'$}
conserver | preserve | aufbewahren | сохранять | przechowywać, zachowywać.

\uventry{konsider$'$}
considérer | consider | betrachten, überlegen, erwägen | соображать | zastanawiać się, rozmyślać.

\uventry{konsil$'$}
conseiller | advise, counsel | rathen | совѣтовать | radzić.

\uventry{konsist$'$}
consister | consist | bestehen (aus...) | состоять | składać się.

\uventry{konsol$'$}
consoler | console | trösten | утѣшать | pocieszać.

\uventry{konsonant$'$}
consonne | consonant | Konsonant | согласная | spółgłoska.

\uventry{konspir$'$}
conspirer | conspire | sich verschwören | дѣлать заговоръ | konspirować.

\uventry{konstant$'$}
constant | constant | beständig | постоянный | stały, ustawiczny.

\uventry{konstat$'$}
constater | prove, verify | konstatiren | подтверждать | konstatować.

\uventry{konstern$'$}
consterner | astonish | bestürzen | озадачивать, смущать | ambarasować.

\uventry{konstru$'$}
construire | construct, build | bauen | строить | budować.

\uventry{konsum$'$}
consumer | consume | zehren, abzehren | истощать, потреблять | konsumować.

\uventry{kontant$'$}
comptant | paid in cash | baar | наличный | w gotówce.

\uventry{kontent$'$}
content | content | zufrieden | довольный | zadowolony.

\uventry{kontor$'$}
comptoir | office, counting-house | Komptor | контора | kantor (biuro).

\uventry{kontrakt$'$}
contracter | contract | einen Vertrag abschliessen | заключать договоръ | zawierać umowę.

\uventry{kontraŭ}
contre | against | gegen | противъ | przeciw.

\uvsubentry{}\uventry{kontraŭ$'$e}
vis-à-vis; au contraire | over against; on the contrary | gegenüber; im Gegentheil | напротивъ; насупротивъ | przeciwnie; naprzeciw.

\uventry{kontur$'$}
contour | outline, contour | Umriss | контуръ | kontur.

\uventry{kontuz$'$}
broyer | contuse | wund stossen, quetschen | контузить,
ушибить | gnieść, tłuc.

\uventry{konval$'$}
muguet | may flower | Maiblümchen | ландышъ | konwalia.

\uventry{konven$'$}
convenir | convenient | sich geziemen | приличествовать | wypadać, przystawać.

\uventry{konvert$'$}
convertir | convert | bekehren | обращать (напр. въ
христіанство) | nawrócić.

\uventry{konvink$'$}
convaincre | convince | überzeugen | убѣждать | przekonać.

\uventry{konvulsi$'$}
convulsion | convulsion | Krampf | судороги | konwulsye.

\uventry{konus$'$}
cône | cone | Konus | конусъ | ostrokrąg.

\uventry{kopi$'$}
copier | copy | copiren | копировать | kopjować.

\uventry{kor$'$}
cœur | heart | Herz | сердце | serce.

\uventry{koran$'$}
Coran | Coran | Koran | Коранъ | Koran.

\uventry{korb$'$}
panier corbeille | basket | Korb | корзина | kosz.

\uventry{kord$'$}
corde | string (piano etc.) | Saite | струна | struna.

\uventry{korekt$'$}
corriger | correct | bessern, corrigiren | исправлять | poprawiać.

\uventry{korespond$'$}
correspondre | correspond | correspondiren | переписываться | korespondować.

\uventry{kork$'$}
bouchon | cork | Kork | пробка | korek.

\uventry{korn$'$}
corne | horn | Horn | рогъ | róg.

\uventry{kornic$'$}
corniche | mantlepiece, shelf | Gesims | карнизъ | gzems.

\uventry{korp$'$}
corps | body | Körper | тѣло | cialo.

\uventry{korporaci$'$}
corporation | corporation | Zunft, Körperschaft | цехъ, корпорація | korporacya, cech.

\uventry{korpus$'$}
corps | body of an army | Corps | корпусъ | korpus.

\uventry{kort$'$}
cour | court | Hof | дворъ | podwórze.

\uvsubentry{}\uventry{kort$'$eg$'$}
cour (d’un souverain) | court | Hof (Königlicher) | Дворъ (царскій) | dwór, pałac.

\uventry{korv$'$}
corbeau | raven | Rabe | воронъ | kruk.

\uventry{kost$'$}
coûter | cost, price | kosten | стоить | kosztować.

\uvsubentry{}\uventry{kost$'$o}
prix | price | Preis | цѣна | koszt, cena.

\uventry{kostum$'$}
costume | costume | Costüm | костюмъ | kostjum.

\uventry{kot$'$}
boue | dirt | Koth, Schmutz | грязь | błoto.

\uventry{koton$'$}
coton | cotton | Baumwolle | хлопчатая бумага | bawełna.

\uventry{koturn$'$}
caille | quail (bird) | Wachtel | перепелъ | przepiórka.

\uventry{kov$'$}
couver | broed, covey | brüten | высиживать птенцовъ | wylęgać.

\uventry{kovert$'$}
enveloppe (à lettres) | envelope | Briefcouvert | конвертъ | koperta.

\uventry{kovr$'$}
couvrir | cover | verdecken, verhüllen | закрывать | zakrywać.

\uvsubentry{}\uventry{kovr$'$il$'$ (de libr$'$o)}
enveloppe | envelop | Umschlag, Hülle | обертка | okładka.


\uvsubentry{}\uventry{kovr$'$il$'$ (de fenestr$'$o)}
contrevent | window shutter | Fensterladen | ставень | okiennica.


\uvsubentry{}\uventry{mal$'$kovr$'$}
découvrir | detect, discover | entdecken | открывать | odkrywać.

\uventry{kraĉ$'$}
cracher | spit | speien | плевать | pluć.

\uventry{krad$'$}
grille | grate, crossbars | Gitter | рѣшетка | krata.

\uventry{krajon$'$}
crayon | pencil | Bleistift | карандашъ | ołówek.

\uventry{krak$'$}
craquer | crack | krachen, knallen, knarren | трещать | trzeszczeć.

\uventry{kraken$'$}
craquelin | cracknel | Bretzel | крендель | ciastko.

\uventry{kramp$'$}
crampon, parenthèse | clamp, holdfast, staple | Krampe,
Klammer | скоба | klamra.

\uventry{kran$'$}
robinet | tap, spigot | Zapfen | кранъ | kran.

\uventry{krani$'$}
crâne | skull | Schädel | черепъ | czaszka.

\uventry{kratag$'$}
aubépine | hawthorn | Weissdorn | боярышникъ | głóg.

\uventry{krater$'$}
cratère | crater | Krater | жерло, кратеръ | krater.

\uventry{kravat$'$}
cravate | cravat | Halsbinde, Cravate | галстукъ | krawat.

\uventry{kre$'$}
créer | create | schaffen, erschaffen | создавать | stwarzać.

\uventry{kred$'$}
croire | believe | glauben | вѣрить | wierzyć.

\uvsubentry{}\uventry{kred$'$ebl$'$}
vraisemblable, probable | verisimilar, probable | wahrscheinlich | вѣроятный | prawdopodobny.

\uventry{krem$'$}
crème | cream | Schmant, Sahne | сливки | śmietana.

\uventry{kren$'$}
raifort | horse-radish | Meerrettig | хрѣнъ | chrzan.

\uventry{krep$'$}
crêpe | crape | Krepp | крепъ | krepa.

\uventry{krepusk$'$}
crépuscule | twilight | Dämmerung | сумерки | zmierzch.

\uventry{kresk$'$}
croître | grow, increase | wachsen | рости | rosnać.

\uvsubentry{}\uventry{kresk$'$aĵ$'$}
plante | plant | Pflanze | растеніе | roślina.

\uventry{kret$'$}
craie | chalk | Kreide | мѣлъ | kreda.

\uventry{krev$'$}
crever | crash | platzen | лопнуть | pęknąć.

\uventry{kri$'$}
crier | cry | schreien | кричать | krzyczeć.

\uventry{kribr$'$}
cribler, tamiser | sieve | sieben, durchsieben | просѣивать | przecedzić, przesiać.

\uventry{krim$'$}
crime | crime | Verbrechen | преступленіе | występek, przestępstwo.

\uventry{kriminal$'$}
criminel | criminal | criminal | уголовный | karny, kryminalny.

\uventry{kring$'$}
craquelin | cracknel | Kringel | бубликъ | obwarzanek.

\uventry{kripl$'$}
estropié | crippled | Krüppel | уродливый | kaleki.

\uventry{krisp$'$}
fraise | ruff | Krause, Gekröse | брыжи, брыжейка, курчавость | kreza.

\uventry{Krist$'$}
Christ | Christ | Christus | Христосъ | Chrystus.

\uvsubentry{}\uventry{krist$'$an}
chrétien | christian | Christ | христіанинъ | chrześci janin.

\uvsubentry{}\uventry{krist$'$nask$'$}
Noël | Christmas | Weihnachten | Рождество Христово | Boże Narodzenie.

\uventry{kristal$'$}
cristal | crystal | Krystall | кристаллъ | krysztal.

\uventry{kritik$'$}
critiquer | critick, criticise | kritisiren | критиковать | krytykować.

\uventry{kriz$'$}
crise | crisis | Krisis | кризисъ | kryzys.

\uventry{kroĉ$'$}
accrocher | hook to, cling to | anhaken, anklammern | цѣплять | czepiać.

\uventry{krokodil$'$}
crocodile | crocodile | Krokodill | крокодилъ | krokodyl.

\uventry{krom}
hors, hormis, excepté | without, except | ausser | кромѣ | oprócz.

\uventry{kron$'$}
couronne | crown | Krone, Kranz | вѣнецъ, вѣнокъ | wieniec.

\uventry{kronik$'$}
chronique | chronicle | Chronik | хроника | kronika.

\uventry{krop$'$}
jabot, gosier | crop, craw | Kropf | зобъ | wole.

\uventry{kroz$'$}
croiser | cruise | kreuzen (von Kriegsschiffen) | крейсировать | krzyżować.

\uventry{kruc$'$}
croix | cross | Kreuz | крестъ | krzyż.

\uvsubentry{}\uventry{kruc$'$um$'$}
crucifier | crucify | kreuzigen | распинать на крестѣ | ukrzyżować.

\uventry{kruĉ$'$}
cruche | jug | Krug | кувшинъ | dzban.

\uventry{krud$'$}
cru, rudé | raw, crude, rough | roh, rauh | сырой, суровый | surowy.

\uventry{kruel$'$}
cruel | cruel | grausam | жестокій | okrutny.

\uventry{krup$'$}
croup | croup | Croup | крупъ | krup.

\uventry{krur$'$}
cuisse, jambe | thigh, shank | Unterschenkel | голень | goleń.

\uventry{krust$'$}
croute | crust | Kruste | струпъ | strup.

\uventry{krut$'$}
roide, escarpé | steep | steil | крутой | stromy.

\uventry{kub$'$}
cube | cube | Kubus | кубъ | sześcian.

\uventry{kubut$'$}
coude | elbow | Eltbogen | локоть | łokieć.

\uventry{kudr$'$}
coudre | sew | nähen | шить | szyć.

\uventry{kuf$'$}
coiffe, huppe | cap, tuft | Haube, Kappe | чепецъ | czepek.

\uventry{kugl$'$}
balle (de fusil) | bullet | Kugel (Schiess-) | пуля | kula.

\uventry{kuir$'$}
faire cuire | cook | kochen | варить | gotować.

\uventry{kuk$'$}
gâteau | cookey | Kuchen | пирогъ | pieroźek.

\uventry{kukol$'$}
coucou | cuckoo | Kuckuck | кукушка | kukułka.

\uventry{kukum$'$}
concombre | cucumber | Gurke | огурецъ | ogórek.

\uventry{kukurb$'$}
citrouille, calebasse | gourd, pumpion | Kürbis | тыква | tykwa.

\uventry{kul$'$}
cousin moucheron | gnat | Mücke | комаръ | komar.

\uventry{kuler$'$}
cuillère | spoon | Löffel | ложка | łyżka.

\uventry{kulp$'$}
coupable | fault, blame | schuldig | виноватый | winny.

\uventry{kun}
avec | with | mit | съ | z.

\uvsubentry{}\uventry{kun$'$e}
ensemble, conjointement | together | zusammen | вмѣстѣ | razem, wraz.

\uventry{kunikl$'$}
lapin | rabbit | Kaninchen | кроликъ | królik.

\uventry{kup$'$}
ventouse | cupping-glass | Schröpfkopf | банка (медиц.) | bańka.

\uventry{kupol$'$}
coupole, dôme | cupola, dome | Kuppel | куполъ | kopuła.

\uventry{kupr$'$}
cuivre | copper | Kupfer | мѣдь | miedź.

\uvsubentry{}\uventry{flav$'$a kupr$'$o}
laiton | latten | Messing | латунь | mosiądz.

\uventry{kur$'$}
courir | run | laufen | бѣгать | biegać, lecieć.

\uventry{kurac$'$}
traiter (une maladie) | cure, heal | kuriren, heilen | лѣчить | leczyć.

\uvsubentry{}\uventry{kurac$'$il$'$}
médecine | medicine | Arznei | лѣкарство | lekarstwo.

\uventry{kuraĝ$'$}
courageux, hardi | courage | kühn, dreist | смѣлый | smialy.

\uventry{kurator$'$}
curateur | curator | Curator | попечитель | kurator, opiekun.

\uventry{kurb$'$}
courbe, tortueux | curve | krumm | кривой | krzywy.

\uventry{kurier$'$}
courrier | courier | Courier | курьеръ, вѣстникъ | kuryer.

\uventry{kurten$'$}
rideau | curtain | Vorhang | занавѣсъ | kurtyna, firanka.

\uvsubentry{}\uventry{flank$'$kurten$'$}
courtine | courtain | Gardine | гардины | firanki.

\uvsubentry{}\uventry{rul$'$kurten$'$}
rouleau | window-shade | Rouleau | шторы | roleta.

\uventry{kusen$'$}
coussin | cushion | Kissen | подушка | poduszka.

\uvsubentry{}\uventry{kusen$'$eg$'$}
lit de plumes | bolster | Pfühl | перина | pierzyna.

\uventry{kuŝ$'$}
être couché | lie (down) | liegen | лежать | leźeć.

\uvsubentry{}\uventry{sub$'$kuŝ$'$}
succomber | succumb | unterliegen | подлегать, подлежать | podlegać.

\uventry{kutim$'$}
s’habituer | custom | sich gewöhnen | привыкать | przyzwyczajać się.

\uventry{kuv$'$}
baignoire, cuve | coop, tub | Wanne | ванна | wanna.

\uventry{kuz$'$}
cousin | cousin | Vetter, Cousin | двоюродный братъ | kuzyn.

\uventry{kvankam}
quoique | although | obgleich | хотя | chociaź.

\uventry{kvant$'$}
quantité | quantity | Quantität | количество | ilość.

\uventry{kvar}
quatre | four | vier | четыре | cztery.

\uventry{kvaranten$'$}
quarantaine | quarantine | Quarantaine | карантинъ | kwarantanna.

\uventry{kvarc$'$}
quartz | quartz | Quarz | кварцъ | kwarc.

\uventry{kvart$'$}
quarte | quart | Quart | кварта | kwarta.

\uventry{kvartal$'$}
quartier | quarter | Quartal | кварталъ | kwartał.

\uventry{kvazaŭ}
comme si | as if | gleichsam, als ob | будто бы | jakoby.

\uventry{kverk$'$}
chêne | oak | Eiche | дубъ | dąb.

\uventry{kviet$'$}
mou, doux, quiet | quiet, calm | sanft | кроткій | łagodny.

\uventry{kvin}
cinq | five | fünf | пять | pięć.

\uventry{kvit$'$}
quitte | quit | quitt | квитъ, въ разсчетѣ | kwit.

\uventry{kvitanc$'$}
quittancer | acquit | quittiren | квитировать, росписаться
въ полученіи | pokwitować.

\uvlitero{L}

\uventry{l’, la}
article défini (le, la, les | the | der, die, das (bestimmter
Arttkel) | членъ опредѣленный (по русски не переводится) | przedimek
określny (nie tłomaczy się).

\uventry{labor$'$}
travailler | labor | arbeiten | работать | pracować.

\uvsubentry{}\uventry{per$'$labor$'$}
gagner par son travail | gain | verdienen, erarbeiten | заработывать | zarabiać.

\uvsubentry{}\uventry{pri$'$labor$'$}
labourer | elaborate, cultivate | bearbeiten | обработывать | obrabiać.

\uventry{lac$'$}
las, fatigué | weary | müde | усталый | zmęczony.

\uventry{lacert$'$}
lézard | lizard | Eidechse | ящерица | jaszczurka.

\uventry{laĉ$'$}
lacs | lace | Schnur | шнуръ | sznur.

\uventry{lad$'$}
tôle, fer-blanc | tinned iron | Blech | жесть | blacha.

\uventry{laf$'$}
lava | lava | Lava | лава | lawa.

\uventry{lag$'$}
lac | lake | See (der) | озеро | jezioro.

\uvsubentry{}\uventry{lag$'$et$'$}
étang | pond | Teich | прудъ | staw.

\uventry{lak$'$}
vernis | varnish | Lack, Firniss | лакъ | pokost.

\uventry{lake$'$}
laquais | lackey | Lackei, Bedienter | лакей | lokaj.

\uventry{laks$'$}
flux du ventre | looseness | Durchfall | поносъ | rozwolnienie.

\uventry{lakt$'$}
lait | milk | Milch | молоко | mleko.

\uvsubentry{}\uventry{lakt$'$um$'$}
laite | soft-roe, milt | Milch der Fische | молоки | mleczko (ryb).

\uventry{lam$'$}
boiteux | lame | lahm | хромой | kulawy.

\uvsubentry{}\uventry{lam$'$baston$'$}
béquille | crutch | Krücke | костыль | kula (do chodzenia).

\uventry{lamp$'$}
lampe | lamp | Lampe | лампа | lampa.

\uventry{lampir$'$}
verluisant | glow-worm | Leuchtkäfer | свѣтлякъ | świetlik.

\uventry{lan$'$}
laine | wool | Wolle | шерсть | wełna.

\uventry{land$'$}
pays | land, country | Land | страна | kraj.

\uventry{lang$'$}
langue (organe) | tongue | Zunge | языкъ (органъ) | język (narząd).

\uventry{lantern$'$}
lanterne | lantern | Laterne | фонарь | latarnia.

\uventry{lanug$'$}
duvet, poils follets | down, fluff | Flaumfeder | пухъ | puch, puszek.

\uventry{lard$'$}
lard | lard | Speck | сало | sadło.

\uventry{larĝ$'$}
large | broad | breit | широкій | szeroki.

\uvsubentry{}\uventry{laŭ$'$larĝ$'$}
à travers | across, in width | quer | поперекъ | poprzek.

\uventry{laring$'$}
larynx | larynx | Kehle | гортань | krtań.

\uventry{larm$'$}
larme | tear (to shed a) | Thräne | слеза | łza.

\uventry{larv$'$}
larve | chrysalis | Larve | личинка | poczwarka, liszka.

\uventry{las$'$}
laisser, abandonner | leave, let alone | lassen | пускать,
оставлять | puszczać, zostawiać.

\uventry{last$'$}
dernier | last, latest | letzt | послѣдній | ostatni.

\uventry{laŭ}
selon, d’après | according to | nach, gemäss | по, согласно | według.

\uventry{laŭb$'$}
tonnelle, berceau | arbor, summerhouse | Laube | бесѣдка | altanka.

\uventry{laŭd$'$}
louer, vanter | praise | loben | хвалить | chwalić.

\uventry{laŭr$'$}
laurier | laurel | Lorbeer | лавръ | wawrzyn, laur.

\uventry{laŭt$'$}
haut (parler) | loud | laut, hörbar | громко | głośno.

\uventry{lav$'$}
laver | wash | waschen | мыть | myć.

\uventry{lavang$'$}
lavanche, avalanche | avalanche | Lawine | лавина | lawina.

\uventry{lecion$'$}
leçon | lesson | Lektion | урокъ | lekcya.

\uventry{led$'$}
cuir, peau (des bêtes) | leather | Leder | кожа | skóra.

\uventry{leg$'$}
lire | read | lesen | читать | czytać.

\uventry{legend$'$}
légende | legend | Sage | легенда | legenda.

\uventry{legi$'$}
légion | legion | Legion | легіонъ | legjon.

\uventry{legom$'$}
légume | legume | Gemüse | овощь | jarzyna, warzywo.

\uventry{leĝ$'$}
loi | law | Gesetz | законъ | prawo.

\uventry{lek$'$}
lécher | lick | lecken | лизать | lizać.

\uventry{lekant$'$}
paquerette | daisy | Gänseblume | маргаритка | złocień pospolity.

\uventry{leksikon$'$}
lexicon | lexicon | Lexicon | лексиконъ | słownik.

\uventry{lent$'$}
lentille | lentil | Linse | чечевица | soczewka.

\uventry{lentug$'$}
lentille, rousseur | freckle | Sommersprosse | веснушка | pieg.

\uventry{leon$'$}
lion | lion | Löwe | левъ | lew.

\uventry{leontod$'$}
dent de lion | dandelion | Hundeblume | одуванчикъ | kaczyniec.

\uventry{leopard$'$}
léopard | leopard | Leopard | леопардъ | lampart.

\uventry{lepor$'$}
lièvre | hare | Hase | заяцъ | zając.

\uventry{lepr$'$}
lèpre | leprosy | Aussatz | проказа | trąd.

\uventry{lern$'$}
apprendre | learn | lernen | учиться | uczyć się.

\uvsubentry{}\uventry{lern$'$ej$'$}
école | school | Schule | школа | szkoła, uczelnia.

\uventry{lert$'$}
adroit, habile, agile | skilful | geschickt, gewandt, geläufig | ловкій | zręczny.

\uventry{lesiv$'$}
lessive | lie, buck | Lauge | щелокъ | ług.

\uventry{leter$'$}
lettre, épître | letter | Brief | письмо | list.

\uventry{leŭtenant$'$}
lieutenant | lieutenant | Lieutenant | поручикъ | porucznik, lejtnant.

\uventry{lev$'$}
lever | lift, raise | aufheben | поднимать | podnosić.

\uvsubentry{}\uventry{lev$'$il$'$}
levier | leaver, lever | Hebel | рычагъ | dźwignia.

\uventry{levkoj$'$}
giroflée | stock-gillyflower | Levkoje | левкой | lewkonja.

\uventry{li}
il, lui | he | er | онъ | on.

\uvsubentry{}\uventry{li$'$a}
son, sa | his | sein | его | jego.

\uventry{lian$'$}
liane, liène | liana | Liane | ліяна | ljana.

\uventry{libel$'$}
demoiselle (ins.) | dragon fly | Libelle | стрекоза | konik
polny.

\uventry{liber$'$}
libre | free | frei | свободный | wolny.

\uventry{libr$'$}
livre | book | Buch | книга | księga; książka.

\uvsubentry{}\uventry{libr$'$o$'$ten$'$ant$'$o}
teneur de livres | bookkeeper | Buchhalter | бухгалтеръ | buchalter.

\uventry{lien$'$}
rate | spleen | Milz | селезенка | śledziona.

\uventry{lig$'$}
lier | bind, tie | binden | связывать | wiązać.

\uventry{lign$'$}
bois | wood (the substance) | Holz | дрова | drzewo, drwa.

\uventry{liken$'$}
dartre, lichen | tetter | Flechte | лишай | liszaj.

\uventry{likvid$'$}
liquider | liquidate | abrechnen, liquidiren | ликвидировать | likwidować.

\uventry{likvor$'$}
liqueur | liquor | Liqueur | ликеръ | likier.

\uventry{lili$'$}
lis | lily | Lilie | лилія | lilja.

\uventry{lim$'$}
limite, borne | limit | Grenze | граница | granica.

\uvsubentry{}\uventry{lim$'$ig$'$}
restreindre | limit | beschränken | ограничивать | ograniczać.

\uventry{limak$'$}
limaçon, escargot | snail | Schnecke | улитка | ślimak.

\uventry{limonad$'$}
limonade | lemonade | Limonade | лимонадъ | lemoniada.

\uventry{lin$'$}
lin | flax | Flachs | ленъ | len.

\uventry{lingv$'$}
langue, langage | language | Sprache | языкъ (рѣчь) | język
(mowa).

\uventry{lini$'$}
ligne | line, file | Linie, Zeile, Reihe | линія, строка | wiersz.

\uventry{link$'$}
lynx, loup-cervier | lynx | Luchs | рысь (животное) | ryś.

\uventry{lip$'$}
lèvre | lip | Lippe | губа | warga.

\uvsubentry{}\uventry{lip$'$har$'$o$'$j}
moustache | mustache | Schnurrbart | усы | wąsy.

\uventry{lir$'$}
lyre | lyre | Leier, Lyra | лира | lira.

\uventry{lit$'$}
lit | bed | Bett | кровать | łóżko.

\uventry{litani$'$}
litanies | litany | Litanei | литанія | litanja.

\uventry{liter$'$}
lettre (de l’alphabet) | letter | Buchstabe | буква | litera.

\uventry{liut$'$}
luth | lute | Laute | лютня | lutnia.

\uventry{liver$'$}
livrer, fournir | deliver, furnish | liefern | доставлять | dostawiać.

\uventry{livre$'$}
livrée | livery | Livrée | ливрея | liberya.

\uventry{lod$'$}
demi-once | half an ounce | Loth | лотъ | łut.

\uventry{log$'$}
attirer, allécher | entice | locken, anlocken | манить | manić, tumanić.

\uvsubentry{}\uventry{de$'$log$'$}
séduire | seduce | verführen | обольщать | zwodzić; uwieżć.

\uventry{loĝ$'$}
habiter, loger | lodge | wohnen | жить, квартировать | mieszkać.

\uventry{loĝi$'$}
loge | cabin, lodge | Loge | ложа | loźa.

\uventry{lojt$'$}
lotte | eel-pout, lote | Aalraupe | налимъ | mięlus.

\uventry{lok$'$}
place, lieu | place | Ort | мѣсто | miejsce.

\uventry{lokomotiv$'$}
locomotive | locomotive | Lokomotive | локомотивъ | lokomotywa.

\uventry{lol$'$}
ivraie | lure, cockleweed | Lolch | плевелъ | kąkol.

\uventry{long$'$}
long | long | lang | долгій, длинный | długi.

\uvsubentry{}\uventry{laŭ$'$long$'$}
le long | along | entlang, der Länge nach | вдоль | wdłuź.

\uventry{lonicer$'$}
chèvrefeuille | honey-suckle | Geissblatt | жимолость | wiciokrzew.

\uventry{lorn$'$}
lorgnette | perspective-glass | Fernglas | лорнетъ | lorneta.

\uventry{lot$'$}
tirer au sort | cast lots | loosen | бросать жребій | losować.

\uvsubentry{}\uventry{lot$'$um$'$}
lotir | allot | verloosen | разыгривать | rozegrać.

\uventry{lu$'$}
louer (location) | rent | miethen | брать въ наемъ | wynajmować.

\uventry{lud$'$}
jouer | play | spielen | играть | bawić się, grać.

\uventry{luks$'$}
luxe | luxe | Luxus | роскошь | komfort, przepych.

\uventry{lul$'$}
bercer | lull asleep, rock | wiegen | качать, баюкать | kołysać.

\uvsubentry{}\uventry{lul$'$il$'$}
berceau | cradle | Wiege | колыбель | kołyska.

\uventry{lum$'$}
luire, lumière | light | leuchten | свѣтить | świecić.

\uvsubentry{}\uventry{mal$'$lum$'$a}
sombre, obscur | dark | dunkel | темный | ciemny.


\uvsubentry{}\uventry{lum$'$tur$'$}
phare | light-house | Leuchtthurm | маякъ | latarnia morska.

\uventry{lumb$'$}
lombes | loins, haunch | Lende | поясница | lędźwie.

\uventry{lun$'$}
lune | moon | Mond | луна | księźyć.

\uventry{lunatik$'$}
lunatique | lunatic | Mondsüchtiger | лунатикъ | lunatyk.

\uventry{lund$'$}
lundi | Monday | Montag | понедѣльникъ | poniedziałek.

\uventry{lup$'$}
loup | wolf | Wolf | волкъ | wilk.

\uventry{lupol$'$}
houblon | hops | Hopfen | хмѣль | chmiel.

\uventry{lustr$'$}
lustre | lustre, chandelier | Kronleuchter | люстра | pająk,
świecznik.

\uventry{lut$'$}
souder | solder | kitten, löten | паять | lutować.

\uventry{lutr$'$}
loutre | common otter | Fischotter | выдра | wydra.

\uvlitero{M}

\uventry{mac$'$}
azyme | unleavened bread | ungesäuertes Brot | опрѣснокъ | praśny chleb.

\uventry{maĉ$'$}
mâcher | chew | kauen | жевать | źuć.

\uventry{magazen$'$}
magasin | store | Kaufladen | лавка, магазинъ | sklep,
magazyn.

\uventry{magi$'$}
magie | magic, black art | Magie | магія | magie.

\uventry{magnet$'$}
aimant | magnet | Magnet | магнитъ | magnes.

\uventry{maiz$'$}
mais | maize | Mais | маисъ | kukurydza.

\uventry{Maj$'$}
Mai | May | Mai | Май | Maj.

\uventry{majest$'$}
majestueux | majesty | erhaben | величественный | majestatyczny, wzniosły.

\uventry{majstr$'$}
maître (dans sa partie) | foreman | Meister | мастеръ | majster.

\uvsubentry{}\uventry{sub$'$majstr$'$}
garçon de métier | journey-man | Handwerksgesell | подмастерье | podmajstrzy, czeladnik.

\uventry{makler$'$}
faire le courtier | play the broker | Mäkler sein | посредничать | pośredniczyć.

\uventry{makul$'$}
tache | stain | Fleck | пятно | plama.

\uventry{makzel$'$}
mâchoire | jaw-bone, cheek-bone | Kinnlade | челюсть | szczęka.

\uventry{mal$'$}
marque les contraires: ex. \uventry{bon$'$} bon ― \uventry{mal$'$bon$'$} mauvais;
\uventry{estim$'$} estimer ― \uventry{mal$'$estim$'$} mépriser | denotes opposites;
e. g. \uventry{alt$'$} high ― \uventry{mal$'$alt$'$} low | bezeichnet einen geraden
Gegensatz; z. B. \uventry{bon$'$} gut ― \uventry{mal$'$bon$'$} schlecht; \uventry{estim$'$} schätzen
― \uventry{mal$'$estim$'$} verachten | прямо противоположно; напр. \uventry{bon$'$}
хорошій ― \uventry{mal$'$bon$'$} дурной; \uventry{estim$'$} уважать ― \uventry{mal$'$estim$'$}
презирать | oznacza przeciwieństwo; np. \uventry{bon$'$} dobry ― \uventry{mal$'$bon$'$}
zły; \uventry{estim$'$} powaźać ― \uventry{mal$'$estim$'$} gardzić.

\uventry{maleol$'$}
ceville | ankle | Knöchel | лодыжка | kłykieć.

\uventry{malgraŭ}
malgré, en dépit de | in spite of | ungeachtet, obgleich | не смотря на | pomimo.

\uventry{malic$'$}
malicieux | malicious | tückisch | коварный | chytry.

\uventry{mam$'$}
mamelle | breast | Brust, Euter | грудь, вымя | pierś, sutka.

\uventry{man$'$}
main | hand | Hand | рука | ręka.

\uvsubentry{}\uventry{man$'$plat$'$}
paume | palm | Handfläche | ладонь | dłoń.


\uvsubentry{}\uventry{man$'$um$'$}
manchette | ruffle | Manschette | манжета | mankiet.

\uventry{mana$'$}
manne | manna | Manna | манна | manna.

\uventry{mangan$'$}
manganèse | manganese | Mangan | марганецъ | brunatnik.

\uventry{manĝ$'$}
manger | eat | essen | ѣсть | jeść.

\uvsubentry{}\uventry{maten$'$manĝ$'$}
déjeuner | breakfast | frühstücken | завтракать | śniadać.


\uvsubentry{}\uventry{tag$'$manĝ$'$}
dîner | dine | zu Mittag essen | обѣдать | obiadować.


\uvsubentry{}\uventry{vesper$'$manĝ$'$}
souper | sup | zu Abend essen | ужинать | jeść kolacyę, wieczerrzać.

\uventry{manier$'$}
manière, façon | manner | Manier, Weise, Art | способъ, образъ | sposób, maniera.

\uventry{manik$'$}
manche | sleeve | Aermel | рукавъ | rękaw.

\uventry{mank$'$}
manquer | want | fehlen | недоставать | brakować.

\uventry{manovr$'$}
manœuvrer | work a ship, take measures | manövriren | маневрировать | manewrować.

\uventry{mantel$'$}
manteau | mantle | Mantel | плащъ | plaszcz.

\uventry{mar$'$}
mer | sea | Meer | море | morze.

\uventry{marĉ$'$}
marais | swamp, marsh | Sumpf | болото | błoto.

\uventry{marĉand$'$}
marchander | bargain, trade | dingen, feilschen | торговаться | targować się.

\uventry{marcipan$'$}
massepain | marchpane | Marzipan | марципанъ | marcepan.

\uventry{mard$'$}
mardi | Tuesday | Dienstag | вторникъ | wterek.

\uventry{mark$'$}
marque | mark | Marke, Briefmarke | марка | marka.

\uventry{markot$'$}
marcotte | layer | Ableger | отпрыскъ | latorośl.

\uventry{marmot$'$}
marmotte | marmot | Murmelthier | сурокъ | świszcz.

\uventry{marmor$'$}
marbre | marble | Marmor | мраморъ | marmur.

\uventry{maroken$'$}
marroquin | marroquin | Saffian | сафьянъ | safian.

\uventry{marŝ$'$}
marcher | march, walk | marschiren | маршировать | marzerować.

\uventry{marŝal$'$}
maréchal | marshal | Marschall | маршалъ | marszałek.

\uventry{Mart$'$}
Mars | March | März | Мартъ | Marzec.

\uventry{martel$'$}
marteau | hammer | Hammer | молотъ | młot.

\uventry{mas$'$}
masse | mass | Masse | масса | massa.

\uventry{maŝ$'$}
nœud coulant, maille | mesh | Schlinge, Masche | петля | pętlica.

\uventry{maŝin$'$}
machine | machine | Maschine | машина | maszyna.

\uventry{masiv$'$}
massif | massive | massiv | массивный | masywny.

\uventry{mask$'$}
masque | mask | Maske | маска | maska.

\uventry{mason$'$}
maçonner | build with stone | mauern | класть стѣны (каменная
работа) | murować.

\uventry{mast$'$}
mât | mast | Mast, Mastbaum | мачта | maszt.

\uventry{mastik$'$}
mastic | mastick | Mastik | мастика | mastyka.

\uventry{mastr$'$}
hôte, maître de maison | master | Wirth | хозяинъ | gospodarz.

\uvsubentry{}\uventry{mastr$'$um$'$}
faire le ménage | keep house, husband | wirthschaften | хозяйничать | gospodarzyć.

\uventry{mat$'$}
natte | mat | Matte | рогожа | rogoźka.

\uventry{maten$'$}
matin | morning | Morgen | утро | poranek.

\uventry{matrac$'$}
matelas | mattress | Matratze | тюфякъ | materac.

\uventry{matur$'$}
mûr | ripe | reif | зрѣлый | dojrzały.

\uventry{mebl$'$}
meuble | furniture | Möbel | мебель | mebel.

\uventry{meĉ$'$}
mèche | wick | Docht | фитиль | knot.

\uvsubentry{}\uventry{meĉ$'$aĵ$'$}
amadou | amadou, match | Feuerschwamm | трутъ | hubka.

\uventry{medal$'$}
médaille | medal | Denkmünze | медаль | medal.

\uventry{medalion$'$}
médaillon | medallion | Medaillon | медаліонъ | medalion.

\uventry{medit$'$}
réfléchir, méditer | reflect, meditate | nachdenken | размышлять | medytować, rozmyślać.

\uventry{meĥanik$'$}
mécanique | mechanics | Mechanik | механика | mechanika.

\uventry{mejl$'$}
mille (mesure itinéraire) | mile | Meile | миля | mila.

\uventry{mel$'$}
blaireau | badger | Dachs | барсукъ | borsuk.

\uventry{meleagr$'$}
dindon | turkey | Truthahn | индюкъ | indyk.

\uventry{melk$'$}
traire | milk (vb.) | melken | доить | doić.

\uventry{melon$'$}
melon | melon | Melone | дыня | melon.

\uvsubentry{}\uventry{akv$'$o$'$melon$'$}
melon d’eau | water-melon | Wassermelone | арбузъ | arbuz.

\uventry{mem}
même (moi-, toi-, etc.) | self | selbst | самъ | sam.

\uventry{membr$'$}
membre | member | Glied | членъ | członek.

\uventry{membran$'$}
membrane | membrane | Häutchen, Membrane | перепонка | błona, błonka.

\uventry{memor$'$}
se souvenir, se rappeler | memory | sich erinnern, im Gedachtniss behalten | помнить | pamiętać.

\uventry{mem$'$star$'$}
indépendant | self-subsistent | selbstständig | самостоятельный | samodzielny.

\uventry{mend$'$}
mander, commettre | commit | bestellen | заказывать | stalować.

\uventry{mensog$'$}
mentir | tell a lie | lügen | врать | kłamać.

\uventry{ment$'$}
menthe | mint | Münze (Botan.) | мята | mięta.

\uventry{menton$'$}
menton | chin | Kinn | подбородокъ | podbródek.

\uventry{merit$'$}
mériter | merit | verdienen | заслуживать | zasługiwać.

\uventry{meriz$'$}
merise | bird-cherry | Vogelkirsche | черешня | czereśnia.

\uventry{merkred$'$}
mercredi | Wednesday | Mittwoch | среда | środa.

\uventry{merl$'$}
merle | blackbird | Amsel | черный дроздъ | kos.

\uventry{mes$'$}
messe | mass | Messe, Gottesdienst | обѣдня, богослуженіе | msza.

\uventry{Mesi$'$}
Messie | Messiah | Messias | Мессія | Messyasz.

\uventry{met$'$}
mettre, placer, poser | put, place | hinthun; kann durch
verschiedene Zeitwörter übersetzt werden | дѣть; можетъ быть
переведено различными глаголами | podziać; moźe być oddane za pomocą
rozmaitych czasowników.

\uvsubentry{}\uventry{el$'$met$'$}
exposer | expose | ausstellen | выставлять | wystawiać.

\uventry{meti$'$}
métier | handicraft | Handwerk | ремесло | rzemiosło.

\uventry{mev$'$}
mouette | sea-gull | Möwe | чайка | czajka.

\uventry{mez$'$}
milieu | middle | Mitte | средина | środek.

\uvsubentry{}\uventry{tag$'$mez$'$}
midi | mid-day | Mittag | полдень | poludnie.


\uvsubentry{}\uventry{mez$'$o$'$nombr$'$}
nombre moyen | average | durchschnittlich | среднимъ числомъ | przecięciowo.

\uventry{mezur$'$}
mesurer | measure | messen | мѣрить | mierzyć.

\uvsubentry{}\uventry{al$'$mezur$'$}
essayer, ajuster | adapt | anpassen | примѣривать | przymierzać.

\uventry{mi}
je, moi | I | ich | я | ja.

\uvsubentry{}\uventry{mi$'$a}
mon, ma | my, mine | mein | мой | mój.

\uventry{miel$'$}
miel | honey | Honig | медъ | miód.

\uventry{mien$'$}
mine, air | mien, air | Miene | мина (выраженіе лица) | mina.

\uventry{migdal$'$}
amande | almond | Mandel | миндаль | migdal.

\uventry{migr$'$}
voyager, courir le monde | migrate | wandern | странствовать | wędrować.

\uventry{miks$'$}
mêler | mix | mischen | смѣшивать | mieszać.

\uventry{mil$'$}
mille (nombre) | thousand | tausend | тысяча | tysiąc.

\uventry{mili$'$}
mil, millet | millet | Hirse | просо | proso.

\uventry{milit$'$}
guerroyer | fight | Krieg führen | воевать | wojować.

\uvsubentry{}\uventry{al$'$milit$'$}
conquérir | conquer | erobern | завоевывать | zawojować.

\uventry{min$'$}
mine, minière | mine | Mine | мина (пороховая) | mina.

\uventry{minac$'$}
menacer | menace, threat | drohen | грозить | grozić.

\uventry{minut$'$}
minute | minute | Minute | минута | minuta.

\uventry{miogal$'$}
rat musqué | musk-rat | Bisamspitzmaus | выхухоль | piźmoszczur.

\uventry{miop$'$}
myope | short-sight | kurzsichtig | близорукій | krótkowzroczny.

\uventry{miozot$'$}
myosotis | forget-me-not | Vergissmeinnicht | незабудка | niezapominajka.

\uventry{mir$'$}
s’étonner, admirer | wonder | sich wundern | удивляться | dziwić się.

\uventry{mirh$'$}
mirrhe | myrrh | Myrrhe | мирра | myrra.

\uventry{mirt$'$}
myrte | myrtle | Myrthe | мирта | myrta.

\uventry{mirtel$'$}
airelle, myrtille | bilberry | Heidelbeere | черника | czarna jagoda.

\uventry{misi$'$}
mission | mission | Mission | миссія | misya.

\uventry{mister$'$}
mystère | mystery | Mysterium | таинство | misterya.

\uvsubentry{}\uventry{mister$'$a}
mystérieux | mysterious | geheimnissvoll | таинственный | tajemniczy.

\uventry{mizer$'$}
misère | distress, misery | Noth | нужда | nędza.

\uventry{mod$'$}
mode | mode | Mode, Modus (gramm.) | мода, наклоненіе | moda, tryb.

\uventry{model$'$}
modèle, spécimen | model | Muster | образецъ | model.

\uventry{moder$'$}
modéré | moderate | mässig | умѣренный | umiarkowany.

\uventry{modest$'$}
modeste | modest | bescheiden | скромный | skromny.

\uventry{mok$'$}
se moquer | mock | spotten | насмѣхаться | szydzić.

\uventry{mol$'$}
mou | soft | weich | мягкій | miękki.

\uvsubentry{}\uventry{mol$'$anas$'$}
canard à duvet | eider-duck | Eiderente | гага | miękopiór.

\uventry{moment$'$}
moment | moment | Augenblick | мгновеніе | chwila, moment.

\uventry{mon$'$}
argent (monnaie) | money | Geld | деньги | pieniądze.

\uventry{monaĥ$'$}
moine | monk, friar | Mönch | монахъ | zakonnik.

\uventry{monarĥ$'$}
monarque | monarch | Monarch | монархъ | monarcha.

\uventry{monat$'$}
mois | month | Monat | мѣсяцъ | miesiąc.

\uventry{mond$'$}
monde, univers | world | Welt | міръ, свѣтъ | świat.

\uventry{moned$'$}
choucas | jack-daw, chough | Dohle | галка | kawka.

\uventry{monstr$'$}
monstre | monster | Ungeheuer | чудовище | potwór, monstrum.

\uventry{mont$'$}
montagne | mountain | Berg | гора | góra.

\uventry{montr$'$}
montrer | show | zeigen | показывать | pokazywać.

\uventry{monument$'$}
monument | monument | Denkmal | памятникъ | pomnik.

\uventry{mops$'$}
mopse | pug-dog | Mops | мопсъ | mops.

\uventry{mor$'$}
mœurs | habit | Sitte | нравъ, обычай | zwyczaj,
obyczaj.

\uventry{morbil$'$}
rougeole | measles | Masern | корь | odra.

\uventry{mord$'$}
mordre | bite | beissen | кусать | kąsać.

\uvsubentry{}\uventry{mord$'$et$'$}
ronger | gnaw | nagen | грызть | gryźć.

\uventry{morgaŭ}
demain | to-morrow | morgen | завтра | jutro.

\uventry{mort$'$}
mourir | die | sterben | умирать | umierać.

\uventry{morter$'$}
mortier | mortar | Mörtel | замазка | zaprawa (wapienna).

\uventry{morus$'$}
mûre | mulberry | Maulbeere | тутовая ягода | morwa.

\uventry{most$'$}
moût | must | Most | виноградный морсъ | moszcz.

\uventry{moŝt$'$}
titre commun; ex. \uventry{vi$'$a reĝ$'$a moŝt$'$o} votre majesté, \uventry{vi$'$a
general$'$a moŝt$'$o} monsieur le général, \uventry{vi$'$a episkop$'$a moŝt$'$o} etc. | universal title; e. g. \uventry{vi$'$a reĝ$'$a moŝt$'$o} your majesty; \uventry{vi$'$a moŝt$'$o}
your honor | allgemeiner Titel; z. B. \uventry{vi$'$a reĝ$'$a moŝt$'$o} Eure
Majestät, \uventry{vi$'$a general$'$a moŝt$'$o} etc. | общій титулъ; напр. \uventry{vi$'$a
reĝ$'$a moŝt$'$o} Ваше Величество; \uventry{vi$'$a general$'$a moŝt$'$o} и т. п. | Mość.

\uventry{mov$'$}
mouvoir | move | bewegen | двигать | ruszać.

\uventry{muel$'$}
moudre | mill | mahlen | молоть | mleć.

\uventry{muf$'$}
manchon | muff | Muff | муфта | mufka.

\uventry{muĝ$'$}
mugir | rush | brausen, zischen | шипѣть | burzyć się, wrzeć.

\uventry{muk$'$}
pituite, glaire | slime | Schleim | слизь | śluz.

\uventry{mul$'$}
mulet | mule | Maulesel | мулъ | muł.

\uventry{mult$'$}
beaucoup, nombreux | much, many | viel | много | wiele.

\uventry{mur$'$}
mur | wall | Wand | стѣна | ściana.

\uventry{murmur$'$}
murmurer, grommeler | murmur | murren, brummen | ворчать | mruczеć.

\uventry{mus$'$}
souris | mouse | Maus | мышь | mysź.

\uventry{muŝ$'$}
mouche | fly (a) | Fliege | муха | mucha.

\uventry{musk$'$}
mousse (la) | moss | Moos | мохъ | mech.

\uventry{muskat$'$}
muscade | nutmeg | Muskatnuss | мушкатный орѣхъ | orzech
muszkatałowy.

\uventry{muskol$'$}
muscle | muscle | Muskel | мускулъ | mięsień, muskuł.

\uventry{muslin$'$}
mousseline | muslin | Nesseltuch | кисея | muślin.

\uventry{mustard$'$}
moutarde | mustard | Senf | горчица | musztarda.

\uventry{mustel$'$}
martre | marten | Marder | куница | kuna.

\uventry{mut$'$}
muet | dumb | stumm | нѣмой | niemy.

\begin{minipage}{\textwidth}
\uvlitero{N}
\begin{outdent}{1em}
\uventry{n}
marque l’accusatif ou complément direct et le lieu où l’on va | ending of the objective, also marks direction | bezeichnet den
Accusativ, auch die Richtung | означаетъ винительный падежъ, также
направленіе | oznacza biernik, również kierunek.
\end{outdent}
\end{minipage}

\uventry{naci$'$}
nation | nation | Nation | нація, народъ | naród, nacya.

\uventry{naĝ$'$}
nager | swim | schwimmen | плавать | pływać.

\uventry{najbar$'$}
voisin | neighbour | Nachbar | сосѣдъ | sąsiad.

\uventry{najl$'$}
clou | nail | Nagel | гвоздь | gwóźdź.

\uventry{najtingal$'$}
rossignol | nightingale | Nachtigall | соловей | słowik.

\uventry{nanken$'$}
nankin | nankeen, nankin | Nanking | нанка | nankin.

\uventry{nap$'$}
chou-navet | cabbage | Kohlrübe | брюква | brukiew.

\uventry{narcis$'$}
narcisse | daffodil | Narcisse | нарцисъ | narcyz.

\uventry{nask$'$}
enfanter, faire naître | bear, produce | gebären | рождать | rodzić.

\uvsubentry{}\uventry{nask$'$iĝ$'$}
naître, provenir | be born | entstehen | возникать | rodzić się.


\uvsubentry{}\uventry{du$'$nask$'$it$'$}
jumeau | twin | Zwilling | близнецъ | bliźnię.

\uventry{natur$'$}
nature | nature | Natur | природа | natura.

\uventry{naŭ}
neuf (9) | nine | neun | девять | dziewięć.

\uventry{naŭz$'$}
dégoûter | nauseate | Uebelkeit erregen | тошнить | mdlić.

\uventry{naz$'$}
nez | nose | Nase | носъ | nos.

\uvsubentry{}\uventry{naz$'$um$'$}
pince-nez | pince-nez | Pince-nez, Nasenklemmer | пенсне | binokle.

\uventry{ne}
non, ne, ne... pas | no, not | nicht, nein | не, нѣтъ | nie.

\uventry{nebul$'$}
brouillard | fog | Nebel | туманъ | mgła.

\uventry{neces$'$}
nécessaire | necessary | nöthig, nothwendig | необходимій | niezbędny.

\uvsubentry{}\uventry{neces$'$ej$'$}
cabinet d’aisance | privy | Abtritt | отхожее мѣсто | ustęp.


\uvsubentry{}\uventry{neces$'$uj$'$}
nécessaire | sewing-desk | Necessär | несесеръ | neseserka.

\uventry{neĝ$'$}
neige | snow | Schnee | снѣгъ | śnieg.

\uventry{negliĝ$'$}
négligé | negligee | Negligé | неглиже | negliź.

\uventry{negoc$'$}
affaire | business | Geschäft | дѣло, занятіе | interes,
zajęcie.

\uventry{nek ― nek}
ni ― ni | neither ― nor | weder ― noch | ни ― ни | ani ― ani.

\uventry{nenia}
aucun | no kind of | kein | никакой | żaden.

\uventry{neniam}
ne... jamais | never | niemals | никогда | nigdy.

\uventry{nenie}
nulle part | nowhere | nirgends | нигдѣ | nigdzie.

\uventry{neniel}
nullement, en aucune façon | nohow | keineswegs, auf keine
Weise | никакъ | w źaden sposób.

\uventry{nenies}
de personne, à personne | no one’s | keinem gehörig | ничей | niczyj.

\uventry{nenio}
rien | nothing | nichts | ничто | nic.

\uventry{neniu}
personne | nobody | Niemand | никто | nikt.

\uventry{nep$'$}
petit-fils | grandson | Enkel | внукъ | wnuk.

\uventry{nepr$'$}
tout à fait | throughout | durchaus | непремѣнно | koniecznie.

\uventry{nest$'$}
nid | nest | Nest, Lager | гнѣздо, притонъ | gniazdo.

\uventry{net$'$}
net | net | in’s Reine | начисто | na czysto.

\uvsubentry{}\uventry{mal$'$net$'$}
brouillon | foul copy | Brouillon | начерно | na brudno.

\uventry{nev$'$}
neveu | nephew | Neffe | племянникъ | siostrzeniec, bratanek.

\uventry{ni}
nous | we | wir | мы | my.

\uvsubentry{}\uventry{ni$'$a}
notre | our | unser | нашъ | nasz.

\uventry{niĉ$'$}
niche | niche | Nische | ниша | nisza.

\uventry{nigr$'$}
noir | black | schwarz | черный | czarny.

\uventry{nivel$'$}
niveau | water-level, plumb | Niveau | уровень | poziom.

\uventry{nj$'$}
après les 1-5 premières lettres d’un prénom féminin lui donne un
caractère diminutif et caressant; ex. \uventry{Mari$'$ ― Ma$'$nj$'$}; \uventry{Emili$'$
― Emi$'$nj$'$}
| diminutive of female names; e. g. \uventry{Henriet$'$} Henrietta
― \uventry{Henri$'$nj$'$, He$'$nj$'$} Hetty | den ersten 1-5 Buchstaben eines
weiblichen Eigennamens beigefügt, verwandelt diesen in ein
Liebkosungswort; z. B. \uventry{Mari$'$ ― Ma$'$nj$'$}; \uventry{Emili$'$ ― Emi$'$nj$'$}
| приставленное къ первымъ 1-5 буквамъ имени собств. женскаго пола,
превращаетъ его въ ласкательное; напр. \uventry{Mari$'$ ― Ma$'$nj$'$}; \uventry{Emili$'$ ―
Emi$'$nj$'$}
| dodane do pierwszych 1-5 liter imienia własnego rodzaju
źeńskiego zmienia takowe w pieszczotliwe; np. \uventry{Mari$'$ ― Ma$'$nj$'$};
\uventry{Emili$'$ ― Emi$'$nj$'$}
.

\uventry{nobel$'$}
noble (subst.), gentilhomme | nobleman | Adeliger, Edelmann | дворянинъ | szlachcic.

\uventry{nobl$'$}
noble (adj) | noble | edel | благородный | szlachetny.

\uventry{nokt$'$}
nuit | night | Nacht | ночь | noc.

\uventry{nom$'$}
nom | name | Name | имя | imię.

\uvsubentry{}\uventry{nom$'$i}
nommer, appeler | name, nominate | nennen | называть | nazywać.


\uvsubentry{}\uventry{nom$'$e}
c’est-à-dire, savoir | namely, viz | nämlich | именно | mianowicie.

\uventry{nombr$'$}
nombre | number | Zahl | число | liczba, ilość.

\uvsubentry{}\uventry{unu$'$nombr$'$}
singulier | singular | Einzahl, Singular | единсвенное число | liczba pojedyńcza.

\uventry{nominativ$'$}
nominatif | nominative | Nominativ | именительный падежъ | mianownik.

\uventry{nord$'$}
nord | north | Norden | сѣверъ | północ.

\uventry{not$'$}
noter | note | notiren | записывать, отмѣчать | notować.

\uvsubentry{}\uventry{not$'$o (muzik$'$a)}
note (de musique) | note (mus.) | Note (Mus.) | нота | nuta.

\uventry{notari$'$}
notaire | notary | Notar | нотаріусъ | notaryusz, rejent.

\uventry{nov$'$}
nouveau | new | neu | новый | nowy.

\uventry{Novembr$'$}
Novembre | November | November | Ноябрь | Listopad.

\uventry{novic$'$}
novice | novice | Noviz | послушникъ | nowicyusz.

\uventry{nu!}
eh bien! | well! | nu! nun! | ну! | no!.

\uventry{nuanc$'$}
nuance | nuance | Schattirung, Abstufung | оттѣнокъ | odcień.

\uventry{nub$'$}
nuage, nuée | cloud | Wolke | облако | obłok, chmura.

\uventry{nud$'$}
nu | naked | nackt | нагой | nagi.

\uventry{nuk$'$}
nuque | neck | Genick | затылокъ | kark.

\uventry{nuks$'$}
noix | nut | Nuss | орѣхъ | orzech.

\uventry{nul$'$}
zéro | null | Null | нуль | nul, zero.

\uventry{numer$'$}
numéro | number (of a magazine, etc.) | Nummer | номеръ | numer.

\uventry{nun}
maintenant | now | jetzt | теперь | teraz, obecnie.

\uventry{nur}
seulement, ne... que | only (adv.) | nur | только | tylko.

\uventry{nutr$'$}
nourrir | nourish | nähren | питать, кормить | karmić.

\uvlitero{O}

\uventry{o}
marque le substantif | ending of nouns (substantive) | bezeichnet
das Substantiv | означаетъ существительное | oznacza rzeczownik.

\uventry{obe$'$}
obéir | obey | gehorchen | повиноваться | być posłusznym.

\uventry{objekt$'$}
objet | object | Gegenstand | предметъ | przedmiot.

\uventry{obl$'$}
marque l’adjectif numéral multiplicatif; ex. \uventry{du} deux ―
\uventry{du$'$obl$'$} double | ...fold; e. g. \uventry{du} two ― \uventry{du$'$obl$'$} twofold,
duplex | bezeichnet das Vervielfachungszahlwort; z. B. \uventry{du} zwei ―
\uventry{du$'$obl$'$} zweifach | означаетъ числительное множительное; напр. \uventry{du}
два ― \uventry{du$'$obl$'$} двойной | oznacza liczebnik wieloraki; np. \uventry{du} dwa
― \uventry{du$'$obl$'$} podwójny.

\uventry{oblat$'$}
pain à cacheter | wafer | Oblate | облатка | opłatek.

\uventry{observ$'$}
observer | observe | beobachten, beaufsichtigen | наблюдать | obserwować.

\uventry{obstin$'$}
entêté, obstiné | obstinate | eigensinnig | упрямый | uparty.

\uventry{obstrukc$'$}
obstruction | obstruction | Verstopfung | запоръ | obstrukcya, zatwardzenie.

\uventry{odor$'$}
sentir, avoir une odeur | odour | riechen, duften | пахнуть | pachnąć.

\uventry{ofend$'$}
offenser | offend | beleidigen | обижать | obraźać,
krzywdzić.

\uventry{ofer$'$}
sacrifier | offer | opfern | жертвовать | ofiarować.

\uventry{ofic$'$}
office, emploi | office | Amt | должность | urząd.

\uvsubentry{}\uventry{ofic$'$ist$'$}
fonctionnaire | officer | Beamter | чиновникъ | urzędnik.


\uvsubentry{}\uventry{ofic$'$ej$'$}
bureau | bureau | Bureau | бюро, канцелярія | biuro, kancelarya.

\uventry{oficir$'$}
officier | officer | Offizier | офицеръ | oficer.

\uventry{oft$'$}
souvent | often | oft | часто | często.

\uventry{ok}
huit | eight | acht | восемь | ośm.

\uventry{okaz$'$}
avoir lieu, arriver | happen | vorfallen | случаться | zdarzać się.

\uvsubentry{}\uventry{okaz$'$o}
occasion | occasion | Ereigniss, Gelegenheit | случай | wypadek, zdarzenie.


\uvsubentry{}\uventry{okaz$'$a}
accidentel | accidental | zufällig | случайный | przypadkowy.

\uventry{okcident$'$}
ouest | west | West, Westen | западъ | zachód.

\uventry{oksigen$'$}
oxygène | oxygen | Sauerstoff | кислородъ | tlen.

\uventry{oksikok$'$}
canneberge | moss-berry | Moosbeere | клюква | źurawina.

\uventry{Oktobr$'$}
Octobre | October | Oktober | Октябрь | Październik.

\uventry{okzal$'$}
oseille | sorrel | Ampfer | щавель | szczaw.

\uventry{okul$'$}
œil | eye | Auge | глазъ | oko.

\uvsubentry{}\uventry{okul$'$har$'$}
cils | eye-lash | Wimper | рѣсница | rzęsa.


\uvsubentry{}\uventry{okul$'$vitr$'$}
lunettes | spectacles | Brille | очки | okulary.

\uventry{okup$'$}
occuper | occupy | einnehmen, beschäftigen | занимать | zajmować.

\uventry{ol}
que (dans une comparaison) | than | als | чѣмъ | niź.

\uventry{ole$'$}
huile | oil | Oel | масло (деревянное) | olej.

\uventry{oliv$'$}
olive | olive | Olive | маслина | oliwa.

\uventry{omar$'$}
homard | lobster | Hummer | морской ракъ | homar.

\uventry{ombr$'$}
ombre | shadow | Schatten | тѣнь | cień.

\uventry{ombrel$'$}
parapluie, ombrelle | umbrella | Schirm | зонтикъ | parasol.

\uventry{on$'$}
marque les nombres fractionnaires; ex. \uventry{kvar} quatre ―
\uventry{kvar$'$on$'$} le quart | marks fractions; e. g. \uventry{kvar} four ―
\uventry{kvar$'$on$'$} a fourth, quarter | Bruchzahlwort; z. B. \uventry{kvar} vier ―
\uventry{kvar$'$on$'$} Viertel | означаетъ числительное дробное; напр. \uventry{kvar}
четыре ― \uventry{kvar$'$on$'$} четверть | oznacza liczebnik ułamkowy;
np. \uventry{kvar} cztery ― \uventry{kvar$'$on$'$} czwarta część, ćwierć.

\uventry{ond$'$}
onde, vague | wave | Welle | волна | fala.

\uventry{oni}
on | one, people, they | man | безличное мѣстоименіе множ. числа | zaimek nieosobisty liczby mnogiej.

\uventry{onkl$'$}
oncle | uncle | Onkel | дядя | wuj, stryj.

\uventry{ont$'$}
marque le participe futur d’un verbe actif | ending of
fut. part. act. in verbs | bezeichnet das Participium fut. act. | означаетъ причастіе будущаго времени дѣйствительнаго залога | oznacza
imiesłów czynny czasu przyszłego.

\uventry{op$'$}
marque l’adjectif numéral collectif; ex. \uventry{du} deux ― \uventry{du$'$op$'$} à
deux | marks collective numerals; e. g. \uventry{tri} three ― \uventry{tri$'$op$'$}
three together | Sammelzahlwort; z. B. \uventry{du} zwei ― \uventry{du$'$op$'$}
selbander, zwei zusammen | означаетъ числительное собирательное;
напр. \uventry{du} два ― \uventry{du$'$op$'$} вдвоемъ | oznacza liczebnik zbiorowy;
np. \uventry{du} dwa ― \uventry{du$'$op$'$} we dwoje.

\uventry{opal$'$}
opale | opal | Opal | опалъ | opal.

\uventry{opini$'$}
penser, croire | mean | meinen | имѣть мнѣніе | sądzić, opiniować.

\uventry{oportun$'$}
commode, ce qui est à propos | opportune, suitable | bequem | удобный | wygodny.

\uventry{or$'$}
or (métal) | gold | Gold | золото | złoto.

\uventry{orakol$'$}
oracle | oracle | Orakel | оракулъ | wyrocznia.

\uventry{oranĝ$'$}
orange | orange | Apfelsine | апельсинъ | pomarańcza.

\uventry{ord$'$}
ordre (arrangement) | order, arrange | Ordnung | порядокъ | porządek.

\uventry{orden$'$}
ordre | order | Orden | орденъ | order.

\uventry{ordinar$'$}
ordinaire | ordinary | gewöhnlich | обыкновенный | zwyczajny.

\uventry{ordon$'$}
ordonner | order, command | befehlen | приказывать | rozkazywać.

\uventry{orel$'$}
oreille | ear | Ohr | ухо | ucho.

\uventry{orf$'$}
orphelin | orphan | Waise | сирота | sierota.

\uventry{orgen$'$}
orgue | organ | Orgel | органъ | organ.

\uventry{orient$'$}
est | east | Osten | востокъ | wschód.

\uventry{ornam$'$}
orner | ornament | putzen | наряждать | zdobić.

\uventry{os}
marque le futur | ending of future tense in verbs | bezeichnet
das Futur | означаетъ будущее время | oznacza czas przyszły.

\uventry{osced$'$}
bâiller | yawn | gähnen | зѣвать | ziewać.

\uventry{ost$'$}
os | bone | Knochen | кость | kość.

\uventry{ostr$'$}
huître | oyster | Auster | устрица | ostryga.

\uventry{ot$'$}
marque le participe futur d’un verbe passif | ending of
fut. part. pass. in verbs | bezeichnet das Participium fut. pass. | означаетъ причастіе будущаго времени страдательнаго залога | oznacza
imiesłów bierny czasu przyszłego.

\uventry{ov$'$}
œuf | egg | Ei | яйцо | jajko.

\uvsubentry{}\uventry{ov$'$uj$'$}
ovaire | ovary | Eierstock | яичникъ | jajnik.


\uvsubentry{}\uventry{ov$'$blank$'$}
blanc d’œuf, aubin | white of an egg | Eiweiss | бѣлокъ | białko.

\uventry{oval$'$}
ovale | oval | oval | овальный | owalny.

\uvlitero{P}

\uventry{pac$'$}
paix | peace | Friede | миръ | pokój, spokój.

\uventry{pacienc$'$}
patience | patience | Geduld | терпѣніе | cierpliwość.

\uventry{paf$'$}
tirer, faire feu | shoot | schiessen | стрѣлять | strzelać.

\uvsubentry{}\uventry{paf$'$il$'$}
fusil | gun | Flinte | ружье | strzelać.


\uvsubentry{}\uventry{paf$'$il$'$eg$'$}
canon | cannon | Kanone | пушка | armata.

\uventry{pag$'$}
payer | pay | zahlen | платить | płacić.

\uvsubentry{}\uventry{de$'$pag$'$}
impôt | duty | Steuer, Abgabe, Zoll | подать, пошлина | podatek.

\uventry{paĝ$'$}
page (d’un livre) | page | Seite (Buch-) | страница | stronica.

\uventry{paĝi$'$}
page | page | Page | пажъ | paź.

\uventry{pajl$'$}
paille | straw | Stroh | солома | słoma.

\uventry{pak$'$}
empaqueter, emballer | pack, put ut | packen, einpacken | укладывать, упаковывать | pakować.

\uventry{pal$'$}
pâle | pale | bleich, blass | блѣдный | blady.

\uventry{palac$'$}
palais | palace | Schloss (Gebäude) | дворецъ | pałac.

\uventry{palat$'$}
palais (de la bouche) | palate | Gaumen | нёбо | podniebienie.

\uventry{paletr$'$}
palette | palette | Palette | палитра | paleta.

\uventry{palis$'$}
pieu, échalas, palissade | pale, stake | Pfahl | тычина, колъ | pal.

\uventry{palm$'$}
palmier | palm | Palme | пальма | palma.

\uventry{palp$'$}
palper | touch, feel | tasten | щупать | macać.

\uventry{palpebr$'$}
paupière | eyelid | Augenlied | вѣко | powieka.

\uvsubentry{}\uventry{palpebr$'$um$'$}
cligner, clignoter | twinkle | blinzeln | моргать | mrugać.

\uventry{pan$'$}
pain | bread | Brot | хлѣбъ | chleb.

\uventry{pantalon$'$}
pantalon | pantaloon, trowsers | Hosen | брюки | spodnie.

\uventry{panter$'$}
panthère | panther | Panther | пантера | pantera.

\uventry{pantofl$'$}
pantoufle | pantofle | Pantoffel | туфель | pantofel.

\uventry{pap$'$}
pape | pope | Papst | папа (Римскій) | papieź.

\uventry{papag$'$}
perroquet | parrot | Papagei | попугай | papuga.

\uventry{papav$'$}
pavot | poppy | Mohn | макъ | mak.

\uventry{paper$'$}
papier | paper | Papier | бумага | papier.

\uventry{papili$'$}
papillon | butterfly | Schmetterling | бабочка | motyl.

\uventry{par$'$}
paire | pair | Paar | пара | para.

\uventry{parad$'$}
faire parade | make parade | prangen | парадировать | paradować.

\uventry{paradiz$'$}
paradis | paradise | Paradies | рай | raj.

\uventry{paraliz$'$}
paralyser | paralyse | paralysiren | парализовать | paralizować.

\uventry{parazit$'$}
parasite | parasite | Schmarotzer | паразитъ | pasorzyt.

\uventry{pardon$'$}
pardonner | forgive | verzeihen | прощать | przebaczać.

\uventry{parenc$'$}
parent | relation | Verwandter | родственникъ | krewny.

\uventry{parentez$'$}
parenthèse | parenthesis | Parenthese | скобка | nawias.

\uventry{parfum$'$}
parfum | parfume | Parfüm | духи | perfuma.

\uventry{parget$'$}
parquet | pit | Parquet | паркетъ | posadzka.

\uventry{park$'$}
parc | park | Park | паркъ | park.

\uventry{parker$'$}
par cœur (de mémoire) | by heart, thoroughly | auswendig | наизусть | na pamięć.

\uventry{paroĥ$'$}
cure, paroisse | parish | Pfarre | приходъ (церковный) | parafia.

\uventry{parol$'$}
parler | speak | sprechen | говорить | mówić.

\uvsubentry{}\uventry{inter$'$parol$'$}
entretien | discourse | Gespräch, Unterhaltung | бесѣда | rozmowa.


\uvsubentry{}\uventry{el$'$parol$'$}
prononcer | pronounce | aussprechen | произносить | wymówić.

\uventry{part$'$}
partie, part | part | Theil | часть | część.

\uvsubentry{}\uventry{part$'$o$'$pren$'$}
participer | participate | Theil nehmen | участвовать | przyjmować udział.

\uventry{parter$'$}
parterre | ground-floor | Parterre | партеръ | parter.

\uventry{parti$'$}
parti | party | Partei, Partie | партія | partya.

\uvsubentry{}\uventry{parti$'$a}
partial | partial | parteiisch | пристрастный | stronny, stronniczy.

\uventry{particip$'$}
participe | participle | Participium | причастіе | imlesłów.

\uventry{paru$'$}
mésange | muskin | Meise | синица | sikora.

\uventry{pas$'$}
passer | pass | vergehen | проходить | przechodzić.

\uventry{paŝ$'$}
faire des pas, enjamber | stride, step | schreiten | шагать | kroczyć.

\uventry{pasament$'$}
passement | lace | Borte, Tresse | позументъ | galon, ślaczek.

\uvsubentry{}\uventry{pasament$'$ist$'$}
passementiér | lace-maker | Posamentirer | позументщикъ | szmuklerz.

\uventry{paser$'$}
passereau | sparrow | Sperling | воробей | wróbel.

\uventry{pasi$'$}
passion | passion | Leidenschaft | страсть | namiętność.

\uventry{Pask$'$}
Pâques, Pâque | Easter | Ostern | Пасха | Wielkanoc.

\uventry{pasport$'$}
passe-port | pass-port | Reisepass | паспортъ | paszport.

\uventry{past$'$}
pâte | paste | Teig | тѣсто | ciasto.

\uventry{paŝt$'$}
paître | pasture, feed animals | weiden lassen | пасти | paść.

\uventry{pasteĉ$'$}
pâté | pasty | Pastete | пастетъ | pasztet.

\uventry{pastel$'$}
pastille | pastil | Pastille | лепешка | pastylka.

\uventry{pastinak$'$}
panais | parsnip | Pastinake | пастернакъ | pasternak.

\uventry{pastr$'$}
prêtre, pasteur | priest, pastor | Priester | жрецъ,
священникъ | kapłan.

\uventry{pat$'$}
poêle (à frire) | frying-pan | Pfanne | сковорода | patelnia.

\uventry{patr$'$}
père | father | Vater | отецъ | ojciec.

\uvsubentry{}\uventry{patr$'$uj$'$}
patrie | fatherland | Vaterland | отечество | ojczyzna.

\uventry{patrol$'$}
patrouille | patrol | Patrouille | патруль | patrol.

\uventry{paŭz$'$}
pause | pause | Pause | пауза | pauza.

\uventry{pav$'$}
paon | peacok | Pfau | павлинъ | paw$'$.

\uventry{pavim$'$}
pavé | pavement | Pflaster (Strassen-) | мостовая | bruk.

\uventry{pec$'$}
morceau | piece | Stück | кусокъ | kawałek.

\uventry{peĉ$'$}
poix | pitch | Pech | смола | smoła.

\uventry{pedik$'$}
pou | louse | Laus | вошь | wesz.

\uventry{peg$'$}
pic (oiseau) | wood-peck | Specht | дятелъ | dzięcioł.

\uventry{pejzaĝ$'$}
paysage | landscape | Landschaft | пейзажъ | krajobraz, landszaft.

\uventry{pek$'$}
pécher | sin | sündigen | грѣшить | grzeszyć.

\uventry{pekl$'$}
saler | pickle | pökeln | солить | solić, peklować.

\uventry{pel$'$}
chasser, renvoyer | pursue, chase out | jagen, treiben | гнать | gonić.

\uventry{pelikan$'$}
pélican | pelican | Kropfgans | пеликанъ | pelikan.

\uventry{pelt$'$}
pelisse | fur | Pelz | шуба | koźuch, futro.

\uventry{pelv$'$}
bassin | basin, pelvis | Becken | тазъ | miednica.

\uventry{pen$'$}
tâcher, s’efforcer de | endeavour | sich bemühen | стараться | starać się.

\uventry{pend$'$}
pendre, être suspendu | hang | hängen (v. n.) | висѣть | wisieć.

\uvsubentry{}\uventry{de$'$pend$'$}
dépendre | hang from | abhängen | зависѣть | zaleźeć.


\uvsubentry{}\uventry{el$'$pend$'$aĵ$'$}
enseigne | ensign | Aushängeschild | вывѣска | szyld.

\uventry{pendol$'$}
pendule, perpendicule | perpendicle | Pendel | маятникъ | wahadło.

\uventry{penetr$'$}
pénétrer | penetrate | dringen | проникать | przenikać.

\uventry{penik$'$}
pinceau, houppe | paintbrush | Pinsel, Quast | кисть | pędzel.

\uventry{pens$'$}
penser | think | denken | думать | myśleć.

\uvsubentry{}\uventry{el$'$pens$'$}
inventer | invent | erfinden | изобрѣтать | wymyśleć.


\uvsubentry{}\uventry{pri$'$pens$'$}
considérer | consider | überlegen, nachdenken | обдумывать | obmyśleć.

\uventry{pensi$'$}
pension | pension | Pension | пенсія, пенсіонъ | emerytura.

\uventry{pent$'$}
se repentir | repent | bereuen, Busse thun | раскаиваться | źałować, pokutować.

\uventry{Pentekost$'$}
Pentecôte | Pentecost, Whitsuntide | Pfingsten | Пятидесятница | Zielone świątki.

\uventry{pentr$'$}
peindre | paint | malen | рисовать | rysować, malować.

\uventry{pep$'$}
gazouiller | warble, purl | pipen | чирикать | piszczeć.

\uventry{per}
par, au moyen, à l’aide de | through, by means of | mittelst,
vermittelst, durch | посредствомъ | pwzez, za pomocą.

\uvsubentry{}\uventry{per$'$a}
médiat | mediate | mittelbar | посредственный | pośredni.


\uvsubentry{}\uventry{per$'$i}
moyenner | mediate, interpose | vermitteln | посредничать | pośredniczyć.

\uventry{perĉ$'$}
perche goujonnière, grémille | perch (fish) | Kaulbars | ершъ | jaźdź.

\uventry{perd$'$}
perdre | lose | verlieren | терять | gubić.

\uventry{perdrik$'$}
perdrix | partridge | Rebhuhn, Feldhuhn | куропатка | kuropatwa.

\uventry{pere$'$}
périr, se perdre | perish | umkommen | погибать | ginąć.

\uvsubentry{}\uventry{pere$'$ig$'$}
ruiner, tuer | murder | umbringen | губить | gubić.

\uventry{perfekt$'$}
parfait | perfect | vollkommen | совершенный | zupełny, doskonały.

\uventry{perfid$'$}
trahir | betray | verrathen | измѣнять, предавать | zdradzić.

\uvsubentry{}\uventry{perfid$'$a}
perfide | perfidious | verrätherisch | измѣнническій | zdradziecki.

\uventry{pergamen$'$}
parchemin | parchment | Pergament | пергаментъ | pergamin.

\uventry{peritone$'$}
péritoine | peritoneum | Darmfell | брюшина | otrzewna,
błona brzuszna.

\uventry{perk$'$}
perche | perch | Barsch, Bars | окунь | okuń.

\uventry{perl$'$}
perle | pearl | Perl | жемчугъ | perła.

\uventry{perlamot$'$}
nacre de perle | mother of pearl | Perlmutter | перламутръ | macica perłowa.

\uventry{permes$'$}
permettre | permit, allow | erlauben | позволять | pozwalać.

\uvsubentry{}\uventry{for$'$permes$'$}
donner congé | give furlough | beurlauben | отпускать | zwalniać, dać urlop.

\uventry{peron$'$}
perron | stoop, front-steps | Freitreppe, Perron | крыльцо | peron.

\uventry{persekut$'$}
poursuivre, persécuter | persecute | verfolgen | преслѣдовать | prześladować.

\uventry{persik$'$}
pêche (fruit) | peach | Pfirsiche | персикъ | brzoskwinia.

\uventry{persist$'$}
persévérer | persist | beharren | настаивать | nalegać.

\uventry{person$'$}
personne (la) | person | Person | особа, лицо | osoba, persona.

\uvsubentry{}\uventry{person$'$a}
personnel | personal | persönlich | личный | osobisty.

\uventry{pes$'$}
peser (prendre le poids) | weigh (vb. act.) | wägen | взвѣшивать | ważyć (kogo, co).

\uvsubentry{}\uventry{pes$'$il$'$}
balance | balance, pair of scales | Wage | вѣсы | waga.

\uventry{pest$'$}
peste (la) | plague | Pest | чума | zaraza.

\uventry{pet$'$}
prier (quelqu’un) | request, beg | bitten | просить | prosić.

\uventry{petol$'$}
faire le polisson, faire des espiègleries | be petulant, be
arch | muthwillig sein | шалить | dokazywać, swawolić.

\uventry{petrol$'$}
petrole | coal-oil, kerosene | Erdöl, Petroleum | нефть | nafta.

\uventry{petromiz$'$}
lamproie | lamprey | Neunauge | минога | minoga.

\uventry{petrosel$'$}
persil | parsley | Petersilie | петрушка | pietruszka.

\uventry{pez$'$}
peser (avoir del poids) | weigh (vb. neut.) | wiegen | вѣсить
(имѣть вѣсъ) | ważyć (mieć wagę).

\uvsubentry{}\uventry{pez$'$il$'$}
poids | weight | Gewicht (zum Wägen) | гиря | cięźarek, waźka.

\uventry{pi$'$}
pieux | pious | fromm | благочестивый, набожный | poboźny.

\uventry{pice$'$}
sapin | fir-tree | Edeltanne | пихта | jodła.

\uventry{pied$'$}
pied | foot | Fuss, Bein | нога | noga.

\uvsubentry{}\uventry{pied$'$ing$'$}
étrier | stirrup | Steigbügel | стремя | strzemię.


\uvsubentry{}\uventry{piedestal$'$}
piédestal | pedestal | Piedestal, Postament | пьедесталъ | postument.

\uventry{pig$'$}
pie (la) | magpie | Elster | сорока | sroka.

\uventry{pik$'$}
piquer; Pique | prick, sting; spade | stechen; Pik (in Karten) | колоть; пика | kłuć; pik.

\uvsubentry{}\uventry{pik$'$il$'$}
aignillon, écharde | sting, thorn | Stachel | жало | źądło.

\uventry{piked$'$}
piquet | picket | Piquet | пикетъ | pikieta.

\uventry{pilk$'$}
balle (à jouer) | ball (to play with) | Ball (Spiel-) | мячикъ | piłka.

\uventry{pilol$'$}
pilule | pill | Pille | пилюля | pigułka.

\uventry{pilot$'$}
pilote-côtier | pilot, loadsmann | Lootsmann | лоцманъ | locman.

\uventry{pin$'$}
pin | pine-tree | Fichte | сосна | sosna.

\uventry{pinĉ$'$}
pincer | pinch | kneifen | щипать | szczypać.

\uventry{pingl$'$}
épingle | pin | Stecknadel, Tangel | булавка, хвоя | szpilka.

\uventry{pini$'$}
pignon | pine-tree | Pinie | сибирскій кедръ | pinela.

\uventry{pint$'$}
pointe, bout | point, tip, peak | Spitze | остріе, носокъ | wierzchołek, szczyt.

\uventry{pionir$'$}
pionnier | pioneer | Pionnier | піонеръ | pionier.

\uventry{pip$'$}
pipe | pipe (tobacco) | Pfeife (Tabaks-) | трубка | fajka,
lulka.

\uventry{pipr$'$}
poivre | pepper | Pfeffer | перецъ | pieprz.

\uventry{pips$'$}
pépie | pip | Pips (Krankheit der Vögel)) | типунъ | pypeć.

\uventry{pir$'$}
poire | pear | Birne | груша | gruszka.

\uventry{pirit$'$}
gravier, pyrite | gravel, pyrites | Kies | колчеданъ | źwir.

\uventry{pirol$'$}
bouvreuil | bullfinch | Dompfaff | снигирь | gil.

\uventry{piroz$'$}
fer-chaud | heartburn | Sodbrennen | изжога | zgaga.

\uventry{pist$'$}
piler, broyer | pound, bruise | kleinstossen | толочь | tłuc.

\uventry{piŝt$'$}
piston | piston, sucker | Kolben | поршень | kolba.

\uventry{pistak$'$}
pistache | pistachio-nut | Pistacie | фисташка | pistacya.

\uventry{pistol$'$}
pistolet | pistol | Pistole | пистолетъ | pistolet.

\uventry{piz$'$}
pois | pea | Erbse | горохъ | groch.

\uventry{plac$'$}
place (publique) | public square | Platz | площадь | plac.

\uventry{plaĉ$'$}
plaire | please | gefallen | нравиться | podobać się.

\uventry{plad$'$}
plat (un) | plate | Schüssel | блюдо | półmisek.

\uventry{plafon$'$}
plafond | ceiling | Zimmerdecke | потолокъ | sufit.

\uventry{plan$'$}
plan | plan | Plan | планъ | plan.

\uventry{pland$'$}
plante du pied, semelle | sole (of the foot) | Sohle | подошва | podeszwa.

\uventry{planed$'$}
planète | planet | Planet | планета | planeta.

\uventry{plank$'$}
plancher | floor | Fussboden | полъ | podłoga.

\uventry{plant$'$}
planter | plant (vb.) | pflanzen | сажать, насаждать | sadzić.

\uventry{plastr$'$}
emplâtre | plaster | Pflaster (medic.) | пластырь | plaster.

\uventry{plat$'$}
plat, plate | flat, plain | flach | плоскій | płaski.

\uventry{platen$'$}
platine | platina | Platina | платина | platyna.

\uventry{plaŭd$'$}
battre, claquer | splash, clap | plätschern, klatschen | плескать | klaskać.

\uventry{plej}
le plus | most | am meisten | наиболѣе | najwięcej.

\uventry{plekt$'$}
tresser | weave, plait | flechten | плесть | pleść.

\uventry{plen$'$}
plein | full | voll | полный | pełny.

\uvsubentry{}\uventry{plen$'$aĝ$'$}
majeur | of full age | mündig | совершеннолѣтній | pełnoletni.


\uvsubentry{}\uventry{plen$'$um$'$}
accomplir | accomplish | erfüllen | исполнять | spełniać.

\uventry{plend$'$}
plaindre, se plaindre | complain | klagen | жаловаться | skarźyć się.

\uventry{plet$'$}
plateau | teaboard | Präsentirteller | подносъ | taca.

\uventry{plezur$'$}
plaisir | pleasure | Vergnügen | удовольствіе | przyjemność.

\uventry{pli}
plus | more | mehr | больше | więcej.

\uventry{plik$'$}
plique | plica | Weichselzopf | колтунъ | kołtun.

\uventry{plor$'$}
pleurer | mourn, weep | weinen | плакать | płakać.

\uventry{plot$'$}
gardon | roach | Plötze | плотва, плотица | płotka.

\uventry{plu}
de plus | farther, further | weiter, ferner | дальше | dalej.

\uventry{plug$'$}
labourer | plough | pflügen | пахать | orać.

\uventry{plum$'$}
plume | pen | Feder | перо | pióro.

\uventry{plumb$'$}
plomb | lead (metall) | Blei | свинецъ | ołów.

\uventry{pluŝ$'$}
peluche | plush | Plüsch | плюшъ | plusz.

\uventry{pluv$'$}
pluie | rain | Regen | дождь | deszcz.

\uventry{po}
numéral distributif qui a le sens de: par, au taux de, sur le
pied de | by (with numbers) | (bei Zahlwörtern) zu | по (при
числительныхъ) | po (przy liczebnikach).

\uventry{poent$'$}
point | stitch, point | Point | очко | oczko.

\uventry{pokal$'$}
bocal, gobelet | cup, goblet | Becher | бокалъ | puhar,
kielich.

\uventry{polic$'$}
police | police | Polizei | полиція | policya.

\uventry{poligon$'$}
blé noir, sarrasin | buckwheat | Buchweizen | греча | gryka.

\uventry{polur$'$}
poli | polish, politure | Glanz, Politur | политура, лоскъ | politura.

\uventry{polus$'$}
pôle | pole | Pol | полюсъ | biegun.

\uventry{polv$'$}
poussière | dust | Staub | пыль | kurz.

\uventry{pom$'$}
pomme | apple | Apfel | яблоко | jabłko.

\uvsubentry{}\uventry{ter$'$pom$'$}
pomme de terre | potatoe | Kartoffel | картофель | kartofel.

\uventry{ponard$'$}
poignard | dagger, poniard | Dolch | кинжалъ | kindżał.

\uvsubentry{}\uventry{ponard$'$eg$'$}
épieu, pique | spear, lance | Pike, Spiess | копье | pika.

\uventry{pont$'$}
pont | bridge | Brücke | мостъ | most.

\uventry{popl$'$}
peuplier | poplar | Pappel | тополь | topola.

\uventry{popol$'$}
peuple | people | Volk | народъ | naród.

\uvsubentry{}\uventry{popol$'$amas$'$}
populace | mob, populace | Pöbel | чернь | pospólstwo, gmin.

\uventry{por$'$}
pour, en faveur de | for | für | для, за | dla, za.

\uventry{porcelan$'$}
porcelain | porcelain | Porzellan | фарфоръ | porcelana.

\uventry{porci$'$}
portion | portion | Portion | порція | porcya.

\uventry{pord$'$}
porte | door | Thür | дверь | drzwi.

\uvsubentry{}\uventry{pord$'$eg$'$}
porte cochère | gate | Thor | ворота | brama.

\uventry{porfir$'$}
porphyre | porphyry | Porphyr | порфиръ | porfir.

\uventry{pork$'$}
cochon | hog | Schwein | свинья | świnia.

\uventry{port$'$}
porter | pack, carry | tragen | носить | nosić.

\uvsubentry{}\uventry{al$'$port$'$}
apporter | bring | bringen | приносить | przynosić.


\uvsubentry{}\uventry{el$'$port$'$}
supporter | bear, support | ertragen | выносить | znosić.

\uventry{porter$'$}
double bière | porter | Porter | портеръ | porter.

\uventry{portret$'$}
portrait | portrait | Portrait | портретъ | portret.

\uventry{poŝ$'$}
poche | pocket | Tasche | карманъ | kieszeń.

\uventry{posed$'$}
posséder | possess | besitzen, mächtig sein | владѣть | posiadać.

\uventry{post}
après | after, behind | nach, hinter | послѣ, за | po, za, poza, potem.

\uventry{poŝt$'$}
poste (la) | post | Post | почта | poczta.

\uvsubentry{}\uventry{sign$'$o de poŝt$'$o}
timbre-poste | postage-stamp | Briefmarke | марка | marka.

\uventry{posten$'$}
poste | post | Posten | постъ, мѣсто | stanowisko.

\uventry{postul$'$}
exiger, requérir | require, claim | fordern | требовать | żądać.

\uventry{pot$'$}
pot | pot | Topf | горшокъ | garnek.

\uventry{potas$'$}
potasse | potash | Pottasche | поташъ | potaż.

\uventry{potenc$'$}
puissance | might, power | Macht | могущество | władza,
siła, potęga.

\uventry{pov$'$}
pouvoir | be able, can | können | мочь | módz.

\uventry{pra$'$}
primitiv, bis- | primordial, great- | ur- | пра- | pra-.

\uventry{praktik$'$}
pratique | practic | Praxis | практика | praktyka.

\uventry{pram$'$}
prame | prame | Prahm | паромъ | prom.

\uventry{prav$'$}
qui a raison, qui est dans le vrai | right (to be in the
right) | Recht habend | правый (напр. я правъ) | mający słuszność.

\uventry{precip$'$}
principalement, surtout | particularly | besonder,
vorzüglich | преимущественно | szczególnie, przedewszystkiem.

\uventry{preciz$'$}
précis, juste | precise | genau, eben | точный | dokładny,
ścisły.

\uventry{predik$'$}
prêcher | preach | predigen | проповѣдывать | kazać (mieć
kazanie).

\uventry{predikat$'$}
attribut | attribute | Prädikat | сказуемое | orzeczenie.

\uventry{prefer$'$}
préférer | prefer | vorziehen | предпочитать | przekładać.

\uventry{preĝ$'$}
prier (Dieu) | pray | beten | молиться | modlić się.

\uventry{prem$'$}
presser, comprimer | press | drücken, pressen | давить | cisnąć, uciskać.

\uventry{premi$'$}
prime | premium, prize | Prämie | премія | premia.

\uventry{pren$'$}
prendre | take | nehmen | брать | brać.

\uvsubentry{}\uventry{pren$'$o}
levée | trick | Stich (kartensp.) | взятка (въ картахъ) | lewa, wziątka.


\uvsubentry{}\uventry{pren$'$il$'$}
tenailles | tongs | Zange | щипцы | szczypce.

\uventry{prepar$'$}
préparer | prepare | bereiten, zubereiten | готовить | przygotowywać.

\uventry{prepozici$'$}
préposition | preposition | Vorwort, Präposition | предлогъ | przyimek.

\uventry{pres$'$}
imprimer | print (vb.) | drucken, prägen | печатать | drukować.

\uventry{preskaŭ}
presque | almost | fast, beinahe | почти | prawie.

\uventry{pret$'$}
prêt, disposé | ready | fertig | готовый | gotowy.

\uventry{pretekst$'$}
prétexte | pretext | Vorwand | предлогъ, отговорка | pretekst, wymówka.

\uventry{pretend$'$}
prétendre | pretend | Anspruch machen | претендовать | pretendować, rościć prawa do czego.

\uventry{preter}
outre | beside, along | vorbei | мимо | mimo.

\uventry{prez$'$}
prix | price | Preis | цѣна | cena.

\uventry{prezent$'$}
présenter | present (vb.) | vorstellen | представлять | przedstawiać.

\uvsubentry{}\uventry{re$'$prezent$'$}
représenter | represent | vertreten | быть представителемъ | reprezentować.

\uventry{prezid$'$}
présider | preside | den Vorsitz haben, präsidiren | предсѣдательствовать | prezydować.

\uventry{pri}
sur, touchant, de | concerning, about | von, über | о, объ | o.

\uventry{primol$'$}
primevère | primrose | Schlüsselblume | баранчикъ (растеніе) | pierwiosnek.

\uventry{princ$'$}
prince, souverain | prince | Fürst, Prinz | принцъ, князь | książe.

\uventry{printemp$'$}
printemps | spring time | Frühling | весна | wiosna.

\uventry{privat$'$}
privé, particulier | private | privat | частный | prywatny.

\uventry{privilegi$'$}
privilège | privilege | Vorrecht | привиллегія | przywilej.

\uventry{pro}
à cause de, pour | for the sake of | um ― willen, wegen | ради | dla.

\uventry{pro$'$cent$'$}
intérêt, pour cent | per cent | Procent | процентъ | procent, odsetka.

\uvsubentry{}\uventry{pro$'$cent$'$eg$'$}
usure | usury | Wucher | лихоимство | lichwa.

\uventry{proces$'$}
procès | lawsuit, process | Process | процесъ | proces, sprawa.

\uventry{produkt$'$}
produite | produce | erzeugen | производить | produkować.

\uventry{profesi$'$}
profession | profession | Profession, Gewerbe | профессія,
занятіе | profesya.

\uventry{profet$'$}
prophète | prophet | Prophet | пророкъ | prorok.

\uventry{profit$'$}
profiter | profit, gain | gewinnen, Nutzen ziehen | имѣть
барышъ | mieć kozyść.

\uventry{profund$'$}
profond | deep | tief | глубокій | głęboki.

\uventry{progres$'$}
avancer | advance, progress | fortschreiten | прогрессировать | postępować.

\uventry{proklam$'$}
proclamer | proclaim | proklamiren | прокламировать | proklamować.

\uventry{prokrast$'$}
remettre, retarder | delay, retard | aufschieben,
verzögern | отстрочивать | prolongować.

\uventry{proksim$'$}
proche, près de | near | nahe | близкій | blizki.

\uventry{promen$'$}
se promener | walk, promenade | spazieren | прогуливаться | spacerować.

\uventry{promes$'$}
promettre | promise | versprechen | обѣщать | obiecywać.

\uventry{promontor$'$}
promontoire, cap | promontory, cape | Vorgebirge | мысъ | przedgórze.

\uventry{pronom$'$}
pronom | pronoun | Fürwort | мѣстоименіе | zaimek.

\uventry{propon$'$}
proposer, offrir | propose, suggest | vorschlagen | предлагать | proponować.

\uventry{propr$'$}
propre (à soi) | own (one’s own) | eigen | собственный | własny.

\uventry{prosper$'$}
réussir | prosper | gelingen | удаваться | udać się.

\uventry{prov$'$}
essayer | attempt, trial | versuchen, probiren | пробовать | próbować.

\uventry{proverb$'$}
proverbe | proverb | Sprichwort | пословица | przysłowie.

\uventry{provinc$'$}
province | province | Provinz | область, провинція | prowincya.

\uventry{proviz$'$}
pourvoir, garnir de | provide | versehen, versorgen | запасать | robić zapasy.

\uventry{prudent$'$}
prudent, raisonnable | prudent | verständig | благоразумный | rozsądny.

\uventry{prujn$'$}
gelée blanche, frimas | rime, hoar frost | Reif
(gefror. Thau) | иней | szron.

\uventry{prun$'$}
prune | plum | Pflaume | слива | śliwka.

\uventry{prunel$'$}
prunell | sloe | Dornschlehe | терновникъ | tarnośliwa.

\uventry{prunt$'$}
en prêt | lent, borrowed | leihen, borgen | взаймы | pożyczać.

\uventry{pruv$'$}
prouver | prove, demonstrate | beweisen | доказывать | dowodzić.

\uventry{publik$'$}
public | public | Publikum | публика | publika.

\uvsubentry{}\uventry{publik$'$a}
public | public, common | öffentlich | публичный | publiczny.

\uventry{pudel$'$}
barbet | spaniel | Pudel | пудель | pudel.

\uventry{pudr$'$}
poudre | powder | Puder | пудра | puder.

\uventry{pugn$'$}
poing | fist | Faust | кулакъ | kułak.

\uventry{pul$'$}
puce | flea | Floh | блоха | pchła.

\uventry{pulm$'$}
poumon | lung | Lunge | легкое | płuco.

\uventry{pulv$'$}
poudre à tirer | gunpowder | Pulver (Schiess-) | порохъ | proch.

\uventry{pulvor$'$}
poudre | powder | Pulver (zu Arznei u. drgl.) | порошокъ | proszek.

\uventry{pumik$'$}
pierre-ponce | pumice-stone | Bimstein | пемза | pumeks.

\uventry{pump$'$}
pomper | pump | pumpen | выкачивать насосомъ | pompować.

\uventry{pun$'$}
punir | punish | strafen | наказывать | karać.

\uventry{punc$'$}
ponceau | crimson red | ponceau | пунцовый | ponsowy.

\uventry{punĉ$'$}
punch | punch | Punsch | пуншъ | poncz.

\uventry{punkt$'$}
point | point | Punkt | точка, пунктъ | punkt, kropka.

\uvsubentry{}\uventry{punkt$'$o$'$kom$'$}
point et virgule | semicolon | Semikolon | точка съ запятою | średnik.


\uvsubentry{}\uventry{du$'$punkt$'$}
deux points | colon | Kolon | двоеточіе | dwukropek.

\uventry{punt$'$}
dentelle | lace | Spitzen | кружево | koronka.

\uventry{pup$'$}
poupée | doll | Puppe | кукла | lalka.

\uventry{pupil$'$}
pupille (de l’œil) | pupil | Pupille | зрачекъ | żrenica.

\uventry{pur$'$}
pur, propre | pure | rein | чистый | czysty.

\uventry{purpur$'$}
pourpre | purple | Purpur | пурпуръ | purpura.

\uventry{pus$'$}
pus | pus, matter | Eiter | гной | gnój, ropa, materya.

\uventry{puŝ$'$}
pousser (impulsion) | push | stossen | толкать | pchać.

\uventry{put$'$}
puits | well (subst.) | Brunnen | колодезь | studnia.

\uventry{putor$'$}
putois, furet | pole-cat, ferret | Iltiss | хорекъ | tchórz.

\uventry{putr$'$}
pourrir | rotten | faulen | гнить | gnić.

\uvlitero{R}

\uventry{rab$'$}
piller | rob | rauben, plündern | грабить | rabować, grabić.

\uventry{rabat$'$}
rabais, concession | rebate, discount | Rabatt | уступка,
скидка | rabat, ustępstwo.

\uventry{raben$'$}
rabbin | Jewish rabbi | Rabbiner | раввинъ | rabin.

\uventry{rabot$'$}
raboter | plane | hobeln | стругать | strugać, heblować.

\uventry{rad$'$}
roue | wheel | Rad | колесо | koło (od woza i t. p.).

\uventry{radi$'$}
rayon (de lumière, de roue) | beam, ray | Strahl | лучъ | promień.

\uventry{radik$'$}
racine | root | Wurzel | корень | korzeń.

\uventry{rafan$'$}
raifort | radish | Rettig | рѣдька | rzodkiew.

\uventry{rafin$'$}
raffiner | refine | raffiniren | рафинировать, изощрять | rafinować.

\uventry{rajd$'$}
aller à cheval | ride | reiten | ѣздить верхомъ | jeździć
konno.

\uventry{rajt$'$}
droit (le) | right, authority | Recht, Befugniss | право | prawo, racya, słuszność.

\uvsubentry{}\uventry{rajt$'$ig$'$}
donner plein pouvoir | empower | bevollmächtigen | уполномочивать | upoważnić, umocować, dać plenipotencyę.

\uventry{rakont$'$}
raconter | tell, relate | erzählen | разсказывать | opowiadać.

\uventry{ramp$'$}
ramper | crawl | kriechen | ползать | pełzać.

\uvsubentry{}\uventry{ramp$'$aĵ$'$}
reptile | reptile | Reptil | пресмыкающееся | gad.

\uventry{ran$'$}
grenouille | frog | Frosch | лягушка | żaba.

\uventry{ranc$'$}
rance | rancid | ranzig | прогорклый | jełki, przygorzki.

\uventry{rand$'$}
bord, extrémité | edge | Rand | край | brzeg.

\uventry{rang$'$}
rang, rangée, dignité | rank, dignity | Rang | рангъ | ranga.

\uventry{ranunkol$'$}
renoncule | ranunculus | Ranunkel | лютикъ | jaśkier.

\uventry{rap$'$}
rave | turnip | Rübe | рѣпа | rzepa.

\uventry{rapid$'$}
rapide, vite | quick, rapid | schnell | быстрый | prędki, bystry.

\uventry{rapir$'$}
fleuret | foil | Rappier | рапира | rapir.

\uventry{raport$'$}
rapporter | report | berichten, melden | доносить,
докладывать | meldować.

\uventry{rasp$'$}
râper | rasp | raspeln | терпужить | raszplować.

\uventry{rast$'$}
râteler | rake | harken | грести, скребать | grabić.

\uventry{rat$'$}
rat | rat | Ratte | крыса | szczur.

\uventry{raŭk$'$}
rauque, enroué | hoarse | heiser | хриплый | ochrypły.

\uventry{raŭp$'$}
chenille | caterpillar | Raupe | гусеница | gąsienica.

\uventry{rav$'$}
ravir, enchanter | ravish | entzücken | восхищать | zachwycić.

\uventry{raz$'$}
raser, faire la barbe | shave | rasiren | брить | golić.

\uventry{re$'$}
de nouveau, de retour; re-, ré- | again, back | wieder, zurück | снова, назадъ | znowu, napowrót.

\uventry{reciprok$'$}
mutuel, réciproque | mutual, reciprocal | gegenseitig | взаимный | wzajemny.

\uventry{redakci$'$}
rédaction | digesting, compiling | Redaktion | редакція | redakcya.

\uventry{redaktor$'$}
rédacteur | compiler, editor | Redakteur | редакторъ | redaktor.

\uventry{reg$'$}
gouverner, régir | rule, reign | regiren | править | rządzić.

\uvsubentry{}\uventry{reg$'$at$'$}
sujet | subject | Unterthan | подданный | poddany.

\uventry{reĝ$'$}
roi | king | König | король, царь | król.

\uventry{regal$'$}
régaler, traiter | entertain, regale | bewirthen | угощать | ugościć.

\uventry{regiment$'$}
regiment | regiment | Regiment | полкъ | półk.

\uventry{region$'$}
région, territoire | region, dominion | Gebiet | область | obręb, okolica.

\uventry{registr$'$}
registrer | register | registriren | регистрировать | rejestrować.

\uventry{regn$'$}
l’Etat | kingdom | Staat | государство | państwo.

\uventry{regol$'$}
roitelet | wren | Goldhähnchen | королекъ (птица) | złotnik,
królik.

\uventry{regul$'$}
règle (principe) | rule | Regel | правило | prawidło, regula.

\uventry{rekomend$'$}
recommander | recommend | empfehlen | рекомендовать | rekomendować, polecać.

\uventry{rekompenc$'$}
recompenser | reward | belohnen | награждать | wynagradzać.

\uventry{rekrut$'$}
recrue | recruit | Rekrut | рекрутъ | rekrut.

\uventry{rekt$'$}
droit, direct | straight | gerade | прямой | prosty.

\uvsubentry{}\uventry{mal$'$rekt$'$}
oblique | oblique, sloping | schief | косой | krzywy.

\uventry{rel$'$}
rail | rail | Schiene | рельса | szyna.

\uventry{religi$'$}
religion | religion | Religion | вѣра, религія | religia.

\uventry{rem$'$}
ramer | row (vb.) | rudern | грести (веслами) | wiosłować.

\uventry{rembur$'$}
rembourer, matelasser | quilt | polstern | набивать (мебель) | wyściełać.

\uventry{rempar$'$}
rempart | rampart | Wall | валъ, окопъ | wał.

\uventry{ren$'$}
rein | kidney | Niere | почка | nerka.

\uventry{renkont$'$}
rencontrer | meet | begegnen | встрѣчать | spotykać.

\uventry{rent$'$}
rente, revenu | rent | Rente | рента, доходъ | renta, dochód.

\uventry{renvers$'$}
renverser | upset | umwerfen, umstürzen | опрокидывать | przewracać.

\uventry{respekt$'$}
respect | respect | Respekt | почтеніе | uszanowanie,
respekt.

\uventry{respond$'$}
répondre | reply | antworten | отвѣчать | odpowiadać.

\uvsubentry{}\uventry{respond$'$ec$'$}
responsable | responsal | Verantwortlichkeit | отвѣтственность | odpowiedzialność.

\uventry{rest$'$}
rester | remain | bleiben | оставаться | pozostawać.

\uventry{restoraci$'$}
restaurant, auberge | eating-house | Speisehaus | ресторація | restauracya.

\uventry{ret$'$}
filet (de mailles) | net | Netz | сѣть | sieć, siatka.

\uventry{rev$'$}
rêver, imaginer | fancy | träumen, schwärmen | мечтать | marzyć.

\uvsubentry{}\uventry{dis$'$rev$'$iĝ$'$}
désenchantement | disenchanting | Enttäuschung | разочаровываться | rozczarować się.

\uventry{rezerv$'$}
réserver | reserve | vorbehalten | сохранить на послѣ | zachować, zarezerwować.

\uventry{rezin$'$}
résine | resin | Harz | смола | źywica.

\uventry{rib$'$}
grosille | currant | Johannisbeere | смородина | porzeczka.

\uventry{ribel$'$}
se révolter | revolt, rebel | aufstehen, sich empören | возставать | powstawać, rokoszować.

\uventry{riĉ$'$}
riche | rich | reich | богатый | bogaty.

\uventry{ricev$'$}
recevoir, obtenir | obtain, get, receive | bekommen | получать | otrzymać.

\uventry{rid$'$}
rire | laugh | lachen | смѣяться | śmiać się.

\uventry{rif$'$}
banc | reef, bank | Riff | рифъ | rafa, skała podwodna.

\uventry{rifuĝ$'$}
se réfugier | refuge | Zuflucht nehmen | искать убѣжища | szukać schronienia.

\uventry{rifuz$'$}
refuser | refuse | verweigern, abschlagen, abdanken | отказывать | odmanwiać.

\uvsubentry{}\uventry{rifuz$'$iĝ$'$}
renoncer | renounce, resign | verzichten | отказываться | wymówić się.

\uventry{rigard$'$}
regarder | behold, look at | schauen | смотрѣть | patrzeć.

\uventry{rigid$'$}
roide, rigide | stiff, rigid | starr | окоченѣлый | stęźały,
zdrętwiały.

\uventry{rigl$'$}
verrouiller | bolt | verriegeln | запирать засовомъ | ryglować.

\uvsubentry{}\uventry{rigl$'$il$'$}
verrou | bolt | Riegel | засовъ | rygiel.

\uventry{rikolt$'$}
récolter, moissoner | reap | ernten, schneiden | жать,
жинать | źąć, sprzątać.

\uvsubentry{}\uventry{rikolt$'$il$'$}
faux, faucille | sickle | Sichel | серпъ | sierp.

\uventry{rilat$'$}
concerner; avoir raport à | be related to | sich beziehen | относиться | odnosić się, tyczeć się.

\uventry{rim$'$}
rime | rhyme | Reim | риѳма | rym.

\uventry{rimark$'$}
remarquer | remark | merken, bemerken | замѣчать | zauwaźać,
postrzegać.

\uventry{rimed$'$}
moyen, expédient | means, remedy | Mittel | средство | środek.

\uventry{rimen$'$}
courroie, lanière | strap | Riemen | ремень | rzemień.

\uventry{ring$'$}
anneau | ring (subst.) | Ring | кольцо | pierścień.

\uvsubentry{}\uventry{ring$'$eg$'$}
cerceau | hoop, circle | Reif (am Fasse) | обручъ | obręcz.

\uventry{rinocer$'$}
rhinocéros | rhinoceros | Nashorn | носорогъ | nosoroźec.

\uventry{rip$'$}
rive, côte | rib | Rippe | ребро | źebro.

\uventry{ripet$'$}
répéter | repeat | wiederholen | повторять | powtarzać.

\uventry{ripoz$'$}
reposer (se reposer) | repose | ruhen | отдыхать | odpoczywać.

\uventry{riproĉ$'$}
reprocher | reproach | vorwerfen | упрекать | zarzucać.

\uventry{risk$'$}
risquer | risk | wagen | рисковать | ryzykować.

\uventry{risort$'$}
ressort | spring | Triebfeder | пружина | sprężyna.

\uventry{river$'$}
rivière, fleuve | river | Fluss | рѣка | rzeka.

\uventry{riverenc$'$}
révérence | reverence | Knicks | присѣдать, дѣлать реверансъ | dygać.

\uventry{riz$'$}
riz | rice | Reis | рисъ | ryź.

\uventry{rod$'$}
rade | road | Rhede | рейдъ | stanowisko okrętów.

\uventry{romp$'$}
rompre, casser | break | brechen | ломать | łamać.

\uventry{rond$'$}
rond, cercle | round, circle | Kreis | кругъ | koło.

\uventry{ronk$'$}
ronfler | snore | schnarchen | храпѣть | chrapać.

\uventry{ros$'$}
rosée | dew | Thau | роса | rosa.

\uventry{rosmar$'$}
morse | waltron, morse | Wallross | моржъ | mors.

\uventry{rosmaren$'$}
romarin | rosemary | Rosmarin | розмаринъ | roźmaryn.

\uventry{rost$'$}
rôtir | roast | braten | жарить | piec, smażyć.

\uventry{rostr$'$}
trompe | trunk | Rüssel | хоботъ | ryj, trąba (słonia).

\uventry{rot$'$}
compagnie (de soldats) | troop | Rotte | рота | rota.

\uventry{roz$'$}
rose | rose | Rose | роза | roźa.

\uventry{rozari$'$}
rosaire | rosary | Rosenkranz | четки | różaniec, paciorki.

\uventry{rub$'$}
décombres, gravois | rubbish | Schutt | мусоръ | gruz.

\uventry{ruband$'$}
ruban, cordon | ribbon | Band (das) | лента | wstąźka.

\uventry{ruben$'$}
rubis | ruby | Rubin | рубинъ | rubin.

\uventry{rubrik$'$}
rubrique | rubric | Rubrik | рубрика | rubryka.

\uventry{ruĝ$'$}
rouge | red | roth | красный | czerwony.

\uventry{ruin$'$}
ruines | ruins, wrack | Ruine | развалины | rozwaliny, zwaliska.

\uvsubentry{}\uventry{ruin$'$ig$'$}
ruiner | ruin | ruiniren | разорять | zniszczyć, rujnować.

\uventry{rukt$'$}
renvoi de l’estomac, rot | ructation | Aufstossen | отрыжка | odbijanie.

\uventry{rul$'$}
rouler | roll | wälzen, rollen | катать | toczyć.

\uvsubentry{}\uventry{rul$'$o}
rouleau, cylindre | roller, cylinder | Walze | валикъ | walec.

\uventry{rum$'$}
rhum | rum | Rum | ромъ | arak.

\uventry{rust$'$}
rouille | rust | rostig | ржавчина | rdza.

\uventry{ruz$'$}
rusé, astucieux | trick, ruse | listig | хитрый | chytry.

\uvlitero{S, Ŝ}

\uventry{sabat$'$}
samedi | Saturday | Sonnabend | суббота | sobota.

\uventry{sabl$'$}
sable | sand | Sand | песокъ | piasek.

\uvsubentry{}\uventry{sabl$'$aĵ$'$}
banc de sable | flat | Sandbank | мель | mielizna.

\uventry{ŝaf$'$}
bélier, mouton | sheep | Schaf | баранъ | owca.

\uventry{safir$'$}
saphir | saphire | Saphir | сапфиръ | szalir.

\uventry{safran$'$}
safran | saffron | Safran | шафранъ | szafran.

\uventry{sag$'$}
flèche | arrow | Pfeil | стрѣла | strzała.

\uventry{saĝ$'$}
sage, sensé | wise | klug, vernünftig | умный | mądry.

\uventry{sagac$'$}
subtil; argutieux | subtle, crafty, sharp | spitzfindig | замысловатый | przebiegły.

\uventry{ŝajn$'$}
sembler | seem | scheinen | казаться | wydawać się.

\uventry{sak$'$}
sac | sack | Sack | мѣшокъ | worek.

\uventry{ŝak$'$}
échecs (jeu) | chess | Schachspiel | шахматы | szachy.

\uventry{ŝakal$'$}
chacal | jackal | Schakal | шакалъ | szakal.

\uventry{sakr$'$}
épine du dos | back-bone | Kreuzbein | крестецъ | kość krzyżowa.

\uventry{sal$'$}
sel | salt | Salz | соль | sól.

\uventry{ŝal$'$}
chàle | shawl | Shawl | шаль | szal.

\uventry{salajr$'$}
salaire, appointements | wages, salary | Gehalt, Gage | жалованье | pensya.

\uventry{salamandr$'$}
salamandre | salamander | Molch | саламандра | salamandra.

\uventry{sal$'$amoniak$'$}
sal ammoniac | sal ammoniac | Salmiak | нашатырь | salmiak.

\uventry{salat$'$}
salade | salad, sallet | Salat | салатъ | sałata.

\uventry{salik$'$}
saule | willow | Weidebaum | верба | wierzba.

\uventry{salm$'$}
saumon | salmon | Lachs | лосось, семга | łosoś.

\uventry{ŝalm$'$}
chalumeau | shalm | Schalmei | свирѣль | świstawka.

\uventry{salon$'$}
salun | saloon | Salon | залъ | salon.

\uventry{salpetr$'$}
salpêtre | saltpetre | Salpeter | селитра | saletra.

\uventry{salt$'$}
sauter, bondir | leap, jump | springen | прыгать | skakać.

\uventry{salut$'$}
saluer | salute, greet | grüssen | кланяться | kłaniać się.

\uventry{salvi$'$}
sauge | sage | Salvei, Salbei | шалфей | szałwia.

\uventry{sam$'$}
même (qui n’est pas autre) | same | selb, selbst
(z. B. derselbe, daselbst) | же, самый (напр. тамъ же, тотъ самый) | źe, sam (np. tamźe, ten sam).

\uventry{ŝam$'$}
peau de chamois | shamoy-leather | sämisches Leder | замша | zamsza.

\uventry{sambuk$'$}
sureau | elder | Hollunder | бузина | bez.

\uventry{san$'$}
sain, en santé | healthy | gesund | здоровый | zdrowy.

\uventry{ŝancel$'$}
chanceler | totter | bewegen, wankend machen | колебать | chwiać, powiewać.

\uventry{sang$'$}
sang | blood | Blut | кровь | krew.

\uventry{ŝanĝ$'$}
changer | change | tauschen, wechseln | мѣнять | zmieniać.

\uventry{sankt$'$}
saint | holy | heilig | святой, священный | święty.

\uvsubentry{}\uventry{sankt$'$ej$'$}
temple | temple | Tempel | храмъ | świątynia.

\uventry{sap$'$}
savon | soap | Seife | мыло | mydło.

\uventry{sardel$'$}
sardine | sardel | Sardelle | сардель, анчоусъ | sardela.

\uventry{ŝarg$'$}
charger (une arme à feu) | load (a gun, etc.) | laden (eine Flinte etc.) | заряжать (ружье) | nabijać (broń).

\uventry{ŝarĝ$'$}
charger | charge, load | laden, aufladen, belasten | нагружать, обременять | obciąźać, naładować.

\uventry{sark$'$}
sarcler | weed | gäten | полоть | pielić.

\uventry{ŝark$'$}
requin | shark, sea-dog | Haifisch | акула | haja, wilk
morski.

\uventry{sat$'$}
rassasié | satiated | satt | сытый | syty.

\uventry{ŝat$'$}
estimer | esteem | viel halten, grossen Werth legen | дорожить | cenić, oceniać, szacować.

\uvsubentry{}\uventry{mal$'$ŝat$'$}
négliger | neglect | gering schätzen, vernachlässigen | пренебрегать | zapoznawać.

\uventry{satur$'$}
rassasier, assouvir | satiate, saturate | sättigen, tränken | насыщать | nasycać.

\uventry{saŭc$'$}
sauce | sauce | Brühe, Sauce | соусъ | sos.

\uventry{ŝaŭm$'$}
écume | foam | Schaum | пѣна | piana.

\uventry{sav$'$}
sauver | save | retten | спасать | ratować.

\uventry{sceptr$'$}
sceptre | sceptre | Scepter | скипетръ | berło.

\uventry{sci$'$}
savoir | know | wissen | знать, вѣдать | wiedzieć.

\uvsubentry{}\uventry{sci$'$ig$'$}
avertir, annoncer | announce, give notice | benachrichtigen | увѣдомлять | uwiadomić.


\uvsubentry{}\uventry{sci$'$iĝ$'$}
apprendre | perceive | erfahren | узнавать | dowiedzieć się.

\uventry{scienc$'$}
science | science | Wissenschaft | наука | nauka, wiedza.

\uventry{sciur$'$}
écureuil | squirrel | Eichhorn | бѣлка | wiewiórka.

\uventry{se}
si | if | wenn | если | jeźeli.

\uventry{seb$'$}
suif | tallow | Schmalz, Talg | сало | smalec, łój.

\uventry{sed}
mais | but | aber, sondern | но | lecz.

\uventry{seg$'$}
scier | saw | sägen | пилить | piłować.

\uventry{seĝ$'$}
chaise | seat | Stuhl | стулъ | krzesło.

\uventry{sek$'$}
sec | dry | trocken | сухой | suchy.

\uventry{sekal$'$}
seigle | rye | Roggen | рожь | źyto.

\uventry{sekc$'$}
disséquer | dissect | seciren | вскрывать (трупъ) | rozkrawać,
rozczłonkować.

\uventry{sekret$'$}
secret | secret | Geheimniss | тайна | tajemnica.

\uventry{sekretari$'$}
secrétaire | secretary | Sekretär | секретарь | sekretarz.

\uventry{seks$'$}
sexe | sex | Geschlecht (männlich oder weiblich) | полъ
(мужской или женскій) | płeć.

\uventry{sekund$'$}
seconde | second | Sekunde | секунда | sekunda.

\uventry{sekv$'$}
suivre | follow | folgen | слѣдовать | nastąpić.

\uventry{sel$'$}
selle | saddle | Sattel | сѣдло | siodło.

\uvsubentry{}\uventry{sel$'$i}
seller | saddle | satteln | сѣдлать | osiodłać.

\uventry{ŝel$'$}
écorce, coque | shell | Schale, Rinde | скорлупа, кора | skorupa.

\uvsubentry{}\uventry{sen$'$ŝel$'$ig$'$}
écorcer, peler | shell, peel | schälen, abschälen | облуплять | obłupiać, obierać.

\uventry{selakt$'$}
petit-lait | whey | Molken | сыворотка | serwatka.

\uventry{ŝelk$'$}
bretelle | breeches-bearer | Hosenträger | подтяжки | szelki.

\uventry{sem$'$}
semer | sow | säen | сѣять | siać.

\uvsubentry{}\uventry{sem$'$o}
semence | seed | Samen | сѣмя | nasienie.

\uventry{semajn$'$}
semaine | week | Woche | недѣля | tydzień.

\uventry{sen}
sans | without | ohne | безъ | bez.

\uventry{senc$'$}
sens, acception | sense | Sinn | смыслъ | znaczenie, sens.

\uventry{send$'$}
envoyer | send | senden, schicken | посылать | posyłać.

\uventry{sent$'$}
ressentir, éprouver (une impression) | feel, perceive | fühlen | чувствовать | czuć.

\uventry{sentenc$'$}
sentence | sentence | Sentenz | изреченіе | sentencya, orzeczenie.

\uventry{sep}
sept | seven | sieben | семь | siedm.

\uventry{sepi$'$}
seiche (poisson) | cuttle-fish | Tintenwurm | каракатица | sepa, pław morski.

\uventry{Septembr$'$}
Septembre | September | September | Сентябрь | Wrzesień.

\uventry{serĉ$'$}
chercher | search | suchen | искать | szukać.

\uventry{ŝerc$'$}
plaisanter | joke | scherzen | шутить | źartować.

\uventry{serĝent$'$}
sergent | sergeant | Sergeant | сержантъ | sierźant.

\uventry{seri$'$}
série | series | Reihe | рядъ, серія | serya.

\uventry{serioz$'$}
sérieux | serious | ernst | серіозный | waźny, powaźny, na seryo.

\uventry{serpent$'$}
serpent | serpent | Schlange | змѣя | wąź.

\uventry{serur$'$}
serrure | lock (subst.) | Schloss (zum Schliessen) | замокъ | zamek (od drzwi).

\uventry{serv$'$}
servir (quelqu’un) | serve | dienen | служить | słuźyć.

\uventry{servic$'$}
service | set of dishes, plates, etc. | Service | сервизъ | serwis.

\uventry{servut$'$}
corvée | soccage | Frohne | барщина | pańszczyzna.

\uventry{ses}
six | six | sechs | шесть | sześć.

\uventry{sever$'$}
sévère | severe | streng | строгій | surowy, ostry, srogi.

\uventry{sezon$'$}
saison | season | Jahrzeit, Zeit | сезонъ | sezon.

\uventry{si}
soi, se | one’s self | sich | себя | siebie.

\uvsubentry{}\uventry{si$'$a}
son, sa | one’s | sein | свой | swój.

\uventry{ŝi}
elle | she | sie (Einzahl) | она | ona.

\uvsubentry{}\uventry{ŝi$'$a}
son, sa | her | ihr | ея | jej.

\uventry{sibl$'$}
siffler, frémir | hiss, whizz | zischen | шипѣть (о
произношеніи) | sykać.

\uventry{sid$'$}
être assis, siéger | sit | sitzen | сидѣть | siedzieć.

\uvsubentry{}\uventry{kun$'$sid$'$}
séance, session | session | Sitzung | засѣданіе | posiedzenie, sesya.

\uventry{sieĝ$'$}
assiéger | besiege | belagern | осаждать | oblegać.

\uventry{sigel$'$}
sceller | seal (vb.) | siegeln | класть печать | pieczętować.

\uvsubentry{}\uventry{sigel$'$vaks$'$}
cire à cacheter | sealing-wax | Siegellack | сургучъ | lak.

\uventry{sign$'$}
signe, marque | sign, token | Zeichen | знакъ | znak.

\uvsubentry{}\uventry{post$'$sign$'$}
trace, vestige | trace, vestige, footstep | Spur | слѣдъ | ślad.

\uventry{signal$'$}
signal | signal | Signal | сигналъ | sygnał.

\uventry{signif$'$}
signifier | signify, mean | bezeichnen, bedeuten | означать | oznaczać.

\uventry{silab$'$}
syllabe | syllable | Sylbe | слогъ | sylaba, zgłoska.

\uvsubentry{}\uventry{silab$'$i}
épeler | to spell | buchstabiren | читать по слогамъ | sylabizować.

\uventry{ŝild$'$}
bouclier | shield | Schild | щитъ | puklerz, tarcza.

\uventry{silent$'$}
se taire | silent | schweigen | молчать | milczeć.

\uventry{silik$'$}
silex, caillou | flint | Kieselstein | кремень | krzemień.

\uventry{silk$'$}
soie | silk | Seide | шелкъ | jedwab.

\uventry{silur$'$}
glanis | shad-fish | Wels | сомъ | sum.

\uventry{silvi$'$}
fauvette | hedge-sparrow, linget | Grasmücke | малиновка | piegźa, gajówka.

\uventry{ŝim$'$}
se moisir | mould | schimmeln | плѣснѣть | pleśnieć.

\uventry{simi$'$}
singe | monkey | Affe | обезьяна | małpa.

\uventry{simil$'$}
semblable | like, similar | ähnlich | похожій | podobny.

\uventry{simpl$'$}
simple | simple | einfach | простой | prosty, zwyczajny.

\uventry{sincer$'$}
sincère | sincere | aufrichtig | чистосердечный | szczery.

\uventry{ŝind$'$}
bardeau, échandole | shingle | Schindel | гонтъ | gont.

\uventry{singult$'$}
avoir le hoquet | hiccough | schlucksen | икота | czkawka.

\uventry{sinjor$'$}
monsieur | Sir, Mr | Herr | господинъ | pan.

\uventry{ŝink$'$}
jambon | bacon | Schinken | ветчина | wędlina.

\uventry{ŝip$'$}
navire | ship | Schiff | корабль | okręt.

\uventry{ŝir$'$}
déchirer | tear, rend | reissen | рвать | rwać.

\uventry{siring$'$}
lilas | lilac | Flieder | сирень | bez.

\uventry{ŝirm$'$}
couvrir, protéger | protekt | beschirmen | заслонять | zasłaniać.

\uventry{sirop$'$}
sirop | syrup | Syrop | сиропъ | ulepek, syróp.

\uventry{sitel$'$}
seau | bucket | Eimer | ведро | wiadro, ceber.

\uventry{situaci$'$}
situation, position | situation | Lage | положеніе | sytuacya, połoźenie.

\uventry{skabi$'$}
gale, déchet | scab, wastings | Krätze | чесотка | świerzbiaczka.

\uventry{skadr$'$}
escadron | squadron | Eskadron | эскадронъ | szwadron.

\uventry{skal$'$}
échelle | scale | Massstab | масштабъ | miara (podziałka na mapie).

\uventry{skapol$'$}
omoplate | omoplate, shoulderplate | Schulterbein | лопатка | łopatka.

\uventry{skarab$'$}
scarabée | beetle | Käfer | жукъ | chrząszcz.

\uventry{skarlat$'$}
écarlate | scarlet | Scharlach | скарлатина | szkarlatyna,
płonica.

\uventry{skarp$'$}
écharpe | scarf | Schärpe | шарфъ | szarfa.

\uventry{skatol$'$}
boîte | small box, case | Büchse, Schachtel | коробка | pudełko.

\uventry{skerm$'$}
escrimer | fight, fence | fechten | фехтовать | fechtować
się.

\uventry{skiz$'$}
esquisser | sketch | skizziren | очертать, очеркъ | szkicować.

\uventry{sklav$'$}
esclave | slave | Knecht | рабъ | niewolnik.

\uventry{skolop$'$}
bécasse | snipe, wood-cock | Schnepfe | куликъ, бекасъ | bekas.

\uventry{skorbut$'$}
scorbut | scurvy | Scorbut | цынга | szkorbut.

\uventry{skorpi$'$}
scorpion | scorpion | Scorpion | скорпіонъ | niedźwiadek.

\uventry{skrap$'$}
râcler, ratisser | shave, scrape | schaben | скоблить | skrobać.

\uventry{skrib$'$}
écrire | write | schreiben | писать | pisać.

\uventry{skrofol$'$}
scrofules | scrofula | Scropheln | золотуха | skrofuły,
zołzy.

\uventry{sku$'$}
secouer | shake | schütteln | трясти | trząść.

\uventry{skulpt$'$}
sculpter | sculpture | aushauen, schnitzen | ваять | rzeźbić.

\uvsubentry{}\uventry{skulpt$'$il$'$}
ciseau | chisel | Meissel | рѣзецъ | dłóto.

\uventry{skurĝ$'$}
fouet | scourge | Geissel, Plette | нагайка | bicz, nahajka.

\uventry{skvam$'$}
écaille (de poisson) | scale (fish etc.) | Schuppe | чешуя | łuska.

\uventry{ŝlim$'$}
limon, bourbe | slime, mire, mud | Schlamm | илъ, тина | muł, szlam.

\uventry{ŝlos$'$}
fermer à clef | lock, fasten | schliessen | запирать на ключъ | zamykać na klucz.

\uvsubentry{}\uventry{ŝlos$'$il$'$}
clef | key | Schlüssel | ключъ | klucz.

\uventry{ŝmac$'$}
baiser | kiss | schmatzen | чмокать | czmokać.

\uventry{smerald$'$}
émeraude | emerald | Smaragd | смарагдъ | szmaragd.

\uventry{ŝmir$'$}
oindre, graisser | smear | schmieren | мазать | smarować.

\uventry{ŝnur$'$}
corde | string | Strick | веревка | sznur, powróz.

\uventry{sobr$'$}
sobre | sober | nüchtern | трезвый | trzeźwy.

\uvsubentry{}\uventry{mal$'$sobr$'$}
enivré | drunk | trunken, betrunken | пьяный | pijany.

\uventry{societ$'$}
société | society | Gesellschaft | общество | społeczeństwo,
towarzystwo.

\uventry{sof$'$}
sofa | sofa | Sofa | софа | sofa.

\uventry{soif$'$}
avoir soif | thirst | dursten | жаждать | pragnąć, doznawać
pragnienia.

\uventry{sojl$'$}
seuil | threshold | Schwelle | порогь | próg.

\uventry{sol$'$}
seul | only, alone | einzig, allein | единственный | jedyny.

\uventry{soldat$'$}
soldat | soldier | Soldat | солдатъ | żolnierz, sołdat.

\uventry{solen$'$}
solennel | solemn | feierlich | торжественный | uroczysty, solenny.

\uventry{solv$'$}
résoudre | loosen, dissolve | auflösen | рѣшать, разрѣшать | rozwiązać.

\uventry{somer$'$}
été | summer | Sommer | лѣто | lato.

\uventry{son$'$}
sonner, rendre des sons, résonner | sound | tönen, lauten | звучать | brzmieć.

\uventry{sonĝ$'$}
songe | dream | träumen | видѣть во снѣ | śnić.

\uventry{sonor$'$}
tinter | give out a sound (as a bell) | klingen | звенѣть | brzęczeć.

\uventry{son$'$serpent$'$}
serpent à sonnettes | rattle-snake | Klapperschlange | гремучая змѣя | grzechotnik.

\uventry{sopir$'$}
soupirer après | fain, long for | sich sehnen | тосковать | tęsknić.

\uventry{sopran$'$}
dessus (musique), discant | descant | Diskant | дискантъ | sopran, dyszkant.

\uventry{sorb$'$}
humer | sip | schlürfen | хлебать | chlipać.

\uventry{sorĉ$'$}
pratiquer la magie, la sorcellerie | witchcraft | zaubern | колдовать | czarować.

\uventry{sorik$'$}
musaraigne | shrew-mouse | Spitzmaus | землеройка | kretomysz.

\uventry{sorp$'$}
sorbe | sorb, service | Ebereschenbeere | рябина | jarzębina.

\uventry{sort$'$}
sort, destinée | fate, lot | Schicksal | судьба | los.

\uventry{ŝov$'$}
pousser, mener | shove | schieben | совать | suwać.

\uventry{sovaĝ$'$}
sauvage | wild, savage | wild | дикій | dziki.

\uventry{ŝovel$'$}
enlever avec une pelle | showel | schaufeln | сгребать
(лопатой) | szuflować, kopać (łopatą).

\uventry{spac$'$}
espace | room, space | Raum | пространство | przestrzeń.

\uventry{spalir$'$}
espalier | espalier | Spalier | шпалеръ | szpaler.

\uventry{ŝpar$'$}
ménager, épargner | be sparing | sparen | сберегать | oszczędzać.

\uventry{spat$'$}
éparvin, spath | spavin, spar | Spath | шпатъ | spat.

\uventry{spec$'$}
espèce | kind, species | Art, Gattung | родъ, сортъ | rodzaj, gatunek.

\uventry{spegul$'$}
miroir | looking-glass | Spiegel | зеркало | zwierciadło.

\uventry{spert$'$}
expérimenté | expert | erfahren, bewandert | опытный | doświadczony.

\uventry{spez$'$}
virement | spend | Umsatz | оборотъ | obrót.

\uvsubentry{}\uventry{el$'$spez$'$}
dépenser | spend (expenses) | verausgaben | расходовать | wydawać.


\uvsubentry{}\uventry{en$'$spez$'$}
avoir des revenus | have revenues | lösen (Geld), Einkünfte haben | имѣть доходъ | mieć dochód.

\uventry{spic$'$}
épice | spice | Gewürz | пряность | przyprawa, korzenie.

\uventry{spik$'$}
épi | ear head (of corn, etc.) | Aehre | колосъ | kłos.

\uventry{spin$'$}
épine du dos, échine | spine, back-bone | Rückgrat | хребетъ | grzbiet.

\uventry{ŝpin$'$}
filer | spin | spinnen | прясть | prząść.

\uventry{spinac$'$}
épinards | spinach | Spinat | шпинатъ | szpinak.

\uventry{spion$'$}
espion | spy | Spion | шпіонъ | szpieg.

\uventry{spir$'$}
respirer | breathe | athmen | дышать | oddychać.

\uventry{spirit$'$}
esprit | spirit | Geist | духъ | duch.

\uventry{spit$'$}
en dépit de | in spite of | zum Trotz | на перекоръ | na przekór.

\uventry{spong$'$}
éponge | sponge | Schwamm | губка | gąbka.

\uventry{sprit$'$}
spirituel, ingénieux | wit | witzig | остроумный | dowcipny.

\uventry{spron$'$}
éperon | spur | Sporn | шпора | ostroga.

\uventry{ŝpruc$'$}
jaillir | sprinkle | spritzen | брызгать | pryskać.

\uventry{sput$'$}
cracher | spit | ausspeien, auswerfen | мокрота | plwocina.

\uventry{ŝrank$'$}
armoire | cupboard | Schrank | шкафъ | szafa.

\uventry{ŝraŭb$'$}
vis | screw | Schraube | винтъ | szruba.

\uventry{stab$'$}
état-major | staff-officers | Stab (milit.) | штабъ | sztab.

\uventry{stabl$'$}
tréteau | trestle | Gestell | станокъ | podstawa, osada.

\uventry{staci$'$}
station | dépôt (railroad) | Station | станція | stacya.

\uvsubentry{}\uventry{staci$'$dom$'$}
embarcadère | station, terminus | Bahnhof | вокзалъ | banhof, dworzec, foksal.

\uventry{stal$'$}
étable | stable | Stall, Schoppen | стойло, сарай | stajnia.

\uventry{ŝtal$'$}
acier | steet | Stahl | сталь | stal.

\uventry{stamp$'$}
estampille, timbre | stamp, mark | stempeln | класть штемпель | stemplować.

\uventry{stan$'$}
étain | tin | Zinn | олово | cyna.

\uvsubentry{}\uventry{stan$'$i}
étamer | tin | verzinnen | лудить | pobielać.

\uventry{standard$'$}
drapeau, étendard | flag | Fahne | знамя | chorągiew, znamię.

\uventry{stang$'$}
perche (bois) | pole | Stange | шестъ | drąg.

\uventry{star$'$}
être debout | stand | stehen | стоять | stać.

\uventry{stat$'$}
état (manière d’être) | state, condition | Stand, Zustand | состояніе | stan.

\uventry{ŝtat$'$}
État | State | Staat | штатъ | stan (państwo).

\uventry{steb$'$}
piquer | quilt | steppen | строчить | pikować, cerować.

\uventry{stel$'$}
étoile | star | Stern | звѣзда | gwiazda.

\uventry{ŝtel$'$}
voler, dérober | steal | stehlen | красть | kraść.

\uventry{step$'$}
lande, step | heath, desert | Steppe | степь | step.

\uventry{sterk$'$}
fumier, engrais | dung, manure | Mist | навозъ | gnój.

\uventry{sterled$'$}
sterlet (poiss.) | sterlet (fish) | Sterläd | стерлядь | czeczuga.

\uventry{stern$'$}
étendre, coucher | lay on the ground | betten | стлать | słać.

\uventry{stertor$'$}
râler | rattle | röcheln | хрипѣть (въ груди) | rzężeć.

\uventry{stil$'$}
style | style | Stil | стиль, слогъ | styl.

\uventry{stip$'$}
genêt | broom | Pfriemgrass | ковыль | trawa piórowa, narduszek.

\uventry{ŝtip$'$}
bloc, billot | block, log | Klotz | колода, чурбанъ | kloc,
pniak.

\uventry{stof$'$}
stofe (mesure) | stofe (measure) | Stof (Hohlmass) | штофъ | sztof (miara).

\uventry{ŝtof$'$}
étoffe | stuff, matter, goods | Stoff | вещество, матерія | sztof (miara).

\uventry{stomak$'$}
estomac | stomach | Magen | желудокъ | żołądek.

\uventry{ŝton$'$}
pierre | stone | Stein | камень | kamień.

\uventry{ŝtop$'$}
boucher | stop, fasten down | stopfen | затыкать | zatykać.

\uventry{strab$'$}
loucher | squint | schielen | косить (глазами) | zezem
patrzeć.

\uventry{strang$'$}
étrange, bizarre | strange | sonderbar | странный | dziwny,
dziwaczny.

\uventry{strat$'$}
rue | street | Strasse | улица | ulica.

\uventry{streĉ$'$}
tendre, tirer | bend, strain, stretch out | spannen,
anstrengen | напрягать | wyprężyć, wytężyć.

\uventry{strek$'$}
rayer, biffer | streak, line | streichen | черкать | kreślić.

\uvsubentry{}\uventry{strek$'$o}
trait | streak, stroke | Strich | черта | kreska.

\uventry{stri$'$}
bande, raie | stripe, streak | Streifen | полоса | pas, pręga.

\uventry{strig$'$}
hibou | owl | Eule | сова | sowa.

\uventry{strik$'$}
grève | strike | Strike | стачка (работниковъ) | strejk.

\uventry{ŝtrump$'$}
bas (vêtement) | stocking | Strumpf | чулокъ | pończocha.

\uventry{strut$'$}
autruche | ostrich | Strauss (Vogel) | страусъ | struś.

\uventry{student$'$}
étudiant | student | Student | студентъ | student.

\uventry{stuk$'$}
couvrir de stuc, crépir | parget | stuckaturen | штукатурить | sztukaterya.

\uventry{stup$'$}
étoupe | tow | Hede | пакля | pacześ.

\uventry{ŝtup$'$}
marche, échelon | step | Stufe | ступень | stopień.

\uvsubentry{}\uventry{ŝtup$'$ar$'$}
escalier, échelle | staircase | Treppe, Leiter | лѣстница | schody, drabina.

\uventry{sturg$'$}
esturgeon | sturgeon | Stör | осетръ | jesiotr.

\uventry{sturn$'$}
étourneau, sansonnet | starling | Star (Vogel) | скворецъ | szpak.

\uventry{ŝu$'$}
soulier | shoe | Schuh | башмакъ | trzewik.

\uventry{sub}
sous | under, beneath, below | unter | подъ | pod.

\uventry{subit$'$}
subit, soudain | sudden | plötzlich | внезапный | nagły.

\uventry{subjekt$'$}
sujet | subject | Subject | подлежащее | podmiot.

\uventry{sublimat$'$}
mercure sublimé | sublimatum | Sublimat | сулема | sublimat.

\uventry{substantiv$'$}
substantif | noun, substantive | Hauptwort | существительное | rzeczownik.

\uventry{suĉ$'$}
sucer | suck | saugen | сосать | ssać.

\uventry{sud$'$}
sud | south | Süden | югъ | południe.

\uventry{sufer$'$}
souffrir, endurer | suffer | leiden | страдать | cierpieć.

\uventry{sufiĉ$'$}
suffisant | sufficient | genug | довольно, достаточно | dosyć, dostatecznie.

\uventry{sufok$'$}
suffoquer, étouffer | suffocate | ersticken (act.) | душить | dusić, zadusić.

\uventry{suk$'$}
jus, suc | sap, juice | Saft | сокъ | sok.

\uventry{sukcen$'$}
succin, ambre joune | amber | Bernstein | янтарь | bursztyn.

\uventry{sukces$'$}
avoir du succès | success | Erfolg haben | имѣть успѣхъ | mieć powodzenie, sukces.

\uventry{suker$'$}
sucre | sugar | Zucker | сахаръ | cukier.

\uventry{ŝuld$'$}
devoir (dette) | owe | schulden | быть должнымъ | być dłużnym.

\uventry{sulfur$'$}
soufre | sulphur | Schwefel | сѣра | siara.

\uventry{sulk$'$}
sillon | furrow | Furche, Runzel | борозда | brózda.

\uventry{ŝultr$'$}
épaule | schoulder | Schulter | плечо | ramię.

\uventry{sum$'$}
somme | sum | Summe | сумма | summa.

\uvsubentry{}\uventry{re$'$sum$'$}
résumer | resume | resumiren | подводить итогъ | sumować.

\uventry{sun$'$}
soleil | sun | Sonne | солнце | słońce.

\uvsubentry{}\uventry{sun$'$flor$'$}
tournesol | girasol, turnsol | Sonnenblume | подсолнечникъ | słonecznik.

\uventry{sup$'$}
soupe, potage | soup | Suppe | супъ | zupa.

\uventry{super}
au-dessus de, sur (sans toucher) | over, above | über,
oberhalb | надъ | nad.

\uvsubentry{}\uventry{super$'$i}
surpasser | surpass, excel | übertreffen | превосходить | przewyższać.


\uvsubentry{}\uventry{super$'$akv$'$}
inonder, submerger | overflow, deluge | überschwemmen | наводнять | zalewać.


\uvsubentry{}\uventry{super$'$flu$'$}
superflu | superfluous | überflüssig | лишній | zbyteczny.


\uvsubentry{}\uventry{super$'$jar$'$}
année bissextile | intercalary year | Schaltjahr | високосный годъ | rok przestępny.

\uventry{superstiĉ$'$}
superstition | superstition | Aberglaube | суевѣріе | zabobon.

\uventry{supoz$'$}
supposer | supose | voraussetzen | предполагать | przypuścić,
suponować.

\uventry{supr$'$}
en haut | upper (adj.) | oben | вверху | na górze.

\uvsubentry{}\uventry{supr$'$o}
sommet, cime | summit, peak | Gipfel | верхушка | szczyt, wierzchołek.


\uvsubentry{}\uventry{supr$'$aĵ$'$}
surface | surface | Oberfläche | поверхность | powierzchnia.

\uventry{sur}
sur (en touchant) | upon, on | auf | на | na.

\uventry{surd$'$}
sourd | deaf | taub | глухой | głuchy.

\uventry{surpriz$'$}
surprendre | surprise | überraschen | дѣлать сюрпризъ | niespodzianka, siurpriza.

\uventry{surtut$'$}
redingote | over-coat | Rock | сюртукъ | surdut.

\uventry{suspekt$'$}
suspecter, soupçonner | suspect | verdächtigen | подозрѣвать | podejrzewać.

\uventry{ŝut$'$}
verser, répandre (pas pour les liquides) | discharge (corn,
etc.) | schütten | сыпать | sypać.

\uventry{svat$'$}
rechercher en mariage, s’entremettre | intermeddle | freien,
werben | сватать | swatać.

\uventry{ŝvel$'$}
enfler | swell | schwellen | пухнуть | puchnąć.

\uventry{sven$'$}
s’évanouir | faint, vanish | in Ohnmacht fallen | падать въ
обморокъ | omdleć.

\uventry{sving$'$}
brandiller | swing, toss | schwingen | махать | machać.

\uventry{ŝvit$'$}
suer | perspire | schwitzen | потѣть | pocić się.

\uvlitero{T}

\uventry{tabak$'$}
tabac | tobacco | Tabak | табакъ | tytóń.

\uventry{taban$'$}
taon | gad-fly | Bremse (Fliege) | слѣпень | giez.

\uventry{tabel$'$}
table, liste | table, index | Tabelle | таблица | tabelka.

\uventry{tabl$'$}
table | table | Tisch | столъ | stół.

\uventry{tabul$'$}
planche | tablet | Tafel, Brett | доска | deska, tablica.

\uventry{taĉment$'$}
détachement | detachment | Abtheilung, Detachement | отрядъ | oddział.

\uventry{taft$'$}
taffetas | taffety | Taffet | тафта | kitajka.

\uventry{tag$'$}
jour | day | Tag | день | dzień.

\uventry{tajlor$'$}
tailleur | tailor | Schneider | портной | krawiec.

\uventry{taks$'$}
taxer | tax, appraise | abschätzen, taxiren | оцѣнивать | taksować.

\uventry{talent$'$}
talent | talent | Talent | талантъ | talent.

\uventry{tali$'$}
taille | tally | Taille | станъ | talia, figura.

\uventry{talp$'$}
taupe | mole (animal) | Maulwurf | кротъ | kret.

\uventry{tambur$'$}
tambour | drum | Trommel | барабанъ | bęben.

\uvsubentry{}\uventry{tambur$'$i}
battre le tambour | drum | trommeln | барабанить | bębnić.

\uventry{tamen}
pourtant, néanmoins | however, nevertheless | doch, jedoch | однако | jednak.

\uventry{tan$'$}
tanner | tan | gärben | дубить | garbować, wyprawiać (skóry).

\uventry{tapet$'$}
tapisserie, tenture | tapestry | Tapete | обои | obicia,
tapety.

\uventry{tapiŝ$'$}
tapis | carpet | Teppich | коверъ | dywan.

\uventry{tas$'$}
tasse | cup | Tasse | чашка | filiżanka.

\uventry{taŭg$'$}
être bon pour..., convenir pour... | be fit for | taugen | годиться | być zdatnym.

\uventry{tavol$'$}
couche, rangée | couch, bed, row | Schicht, Scheibe | слой | warstwa.

\uventry{te$'$}
thé | tea | Thee | чай | herbata.

\uvsubentry{}\uventry{te$'$kruĉ$'$}
théière | tea-pot | Theekanne | чайникъ | czajnik.


\uvsubentry{}\uventry{te$'$maŝin$'$}
bouilloire | tea-kettle | Theemaschine | самоваръ | samowar.

\uventry{ted$'$}
provoquer la satiété, ennuyer | tedious | Ueberdruss erregen | надоѣдать | dokuczać.

\uventry{teg$'$}
mettre par-dessus, couvrir | overlay, cover | beziehen,
überziehen | наволакивать | powłóczyć.

\uvsubentry{}\uventry{teg$'$o}
taie | pillowcase, bedtick | Ueberzug | наволочка | powłoczka.

\uventry{tegment$'$}
toit | roof | Dach | крыша | dach.

\uvsubentry{}\uventry{sub$'$tegment$'$}
galetas | garret | Boden, Dachstube | чердакъ | poddasze.

\uventry{teks$'$}
tisser | weave | weben | ткать | tkać.

\uventry{teler$'$}
assiette | plate | Teller | тарелка | talerz.

\uventry{tem$'$}
thème | thema | Thema | тема, задача | temat, zadanie.

\uventry{temp$'$}
temps (durée) | time | Zeit | время | czas.

\uventry{tempi$'$}
tempe | temple (of forehead) | Schläfe | високъ | skroń.

\uventry{ten$'$}
tenir | hold, grasp | halten | держать | trzymać.

\uvsubentry{}\uventry{ten$'$il$'$}
manche, anse | touch, hold, handle | Stiel, Griftf | рукоятка | rączka.


\uvsubentry{}\uventry{de$'$ten$'$}
retenir | keep off, detain | enthalten, abhalten | удерживать | zatrzymywać.


\uvsubentry{}\uventry{sub$'$ten$'$}
étayer, appuyer | prop, stay | stützen, unterhalten | поддерживать | podtrzymywać.

\uventry{tend$'$}
tente, pavillon | tent, pavilion | Zelt | палатка | namiot.

\uventry{tenden$'$}
tendon | tendon | Sehne | тетива, сухожиліе | ścięgno,
żyła.

\uventry{tent$'$}
tenter | tempt, try | prüfen, versuchen | искушать | kusić.

\uventry{ter$'$}
terre | earth | Erde | земля | ziemia.

\uvsubentry{}\uventry{en$'$ter$'$ig$'$}
ensevelir | bury | begraben | хоронить | pochować, pogrzebać.


\uvsubentry{}\uventry{ter$'$kol$'$}
isthme | isthmus | Landenge | перешеекъ | międzymorze, przesmyk.


\uvsubentry{}\uventry{ter$'$pom$'$}
pomme de terre | potato | Kartoffel | картофель | kartofel.

\uventry{teras$'$}
terrasse | terrace | Terasse | терасса | taras.

\uventry{terebint$'$}
térébenthine | turpentine | Terpentin | терпентинъ | terpentyna.

\uventry{termin$'$}
terme | term | Termin | терминъ | termin.

\uventry{tern$'$}
éternuer | sneeze | niessen | чихать | kichać.

\uventry{terur$'$}
terreur, effroi | terror | Schrecken | ужасъ | przerażenie.

\uventry{testament$'$}
testament | testament | Testament | завѣщаніе | testament.

\uventry{testik$'$}
testicule | testicle | Ei (anatom.) | яичко (анатом.) | jajko.

\uventry{testud$'$}
tortue | tortoise | Schildkröte | черепаха | żółw.

\uventry{tetan$'$}
tetanos | tetanus | Starrkrampf | столбнякъ | tężec.

\uventry{tetr$'$}
tétras, coq de bruyère | grouse | Birkhahn | тетеревъ | cietrzew.

\uventry{tetra$'$}
gélinotte de bois | hazel-hen | Haselhuhn | рябчикъ | jarząbek.

\uventry{tez$'$}
thèse | thesis | Satz, Thesis | положеніе, тезисъ | teza.

\uventry{tia}
tel | such | solcher | такой | taki.

\uventry{tial}
c’est pourquoi | therefore | darum, deshalb | потому | dla
tego.

\uventry{tiam}
alors | then | dann | тогда | wtedv.

\uventry{tibi$'$}
os de la jambe, tibia | shin bone | Schienbein | голень | goleń.

\uventry{tie}
là-bas, là, y | there | dort | тамъ | tam.

\uventry{tiel}
ainsi, de cette manière | thus, so | so | такъ | tak.

\uventry{tigr$'$}
tigre | tiger | Tiger | тигръ | tygrys.

\uventry{tikl$'$}
chatouiller | tickle | kitzeln | щекотать | łechtać.

\uventry{tili$'$}
tilleul | lime-tree | Linde | липа | lipa.

\uventry{tim$'$}
craindre | fear | fürchten | бояться | obawiać się.

\uventry{timian$'$}
thym | thyme | Thymian | ѳиміамъ | macierzanka.

\uventry{timon$'$}
timon | thill, coach-beem | Deichsel | дышло | dyszel.

\uventry{tine$'$}
teigne | moth | Motte | моль | mól.

\uventry{tint$'$}
tinter | chink, clank, jingle | klirren | бряцать | brząkać.

\uventry{tio}
cela | that one | das, jenes | то, это | to, tamto.

\uventry{tiom}
autant, tant | so much | so viel | столько | tyle.

\uventry{tir$'$}
tirer | draw, pull, drag | ziehen | тянуть | ciągnąć.

\uvsubentry{}\uventry{kun$'$tir$'$}
astreindre | astringe | zusammenziehen, adstringiren | стягивать | ściągać.

\uventry{titol$'$}
titre | title | Titel | титулъ | tytuł.

\uventry{tiu}
celui-là | that | jener | тотъ | tamten.

\uventry{tol$'$}
toile | linen | Leinwand | полотно | płótno.

\uvsubentry{}\uventry{tol$'$aĵ$'$}
linge | linen | Wäsche | бѣлье | bielizna.

\uventry{toler$'$}
tolérer | tolerate | toleriren | терпѣть | tolerować,
cierpieć.

\uventry{tomb$'$}
tombe | tomb | Grab | могила | grób, mogiła.

\uventry{tombak$'$}
tombac | pinchbeck | Tomback | томпакъ | tombak.

\uventry{ton$'$}
ton, son, | tone, sound | Ton | тонъ | tom.

\uventry{tond$'$}
tondre | clip, shear | scheeren | стричь | strzydz.

\uvsubentry{}\uventry{tond$'$il$'$}
ciseaux | scissors | Scheere | ножницы | nożyce.

\uventry{tondr$'$}
tonner | thunder | donnern | гремѣть | grzmieć.

\uventry{topaz$'$}
topaze | topaz | Topas | топазъ | topaz.

\uventry{torĉ$'$}
torche | torch | Fackel | факелъ | pochodnia.

\uventry{tord$'$}
tordre | wind, twist | drehen, winden (z. B. Stricke) | крутить | kręcić.

\uventry{torf$'$}
torf | turf | Torf | торфъ | torf.

\uventry{torn$'$}
tourner (avec un tour) | turn (on a lathe) | drechseln | точить | toczyć.

\uventry{tornistr$'$}
havresac | knapsack | Ranzen, Tornister | ранецъ | tornister.

\uventry{tort$'$}
tourte | tart | Torte | тортъ | tort.

\uventry{tra}
à travers | through | durch | черезъ, сквозь | przez (wskroś).

\uventry{trab$'$}
poutre | beam (of wood) | Balken | бревно | belka.

\uventry{traduk$'$}
traduire | translate | übersetzen | переводить | tłomaczyć.

\uventry{traf$'$}
toucher le but | strike, meet, fall in with | treffen | попадать | trafić.

\uventry{traĥe$'$}
trachée-artère | wind-pipe | Luftröhre | дыхательное горло | tchawica.

\uventry{trajt$'$}
trait | lineament, touch | Zug (Gesichts- etc.) | черта
(напр. лица) | rys (twarzy).

\uventry{trakt$'$}
négocier (faire des négociations) | transact | unterhandeln | вести переговоры | porozumiewać się, układać się.

\uventry{tranĉ$'$}
trancher, couper | cut | schneiden | рѣзать | rżnąć.

\uvsubentry{}\uventry{al$'$tranĉ$'$}
couper, tailler | cut out | zuschneiden | кроить | przykroić.

\uventry{trankvil$'$}
tranquille | quiet | ruhig | спокойный | spokojny.

\uventry{trans}
au-delà, trans- | across | jenseit, über | черезъ (надъ),
пере- | przez, prze-.

\uventry{tre}
très, fort, bien (adv.) | very | sehr | очень | bardzo.

\uventry{tref$'$}
trèfle | club | Treff (Kartsp.) | трефы | tref (w kartach).

\uventry{trem$'$}
trembler | tremble | zittern | дрожать | drżeć, trząść się.

\uventry{tremol$'$}
tremble | asp | Espe | осина | osina.

\uventry{tremp$'$}
tremper | dip, steep | tunken | макать | umoczyć.

\uventry{tren$'$}
traîner | drag, trail | schleppen | влачить | wlec.

\uventry{trezor$'$}
trésor | treasure | Schatz | сокровище | skarb.

\uventry{tri}
trois | three | drei | три | trzy.

\uventry{tribun$'$}
tribune | orator’s pulpit | Rednerbühne | трибуна | trybuna,
mównica.

\uvsubentry{}\uventry{tri$'$foli$'$}
trèfle | trefoil | Klee | трилистникъ | koniczyna.

\uventry{trik$'$}
tricoter | knit | stricken | вязать (чулки) | robić
pończochy.

\uventry{trikot$'$}
tricot | cudgel, knit | Tricot | трико | trykot.

\uventry{tril$'$}
tril, trille | trill | Triller | трель | tryl.

\uventry{trink$'$}
boire | drink | trinken | пить | pić.

\uventry{trip$'$}
tripes | tripes | Kaldaunen, Kutteln | потроха | bebechy, flaki.

\uventry{tritik$'$}
froment | wheat | Weizen | пшеница | pszenica.

\uventry{trivial$'$}
trivial | trivial | abgedroschen | избитый, пошлый | trywialny, gminny.

\uventry{tro}
trop | too | zu, zu viel | слишкомъ | zbyt.

\uventry{trog$'$}
auge | trough | Trog | корыто | koryto.

\uventry{tromb$'$}
trombe | water-spout | wirbelwindartiger Orkan | смерчъ | zawierucha.

\uventry{tromp$'$}
tromper, duper | deceive, cheat | betrügen | обманывать | oszukiwać.

\uventry{tron$'$}
trône | throne | Thron | престолъ | tron.

\uventry{tropik$'$}
tropique | tropic | Tropicus, Wendekreis | тропикъ | zwrotnik.

\uventry{trot$'$}
trotter | trot | traben | бѣжать рысью | kłusować.

\uventry{trotuar$'$}
trottoir | side-walk | Trottoir | тротуаръ | trotuar, chodnik.

\uventry{tro$'$uz$'$}
abuser | abuse | missbrauchen | злоупотреблять | nadużywać.

\uventry{trov$'$}
trouver | find | finden | находить | znajdować.

\uventry{tru$'$}
trou | hole | Loch | дыра | dziura.

\uventry{trud$'$}
contraindre de prendre | press upon, obtrude | dringen,
aufdringen | навязывать | nalegać.

\uventry{truf$'$}
truffe | truffle | Trüffel | трюфель | trufla.

\uventry{trul$'$}
truelle | trowel | Kelle | лопатка | kielnia.

\uventry{trumpet$'$}
trompette | trumpet | Trompete | труба (музык.) | trąba.

\uventry{trunk$'$}
tronc, tige | trunk, stem | Stamm, Rumf | стволъ | pień, tułów.

\uvsubentry{}\uventry{trunk$'$et$'$}
tige, queue | stalk | Stengel | стебель | łodyga, trzonek.

\uventry{trut$'$}
truite | trout | Forelle | форель | pstrąg.

\uventry{tualet$'$}
toilette | toilet | Toilette | туалетъ | tualeta.

\uventry{tub$'$}
tuyau | tube | Röhre | труба | rura.

\uventry{tuber$'$}
tubérosité | bulb | Knolle | шишка, бугоръ | guz, opuszka.

\uventry{tuf$'$}
touffe | tuft | Büschel | хохолъ, пучекъ | kosmek, pęczek.

\uventry{tuj}
tout de suite, aussitôt | immediate | bald, sogleich | сейчасъ | natychmiast.

\uventry{tuk$'$}
mouchoir | cloth | Tuch (Hals-, Schnupf- etc.) | платокъ | chustka.

\uventry{tul$'$}
tulle | tulle | Tüll | тюль | tiul.

\uventry{tulip$'$}
tulipe | tulip | Tulpe | тюльпанъ | tulipan.

\uventry{tumult$'$}
tumulte | tumult | Aufruhr | суматоха | zamięszanie.

\uventry{tur$'$}
tour (édifice) | tower | Thurm | башня | wieża.

\uventry{turban$'$}
turban | turban | Turban | тюрбанъ | turban.

\uventry{turd$'$}
grive | thrush | Drossel | дроздъ | drozd.

\uventry{turkis$'$}
turquoise | turquoise | Türkis | бирюза | turkus.

\uventry{turment$'$}
tourmenter | torment | quälen, martern | мучить | męczyć.

\uventry{turn$'$}
tourner | turn (vb.) | drehen, wenden | вращать, обращать | obracać.

\uventry{turnir$'$}
tournoi | tourney | Turnier | турниръ | turniej, gonitwa.

\uventry{turt$'$}
tourterelle | turtle-dove | Turteltaube | горлица | gruchawka.

\uventry{tus$'$}
tousser | cough | husten | кашлять | kaszleć.

\uventry{tuŝ$'$}
toucher | touch | rühren | трогать | ruszać, dotykać.

\uventry{tut$'$}
entier, total | whole | ganz | цѣлый | caly.

\uvlitero{U}

\uventry{u}
marque l’impératif | ending of the imperative in verbs | bezeichnet
den Imperativ | означаетъ повелительное наклоненіе | oznacza tryb
rozkazujący.

\uventry{uj$'$}
qui porte, qui contient, qui est peuplé de; ex. \uventry{pom$'$} pomme ―
\uventry{pom$'$uj$'$} pommier; \uventry{cigar$'$} cigare ― \uventry{cigar$'$uj$'$} porte-cigares;
\uventry{Turk$'$} Turc ― \uventry{Turk$'$uj$'$} Turquie | filled with; e. g. \uventry{ink$'$} ink ―
\uventry{ink$'$uj$'$} ink-pot; \uventry{pom$'$} apple ― \uventry{pom$'$uj$'$} apple-tree; \uventry{Turk$'$uj$'$}
Turkey | Behälter, Träger (d. h. Gegenstand worin... aufbewahrt
wird,... Früchte tragende Pflanze, von... bevölkertes Land);
z. B. \uventry{cigar$'$} Cigarre ― \uventry{cigar$'$uj$'$} Cigarrenbüchse; \uventry{pom$'$} Apfel
― \uventry{pom$'$uj$'$} Apfelbaum; \uventry{Turk$'$} Türke ― \uventry{Turk$'$uj$'$} Türkei | вмѣститель, носитель (т. е. вещь, въ которой храниться..., растеніе
несущее... или страна заселенная...); напр. \uventry{cigar$'$} сигара ―
\uventry{cigar$'$uj$'$} портъ-сигаръ; \uventry{pom$'$} яблоко ― \uventry{pom$'$uj$'$} яблоня; \uventry{Turk$'$}
Турокъ ― \uventry{Turk$'$uj$'$} Турція | zawierający, noszący (t. j. przedmtot,
w którym się coś przechowuje, roślina, która wydaje owoc, lub kraj,
względem zaludniających go mieszkańców; np. \uventry{cigar$'$} cygaro ―
\uventry{cigar$'$uj$'$} cygarnica; \uventry{pom$'$} jabłko ― \uventry{pom$'$uj$'$} jabłoń; \uventry{Turk$'$}
turek ― \uventry{Turk$'$uj$'$} Turcya.

\uventry{ul$'$}
qui est caractérisé par telle ou telle qualité, telle façon
d’être; ex. \uventry{bel$'$} beau ― \uventry{bel$'$ul$'$} bel homme; \uventry{mal$'$jun$'$} vieux ―
\uventry{mal$'$jun$'$ul$'$} vieillard | person noted for...; e. g. \uventry{avar$'$} covetous
― \uventry{avar$'$ul$'$} miser, covetous person | Person, die sich
durch... unterscheidet; z. B. \uventry{jun$'$} jung ― \uventry{jun$'$ul$'$} Jüngling;
\uventry{avar$'$} geizig ― \uventry{avar$'$ul$'$} Geizhals | особа отличающаяся даннымъ
качествомъ; напр. \uventry{bel$'$} красивый ― \uventry{bel$'$ul$'$} красавецъ; \uventry{avar$'$}
скупой ― \uventry{avar$'$ul$'$} скряга | człowiek, posiadający dany przymiot;
np. \uventry{riĉ$'$} bogaty ― \uventry{riĉ$'$ul$'$} bogacz.

\uventry{ulcer$'$}
ulcère | ulcer | Geschwür | язва | wrzód, owrzodzenie.

\uventry{ulm$'$}
orme | elm | Ulme, Rüster | вязъ | wiąz.

\uventry{uln$'$}
aune | ell, yard | Elle | локоть | łokieć.

\uventry{um$'$}
suffixe peu employé, et qui reçoit différents sens aisément
suggérés par le contexte et la signification de la racine à laquelle
il est joint | this syllable has no fixed meaning | Suffix von
verschiedener Bedeutung | суффиксъ безъ постояннаго значенія | przyrostek, nie mający stlałego znaczenia.

\uventry{umbilik$'$}
nombril | navel | Nabel | пупъ | pępek.

\uventry{unc$'$}
once | ounce | Unze | унція | uncya.

\uventry{ung$'$}
ongle | nail (finger) | Nagel (am Finger) | ноготь | paznokieć.

\uvsubentry{}\uventry{ung$'$eg$'$}
griffe, serre | claw, clutch | Kralle | коготь | pazur.

\uventry{uniform$'$}
uniforme | uniform | Uniform | мундиръ | mundur, uniform.

\uventry{univers$'$}
univers | universe | Weltall | вселенная | wszechświat.

\uventry{universal$'$}
universel | universal | allgemein | всеобщій | ogólny, uniwersalny.

\uventry{universitat$'$}
université | university | Universität | университетъ | uniwersytet.

\uventry{unu}
un | one | ein, eins | одинъ | jeden.

\uventry{ur$'$}
ure, bœuf sauvage | ure-ox, wid-bull | Auerochs | зубръ | żubr.

\uventry{urb$'$}
ville | town | Stadt | городъ | miasto.

\uvsubentry{}\uventry{antaŭ$'$urb$'$}
faubourg | suburb | Vorstadt | предмѣстье | przedmieście.

\uventry{urin$'$}
uriner | piss | pissen | мочиться | urynować.

\uventry{urn$'$}
urne | urn | Urne | урна | urna.

\uventry{urogal$'$}
coq de bruyère | wild cock | Auerhahn | глухарь | głuszec.

\uventry{urs$'$}
ours | bear (animal) | Bär | медвѣдь | niędźwiedź.

\uventry{urtik$'$}
ortie | nettle | Nessel | крапива | pokrzywa.

\uventry{us}
marque le conditionnel (ou le subjonctif) | ending of conditional
in verbs | bezeichnet den Konditionalis (oder Konjunktiv) | означаетъ
условное наклоненіе | oznacza tryb warunkowy.

\uventry{uter$'$}
matrice | matrix | Gebärmutter | матка (анат.) | macica.

\uventry{util$'$}
utile | useful | nützlich | полезный | pożyteczny.

\uvsubentry{}\uventry{mal$'$util$'$}
nuisible | noxious | schädlich | вредный | szkodliwy.

\uventry{uz$'$}
employer | use | gebrauchen | употреблять | używać.

\uvsubentry{}\uventry{uz$'$aĵ$'$}
outils | furniture | Geräthschaft | утварь | sprzęt.


\uvsubentry{}\uventry{tro$'$uz$'$}
abuser | abuse | missbrauchen | злоупотреблять | nadużywać.

\uventry{uzurp$'$}
usurper | usurp | usurpiren | беззаконно захватывать | uzurpować.

\uvlitero{V}

\uventry{vafl$'$}
gaufre, oublie | wafer | Waffel | вафля | wafel.

\uventry{vag$'$}
vaguer | rove, extravagate | herumschweifen | бродить, шляться | włóczyć się.

\uventry{vagon$'$}
wagon | waggon | Wagon | вагонъ | wagon.

\uvsubentry{}\uventry{vagon$'$ar$'$}
train | train | Zug (Bahn-) | поѣздъ | pociąg.

\uventry{vakcini$'$}
airelle rouge | red bilberry | Preisselbeere | брусника | borówka czerwona.

\uventry{vaks$'$}
cire | wax | Wachs | воскъ | wosk.

\uvsubentry{}\uventry{vaks$'$tol$'$}
toile cirée | cerecloth, oil cloth | Wachsleinwand | клеенка | cerata.


\uvsubentry{}\uventry{sigel$'$vaks$'$}
cire à cacheter | sealing-wax | Siegellack | сургучъ | lak.

\uventry{val$'$}
vallée | valley | Thal | долина | dolina.

\uventry{valiz$'$}
valise | valise | Felleisen | чемоданъ | waliza.

\uventry{vals$'$}
valse | waltz | Walzer | вальсъ | walc.

\uventry{van$'$}
vain | vain, needless | vergeblich | напрасный | daremny.

\uventry{vang$'$}
joue | check | Wange | щека | policzek.

\uvsubentry{}\uventry{vang$'$har$'$o$'$j}
favoris | whiskers | Backenbart | бакенбарды | faworyty.

\uventry{vanil$'$}
vanille | West-India nut | Vanille | ваниль | wanilia.

\uventry{vant$'$}
vain, frivole | vain | eitel | суетный | czczy, marny.

\uventry{vapor$'$}
vapeur | steam | Dampf | паръ | para.

\uventry{varb$'$}
enrôler, engager | list, levy | werben (z. B. zu
Kriegsdiensten) | вербовать | werbować, zaciągać.

\uventry{variol$'$}
variole | smallpox | Blattern, Pocken | оспа | ospa.

\uventry{varm$'$}
chaud | warm | varm | теплый | ciepły.

\uvsubentry{}\uventry{mal$'$varm$'$um$'$}
se refroidir | catch cold | sich erkälten | простудиться | przeziębić się.

\uventry{vart$'$}
soigner | take care | warten, pflegen | нянчить, ухаживать | pielęgnować, piastować.

\uventry{vasal$'$}
vassal | vassal | Vasall | вассалъ | dannik, wasal.

\uventry{vast$'$}
vaste, étendu | wide, vast | weit, geräumig | обширный,
просторный | obszerny.

\uvsubentry{}\uventry{mal$'$vast$'$}
étroit | strait, angust | eng | тѣсный | ciasny.


\uvsubentry{}\uventry{vast$'$ig$'$}
répandre | spread | verbreiten | распространять | rozszerzać, rozprzestrzeniać.

\uventry{vat$'$}
ouate | wad | Watte | вата | wata.

\uventry{vaz$'$}
vase | vase | Gefäss | сосудъ | naczynie.

\uventry{ve!}
malheur! | wo, woe! | wehe! | увы! | och!.

\uventry{veget$'$}
végéter | vegetate | vegetiren | прозябать | wegetować.

\uventry{vejn$'$}
veine | vein | Ader | жила, вена | żyła.

\uventry{vek$'$}
réveiller, éveiller | wake, arouse | wecken | будить | budzić.

\uventry{vekt$'$}
fléau | flail | Schulterjoch, Wagebalken | коромысло | bela wagowa.

\uventry{vel$'$}
voile | sail (subst.) | Segel | парусъ | żagiel.

\uventry{velen$'$}
vélin | vellum | Velin | веленевая бумага | papier welinowy.

\uventry{velk$'$}
se faner, se flétrir | fade | welken | вянуть | więdnąć.

\uventry{velur$'$}
velours | velvet | Sammet | бархатъ | aksamit.

\uventry{ven$'$}
venir | come | kommen | приходить | przychodzić.

\uvsubentry{}\uventry{de$'$ven$'$}
descendre | descend | abstammen | происходить | pochodzić.

\uventry{vend$'$}
vendre | sell | verkaufen | продавать | sprzedawać.

\uventry{vendred$'$}
vendredi | Friday | Freitag | пятница | piątek.

\uventry{venen$'$}
poison | poison | Gift | ядъ | trucizna.

\uventry{venĝ$'$}
se venger | vengeance | rächen | мстить | mścić się.

\uventry{venk$'$}
vaincre | conquer | siegen | побѣждать | zwyciężać.

\uventry{vent$'$}
vent | wind | Wind | вѣтеръ | wiatr.

\uvsubentry{}\uventry{vent$'$um$'$}
éventer | fan | fächeln | вѣять | wiać.

\uventry{ventol$'$}
éventer, vanner | ventilate | ventiliren | вентилировать | wentylować, wietrzyć.

\uventry{ventr$'$}
ventre | belly | Bauch | брюхо | brzuch.

\uventry{ver$'$}
vérité | true | Wahrheit | истина | prawda.

\uventry{verb$'$}
verbe | verb | Zeitwort | глаголъ | czasownik.

\uventry{verd$'$}
vert | green | grün | зеленый | zielony.

\uventry{verdigr$'$}
vert-de-gris | verdigris | Grünspan | мѣдянка | rdza
miedziana, grynszpan.

\uventry{verg$'$}
verge | rod | Ruthe | розга | rózga.

\uventry{verk$'$}
composer, faire des ouvrages (littér.) | work (literary) | verfassen, schreiben (Bücher etc.) | сочинять | tworzyć, układać,
pisać (dzieło).

\uventry{verm$'$}
ver | worm | Wurm | червь | robak.

\uventry{vermiĉel$'$}
vermicelle | vermicelli | Nudel | лапша | makaron.

\uventry{vers$'$}
vers | verse | Vers | стихъ | wiersz.

\uventry{verŝ$'$}
verser | pour | giessen | лить | lać.

\uventry{verst$'$}
verste | verst | Werst | верста | wiorsta.

\uventry{vert$'$}
sommet de la tête | crown ot the head | Scheitel (auf dem
Kopfe) | темя, макушка | ciemię.

\uventry{vertebr$'$}
vertèbre | chine-bone | Wirbel (Rücken-) | позвонокъ | kręg.

\uventry{vertikal$'$}
vertical | vertical | senkrecht | вертикальный | pionowy.

\uventry{veruk$'$}
verrue | wart | Warze | бородавка | brodawka.

\uventry{vesp$'$}
guêpe | wasp | Wespe | оса | osa.

\uventry{vesper$'$}
soir | evening | Abend | вечеръ | wieczór.

\uventry{vespert$'$}
chauve-souris | bat | Fledermaus | летучая мышь | nietoperz.

\uventry{vest$'$}
vêtir, habiller | clothe | ankleiden | одѣвать | odziewać,
ubierać.

\uvsubentry{}\uventry{vest$'$o}
habit | clothes | Kleid | платье | ubiór, odzież.

\uventry{vestibl$'$}
vestibule | floor | Hausflur | сѣни | sień.

\uventry{veŝt$'$}
gilet | vest | Weste | жилетка | kamizelka.

\uventry{vet$'$}
parier | bet, wager | wetten | биться объ закладъ | założyć
się.

\uventry{veter$'$}
temps (température) | weather | Wetter | погода | pogoda.

\uventry{vetur$'$}
aller, partir, à l’aide d’un véhicule quelconque: bateau,
voiture, etc. | journey, travel | fahren | ѣхать | jechać.

\uventry{vezik$'$}
vessie | blister, bladder | Blase | пузырь | pęcherz.

\uventry{vezir$'$}
vizir | visier | Vezier | визирь | wezyr.

\uventry{vi}
vous, toi, tu | you | Ihr, du, Sie | вы, ты | wy, ty.

\uvsubentry{}\uventry{vi$'$a}
votre, ton | your | Ihr, euer, dein | вашъ, твой | wasz, twój.

\uventry{viand$'$}
viande | meat, flesh | Fleisch | мясо | mięso.

\uventry{viburn$'$}
aubier | sap | Schlingstrauch | калина | kalina.

\uventry{vic$'$}
rang, série, tour | row, rank | Reihe, Reihenfolge | рядъ | rząd.

\uventry{vid$'$}
voir | see | sehen | видѣть | widzieć.

\uventry{vidv$'$}
veuf | widower | Wittwer | вдовецъ | wdowiec.

\uventry{vigl$'$}
éveillé, vigilant | awake, gay, vigilant | munter | бодрый | czujny.

\uventry{vikari$'$}
vicaire | vicar | Stellvertreter | намѣстникъ | zastępcar.

\uventry{vil$'$}
touffe, villosité | rag, tuft | Zotte | косма | kłak, kosmyk.

\uventry{vilaĝ$'$}
village | village | Dorf | деревня | wieś.

\uvsubentry{}\uventry{vilaĝ$'$an$'$}
paysan | countryman | Bauer | крестьянинъ | wieśniak.

\uventry{vin$'$}
vin | vine | Wein | вино | wino.

\uvsubentry{}\uventry{vin$'$ber$'$}
raisin | grape | Weintraube | виноградъ | winogrono.


\uvsubentry{}\uventry{sek$'$vin$'$ber$'$}
raisin sec | raisin | Rosine | изюмъ | rodzynka.

\uventry{vinagr$'$}
vinaigre | vinegar | Essig | уксусъ | ocet.

\uventry{vind$'$}
tortiller | wind, twist | winden | пеленать | powijać.

\uventry{vintr$'$}
hiver | winter | Winter | зима | zima.

\uventry{viol$'$}
violette | violet | Veilchen | фіалка | fiołek.

\uventry{violon$'$}
violon | violin | Geige | скрипка | skrzypce.

\uventry{violonĉel$'$}
violoncelle | violoncello | Violoncell | віолончель | wiolonczela.

\uventry{vip$'$}
fouet | whip | Peitsche | бичъ | bicz.

\uventry{vipur$'$}
vipère | viper | Viper | ехидна | źmija.

\uventry{vir$'$}
homme (sexe) | man | Mann | мужъ, мужчина | mężczyzna, mąż.

\uventry{virg$'$}
virginal | virginal | jungfräulich | дѣвственный | dziewiczy.

\uvsubentry{}\uventry{mal$'$virg$'$ig$'$}
déshonorer, violer | dishonour, violate, deflower | schänden (eine Jungfrau) | изнасиловать | zgwałcić, bezecnić.

\uventry{virt$'$}
vertu | virtue | Tugend | добродѣтель | cnota.

\uventry{virtuoz$'$}
virtuose | virtuoso | Virtuos | виртуозъ | wirtuoz.

\uventry{viŝ$'$}
essuyer | wipe | wischen | обтирать | ocierać.

\uvsubentry{}\uventry{viŝ$'$il$'$}
essuie-main | towel | Handtuch | полотенцо | ręcznik.

\uventry{vitr$'$}
verre (matière) | glass (substance) | Glas | стекло | szkło.

\uvsubentry{}\uventry{okul$'$vitr$'$o}
lunettes | spectacles | Brille | очки | okulary.

\uventry{vitriol$'$}
vitriol | vitriol | Vitriol | купоросъ | witryol.

\uventry{viv$'$}
vivre | live | leben | жить | żyć.

\uventry{vizaĝ$'$}
visage, figure | face | Gesicht | лицо | twarz.

\uventry{vizier$'$}
visière | visor | Visir | забрало | przedoblicze.

\uventry{vizit$'$}
visiter | visit | besuchen | посѣщать | odwiedzić,
wizytować.

\uventry{voĉ$'$}
voix | voice | Stimme | голосъ | glos.

\uventry{voj$'$}
route, voie | way, road | Weg | дорога | droga.

\uventry{vojaĝ$'$}
voyager | voyage | reisen | путешествовать | podróżować.

\uventry{vok$'$}
appeler | call | rufen | звать | wołać.

\uventry{vokal$'$}
voyelle | vowel | Vokal | гласная | samogłoska.

\uventry{vol$'$}
vouloir | wish, will | wollen | хотѣть | chcieć.

\uventry{volont$'$}
volontiers | willingly | gern | охотно | chętnie.

\uventry{volum$'$}
tome, volume | volume | Band (der) | томъ | tom.

\uventry{volupt$'$}
volupté | sensual pleasure | Wollust | сладострастіе | rozkosz, lubieżność.

\uventry{volv$'$}
rouler, enrouler | turn round, roll up | wickeln | вить | wić.

\uventry{vom$'$}
vomir | vomit | sich erbrechen | рвать, блевать | wymiotować.

\uventry{vort$'$}
mot | word | Wort | слово | słowo, wyraz.

\uventry{vost$'$}
queue | tail | Schwanz, Schweif | хвостъ | ogon.

\uventry{vual$'$}
voile | veil | Schleier | вуаль | wual.

\uventry{vulkan$'$}
volcan | vulcan | Vulkan | вулканъ | wulkan.

\uventry{vulp$'$}
renard | fox | Fuchs | лисица | lis.

\uventry{vultur$'$}
vautour | vultur | Geier | коршунъ | sęp.

\uventry{vund$'$}
blesser | wound | verwunden | ранить | ranić.

\uvlitero{Z}

\uventry{zebr$'$}
zèbre | zebra | Zebra | зебра | dziki koń.

\uventry{zenit$'$}
zénith | zenith | Zenit | зенитъ | zenit, szczyt.

\uventry{zibel$'$}
zibeline | sable | Zobel | соболь | sobol.

\uventry{zingibr$'$}
gingembre | ginger | Ingwer | имбирь | imbier.

\uventry{zink$'$}
zinc | zinc | Zink | цинкъ | cynk.

\uventry{zizel$'$}
zizel | zizel | Ziselmaus | сусликъ | suseł.

\uventry{zon$'$}
ceinture | girdle | Gürtel | поясъ, кушакъ | pas.

\uventry{zorg$'$}
avoir soin, prendre soin de | care, anxiety | sorgen | заботиться | troszczyć się.

\uvsubentry{}\uventry{zorg$'$ant$'$}
tuteur | tutor | Vormund | опекунъ | opiekun.

\uvsubentry{}\uventry{zorg$'$at$'$}
pupille | pupil | Pflegling | питомецъ | wychowaniec.

\uventry{zum$'$}
bourdonner | hum, buzz | summen | жужжать | brzęczeć, mruczeć.

\end{outdent}
\normalsize

\cleardoublepage

% Enhavo
%
% Enhavo
%
\selectlanguage{esperanto}
\begin{center}
\narrow{\Large{TABELO DE LA ENHAVO}}
\thispagestyle{plain}
\end{center}
{\setlength{\parindent}{0pt}

{\centering \rule{13mm}{0.4pt}\par}

{\small
\begin{flushright}
\begin{tblr}{Xr}
ANTAŬPAROLO \dotfill & \pageref{antau} \\
\end{tblr}
\end{flushright}

\vspace{1em}

{\centering
FUNDAMENTA GRAMATIKO \fauxsc{de la} LINGVO ESPERANTO \\
\fauxsc{en} KVIN LINGVOJ\par}

\vspace{1em}

\newlength{\gramlen}
\settowidth{\gramlen}{Gramatiko}
\begin{flushright}
\begin{tblr}{lXr}
\fauxsc{Grammaire} & (Gramatiko Franca) \dotfill & \pageref{gram:franca} \\
\fauxsc{Grammar} & (\makebox[\gramlen][c]{—} Angla) \dotfill & \pageref{gram:angla} \\
\fauxsc{Grammatik} & (\makebox[\gramlen][c]{—} Germana) \dotfill & \pageref{gram:germana} \\
\fauxsc{Грамматика} & (\makebox[\gramlen][c]{—} Rusa) \dotfill & \pageref{gram:rusa} \\
\fauxsc{Gramatyka} & (\makebox[\gramlen][c]{—} Pola) \dotfill & \pageref{gram:pola} \\
\end{tblr}
\end{flushright}

\begin{flushright}
\begin{tblr}{Xr}
EKZERCARO \dotfill & \pageref{ekzercaro} \\
UNIVERSALA VORTARO \dotfill & \pageref{vortaro} \\
\end{tblr}
\end{flushright}

} % end "small" before the antaŭparolo
} % end "setlength" before the rule at top

\vspace*{\fill}

\begin{center}
\rule{0.3\textwidth}{0.4pt}

\scriptsize 54941. — Imprimerie \fauxsc{Lahure}, 9, rue de Fleurus, Paris.
\end{center}

\vspace*{\fill}


% License
\newpage
\thispagestyle{empty}
\vspace*{\fill}
\begin{center}
Ĉi tiu verko estas permesita per Creative Commons \\
Attribution-NonCommercial-ShareAlike 4.0 \\
Internacia Permesilo.

La originala verko de D-ro Zamenhof \\
estas senkopirajta.\\[2ex]

\ccbyncsa\\[2ex]

This work is licensed under a Creative Commons \\
Attribution-NonCommercial-ShareAlike 4.0 \\
International License.

The original work by Dr. Zamenhof \\
is in the public domain.
\vspace*{\fill}
\end{center}

\end{document}