%
% Esperanta gramatiko por poloj
%
\label{gram:pola}
\markboth{GRAMATYKA}{FUNDAMENTO DE ESPERANTO}
\thispagestyle{plain}
\begin{center}
\phantomsection
\narrow{\LARGE\bf \spaceoutless{GRAMATYKA}}
\addcontentsline{toc}{section}{Gramatyka (Gramatiko Pola)}
\selectlanguage{polish}

\rule{13mm}{0.4pt}
\vspace{2em}

{\large\gramsec{A) ABECADŁO}}
\vspace{1em}

\setstretch{1}
\begin{tabu} to \textwidth{+Y@{}ZY@{}ZY@{}ZY@{}ZY@{}ZY@{}ZY}
\rowstyle{\Large\arbfont} Aa, & Bb, & Cc, & Ĉĉ, & Dd, & Ee, & Ff, \\
\rowstyle{\footnotesize} a & b & c & cz & d & e & f \\[1ex]
\rowstyle{\Large\arbfont} Gg, & Ĝĝ, & Hh, & Ĥĥ, & Ii, & Jj, & Ĵĵ, \\
\rowstyle{\footnotesize} g &  dż & h & ch & i & j & ż \\[1ex]
\rowstyle{\Large\arbfont} Kk, & Ll, & Mm, & Nn, & Oo, & Pp, & Rr, \\
\rowstyle{\footnotesize} k & l & m & n & o & p & r \\[1ex]
\rowstyle{\Large\arbfont} Ss, & Ŝŝ, & Tt, & Uu, & Ŭŭ, & Vv, & Zz. \\
\rowstyle{\footnotesize} s & sz & t & u & u (krótkie) & w & z \\
\end{tabu}
\end{center}

{\footnotesize {\so{\textbf{UWAGA.}}― Drukarnia, nie posiadająca czcionek ze znaczkami, może zamiast Ĉ, Ĝ, Ĥ, Ĵ, Ŝ, Ŭ, drukować ch, gh, hh, jh, sh, u.}}
\begin{center}
\large \gramsec{B) CZĘSCI MOWY}
\end{center}

\textbf{1. Przedimka} nieokreślnego niema; jest tylko określny \textbf{la}, wspólny dla wszystkich rodzajów, przypadków i liczb.

\textbf{2. Rzeczownik} kończy się zawsze na \textbf{o}. Dla utworzenia liczby mnogiéj dodaje się końcówka \textbf{j}. Przypadków jest dwa: mianownik (nominativus) i biernik (accusativus); ten ostatni powstaje z mianownika przez dodanie zakończenia \textbf{n}. Resztę przypadków oddaje się za pomocą przyimków: dla dopełniacza (genitivus) ― \textbf{de} (od), dla celownika (dativus) ― \textbf{al} (do), dla narzędnika (instrumentalis) ― \textbf{per} (przez), lub inne przyimki odpowiednio do znaczenia. \textbf{Przykłady:} \textbf{patr/o} ojciec, \textbf{al patr/o} ojcu, \textbf{patr/o/n} ojca (przypadek czwarty), \textbf{por patr/o/j} dla ojców, \textbf{patro/j/n} ojców (przyp. czwarty).

\textbf{3. Przymiotnik} zawsze kończy się na \textbf{a}. Przypadki i liczby też same co dla rzeczownika. Stopień wyższy tworzy się przez dodanie wyrazu \textbf{pli} (więcéj), a najwyższy przez dodanie \textbf{plej} (najwięcéj); wyraz „niż“ tłomaczy się przez \textbf{ol}. \textbf{Przykład: Pli blank/a ol neĝ/o} bielszy od śniegu.

\textbf{4. Liczebniki} główne nie odmieniają się: \textbf{unu} (1), \textbf{du} (2), \textbf{tri} (3), \textbf{kvar} (4), \textbf{kvin} (5), \textbf{ses} (6), \textbf{sep} (7), \textbf{ok} (8), \textbf{naŭ} (9), \textbf{dek} (10), \textbf{cent} (100), \textbf{mil} (1000). Dziesiątki i setki tworzą się przez proste połączenie liczebników. Dla utworzenia liczebników porządkowych dodaje się końcówka przymiotnika, dla wielorakich ― przyrostek \textbf{obl}, dla ułamkowych ― \textbf{on}, dla zbiorowych ― \textbf{op}, dla podziałowych ― wyraz \textbf{po}. Prócz tego mogą być liczebniki rzeczowne i przysłówkowe. \textbf{Przykłady: kvin/cent tri/dek tri} = 533; \textbf{kvar/a} czwarty; \textbf{unu/o} jednostka; \textbf{du/e} powtóre; \textbf{tri/obl/a} potrójny, trojaki; \textbf{kvar/on/o} czwarta część; \textbf{du/op/e} we dwoje; \textbf{po kvin} po pięć.

\textbf{5. Zaimki} osobiste: \textbf{mi} (ja), \textbf{vi} (wy, ty) \textbf{li} (on), \textbf{ŝi} (ona), \textbf{ĝi} (ono; o rzeczy lub zwierzęciu), \textbf{si} (siebie) \textbf{ni} (my), \textbf{ili} (oni, one), \textbf{oni} (zaimek nieosobisty liczby mnogiéj); dzierżawcze tworzą się przez dodanie końcówki przymiotnika. Zaimki odmieniają się jak rzeczowniki. \textbf{Przykłady:} \textbf{mi/n} mnie (przyp. czwarty); \textbf{mi/a} mój.

\textbf{6. Słowo} nie odmienia się przez osoby i liczby. Np. \textbf{mi far/as} ja czynię, \textbf{la patr/o far/as} ojciec czyni, \textbf{ili far/as} oni czynią. Formy słowa:

a) Czas teraźniejszy ma zakończenie \textbf{as}. (Przykład: \textbf{mi far/as} ja czynię).

b) Czas przeszły ― \textbf{is} (\textbf{li far/is} on czynił).

c) Czas przyszły ― \textbf{os} (\textbf{ili far/os} oni będą czynili).

ĉ) Tryb warunkowy ― \textbf{us} (\textbf{ŝi far/us} ona by czynila).

d) Tryb rozliazujący ― \textbf{u} (\textbf{far/u} czyń, czyńcie).

e) Tryb bezokoliczny ― \textbf{i} (\textbf{far/i} czynić).

Imiesłowy (odmienne i nieodmienne):

f) Imiesłów czynny czasu teraźniejszego ― \textbf{ant} (\textbf{far/ant/a} czyniący, \textbf{far/ant/e} czyniąc).

g) Imiesłów czynny czasu przeszłego ― \textbf{int} (\textbf{far/int/a} który uczynił).

ĝ) Imiesłów czynny czasu przyszłego ― \textbf{ont} (\textbf{far/ont/a} który uczyni).

h) Imiesłów bierny czasu teraźn. ― \textbf{at} (\textbf{far/at/a} czyniony).

ĥ) Imiesłów bierny czasu przeszłego ― \textbf{it} (\textbf{far/it/a} uczyniony).

i) Imiesłów bierny czasu przyszłego ― \textbf{ot} (\textbf{far/ot/a} mający być uczynionym).

Wszystkie formy strony biernéj tworzą się zapomocą odpowiedniéj formy słowa \textbf{est} być i imiesłow biernego danego słowa; używa się przytem przyimka \textbf{de} (np. ŝi est/as am/at/a de ĉiu/j ― ona kochana jest przez wszystkich).

\textbf{7. Przysłówki} mają zakończenie \textbf{e}. Stopniowanie podobnem jest do stopniowania przymiotników (np. \textbf{mi/a frat/o pli bon/e kant/as ol mi} ― brat mój lepiej śpiewa odemnie).

\textbf{8. Przyimki} rządzą wszystkie przypadkiem pierwszym.

\begin{center}
\large \gramsec{C) PRAWIDŁA OGOLNE}
\end{center}

\textbf{9.} Każdy wyraz tak się czyta, jak się pisze.

\textbf{10.} Akcent pada zawsze na przedostatnią zgłoskę.

\textbf{11.} Wyrazy złożone tworzą się przez proste połączenie wyrazów (główny na końcu). \textbf{Przykład: vapor/ŝip/o,} parostatek ― z \textbf{vapor} para, \textbf{ŝip} okręt, \textbf{o} ― końcówka rzeczownika.

\textbf{12.} Przy innym przeczącym wyrazie opuszcza się przysłówek przeczący \textbf{ne} (np. \textbf{mi neniam vid/is} nigdy nie widziałem).

\textbf{13.} Na pytanie „dokąd“ wyrazy przybierają końcówkę przypadku czwartego (np. \textbf{tie} tam (w tamtem miejscu) ― \textbf{tie/n} tam (do tamtego miejsca); \textbf{Varsovi/o/n} (do Warszawy).

\textbf{14.} Każdy przyimek ma określone, stałe znaczenie; jeżeli należy użyć przyimka w wypadkach, gdzie wybór jego nie wypływa z natury rzeczy, używany bywa przyimek \textbf{je}, który nie ma samoistnego znaczenia (np. \textbf{ĝoj/i je tio} cieszyć się \textbf{z} tego; \textbf{mal/san/a je la okul/o/j} chory \textbf{na} oczy; \textbf{enu/o je la patr/uj/o} tęsknota \textbf{za} ojczyzną i t. p. Jasność języka wcale wskutek tego nie szwankuje, albowiem w tym razie wszystkie języki używaja jakiegokolwiek przyimka, byle go tylko zwyczaj uświęcił; w języku zaś międzynarodowym sankcja we wszystkich podobnych wypadkach nadaną została \textbf{jednemu} tylko przyimkowi \textbf{je}). Zamiast przmyika\eraro{przyimka} \textbf{je} używać też można, przypadku czwartego bez przyimka tam, gdzie nie zachodzi obawa dwuznaczności.

\textbf{15.} Tak zwane wyrazy „cudzoziemskie“ t. j. takie, które większość języków przyjęła z jednego obcego źródła, nie ulegają w języku międzynarodowym żadnéj zmianie, lecz otrzymują tylko pisownię międzynarodową; przy rozmaitych wszakże wyrazach jednego źródłosłowu, lepiéj używać bez zmiany tylko wyrazu pierwotnego, a inne tworzyć według prawideł języka międzynarodowego (np. \textbf{teatr/o} ― teatr, lecz teatralny ― \textbf{teatr/a}).

\textbf{16.} Końcówkę rzeczownika i przedimka można opuścić i zastąpić apostrofem (np. \textbf{Ŝiller’} zam. \textbf{Ŝiller/o}; \textbf{de l’ mond/o} zamiast \textbf{de la mond/o}).

\cleardoublepage
