%
% Esperanta gramatiko por germanoj
%
\label{gram:germana}
\markboth{GRAMMATIK}{FUNDAMENTO DE ESPERANTO}
\thispagestyle{plain}
\begin{center}
\phantomsection
{\fr\huge Grammatik}
\addcontentsline{toc}{section}{Grammatik (Gramatiko Germana)}
\selectlanguage{german}
\vspace{1em}

\rule{13mm}{0.4pt}
\vspace{1em}

\large\gramsec{A)} {\Large\bf\fr{Das Alphabet.}}
\vspace{1ex}

\setstretch{1}
\begin{tabu} to \textwidth{+Y@{}ZY@{}ZY@{}ZY@{}ZY@{}ZY@{}ZY}
\rowstyle{\Large\arbfont} Aa, & Bb, & Cc, & Ĉĉ, & Dd, & Ee, & Ff, \\
\rowstyle{\small} a & b & c, z & tsch & d & e & f \\[1ex]
\rowstyle{\Large\arbfont} Gg, & Ĝĝ, & Hh, & Ĥĥ, & Ii, & Jj, & Ĵĵ, \\
\rowstyle{\small} g & dsch & h & ch & i & j  & sh \\[1ex]
\rowstyle{\Large\arbfont} Kk, & Ll, & Mm, & Nn, & Oo, & Pp, & Rr, \\
\rowstyle{\small} k  & l & m & n & o & p & r \\[1ex]
\rowstyle{\Large\arbfont} Ss, & Ŝŝ, & Tt, & Uu, & Ŭŭ, & Vv, & Zz. \\
\rowstyle{\small} ss & sch & t & u & kurzes u & w & s \\
\end{tabu}
{\raggedleft\scriptsize(wie in \glqq{}lesen\grqq{})\par}

\end{center}

{\fr
\small \spaceout{Anmerkung:} ~\std{ĝ} lautet wie das engliſche \std{\glqq{}g\grqq{}} in \std{\glqq{}gentleman\grqq{}}; \std{ĵ —}wie das franzöſiſche \std{\glqq{}j\grqq{}} in \std{\glqq{}journal\grqq{}}; \std{u —} wie das kurze \glqq{}u\grqq{} in \glqq{}glauben\grqq{} (wird nur nach einem Vokal gebraucht). Bei mangelnden Typen im Druck erſetzt man \std{ĉ, ĝ, ĥ, ĵ, ŝ, ŭ} durch \std{ch, gh, hh, jh, sh, u.}}
\begin{center}
\large\gramsec{B)} \Large\bf\fr{Redetheile.}
\end{center}

{\fr \large
    \textbf{1.} Der beſtimmte \textbf{Artikel} iſt \std{la}, für alle Geſchlechter und Fälle, für die Einzahl und Mehrzahl. Einen unbeſtimmten Artikel gibt es nicht.
    
    \textbf{2.} Das \textbf{Hauptwort} bekommt immer die Endung \std{o}. Der Plural bekommt die Endung \std{j}. Es gibt nur zwei Fälle: Nominativ und Akkuſativ; der letztere entſteht aus dem Nominativ, indem die Endung \std{n} hinzugefügt wird. Die übrigen Fälle werden vermittelſt der Präpoſitionen ausgedrückt: der Genitiv durch \std{de} (von), der Dativ durch \std{al} (zu), der Ablativ durch \std{kun} (mit), oder andere, dem Sinne entſprechende, Präpoſitionen. Z.~B. \std{la patr$'$o}, der Vater; \std{al la patr$'$o}, dem Vater; \std{la patr$'$o$'$n}, den Vater; \std{la patr$'$o$'$j$'$n}, die Väter (Akkuſativ).
    
    \textbf{3}. Das \textbf{Eigenſchaftswort} endet immer auf \std{a}. Deklinationen wie beim Subſtantiv. Der Komparativ wird mit Hülfe des Wortes \std{pli} (mehr), der Superlativ durch \std{plej} (am meiſten) gebildet. Das Wort \glqq{}als\grqq{} heißt \std{ol}. Z.~B.: \std{pli blank$'$a ol neĝ$'$o,} weißer als Schnee.
    
    \textbf{4.} Die \textbf{Grundzahlwörter} (undeklinirbar) ſind folgende: \std{unu (1), du (2), tri (3), kvar (4), kvin (5), ses (6), sep (7), ok (8), naŭ (9), dek (10), cent (100), mil (1000).} Zehner und Hunderte werden durch einfache Anreihung der Zahlwörter gebildet; z.~B.: \std{kvin$'$cent tri$'$dek tri = 533.} Ordnungszahlwörter entſtehen, indem ſie die Endung des Adjektivs annehmen; z.~B. \std{kvar$'$a}, vierter. Vervielfältigungszahlwörter \std{—} durch Einſchiebung des Suffixes \std{obl;} z.~B.: \std{tri$'$obl$'$a,} dreifach. Bruchzahlwörter \std{—} durch \std{on}; z.~B. \std{kvar$'$on$'$o,} ein Viertel. Sammelzahlwörter \std{—} durch \std{op}; z.~B. \std{du$'$op$'$e,} ſelbander. Diſtributive Zahlwörter \std{—} durch das Wort \std{po;} z.~B. \std{po kvin,} zu fünf. Außerdem gibt es Subſtantiv- und Adverbialzahlwörter; z.~B. \std{cent$'$o,} das Hundert, \std{du$'$e}, zweitens.
    
    \textbf{5.} Die persönlichen \textbf{Fürwörter} sind: \std{mi} (ich), \std{vi} (du, Ihr), \std{li} (er), \std{ŝi} (sie), \std{ĝi} (es; von Thieren oder Sachen), \std{si} (sich), \std{ni} (wir), \std{ili} (sie [Mehrzahl]), \std{oni} (man). Poſſeſſive Pronomina werden durch die Hinzufügung der Endung des Adjektivs gebildet. Die Pronomina werden gleich den Subſtantiven deklinirt. Z.~B.: \std{mi$'$a,} mein, \std{mi$'$n,} mich.
       
    \textbf{6.} Das \textbf{Zeitwort} hat weder Personen noch Mehrzahl; z.~B. \std{mi far$'$as,} ich mache; \std{la patr$'$o far$'$as,} der Vater macht; \std{ili far$'$as,} sie machen.

\begin{minipage}{\textwidth}
\begin{center}
\textbf{Formen des Zeitwortes :}
\end{center}

        \std{a)} Das Präſens endet auf \std{as}; z.~B. \std{mi far$'$as,} ich mache.
\end{minipage}
        
        \std{b)} Die vergangene Zeit \std{―} auf \std{is}; z.~B. \std{li far$'$is,} er hat gemacht.
        
        \std{c)} Das Futurum \std{―} auf \std{os}; z.~B. \std{ili far$'$os,} ſie werden machen.
        
        \std{ĉ)} Der Konditionalis \std{―} auf \std{us}; z.~B. \std{ŝi far$'$us,} ſie würde machen.
        
        \std{d)} Der Imperativ \std{―} auf \std{u}; z.~B. \std{far$'$u,} mache, macht; \std{ni far$'$u,} laſſet uns machen.
        
        \std{e)} Der Infinitiv \std{―} auf \std{i}; z.~B. \std{far$'$i,} machen.
        
        \std{f)} Partizipium präſentis aktivi \std{―} auf \std{ant}; z.~B. \std{far$'$ant$'$a}, machender; \std{far$'$ant$'$e,} machend.
        
        \std{g)} Partizipium perfekti aktivi \std{―} \std{int}; z.~B. \std{far$'$int$'$a,} der gemacht hat.
        
        \std{ĝ)} Partizipium futuri aktivi \std{―} \std{ont}; \std{far$'$ont$'$a,} der machen wird.
        
        \std{h)} Partizipium präſentis paſſivi \std{―} \std{at}; z.~B. \std{far$'$at$'$a,} der gemacht wird.
        
        \std{ĥ)} Partizipium perfekti paſſivi \std{―} \std{it}; z.~B. \std{far$'$it$'$a,} gemacht.
        
        \std{i)} Partizipium futuri paſſivi \std{―} \std{ot}; \std{far$'$ot$'$a,} der gemacht werden wird.

    Alle Formen des Paſſivs werden mit Hülfe der entſprechenden Form des Wortes \std{est} (ſein) und des Partizipium paſſivi des gegebenen Zeitwortes gebildet, wobei die Präpoſition \std{de} gebraucht wird; z.~B. \std{ŝi est$'$as am$'$at$'$a de ĉiu$'$j,} ſie wird von Allen geliebt.
    
    \textbf{7.} Das \textbf{Adverbium} endet auf \std{e}; Komparation wie beim Adjektiv. Z.~B: \std{mi$'$a frat$'$o pli bon$'$e kanta$'$as ol mi} = mein Bruder ſingt beſſer als ich.
    
    \textbf{8.} Alle \textbf{Präpoſitionen} regieren den Nominativ.
}

\begin{center}
\Large \bf C) \fr{Allgemeine Regeln.}
\end{center}

\enlargethispage{-\baselineskip}
{\fr \large
    \textbf{9.} Jedes Wort wird geleſen ſo wie es geſchrieben ſteht.
    
    \textbf{10.} Der Accent fällt immer auf die vorletzte Silbe.
    
    \textbf{11.} Zuſammengeſetzte Wörter entſtehen durch einfache Anreihung der Wörter, indem man ſie durch hochſtehende Striche trennt\footnote{\fr \small Im Briefwechſel mit ſolchen Perſonen, die der internationalen Sprache ſchon mächtig ſind, oder in Werken, die für eben ſolche Perſonen beſtimmt ſind; fallen die hochſtehenden Striche zwiſchen den verſchiedenen Theilen der Wörter weg.}. Das Grundwort kommt zuletzt. Grammatikaliſche Endungen werden als ſelbſtſtändige Wörter betrachtet. Z.~B. \std{vapor$'$ŝip$'$o} (Dampfschiff) beſteht aus \std{vapor,} Dampf, \std{ŝip,} Schiff, und \std{o}=Endung des Subſtantivs.

    \textbf{12.} Wenn im Satze ein Wort vorkommt, das von ſelbſt eine verneinende Bedeutung hat, ſo wird die Negation ne weggelaſſen; z.~B. \std{mi nenio$'$n vid$'$is,} ich habe Nichts geſehen.
    
    \textbf{13.} Auf die Frage \glqq{}wohin\grqq{} nehmen die Wörter die Endung des Akkuſativs an; z.~B. \std{tie,} da; \std{tie$'$n,} dahin; \std{Varsovi$'$o$'$n,} nach Warschau.
    
    \textbf{14.} Jede Präpoſition hat eine beſtimmte, feſte Bedeutung; iſt es aber aus dem Sinne des Satzes nicht erſichtlich, welche Präpoſition anzuwenden iſt, ſo wird die Präpoſition \std{je} gebraucht, welche keine ſelbſtſtändige Bedeutung hat; z.~B. \std{ĝoj$'$i je tio,} ſich darüber freuen; \std{rid$'$i je tio,} darüber lachen; \std{enu$'$o je la patr$'$uj$'$o,} Sehnſucht nach dem Vaterlande, \&c. Die Klarheit leidet keineswegs darunter, da doch dasſelbe in allen Sprachen geſchieht, nämlich, daß man in ſolchen Fällen eine beliebige Präpoſition begraucht, wenn ſie nur einmal angenommen iſt. In der internationalen Sprache wird in ſolchen Fällen immer nur die eine Präpoſition \std{je} angewendet. Statt der Präpoſition \std{je} kann man auch den Akkuſativ ohne Präpoſition gebrauchen, wo kein Doppelſinn zu befürchten iſt.
    
    \textbf{15.} Sogenannte Fremdwörter, d.~h. ſolche Wörter, welche die Mehrheit der Sprachen aus einer und derſelben fremden Quelle entlehnt hat, werden in der internationalen Sprache unverändert gebraucht, indem ſie nur die internationale Orthographie annehmen; aber bei verſchiedenen Wörtern, die eine gemeinſame Wurzel haben, ist es beſſer, nur das Grundwort unverändert zu gebrauchen, die abgeleiteten Wörter aber \std{―} nach den Regeln der internationalen Sprache zu bilden; z.~B. Theater, \std{teatr$'$o;} theatraliſch, \std{teatr$'$a.}
    
    \textbf{16.} Die Endung des Subſtantivs und des Artikels kann ausgelaſſen werden, indem man dieſelbe durch einen Apoſtroph erſetzt; z.~B. \std{Ŝiller’,} ſtatt \std{Ŝiller$'$o;} \std{de l’ mond$'$o,} ſtatt \std{de la mond$'$o.} 
}
\newpage
