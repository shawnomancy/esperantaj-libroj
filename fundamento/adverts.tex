%
% adverts page
%
\thispagestyle{plain}
{\centering
{\footnotesize \so{LIBREJO HACHETTE KAJ K\textsuperscript{o}, PARIS}\par}
\rule[0.9ex]{\textwidth}{0.4pt}

\times{\Large TUTMONDA}\\[1ex]
\parbox{0.7\textwidth}{\times{\Huge \spaceoutmed{Jarlibro}\vskip 1ex\hfill \spaceoutmed{Esperantista}}}\\[1ex]
{\footnotesize ENHAVANTA LA}\par

{\csfont{\textbf{\spaceoutless{Adresaron de D\textsuperscript{ro} ZAMENHOF}}}\par}
}

{\footnotesize Tiu ĉi jarlibro eldoniĝas ĉiujare dum Marto.  Ĝi entenas : pli ol 20,000 adresojn de Esperantistoj de l’ tuta mondo, plenajn sciigojn pri la propagandaj Societoj, pri la libroj, la gazetoj, la ĉiuspecaj eldonaĵoj en Esperanto.}

\newlength{\tempindent}
\setlength{\tempindent}{\parindent}
\setlength{\parindent}{0pt}
{\footnotesize Unu volumo, in-16, broŝurita \dotfill 2 fr. 50}
\setlength{\parindent}{\tempindent}

{\centering
\rule[0.9ex]{\textwidth}{0.4pt}
{\times{\LARGE Internacia}}

\vspace{1em}

{\times{\Huge \spaceoutmed{Scienca Revuo}}}

\vspace{1em}

{\small \bf Monata scienca revuo redaktita} \\
\scalebox{1.5}[1]{\spaceoutmed{EN ESPERANTO}} \\
{\large Eldonata de la 1\textsuperscript{a} de Januaro 1904}

\rule[0.9ex]{13mm}{0.4pt}

{\footnotesize PATRONARO :\par}
}

{\footnotesize Franca Societo de Fiziko, Internacia Societo de Elektristoj, S\textsuperscript{oj} Adelsköld, Appel, d’Arsonval, Baudoin de Courtenay, Becquerel, Berthelot, Bouchard, Brouardel, Deslandres, G\textsuperscript{al} Sébert, anoj de diversaj akademioj.}

{\centering
\begin{tabu} to \textwidth{YY}
\scriptsize REDAKCIO : & \scriptsize ADMINISTRACIO : \\
\fjallafont{\spaceout{P. FRUICTIER}} & \fjallafont{\spaceout{HACHETTE \& K\textsuperscript{o}}} \\
\footnotesize 27, {\it boulevard Arago,} & \footnotesize 79, {\it boulevard Saint-Germain,} \\
\footnotesize\it PARIS & \footnotesize\it PARIS
\end{tabu}

\rule{13mm}{0.4pt}

{\scriptsize JARA ABONO\par}

Francujo \dotfill 6 fr. 50 | Ceteraj landoj \dotfill 7 fr.

\footnotesize\fauxsc{unu numero} : 60 centimoj\par

\rule{\textwidth}{0.4pt}
}
\begin{tabu} to \textwidth{X@{}r}
\footnotesize D\textsuperscript{ro} \fauxsc{Helte.} — {\bf Pri la Teorio de l’Jonoj} \dotfill & \footnotesize » 30\\
\footnotesize \fauxsc{Mendelejev.} — {\bf Provo de Kemia Kompreno de l’Monda Etero} \dotfill & \footnotesize » 30
\end{tabu}

