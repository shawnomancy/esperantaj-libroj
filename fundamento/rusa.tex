%
% Esperanta gramatiko por rusoj
%
\label{gram:rusa}
\markboth{FUNDAMENTO DE ESPERANTO}{ГРАММАТИКА}
\thispagestyle{plain}
\begin{center}
\phantomsection
\narrow{\LARGE\textbf{ГРАММАТИКА}}
\addcontentsline{toc}{section}{Грамматика (Gramatiko Rusa)}
\selectlanguage{russian}

\rule{13mm}{0.4pt}
\vspace{1em}

{\large\gramsec{A) АЗБУКА}}
\vspace{1em}

\setstretch{1}
\begin{tblr}{YYYYYYY}
\SetRow{font=\Large\arbfont} Aa, & Bb, & Cc, & Ĉĉ, & Dd, & Ee, & Ff, \\
\SetRow{font=\small} а & б & ц & ч & д & э & ф \\[1ex]
\SetRow{font=\Large\arbfont} Gg, & Ĝĝ, & Hh, & Ĥĥ, & Ii, & Jj, & Ĵĵ, \\
\SetRow{font=\small} г & дж & (г, х) & х & и & й  & ж \\[1ex]
\SetRow{font=\Large\arbfont} Kk, & Ll, & Mm, & Nn, & Oo, & Pp, & Rr, \\
\SetRow{font=\small} к  & л & м & н & о & п & п \\[1ex]
\SetRow{font=\Large\arbfont} Ss, & Ŝŝ, & Tt, & Uu, & Ŭŭ, & Vv, & Zz. \\
\SetRow{font=\small} с & ш & т & у & у (краткое) & в & з \\
\end{tblr}
\end{center}

{\footnotesize \br{Примѣчаніе I.} Буква \es{h} произносится какъ \es{h} въ языкахъ нѣмецкомъ, латинскомъ и другихъ; буква \es{ŭ} (которая употребляется только послѣ гласной), произносится какъ корокое \es{u} въ нѣмецкомъ словѣ \es{kaufen} или въ латинскомъ \es{laudo.} Лица, не знающія никакой другой азбуки кромѣ русской, могутъ на первыхъ порахъ произносить \es{h} какъ русское \br{х}, а \es{ŭ} какъ русское \br{у}.

\br{Примѣчаніе II.} Типографіи, не имѣющія буквъ \es{ĉ, ĝ, ĥ, ĵ, ŝ, ŭ}, могутъ на первыхъ порахъ употреблять \es{ch, gh, hh, jh, sh, u.}}
\begin{center}
{\large\gramsec{B) ЧАСТИ РѢЧИ}}
\end{center}

\br{1. Члена} неопредѣленнаго нѣтъ; есть только опредѣленный (\es{la}), одинаковый для всѣхъ родовъ, падежей и чиселъ.

{\footnotesize \br{Примѣчаніе.} Употребленіе члена такое же, какъ въ языкахъ нѣмецкомъ, французскомъ и другихъ. Лица, для которыхъ употребленіе члена представляетъ трудности, могутъ совершенно его не употреблять.}
    
    \br{2. Существительное} всегда оканчивается на \es{o}. Для образованія множественнаго числа прибавляется окончаніе \es{j}. Падежей есть только два: именительный и винительный; послѣдний получается изъ именительнаго прибавленіемъ окончанія \es{n}. Остальные падежи выражаются помощью предлоговъ: для родительнаго ― \es{de} (отъ), для дательнаго ― \es{al} (къ), для творительнаго ― \es{per} (посредствомъ) или другіе предлоги соотвѣтственно смыслу. (\br{Примѣры:} \es{patr\|o} отецъ, \es{al patr\|o} отцу, \es{patr\|o\|n} отца (винит. пад.), \es{por patr\|o\|j} для отцовъ, \es{patr\|o\|j\|n} отцовъ, (винит. пад.).
    
    \br{3. Прилагательное} всегда оканчивается на a. Падежи и числа какъ у существительнаго. Сравнительная степень образуется помощью слова \es{pli} (болѣе), а превосходная ― \es{plej} (наиболѣе); слово „чѣмъ“ переводится ol. (\br{Прим.:} \es{pli blank\|a ol neĝ\|o} бѣлѣе снѣга; \es{mi hav\|as la plej bon\|a\|n patr\|in\|o\|n} я имѣю самую лучшую мать).
    
    \br{4. Числительныя} количественныя (не склоняются): \es{unu} (1), \es{du} (2), \es{tri} (3), \es{kvar} (4), \es{kvin} (5) \es{ses} (6), \es{sep} (7), \es{ok} (8), \es{naŭ} (9), \es{dek} (10), \es{cent} (100), \es{mil} (1000). Десятки и сотни образуются простымъ сліяніемъ числительныхъ. Для образованія порядковыхъ прибавляется окончаніе прилагательнаго; для множительныхъ ― вставка \es{obl}, для дробныхъ ― \es{on}, для собирательныхъ ― \es{op}, для раздѣлительныхъ ― слово \es{po}. Кромѣ того могутъ быть числительныя существительныя и нарѣчныя. (\br{Примѣры:} \es{Kvin\|cent tri\|dek tri} = 533; \es{kvar\|a} четвертый; \es{unu\|o} единица; \es{du\|e} во вторыхъ; \es{tri\|obl\|a} тройной, \es{kvar\|on\|o} четверть; \es{du\|op\|e} вдвоемъ; \es{po kvin} по пяти).
    
    \br{5. Мѣстоименія} личныя: \es{mi} (я), \es{vi} (вы, ты), \es{li} (онъ), \es{ŝi} (она), \es{ĝi} (оно; о вещи или о животномъ), \es{si} (себя), \es{ni} (мы), \es{ili} (они, онѣ), \es{oni} (безличное множественнаго числа); притяжательныя образуются прибавленіемъ окончанія прилагательнаго. Склоненіе какъ у существительныхъ (\br{Примѣры:} \es{mi\|n} меня (винит.); \es{mi\|a} мой).
    
    \br{6. Глаголъ} по лицамъ и числамъ не измѣняется (наприм.: \es{mi far\|as} я дѣлаю, \es{la patr\|o far\|as} отецъ дѣлаетъ, \es{ili far\|as} они дѣлаютъ). Формы глагола:
    
        a) Настоящее время принимаетъ окончаніе \es{as} (напримѣръ: \es{mi far\|as} я дѣлаю).
        
        b) Прошедшее ― \es{is} (\es{li far\|is} онъ дѣлатъ).
        
        c) Будущее ― \es{os} (\es{ili far\|os} они будутъ дѣлать).
        
        ĉ) Условное наклоненіе ― \es{us} (\es{ŝi far\|us} она бы дѣлала).
        
        d) Повелительное наклоненіе ― \es{u} (\es{far\|u} дѣлайте).
        
        e) Неопредѣленное наклоненіе ― \es{i} (\es{far\|i} дѣлать).
        
\begin{center}
Причастія (и дѣепричастія):
\end{center}
    
        f) Дѣйствит. залога настоящаго времени ― \es{ant} (\es{far\|ant\|a} дѣлающій, \es{far\|ant\|e} дѣлая).
        
        g) Дѣйствит. залога прошедш. времени ― \es{int} (\es{far\|int\|a} сдѣлавшій).
        
        ĝ) Дѣйствит. залога будущ. времени ― \es{ont} (\es{far\|ont\|a} который сдѣлаетъ).
        
        h) Страдат. залога настоящ. времени ― \es{at} (\es{far\|at\|a} дѣлаемый).
        
        ĥ) Страдат. залога прошедш. времени ― \es{it} (\es{far\|it\|a} сдѣланный).
        
        i) Страдат. залога будущ. времени ― \es{ot} (\es{far\|ot\|a} имѣющій быть сдѣланнымъ).

    Всѣ формы страдательнаго залога образуются помощью соотвѣтственной формы глагола \es{est} (быть) и причастія страдательнаго залога даннаго глагола; предлогъ при этомъ употребляется de (\br{Примѣръ:} \es{ŝi est\|as am\|at\|a de ĉiu\|j} она любима всѣми).
    
    \br{7. Нарѣчія} оканчиваются на \es{e}. Степени сравненія какъ у прилагательныхъ (\br{Примѣръ:} \es{mi\|a frat\|o pli bon\|e kant\|as} ol mi мой братъ лучше меня поетъ).
    
    \br{8. Предлоги} всѣ требуютъ именительнаго падежа.

\begin{samepage} 
\begin{center}
{\large\gramsec{C) ОБЩІЯ ПРАВИЛА.}}
\end{center}

    \br{9.} Каждое слово читается такъ, какъ оно написано.
\end{samepage}

    \br{10.} Удареніе всегда находится на предпослѣднемъ слогѣ.
    
    \br{11.} Сложныя слова образуются простымъ сліяниемъ словъ (главное на концѣ), которыя пишутся вмѣстѣ, но отдѣляются другъ отъ друга черточкой\footnote{Въ письмахъ и сочиненіяхъ, назначенныхъ для лицъ, владѣющихъ уже международнымъ языкомъ, черточки между частями словъ не употребляются.}.) Грамматическія окончанія разсматриваются также какъ самостоятельныя слова (\br{Примѣръ:} \es{vapor\|ŝip\|o,} пароходъ ― изъ \es{vapor} паръ, \es{ŝip} корабль, \es{o} окончаніе существительныхъ).
    
    \br{12.} При другомъ отрицательномъ словѣ отрицаніе ne опускается (\br{Примѣръ:} \es{mi neniam vid\|is} я никогда не видалъ).
    
    \br{13.} На вопросъ „куда“ слова принимаютъ окончаніе винительнаго падежа (\br{Примѣры:} \es{tie} тамъ ― \es{tie\|n} туда; \es{Varsovi\|o\|n} въ Варшаву).
    
    \br{14.} Каждый предлогъ имѣтъ опредѣленное постоянное значеніе; если же нужно употребить предлогъ, а прямой смыслъ не указываетъ, какой именно, то употребляется предлогъ je, который самостоятельнаго значенія не имѣетъ (\br{Примѣры:} \es{ĝoj\|i je tio} радоваться этому; \es{rid\|i je tio} смѣяться надъ этимъ; \es{enu\|o je la patr\|uj\|o} тоска по родинѣ и т. д.).
    
    Ясность отъ этого не страдаетъ, потому что во всѣхъ языкахъ въ этихъ случаяхъ употребляется какой угодно предлогъ, лишь бы обычай далъ ему санкцію; въ междунатодномъ же языкѣ санкція на всѣ подобные случаи дана \br{одному} предлогу \es{je}.
    
    Вмѣсто предлога \es{je} можно также употребить винительный падежъ.
    
    \br{15.} Такъ называемыя „иностранныя“ слова, т.е. такія, которыя большинствомъ языковъ взяты изъ одного чужого источника, употребляются въ международномъ языкѣ безъ измѣненія, принимая только орѳографію этого языка; но при различныхъ словахъ одного корня лучше употреблять безъ измѣненія только основное слово, а другія образовать по правиламъ международнаго языка (\br{Примѣръ:} театръ ― \es{teatr\|o,} но театральный ― \es{teatr\|a}).
    
    \br{16.} Окончанія существительнаго и члена могутъ быть опущены и замѣнены апострофомъ (\br{Примѣры:} \es{dom’} вм. \es{dom\|o; de l’mond\|o} вм. \es{de la mond\|o.}
    
\newpage
