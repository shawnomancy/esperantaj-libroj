\begin{center}
\footnotesize (El la lingvo sveda tradukis B. G. \fsc{Jonson el Östersund})
\end{center}

   En la voja\^go, kiun mi entreprenis anta\u u nelonge de \^Cikago al
Nov-Jorko, mi trovis, kiam mi maldormi\^gis en la mateno, ke la
vagonaro haltis. La kelnero rakontis, ke \^gi staras jam 1 1/2
horojn. Mi vestis min, kaj elirinte eksteren, mi trovis, ke ni estas
\^ce malgranda stacio sur la kampo. Mi eniris en la man\^gosalonon
kaj matenman\^gis kaj poste promenadis sur la perono.

   Sur la lokomotivo la lokomotivestro sidis sola kaj atendis. Mi haltis
kaj babilis kun li kelkan tempon pri la ma\^sino. Kiam mi proponis
al li cigaron, kiun li kun danko akceptis, li min petis eniri en la
malgrandan kaverneton de lia lokomotivo.

   La estro, bela, granda viro en la a\^go de 40 jaroj, klarigis al mi,
kiel oni uzas la apartajn partojn de la ma\^sino. \^Cio, kio nur
povis esti briligita, radiis kiel la suno; \^car la lokomotivestroj
estas egale fieraj, kiam iliaj ma\^sinoj estas ornamitaj, kiel
domestrino, kiam \^siaj \^cambroj estas ordigitaj.

 --- Kia ornamo tio \^ci estas? mi demandis, montrante sur ion, kiu
estis simila je insekto kaj, enkadrigita en oran kadron, pendis sur
la mureto. La estro ridis. \^Gi estas malpli ornamo ol
memora\^{\j}o, li diris; mi \^gin pendigis tien \^ci, \^car \^gi
savis mian vivon kaj la vivon de 250 aliaj homoj.

 --- Kiel do insekto povas savi la vivon de homoj? mi demandis.

 --- Mi \^gin rakontos al vi. Ni havas multon da tempo anta\u u la
forvoja\^go.

   Mi sidigis min sur la lokon de la forestanta hejtisto kaj preparis
min al la a\u uskultado.

 --- \^Gi okazis anta\u u nelonge, komencis la estro, anta\u u nur unu jaro,
en printempo. Mi voja\^gis sur tiu samo vojo, kiel nun, kaj havis
tiun saman ma\^sinon, kiel nun, --- la karan 499. Mia hejtisto estis
tiu sama, kiun mi havas nun, --- Jim Moode. Jim estas bonega knabo,
sed tre ema kredi spiritojn, son\^gojn kaj anta\u usignojn. Komence
mi ridadis je lia malprudenteco, sed nun mi jam ne \^sercas tiel
multe pri li, --- de tiu tempo, kiam ni vidis la nigran virinon.

   Mi devis forlasi M. \^cirka\u u la unua horo matene kaj esti en S. je
la sesa horo. En tiu nokto blovis terura ventego kaj la pluvo
faladis per riveroj jam de la komenco de la vespero. Kiam mi venis
al la lokomotivejo, la ventego blovis plej terure.

   Kiam Jim kaj mi estis en la vojo al la stacio kun la lokomotivo, li
diris: Ni havos malgajan veturon, Frank; mi dezirus, ke ni estu jam
feli\^ce en S.

   Mi ridis kaj demandis: --- kio tiel malkura\^gigas vin en tiu \^ci
nokto?

 --- Mi sentas, ke io okazos, li diris.

   Por diri la veron, mi anka\u u sentis min iom malkura\^ga mem.

   La vagonaro, kiun mi devis konduki, estis longa, multepeza kaj
konsistis preska\u u nur el personvagonoj. Mi fari\^gis nerva \^ce
la penso havi sub mia flego kaj respondeco tiel multajn centojn da
personoj.

   Mi ridis je mi mem pro mia malkura\^go, kiam mi unuigis la lokomotivon
kun la vagonaro kaj poste esploris kaj trovis, ke \^cio estas preta.
La signo eksonis, kaj ni ekvoja\^gis en la blovegadon. La mallumo
estis netrapenetrebla, nur de la lanterno sur la anta\u ua\^{\j}o de
la lokomotivo estis distrata anta\u uen elektra lumo. Jim metadis
diligente en la fajron kaj subtenadis la plej altan premon de aero
tiel, ke ni iris anta\u uen, kiel furioj.

   Apud la unua stacio, kio ni haltis, por preni akvon, mi ekzamenis
precize, \^cu \^cio estas en ordo, kaj Jim esploris la lanternon.
\^Cio estis bona, kaj ni veturis pluen.

   La mallumo fari\^gis, se estas eble, pli firma. La pluvo falis
ankora\u u per riveroj. Subite mi vidis tra la pluvo kaj nebulo
glitantan anta\u u ni gigantan virinan figuron, envolvitan en longan
nigran mantelon, kiu flugis en la blovado. \^Si \^{\j}etis siajn
brakojn posten kaj anta\u uen, \^gis \^si nevidebli\^gis.

   Mi estis tute muta de miro kaj forgesis fari ian signon al Jim, kiu
staris anta\u u la forno. Kiam li ekrigardis returnen, li ekkriis:
Halo, Frank! kio estas? vi elrigardas, kiel vidinto de spiritoj!

   Mi nenion respondis. Miaj pensoj estis okupitaj je la stranga figuro,
kiun mi vidis.

   Nun ni estis proksime de Rock Creek, kie ponto kondukas super
profunda rivero.

   Mi fari\^gis pli nerva ol anta\u ue. Ni veturis rapide, kaj signo estis
donita de la stacio de Rock Creek, kiu estas nur unu mejlon
malproksima de la ponto. Kiam ni pasis preter la stacio, mi a\u
udis, ke Jim ekkriis. Mi kuris al li kaj vidis lin tremantan de
teruro. Li montris eksteren en la mallumon, kaj kiam mi ekrigardis,
teruro atakis min mem.

   Tie sur la reloj montri\^gis tiu sama giganta virino, kiel anta\u ue,
jen tran\-kvi\-la, jen en la plej sova\^ga danco. --- Frank,
murmuretis Jim kun malfacileco, ne iru sur la ponton! Pro la
\^cielo, ne faru tion \^ci! Ne iru, anta\u u ol vi scias, ke \^cio
estas en ordo!

   Mi ne povis kontra\u ustari al la penso haltigi la vagonaron kaj
malfermis la ventolilon kiel eble plej forte. Apena\u u ni haltis,
mi povis a\u udi la akvon, kiu mu\^gis en Rock Creek rekte anta\u u
ni. Kiam mi eliris el la ma\^sino, la konduktoro venis al mi
renkonte.

 --- Kio estas? Kio estas? li demandis. Mi sentis min tre konfuzita.
Nun estis videbla jam nenia giganta virino. Ni ne povis vidi pli
malproksime ol unu metron a\u u du anta\u uen super la reloj. Mi
nenion vidis, sed diris: --- Mi ne scias, kio estas, sed \^sajnis al
mi, ke mi vidis grandan nigran spiriton, kiu etendis la brakojn kaj
faris al mi signon ne iri plu. La konduktoro rigardis min tute
mirigita. --- \^Cu vi estas freneza, Frank? li diris. --- Oni \^gin
preska\u u povus kredi. Sed ni estas ja en la proksimeco de la
rivero kaj ni povas esplori.

   Ni prenis niajn lanternojn kaj iris anta\u uen. Jim ricevis la ordonon
gardi la ma\^sinon. Sed apena\u u ni faris kelkajn dekojn da
pa\^soj, ni haltis, rigidigitaj de teruro. Anta\u u niaj piedoj
estis profunda fa\u uko, kie la rivero mu\^gis, \^sveligita de la
printempaj pluvoj. Kiam ni nin returnis, ni vidis la nigran virinan
figuron, kiu dancadis en sova\^gaj turnoj. La konduktoro rigardis
anta\u ue la fa\u ukon, poste min.

 --- \^Cu tion \^ci vi vidis, kiam vi haltigis la vagonaron?

 --- Jes. --- Io alia, ol feli\^co, nin savis tiun \^ci nokton.

   Ni reiris malrapide al la vagonaro, plenaj de pensoj kaj kun malgaja
animo. Diversaj veturantoj venis renkonte al ni. Inter ili sin
trovis dekokjara junulo el \^Cikago, kiu estis pli rapidepensanta,
ol iu el ni. Kiam li ekvidis la nigran virinon, li iris al la
lokomotivo kaj enrigardis en la lanternojn, kiuj staris tie. --- Tie
\^ci estas nia nigra virino! diris la \^Cikaga junulo. Kaj tie estis
efektive tiu sama insekto, kiun vi vidas sub tiu \^ci vitro. Kiam mi
malfermis la lanternon, \^gi flugis kontra\u u la reflektoron.

   Jen estas la tuta historio, mia sinjoro. Kiam la insekto flugis
anta\u u la elektra lumo, \^gi \^{\j}etis ombron, kiu similis
virinon, svingantan la brakojn. Ni ne scias, kiel \^gi eniris, sed
kredeble \^gi eniris tiam, kiam Jim esploris la lanternon apud la
akvostacio. Kiel ajn \^gi estis, \^gi savis nian vivon per tio, ke
\^gi min timigis per tiu nigra virino.

   Jen estas la ka\u uzo, ke tiu \^ci malgranda insekto estas sub vitro kaj
en kadro. Tio \^ci estas, por min memorigi, kiel ni estis savitaj
per tiu \^ci insekto. Jes, vi nomas \^gin okazo, --- mi kredas, ke
\^gi estis senda\^{\j}o de Dio.

 --- \^Cio en ordo! ekkriis la konduktoro, elirante el la telegrafa
stacio, portante paperon en la mano.

   Jim, la hejtisto, venis en la ma\^sinon kaj mi en mian vagonon.

\smallrule{}
