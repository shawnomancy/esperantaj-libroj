   La Bremena vapor\^sipo "Lloyd" staras en la haveno de Nov-Jorko, preta
al forveturo; jam la duan fojon estas donita la signalo per la
sonorilo, kiu admonas \^ciujn personojn, kiuj ne veturas, forlasi la
\^sipon. La kaldronoj de la \^sipo kun mu\^go kaj bruo ellasas
vaporon, la piloto estas sur la \^sipo, la amerika havena komisio
\^{\j}us per la trans\^{\j}etita ponto forlasas la \^sipon, trovinte
\^ciujn paperojn en ordo kaj doninte la permeson por elveturo el la
haveno.

   Jen supre sur la bordo de la haveno en la lasta minuto rapide
alkuras ankora\u u kelkaj kale\^soj. Sakoj de diversa grandeco estas
rapide \^{\j}etataj malsupren kaj poste kun granda rapideco portataj
sur la \^sipon kaj internen, kaj post kelkaj minutoj la \^sipo en
plena veturado forkuras de la havena ponto kaj, lasante la grandegan
monumenton de la libereco sur la insulo Bedloe dekstre, elveturas el
la golfo de Nov-Jorko.

   Tiu lasta paka\^{\j}o, kiu venis ankora\u u en la lasta minuto sur la
\^sipon, estis la po\^sto. La vapor\^sipo portas sur la posta flanko
la germanan standardon, nigra-blanka-ru\^gan, sur la anta\u ua masto
la standardon de Bremeno, blanka-ru\^gan, sur la meza masto la
standardon de la Nord-Germana Lloyd, bluan ankron kaj bluan
\^slosilon kruci\^gantajn sur blanka kampo, kaj sur la krucmasto la
standardon de la germana po\^sto, tute similan al la standardo de la
milita \^siparo, nur havantan en la maldekstra malsupra kampo flavan
kornon de po\^sto.

   Se ni sekvos post la po\^sta paka\^{\j}o, kiu tra la malfermoj estis
transportita en la malsupran parton de la \^sipo, ni venos en
sufi\^ce grandan salonon, kiu estas aran\^gita tiel same elegante,
kiel la unua kaj dua kajutoj de \^ciuj Lloyd'aj vapor\^sipoj.
Granda, per verda drapo kovrita tablo kun multaj fakoj super la
skriba plata\^{\j}o, apud kiu oportune povas labori kvar homoj,
prezentas la \^cefan meblon de tiu \^ci salono, almena\u u la
moveblan. Unu el la muroj la\u ulongaj estas okupita de polurita
masiva breto, kiu enhavas grandegajn fakojn, kies fundoj \^ciuj
estas klinitaj malrekte al la muro. Kelkaj \^sranketoj por
konservado de formularoj, stampiloj, koloriga\^{\j}o, fadenoj,
\^snuretoj, sigelvakso k. t. p. plenigas la meblaron de tiu \^ci
na\^ganta po\^sta oficejo. \^Sova pordo kondukas en grandegan
\^cambron najbaran, kiu tuta estas plenigita de paketoj; tiuj \^ci
lastaj estas tiel firme kunmetitaj kaj tiel lerte pakitaj unuj sur
la aliaj, ke ili \^ce \^ciu movo de la \^sipo ne ruli\^gas tien kaj
reen, sed e\^c en granda ventego, kiam la \^sipo saltas kaj batas,
ili ne tre multe dis\^suti\^gas; en alia okazo ne multaj el ili
venus E\u uropon.

   Se ni eliros en la koridoron, ni venos en elegante aran\^gitan
\^cambron kun du litoj unu super la dua kaj alia mebla\^{\j}o,
simila al tiu, kiun havas la voja\^gantoj de la dua klaso. La
\^cambro estas difinita por la du po\^staj oficistoj, la germana kaj
amerika. Malpli eleganta, sed tre praktike meblita apud\^cambro
estas difinita por la du suboficistoj, la germana kaj amerika.

   De la komenco de la jaro 1891, post longaj kaj detalaj traktadoj
inter Nord-Ameriko kaj Germanujo, oni fine decidis, ke \^ciu
vapor\^sipo devas esti akompanata de amerika kaj germana po\^sta
oficisto kaj po unu helpanto pro \^ciu el ili, por ke oni povu
ankora\u u en la da\u uro de la septaga transveturo tiel prepari la
po\^ston, ke \^gi, veninte Bremenon a\u u Nov-Jorkon, povu tuj esti
dissendata pluen, sen perdo de duontago a\u u de tuta tago por la
specigado. Anta\u ue la \^sipestro a\u u la unua oficiro akceptadis
la po\^ston; sub lia observo \^gi estis enportata de la amerikanoj,
poste la \^cambroj estis \^slosataj kaj la \^sipestro prenadis la
\^slosilon, kiun li transdonadis poste nur al la germana po\^sta
estraro en Bremeno a\u u en Nov-Jorko al la amerikaj po\^staj
oficistoj.

   Per tiu \^ci plej nova po\^sta progreso, la aran\^go de na\^gantaj
po\^staj oficejoj kun germana kaj amerika oficisto, oni atingas en
la dissendado de la po\^sta\^{\j}oj plifruigon de preska\u u tuta
tago. La oficistoj kaj suboficistoj, tiel la amerikaj kiel anka\u u
la germanaj, estas "borduloj", t. e. homoj, kiuj konas la marajn
cirkonstancojn, faris marajn veturojn kaj estas sufi\^ce harditaj
kontra\u u la mara malsano. Tiu \^ci cirkonstanco estas tre grava,
\^car en a\u utuno kaj vintro la transveturo de la vapor\^sipo estas
ofte tiel malkvieta, ke kredeble la tuta po\^sto restus nepreparita,
se la po\^staj oficistoj estus atakitaj de mara malsano.

   La oficistoj ricevas tablon oficiran, kompreneble je la kostoj de
Germanujo kaj de la Unuigitaj \^Statoj de Ameriko, kaj man\^gas kune
kun certa grupo da \^sipaj oficiroj. La suboficistoj en rilato de
tenado kaj man\^gado estas kalkulataj en tiu sama rango, kiel la
maatoj, t. e. homoj starantaj en rango inter \^sipsoldatoj kaj
\^sipoficiroj.

   En la veturo al Germanujo, sekve en tiu, en kiu ni en spirito
partoprenas, la amerika oficisto estas la kondukanto. La po\^sto,
kiu estas transveturigata, estas ja amerika propra\^{\j}o \^gis la
momento, en kiu \^gi en Bremeno estas transdonata al la germana
regna po\^sto; kontra\u ue, en la veturo al Nov-Jorko la germana
oficisto estas la kondukanto kaj komandanto en la oficejo.

   Ni rigardu iom pli proksime la po\^ston, kiu estas transportata
de Nov-Jorko al Bremeno. \^Gi enhavas anta\u u \^cio la tutan
po\^ston, kiu kolekti\^gis en Nov-Jorko kaj en \^gia \^cirka\u
ua\^{\j}o, kaj tio \^ci estas ne malgranda, \^car inter Nov-Jorko
kaj Germanujo estas komercaj komuniki\^goj, kiuj prezentas la spezon
de centoj da milionoj markoj \^ciujare. Al tio \^ci venas la tuta
po\^sto de la okcidentaj Unuigitaj \^Statoj \^gis San-Francisko kaj
plu ankora\u u parto de la po\^sto Azia kaj A\u ustralia, kiu tra la
Granda Oceano venas per \^sipoj al San-Francisko, de tie \^ci iras
en la da\u uro de kelkaj tagoj per la fervojoj al Nov-Jorko kaj tie
estas \^sar\^gata sur la vapor\^sipojn, por esti transveturigita tra
la Atlantan Oceanon. Se oni konsideros, kiel granda estas la komerca
komuniki\^go de Germanujo kun la Unuigitaj \^Statoj de Ameriko, kiom
multaj centoj da miloj da elmigrantoj \^ciujare iras Amerikon,
restante ja \^ciam en rilatoj kun sia patrujo, kun siaj parencoj,
kiom multe da germanoj vivas en Ameriko, tiam oni facile komprenos,
ke la po\^sta komuniki\^go estas grandega kaj ke \^gi konstante
kreskas.

   Sed tiu \^ci po\^sto alportas objektojn ne sole por Germanujo, sed
anka\u u transire por Polujo, Rusujo, A\u ustro-Hungarujo, Turkujo,
Svedo-Norvegujo, e\^c por la norda Italujo. Precipe grandega nombro
da gazetoj estas sendataj sub banderolo el Ameriko E\u uropon, ne
alportante grandan \^gojon al la germanaj po\^staj oficistoj, kiuj
nomas tiujn \^ci gazetajn senda\^{\j}ojn "bastonoj". La amerikanoj
ordinare ne faldas la gazetojn kvarangule, kiel la e\u uropanoj, sed
kunrulas ilin kaj poste \^cirka\u uigas ilin per papera strio kun la
adreso. Tiuj \^ci rulitaj, rondaj, bastonformaj po\^sta\^{\j}oj
estas pli facile pakeblaj, sed la adresoj estas legataj \^ce la
specigado tre malfacile, kaj sperta oficisto specigas egalan nombron
da "bastonoj", t. e. amerikaj rulitaj gazetoj, trioble pli
malrapide, ol leterojn e\^c kun sufi\^ce malklaraj adresoj.

   Krom leteroj, banderoloj kaj \^{\j}us nomitaj po\^staj paketoj, oni devas
an\-ko\-ra\u u prilabori la leterojn rekomenditajn, t. e.
enskribitajn, kaj leterojn \^sar\^gitajn, t. e. enhavantajn monon
kaj kosta\^{\j}ojn. \^Ciuj pecoj kune, kiujn unu sola na\^ganta
po\^sto transportas, prezentas la nombron de \^cirka\u u 400 000
po\^sta\^{\j}oj de diversaj specoj; sed ofte tiu \^ci nombro
alti\^gas \^gis pli ol miliono da pecoj. La devo de la po\^staj
oficistoj, kiuj veturas kun la \^sipo, estas dividi la
po\^sta\^{\j}ojn la\u u la apartaj po\^staj kursoj. La objektoj,
kiuj estas difinitaj por la trairo transite, estas metataj en unu el
la fakoj de la grandaj bretoj, kaj kiam la \^sipo estas en movo,
tiam oni vidas, kial la fundo de la apartaj fakoj en la grandegaj
bretoj iras tiel malrekte malsupren. Se la fundo estus plata, tiam
\^ce \^ciu levi\^go de la \^sipo \^ciuj po\^sta\^{\j}oj elfalus, el
la diversaj fakoj la objektoj denove miksi\^gus inter si, kaj la
tuta laboro estus vana.

   Krom la transira komuniki\^go, kiu iras tra Berlino, se la afero
ne tu\^sas Holandon a\u u Svedo-Norvegujon, devas ja anta\u ue jam
esti pakitaj la kursoj, kiuj iras el Bremeno. \^Cefa diskruci\^gejo
por la norda, orienta kaj suda Germanujo estas Hannovero. La
transoceana po\^sto venas el Bremeno Hannoveron, kaj tie \^ci en la
da\u uro de kelkaj minutoj, dum la kruci\^go de la vagonaroj, la
grandegaj sakoj kun la amerikaj po\^sta\^{\j}oj devas esti
disdividitaj en la po\^stajn vagonojn de la fervojoj, por ke tuj
poste, en la tempo de la plua veturado, oni povu specigi ilin la\u u
la diversaj stacioj. La specigantaj oficistoj devas scii preska\u u
pri \^ciu loko en Germanujo, kie \^gi sin trovas kaj al kiu po\^sta
kurso \^gi apartenas. Sako post sako el la amerikaj po\^sta\^{\j}oj
estas malfermata, la enhavo estas \^sutata en korbojn kaj nun
dividata en la diversajn fakojn. Kiam unu fako estas tute a\u u
duone plena, tiam oni pakas la leterojn, po\^stajn kartojn,
po\^stajn asignojn, banderolojn k. t. p. en sakojn kun diversaj
surskriboj, kaj la rekomenditajn kaj monajn senda\^{\j}ojn denove en
aliajn sakojn, kiuj anka\u u portas sur si la surskribojn de la
respondaj po\^staj kursoj.

   Aparta ekzerciteco estas necesa, por legi la amerikajn adresojn. La
a\-me\-ri\-ka\-no ekzemple havas la kutimon meti tuj sub la nomo de
la adresato la straton kaj numeron, dum ni \^gin skribas ordinare
sub la nomo de la urbo. Al tio \^ci \^guste la amerikaj leteroj,
kiujn la elmigrintoj sendas hejmen, estas pleje skribitaj de homoj,
kiuj ne bone posedas la plumon kaj nur malofte okupas sin per
skribado de leteroj. Sur tia letero la diversaj partoj de la adreso
kuras unu sur alian, duone germane, duone angle, kaj la amerika
kolego, kiu estas kompetenta en tiaj aferoj, devas tiam helpi en la
de\^cifrado de la adresoj, tiel same kiel anka\u u la enportadon de
la rekomenditaj kaj monaj leteroj en la registrojn li faras kune kun
la germana kolego.

   Se la vetero estas bona kaj la preparado de la po\^sto iras bone, tiam
la po\^staj oficistoj ofte havas anka\u u tempon, por iri sur la
ferdekon kaj \^gui la belegan maran aeron; sed ordinare restas al
ili tempo nur por la man\^gado, dormado kaj por la plej necesa
ripozo vespere en la oficira salono a\u u en la fuma salono de la
dua kajuto. Ili devas forte labori, se ili volas veni al fino kun
sia laboro. Kiam la oceano estas traveturita kaj oni alproksimi\^gas
al la kanalo, tiam parto de la po\^sto devas esti tute preta, \^car
ja anka\u u anglaj po\^sta\^{\j}oj por Southampton estas alportitaj
el Ameriko.

   En la lastaj horoj komenci\^gas la fermado de la sakoj da leteroj.
\^Ciuj amerikaj sakoj estas jam malplenigitaj, kaj la specigita
enhavo migris en la sakojn germanajn. \^Ciu sako estas fermita per
\^snuro, kiun oni povas kuntiri. La finojn de tiu \^ci sufi\^ce dika
\^snuro oni tratiras tra du truoj en la fundo de lada pladeto, kiu
havas pli-malpli la grandecon de monero dumarka. Sur la fundo de tiu
\^ci pladeto staras la surskribo "U. S. Mail", t. e. po\^sto de la
Unuigitaj \^Statoj. La finojn de la \^snuro oni ligas, kaj en la
kavon de la pladeto oni nun fandas sigelvakson, kiu tute kovras la
\^snuron, kaj poste oni premas sur \^gin la amerikan sigelon, kiu
portas sur si la surskribon: "U. S. German Sea Post" (Mara po\^sto
de la Unuigitaj \^Statoj al Germanujo).

   Nun la po\^sta\^{\j}oj estas kvankam jam specigitaj la\u u la germanaj
kursoj, sed ili \^ciam ankora\u u estas amerika propra\^{\j}o kaj la
Unuigitaj \^Statoj de Ameriko estas respondaj por tiuj \^ci
objektoj.

   La fajra \^sipo en la Weser fari\^gas videbla. La vapor\^sipo
alproksimi\^gas al la hejma haveno. Jam prete pakitaj estas \^ciuj
sakoj, anka\u u la paketojn oni en la lastaj tagoj elprenis el la
kestoj kaj specigis ilin la\u u la kursoj kaj lokoj de difino. La
lumturo de Rothesand estas preterveturita, pli malrapide veturas la
Lloyda vapor\^sipo la\u u la Weser, \^gis \^gi venas al la granda
alvetura ponto de Nordenham. Kun bruo la ankroj estas delasitaj, la
estraro venas sur la \^sipon, kaj kun ili la reprezentantoj de la
germana imperia po\^sto, kiuj tuj iras en la \^sipan oficejon, kaj
tie \^ci en la apudesto de la amba\u u germanaj oficistoj, kiuj
kunvenis de trans la oceano, ili transprenas de la amerikaj
oficistoj la senda\^{\j}on. La diversaj pozicioj en la formularoj
estas komparataj, oni kvitancas, rapidaj manoj portas trans la
havena ponto la po\^stajn sakojn el la \^sipo sur la bordon, kie jam
staras la grandaj flavaj kale\^soj por la akceptado de la sakoj kaj
paketoj, kaj rapide oni veturas a\u u al la po\^sta oficejo en
Nordenham, a\u u al la stacidomo de la fervojo, kie en la vagonaron
estas enaran\^gitaj apartaj po\^staj vagonoj, kiuj la tutan
transmaran po\^ston portas al Bremeno. Tie \^ci oni disapartigas la
po\^stajn kursojn, kaj parton de la paketoj kaj sakoj oni lasas. Sed
la plej granda parto iras senpere kun la kuriera vagonaro, kiu
ali\^gas en Bremeno, pluen al Hannovero, kie \^ce la alveno de la
transmara po\^sto denove disvolvi\^gas viva po\^sta movado. El la
Bremenaj po\^staj vagonoj elflugas la sakoj kaj paketoj, ili estas
\^sar\^gataj en la po\^stajn vagonojn de la aliaj vagonaroj, kiuj
staras sur la diversaj relaj vojoj, kaj post kelkaj minutoj
elveturas preska\u u la\u u \^ciuj direktoj la po\^staj vagonaroj,
kiuj disportas la transmaran po\^ston por Germanujo kaj la alilando.

   La Berlina po\^sto sendas \^ciutage Hannoveron specigistojn, por ke
ili la leterojn difinitajn por Berlino tuj specigu la\u u la
po\^staj oficejoj. En la tagoj, kiam venas la transmara po\^sto
--- kaj tio \^ci nun havas lokon kelkajn fojojn en la semajno
--- tiuj \^ci specigistoj ricevas helpantojn, kaj dum la vagonaro
kuras de Hannovero Berlinon, ili uzas sian tutan forton kaj sian
mirindan sciadon de la Berlinaj stratoj kaj domoj, por tiel pretigi
la po\^sta\^{\j}ojn, ke ili en Berlino jam de la stacidomo povas
esti disdividitaj per kale\^setoj al la apartaj po\^staj oficejoj,
kie tuj post la alveno komenci\^gas la disportado de la
po\^sta\^{\j}oj. La po\^sta\^{\j}oj, kiujn oni ne havis tempon
pretigi, kiel anka\u u la po\^sta\^{\j}oj difinitaj por la
eksterlando kaj por transita trairo, per la grandaj kale\^soj de la
stacidomo estas alportataj rekte al la Berlina \^cefa urba po\^stejo
en la Re\^ga Strato, tie \^ci ili estas specigataj kaj iras kun la
plej proksimaj kurieraj vagonaroj pluen al la orienta kaj suda
Germanujo.

   En tia maniero estas eble, ke letero, kiu en Nov-Jorko estis donita
sur la po\^ston en la tago de forveturo de Lloyda vapor\^sipo, la
plej malfrue post ok tagoj trovas sin jam en la manoj de la adresato
e\^c en la plej malproksima anguleto de Germanujo. Se oni al tio
\^ci konsideros la malkaran po\^stan pagon de dudek pfenigoj, tiam
oni efektive devas miregi pro la progresoj de la po\^staj aferoj,
precipe se oni ekmemoros, ke ankora\u u anta\u u \^cirka\u u
sesdekjaroj letero el la suda Germanujo al la orienta Prusujo iris
tiom same longe, kiel hodia\u u letero el Nov-Jorko \^gis \^cia
urbeto en Germanujo, ke tia letero tiam kostis \^cirka\u u unu
talero kaj ke ordinare nur unu fojon en semajno iris po\^sto, kiu
prenis kun si leterojn, tiel ke la respondo povis veni ne pli frue
ol post dek kvar tagoj.

   La komuniki\^goj inter la Unuigitaj \^Statoj de Nord-Ameriko kaj
E\u uropo, precipe Germanujo, kreskas kun \^ciu tago, tiel ke jam
nun fari\^gas malfacile preti\^gi kun la po\^sta komuniki\^go.
Balda\u u oni devos pensi enporti ankora\u u plibonigojn, kaj
kredeble la tutmonda ekspozicio en \^Cikago helpos al tio, ke novaj
faciligoj kaj rapidigoj estos enkondukitaj por la transsendado de la
po\^sta\^{\j}oj inter Ameriko kaj E\u uropo. [El "Das Buch für
Alle".]

\smallrule{}
