\begin{verse}
\begin{center}
\footnotesize (El \fsc{Heine.})
\end{center}

                     Ne scias mi, kial subita\\
                     Malgaj' en la koro naski\^gis;\\
                     El tempo jam enterigita\\
                     Legendo al mi revivi\^gis.

                      \vin  Jam malvarmeti\^gas l'aero,\\
                      \vin  La Rejno malla\u ute babilas,\\
                      \vin  Per oro de l' sun' en vespero\\
                      \vin  La supro de l' monto rebrilas.
\newpage
                     Plej belan knabinon mi vidas:\\
                     En ora ornamo brilante\\
                     Sur supro de l' monto \^si sidas,\\
                     La harojn mistere kombante.

                     \vin   La oran kombilon \^si movas\\
                     \vin   Kaj kantas tra l' pura aero,\\
                     \vin   Kaj forto mirinda sin trovas\\
                     \vin   En tiu \^ci kant' de l' vespero.

                     \^Sipet' iras sur la rivero,\\
                     \^Sipisto ektremis de l' kanto,\\
                     Kaj blinda por \^ciu dan\^gero\\
                     Rigardas li al la kantanto.

                      \vin  Ha, balda\u u \^sipisto la bela\\
                      \vin  Perdi\^gis sub l'akvoturnado;\\
                      \vin  \^Gin Lorelej' faris kruela,\\
                      \vin  Per sia mirinda kantado.

%L. L. ZAMENHOF.
\end{verse}

\citsc{L. Zamenhof.}

\smallrule{}
