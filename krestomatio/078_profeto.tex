\begin{verse}
\begin{center}
\footnotesize (El \fsc{Lermontov}.)
\end{center}

                     De kiam, la\u u volo de Supra Ju\^ganto,\\
                     Mi \^cionscianta fari\^gis profeto,\\
                     Malvirton mi sentas en \^ciu vivanto,\\
                     Malicon mi vidas tra \^ciu rideto.

                     Mi altajn principojn de amo kaj vero\\
                     Komencis prediki al miaj kunfratoj,\\
                     Sed ili min batis kun granda kolero\\
                     Per \^stonoj, ku\^santaj sur placoj kaj stratoj.

                     Kun cindro sur kapo, kun kora doloro,\\
                     Mi urbojn forlasis sen ia hava\^{\j}o,\\
                     Kaj nun en dezerto --- de Dia favoro\\
                     Nur ion mi havas por mia nutra\^{\j}o.

                     La\u u sankta ordono de l'Anta\u utempulo\\
                     Al mi \^ciuj bestoj volonte servadas,\\
                     Kaj steloj \^ce vortoj de mi, dezertulo,\\
                     Pli \^goje radias, pli hele lumadas.

                     Sed se iafoje tra urbo bruanta\\
                     Mi en rapideco pretere pasadas,\\
                     Min popolamaso renkontas mokanta\\
                     Kaj maljunularo memfide diradas:

                     Rigardu, infanoj, vin edifigante,\\
                     Li estis fiera, kun ni malpacadis;\\
                     Malsa\^ga, li volis kredigi nin vante,\\
                     Ke Dio al homoj per li paroladis.
\newpage
                     Jen frukto de lia obstina vanteco!\\
                     Jen pala, malgrasa, malgaja profeto,\\
                     En vesto \^cifona, simila al reto ---\\
                     \^Ce \^ciuj li estas en malestimeco.

%I. LOJKO.
\end{verse}
\citsc{I. Lojko.}

\smallrule{}