\begin{center}
\footnotesize Raporto verkita de anonima a\u utoro kaj legita (en formo iom
\^san\^gita kaj mallongigita) de s-ro L. de \fsc{Beaufront} en la kongreso
de l'Association Française pour l'Avancement des Sciences (Parizo,
1900).
\end{center}
\begin{center}
\textbf{I}
\end{center}

   \^Ciuj ideoj, kiuj estas ludontaj gravan rolon en la historio de la
homaro, havas \^ciam tiun saman egalan sorton: kiam ili ekaperas, la
samtempuloj renkontas ilin ne sole kun rimarkinde obstina
malkonfido, sed e\^c kun ia neklarigebla malamikeco; la pioniroj de
tiuj \^ci ideoj devas multe batali kaj multe suferi; oni rigardas
ilin kiel homojn frenezajn, infane malsa\^gajn, a\u u fine e\^c
rekte kiel homojn tre malutilajn. Dum la homoj, kiuj okupas sin je
\^cia plej sencela kaj senutila sensenca\^{\j}o, se \^gi nur estas
en modo kaj konforma al la rutinaj ideoj de la amaso, \^guas ne sole
\^ciujn bonojn de la vivo, sed anka\u u la honoran nomon de
"instruituloj" a\u u "utilaj publikaj agantoj", la pioniroj de
novaj ideoj renkontas nenion krom mokoj kaj atakoj; la unua
renkontita tre malmulte lerninta bubo rigardas ilin de alte kaj
diras al ili, ke ili okupas sin je malsa\^ga\^{\j}oj; la unua
renkontita gazeta felietonisto skribas pri ili "spritajn"
artikolojn kaj notojn, ne preninte sur sin la laboron almena\u u iom
ekscii, super kio ili propre laboras; kaj la publiko, kiu \^ciam
iras kiel anaro da \^safoj post la kriemuloj, ridas kaj ridegas kaj
e\^c por unu minuto ne faras al si la demandon, \^cu ekzistas e\^c
guto da senco kaj logiko en \^ciuj tiuj \^ci "spritaj" mokoj. Pri
tiuj \^ci ideoj "estas modo" paroli ne alie, ol kun ironia kaj
malestima rideto, tial tiel agas anka\u u A kaj B kaj C, kaj \^ciu
el ili timas enpensi\^gi serioze e\^c unu minuton pri la mokata
ideo, \^car li "scias anta\u ue", ke "\^gi krom malsa\^ga\^{\j}o
enhavas ja nenion", kaj li timas, ke oni iel alkalkulos lin mem al
la nombro de "tiuj malsa\^guloj", se li e\^c en la da\u uro de unu
minuto provos rilati serioze al tiu \^ci malsa\^ga\^{\j}o. La homoj
miras, "kiamaniere en nia praktika tempo povas aperi tiaj
malsa\^gaj fantaziuloj kaj kial oni ne metas ilin en la domojn por
frenezuloj".

   Sed pasas kelka tempo. Post longa vico da batalado kaj suferoj la
"buboj-fantaziuloj" atingis la celon. La homaro fari\^gis pli
ri\^ca per unu nova grava akiro kaj eltiras el \^gi la plej vastan
kaj diversforman utilon. Tiam la cirkonstancoj \^san\^gi\^gas. La
jam forti\^ginta nova afero \^sajnas al la homoj tiel simpla, tiel
"komprenebla per si mem", ke la homoj ne komprenas, kiamaniere oni
povis tutajn miljarojn vivi sen \^gi. Kiam la posteuloj legas la
rakontojn pri tio, kiel sin tenis kontra\u u la dirita ideo la
samtempuloj de \^gia naski\^go, ili absolute ne volas kredi kaj
pensas, ke \^cion tion \^ci elpensis la historioskribantoj pro
mokado je la foririntaj generacioj. "\^Cu efektive", ili diras,
"la tuta mondo tiam konsistis el idiotoj? \^Cu efektive ekzistis
homoj, kiuj elpa\^sadis kontra\u u la pioniroj kun tiaj sensencaj
kontra\u uparoloj kaj la ceteraj homoj silentadis kaj la unua
renkontita kvinjara infano ne diradis al tiuj kritikantoj:
"sinjoroj, vi ja parolas teruran, sur nenio fonditan
sensenca\^{\j}on, kies rebato sin trovas ja tuj anta\u u via nazo!"
? Absolute nekompreneble! La historiistoj certe trograndigas!"

   Legu la historion de naski\^go de la kristaneco kaj de diversaj
grandaj ideoj en la regiono de moralo, filozofio kaj scienco; legu
la historion de la eltrovo de Ameriko, de la enkonduko de fervojoj
k. t. p. k. t. p. \^Cie tute tio sama. "Es ist eine alte
Geschichte, doch bleibt sie immer neu". La lumo aperas kiel necesa
bezonata\^{\j}o al tiu, kiu staras malproksime, sed al la proksime
starantaj \^gi tran\^cas la okulojn kaj ili penas estingi \^gin. La
ideo de Kolumbo, ke "devas ekzisti okcidenta vojo Hindujon",
\^sajnas al ni nun tiel simpla, tiel natura, kaj ni simple ne volas
kredi, ke povis iam ekzisti homoj, kiuj, sciante jam, ke la tero
estas globo, povis dubi, ke al \^cia lando oni povas veni ne sole de
oriento, sed anka\u u de okcidento, kaj ke en tiu \^ci ne esplorita
okcidento povas eble trovi\^gi ne konataj al ni interesaj landoj.
Kiam ni legas tiujn kontra\u uparolojn, kiujn oni tiam faradis al
Kolumbo, ekzemple, ke neniu okcidenten de E\u uropo veturis, sekve
\^gi estas ne ebla, ke Dio malpermesis tion \^ci fari, ke la \^sipoj
mallevi\^gados malsupren kaj ne povos returne levi\^gadi supren\dots
k. t. p., --- ni kontra\u uvole demandas nin, kiamaniere homoj
matura\^gaj povis paroli tiajn sensenca\^{\j}ojn, pro kiuj en nia
tempo ru\^gi\^gus \^cia infano. Kaj tamen en tiu tempo \^guste tiuj
\^ci naivaj kontra\u uparoloj estis rigardataj kiel veroj, ne
ebligantaj ian dubon, kiel plej logika opinio de la tuta prudenta
mondo, kaj la ideoj de Kolumbo estis kalkulataj kiel infana\^{\j}o,
kiu estas inda nenian atenton. Kiam oni montris al la homoj la
forton de la vaporo kaj \^gian uzeblecon, \^sajnis, ke kia prudenta
homo povus ion kontra\u uparoli kontra\u u \^gi? Kaj tamen kiom da
multjara batalado, suferoj kaj mokoj la elpensinto devis elporti!
kaj e\^c tiam, kiam fine prosperis jam atingi la celon, kiam en
Anglujo jam dum tutaj tri jaroj la lokomotivoj kursadis kaj
alportadis grandegan utilon, sur la kontinento de E\u uropo
instruitaj homoj kaj e\^c tutaj instruitaj korporacioj, anstata\u u
simple ekrigardi kaj konvinki\^gi, skribadis ankora\u u
profundapensajn traktatojn pri tio, ke konstruado de lokomotivoj
estas infana entrepreno, ke \^gi estas ne ebla, ke \^gi estas
malutila k. t. p. Kio tio \^ci estas? ni demandas nin; \^cu tio \^ci
estis ia \^ciuhoma epidemia idioteco? \^cu efektive ekzistis tiaj
generacioj? Jes, ekzistis tiaj generacioj, kaj ni, kiuj nun miregas,
ni en efektiveco estas ne pli bonaj ol ili, kaj niaj nepoj estos ne
pli bonaj ol ni. \^Ciuj tiuj \^ci homoj kun iliaj indignige
sensencaj kontra\u uparoloj kaj atakoj estis tamen ne idiotoj,
kvankam ili nun eble \^sajnas al ni tiaj. Ilia tuta kulpo konsistis
nur en tio, ke, dank' al la natura spirita inercio de \^ciu el ni,
ili a\u u tute ne volis priju\^gi la naski\^gantajn novajn aperojn,
plivolante limigi sin per sanosubtenanta ridado, a\u u alpa\^sadis
al la priju\^gado kun anta\u ue jam preta konvinko, ke la afero
proponata al ili estas neplenumebla, kaj \^ciujn siajn argumentojn
ili penadis konformigadi al tiu auta\u ue farita decido, ne
rimarkante la tutan senfundamentecon de tiuj \^ci argumentoj, kaj
kontra\u u la argumentoj de la defendantoj de la nova ideo ili
fermadis sian cerbon per la plej fortikaj seruroj, kaj tial tiuj
\^ci lastaj argumentoj, kiuj penadis pruvi la eblecon de tio, devis
\^sajni al tiuj inerciaj homoj tiel same infanaj, kiel al ni nun
\^sajnas iliaj tiamaj kontra\u uparoloj.

   Al tiaj ideoj, kiuj al la samtempuloj \^sajnas senenhava fantazio
kaj al la posteuloj \^sajnas tia natura afero, ke ili ne komprenas,
kiamaniere la homoj miljarojn vivis sen \^gi, --- al tiaj ideoj
apartenas anka\u u la ideo de enkonduko de komuna lingvo por la
komuniki\^goj inter diversaj popoloj. Kiam niaj posteuloj legos en
la historio, ke la homoj, tiuj \^ci re\^goj de la tero, tiuj \^ci
plej altaj reprezentantoj de la monda inteligenteco, tiuj \^ci
duon-dioj, en la da\u uro de tutaj miljaroj vivis unuj apud la
aliaj, ne komprenante unuj la aliajn, ili simple ne volos kredi.
"Por tio \^ci oni ja bezonis nenian supernaturan forton, ili diros;
\^ciu el tiuj \^ci homoj posedis ja kolekton da kondi\^caj sonoj,
per kiuj li tute precize kompreni\^gadis kun siaj plej proksimaj
najbaroj, --- kiel do ne venis al ili en la kapon konsenti\^gi inter
si, ke unu el tiaj kolektoj da kondi\^caj sonoj estu enkondukita por
la reciproka kompreni\^gado inter {\sl \^ciuj}, simile al tio, kiel
por la plimulto de la kulturaj popoloj estis enkondukita jam longe
unu kondi\^ca kolekto da mezuroj, unu kondi\^ca alfabeto, unuj
kondi\^caj muzikaj signoj k. t. p.!" Niaj posteuloj indignos, kiam
ili ekscios, ke la homojn, kiuj penadis pri la enkonduko de komuna
lingvo, la samtempuloj montradis per la fingroj, kiel maniulojn,
bubojn, ne meritantajn la nomon de seriozaj homoj; ke pri tiuj \^ci
homoj \^ciu malplenkapulo povis spritadi en la gazetoj, kiom li
volis, kaj trovi\^gis neniu, kiu dirus al tiuj malplenkapuloj: "vi
povas trovi tiujn \^ci ideojn plenumeblaj a\u u ne plenumeblaj,
--- sed moki ilin, e\^c ne konati\^ginte kun ili, estas honte,
sinjoroj!" Kore ridegos niaj posteuloj, kiam ili a\u udos tiujn
naivajn kontra\u uparolojn, kiujn multaj el niaj samtempuloj faradis
kontra\u u la ideo de lingvo internacia entute kaj de lingvo arta
speciale. Simile al tio, kiel ni kun rideto de kompato rilatas al
tiu el niaj pra-praavoj, kiu anta\u u kelke da miljaroj eble
protestis kontra\u u la enkonduko de arta alfabeto, kriante kun la
aplombo de instruitulo, sed tute senpruve, ke rimedo por la
esprimado de niaj pensoj estas objekto organa, natura, kreita de la
historio (skribado per hieroglifaj desegna\^{\j}oj) kaj ne povanta
"esti kreita en kabineto", --- tiel niaj posteuloj, mokados tiujn
niajn samtempulojn, kiuj nur pro tiu nenion diranta cirkonstanco, ke
la nunaj lingvoj krei\^gis blinde per si mem, a\u utoritate
certigas, ke lingvo ne povas esti kreita arte. "\^Gis nun ne estis,
sekve ne povas esti!" --- "Kiel mi povas kredi", diros en la
venonta centjaro ia dekjara lernanto al sia instruanto, "ke
ekzistis homoj, kiuj neadis la eblecon de ekzistado de arta lingvo,
kiam anta\u u ilia nazo tia lingvo jam {\sl ekzistis}, havis jam
ri\^can literaturon kaj bonege {\sl plenumadis jam en la praktiko}
\^ciujn funkciojn, kiujn oni povas postuli de lingvo internacia, kaj
tiuj \^ci sinjoroj, anstata\u u babiladi \^ciam teorian
sensenca\^{\j}on, bezonis nur malfermi la okulojn kaj {\sl
ekrigardi}! \^Cu estas eble, ke homoj matura\^gaj parolus \^ciam
frazistan sensenca\^{\j}on pri ia diferenco de la vo\^caj organoj
\^ce la popoloj, kiam \^ciu infano vidis sur \^ciu pa\^so membrojn
de unu popolo, bonege parolantajn en la lingvo de alia popolo!" Kaj
la instruanto respondos: "\^gi estas efektive nekredebla, kaj tamen
\^gi tiel estis!"

   Cetere en la nuna tempo en la afero de lingvo internacia la rutino
kaj spirita inercio komencas iom post iom cedadi al la sana
prudento. Jam longe tie a\u u aliloke en diversaj gazetoj kaj revuoj
aperas artikoloj plenaj de aprobo por la ideo mem kaj por \^giaj
batalantoj. Sed tiuj \^ci artikoloj estas ankora\u u senkura\^gaj,
kvaza\u u la a\u utoroj timas, ke oni ne elmetu ilin al publika
malhonoro. Tiuj \^ci senkura\^gaj vo\^coj perdi\^gas en la la\u
utega \^horo de la kriistoj kaj mokistoj, tiel ke la grandega
plimulto de la publiko, kutiminta iradi nur tien, kie oni krias la
plej la\u ute, kaj opiniadi \^ciun mokanton sa\^gulo, \^ciun
atakanton bravulo kaj \^ciun atakaton kulpulo, \^ciam ankora\u u
rigardas la ideon de lingvo internacia kiel sensencan infanan
fantazion. Tiun \^ci publikon konvinki ni ne entreprenas, \^car
\^ciuj niaj vortoj pereus vane. \^Gin konvinkos nur la tempo.
Morga\u u \^gi al la pioniroj de la ideo konstruos monumentojn kun
tia sama anara sento, kun kiu \^gi hodia\u u super\^{\j}etas ilin
per koto. Nia parolo estas difinita nur por tiuj, kiuj provis rilati
al nia ideo kun ju\^go memstara, sed sub la influo de diversaj a\u
uditaj opinioj perdis la egalpezon, ne scias, kiel ili devas sin
teni, dezirus kredi kaj samtempe turmenti\^gas per konstantaj duboj.
Por ili ni tie \^ci analizos la demandon, \^cu efektive ni, la
amikoj de la ideo de lingvo internacia, laboras por ia utopio, kaj
\^cu minacas al ni la dan\^gero, ke \^ciuj niaj laboroj pereos vane,
kiel kredigas niaj kontra\u uuloj, a\u u \^cu ni iras al celo klare
difinita, senduba kaj nepre atingota.

   Ni scias, estimataj a\u uskultantoj, ke vi kutimis rilati kun estimo
nur al tiaj argumentoj, kiuj estas plenigitaj per multo da citatoj,
traplektitaj per multo da la\u utaj a\u utoritataj nomoj kaj brilas
per amasego da alteflugaj kvaza\u u-sciencaj frazoj. Ni avertas vin,
ke \^cion tion \^ci vi en nia parolo ne trovos. Se vi trovas
atentinda nur tion, kio estas ligita kun la\u utaj nomoj, legu ian
verkon pri lingvo internacia, kaj vi trovos tie longan serion da
gloraj kaj a\u utoritataj scienculoj, kiuj laboris por la ideo de
lingvo internacia. Sed ni tie \^ci forlasos \^cian superfluan
balaston kaj parolos al vi nur en la nomo de la nuda {\sl logiko}.
Ne turnu atenton sur tion, kion diras Petro a\u u Johano, sed
pripensu {\sl mem}. Se niaj argumentoj estas \^gustaj, akceptu ilin,
--- se ili estas mal\^gustaj, for\^{\j}etu ilin, se e\^c miloj da
la\u utaj nomoj starus post ili.

   Ni analizos sisteme la sekvantajn demandojn: 1) \^cu lingvo
internacia estas bezona; 2) \^cu \^gi estas ebla en principo; 3)
\^cu ekzistas espero, ke \^gi efektive estos enkondukita praktike;
4) kiam kaj kiamaniere tio \^ci estos farita kaj kia lingvo estos
enkondukita; 5) \^cu nia nuna laborado kondukas al ia difinita celo,
a\u u ni agas ankora\u u blinde kaj riskas, ke nia laborado pereos
vane, kaj prudentaj homoj devas ankora\u u sin teni flanke de ni,
\^gis "la afero klari\^gos".

\begin{center}
\textbf{II}
\end{center}

   \^Cu lingvo internacia estas bezona? Tiu \^ci demando per sia
naiveco elvokos ridon \^ce la estontaj generacioj, tiel same kiel
niaj samtempuloj ekridus ekzemple \^ce la demando "\^cu po\^sto
estas bezona"? La plimulto de la inteligenta mondo jam nun trovos
tiun \^ci demandon tute superflua; tamen pro konsekvenco ni metas
tiun \^ci demandon dank'al tio, ke ekzistas ankora\u u multe da
homoj, kiuj respondas je tiu \^ci demando per "ne". La sola
motivo, kiun kelkaj el tiuj \^ci homoj elmetas, estas sekvanta:
"lingvo internacia detruos la lingvojn naciajn kaj la naciojn". Ni
konfesas, ke kiom ajn ni rompis al ni la kapon, ni neniel povis
kompreni, en kio nome konsistus la malfeli\^co por la homaro, se en
unu bela tago montri\^gus, ke ne ekzistas jam plu nacioj kaj lingvoj
naciaj, sed ekzistas nur unu \^ciuhoma familio kun unu \^ciuhoma
lingvo. Sed ni supozu, ke tio \^ci efektive estus io terura, kaj ni
rapidos trankviligi tiujn \^ci sinjorojn. Lingvo intornacia deziras
nur doni al la homoj de {\sl malsamaj} popoloj, kiuj staras unu
anta\u u alia kiel mutuloj, la eblon komprenadi unu alian, sed \^gi
neniel intencas enmiksi\^gi en la internan vivon de la popoloj.
Timi, ke lingvo internacia detruos la lingvojn naciajn, estas tiel
same ridinde, kiel ekzemple timi, ke la po\^sto, kiu donas al homoj
malproksimaj unu de alia la eblon komuniki\^gadi, minacas neniigi la
bu\^sajn interparoladojn inter la homoj! "Lingvo internacia" kaj
"lingvo tutmonda" estas du tute malsamaj objektoj, kiujn miksi
inter si oni neniel devas. Se ni supozus, ke fari\^gus iam
kunflui\^go de la homoj en unu \^ciuhoman popolon, en tiu \^ci
"malfeli\^co" (kiel nomos \^gin la naciaj \^sovinistoj) estos
kulpa ne la lingvo internacia, sed la alii\^gintaj konvinkoj kaj
opinioj de la homoj. {\sl Tiam} efektive la lingvo internacia
faciligos al la homoj la atingon de tio, kio anta\u ue estos
principe decidita de ili kiel dezirinda; sed se la celado al
kunflui\^go ne naski\^gos \^ce la homoj {\sl memstare}, la lingvo
internacia per si mem certe ne volos {\sl altrudi} al la homoj tian
unui\^gon. Lasante tute flanke la demandon pri la dezirindeco a\u u
nedezirindeco de nacia \^sovinismo, ni notos nur tion, ke celadon al
lingvo internacia ne devas escepti e\^c la plej varmega blinda
\^sovinismo; \^car la rilato inter celado al lingvo internacia kaj
inter nacia \^sovinismo estas tia sama, kiel inter nacia patriotismo
kaj amo al sia familio: \^cu iu povas diri, ke la pligrandigo de
reciprokaj komuniki\^goj kaj interkonsentoj inter homoj de tiu sama
lando (celado {\sl patriota}) minacas per io al la amo {\sl
familia}? Per si mem la lingvo internacia ne sole ne povas
malfortigi la lingvojn naciajn, sed kontra\u ue, \^gi sendube devas
konduki al ilia granda forti\^gado kaj plena ekflorado: dank' al la
neceseco ellernadi diversajn fremdajn lingvojn, oni nun malofte
povas renkonti homon, kiu posedus perfekte sian patran lingvon, kaj
la lingvoj mem, konstante kunpu\^si\^gante unuj kun la aliaj, \^ciam
pli kaj pli konfuzi\^gas, kripli\^gas kaj perdas sian naturan
ri\^cecon kaj \^carmon; sed kiam \^ciu el ni devos ellernadi nur
{\sl unu} fremdan lingvon (kaj ankora\u u tre facilan), \^ciu el ni
havos la eblon ellerni sian lingvon fonde, kaj \^ciu lingvo,
liberi\^ginte de la premado de multaj najbaroj kaj konservinte plene
por si sola \^ciujn fortojn de sia popolo, disvolvi\^gos balda\u u
plej potence kaj brile.

   La dua motivo, kiun elmetas la malamikoj de lingvo internacia, estas
la timo, ke kiel lingvo internacia estos eble elektita ia el la
lingvoj {\sl naciaj} kaj ke tiam la homoj ne {\sl alproksimi\^gos}
al si reciproke, sed simple ia unu popolo {\sl dispremos} kaj {\sl
englutos} \^ciujn aliajn popolojn, dank' al la grandega superforto,
kiujn \^gi ricevos super \^ciuj aliaj popoloj. Tiu \^ci motivo estas
ne tute senfundamenta; sed \^gi povas esti elmetita nur kontra\u u
tia a\u u alia nepripensita kaj mal\^gusta {\sl formo} de lingvo
internacia. Tiu \^ci motivo kompreneble perdas \^cian signifon, se
ni turnos atenton, ke lingvo internacia povas esti kaj estos nur ia
{\sl ne\u utrala} lingvo, kiel ni malsupre montros.

   Sekve se ni lasos por kelka tempo flanke la demandon pri la ebleco
a\u u neebleco de la enkonduko de lingvo internacia (pri tiu \^ci
punkto ni parolos malsupre), se ni supozos, ke la enkonduko de tia
lingvo dependas nur de nia {\sl deziro}, kaj se ni esceptos ian
kriantan erarpa\^son en la {\sl elekto} de la lingvo, \^ciu devas
konsenti, ke pri {\sl malutilo} de lingvo internacia neniel povas
esti e\^c la plej malgranda parolo. Sed la {\sl utilo}, kiun tia
lingvo alportus al la mondo, estas tiel grandega kaj videbla por
\^ciu, ke pri tio \^ci ni propre ne bezonus paroli. Tamen {\sl
kelke} da vortoj ni diros pri tio \^ci, se e\^c simple pro pleneco
de nia analizo.

   \^Cu vi ekpensis iam pri tio, {\sl kio} propre levis la homaron tiel
neatingeble alte super \^ciuj aliaj bestoj, kiuj ja en efektiveco
estas konstruitaj la\u u tiu sama tipo, kiel la homo? La tutan nian
altan kulturon kaj civilizacion ni dankas nur al unu objekto: al la
{\sl posedado de lingvo}, kiu ebligis al ni la {\sl
inter\^san\^gadon de pensoj}. Kio estus kun ni, fieraj re\^goj de la
mondo, se ni ne povus {\sl lingve komuniki\^gadi} unuj kun aliaj, se
sian tutan scion kaj inteligentecon \^ciu el ni devus de la komenco
mem ellaboradi al si mem, anstata\u u faradi uzon --- dank' al
inter\^san\^go de pensoj --- de la jam pretaj fruktoj de la sperto
kaj diversaj scioj de tutaj miljaroj, de tutaj milionoj kaj
miliardoj da aliaj similaj al ni kreita\^{\j}oj? Ni tiam e\^c per
unu plej malgranda \^stupeto ne starus pli alte ol tiuj diversaj
bestoj, kiuj nin \^cirka\u uas kaj kiuj estas tiel sensa\^gaj kaj
senhelpaj! Forprenu de ni la manojn kaj la piedojn kaj kion vi
volas, sed lasu al ni nur la povadon inter\^san\^gadi la pensojn,
--- kaj ni restos tiuj samaj re\^goj de la naturo kaj ni konstante
kaj senfine perfekti\^gados; sed donu al \^ciu el ni e\^c centon da
manoj, donu al ni e\^c centon da diversaj \^gis nun ne konataj
sentoj kaj povoj, sed forprenu de ni la povon de inter\^san\^gado de
pensoj --- kaj ni restos sensa\^gaj kaj senhelpaj bestoj. Sed se la
tre neplena kaj tre limigita ebleco de inter\^san\^go de pensoj
havis por la homaro tian grandegan signifon, ekpensu pri tio, kian
grandegan kaj kun nenio kompareblan utilon donus al la homaro tiu
lingvo, kiu farus la inter\^san\^gadon de pensoj {\sl plena}, kaj
dank' al kiu ne sole A havus la eblon kompreni\^gadi kun B, C kun D,
E kun F, sed {\sl \^ciu} el ili povus kompreni\^gadi kun {\sl \^ciu}
el la aliaj! Tuta cento da plej grandaj elpensoj ne faros en la vivo
de la homaro tian grandan kaj bonfaran revolucion, kian faros la
enkonduko de lingvo internacia! Ni prenu kelkajn malgrandajn
ekzemplojn. Ni penas tradukadi la verkojn de \^ciu nacio en la
lingvojn de \^ciuj aliaj nacioj; sed tio \^ci englutas ja
neproduktive grandegan multon da laboroj kaj mono kaj tamen malgra\u
u \^cio ni povas traduki nur la plej sensignifan parton de la homa
literaturo, kaj la grandega plimulto de la homa literaturo kun
ri\^caj provizoj da diversaj pensoj por \^ciu el ni restas
neakirebla. Sed kiam ekzistus lingvo internacia, tiam \^cio, kio
aperas en la regiono de la homa penso, estus tradukata nur en tiun
\^ci {\sl unu} ne\u utralan lingvon kaj multaj verkoj estus
skribataj rekte en tiu \^ci lingvo, kaj \^ciuj produktoj de la homa
spirito fari\^gus akireblaj por \^ciu el ni. Por la perfektigado de
tiu a\u u alia bran\^co de la homaj scioj ni aran\^gas sur \^ciu
pa\^so internaciajn kongresojn, --- sed kian mizeran rolon ili
ludas, kiam partopreni en ili povas ne tiu, kiu efektive kun utilo
dezirus ion a\u udi, ne tiu, kiu efektive ion gravan volus komuniki,
sed nur tiu, kiu scias babiladi en kelkaj lingvoj. Nia vivo estas
mallonga kaj la scienco estas vasta; ni devas lerni, lerni, lerni!
Al la lernado ni povas dedi\^ci nur parton de nia mallonga vivo,
nome niajn infanajn kaj junulajn jarojn; sed ho ve! granda parto de
tiu \^ci kara tempo foriras tute neproduktive por la ellernado de
lingvoj! Kiom multe ni gajnus, se, dank' al ekzistado de lingvo
internacia, ni povus la tutan tempon, dedi\^catan nun al la
neproduktiva lernado de lingvoj, dedi\^ci al la lernado de efektivaj
kaj pozitivaj {\sl sciencoj}! Kiel alte tiam levi\^gus la
homaro!\dots

   Sed ni ne parolos plu pri tiu \^ci punkto, \^car kiel ajn \^ciu el niaj
a\u uskultantoj rilatus al tiu a\u u alia {\sl formo} de lingvo
internacia, ni dubas, \^cu trovi\^gos inter ili e\^c unu, kiu dubus
la {\sl utilecon mem} de tia lingvo. Sed \^car al multaj homoj, kiuj
ne kutimis donadi al si precizan kalkulon pri siaj simpatioj kaj
antipatioj, ordinare \^sajnas, ke se ili ne aprobas tiun a\u u alian
formon de ia ideo, ili nepre devas ataki la ideon mem entute,
--- tial ni, pro sistemeco de nia analizo, petas \^ciun el la
estimataj a\u uskultantoj anta\u u \^cio noti al si bone en la
memoro, ke pri la utileco de lingvo internacia {\sl entute} --- se
tia estus enkondukita --- li ne dubas. Ekmemoru do, sinjoroj, bone
la unuan konkludon, al kiu ni venis, notu al vi kaj ekmemoru, ke kun
tiu \^ci konkludo vi konsentas, nome:

   \emph{La ekzistado de lingvo internacia, per kiu povus kompreni\^gadi
   inter si la homoj de \^ciuj landoj kaj popoloj, alportus al la
   homaro grandegan utilon.}

\begin{center}
\textbf{III}
\end{center}

   Nun ni transiros al la dua demando: "\^cu lingvo internacia estas
ebla?" Anka\u u pri tio \^ci nenia senanta\u uju\^ga homo e\^c unu
minuton povas dubi, \^car ne sole ne ekzistas e\^c la plej
malgrandaj faktoj, kiuj parolus {\sl kontra\u u} tia ebleco, sed ne
ekzistas e\^c la plej malgrandaj ka\u uzoj, kiuj devigus e\^c unu
minuton {\sl dubi} pri tia obleco. Ekzistas en efektiveco homoj,
kiuj kun scienca aplombo kredigas, kvaza\u u lingvo estas objekto
natura, organa, kiu dependas de apartaj fiziologiaj ecoj de la
organoj de parolo de \^ciu popolo, de la klimato, heredeco,
kruci\^gado de rasoj, historiaj kondi\^coj k. t. p. Kaj al la amaso
tiaj instruitaj paroloj tre imponas, precipe se ili en sufi\^ca
mezuro estas traplektitaj per diversaj citatoj kaj per misteraj por
la amaso terminoj te\^hnikaj. Sed homo klera, kiu kura\^gas havi
propran ju\^gon, scias ja tre bone, ke \^cio tio \^ci estas nur
senenhava pse\u udo-scienca babilado, kiu havas nenian sencon kaj
kiun rebati povus tre facile la unua renkontita infano. El la
\^ciutaga sperto ni \^ciuj ja scias tre bone, ke se ni prenos
infanon el kiu ajn lando a\u u nacio kaj de la tago de \^gia
naski\^go edukados \^gin inter personoj de nacio tute fremda kaj
e\^c antipoda por \^gi, \^gi parolados en la lingvo de tiu \^ci
nacio tiel same bonege kaj pure, kiel \^ciu natura filo de tiu \^ci
nacio. Se por homo {\sl matura\^ga} estas ordinare malfacile ellerni
fremdan lingvon, tio \^ci ja tute ne venas de la konstruo de liaj
organoj de parolo, sed simple de tio, ke li ne havas paciencon, ne
havas tempon, ne havas instruantojn, ne havas rimedojn k. t. p. Tiu
\^ci sama matura\^gulo renkontus ja tiujn \^ci samajn
malfacila\^{\j}ojn \^ce la ellernado de sia {\sl hejma} lingvo, se
li en la infaneco ne estus edukita en tiu \^ci lingvo, sed devus
ellernadi \^gin per helpo de lecionoj. Fine \^ciu klera homo ja
anka\u u nun {\sl devas} ellernadi kelkajn fremdajn lingvojn, kaj li
certe ne elektas tiujn lingvojn, kiuj kvaza\u u estas konformaj al
liaj organoj de parolado, sed nur tiujn, kiujn li {\sl bezonas};
estas sekve nenio neebla en tio, ke anstata\u u ke \^ciu lernas {\sl
diversajn} lingvojn, \^ciuj ellernadu {\sl unu saman} lingvon kaj
sekve povu komprenadi unu alian. Se e\^c \^ciu posedus la
komuneakceptitan lingvon ne en plena perfekteco, e\^c tiam la
demando de lingvo internacia estus jam decidita kaj la homoj \^cesus
staradi unu anta\u u alia kiel surda-mutaj. Kaj oni devas ja memori,
ke se \^cie estus sciate, ke por komuniki\^goj kun la tuta mondo oni
devas ellerni nur {\sl unu} lingvon --- \^cie ekzistus multego da
bonaj instruistoj de tiu \^ci lingvo, da specialaj lernejoj, \^ciu
ellernadus tiun \^ci lingvon kun la plej granda volonteco kaj
fervoreco, kaj fine \^ciuj gepatroj alkutimigadus siajn infanojn al
tiu \^ci lingvo en {\sl la infaneco}, paralele kun la patra lingvo.
Sekve, lasante dume flanke la demandon pri tio, \^cu la homoj {\sl
volos} elekti ian unu lingvon por la rolo de internacia kaj \^cu
prosperos al ili veni al interkonsento pri tiu \^ci elekto, ni dume
konstatas tiun fakton, kiu kun plena sendubeco sekvas el \^cio, kion
ni diris supre, nome, ke la {\sl ekzistado mem} de lingvo internacia
estas tute ebla. Notu do al vi bone en la memoro tiujn du sendubajn
konkludojn, al kiuj ni venis \^gis nun, nome:

1. \emph{Lingvo internacia alportus al la homaro grandegan utilon;}

2. \emph{La ekzistado de lingvo internacia estas plene ebla.}

\begin{center}
\textbf{IV}
\end{center}

   \^Cu lingvo internacia {\sl estos} iam enkondukita? Se ni venis al la
konkludo, ke lingvo internacia alportus al la homaro grandegan
utilon kaj ke \^gia ekzistado estas ebla, el tiuj \^ci du konkludoj
jam per si mem elfluas la konkludo, ke tia lingvo pli a\u u malpli
frue nepre estos enkondukita, \^car alie ni devus nei \^ce la homaro
la ekzistadon de \^cia e\^c plej elementa inteligenteco. Se lingvo,
povanta plenumadi la rolon de internacia, \^gis nun ankora\u u ne
ekzistus, sed devus ankora\u u esti kreita, tiam respondo je la
demando metita en la komenco de tiu \^ci \^capitro estus duba, \^car
estus nesciate ankora\u u, \^cu oni povos krei tian lingvon. Sed ni
ja scias, ke da lingvoj ekzistas tre multe kaj ke {\sl \^ciu} el ili
en okazo de bezono povus esti difinita kiel internacia, nur kun tia
diferenco, ke unu el ili {\sl pli} ta\u ugus por tiu \^ci celo kaj
alia {\sl malpli}. Ni havas sekve \^cion pretan kaj ni bezonas nur
{\sl ekdeziri} kaj {\sl elekti}, --- kaj en tia okazo la respondo je
la supre metita demando jam ne povas esti duba. La homoj vivas per
vivo konscia kaj sen\^cese celadas al sia bono; tial se ni scias, ke
tiu a\u u alia afero promesas al la homoj grandegan kaj senduban
utilon kaj ke \^gi estas por ili atingebla, ni \^ciam kun plena
certeco povas anta\u udiri, ke de tiu momento, kiam la homoj nur
ekturnis sian atenton al tiu \^ci afero, ili jam obstine celados al
\^gi \^ciam pli kaj pli kaj ne \^cesos en sia celado tiel longe,
\^gis ili la aferon atingos. Se du homaj grupoj estas disigitaj unu
de alia per rivereto, sed scias, ke por ili estus tre utile
komuniki\^gadi inter si, kaj ili vidas, ke tabuloj por la kunigo de
amba\u u bordoj ku\^sas tute pretaj apud iliaj manoj, tiam oni ne
bezonas esti profeto, por anta\u uvidi kun plena certeco, ke pli a\u
u malpli frue tabulo estos trans\^{\j}etita trans la rivereto kaj
komuniki\^gado estos aran\^gita. Estas vero, ke pasas ordinare kelka
tempo en \^sanceli\^gado, kaj tiu \^ci \^sanceli\^gado estas
ordinare ka\u uzata de la plej sensencaj pretekstoj: sa\^gaj homoj
diras, ke celado al aran\^go de komuniki\^go estas infana\^{\j}o,
\^car neniu el ili okupas sin je metado de tabuloj trans rivereto
kaj tiu \^ci afero estas tute ne en modo; spertaj homoj diras, ke la
anta\u uuloj ne metadis tubulojn trans riveveto, sekve \^gi estas
utopio; instruitaj homoj pruvas, ke komuniki\^gado povas esti nur
afero natura kaj ke la homa organismo ne povas sin movadi sur
tabuloj, k. t. p. Tamen pli a\u u malpli frue tabulo estas
transmetata kaj la komuniki\^gado estas aran\^gata. Tiel estis kun
\^ciu utila ideo, tiel estis kun \^ciu utila elpenso; apena\u u la
senanta\u uju\^gaj homoj venadis al senduba konkludo, ke la donita
afero estas tre utila kaj samtempe efektivigebla, ili povis \^ciam
scii anta\u ue kun plena certeco, ke pli a\u u malpli frue la afero
{\sl nepre} estos akceptita, malgra\u u \^cia batalado de la flanko
de la rutinuloj; \^car tion \^ci garantias ne sole la natura
inteligenteco de la homaro, sed anka\u u \^gia celado al sia
praktika bono kaj profito. Tiel estos anka\u u kun la lingvo
internacia. En la da\u uro de multaj centjaroj la homoj, ankora\u u
ne tre bezonante lingvon internacian, ne enpensi\^gadis pri tiu \^ci
demando; sed nun, kiam la forti\^gintaj komuniki\^goj inter la homoj
turnis ilian atenton al tiu \^ci demando, nun, kiam la homoj
komencis konvinki\^gadi, ke lingvo internacia alportos al ili
grandegan utilon kaj ke \^gi estas atingebla, ili sen ia dubo jam
celados al \^gi \^ciam pli kaj pli, \^gia neceseco fari\^gados por
ili kun \^ciu tago \^ciam pli sentebla, kaj ili jam ne
trankvili\^gos tiel longe, \^gis la demando estos solvita. \^Cu vi
povas tion \^ci dubi? Certe ne! {\sl Kiam} tio \^ci venos --- ni ne
intencas nun anta\u udiri: povas esti, ke \^gi venos post unu jaro,
post dek jaroj, post cent jaroj a\u u e\^c post kelkaj centoj da
jaroj, --- sed unu afero estas jam senduba, ke kiom ajn devos suferi
la unuaj pioniroj de tiu \^ci ideo, kaj se e\^c tiu \^ci ideo
multajn fojojn endormi\^gadus je tutaj dekjaroj, \^gi jam neniam
mortos: \^ciam pli ofte kaj pli obstine eksonados vo\^coj,
postulantaj enkondukon de lingvo internacia, kaj fine, pli a\u u
malpli frue --- se la demando ne estos solvita de la societo mem
--- la registaroj de \^ciuj landoj {\sl devos} cedi, aran\^gi internacian
kongreson kaj elekti ian unu lingvon kiel internacian. Tie \^ci
povas esti nur demando pri la {\sl tempo}: unuj el vi diros, ke tio
\^ci venos tre balda\u u, aliaj diros, ke \^gi venos nur en plej
malproksima estonteco; sed ke tiu \^ci fakto entute iam {\sl venos}
kaj ke la homaro, vidante la grandegan utilecon kaj samtempe la
atingeblecon de lingvo internacia, ne restos eterne indiferenta por
tiu \^ci afero kaj senhelpa anaro da ekzista\^{\j}oj, ne
komprenantaj unu alian --- pri tio \^ci certo neniu el vi dubas e\^c
minuton. Ni petas vin tial noti al vi en la memoro ia trian
konkludon, al kiu ni venis, nome:

\emph{"Pli a\u u malpli frue lingvo internacia nepre estos enkondukita."}

   Tie \^ci ni faros malgrandan pa\u uzon kaj diros kelke da vortoj pri
ni, batalantoj pro la ideo de lingvo internacia. El \^cio pruvita de
ni vi vidas, ke ni tute ne estas tiaj fantaziistoj kaj utopiistoj,
kiajn multaj el vi eble vidis en ni kaj kiajn nin pentras multaj
gazetoj, ne dezirantaj eni\^gadi en la esencon de tio, pro kio ni
batalas. Vi vidas, ke ni batalas pro afero, kiu alportos al la
homaro grandegan utilon kaj kiu pli a\u u malpli frue {\sl nepre}
estos atingita. \^Ciu prudenta homo povas sekve kura\^ge ali\^gi al
ni, ne timante la mokojn de la malsa\^ga kaj nepensanta amaso. Ni
batalas pro afero tute pripensita kaj certa, kaj tial neniaj mokoj
nek atakoj nin debatos de la vojo. La estonteco apartenas al {\sl
ni}. Ni supozu e\^c, ke tiu {\sl formo} de lingvo internacia, pro
kiu ni batalas, montri\^gos en la estonteco kiel erara kaj ke
venonta lingvo internacia estos ne tiu, kiun ni elektis, --- sed tio
\^ci ja tute nin ne devas konfuzi, \^car ni batalas ne pro la {\sl
formo}, sed pro la {\sl ideo}, kaj formon konkretan al nia batalado
ni donis nur tial, ke \^cia batalado abstrakta kaj teoria ordinare
al nenio kondukas. Malsupre ni pruvos, ke e\^c anka\u u tiu konkreta
{\sl formo} de la lingvo estas tute pripensita kaj havas senduban
estontecon; sed se vi e\^c tion \^ci dubus, la formo ja neniom nin
ligas: se tiu \^ci formo montri\^gos erara, ni morga\u u \^gin
\^san\^gos, kaj en okazo de bezono ni \^gin post-morga\u u ankora\u
u unu fojon \^san\^gos, sed ni batalados pro nia ideo tiel longe,
\^gis \^gi pli a\u u malpli frue estos plene efektivigita. Se ni,
obeante la vo\^con de la indiferenta egoismo, detenadus nin de nia
laborado nur tial, ke kun la tempo la formo de la lingvo internacia
eble estos alia, ol tiu, pro kiu ni nun laboras, tio \^ci signifus
tion saman, kiel ekzemple rifuzi la uzadon de vaporo tial, ke poste
eble estos trovita pli bona rimedo de komuniki\^gado, a\u u rifuzi
regnajn plibonigojn tial, ke poste iam eble estos trovitaj pli bonaj
formoj por la regna konstruo. Nun ni estas ankora\u u malfortaj kaj
\^cia bubo povas ankora\u u mokadi nin kaj montradi nin per la
fingroj: sed plej bone ridas tiu, kiu ridas la lasta. Nia afero iras
malrapide kaj malfacile; tre povas esti, ke la plimulto de ni ne
\^gisvivos tiun momenton, kiam montri\^gos la fruktoj de nia agado
kaj \^gis la morto mem ni estos objekto de mokoj; sed ni iros en la
tombon kun la konscio, ke nia afero ne mortos, ke \^gi morti {\sl
neniam povas}, ke pli a\u u malpli frue \^gi {\sl devas} atingi la
celon. Kaj se e\^c, lacaj de la sendanka laborado, ni kun malespero
kaj apatio lasus fali la manojn, --- tute egale, la afero ne mortos:
anstata\u u la lacigitaj batalantoj aperos batalantoj novaj; \^car
ni denove ripetas, ke se estas ekster dubo, ke lingvo internacia
alportus al la homaro grandegan utilon kaj ke \^gi estas atingebla,
en tia okazo por nenia homo ne blindigita de rutino ne povas esti ia
dubo, ke \^gi pli a\u u malpli frue nepre {\sl estos} atingita, kaj
nia konstanta laborado estos por la homaro eterna memorigado tiel
longe, \^gis la ideo de lingvo internacia estos efektivigita. La
posteuloj benos nian memoron, kaj al tiuj sa\^gaj homoj, kiuj nun
nomas nin fantaziistoj, ili rilatos tiel, kiel ni nun rilatas al la
sa\^gaj samtempuloj de la eltrovo de Ameriko, de la elpenso de
vaporveturiloj k. t. p.

\begin{center}
\textbf{V}
\end{center}

   Sed ni revenu al nia interrompita analizado. Ni pruvis, ke lingvo
internacia pli a\u u malpli frue nepre estos enkondukita; sed restas
la demando: {\sl kiam} kaj {\sl kiamaniere} \^gi venos? Povas esti,
ke tio \^ci venos nur post centoj a\u u eble e\^c post miloj da
jaroj? \^Cu por tio \^ci estas bezona nepre reciproka interkonsento
de la registaroj do \^ciuj landoj? Por doni pli malpli kontentigajn
respondojn je tiuj \^ci demandoj, ni devas anta\u ue analizi alian
demandon, nome: "\^cu oni povas anta\u uvidi, {\sl kia} lingvo
estos internacia?" Inter la unuaj demandoj kaj la lasta ekzistas la
sekvanta malvasta ligitceo: se oni ne povas anta\u uvidi, {\sl kia}
lingvo estos farita internacia, kaj se diversaj lingvoj havas por
tio \^ci pli-malpli egalajn \^sancojn, tiam oni devas atendi \^gis
la registaroj de \^ciuj (almena\u u la plej gravaj) regnoj decidos
aran\^gi por tiu \^ci celo kongreson kaj decidi la demandon pri
lingvo internacia. Kiu scias, kun kia granda malfacileco la
registaroj decidi\^gas por \^cia nova afero, tiu komprenos, ke pasos
ankora\u u tre kaj tre multe da jaroj, anta\u u ol la registaroj
trovos la demandon de lingvo internacia sufi\^ce maturi\^ginta kaj
inda je ilia enmiksi\^go, kaj poste pasos kredeble ankora\u u vico
da jaroj por la laborado de diversaj komitatoj kaj diplomatiistoj,
anta\u u ol la afero estos decidita. Apartaj personoj kaj societoj
tie \^ci nenion povus fari; ili povus nur konstante instigadi la
registarojn, sed mem solvi la demandon sen la enmiksi\^go de la
registaroj ili ne povus. \^Gis la solvo de la demando en tia okazo
estus sekve ankora\u u tre kaj tre malproksime. Sed tute alia afero
estus, se montri\^gus, ke oni povas anta\u ue {\sl anta\u uvidi} kun
plena precizeco kaj plena certeco, kia nome lingvo estos iam
internacia: tiam oni jam ne bezonus atendi eble senfinan multon da
jaroj, tiam \^cia societo, \^cia aparta persono povus la\u u sia
propra gvido labori por la disvastigo de tiu \^ci lingvo; la nombro
da adeptoj de tiu \^ci lingvo kreskadus kun \^ciu horo, \^gia
literaturo rapido ri\^ci\^gadus, kongresoj internaciaj tuj povus
komenci uzadi gin por la reciproka kompreni\^gado de siaj membroj,
kaj en la plej mallonga tempo tiu \^ci lingvo tiom fortiki\^gus en
la tuta mondo, ke al la registaroj restus nur doni sian sankcion al
fakto jam plenumi\^ginta. \^Cu ni povas anta\u uvidi, kia lingvo
estos internacia? Feli\^ce ni povas respondi tiun \^ci demandon tute
pozitive: "jes, ni povas anta\u uvidi, kia lingvo estos internacia,
ni povas tion \^ci anta\u uvidi kun plena precizeco kaj certeco, sen
ia ombro da dubo".

   Por konvinki pri tio \^ci niajn a\u uskultantojn, ni petas ilin
prezenti al si, ke kongreso da reprezentantoj de diversaj regnoj jam
efektivi\^gis, kaj ni trarigardos, kian lingvon ili povus elekti. Ne
malfacile estos por ni pruvi, ke ekzistas nur {\sl unu sola} lingvo,
kiun ili povus elekti, kaj ke \^cia elekto de ia alia lingvo estus
por ili rekte {\sl ne ebla}, se ili e\^c {\sl volus} \^gin elekti,
kaj ke se ili kontra\u u \^ciu atendo kaj spite \^ciuj argumentoj de
la sana prudento tamen elektus ian alian lingvon, tiam kontra\u u
tio \^ci protestus la vivo mem kaj ilia elekto restus nur malviva
litero.

   Tiel ni prezentu al ni, ke la reprezentantoj de diversaj regnoj
kunveturi\^gis kaj ke ili alpa\^sas al la elekto de lingvo
internacia. Al ili anta\u ustarus la sekvanta: 1) a\u u elekti ian
el la ekzistantaj lingvoj {\sl vivantaj}, 2) a\u u elekti ian el la
lingvoj {\sl mortintaj} (ekzemple latinan, grekan, hebrean), 3) a\u
u elekti ian el la ekzistantaj lingvoj {\sl artaj}, 4) a\u u difini
komitaton, kiu okupus sin je la kreado de lingvo tute {\sl nova},
ankora\u u ne ekzistanta. Por ke niaj a\u uskultantoj povu pense
partopreni en la laboroj kaj konsideroj de la elektantoj, ni devas
anta\u ue konatigi ilin iom kun la karaktero de la nomitaj lingvaj
kategorioj. La karaktero de la lingvoj vivantaj kaj mortintaj estas
al la a\u uskultantoj pli a\u u malpli konata, ni diros tial kelke
da vortoj nur pri la lingvoj {\sl artaj}, kiuj por la plimulto de
niaj a\u uskultantoj prezentas kredeble absolutan "terra
incognita".

   Kiamaniere \^ce la homoj naski\^gis la ideo de lingvo arta, kiel tiu
\^ci ideo disvolvi\^gadis, trairadis diversajn stadiojn, komencante
de la plej malperfektaj pazigrafioj \^gis la plej perfekta tipo de
plena kaj ri\^ca lingvo, kia grandega multo da provoj estis farita
en tiu \^ci direkto, kia grandega multo da laboroj iris ofere por
tiu \^ci ideo en la da\u uro de la lastaj du centjaroj, --- pri
\^cio tio \^ci ni ne parolos, \^car por ela\u uskulti \^cion tion
\^ci vi ne havus sufi\^ce da tempo nek pacienco. Ni diros nur ion
pri la specialaj {\sl ecoj} de la artaj lingvoj, \^ce kio ni
kompreneble havos anta\u u la okuloj ne la diversajn malprosperajn
provojn anta\u uajn, kiuj la plimulton da analizataj de ni ecoj ne
posedas, sed la plej perfektan formon de lingvo internacia,
ekzistantan en la nuna tempo.

   Krom plena ne\u utraleco en rilato nacia, arta lingvo distingi\^gas per
la sekvantaj ecoj:

   1) \^Gi estas mireginde kaj nekredeble {\sl facila por ellernado}: sen
trograndigo oni povas diri, ke \^gi estas almena\u u {\sl kvindek}
fojojn pli facila, ol \^cia lingvo natura. Kiu ne konati\^gis kun
lingvo arta, ne povas e\^c kredi, \^gis kia grado atingas \^gia
facileco. La granda verkisto kaj filozofo Leono Tolstoj, kiun certe
neniu en la tuta mondo kura\^gos suspekti en tio, ke li volas fari
reklamon al la lingvo internacia, diris pri la lingvo Esperanto
jenon: "La facileco de \^gia ellernado estas tia, ke, ricevinte
Esperantan gramatikon, vortaron kaj artikolojn skribitajn en tiu
\^ci lingvo, mi post ne pli ol du horoj da sinokupado havis la eblon
se ne skribi, tamen libere legi en tiu \^ci lingvo. En \^cia okazo,
la oferoj, kiujn alportos \^ciu homo de nia E\u uropa mondo,
dedi\^cinte kelkan tempon al la ellernado de tiu \^ci lingvo, estas
tiel {\sl sensignifaj}, kaj la sekvoj, kiuj povas veni, se \^ciuj,
almena\u u la E\u uropanoj kaj Amerikanoj, aligos al si tiun \^ci
lingvon, estas tiel grandegaj, ke oni ne povas ne fari tiun \^ci
provon". Komprenu, sinjoroj, kion tio \^ci signifas: "{\sl post ne
pli ol du horoj da sinokupado}". Kaj en tiu \^ci sama maniero
esprimis sin pri la lingvo Esperanto \^ciuj tiuj senanta\u uju\^gaj
kaj honestaj homoj, kiuj, anstata\u u filozofadi pri \^gi blinde,
prenis sur sin la malgrandan laboron efektive konati\^gi kun \^gi.
Estas vero, ke homoj instruitaj povas ellerni Esperanton pli rapide,
ol homoj neinstruitaj, sed anka\u u la lastaj ellernas \^gin treege
kaj mireginde facile, \^car por la ellernado de la lingvo Esperanto
de la lernanto estas postulataj neniaj anta\u uaj scioj. Inter la
esperantistoj vi trovos multe da homoj tiom neinstruitaj, ke ili
\^gis nun en sia {\sl propra, patra} lingvo skribas treege malbone
kaj plene da eraroj, kaj tamen en la lingvo Esperanto ili skribas
tuto senerare, --- kaj ili ellernis tiun \^ci lingvon en la da\u uro
de iaj kelke da semajnoj, dum la ellernado de la lingvo natura \^ce
tiuj samaj personoj devus okupi almena\u u 4 a\u u 5 jarojn.

   Kiam en la jaro 1895 Odeson venis svedaj studentoj, kiuj sciis sole
svede kaj Esperante, unu \^{\j}urnalisto, kiu deziris interparoli
kun ili, matene la unuan fojon en sia vivo prenis en la manojn
lernolibron Esperantan kaj vespero en tiu sama tago li povis jam
sufi\^ce bone paroli kun la svedoj.

   De kie venas tia nekredebla facileco de lingvo internacia? \^Ciu
lingvo natura konstrui\^gis {\sl blinde}, per la vojo de amasi\^gado
de la plej diversaj kaj pure okazaj cirkonstancoj; tie agadis nenia
logiko, nenia difinita plano, sed simple nur la uzo: tian vorton
estas {\sl akceptite} uzi tiel, kaj tial ni devas \^gin uzi tiel,
--- tian vorton estas akceptite uzi alie, kaj tial ni devas \^gin uzi
alie. Jam anta\u ue tial oni povas diri, ke sistemo da sonoj por la
esprimado de pensoj, kiun kreos homa {\sl inteligenteco} konscie kaj
la\u u severe difinitaj kaj logikaj le\^goj, devos esti multe, tre
multe da fojoj pli facila, ol tia sistemo da sonoj, kiu
konstrui\^gis okaze kaj senkonscie. Ni ne havas la eblon analizi tie
\^ci la tutan tiun iron de la pensoj, je kiu gvidadis sin la a\u
utoroj de artaj lingvoj, nek montri detale \^ciujn tiujn grandegajn
faciligojn, kiujn arta lingvo posedas en komparo kun la naturaj,
\^car tio \^ci postulus tutan vastan traktaton, --- ni prenos tial
nur simple kelkajn {\sl ekzemplojn}. Tiel ekzemple preska\u u en
\^ciuj lingvoj \^ciu substantivo ial apartenas al tia a\u u alia
{\sl sekso}, ekzemple en la lingvo germana "kapo" havas viran
sekson, en la lingvo franca virinan kaj en la lingvo latina ne\u
utralan; \^cu ekzistas en tio \^ci ia e\^c plej malgranda senco a\u
u celo? Kaj tamen kian grandegan malfacilecon prezentas por la
lernanto la memorado de la sekso de \^ciu substantivo! Kiom multe la
lernanto devas ekzerci\^gadi, ekzerci\^gadi kaj denove
ekzerci\^gadi, anta\u u ol li venos al tiu perfekteco, ke li jam plu
ne konfuzi\^gados kaj ne diros ekzemple "le fin" anstata\u u "la
fin" a\u u "das Strick" anstata\u u "der Strick"! En lingvo
arta tiu \^ci sekso de la substantivoj estas {\sl tute
el\^{\j}etita}, \^car montri\^gis, ke \^gi ne havas en la lingvo
e\^c la plej malgrandan celon. Jen vi havas sekve jam unu ekzemplon
de tio, kiel per la plej bagatela rimedo estas atingita plej
grandega faciligo de la lingvo. En la lingvoj naturaj ekzistas la
plej komplikitaj kaj konfuzitaj deklinacioj kaj konjugacioj kun
grandega multo da diversaj formoj ne sole por \^ciuj deklinacioj kaj
konjugacioj, sed en \^ciu el ili ankora\u u tuta serio da formoj;
ekzemple en la konjugacioj vi havas ne sole por \^ciu tempo kaj modo
tutajn seriojn da formoj, sed en \^ciu el tiuj \^ci tempoj kaj modoj
ankora\u u apartajn formojn por \^ciu persono kaj nombro. Oni
ricevas vicon da grandegaj gramatikaj tabeloj, kiujn oni devas
ellerni kaj konservadi en la memoro; sed tio \^ci estas ankora\u u
nur komenco: al tio \^ci ali\^gas multego da deklinacioj kaj
konjugacioj {\sl neregulaj}, \^ciu kun aparta serio da formoj, kaj
\^cion tion \^ci oni devas ne sole ellerni kaj konservadi en la
memoro, sed konstante memori, kia vorto \^san\^gi\^gas la\u u la
regulaj deklinacioj a\u u konjugacioj kaj kia la\u u la malregulaj
kaj la\u u kia nome de tiuj \^ci regulaj a\u u malregulaj tabeloj la
donita vorto \^san\^gi\^gas. La ekposedo de \^cio tio \^ci postulas
inferan paciencon, multegon da tempo kaj konstantan sen\^cesan
ekzerci\^gadon. Dume lingvo arta anstata\u u tiu \^ci tuta \^haoso,
kiu postulas tutajn jarojn da pacienca laborado, donas al vi sole
nur 6 vortetojn "i, as, is, os, us, u", kiujn vi povas perfekte
ekposedi en la da\u uro de kelke da minutoj kaj kiujn vi jam neniam
forgesos kaj neniam konfuzos. Vi kun mirego demandas: "Kiel tio
\^ci estas ebla?" Jen, tre simple: Esperanto diras al vi, ke
deklinacioj {\sl tute} estas bezonaj {\sl neniaj}, \^car ili plene
anstata\u ui\^gas per la prepozicioj, kiujn ni ja sen tio uzas, kaj
en la konjugacioj ne sole sufi\^cas {\sl unu} tabelo por {\sl
\^ciuj} verboj, sed e\^c por tiu \^ci tabelo tute sufi\^cas, se \^gi
enhavas en si (krom participoj, prezentantaj apartan formon) sole
nur 6 fini\^gojn, nome: por la tempoj estanta, estinta kaj estonta
kaj por la modoj sendifina, kondi\^ca kaj ordona. Vi certe en la
unua minuto ekpensos, ke dank' al tiu \^ci malgrandega tabelo da
konjugacioj la lingvo perdas sian flekseblecon? Tute neniel:
konati\^gu kun la lingvo arta, kaj vi ekvidos, ke \^gia konjugacio
esprimas \^ciujn nuancojn de la penso senkompare pli bone kaj pli
precize, ol la plej komplikitaj kaj konfuzitaj tabeloj de la lingvoj
naturaj, \^car la arta lingvo for\^{\j}etis ne tion, kion la lingvo
bezonas, sed nur tion, kio prezentas en \^gi absolute superfluan kaj
tute al nenio servantan balaston. Efektive, por kio ni bozonas
apartan serion da fini\^goj por \^ciu persono kaj nombro, kaj al tio
\^ci en \^ciu tempo kaj en \^ciu modo apartan {\sl novan} serion da
tiuj \^ci fini\^goj, se \^ciuj tiuj \^ci fini\^goj estas ja tute
superfluaj, \^car la pronomo, kiu staras anta\u u la verbo, jam tute
sufi\^ce montras \^gian personon kaj nombron?

   La ortografio en la plimulto da lingvoj (kaj al tio \^ci la plej multe
\^guste en tiuj lingvoj, kiuj la plej multe havas da \^sancoj por
esti elektitaj kiel internaciaj) prezentas veran krucon por la
lernanto: en unu vorto la donita litero estas elparolata, en alia
\^gi ne estas elparolata a\u u estas elparolata alie, en unu vorto
la donita sono estas skribata tiel, en alia vorto alie\dots Tutajn
jarojn devas uzi franco a\u u anglo por tio, ke li ekpovu regule
skribi en sia patra lingvo! Radikale \^san\^gi tiun \^ci ortografion
estas absolute ne eble, \^car tiam grandega multo da vortoj, kiuj
diferencas unu de alia a\u u neniel, a\u u nur per ia apena\u u
rimarkebla nuanco en la elparolado, fari\^gus en la skribado tute
nediferencigeblaj unu de alia. La lingvo arta donis al \^ciu sia
litero klaran, severe difinitan kaj \^ciam egalan elparoladon, kaj
dank' al tio en la lingvo arta la demando de ortografio tute ne
ekzistas, kaj jam post, kvarono da horo da sinokupado je la arta
lingvo (t. e. simple post la konati\^go kun \^gia plej simpla
alfabeto) \^ciu skribos en \^gi diktaton tute senerare, dum en
lingvo natura li atingos tion \^ci nur post tutaj jaroj da malfacila
kaj enua laborado.

   Jam el tiuj kelkaj ekzemploj, kiujn ni donis, vi povas ricevi ideon
pri tio, kian grandegan faciligon donas al la lingvo la enmiksi\^go
de konscia arto. Ni povus citi kompreneble ankora\u u multe da aliaj
ekzemploj, \^car sur \^ciu pa\^so ni renkontas en la lingvoj naturaj
grandegajn malfacila\^{\j}ojn kaj konfuza\^{\j}ojn, kiuj en lingvo
arta estas a\u u tute el\^{\j}etitaj kiel superflua balasto, a\u u
alkondukitaj al ia unu a\u u du mallongaj vortetoj a\u u reguloj,
sen ia e\^c plej malgranda difekto por la fleksebleco, ri\^ceco kaj
precizeco de la lingvo. Tamen ni ne parolos pli pri tio \^ci, sed ni
diros sole tion, ke {\sl la tuta gramatiko de la lingvo Esperanto
konsistas tuta nur el 16 mallongaj reguletoj, kiujn \^ciu povas
bonege ellerni en la da\u uro de duonhoro!} Post unu sola duonhoro
da laborado super Esperanto la lernanto en tia grado ekposedas la
tutan gramatikon kaj tutan konstruon de la lingvo, ke al li restas
jam nur la simpla kaj facila akirado de provizo da vortoj! Por
kompreni kaj taksi la tutan gravecon de tio \^ci, imagu al vi, ke vi
entreprenis la ellernadon de ia lingvo natura kaj ke post kelke da
jaroj da pacienca laborado vi fine venis al tio, ke vi ekposedis la
konstruon de la lingvo en perfekteco kaj ke vi estas certaj, ke vi
jam plu neniam povas fari en tiu \^ci lingvo ian gramatikan a\u u
ortografian eraron kaj ke vi nun jam nur bezonas simple lernadi kiom
eble pli da {\sl vortoj}, --- vi ja tiam sentus vin feli\^ca kaj
dirus, ke la plej malfacilan kaj plej enuan parton de la laboro vi
jam pasis\dots \^Cu ne vere? En la lingvo Esperanto vi atingas tion
\^ci {\sl jam post duonhoro da laborado!}

   Sekve se Esperanto havus e\^c {\sl nur} tiun econ, pri kiu ni supre
parolis, t. e. grandegan facilecon kaj regulecon de la gramatiko kaj
ortografio, ni jam devus diri pri \^gi, ke \^gi estas multajn,
multajn fojojn pli facila, ol \^ciu lingvo natura. Sed per tio \^ci
ne fini\^gas ankora\u u la facileco de la lingvo Esperanto. Kiam vi
venis al tio, ke al vi restas jam nur simpla ellernado de {\sl
vortoj}, vi e\^c tie \^ci renkontos ankora\u u grandegajn
faciligojn. Tiel ekzemple jam la reguleco mem de la lingvo donas al
vi grandegan ekonomion en la nombro de la vortoj, kiujn vi bezonos
lerni; \^car sciante de la vorto la formon substantivan, vi jam
anta\u ue, kaj sen ia lernado, scias anka\u u \^gian adjektivon kaj
\^gian adverbon kaj \^gian verbon k. t. p., dum en \^ciu lingvo
natura tre multaj esprimoj havas por \^ciu parto de parolo apartan
vorton (ekzemple: {\sl parler}, {\sl oral}, {\sl verbalement}).
Havante plenan kaj ne limigitan rajton kunigi \^cian vorton kun
\^cia prepozicio kaj kun \^cia alia vorto, vi estas liberigitaj de
la lernado de multego da vortoj, kiuj en la lingvoj naturaj havas
por si apartajn radikojn nur tial, ke tia a\u u alia kunigo de
vortaj estas ial ne permesita. Sed krom tiuj \^ci {\sl naturaj}
vortfaraj oportuna\^{\j}oj, kiujn la lingvo Esperanto havas, vi
trovos en \^gi ankora\u u apartajn, tiel diri {\sl artajn} rimedojn,
kiuj enkondukas grandegan ekonomion \^ce la lernado de la vortoj.
Tiaj ekzemple estas \^giaj sufiksoj kaj prefiksoj, el kiuj ni citos
nur kelkajn pro ekzemplo: la prefikso "mal" donas al la vorto
sencon rekte kontra\u uan ("bona" bon
--- "malbona" mauvais), --- sekve sciante la vortojn "bona, mola,
varma, lar\^ga, supre, ami, estimi" k. t. p., vi jam mem povas
formi la vortojn "malbona, malmola, malvarma, mallar\^ga, malsupre,
malami, malestimi" k. t. p., aldonante al la vorto jam konata de vi
nur la prefikson "mal"; la sufikso "in" signifas la virinan
sekson ("patro" pere --- "patrino" mere), --- sekve sciante la
vortojn "patro, frato, onklo, fian\^co, bovo, koko" k. t. p., vi
jam estas liberigitaj de la lernado de la vortoj "patrino, fratino,
onklino, fian\^cino, bovino, kokino" k. t. p.; la sufikso "il"
signifas ilon ("tran\^ci" trancher --- "tran\^cilo" couteau),
--- sekve sciante la vortojn "tran\^ci, kombi, tondi, pafi, sonori,
plugi" k. t. p., vi jam mem scias la vortojn "tran\^cilo, kombilo,
tondilo, pafilo, sonorilo, plugilo" k. t. p. Da tiaj vortopartetoj,
kiuj treege malgrandigas la vortaron de la lingvo, ekzistas ankora\u
u multaj aliaj.

   Rememoru do nun \^cion, kion ni diris pri la konstruo de arta lingvo,
kaj tiam vi facile konsentos, ke se ni diris, ke lingvo arta estas
almena\u u 50 fojojn pli facila, ol natura, en tio \^ci estis nenia
trograndigo. Notu al vi en la memoro tiun \^ci grandegan facilecon
de arta lingvo, \^car ni al \^gi poste ankora\u u revenos.

   2.) La dua distingi\^ga eco de lingvo arta estas \^gia {\sl perfekteco},
kiu konsistas en matematika precizeco, fleksebleco kaj senlima
ri\^ceco. Ke lingvo arta posedos tian econ, tion \^ci ankora\u u
anta\u u la apero de la unua arta lingvo anta\u uvidis kaj anta\u
udiris \^ciuj tiuj eminentaj kapoj, kiuj rilatis al tiu \^ci por la
homaro treege grava ideo pli serioze, ol diversaj nuntempaj
Jupiteroj, kiuj pensas, ke \^cia e\^c plej supra\^{\j}a konati\^go
kun la esenco de lingvoj artaj malaltigus ilian honoron kaj indecon.
Ni povus citi tiajn grandajn lumojn, kiel ekzemple Bacon, Leibnitz,
Pascal, de Brosses, Condillac, Descartes, Voltaire, Diderot, Volney,
Ampere, Max Müller k. t. p., --- sed ni rigardas la citatojn kiel
batalilon nur de pse\u udoinstruitaj sofistoj; tial, ne fanfaronante
per citatoj, ni penos pruvi \^cion nur per la sola logiko. Ke lingvo
arta ne sole povas, se {\sl devas} esti pli perfekta, ol lingvoj
naturaj, tion \^ci komprenos \^ciu mem, se li konsideros la jenon:
\^Cia lingvo natura konstrui\^gadis per tia vojo, ke unu ripetadis
tion, kion li a\u udis de aliaj; nenia logiko, nenia konscia decido
de la homa inteligenteco tie \^ci havis ian forton. \^Cian esprimon,
kiun vi multajn fojojn {\sl a\u udis}, vi povas uzi, kaj \^cian
esprimon, kiun vi ankora\u u neniam a\u udis, estas al vi
malpermesite uzi. Tial ni en \^cia lingvo natura sur \^ciu pa\^so
renkontas la sekvantan aperon: en via cerbo aperas ia
komprena\^{\j}o, sed\dots vi ne havas la eblon esprimi \^gin per
bu\^sa vorto kaj devas tial helpi al vi per tuta multvorta kaj tre
neoportuna {\sl priskribo} de tiu komprena\^{\j}o, kiu en via cerbo
ekzistas kiel {\sl unu} komprena\^{\j}o, kiel unu spirita vorto.
Tiel ekzemple, dank'al tio, ke je lavado de tola\^{\j}o sin okupas
ordinare virinoj, vi en \^ciu lingvo havas vorton por esprimo de la
komprena\^{\j}o "lavistino"; sed se {\sl viro} ekvolos okupadi sin
je lavado de tola\^{\j}o, vi en tre multaj lingvoj staras jam
senhelpe kaj ne scias, kiel nomi tian homon, \^car nomon de viro,
kiu okupas sin je lavado de tola\^{\j}o, vi neniam a\u udis! Je
kuracado \^gis nun okupadis sin sole viroj; sed kiam ekaperis
kuracistinoj, a\u u virinoj posedantaj ian sciencan rangon, por ili
en la plimulto da lingvoj ne trovi\^gis vorto! Por la esprimo de
ilia nomovorto oni bezonis jam helpi al si per priskriba uzo de
kelke da vortoj, kaj kiam vi ankora\u u el ilia titolo volas fari
adjektivon, verbon k. t. p. --- tio \^ci estas jam tute ne ebla! En
\^ciu lingvo vi trovos multe da substantivoj, kiuj ne havas tiun a\u
u alian sekson, tiun a\u u alian kazon, tiun a\u u alian devenan
formon; adjektivojn, kiuj ne havas tiun a\u u alian gradon de
komparado, tiun a\u u alian formon; verbojn, kiuj ne havas tiun a\u
u alian tempon, personon, modon k. t. p.; de tia substantivo vi ne
povas fari adjektivon, de tia verbo vi ne povas fari substantivon k.
t. p. \^Car, ni ripetas, \^ciu lingvo natura estas fondita ne sur la
logiko, sed sur la blinda "oni tiel parolas" a\u u "oni tiel ne
parolas" ; sekve \^cian komprena\^{\j}on, kiu naski\^gas en via
cerbo, sed por kiu vi \^gis nun vortan esprimon ne a\u udis, vi
ordinare esprimi ne havas la eblon kaj vi devas helpi al vi per
priskriboj. Sed en lingvo arta, konscie fondita sur la severaj,
permesantaj nenian escepton nek arbitron le\^goj de pensado, nenio
simila povas havi lokon. Esprimoj en la speco de "tia vorto ne
havas tiajn formojn a\u u ne permesas tiajn ideajn kuni\^gojn" ---
en lingvo arta estas tute ne eblaj. Supozu ekzemple, ke morga\u u
viro ricevas la eblon naski infanojn a\u u nutri ilin per siaj
mamoj, --- kaj por li tuj ekzistas en la lingvo preta vorto, \^car
en lingvo arta estas ne ebla la ekzistado de ia vorto por unu sekso
kaj neekzistado por la dua sekso. Supozu, ke morga\u u iu elektas al
si ian novan, e\^c la plej strangan profesion, ekzemple laboradon je
aero, --- por li tuj ekzistas prete vorto, \^car se en lingvo arta
nur ekzistas sufikso por la esprimo de profesio, \^gi donas al vi
jam la eblon esprimi {\sl \^cian} profesion, kia nur povus aperi en
via cerbo.

   Krom tio ne forgesu, ke la perfekti\^gado de lingvo arta estas
ebla \^gis senfineco, \^car {\sl \^ciun} bonan regulon, bonan
formon, bonan esprimon, kiu ekzistas en kiu ajn lingvo, lingvo arta
havas plenan rajton aligi al si, \^cian mankon, kiu povus trovi\^gi
en \^gi, \^gi havas la rajton plibonigi kaj \^san\^gi, dum en lingvo
natura pri nenio simila povas esti parolo, \^car tiam lingvo natura
transformi\^gus jam en artan.

   Krom la analizitaj de ni du grandegaj plibona\^{\j}oj de lingvo arta
(eksterordinara facileco kaj perfekteco), ekzistas ankora\u u aliaj,
pri kiuj ni ne parolos. Ni transiru nun rekte al la {\sl
malbona\^{\j}oj} de lingvo arta. Kiu e\^c malmulte devus esti unu
vorto \^ci tiel] konati\^gis kun lingvo arta kaj havas sufi\^ce da
kura\^go por kredi tion, kion li {\sl vidas}, kaj ne kun fermitaj
okuloj ripetadi fremdajn frazojn, tiu povas veni nur al unu
konkludo, nome, ke mankojn en komparo kun lingvo natura lingvo arta
posedas {\sl neniajn}. \^Ciu el vi, vere, havis la okazon a\u udi
tre multajn atakojn kontra\u u lingvo arta; sed kontra\u u \^ciuj
tiuj \^ci atakoj ni povas rediri nur unu respondon: \^ciuj ili
eliras el la bu\^so de homoj, kiuj pri lingvo arta havas nenian
scion kaj neniam \^gin e\^c vidis, --- ne sole ne vidis kaj ne
esploris, sed e\^c neniam logike enpensi\^gis pri \^gia esenco, kaj
anstata\u u ekpensi pri tio, kion ili parolas, preferas blinde
\^{\j}etadi dekstren kaj maldekstren la\u utajn kaj modajn sed
sensencajn frazojn. Se ili e\^c iom konati\^gus kun lingvo arta, ili
ekvidus, ke iliaj frazoj estas tute falsaj; se ili, e\^c tute ne
konati\^gante kun la lingvo, simple ekpensus pri \^gi teorie, ili
ekvidus, ke \^ciuj iliaj frazoj ne havas e\^c la plej malgrandan
fundamenton. Se iu ekdeziros kredigi, ke en la najbara urbo \^ciuj
domoj estas konstruitaj el papero kaj ke \^ciuj homoj tie estas sen
manoj kaj sen piedoj, --- li povas imponi per tio \^ci al la {\sl
amaso}, kiu \^ciun vorton elparolitan per a\u utoritata tono de
scianta sankte kredas; sed homo {\sl prudenta} jam de anta\u ue
rilatos al tiuj \^ci vortoj tre kritike, \^car jam en sia sa\^go li
trovos neniajn akcepteblajn fundamentojn por tiuj \^ci frazoj; kaj
kiam \^ce li restos ia dubo, li simple iros en la najbaran urbon kaj
{\sl rigardos}, kaj tiam li konvinki\^gos, ke \^ciuj frazoj, kiujn
li a\u udis, estas la plej absoluta sensenca\^{\j}o. Tiel estas
anka\u u kun lingvo arta: anstata\u u blinde ripetadi frazojn, vi
bezonas nur simple {\sl ekpensi} pri la esenco de tiuj \^ci frazoj,
kaj vi jam komprenos, ke ili ne havas e\^c la plej malgrandan
fundamenton ; kaj se teoria pripenso estos por vi ankora\u u
nesufi\^ca, tiam iru simple kaj {\sl rigardu}, \^{\j}etu okulon en
lernolibron de lingvo arta, konati\^gu kun la konstruo de tiu \^ci
lingvo, profundi\^gu iom en \^gian jam tre ri\^can kaj diversaspecan
literaturon, faru ian provon, rigardu la {\sl faktojn}, kiuj sur
\^ciu pa\^so sin trovas anta\u u via nazo,
--- kaj tiam vi komprenos, kia grandega sensenca\^{\j}o sin trovas en
\^ciuj tiuj frazoj, kiujn vi a\u udadis kontra\u u lingvo arta. Vi
a\u udis ekzemple frazon "lingvo ne povas esti kreita en kabineto,
kiel viva ekzista\^{\j}o ne povas esti kreita en la retorto de
\^hemiisto"; tiu \^ci frazo sonas tiel bele kaj "sa\^ge", ke por
la grandega plimulto da homoj \^gi jam lasas nenian dubon pri tio,
ke lingvo arta estas infana\^{\j}o. Kaj tamen se tiuj \^ci homoj
havus tiom da propra kritika kapablo, por meti malgrandan, tre
malgrandan demandon "kial?" --- tiam tiu \^ci la\u uta frazo per
unu fojo perdus en iliaj okuloj \^cian sencon, \^car ili ekvidus, ke
nenia logika respondo ekzistas, ke tiu \^ci frazo estas bele sonanta
kolekto da vortoj, kiu ne havas e\^c la plej malgrandan logikan
fundamenton. Tute tian saman frazon oni ja povus uzi anka\u u
kontra\u u la arta alfabeto, kiun la homaro jam tiel longe uzas kun
la plej granda utilo por si, kaj kontra\u u la arta veturado per
helpo de vaporo a\u u velocipedo, kaj kontra\u u nia tuta arta
civilizacio! Kaj tiun \^ci kaj similajn frazojn la homoj obstine
ripetadas \^ciun fojon, kiam aperas ia nova utila ideo\dots Ho,
frazo, frazo, frazo, kiam vi \^cesos regni super la spiritoj de la
homoj!

   Vi a\u udis, ke arta lingvo estas ne ebla, ke en \^gi homoj ne
komprenados unu alian, ke \^ciu popolo uzados \^gin alie, ke en \^gi
oni nenion povas esprimi, k. t. p. k. t. p. Se ni turnos atenton, ke
\^cio tio \^ci estas aferoj, kiujn \^ce plej malgranda dozo da
honesteco kaj bona volo \^ciu povas facile praktike {\sl
trakontroli}, kaj ke \^ciuj tiuj \^ci frazistoj simple ne {\sl
volas} trakontroli tion, pri kio ili parolas kun tia a\u utoritata
tono, sed pro la apla\u udado de la amaso ili preferas fermi la
okulojn kaj \^{\j}etadi koton sur la aferon nur tial, ke \^gi estas
nova kaj ankora\u u ne moda, --- tiam \^ciuj tiuj \^ci frazoj
montri\^gas jam ne sole ridindaj, sed rekte {\sl indignigaj}.
Anstata\u u blinde \^{\j}etadi frazojn, iru kaj {\sl rigardu}, ---
kaj tiam vi vidos, ke \^ciuj viaj vortoj estas simple senceremonia
{\sl mensogado}: vi ekvidos, ke en efektiveco lingvo arta fakte de
longe {\sl jam ekzistas}, ke homoj de la plej diversaj nacioj jam
longe kun la plej granda utilo por si {\sl uzas} \^gin, ke ili unu
alian {\sl komprenas bonege kaj plej precize}, kiel skribe, tiel
anka\u u bu\^se ; ke homoj de \^ciuj nacioj uzas \^gin \^ciuj {\sl
tute egale}; \^gia jam tre ri\^ca kaj diversaspeca literaturo
montros al vi tute okulvideble, ke \^ciuj nuancoj de la homa penso
kaj sento povas esti esprimitaj en \^gi en la plej bona maniero\dots
Anstata\u u blinde babiladi diversan teorian sensenca\^{\j}on, iru
kaj rigardu la {\sl faktojn}, la longe jam ekzistantajn, de \^ciu
facile kontroleblajn sendubajn kaj nemalkonfeseblajn faktojn, ---
kaj tiam por vi restos nenia dubo pri tio, ke iaj motivoj,
parolantaj kontra\u u la enkonduko de lingvo arta en komunan uzadon,
ekzistas absolute {\sl neniaj}.

   Ni revenu nun al tio, pri kio ni parolis en la komenco de tiu
\^ci \^capitro, t. e. ni prezentu al ni, ke kolekti\^gis kongreso el
reprezentantoj de \^ciuj plej gravaj regnoj, por elekti lingvon
internacian. Ni rigardu, kian lingvon ili povas elekti. Ne malfacile
estos pruvi, ke ilian elekton ni povas anta\u uvidi ne sole kun tre
granda {\sl kredebleco}, sed e\^c kun plena {\sl certeco} kaj
precizeco.

   El \^cio, kion ni supre diris pri la grandegaj plibona\^{\j}oj de la
lingvoj artaj en komparo kun la lingvoj naturaj, jam per si mem
sekvus, ke esti elektita povas nur lingvo {\sl arta}. Ni supozu
tamen por unu minuto, ke la tuta kongreso malfeli\^ce konsistos
plene el la plej obstinaj rutinuloj kaj malamikoj de \^cio nova kaj
ke al ili venos en la kapon la ideo pli bone elekti ian en \^ciuj
rilatoj maloportunan lingvon {\sl naturan}, ol centoble pli
oportunan lingvon artan. Ni rigardu, kio tiam estos. Se ili
ekdeziros elekti ian lingvon vivantan de ia el la ekzistantaj
nacioj, tiam kiel grandega malhelpo tie \^ci aperos ne sole la
reciproka envio de la popoloj, sed anka\u u la tute natura timo de
\^ciu nacio jam simple pro sia ekzistado: \^car estas afero tute
komprenebla, ke tiu popolo, kies lingvo estos elektita kiel
internacia, balda\u u ricevos tian grandegan superforton super
\^ciuj aliaj popoloj, ke \^gi ilin simple dispremos kaj englutos.
Sed ni sipozu, ke la delegatoj de la kongreso jam tute ne atentos
tion \^ci, a\u u ke ili, por eviti reciprokan envion kaj engluton,
elektos ian lingvon {\sl mortintan}, ekzemple la latinan, --- kio do
tiam estos? Estos tre simple tio, ke la decido de la kongreso restos
simple {\sl malviva litero} kaj fakte {\sl neniam atingos
efektivi\^gon}. \^Ciu lingvo natura, kiel viva, tiel ankora\u u pli
mortinta, estas tiel terure malfacila, ke la almena\u u iom fonda
ellernado de \^gi estas ebla nur por personoj, kiuj posedas grandan
kvanton da libera tempo kaj grandajn monajn rimedojn. Ni havus sekve
ne lingvon internacian en la vera senco de tiu \^ci vorto, sed nur
lingvon internacian por {\sl la pli altaj klasoj de la societo}. Ke
la afero starus tiel kaj ne alie, tion \^ci montras al ni ne sole la
logiko, sed tion \^ci jam longe okulvideble montris al ni {\sl la
vivo mem}: en efektiveco la lingvo latina estas ja de longe jam
elektita de \^ciuj registaroj kiel internacia, kaj jam longe en la
gimnazioj en \^ciuj landoj la\u u ordono de la registaroj la
junularo devigite dedi\^cas tutan vicon da jaroj al la ellernado de
tiu \^ci lingvo,
--- kaj tamen \^cu ekzistas multe da homoj, kiuj libere posedas la
lingvon latinan? La decido de la kongreso sekve donus al ni nenion
novan, sed aperus nur kiel sencela kaj senfrukta ripeto de tiu
decido, kiu longe jam estas farita kaj e\^c efektivigita, sed sen ia
rezultato. En nia tempo nenia decido e\^c de plej autoritata
kongreso neniam povas jam doni al la lingvo latina tiun forton, kiun
\^gi havis en la mezaj centjaroj: tiam ne sole por \^gia
internacieco, sed e\^c por \^gia absoluta re\^gado unuanime staris
\^ciuj registaroj, la tuta societo, la tuta \^ciopova eklezio, kaj
e\^c la vivo mem, tiam \^gi prezentis la fundamenton de \^cia
scienco kaj de \^cia scio, tiam al \^gi estis dedi\^cata la pli
granda parto de la vivo, \^gi elpu\^sis per si \^ciujn lingvojn
patrajn, \^gi estis ellernata kaj prilaborata {\sl devigite}, jam
simple pro tio, ke en siaj patraj lingvoj la instruituloj simple ne
havis la {\sl eblon} esprimadi sin,
--- kaj tamen malgra\u u \^cio la lingvo ne sole falis, sed e\^c en
sia plej bona tempo \^gi estis kaj povis esti nur apartena\^{\j}o de
la {\sl elektitaj klasoj} de la societo! Dume en okazo de elekto de
lingvo {\sl arta} \^gin post kelke da monatoj povus bonege posedi
jam per unu fojo la tuta mondo, {\sl \^ciuj} sferoj de la homa
societo, ne sole la inteligentaj kaj ri\^caj, sed e\^c la plej
malri\^caj kaj senkleraj vila\^ganoj.

   Vi vidas sekve, ke la estonta kongreso tute {\sl ne povas} elekti ian
alian lingvon krom lingvo arta. Elekti ian lingvon naturan, kiam ni
havas la eblon elekti lingvon artan, kiu havas anta\u u la lingvoj
naturaj en \^ciuj rilatoj senduban kaj okulvideblan grandegan
plibonecon, estas tia sama malsa\^ga\^{\j}o, kiel ekzemple sendi ion
el Parizo Peterburgon per \^cevaloj, kiam ni havas la eblon fari
tion \^ci per fervojo. Nenia kongreso povas fari tian elekton; sed
se ni e\^c supozus, ke la kongreso tiom malmulte pripensos kaj estos
tiel blindigita per la rutinaj kutimoj, ke \^gi tamen faros tian
absurdan elekton, tiu \^ci elekto per la forto de la cirkonstancoj
tute egale restos malviva litero kaj en la vivo la demando pri
lingvo internacia fakte restos nesolvita tiel longe, \^gis pli a\u u
malpli frue kolekti\^gos nova kongreso kaj elektos lingvon artan.

   Tiel ni petas vin noti al vi en la memoro tiun konkludon, al kiu ni
venis, nome: {\sl lingvo internacia de la venontaj generacioj estos
sole kaj nepre nur lingvo arta}.

\begin{center}
\textbf{VI}
\end{center}

   Nun restas ankora\u u solvi la demandon, {\sl kia} nome arta lingvo estos
enkondukita en komunan uzadon. En la unua minuto \^sajnas, ke
respondi tiun \^ci demandon ekzistas nenia eblo, \^car --- kiel vi
sendube diros --- "da lingvoj artaj ekzistas ja tre multe kaj ilia
nombro povas esti ankora\u u mil fojojn pli granda, \^car \^ciu
aparta homo povas krei apartan lingvon artan la\u u sia arbitro".
\^Cu ekzistas tial ia eblo anta\u uvidi, kiu el ili estos elektita?
Tiel efektive \^sajnas de la unua rigardo al la homoj ne konantaj la
aferon, kaj tamen anta\u uvidi, kaj anta\u udiri, kia lingvo arta
estos elektita, estas tre facile. Tio \^ci venas de tio, ke la supre
dirita en la publiko tre populara opinio pri la nombro de la
ekzistantaj kaj povantaj ankora\u u aperi lingvoj artaj estas tute
erara kaj estas fondita sur plena nesciado de la historio kaj esenco
de la artaj lingvoj.

   Anta\u u \^cio ni konstatas la fakton, ke, malgra\u u la grandega nombro
da personoj, kiuj laboris a\u u laboras super lingvoj artaj jam en
la da\u uro de 200 jaroj, \^gis nun aperis sole nur {\sl du}
efektive pretaj lingvoj, nome {s\l Volap\"uk} kaj {\sl Esperanto}.
Turnu atenton sur tion \^ci: sole nur {\sl du} artaj lingvoj. Estas
vero, ke vi preska\u u \^ciutage legas en la gazetoj, ke jen tie jen
aliloke aperis ankora\u u unu a\u u ankora\u u kelkaj artaj lingvoj,
oni citas al vi ilian nomon, oni donas al vi ofte e\^c notojn pri la
konstruo de tiuj \^ci lingvoj, oni alportas al vi kelkajn frazojn en
tiuj \^ci tiel nomataj novaj lingvoj, kaj al la publiko \^sajnas, ke
novaj artaj lingvoj elkreskadas kiel fungoj post pluvo. Sed tiu \^ci
opinio estas tute erara kaj venas de tio, ke la gazetoj ne trovas
necesa eni\^gi en tion, pri kio ili skribas, kaj kontenti\^gas nur
per tio, ke ili havas la eblon regali la legantojn per ridinda
nova\^{\j}o a\u u fari sprita\^{\j}on. Sciu do, ke \^cio tio, kio
\^ciutage estas alportata al vi de la gazetoj sub la la\u uta nomo
de "novaj lingvoj internaciaj", estas nur {\sl projektoj}, en
rapideco kaj sen sufi\^ca pripenso elbakitaj projektoj, kiuj de
reali\^go staras ankora\u u tre kaj tre malproksime. Tiuj \^ci
projektoj aperas jen en formo de mallongaj folietoj, jen e\^c en la
formo de dikaj libroj kun la plej la\u utaj kaj multepromesaj
frazoj,
--- aperas kaj tuj malaperas de la horizonto, kaj vi jam plu neniam
a\u udas ion pri ili; kiam la a\u utoroj de tiuj \^ci projektoj
alpa\^sas al ilia {\sl realigado}, ili tuj konvinki\^gas, ke tio
\^ci estas tute ne la\u u iliaj fortoj kaj ke tio, kio en teorio
\^sajnis afero tiel facila, en la praktiko montri\^gas tre malfacila
kaj neplenumebla. Kial la efektivigo de tiuj \^ci projektioj estas
tiel malfacila kaj tial \^gis hodia\u u aperis sole nur du efektive
pretaj kaj vivipovaj lingvoj --- pri tio ni parolos malsupre, kaj
dume ni nur turnas vian atenton sur tion, ke da lingvoj artaj \^gis
nun ekzistas sole nur du kaj ke sekve se la kongreso hodia\u u
efektivi\^gus, \^gi el la jam ekzistantaj artaj lingvoj havus por
elekto nur du. La problemo de la kongreso estus sekvo jam tute ne
tiel malfacila, kiel povus \^sajni de la unua rigardo. {\sl Kian} el
tiuj \^ci du lingvoj elekti --- la kongreso anka\u u ne
\^sanceli\^gus e\^c unu minuton, \^car la vivo mem jam longe solvis
tiun \^ci demandon en la plej senduba maniero kaj Volap\"uk \^cie
estas jam elpu\^sita de la lingvo Esperanto. La pliboneco de
Esperanto anta\u u Volap\"uk estas tiel frapante granda, ke \^gi
falas al \^ciu en la okulojn tuj de la unua rigardo kaj ne estas
malkonfesata e\^c de la plej fervoraj Volap\"ukistoj. Estos
sufi\^ce, se ni diros al vi la jenon: Volap\"uk aperis tiam, kiam la
entuziasmo de la publiko por la nova ideo estis ankora\u u tute
fre\^sa, kaj Esperanto, dank'al financaj malfacila\^{\j}oj de la a\u
utoro, aperis anta\u u la publiko kelkajn jarojn pli malfrue kaj
renkontis jam \^cie pretajn malamikojn; la Volap\"ukistoj havis por
sia agitado grandajn rimedojn kaj agadis per la plej vasta, pure
Amerika reklamo, kaj la Esperantistoj agadis la tutan tempon
preska\u u tute sen iaj materialaj rimedoj kaj montradis en sia
agado grandegan nelertecon kaj senhelpecon; kaj malgra\u u \^cio de
la unua momento de la apero de Esperanto ni vidas grandegan multon
da Volap\"ukistoj, kiuj transiris malka\^se al Esperanto, kaj
ankora\u u pli grandan multon da tiaj, kiuj, konsciante, ke
Volap\"uk staras multe pli malalte el Esperanto, sed ne dezirante
montri sin venkitaj, tute defalis de la ideo de lingvo internacia
entute,
--- dume en la tuta tempo de ekzistado de Esperanto (13 jaroj) nenie
sur la tuta tera globo trovi\^gis e\^c unu persono, --- ni ripetas,
e\^c {\sl unu} --- kiu transirus de Esperanto al Volap\"uk! Dum
Esperanto, malgra\u u la grandegaj malfacila\^{\j}oj, kun kiuj \^gi
devas batali, da\u ure vivas kaj floras kaj konstante pliforti\^gas,
Volap\"uk jam longe estas forlasita preska\u u de \^ciuj kaj povas
esti nomita jam de longe mortinta.

   En kio konsistas la pliboneco de Esperanto anta\u u Volap\"uk, ni
ne povas kompreneble analizi tie \^ci tute detale; pro ekzemplo ni
montros nur kelkajn punktojn: 1) Dum Volap\"uk sonas tre sova\^ge
kaj maldelikate, Esperanto estas plena je harmonio kaj estetiko kaj
memorigas per si la lingvon italan. 2) E\^c por tute neinstruitaj
Esperanto estas multe pli facila ol Volap\"uk, sed por instruitaj
\^gi estas tre multe pli facila, \^car \^giaj vortoj --- krom kelkaj
tre malmultaj --- ne estas arbitre elpensitaj, sed prenitaj el la
lingvoj romana-germanaj en tia formo, ke \^ciu ilin facile rekonas.
Tial \^ciu iom civilizita homo jam post kelkaj horoj da lernado
povas libere legi \^cian verkon en Esperanto jam preska\u u tute sen
vortaro. 3) Dum la uzanto de Volap\"uk nepre devas \^gin konstante
ripetadi, \^car alie li \^gin balda\u u forgesas (dank'al la plena
elpensiteco de la vortoj), la uzanto de Esperanto, unu fojon \^gin
ellerninte, jam \^gin ne forgesas, se li e\^c longan tempon \^gin ne
uzas. 4) Esperanto jam en la komenco estas tre facila por bu\^sa
{\sl interparolado}, dum en Volap\"uk oni devas tre longe kaj
pacience ekzercadi sian orelon, \^gis oni alkutimi\^gas al klara
diferencigado de la multaj reciproke tre simile sonantaj vortoj
(ekzemple "bap, pab, pap, paep, pep, poep, peb, boeb, bob, pop,
pup, bub, pub, pueb, bib, pip, puep" k. t. p., kiuj fari\^gas
ankora\u u pli similaj inter si, se oni prenas ilin en la
multenombro, t. e. kun {\sl s} en la fino). 5) En Volap\"uk, dank'al
kelkaj fundamentaj eraroj en la principoj de la konstruo (ekzemple:
vokaloj en la komenco a\u u fino de vorto ne povas esti uzataj,
\^car ili estas signoj gramatikaj), \^ciu nove bezonata vorto nepre
devas esti {\sl kreita} de la a\u utoro (e\^c \^ciuj nomaj {\sl
propraj}, ekzemple Ameriko = Melop, Anglujo = Nelij). Tio \^ci donas
ne sole grandegan nombron da tute superfluaj vortoj por lernado, sed
faras \^cian disvolvi\^gan pa\^son de la lingvo \^ciam dependa de
\^gia a\u utoro a\u u de ia ordonanta Akademio. Dume en Esperanto,
dank'al la plena seninflueco de la gramatiko sur la vortaron kaj
dank'al la regulo, ke \^ciuj vortoj "fremdaj", kiuj jam de si mem
estas internaciaj, estas uzataj sen\^san\^ge egale kiel en la aliaj
lingvoj, --- ne sole grandega multo da vortoj fari\^gas tute
senbezona por lernado, sed la lingvo ricevas la eblon disvolvi\^gadi
eterne \^ciam pli kaj pli, sen ia dependo de la a\u utoro a\u u de
ia Akademio.

   Parolante pri la pliboneco de Esperanto anta\u u Volap\"uk, ni tute
ne intencas per tio \^ci malgrandigi la meritojn de la elpensinto de
Volap\"uk. La meritoj de Schleyer estas grandegaj kaj lia nomo
\^ciam staros sur la plej honora loko en la historio de la ideo de
lingvo internacia. Ni volis nur montri, ke se hodia\u u
efektivi\^gus kongreso por la elekto de lingvo internacia, tiam \^gi
inter la amba\u u nun ekzistantaj lingvoj artaj ne povus
\^sanceli\^gi e\^c unu minuton.

   Ni pruvis sekve, ke se hodia\u u efektivi\^gus kongreso por la elekto
de lingvo internacia, tiam malgra\u u la grandega nombro da
ekzistantaj lingvoj ni povus jam {\sl nun} kun plena certeco kaj
precizeco anta\u uvidi, {\sl kian} lingvon \^gi elektos, nome: el
\^ciuj ekzistantaj lingvoj vivaj, mortintaj kaj artaj la kongreso
povas elekti sole nur {\sl unu} lingvon: {\sl Esperanto}. Kia ajn
estus la konsisto de la kongreso, kiaj ajn estus la politikaj
kondi\^coj, je kiaj ajn konsideroj, anta\u uju\^goj, simpatioj a\u u
antipatioj la kongreso sin gvidus, \^gi absolute {\sl ne povus}
elekti ian alian lingvon krom Esperanto, \^car por la rolo de lingvo
internacia Esperanto estas nun la {\sl sola} kandidato en la tuta
mondo, la sola, tute sen iaj konkurantoj. \^Car e\^c \^ce la plej
malprospera konsisto de la kongreso en \^gi tamen sidos homoj
pensantaj, kaj la grandega pliboneco de la lingvo Esperanto anta\u u
\^ciuj aliaj lingvoj tro forte falas en la okulojn al \^ciu, kiu
almena\u u iom konati\^gis kun tiu \^ci lingvo, tial estas tute ne
eble supozi, ke la kongreso elektos ian alian lingvon. Se tamen,
kontra\u u \^cia atendo, la kongreso estos tiom blindigita, ke \^gi
ekdeziros ian alian lingvon, tiam --- kiel ni pruvis supre --- la
{\sl vivo mem} zorgos pri tio, ke la decido de la kongreso restu
sole malviva litero, tiel longe, \^gis kolekti\^gos nova kongreso
kaj faros elekton \^gustan.

\begin{center}
\textbf{VII}
\end{center}

   Nun restas al ni respondi ankora\u u unu lastan demandon, nome: En la
{\sl nuna} minuto Esperanto, vere, aperas kiel {\sl sola} kandidato
por lingvo internacia; sed \^car kongreso de reprezentantoj de
diversaj regnoj por la elekto de lingvo internacia efektivi\^gos
kredeble ankora\u u ne balda\u u, eble post dek kaj eble post cent
jaroj, tre povas ja esti, ke \^gis tiu tempo aperos multaj {\sl
novaj} artaj lingvoj, kiuj staros multe pli alte ol Esperanto, kaj
sekve unu el {\sl ili} devos esti elektita de la kongreso? A\u u
eble la kongreso mem aran\^gos kompetentan komitaton, kiu okupos sin
je la kreo de nova arta lingvo?

   Je tio \^ci ni povas respondi jenon. La ebleco de apero de nova lingvo
per si mem estas tre duba, kaj komisii al komitato la kreadon de
nova lingvo estus tiel same sensence, kiel ekzemple komisii al
komitato verki bonan poemon. \^Car la kreado de plena, en \^ciuj
rilatoj ta\u uga kaj vivipova lingvo, kiu al multaj \^sajnas tia
facila kaj \^serca afero, en efektiveco estas afero tre kaj tre
malfacila. \^Gi postulas de unu flanko specialan talenton kaj
inspiron kaj de la dua flanko grandegan energion, paciencon kaj
varmegan, senfine aldonitan amon al la entreprenita afero. Multajn
niaj vortoj tre mirigos, \^car \^sajnas al ili, ke oni bezonas nur
decidi al si, ke tablo ekzemple estos "bam", se\^go estos "bim"
k. t. p. k. t. p., kaj lingvo jam estos preta. Kun la kreado de
pleneca, ta\u uga kaj vivipova lingvo estas tute tiel same, kiel
ekzemple kun la ludado sur fortepiano a\u u kun la trairado de
densega arbaro. Al homo, kiu ne konas la esencon de muziko,
\^sajnas, ke nenio estas pli facila, ol ludi fortepianon, --- oni ja
bezonas nur ekfrapi unu klavon, kaj estos ricevita tono, vi ekfrapos
alian klavon kaj vi ricevos alian tonon, --- vi frapados en la da\u
uro de tuta horo diversajn klavojn, kaj vi ricevos tutan
kompozicion\dots \^sajnas, ke nenio estas pli facila; sed kiam li
komencas ludi sian improvizitan kompozicion, \^ciuj kun ridego
diskuras, kaj e\^c li mem, a\u udante la ricevatajn de li sova\^gajn
sonojn, balda\u u komencos komprenetadi, ke la afero iel estas ne
glata, ke muziko ne konsistas en sola frapado de klavoj, --- kaj tiu
heroo, kiu kun tia memfida mieno sidi\^gis anta\u u la fortepiano,
fanfaronante, ke li ludos pli bone ol \^ciuj, kun honto forkuras kaj
jam plu ne montras sin anta\u u la publiko. Al homo, kiu neniam
estis en granda arbaro, \^sajnas, ke nenio estas pli facila, ol
trairi arbaron de unu fino \^gis la dua: "kio da artifika tie \^ci
estas? \^Ciu infano ja povas tion \^ci fari; oni bezonas nur eniri,
iri \^ciam rekte anta\u uen, --- kaj post kelke da horoj a\u u post
kelke da tagoj vi trovos vin en la kontra\u ua fino de la arbaro".
Sed apena\u u li eniris iom en la profunda\^{\j}on de la arbaro, li
balda\u u tiel perdas la vojon, ke li a\u u tute ne povas elrampi el
la arbaro, a\u u post longa vagado li eliras, sed tute, tute ne en
tiu loko, kie li eliri devis. Tiel estas anka\u u kun arta lingvo:
{\sl entrepreni} la kreadon de lingvo, doni al \^gi jam anta\u ue
nomon, trumpeti pri \^gi al la leganta mondo --- \^cio tio \^ci
estas tre facila, --- sed feli\^ce {\sl fini} tiun \^ci laboron
estas tute ne tiel facile. Kun memfida mieno multaj entreprenas tian
laboron; sed apena\u u ili iom enprofundi\^gis en \^gin, ili a\u u
ricevas senordan kolekton da sonoj sen ia difinita plano kaj sen ia
indo, a\u u ili pu\^si\^gas je tiom da malhelpoj, je tiom da
reciproke kontra\u uparolantaj postuloj, ke ili perdas \^cian
paciencon, \^{\j}etas la laboron kaj jam plu ne montras sin anta\u u
la publiko.

   Ke la kreado de ta\u uga kaj vivipova lingvo ne estas tiel facila
afero, kiel al multaj \^sajnas, oni povas interalie la plej bone
vidi el la sekvanta fakto: oni scias, ke \^gis la apero de Volap\"uk
kaj Esperanto estis grandega multo da diversaj provoj krei artan
lingvon internacian; ne malmulte da provoj aperis anka\u u post la
apero de la diritaj du lingvoj; grandegan serion da nomoj de tiuj
\^ci provoj kaj iliaj a\u utoroj vi trovos en \^cia historio de la
ideo de lingvo internacia; tiuj \^ci provoj estis farataj kiel de
privataj personoj, tiel anka\u u de tutaj societoj; ili englutis
grandegan multon da laboroj kaj kelkaj el ili englutis anka\u u tre
grandajn kapitalojn; --- kaj tamen el tiu \^ci tuta grandega nombro
nur {\sl du}, sole nur {\sl du} atingis efektivi\^gon kaj trovis
adeptojn kaj praktikan uzon! Sed anka\u u tiuj \^ci du aperis nur
{\sl okaze}, dank' al tio, ke unu el la a\u utoroj ne sciis pri la
laborado de la dua. La a\u utoro de la lingvo Esperanto, kiu
dedi\^cis al sia ideo sian tutan vivon, komencante de la plej frua
infaneco, kiu kun tiu \^ci ideo kunkreski\^gis kaj estis preta
\^cion oferi al \^gi, konfesas mem, ke lian energion subtenadis nur
tiu konscio, ke li kreas ion tian, kio ankora\u u neniam ekzistis,
kaj ke la malfacila\^{\j}oj, kiujn li renkontadis en la da\u uro de
sia laborado, estis tiel grandaj kaj postulis tiom multe da
pacienco, ke se Volap\"uk estus aperinta 5-6 jarojn pli frue, kiam
Esperanto ne estis ankora\u u finita, li (la a\u utoro de Esperanto)
certe perdus la paciencon kaj rifuzus la pluan laboradon super sia
lingvo, malgra\u u ke li tute bone konsciis la grandegan plibonecon
de sia lingvo anta\u u Volap\"uk.

   El \^cio supredirita vi komprenos, ke nun, kiam la tuta mondo scias,
ke du tute plenaj artaj lingvoj jam longe ekzistas, estas tre dube,
ke trovi\^gu iu, kiu entreprenus nun similan Sizifan laboron de la
komenco kaj havus sufi\^ce da energio, por alkonduki \^gin al
feli\^ca fino, tiom pli ke lin ne vigligus la espero doni iam ion
pli bonan ol tio, kio jam ekzistas. Kiom malmulte da espero havus
tia entreprenanto, oni vidas la plej bone el tiuj tre multaj provoj
kaj projektoj, kiuj aperis post Esperanto: malgra\u u ke la a\u
utoroj havis anta\u u si jam tute pretan modelon, la\u u kiu ili
povis labori, nenia el tiuj \^ci provoj ne sole ne eliris el la
regiono de projektoj, sed e\^c jam el tiuj \^ci projektoj mem oni
klare vidas, ke se iliaj a\u utoroj, havus la paciencon kaj povon
alkonduki ilin \^gis fino tiuj \^ci laboroj prezenti\^gus ne pli
bone, sed kontra\u ue, multe {\sl malpli bone} ol Esperanto. Dum
Esperanto bonege kontentigas {\sl \^ciujn} postulojn, kiuj povas
esti farataj al lingvo internacia (eksterordinara facileco,
precizeco, ri\^ceco, natureco, vivipoveco, fleksebleco, sonoreco k.
t. p.), \^ciu el tiuj projektoj penas plibonigi {\sl unu} ian
flankon de la lingvo, oferante por tio \^ci kontra\u uvole \^ciujn
{\sl aliajn} flankojn. Tiel ekzemple multaj el la plej novaj
projektistoj uzas la sekvantan ruza\^{\j}on: sciante, ke la publiko
taksos \^ciun projekton konforme al tio, kiel al \^gi rilatos la
instruitaj {\sl lingvistoj}, ili zorgas ne pri tio, ke ilia projekto
estu efektive ta\u uga por io en la praktiko, sed nur pri tio, ke
\^gi en la unua minuto faru bonan impreson sur la lingvistojn; por
tio ili prenas siajn vortojn preska\u u tute sen ia \^san\^go el la
plej gravaj jam {\sl ekzistantaj} lingvoj naturaj. Ricevinte frazon
skribitan en tia projektita lingvo, la lingvistoj rimarkas, ke ili
per la unua fojo komprenis tiun \^ci frazon multe pli facile ol on
Esperanto, --- kaj la projektistoj jam triumfas kaj anoncas, ke ilia
"lingvo" (se ili iam finos \^gin) estos {\sl pli bona} ol
Esperanto. Sed \^ciu prudenta homo tuj konvinki\^gas, ke tio \^ci
estas nur {\sl iluzio}, ke al la {\sl malgrava} principo, elmetita
pro montro kaj allogo, tie \^ci estas oferitaj la principoj plej
{\sl gravaj} (kiel ekzemple la facileco de la lingvo por la
nekleruloj, fleksebleco, ri\^ceco, precizeco k. t. p.), kaj ke se
simila lingvo e\^c povus esti iam finita, \^gi en la fino absolute
nenion donus! \^Car se la plej grava merito de lingvo internacia
konsistus en tio, ke \^gi kiel eble plej facile estu tuj komprenata
de la instruitaj {\sl lingvistoj}, ni ja por tio \^ci povus simple
preni ian lingvon, ekzemple la latinan, {\sl tute sen iaj
\^san\^goj}, --- kaj la instruitaj lingvistoj \^gin ankora\u u pli
facile komprenos per la unua fojo! La principo de kiel eble plej
malgranda \^san\^gado de la naturaj vortoj ne sole estis bone konata
al la a\u utoro de la lingvo Esperanto, sed \^guste de {\sl li} la
novaj projektistoj ja prenis tiun \^ci principon: sed dum Esperanto
prudente kontentigas tiun \^ci principon {\sl la\u u mezuro de
ebleco}, penante plej zorge, ke \^gi ne kontra\u uagadu al aliaj
{\sl pli gravaj} principoj de lingvo internacia, la projektistoj
turnas la lutan atenton {\sl nur} sur tiun \^ci principon, kaj
\^cion alian, nekompareble pli gravan, ili fordonas kiel oferon,
\^car kunigi kaj konsentigi inter si {\sl diversajn} principojn ili
ne povas kaj e\^c ne havas deziron, \^car ili e\^c mem ne esperas
doni ion pretan kaj ta\u ugan, sed ili volas nur fari efekton.

   El \^cio supredirita vi vidas, ke ne ekzistas e\^c la plej malgranda
ka\u uzo por timi, ke aperus ia nova lingvo, kiu elpu\^sus
Esperanton, --- tiun lingvon, en kiun estas enmetita tiom da
talento, tiom da oferoj kaj tiom da jaroj da pacienca kaj varmege
aldonita laboro, la lingvon, kiu en la da\u uro de multaj jaroj
estas jam elprovita en \^ciuj rilatoj kaj en la praktiko tiel bonege
plenumas \^cion tion, kion ni povas atendi de lingvo internacia. Sed
por vi, estimataj a\u uskultantoj, tio \^ci estas ne sufi\^ca: vi
deziras, ke ni donu al vi plenan kaj senduban logikan {\sl certecon}
pri tio, ke la lingvo Esperanto ne havos konkurantojn. Feli\^ce ni
trovas nin en tia situacio, ke ni povas doni al vi tiun \^ci plenan
certecon.

   Se la tuta esenco de lingvo arta konsistus en \^gia {\sl gramatiko}, tiam
de la momento de la apero de Volap\"uk la demando de lingvo
internacia estus solvita por \^ciam kaj iaj konkurantoj al la lingvo
Volap\"uk aperi jam ne povus; \^car malgra\u u diversaj eraroj la
gramatiko de Volap\"uk estas tiel facila kaj tiel simpla, ke doni
ion multe pli facilan kaj pli simplan oni jam ne povus. Lingvo nova
povus diferenci de Volap\"uk nur per kelkaj {\sl bagateloj}, kaj
\^ciu komprenas, ke pro {\sl bagateloj} neniu entreprenus la kreadon
de nova lingvo, kaj la mondo pro bagateloj ne rifuzus la jam tute
pretan kaj elprovitan lingvon. En la ekstrema okazo la estonta
akademio a\u u kongreso farus en la Volap\"uka gramatiko tiujn
negrandajn \^san\^gojn, kiuj montri\^gus utilaj, kaj lingvo
internacia sen ia dubo restus Volap\"uk, kaj \^cia konkurado estus
por \^ciam esceptita. Sed lingvo konsistas ne sole el gramatiko, sed
anka\u u el {\sl vortaro}, kaj la ellernado de la vortaro postulas
en lingvo arta cent fojojn pli da tempo, ol la ellernado de la
gramatiko. Dume Volap\"uk solvis {\sl nur} la demandon de la
gramatiko, kaj la vortaron \^gi lasis tute sen atento, doninte
simple tutan kolekton da diversaj elpensitaj vortoj, kiujn \^ciu
nova a\u utoro havus la rajton elpensi al si la\u u sia propra
deziro. Jen kial jam en la komenco mem de la ekzistado de Volap\"uk
e\^c la plej fervoraj Volap\"ukistoj nature timis, ke morga\u u
aperos nova lingvo, tute ne simila je Volap\"uk kaj inter la amba\u
u lingvoj komenci\^gos batalado. Tute alia afero estas kun
Esperanto: oni scias --- kaj tion \^ci e\^c por unu minuto nenia
esploranto neas, --- ke Esperanto solvis ne sole la demandon de la
gramatiko, sed anka\u u la demandon de la vortaro, sekve ne unu {\sl
malgrandan parton} de la problemo, sed la {\sl tutan} problemon. Kio
do en tia okazo restis por fari al la a\u utoro de ia nova lingvo,
se tia iam aperus? Al li restus jam nenio ol\dots eltrovi la jam
eltrovitan Amerikon! Ni prezentu al ni efektive, ke nun, malgra\u u
la jam ekzistanta, bonega en \^ciuj rilatoj, \^ciuflanke elprovita,
havanta jam multegon da adeptoj kaj vastan literaturon lingvo
Esperanto, aperis tamen homo, kiu decidis dedi\^ci tutan serion da
jaroj al la kreado de nova lingvo, ke li sukcesis alkonduki sian
laboron \^gis la fino kaj ke la lingvo proponita de li montri\^gas
efektive pli bona ol Esperanto, --- ni rigardu do, kian vidon havus
tiu \^ci lingvo. Se la gramatiko de la lingvo Esperanto, kiu donas
plenan eblon esprimi en la plej preciza maniero \^ciujn nuancojn de
la homa penso, konsistas tute el 16 malgrandaj reguletoj kaj povas
esti ellernita en duono da horo, --- tiam kion la nova a\u utoro
povus doni pli bonan? En ekstrema okazo li donus eble anstata\u u 16
reguloj 15 kaj anstata\u u 30 minutoj da laborado postulus 25
minutojn? \^Cu ne vere? Sed \^cu deziros iu pro tio \^ci krei novan
lingvon kaj \^cu la mondo pro tia bagatelo rifuzos la jam
ekzistantan kaj \^ciuflanke elprovitan? Sendube ne; en ekstrema
okazo la mondo diros: "se en via gramatiko ia bagatelo estas pli
bona ol en Esperanto, ni tiun \^ci bagatelon enkondukos en
Esperanton kaj la afero estos finita". Kia estos la vortaro de tiu
\^ci lingvo? En la nuna tempo nenia esploranto jam dubas, ke la
vortaro de lingvo internacia ne povas konsisti el vortoj arbitre
elpensitaj, sed devas konsisti nepre el vortoj romana-germanaj en
ilia plej komune uzata formo; tio \^ci estas ne por tio, ke --- kiel
opinias multaj pli novaj projektistoj
--- la instruitaj lingvistoj povu tuj kompreni tekston skribitan en
tiu \^ci lingvo (en tia afero, kiel lingvo internacia, la instruitaj
lingvistoj ludas la {\sl lastan} rolon, \^car por ili ja tia lingvo
estas {\sl malplej} bezona), sed pro aliaj, pli gravaj ka\u uzoj.
Tiel ekzemple ekzistas grandega nombro da vortoj tiel nomataj
"fremdaj", kiuj en \^ciuj lingvoj estas uzataj egale kaj al \^ciuj
estas konataj sen ellernado kaj kiujn ne uzi estus rekta absurdo; al
ili unisone devas soni anka\u u \^ciuj aliaj vortoj de la vortaro,
\^car alie la lingvo estus sova\^ga, sur \^ciu pa\^so estus
kunpu\^si\^go de elementoj, malkompreni\^goj, kaj la konstanta
regula ri\^ci\^gado de la lingvo estus malfaciligita. Ekzistas
ankora\u u diversaj aliaj ka\u uzoj, pro kiuj la vortaro devas esti
kunmetita nur el tiaj vortoj kaj ne el aliaj, sed pri tiuj \^ci ka\u
uzoj, kiel tro specialaj, ni tie \^ci detale ne parolos. Estos
sufi\^ce, se ni nur diros, ke \^ciuj plej novaj esplorantoj akceptas
tiun \^ci le\^gon por la vortaro kiel allasantan jam nenian dubon.
Kaj \^car la lingvo Esperanto \^guste per tiu \^ci le\^go sin gvidis
kaj \^car \^ce tiu \^ci le\^go granda arbitro en la elekto de vortoj
ekzisti ne povas, restas la demando, kion do povus al ni doni a\u
utoro de nova lingvo, se tia estus kreita? Estas vero, ke al tiu a\u
u alia vorto oni povus doni pli oportunan formon, --- sed da tiaj
vortoj ekzistas tre nemulte. Tion \^ci oni la plej bone vidas el
tio, ke kian ajn el la multaj projektoj aperintaj post Esperanto vi
prenus, vi
trovos en \^ciu el ili almena\u u 60\% da vortoj, kiuj havas tute tiun saman
formon, kiel en Esperanto. Kaj se vi al tio \^ci ankora\u u aldonos,
ke
anka\u u la restaj 40\% da vortoj diferencas de la Esperanta formo pleje
nur tial, ke la a\u utoroj de tiuj projektoj a\u u ne turnis atenton
sur diversajn principojn, kiuj por lingvo internacia estas treege
gravaj, a\u u simple \^san\^gis la vortojn tute sen ia bezono, ---
vi facile venos al la konkludo, ke la {\sl efektiva} nombro da
vortoj, al kiuj oni anstata\u u la formo Esperanta povus doni formon
pli oportunan, prezentas ne pli ol
iajn 10\%. Sed se en la Esperanta gramatiko oni preska\u u nenion povas
\^san\^gi kaj en la vortaro oni povus \^san\^gi nur iajn 10\% da vortoj,
tiam estas la demando, kion do prezentus per si la {\sl nova}
lingvo, se \^gi iam estus kreita kaj se \^gi efektive montri\^gus
kiel lingvo ta\u uga en \^ciuj rilatoj? Tio \^ci estus ne nova
lingvo, sed nur iom \^san\^gita lingvo Esperanto! Sekve la tuta
demando pri la estonteco de la lingvo internacia alkonduki\^gas nur
al tio, \^cu Esperanto estos akceptita sen\^san\^ge en \^gia {\sl
nuna} formo, a\u u en \^gi estos faritaj iam iaj \^san\^goj! Sed tiu
\^ci demando por la esperantistoj havas jam nenian signifon; ili
protestas nur kontra\u u tio, se {\sl apartaj} personoj volas
\^san\^gi Esperanton la\u u sia bontrovo; sed se iam a\u utoritata
kongreso a\u u akademio decidos fari en la lingvo tiajn a\u u aliajn
\^san\^gojn, la esperantistoj akceptos tion \^ci kun plezuro kaj
nenion perdos de tio \^ci: ili ne bezonos tiam de la komenco ellerni
ian novan malfacilan lingvon, sed ili bezonos nur oferi unu a\u u
kelkajn tagojn por la ellerno de tiuj \^sangoj en la lingvo, kiuj
estos faritaj, kaj la afero estos finita.

   La esperantistoj tute ne pretendas, ke ilia lingvo prezentas ion
{\sl tiom} perfektan, ke nenio pli alta jam povus ekzisti. Kontra\u
ue: kiam efektivi\^gos a\u utoritata kongreso, pri kiu oni scios, ke
\^gia decido havos {\sl forton} por la mondo, la esperantistoj {\sl
mem} proponos al \^gi difini komitaton, kiu okupus sin je la
trarigardo de la lingvo kaj farus en \^gi \^ciujn utilajn
plibonigojn, se e\^c por tio \^ci oni devus \^san\^gi la lingvon
\^gis plena nerekonebleco; sed \^car ekzistas nenia eblo anta\u
uvidi, \^cu tiu \^ci laboro entute sukcesos al la komitato, \^cu
\^gi ne da\u uros senfinan serion da jaroj, \^cu \^gi en konsenteco
estos alkondukita al feli\^ca fino kaj \^cu la finita laboro en la
praktiko montri\^gos tute ta\u uga, sekve kompreneble estus tre
malsa\^ge kaj nepardoneble de la flanko de la komitato, se \^gi pro
la problema {\sl estonta\^{\j}o} rifuzus la faktan kaj en \^ciuj
rilatoj finitan kaj elprovitan {\sl nuna\^{\j}on}; sekve se e\^c la
kongreso venus al la konkludo, ke Esperanto ne estas bona, \^gi
povus decidi nur la jenon: akcepti {\sl dume} la lingvon Esperanto
en \^gia nuna forma kaj {\sl paralele} kun tio \^ci difini
komitaton, kiu okupus sin je la perfektigo de tiu \^ci lingvo a\u u
je la kreo de ia nova lingvo pli ideala; kaj nur tiam, kiam kun la
tempo montri\^gus, ke la laborado de la komitato estas feli\^ce
alkondukita al fino kaj post multaj provoj montri\^gis tute ta\u
uga, nur tiam oni povus anonci, ke la nuna formo de la lingvo
internacia estas eksigata kaj anstata\u u \^gi eniras en la vivon la
formo nova. \^Ciu prudenta homo konsentos, ke la kongreso povas agi
nur tiel kaj ne alie. Sekve se ni e\^c supozos, ke fina lingvo de la
estontaj generacioj estos ne Esperanto, sed ia alia ankora\u u
ellaborota lingvo, en \^cia okazo la vojo al tiu lingvo nepre devas
konduki tra {\sl Esperanto}.

   Sekve resumante \^cion, kion ni diris de la komenco \^gis la nuna
minuto, ni turnas vian atenton sur tion, ke ni venis al la sekvantaj
konkludoj:

1. La enkonduko de lingvo internacia alportus al la homaro
grandegan utilon;

2. La enkonduko de lingvo internacia estas tute ebla;

3. La enkonduko de lingvo internacia pli a\u u malpli frue nepre kaj
sendube efektivi\^gos, kiom ajn la rutinistoj batalus kontra\u u tio
\^ci;

4. Kiel internacia neniam estos elektita ia alia lingvo krom arta;

5. Kiel internacia neniam estos elektita ia alia lingvo krom
Esperanto; \^gi a\u u estos lasita por \^ciam en \^gia nuna formo,
a\u u en \^gi estos poste faritaj iaj \^san\^goj.

\begin{center}
\textbf{VIII}
\end{center}

   Kaj nun ni rigardu, kio sekvas el \^cio, kion ni supre diris. Unue
sekvas tio, ke la esperantistoj tute ne estas tiaj fantaziistoj,
kiaj ili \^sajnas al multaj tiel nomataj "sa\^gaj" kaj
"praktikaj" homoj, kiuj ju\^gas pri \^cio supra\^{\j}e kaj sen ia
logika pripenso kaj mezuras \^cion per mezurilo de la modo. Ili
batalas por afero, kiu ne sole havas grandegan gravecon por la
homaro, sed kiu al tio havas en si nenion fantazian kaj kiu pli a\u
u malpli frue {\sl devas} efektivi\^gi kaj {\sl nepre}
efektivi\^gos, kiom ajn la inerciuloj batalus kontra\u u \^gi, kiom
ajn la sa\^guloj \^sercadus pri \^gi. Kiel estas sendube, ke post la
nokto venas mateno, tiel sendube estas, ke post mallonga a\u u longa
batalado Esperanto pli a\u u malpli frue estos enkondukita en
komunan uzon por la komuniki\^goj internaciaj. Ni konfirmas tion
\^ci kura\^ge ne tial, ke ni tiel volas, ke ni tion esperas, sed
tial, ke la konkludoj de simpla logiko diras, ke tiel esti {\sl
devas} kaj ke {\sl alie esti ne povas}. Longan tempon eble ankora\u
u la esperantistoj devos bataladi, longan tempon eble ankora\u u
\^cia bubo \^{\j}etados sur ilin \^stonojn, koton kaj malsa\^gajn
sprita\^{\j}ojn, sed tio, kio devas veni, pli a\u u malpli frue
venos. La iniciatoroj de la Esperanta afero eble ne \^gisvivos \^gis
tiu tempo, kiam fari\^gos videblaj la fruktoj de ilia agado, ili
iros eble en la tombon kun la malestima nomo de homoj, kiuj okupadis
sin je infana\^{\j}oj, sed pli a\u u malpli frue por la maldol\^ca
kaliko, kiun ili trinkas el la manoj de la samtempuloj, la posteuloj
konstruos al ili monumentojn kaj elparolados ilian nomon kun la plej
granda danko. Longe ankora\u u eble ili \^sajnos al la mondo
senfortaj, multajn fojojn eble ankora\u u ilia afero \^sajnos al la
mondo e\^c mortinta kaj je \^ciam enterigita --- sed tiu \^ci afero
jam neniam mortos, \^car \^gi morti jam {\sl neniam povas}. La afero
vivos kaj konstante rememorigados pri si; post \^ciu nova silenta
tempo, se \^gi e\^c da\u urus dekon da jaroj, aperos nova
revigli\^go; kiam laci\^gos unuj batalantoj, aperos pli a\u u malpli
frue novaj energiaj batalantoj, kaj tiel la afero iros tiel longe,
\^gis fine \^gi plene atingos sian celon. Ne mal\^goju tial,
Esperantistoj, se nesa\^gaj homoj ironie rimarkas al vi, ke vi estas
ankora\u u tre malmultaj, ne perdu la kura\^gon, se via afero iras
malrapide. La afero konsistas ne en rapideco, sed en certeco. Multe
da sencelaj aferoj ekbrilis anta\u u la mondo rapide, sed anka\u u
rapide falis; afero bona kaj certa progresas ordinare malrapide kaj
kun grandaj malhelpoj.

   Sur la supre montritajn 5 konkludojn ni turnas apartan atenton de
tiuj esperantistoj, kiuj batalas por sia ideo {\sl senkonscie} kaj
tial \^ce la plej malgranda rimarko de la kontra\u uuloj ekstaras
senhelpe kaj ne scias, kion respondi, a\u u perdas la kura\^gon.
\^Ciuj tiuj konkludoj prezentas produkton de la simpla kaj severa
{\sl logiko}. Tial se oni diras al vi "la mondo ne volas vian
lingvon", respondu kura\^ge: "\^cu la mondo volas a\u u ne volas,
\^gi pli a\u u malpli frue {\sl devos} akcepti \^gin, \^car alie
esti ne povas". Kiam vi a\u udos "oni diras, ke aperis nova
lingvo, oni diras, ke tia a\u u alia instruita societo a\u u
kongreso deziras elekti tian a\u u alian lingvon a\u u krei novan
lingvon", respondu kura\^ge: "\^ciuj tiuj \^ci famoj a\u u
entreprenoj estas fonditaj sur la plej absoluta nekomprenado de la
esenco kaj historio de la ideo de lingvo internacia; tiaj provoj de
la flanko ne sole de privataj personoj, sed anka\u u de tutaj
societoj ripeti\^gis jam multajn fojojn kaj \^ciufoje fini\^gadis
kaj devis fini\^gadi per la plej plena fiasko; lingvo internacia
povas esti nur Esperanto, \^car la\u u la le\^goj de la logiko kaj
la\u u la esenco de la afero alie esti {\sl neniel povas}".

   Se oni diras al vi: "tiu a\u u alia esperantisto a\u u societo
esperantista en tro granda sed ne prudenta fervoro faris ian falsan
pa\^son kaj ridindigis a\u u diskreditigis per tio \^ci vian tutan
aferon", tiam respondu: "La afero Esperanta dependas de nenia
persono nek societo, kaj nenia homo per siaj privataj falsaj pa\^soj
povas havi ian influon sur \^gian sorton; e\^c la a\u utoro de
Esperanto mem estas nun por Esperanto absolute seninflua, \^car
Esperanto jam longe fari\^gis afero pure {\sl publika}".

   La dua, kio sekvas el \^cio, kion ni supre diris, estas la
sekvanta: Se la elekto de lingvo internacia dependus de ia kongreso
de reprezentantoj de diversaj regnoj, ni longe, tre longe kredeble
devus tion \^ci atendi kaj neniu el ni ion povus fari por tio \^ci.
Sed se, kiel ni vidis supre, oni jam nun povas kun plena certeco kaj
precizeco anta\u uvidi, al {\sl kia} nome lingvo la sorto difinis
fari\^gi iam internacia, tiam la afero \^san\^gi\^gas. Ni ne bezonas
jam nun atendi kongresojn: la celo estas tute klara kaj \^ciu povas
sin tiri al \^gi. Ne bezonante rigardadi, kion diras a\u u faras
aliaj, \^ciu povas alporti sian \^stonon por la kreskanta konstruo.
Nenia \^stono perdi\^gos. Nenia laboranto tie \^ci dependas de la
alia, \^ciu povas agadi aparte, en sia sfero, la\u u siaj fortoj,
kaj ju pli da laborantoj estos; tiom pli rapide estos finita la
granda konstruo. Precipe ni turnas nin al diversaj sciencaj societoj
kaj kongresoj. Ne rigardante, kion faras aliaj, ne atendante, ke
aliaj prenu sur sin la iniciativon, \^ciu societo a\u u kongreso
aparte decidu ion tian, kio alproksimigus la grandan komunehoman
celon almena\u u je unu pa\^so.

\smallrule{}
