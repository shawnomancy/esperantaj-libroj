\begin{center}
\sl A. Alfabeto.
\end{center}

   Aa, Bb, Cc, \^C\^c, Dd, Ee, Ff, Gg, \^G\^g, Hh, \^H\^h, Ii, Jj, \^J\^{\j}, Kk,
   Ll, Mm, Nn, Oo, Pp, Rr, Ss, \^S\^s, Tt, Uu, \u U\u u, Vv, Zz.

   {\it Rimarko}. Presejoj, kiuj ne posedas la literojn \^c, \^g, \^h, \^{\j}, \^s,
\u u, povas anstata\u u ili uzi ch, gh, hh, jh, sh, u.

\begin{center}
\sl B. Reguloj.
\end{center}


1) {\sl Artikolo} nedifinita ne ekzistas; ekzistas nur artikolo
difinita (\emph{la}), egala por \^ciuj seksoj, kazoj kaj nombroj.

   {\it Rimarko}. La uzado de la artikolo estas tia sama, kiel en la aliaj
lingvoj. La personoj, por kiuj la uzado de la artikolo prezentas
malfacila\^{\j}on, povas en la unua tempo tute \^gin ne uzi.

2) La \emph{substantivoj} havas la fini\^gon \emph{o}. Por la formado de la
multenombro oni aldonas la fini\^gon \emph{j}. Kazoj ekzistas nur du:
nominativo kaj akuzativo; la lasta estas ricevata el la nominativo
per la aldono de la fini\^go \emph{n}. La ceteraj kazoj estas
esprimataj per helpo de prepozicioj (la genitivo per \emph{de}, la
dativo per \emph{al}, la ablativo per \emph{per} a\u u aliaj prepozicioj
la\u u la senco).

3) La \emph{adjektivo} fini\^gas per \emph{a}. Kazoj kaj nombroj kiel \^ce la
substantivo. La komparativo estas farata per la vorto \emph{pli}, la
superlativo per \emph{plej}; \^ce la komparativo oni uzas la
prepozicion \emph{ol}.

4) La \emph{numeraloj} fundamentaj (ne estas deklinaciataj) estas: \emph{unu,
du, tri, kvar, kvin, ses, sep, ok, na\u u, dek, cent, mil.} La dekoj
kaj centoj estas formataj per simpla kunigo de la numeraloj. Por la
signado de numeraloj ordaj oni aldonas la fini\^gon de la adjektivo;
por la multoblaj --- la sufikson \emph{obl}, por la nombronaj ---
\emph{on}, por la kolektaj --- \emph{op}, por la disdividaj --- la vorton
\emph{po}. Krom tio povas esti uzataj numeraloj substantivaj kaj
adverbaj. 

5) \emph{Pronomoj} personaj: \emph{mi, vi, li, \^si, \^gi} (pri objekto a\u u besto),
\emph{si, ni, vi, ili, oni}; la pronomoj posedaj estas formataj per la
aldono de la fini\^go adjektiva. La deklinacio estas kiel \^ce la
substantivoj.

6) La \emph{verbo} ne estas \^san\^gata la\u u personoj nek nombroj. Formoj
de la verbo: la tempo estanta akceptas la fini\^gon \emph{-as}; la
tempo estinta \emph{-is}; la tempo estonta \emph{-os}; la modo kondi\^ca
\emph{-us}; la modo ordona \emph{-u}; la modo sendifina \emph{-i}. Participoj
(kun senco adjektiva a\u u adverba): aktiva estanta \emph{-ant}; aktiva
estinta \emph{-int}; aktiva estonta \emph{-ont}; pasiva estanta \emph{-at};
pasiva estinta \emph{-it}; pasiva estonta \emph{-ot}. \^Ciuj formoj de la
pasivo estas formataj per helpo de responda formo de la verbo
\emph{esti} kaj participo pasiva de la bezonata verbo; la prepozicio
\^ce la pasivo estas \emph{de}.

7) La \emph{adverboj} fini\^gas per \emph{e}; gradoj de komparado kiel \^ce la
adjektivoj.

8) \^Ciuj \emph{prepozicioj} postulas la nominativon.

9) \^Ciu vorto estas legata, kiel \^gi estas skribita.

10) La akcento estas \^ciam sur la anta\u ulasta silabo.

11) Vortoj kunmetitaj estas formataj per simpla kunigo de la vortoj
(la \^cefa vorto staras en la fino); la gramatikaj fini\^goj estas
rigardataj anka\u u kiel memstaraj vortoj.

12) \^Ce alia nea vorto la vorto ne estas forlasata.

13) Por montri direkton, la vortoj ricevas la fini\^gon de la
akuzativo.

14) \^Ciu prepozicio havas difinitan kaj konstantan signifon; sed se
ni devas uzi ian prepozicion kaj la rekta senco ne montras al ni,
kian nome prepozicion ni devas preni, tiam ni uzas la prepozicion
\emph{je}, kiu memstaran signifon ne havas. Anstata\u u la prepozicio
\emph{je} oni povas anka\u u uzi la akuzativon sen prepozicio.

15) La tiel nomataj vortoj \emph{fremdaj}, t. e. tiuj, kiujn la plimulto
de la lingvoj prenis el unu fonto, estas uzataj en la lingvo
Esperanto sen \^san\^go, ricevante nur la ortografion de tiu \^ci
lingvo; sed \^ce diversaj vortoj de unu radiko estas pli bone uzi
sen\^san\^ge nur la vorton fundamentan kaj la ceterajn formi el tiu
\^ci lasta la\u u la reguloj de la lingvo Esperanto.

16) La fina vokalo de la substantivo kaj de la artikolo povas esti
forlasata kaj anstata\u uigata de apostrofo.

\smallrule{}
