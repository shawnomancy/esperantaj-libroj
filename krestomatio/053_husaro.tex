\begin{verse}
\begin{center}
\footnotesize (El \fsc{Poŝkin}.)
\end{center}
                        \^Cevalon sian li purigis,\\
                        Estante forte kolerega,\\
                        Kaj diris: "min diablo igis\\
                        En loko lo\^gi malbonega!

                        \vin      Ni \^ciam tie \^ci restadas\\
                          \vin    En turka kvaza\u u batalado:\\
                          \vin    Nur solan supon ni man\^gadas\\
                         \vin     Kaj e\^c ne pensas pri drinkado.

                        Kaj mastro tie \^ci malbona,\\
                        Malbonan havas li edzinon\dots\\
                        Nek via skur\^g', nek vorto bona\\
                        Admonas tie \^ci virinon!

                         \vin     Jen urbo Kiev! Lando kia!\\
                         \vin     En bu\^son falas mem buletoj,\\
                         \vin     Kaj vinon havas domo \^cia,\\
                         \vin     Kaj kiaj tie virinetoj!\dots

                        Ne estas e\^c doma\^g', je Dio,\\
                        Sin mem pro ili pereigi!\\
                        Malbone estas nur per tio\dots"\\
                        Per kio? volu min sciigi! ---

                          \vin    Kaj li komencis plibonigi\\
                          \vin    Lipharon. "Eble ne timulo",\\
                          \vin    Li diris, estas vi, sed igi\\
                           \vin   Vin miri povas mi --- spertulo.

                        A\u uskultu: regimento nia\\
                        \^Ce Dnepro lar\^ga enloki\^gis.\\
                        Mastrino estis bona mia,\\
                        Kaj tombon \^sia edz' fori\^gis.

                         \vin     Kun \^si mi balda\u u amiki\^gis;\\
                         \vin     Ni vivis pace, tute bone.\\
                        \vin      Al mi Mario humili\^gis,\\
                         \vin     Ne diris vorton e\^c malbone.

                        Se mi malsobra estis tute,\\
                        \^Si mem min zorge ku\^sigadis;\\
                        \^Si \^ciam tute sendispute\\
                        Je \^cio tuj mem konsentadis.

                        \vin      Feli\^can vivon mi pasigis,\\
                         \vin     Kaj longe tiel vivus eble, ---\\
                         \vin     Sed jen subite \^{\j}aluzigis\\
                         \vin     Min io, --- mem diabl' kredeble.

                        Mi pensis: kial baptanino\\
                        Sin levas nokte? kion volas?\\
                        Tre juna estas \^si virino,\\
                        Kaj tial eble \^si petolas!\dots

                         \vin     Mi observadi \^sin intencis.\\
                         \vin     Jen foj' ne dormis mi (en korto\\
                         \vin     Jam estis nokto, kaj komencis\\
                         \vin     Bruegi vent' kun granda forto).

                        Kaj vidas mi: \^si forrampetis,\\
                        De forno prenis \^si karbeton,\\
                        Min tre facile pripalpetis,\\
                        Disblovis apud forn' fajreton;

                          \vin    \^Si ekbruligis kandeleton,\\
                          \vin    Kun \^gi angulon \^si fori\^gis,\\
                          \vin    De tie prenis boteleton,\\
                         \vin     Sur balailon \^si sidi\^gis;

                        \^Si senvesti\^gis kaj, sorbinte\\
                        El boteleto, \^si tra tubo,\\
                        Sur balailo sidi\^ginte,\\
                        Tuj malaperis, kvaza\u u nubo.

                         \vin     --- \^Si eble estas sor\^cistino, ---\\
                         \vin     Mi pensis kaj de forn' rapidas,\\
                         \vin     Por vidi sor\^con de virino;\\
                         \vin     Kaj jen mi boteleton vidas;

                        Mi flaris: ia acida\^{\j}o\dots\\
                        Mi plankon \^sprucis el botelo:\\
                        Forflugis --- kia mirinda\^{\j}o! ---\\
                        Kaj forna forko kaj sitelo!

                         \vin     Sub benko katon mi ekvidis,\\
                         \vin     Sur \^gin mi \^sprucis el botelo:\\
                         \vin     \^Gi ternis tuj kaj ekrapidis\\
                         \vin     Subite fornon post sitelo.

                        Sur \^cion \^sprucis sen kompato\\
                        Jam mi per tuta forto mia, ---\\
                        Kaj \^cio: tablo, benko, pato\\
                        Forflugis unu post alia.

                         \vin     Al ili mi ne volis cedi\\
                          \vin    Kaj trinkis mem per unu fojo\\
                         \vin     Resta\^{\j}on\dots \^Cu vi povas kredi?\\
                         \vin     Mi ekaperis tuj en vojo:

                        Mi flugas, --- kien? mem ne scias,\\
                        Kun forto tran\^cas nur aeron,\\
                        Al steloj mi: "pli dekstren" krias\dots\\
                        Kaj jen ekfalis mi sur teron,

                         \vin     Jen mont'. Sur \^gi kaldronoj bolas,\\
                         \vin     Kaj oni faras ian ludon.\\
                         \vin     Kantadas, fajfas kaj petolas,\\
                         \vin     Kun granda ran' edzigas judon.

                        Mi kra\^cis\dots Flugas jen Mario:\\
                        "For! hejmen kuru, petolulo!\\
                        Vin oni tuj forman\^gos!\dots" --- Kio?!\\
                        Nu, mi ne estas timemulo!

                        \vin      Min ne timigu! Kie vojo\\
                        \vin      Al hejmo estas? --- "Jen sidi\^gu\\
                         \vin     Sur fornan feron, --- kun mal\^gojo\\
                         \vin     Respondas \^si, --- kaj tuj fori\^gu!"

                        Vi volas, ke husar' sidi\^gu\\
                        Sur fornan feron?! Baptanino,\\
                        Mi petas vin, ne frenezi\^gu,\\
                        Ne estu ja malsa\^gulino!\dots

                         \vin     \^Cevalon! "Prenu, malsa\^gulo!"\\
                        \vin      \^Si efektive diris veron:"\\
                         \vin     Jen anta\u u mi \^ceval'-bravulo\\
                        \vin      Per hufo forte batas teron.

                        Sidi\^gas mi kun brava vido\\
                        Sur \^gin kaj trovi bridon penas,\\
                        Sed vane! Rajdas mi sen brido ---\\
                        Kaj tuj al forno ni alvenas.

                        \vin      Kaj jen mi tion saman vidas\dots\\
                        \vin      \^Cevalo tute jam forestis,\\
                        \vin      Kaj mi sur benko rajde sidas\dots\\
                         \vin     Jen kia strang'-okazo estis!"

                        Kaj li da\u urigis plibonigi\\
                        Lipharon. "Eble ne timulo,\\
                        Li diris, estas vi, sed igi\\
                        Vin miri povas mi spertulo."

%V. DEVJATNIN.
\end{verse}

\citsc{V. Devjatnin.}

\smallrule{}
