\begin{verse}
                        Jam malsupre suno staras,\\
                        Jam rapide vesperi\^gas\dots\\
                        Kion do knabino faras?\\
                        Kial do \^si ne montri\^gas?

                        \^Cu patrino ne permesas?\\
                        A\u u alia jam \^sin tenas?\\
                        Kaj \^si dol\^ce lin karesas\\
                        Kaj al mi nun ne elvenas?

                        "Ho, knabino! per fiero\\
                        Vi fari\^gis nefidela:\\
                        Kial do vi en vespero\\
                        Ne elvenis, mia bela?

                        --- Ho, karulo! tempo pasis,\\
                        \^Gustatempe mi ne venis,\\
                        \^Car gepatroj min ne lasis,\\
                        \^Car gefratoj min fortenis"\dots

                        "Ne, ne ili vin fortenis!\\
                        Diru simple kaj sincere:\\
                        Malbelulo, mi ne venis,\\
                        \^Car ne amas vin plu vere!"

                        Ha, mi vidas, kion faras\\
                        Karulino malfidela:\\
                        Kun alia \^si nun staras,\\
                        Pli feli\^ca kaj pli bela\dots

%I. LOJKO
\end{verse}
\citsc{I. Lojko.}

\smallrule{}