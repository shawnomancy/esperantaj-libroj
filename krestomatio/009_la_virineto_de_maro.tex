\begin{center}
\footnotesize Fabelo de Andersen.
\end{center}

   Malproksime en la maro la akvo estas tiel blua, kiel la folioj de la
plej bela cejano, kaj klara, kiel la plej pura vitro, sed \^gi estas
tre profunda, pli profunda, ol povas atingi ia ankro; multaj turoj
devus esti starigitaj unu sur la alia, por atingi de la fundo \^gis
super la akvo. Tie lo\^gas la popolo de maro.

   Sed ne pensu, ke tie estas nuda, blanka, sabla fundo; ne, tie kreskas
la plej mirindaj arboj kaj kreska\^{\j}oj, de kiuj la trunketo kaj
folioj estas tiel flekseblaj kaj elastaj, ke ili \^ce la plej
malgranda fluo de la akvo sin movas, kiel vivaj esta\^{\j}oj. \^Ciuj
fi\^soj, malgrandaj kaj grandaj, traglitas inter la bran\^coj, tute
tiel, kiel tie \^ci supre la birdoj en la aero. En la plej profunda
loko staras la palaco de la re\^go de la maro. La muroj estas el
koraloj, kaj la altaj fenestroj el la plej travidebla sukceno; la
tegmento estas farita el konkoj, kiuj sin fermas kaj malfermas la\u
u la fluo de la akvo. Tio \^ci estas belega vido, \^car en \^ciu
konko ku\^sas brilantaj perloj, el kiuj \^ciu sola jam estus
efektiva beliga\^{\j}o en la krono de re\^gino.

   La re\^go de la maro perdis jam de longe sian edzinon, sed lia maljuna
patrino kondukis la mastra\^{\j}on de la domo. \^Si estis sa\^ga
virino, sed tre fiera je sia nobeleco, tial \^si portis dekdu
ostrojn sur la vosto, dum aliaj nobeloj ne devis porti pli ol ses.
Cetere \^si meritis \^cian la\u udon, la plej multe \^car \^si
montris la plej grandan zorgecon kaj amon por la malgrandaj
re\^gidinoj, \^siaj nepinoj. Ili estis ses belegaj infanoj, sed la
plej juna estis tamen la plej bela el \^ciuj; \^sia ha\u uto estis
travidebla kaj delikata, kiel folieto de rozo, \^siaj okuloj bluaj
kiel, la profunda maro; sed, kiel \^ciuj aliaj, \^si ne havis
piedojn, la korpo fini\^gis per fi\^sa vosto.

   La tutan tagon ili povis ludi en la palaco, en la grandaj salonoj,
kie vivaj floroj elkreskadis el la muroj. La grandaj sukcenaj
fenestroj estis malfermataj, kaj tiam la fi\^soj enna\^gadis al ili,
tute kiel \^ce ni enflugas la hirundoj, kiam ni malfermas la
fenestrojn. La fi\^soj alna\^gadis al la malgrandaj re\^gidinoj,
man\^gadis el iliaj manoj kaj lasadis sin karesi.

   Anta\u u la palaco sin trovis granda \^gardeno kun ru\^gaj kaj nigrabluaj
floroj; la fruktoj brilis kiel oro kaj la floroj kiel fajro, dum la
trunketoj kaj folioj sin movis sen\^cese. La tero estis la plej
delikata sablo, sed blua, kiel flamo de sulfuro. Super \^cio estis
ia blua brileto. Oni povus pli diri, ke oni sin trovas alte supre,
en la aero, havante nur \^cielon super kaj sub si, ol ke oni estas
sur la fundo de la maro. En trankvila vetero oni povis vidi la
sunon, kiu elrigardis kiel purpura floro, el kiu venis lumo
\^cirka\u uen.

   \^Ciu el la malgrandaj re\^gidinoj havis en la \^gardeno sian apartan
loketon, kie \^si povis la\u u sia pla\^co kaj bontrovo fosi kaj
planti. Unu donis al sia florejo la formon de baleno; alia trovis
pli bone, ke \^sia florejo similas je virineto de maro; sed la plej
juna donis al la sia formon rondan, kiel la suno, kaj havis nur
florojn ru\^ge brilantajn, kiel tiu. \^Si estis en \^cio originala
infano, silenta kaj pensanta, kaj kiam la aliaj fratinoj sin ornamis
per la mirindaj objektoj, kiujn ili ricevis de \^sipoj, venintaj al
la fundo, \^si volis havi, krom la ru\^gaj floroj, kiuj similis je
la suno, nur unu belan statuon, kiu prezentis tre belan knabon. \^Gi
estis farita el blanka klara marmoro kaj falis sur la fundon de la
maro \^ce la rompi\^go de la \^sipo. \^Si plantis apud la statuo
rozaru\^gan salikon, kiu kreskis belege kaj tenis siajn bran\^cojn
super la statuo, \^gis la blua sablo de la fundo, kie la ombro havis
koloron nigra-bluan kaj movadis sin sen\^cese, kiel la bran\^coj.
\^Gi elrigardis, kiel la supro kaj la radikoj ludus inter si kaj sin
kisus.

   Ne estis pli granda plezuro por la juna re\^gidino, ol a\u udi pri la
mondo supra kaj la homoj. La maljuna avino devis rakontadi \^cion,
kion \^si sciis pri \^sipoj kaj urboj, homoj kaj bestoj. Sed la plej
bela kaj mirinda estis por \^si, ke la floroj supre sur la tero
odoras, kion ne faras la floroj sur la fundo de la maro, kaj ke la
arbaroj estas verdaj kaj la fi\^soj, kiujn oni vidas tie inter la
bran\^coj, tiel la\u ute kaj agrable kantas, ke estas plezuro ilin
a\u udi. \^Gi estis la birdetoj, kiujn la avino nomis fi\^soj, \^car
alie \^siaj nepinoj, kiuj ne vidis ankora\u u birdojn, ne povus
\^sin kompreni.

 --- Kiam vi atingos la a\^gon de dekkvin jaroj, diris la avino, vi
ricevos la permeson sin levi el la maro, sidi sub la lumo de la luno
sur la maraj \^stonegoj kaj rigardi la grandajn \^sipojn, kiuj
veturos anta\u u vi; vi vidos anka\u u arbarojn kaj urbojn! Balda\u
u unu el la fratinoj devis atingi la a\^gon de dekkvin jaroj, sed la
aliaj\dots \^ciu el ili estis unu jaron pli juna, ol la alia, tiel
ke al la plej juna mankis ankora\u u plenaj kvin jaroj, \^gis \^si
povus sin levi de la fundo de la maro kaj vidi, kiel nia mondo
elrigardas. Sed unu promesis al la alia rakonti al \^si, kion \^si
vidis kaj kio la plej multe pla\^cis al \^si en la unua tago; \^car
ilia avino ne sufi\^ce ankora\u u rakontis al ili, estis ankora\u u
multaj aferoj, pri kiuj ili volus scii.

   Sed neniu el ili estis tiel plena je deziroj, kiel la plej juna,
\^guste \^si, kiu devis ankora\u u la plej longe atendi kaj estis
tiel silenta kaj plena je pensoj. Ofte en la nokto \^si staris
anta\u u la malfermita fenestro kaj rigardis supren tra la mallume
blua akvo, kiel la fi\^soj per siaj na\^giloj kaj vostoj \^gin
batis. La lunon kaj stelojn \^si povis vidi; ili \^sajnis tre palaj,
sed por tio ili tra la akvo elrigardis multe pli grandaj, ol anta\u
u niaj okuloj. Se sub ili glitis io kiel nigra nubo, \^si sciis, ke
tio \^ci estas a\u u baleno na\^ganta super \^si, a\u u e\^c \^sipo
kun multaj homoj. Ili certe ne pensis, ke bela virineto de maro
staras sur la fundo kaj eltiras siajn blankajn manojn kontra\u u la
\^sipo.

   Jen la plej maljuna re\^gidino atingis la a\^gon de dekkvin jaroj kaj
ricevis la permeson sin levi super la supra\^{\j}on de la maro. Kiam
\^si revenis, \^si havis multege por rakonti; sed la plej bela
estas, \^si diris, ku\^si en la lumo de la luno sur supermara
sabla\^{\j}o sur trankvila maro kaj rigardi la grandan urbon, kiu
sin trovas tuj sur la bordo de la maro, belan urbon, kie la lumoj
brilas kiel centoj da steloj, a\u udi la muzikon kaj la bruon de
kale\^soj kaj homoj, vidi la multajn pre\^gejojn kaj turojn kaj a\u
uskulti la sonadon de la pre\^gejaj sonoriloj. \^Guste \^car al la
plej juna estis ankora\u u longe malpermesite sin levi, \^si la plej
multe sentis grandan deziregon je \^cio \^ci. Kun kia atento \^si
a\u uskultis tiujn \^ci rakontojn! Kaj kiam \^si poste je la vespero
staris anta\u u la malfermita fenestro kaj rigardis tra la
mallume-blua akvo, \^ciuj \^siaj pensoj estis okupitaj je la granda
urbo kun \^gia bruado, kaj \^sajnis al \^si, ke la sonado de la
pre\^gejaj sonoriloj trapenetras malsupren al \^si.

   Post unu jaro la dua fratino ricevis la permeson sin levi tra la
akvo kaj na\^gi kien \^si volas. \^Si suprenna\^gis \^ce la subiro
de la suno, kaj tiu \^ci vido la\u u \^sia opinio estis la plej
bela. La tuta \^cielo elrigardis kiel oro, \^si diris, kaj la nuboj
--- ha, ilian belecon \^si ne povus sufi\^ce bone priskribi! Ru\^gaj
kaj nigrabluaj ili sin transportis super \^si, sed multe pli rapide
ol ili, flugis, kiel longa blanka kovrilo, amaso da sova\^gaj cignoj
super la akvo al la fori\^ganta suno. Ili flugis al la suno, sed tiu
\^ci mallevi\^gis kaj la roza brilo estingi\^gis sur la maro kaj la
nuboj.

   En la sekvanta jaro la tria fratino sin levis; \^si estis la plej
kura\^ga el \^ciuj kaj na\^gis tial supren en unu lar\^gan riveron,
kiu fluas al la maro. Palacoj kaj vila\^gaj dometoj estis vidataj
tra belegaj arbaroj: \^Si a\u udis kiel \^ciuj birdoj kantis, kaj la
suno lumis tiel varme, ke \^si ofte devis na\^gi sub la akvon, por
iom malvarmigi sian brule varmegan viza\^gon. En unu malgranda golfo
\^si renkontis tutan amason da belaj homaj infanoj; ili kuradis tute
nudaj kaj batadis en la akvo. \^Si volis ludi kun ili, sed kun
teruro ili forkuris. Poste venis malgranda nigra besto, tio estis
hundo, sed \^si anta\u ue neniam vidis hundon; tiu \^ci komencis
tiel furioze \^sin boji, ke \^si tute ektimi\^gis kaj ree forkuris
en la liberan maron. Sed neniam \^si forgesos la belegajn arbarojn,
la verdajn montetojn kaj la malgrandajn infanojn, kiuj povis na\^gi,
kvankam ili ne havis fi\^san voston.

   La kvara fratino ne estis tiel kura\^ga; \^si restis en la mezo de la
sova\^ga maro kaj rakontis, ke \^guste tie estas la plej bele. Oni
povas rigardi tre malproksime \^cirka\u ue, kaj la \^cielo staras
kiel vitra klo\^so super la maro. \^Sipojn \^si vidis, sed nur tre
malproksime, ili elrigardis kiel mevoj; la ludantaj delfenoj sin
renversadis kaj la grandaj balenoj \^sprucis akvon el la truoj de
siaj nazoj supren, ke \^gi elrigardis \^cirka\u ue kiel centoj da
fontanoj.

   Nun venis la tempo por la kvina fratino. \^Sia naskotago estis \^guste
en la mezo de vintro, kaj tial \^si vidis, kion la aliaj la unuan
fojon ne vidis. La maro estis preska\u u tute verda, kaj \^cirka\u
ue na\^gis grandaj montoj de glacio, el kiuj, la\u u \^sia rakonto,
\^ciu elrigardis kiel perlo kaj tamen estis multe pli granda ol la
turoj, kiujn la homoj konstruas. En la plej mirindaj formoj ili sin
montris kaj brilis kiel diamantoj. \^Si sidi\^gis sur unu el la plej
grandaj, kaj \^ciuj \^sipistoj forkuradis kun teruro de la loko, kie
\^si sidis kaj lasis la venton ludi kun \^siaj longaj haroj. Sed je
la vespero la \^cielo kovri\^gis je nuboj, \^gi tondris kaj fulmis,
dum la nigra maro levis la grandajn pecegojn da glacio alte supren
kaj briligis ilin en la forta fulmado. Sur \^ciuj \^sipoj oni
mallevis la velojn; estis timego, teruro, sed \^si sidis trankvile
sur sia na\^ganta glacia monto kaj vidis, kiel la fajraj fulmoj
zigzage sin \^{\j}etadis en la \^sa\u uman maron.

   Kiam iu el la fratinoj venis la unuan fojon super la akvon, \^ciu
estis ravata de la nova kaj bela, kiun \^si vidis; sed nun, kiam ili
kiel granda\^gaj knabinoj havis la permeson supreniri \^ciufoje la\u
u sia volo, \^gi fari\^gis por ili indiferenta; ili deziregis ree
hejmen, kaj post la paso de unu monato ili diris, ke malsupre en la
domo estas la plej bele kaj tie oni sin sentas la plej oportune.

   Ofte en vespera horo la kvin fratinoj \^cirka\u uprenadis unua la alian
per la manoj kaj suprenna\^gadis en unu linio super la akvon. Ili
havis belegajn vo\^cojn, pli belajn ol la vo\^co de homo, kaj kiam
ventego komenci\^gis kaj ili povis supozi, ke rompi\^gos \^sipoj,
ili na\^gadis anta\u u la \^sipoj kaj kantadis agrable pri la beleco
sur la fundo de la maro, kaj petis la \^sipanojn ne timi veni tien.
Sed tiuj \^ci ne povis kompreni la vortojn, ili pensis ke \^gi estas
la ventego; kaj ili anka\u u ne povis vidi la bela\^{\j}ojn de
malsupre, \^car se la \^sipo iris al la fundo, la homoj mortis en la
akvo kaj venis nur jam kiel malvivuloj en la palacon de la re\^go de
la maro.

   Kiam je la vespero la fratinoj tiel mano en mano sin alte levis tra
la maro, tiam la malgranda fratino restis tute sola kaj rigardis
post ili, kaj estis al \^si, kiel \^si devus plori; sed virino de
maro ne havas larmojn, kaj tial \^si suferas multe pli multe.

 --- Ho, se mi jam havus la a\^gon de dekkvin jaroj! \^si diris. Mi
scias, ke \^guste mi forte amos la mondon tie supre kaj la homojn,
kiuj tie lo\^gas kaj konstruas!

   Fine \^si atingis la a\^gon de dekkvin jaroj.

 --- Jen vi elkreskis, diris \^sia avino, la maljuna re\^gino-vidvino.
Venu, kaj mi vin ornamos kiel viajn aliajn fratinojn! \^Si metis al
\^si kronon el blankaj lilioj sur la harojn, sed \^ciu folieto de
floro estis duono de perlo; kaj la maljuna avino lasis almordi\^gi
al la vosto de la re\^gidino ok grandajn ostrojn, por montri \^sian
altan staton.

 --- Tio \^ci doloras! diris la virineto de maro.

 --- Jes, kortega ceremonio postulas maloportunecon! diris la avino.

   Ho, \^si fordonus kun plezuro \^ciun \^ci belega\^{\j}on kaj de\^{\j}etus la
multepezan kronon! \^Siaj ru\^gaj \^gardenaj floroj staris al \^si
multe pli bone, sed nenio helpis. "Adia\u u!" \^si diris kaj levis
sin facile kaj bele tra la akvo.

   Anta\u u momento subiris la suno, kiam \^si levis la kapon super la
supra\^{\j}o de la maro; sed \^ciuj nuboj brilis ankora\u u kiel
rozoj kaj oro, kaj tra la palru\^ga aero lumis la stelo de la
vespero, la aero estis trankvila kaj fre\^sa kaj nenia venteto movis
la maron. Sur la maro staris granda trimasta \^sipo, nur unu velo
estis levita, \^car estis nenia bloveto, kaj inter la \^snuregaro
sidis \^cie \^sipanoj. Oni a\u udis muzikon kaj kantadon, kaj kiam
la vespero fari\^gis pli malluma, centoj da koloraj lanternoj estis
ekbruligitaj; \^gi elrigardis, kvaza\u u la flagoj de \^ciuj nacioj
blovi\^gas en la aero. La virineto de maro alna\^gis \^gis la
fenestro de la kajuto, kaj \^ciufoje, kiam la akvo \^sin levis, \^si
provis rigardi tra la klaraj vitroj de la fenestroj en la kajuton,
kie staris multaj ornamitaj homoj, sed la plej bela el ili estis la
juna re\^gido kun la grandaj nigraj okuloj. Li povis havi la a\^gon
de dekses jaroj, lia tago de naski\^go \^guste nun estis festata,
kaj pro tio \^ci regis la tuta tiu gajeco kaj belegeco. La \^sipanoj
dancis sur la ferdeko, kaj kiam la re\^gido al ili eliris, pli ol
cent raketoj levi\^gis en la aeron. Ili lumis kiel luma tago, tiel
ke la virineto de maro ektimis kaj subi\^gis sub la akvon. Sed
balda\u u \^si denove ellevis la kapon kaj tiam \^sajnis al \^si,
kvaza\u u \^ciuj steloj de la \^cielo defalas al \^si. Neniam
ankora\u u \^si vidis tian fajra\^{\j}on. Grandaj sunoj mu\^gante
sin turnadis, belegaj fajraj fi\^soj sin levadis en la bluan aeron
kaj de \^cio estis vidata luma rebrilo en la klara trankvila maro.
Sur la \^sipo mem estis tiel lume, ke oni povis vidi e\^c la plej
malgrandan \^snuron, ne parolante jam pri la homoj. Kiel bela la
juna re\^gido estis, kaj li premis al la homoj la manojn kun afabla
ridetado, dum la muziko sonis tra la bela nokto.

   Jam fari\^gis malfrue, sed la virineto de maro ne povis deturni
la okulojn de la \^sipo kaj de la bela re\^gido. La koloraj
lanternoj estingi\^gis, la raketoj sin ne levadis pli en la aeron,
la pafilegoj jam pli ne tondris, sed profunde en la maro estis
movado kaj bruado. La virineto de maro sidis sur la akvo kaj
balanci\^gis supren kaj malsupren, tiel ke \^si povis enrigardadi en
la kajuton. Sed pli rapide la \^sipo kuris super la ondoj, unu velo
post la alia estis levataj, pli forte batis la ondoj, nigra nubego
sin montris kaj fulmoj malproksimaj ekbrilis. Terura ventego balda\u
u komenci\^gis. Tial la \^sipanoj demetis la velojn. La granda
\^sipo balanci\^gis en rapidega kurado sur la sova\^ga maro; kiel
grandaj nigraj montoj sin levadis la \^sa\u umanta akvo, minacante
sin trans\^{\j}eti super la mastoj, sed la \^sipo sin mallevadis
kiel cigno inter la altaj ondoj kaj levadis sin ree sur la akvajn
montojn. Al la virineto de maro \^gi \^sajnis kiel veturo de
plezuro, sed la \^sipanoj tute alie tion \^ci rigardis; la \^sipo
\^gemegis kaj krakis, la \^cefa masto rompi\^gis en la mezo kiel
kano, kaj la \^sipo ku\^si\^gis sur flanko kaj la akvo eniris en la
interna\^{\j}on de la \^sipo. Nun la virineto de maro vidis, ke la
\^sipanoj estas en dan\^gero, \^si devis sin mem gardi anta\u u la
traboj kaj rompa\^{\j}oj de la \^sipo, kiuj estis pelataj super la
akvo. Kelkan tempon estis tiel mallumege, ke \^si nenion povis vidi;
sed jen ekfulmis kaj fari\^gis denove tiel lume, ke la virineto de
maro klare revidis \^ciujn sur la \^sipo. \^Ciu penis sin savi, kiel
li povis. La plej multe \^si observadis la junan re\^gidon kaj \^si
vidis, kiam la \^sipo rompi\^gis, ke li falis en la profundan maron.
En la unua momento \^si estis tre \^goja, \^car nun li ja venos al
\^si, sed balda\u u venis al \^si en la kapon, ke la homoj ja ne
povas vivi en la akvo kaj tiel li nur malviva venos en la palacon de
la re\^go de la maro.

   Ne, morti li ne devas! Tial la virineto de maro ekna\^gis inter
la traboj kaj tabuloj, kiuj kuradis sur la maro forgesis la propran
dan\^geron, subna\^gis profunde en la akvon kaj sin levis denove
inter la altaj ondoj. Fine \^si atingis la junan re\^gidon, kiu jam
apena\u u sin tenis sur la supra\^{\j}o de la sova\^ga maro. Liaj
manoj kaj piedoj komencis jam laci\^gi, la belaj okuloj sin fermis,
li devus morti, se \^si ne alvenus. \^Si tenis lian kapon super la
akvo kaj lasis sin peli de la ondoj kien ili volis.

   \^Cirka\u u la mateno la ventego fini\^gis. De la \^sipo ne restis e\^c unu
ligneto. Ru\^ga kaj brilanta sin levis la suno el la akvo, kaj
\^sajnis, ke per tio \^ci la vangoj de la re\^gido ricevis vivon,
tamen liaj okulaj restis fermitaj. La virineto de maro kisis lian
altan belan frunton kaj reordigis liajn malsekajn harojn. \^Sajnis
al \^si, ke li estas simila je la marmora statuo en \^sia malgranda
\^gardeno. \^Si lin kisis kaj rekisis kaj forte deziris, ke li restu
viva.

   Jen montri\^gis tero, altaj bluaj montoj, sur kies suproj brilis
la blanka ne\^go, kiel amaso da cignoj. Malsupre sur la bordo staris
belegaj verdaj arbaroj kaj anta\u u ili staris pre\^gejo a\u u
mona\^hejo, estis ankora\u u malfacile bone \^gin diferencigi, sed
konstruo \^gi estis. Citronaj kaj oran\^gaj arboj kreskis en la
\^gardeno kaj anta\u u la eniro staris altaj palmoj. La maro faris
tie \^ci malgrandan golfeton, en kiu la akvo estis tute trankvila,
sed tre profunda, kaj kiu estis limita de superakvaj \^stonegoj,
kovritaj de delikata blanka sablo. Tien \^ci alna\^gis la virineto
de maro kun la bela princo, ku\^sigis lin sur la sablo kaj zorgis,
ke lia kapo ku\^su alte en la varma lumo de la suno.

   Eksonis la sonoriloj en la granda blanka konstruo kaj multaj junaj
knabinoj sin montris en la \^gardeno. Tiam la virineto de maro
forna\^gis kaj ka\^sis sin post kelkaj altaj \^stonoj, kiuj elstaris
el la akvo, kovris la harojn kaj bruston per \^sa\u umo de la maro,
tiel ke neniu povis vidi \^sian belan viza\^gon, kaj \^si nun
observadis, kiu venos al la malfeli\^ca re\^gido.

   Ne longe da\u uris kaj alvenis juna knabino. \^Si videble forte ektimis,
sed nur unu momenton, kaj balda\u u \^si vokis multajn homojn, kaj
la virineto de maro vidis, ke la re\^gido denove ricevis la konscion
kaj ridetis al \^ciuj \^cirka\u ustarantoj, sed al sia savintino li
ne ridetis, li ja e\^c ne sciis, ke al \^si li devas danki la vivon.
\^Si sentis sin tiel mal\^goja, ke \^si, kiam oni lin forkondukis en
la grandan konstruon, malgaje subna\^gis sub la akvon kaj revenis al
la palaco de sia patro.

   \^Si estis \^ciam silenta kaj enpensa, sed nun \^gi fari\^gis ankora\u u
pli multe. La fratinoj \^sin demandis, kion \^si vidis la unuan
fojon tie supre, sed \^si nenion al ili rakontis.

   Ofte en vespero kaj mateno \^si levadis sin al la loko, kie \^si
forlasis la re\^gidon. \^Si vidis, kiel la fruktoj de la \^gardeno
fari\^gis maturaj kaj estis de\^siritaj, \^si vidis, kiel la ne\^go
fluidi\^gis sur la altaj montoj, sed la re\^gidon \^si ne vidis, kaj
\^ciam pli mal\^goja \^si tial revenadis hejmen. \^Sia sola plezuro
estis sidi en sia \^gardeneto kaj \^cirka\u upreni per la brakoj la
belan marmoran statuon, kiu estis simila je la re\^gido; sed siajn
florojn \^si ne flegis, kaj sova\^ge ili kreskis super la vojetoj,
kaj iliaj longaj trunketoj kaj folioj sin kunplektis kun la
bran\^coj de la arboj tiel, ke tie fari\^gis tute mallume.

   Fine \^si ne povis pli elteni kaj rakontis al unu el siaj fratinoj,
kaj balda\u u tiam scii\^gis \^ciuj aliaj, sed je vorto de honoro
neniu pli ol tiu, kaj kelkaj aliaj virinetoj de maro, kiuj tamen
rakontis \^gin nur al siaj plej proksimaj amikinoj. Unu el ili povis
doni scia\^{\j}on pri la re\^gido, \^si anka\u u vidis la feston
naskotagan sur la \^sipo, sciis, de kie la re\^gido estas kaj kie
sin trovas lia regno.

   "Venu, fratineto!" diris la aliaj re\^gidinoj, kaj interplektinte la
brakojn reciproke post la \^sultroj, ili sin levis en longa linio el
la maro tien, kie sin trovis la palaco de la re\^gido.

   Tiu \^ci palaco estis konstruita el helflava brilanta speco de \^stono,
kun grandaj marmoraj \^stuparoj, el kiuj unu kondukis rekte en la
maron. Belegaj orkovritaj kupoloj sin levadis super la tegmentoj,
kaj inter la kolonoj, kiuj \^cirka\u uis la tutan konstruon, staris
marmoraj figuroj, kiuj elrigardis kiel vivaj ekzista\^{\j}oj. Tra la
klara vitro en la altaj fenestroj oni povis rigardi en la plej
belegajn salonojn, kie pendis multekostaj silkaj kurtenoj kaj
tapi\^soj kaj \^ciuj muroj estis ornamitaj per grandaj
pentra\^{\j}oj, ke estis efektiva plezuro \^cion tion \^ci vidi.

   En la mezo de la granda salono batis alta fontano, \^giaj radioj sin
levadis \^gis la vitra kupolo de la plafono, tra kiu la suno lumis
sur la akvo kaj la plej belaj kreska\^{\j}oj, kiuj sin trovis en la
granda baseno.

   Nun \^si sciis, kie li lo\^gas, kaj tie \^si montradis sin ofte en
vespero kaj en nokto sur la akvo; \^si alna\^gadis multe pli
proksime al la tero, ol kiel kura\^gus ia alia, \^si e\^c sin
levadis tra la tuta mallar\^ga kanalo, \^gis sub la belega marmora
balkono. Tie \^si sidadis kaj rigardis la junan re\^gidon, kiu
pensis, ke li sidas sola en la klara lumo de la luno.

   Ofte en vespero \^si vidadis lin forveturantan sub la sonoj de muziko
en lia belega \^sipeto ornamita per flagoj; \^si rigardadis tra la
verdaj kanoj, kaj se la vento kaptadis \^sian longan blankan vualon
kaj iu \^sin vidis, li pensis, ke \^gi estas cigno, kiu disvastigas
la flugilojn.

   Ofte en la nokto, kiam la fi\^sistoj kaptadis fi\^sojn sur la maro \^ce
lumo de tor\^coj, \^si a\u udadis, ke ili rakontas multon da bona
pri la juna re\^gido, kaj \^si \^gojadis, ke \^si savis al li la
vivon, kiam li preska\u u sen vivo estis pelata de la ondoj, kaj
\^si ekmemoradis, kiel firme lia kapo ripozis sur \^sia brusto kaj
kiel kore \^si lin kisis. Pri tio \^ci li nenion sciis, li ne povis
e\^c son\^gi pri \^si.

   \^Ciam pli kaj pli kreskis \^sia amo al la homoj, \^ciam pli \^si deziris
povi sin levi al ili kaj vivi inter ili, kaj ilia mondo \^sajnis al
\^si multe pli granda ol \^sia. Ili ja povas flugi sur \^sipoj trans
la maron, sin levi sur la altajn montojn alte super la nubojn, kaj
la landoj, kiujn ili posedas, sin etendas kun siaj arbaroj kaj
kampoj pli malproksime, ol \^si povas atingi per sia rigardo. Multon
\^si volus scii, sed la fratinoj ne povis doni al \^si respondon je
\^cio, kaj tial \^si demandadis pri tio la maljunan avinon, kiu bone
konis la pli altan mondon, kiel \^si nomadis la landojn super la
maro.

 --- Se la homoj ne dronas, demandis la virineto de maro, \^cu ili tiam
povas vivi eterne, \^cu ili ne mortas, kiel ni tie \^ci sur la fundo
de la maro?

 --- Ho jes, diris la maljunulino, ili anka\u u devas morti, kaj la tempo
de ilia vivo estas ankora\u u pli mallonga ol de nia. Ni povas
atingi la a\^gon de tricent jaroj, sed kiam nia vivo \^cesi\^gas, ni
turni\^gas en \^sa\u umon kaj ni ne havas tie \^ci e\^c tombon inter
niaj karaj. Ni ne havas senmortan animon, ni jam neniam reveki\^gas
al vivo, ni similas je la verda kreska\^{\j}o, kiu, unu fojon
detran\^cita, jam neniam pli povas revivi\^gi. Sed la homoj havas
animon, kiu vivas eterne, kiam la korpo jam refari\^gis tero. La
animo sin levas tra la etero supren, al la brilantaj steloj! Tute
tiel, kiel ni nin levas el la maro kaj rigardas la landojn de la
homoj, tiel anka\u u ili sin levas al nekonataj belegaj lokoj, kiujn
ni neniam povos vidi.

 --- Kial ni ne ricevis senmortan animon? mal\^goje demandis la virineto
de maro. Kun plezuro mi fordonus \^ciujn centojn da jaroj, kiujn mi
povas vivi, por nur unu tagon esti homo kaj poste preni parton en la
\^ciela mondo!

 --- Pri tio \^ci ne pensu! diris la maljunulino, nia sorto estas multe
pli feli\^ca kaj pli bona, ol la sorto de la homoj tie supre!

 --- Sekve mi mortos kaj disflui\^gos kiel \^sa\u umo sur la maro, mi jam
ne a\u udados la muzikon de la ondoj, ne vidados pli la belajn
florojn kaj la luman sunon! \^Cu mi nenion povas fari, por ricevi
eternan animon?

 --- Ne! diris la maljunulino, nur se homo vin tiel ekamus ke vi estus
por li pli ol patro kaj patrino, se li alligi\^gus al vi per \^ciuj
siaj pensoj kaj sia amo kaj petus, ke la pastro metu lian dekstran
manon en la vian kun la sankta promeso de fideleco tie \^ci kaj
\^cie kaj eterne, tiam lia animo transfluus en vian korpon kaj vi
anka\u u ricevus parton en la feli\^co de la homoj. Li donus al vi
animon kaj retenus tamen anka\u u sian propran. Sed tio \^ci neniam
povas fari\^gi! \^Guste tion, kio tie \^ci en la maro estas nomata
bela, vian voston de fi\^so, ili tie supre sur la tero trovas
malbela. \^Car mankas al ili la \^gusta komprenado, tie oni devas
havi du malgraciajn kolonojn, kiujn ili nomas piedoj, por esti
nomata bela!

   La virineto de maro ek\^gemis kaj rigardis mal\^goje sian fi\^san
voston.

 --- Ni estu gajaj! diris la maljunulino, ni saltadu kaj dancadu en
la tricent jaroj, kiujn la sorto al ni donas; \^gi estas certe
sufi\^ca tempo, poste oni tiom pli senzorge povas ripozi. Hodia\u u
vespere ni havos balon de kortego!

   Kaj \^gi estis efektive belego, kiun oni sur la tero neniam vidas. La
muroj kaj la plafono de la granda salono estis de dika, sed
travidebla vitro. Multaj centoj da grandegaj konkoj, ru\^gaj kiel
rozoj kaj verdaj kiel herbo, staris en vicoj sur \^ciu flanko kun
blue brulanta flamo, kiu lumigis la tutan salonon kaj tralumis e\^c
tra la muroj, tiel ke la maro \^cirka\u ue estis tute plena je lumo.
Oni povis bone vidi \^ciujn la sennombrajn fi\^sojn, grandajn kaj
malgrandajn, kiel ili alna\^gadis al la vitraj muroj; sur unuj la
skvamoj \^sajnis purpuraj, sur aliaj ili brilis kiel oro kaj
ar\^gento.

   En la mezo tra la salono fluis lar\^ga kvieta rivero, kaj sur \^gi
dancis la viroj kaj virinetoj de maro la\u u sia propra dol\^ca
kantado. Tiel belajn vo\^cojn la homoj sur la tero ne havas. La plej
juna re\^gidino kantis la plej bele, oni apla\u udadis al \^si, kaj
por unu momento \^si sentis \^gojon en la koro, \^car \^si sciis, ke
\^si havas la plej belan vo\^con en la maro kaj sur la tero. Sed
balda\u u \^si denove komencis pensadi pri la mondo supra. \^Si ne
povis forgesi la belan re\^gidon kaj sian doloron, ke \^si ne havas
senmortan animon kiel li. Tial \^si el\^steli\^gis el la palaco de
sia patro, kaj dum interne \^cio kantadis kaj \^gojadis, \^si sidis
malgaja en sia \^gardeneto. Tiam subite \^si eka\u udis la sonon de
korno penetrantan al \^si tra la akvo, kaj \^si ekpensis: "nun li
kredeble \^sipas tie supre, li, kiun mi amas pli ol patron kaj
patrinon, li, al kiu mi pendas per \^ciuj miaj pensoj kaj en kies
manon mi tiel volonte metus la feli\^con de mia vivo. \^Cion mi
riskos, por akiri lin kaj senmortan animon! Dum miaj fratinoj dancas
en la palaco de mia patro, mi iros al la mara sor\^cistino. \^Gis
hodia\u u mi \^ciam sentis teruron anta\u u \^si, sed eble \^si
povas doni al mi konsilon kaj helpon".

   La re\^gidino eliris el sia \^gardeneto kaj rapidis al la bruanta
akvoturnejo, post kiu lo\^gis la sor\^cistino. Sur tiu \^ci vojo
\^si anta\u ue neniam iris; tie kreskis nenia floro, nenia herbeto,
nur nuda griza sabla tero sin etendis \^gis la akvoturnejo, kie la
akvo \^sa\u ume batadis, turni\^gante kiel bruantaj radoj muelaj,
kaj \^cion, kion \^gi kaptadis, tiregadis kun si en la profundegon.
Inter tiuj \^ci \^cion ruinigantaj akvaj turna\^{\j}oj la re\^gidino
devis trapa\^si, por atingi la tera\^{\j}on de la sor\^cistino, kaj
tie \^si devis ankora\u u longe iri sur varma \^sanceli\^ganta
\^slimo, kiun la sor\^cistino nomadis sia torfa\^{\j}o. Post tiu
staris \^sia domo en la mezo de stranga arbaro. \^Ciuj arboj kaj
arbeta\^{\j}oj estis polipoj, duone besto duone kreska\^{\j}o, ili
elrigardis kiel centkapaj serpentoj, kiuj kreskis el la tero; \^ciuj
bran\^coj estis longaj \^slimaj brakoj kun fingroj kiel fleksaj
vermoj, kaj \^ciu membro sin movadis, de la radiko \^gis la plej
alta supro. \^Cion, kion ili povis en la maro kapti, ili \^cirka\u
uvolvadis fortike, por jam neniam \^gin ellasi. La malgranda
re\^gidino kun teruro haltis anta\u u tiu \^ci arbaro; \^sia koro
batis de timego; \^si jam estis preta iri returne, sed \^si ekpensis
pri la re\^gido kaj pri la akirota homa animo kaj fari\^gis denove
kura\^ga. Siajn longajn, libere defalantajn harojn \^si alligis
fortike \^cirka\u u la kapo, por ke la polipoj ne povu per tio \^ci
\^sin ekkapti, la amba\u u manojn \^si almetis kruce al la brusto
kaj rapide iris anta\u uen, kiel fi\^so rapidanta tra la akvo, inter
la teruraj polipoj, kiuj etendadis post \^si siajn fleksajn brakojn
kaj fingrojn. \^Si rimarkis, kiel \^ciu el ili objekton unu fojon
kaptitan tenis forte per centoj da malgrandaj brakoj, kiel per feraj
ligiloj. Homoj pereintaj en la maro kaj falintaj al la fundo
elrigardadis kiel blankaj ostaroj el la brakoj de la polipoj. En la
brakoj ili tenis partojn de \^sipoj kaj kestojn, ostarojn de bestoj
surteraj kaj, kio \^sajnis al la re\^gidino la plej terura
--- virineton de maro, kiun ili estis kaptintaj kaj sufokintaj.

   Jen \^si venis al granda, preska\u u \^cie kovrita de \^slimo, placo en la
arbaro, kie grandaj grasaj maraj serpentoj sin volvadis kaj
montradis sian abomenan blankflavan ventron. En la mezo de la placo
staris domo, konstruita el la blankaj ostoj do homoj pereintaj en la
maro; tie sidis la sor\^cistino de la maro kaj man\^gigadis testudon
el sia bu\^so, kiel la homoj donas sukeron al kanarieto. La
malbelajn grasajn serpentojn de la maro \^si nomadis siaj amataj
kokidoj kaj lasis ilin ludi sur sia granda sponga brusto.

 --- Mi jam scias, kion vi volas! diris la sor\^cistino, tre malsa\^ge!
Tamen via volo estos plenumita, \^car \^gi \^{\j}etos vin en
malfeli\^con, mia bela re\^gidineto. Vi volus liberi\^gi de via
fi\^sa vosto kaj por tio havi du trabojn por la irado, kiel la
homoj, por ke la juna re\^gido enami\^gu en vin kaj vi ricevu lin
kaj senmortan animon! \^Ce tio \^ci la sor\^cistino ridis tiel la\u
ute kaj malbele, ke la testudo kaj la serpentoj falis sur la teron.
Vi ne povis veni en pli oportuna tempo. Morga\u u post la levi\^go
de la suno mi jam ne povus al vi helpi, \^gis ree pasus jaro. Mi
kuiros al vi trinka\^{\j}on, kun kiu vi ankora\u u anta\u u la
levi\^go de la suno devas na\^gi al la tero, sidi\^gi sur la bordo
kaj eltrinki la trinka\^{\j}on; tiam via fi\^sa vosto turni\^gos en
tion, kion la homoj nomas graciaj piedoj, sed \^gi doloros, \^gi
estos al vi, kiel se akra glavo vin tran\^cus. \^Ciuj, kiuj vin
vidos, diros, ke vi estas la plej bela homa esta\^{\j}o, kiun ili
vidis. Restos al vi via facila flugetanta irado, nenia dancistino
povos flirtadi tiel gracie kiel vi, sed \^ce \^ciu pa\^so, kiun vi
faros, vi sentos, kiel vi pa\^sus sur akran tran\^cilon. Se vi volas
\^cion \^ci elteni, tiam mi al vi helpos!

 --- Jes! diris la virineto de maro kun tremanta vo\^co kaj pensis pri
la re\^gido kaj pri la ricevota animo.

 --- Sed memoru, diris la sor\^cistino, se vi unu fojon fari\^gis homo,
vi jam neniam pli povas refari\^gi virineto de maro! vi jam neniam
pli povos sin mallevi tra la akvo al viaj fratinoj kaj al la palaco
de via patro, kaj se vi ne akiros la amon de la princo, tiel ke li
pro vi forgesu patron kaj patrinon, alligi\^gu al vi per \^ciuj siaj
pensoj kaj ordonu al la pastro, ke tiu metu viajn manojn en la liajn
tiel, ke vi fari\^gu edzo kaj edzino, tiam vi ne ricevos senmortan
animon! Je la unua mateno post lia edzi\^go kun alia via koro
dis\^siri\^gos kaj fari\^gos \^sa\u umo sur la maro.

 --- Mi \^gin volas! diris la virineto de maro kaj \^si estis pala kiel
la morto.

 --- Sed vi devas anka\u u pagi al mi! diris la sor\^cistino, kaj ne
malmulton mi postulas. Vi havas la plej belan vo\^con el \^ciuj tie
\^ci sur la fundo de la maro, per \^gi vi esperas tie ensor\^ci la
re\^gidon, sed la vo\^con vi devas doni al mi. La plej bonan, kion
vi posedas, mi volas ricevi por mia kara trinka\^{\j}o! Mian propran
sangon mi ja devas doni por \^gi, por ke la trinka\^{\j}o fari\^gu
akra, kiel glavo amba\u uflanke tran\^canta!

 --- Sed se vi prenos de mi mian vo\^con, diris la virineto de maro, kio
do tiam restos al mi?

 --- Via bela ekstera\^{\j}o, diris la sor\^cistino, via flugetanta irado
kaj viaj parolantaj okuloj, per kiuj vi jam povas malsa\^gigi homan
koron. Nu, \^cu vi perdis la kura\^gon? Eletendu vian malgrandan
langon, kaj mi detran\^cos \^gin por mia peno kaj vi ricevos la
karan trinka\^{\j}on!

 --- \^Gi fari\^gu! diris la virineto de maro, kaj la sor\^cistino metis
la kaldronon, por kuiri la sor\^can trinka\^{\j}on. Pureco estas
duono de vivo! \^si diris kaj elfrotis la kaldronon per la
serpentoj, kiujn \^si kunligis en unu tuberon. Poste \^si gratis al
si mem la bruston kaj engutigis en la kaldronon sian nigran sangon.
La vaporo formis la plej strangajn flgurojn, tiel ke estis terure
ilin vidi. \^Ciun momenton la sor\^cistino metis novajn objektojn en
la kaldronon, kaj kiam \^gi komencis bone boli, estis tute, kiel se
plorus krokodilo. Fine la trinka\^{\j}o estis preta; \^si elrigardis
kiel la plej klara akvo.

 --- Jen ricevu \^gin! diris la sor\^cistino kaj detran\^cis al la
virineto de maro la langon; la virineto nun jam estis muta, povis
jam nek kanti nek paroli.

 --- Se la polipoj volos vin kapti, kiam vi iros returne tra mia
arbaro, diris la sor\^cistino, \^sprucu sur ilin nur unu solan guton
de tiu \^ci trinka\^{\j}o, tiam iliaj brakoj kaj fingroj disfalos en
mil pecojn! Sed tion \^ci la re\^gidino ne bezonis, la polipoj
timigite sin retiris de \^si, kiam ili vidis la brilantan
trinka\^{\j}on, kiu lumis en \^sia mano kiel radianta stelo. Tiel
\^si balda\u u pasis la arbaron, la mar\^con kaj la akvoturnejon.

   \^Si povis vidi la palacon de sia patro; la tor\^coj en la granda
salono de dancado estis estingitaj; tie certe jam \^ciuj dormis,
tamen \^si ne kura\^gis iri al ili nun, kiam \^si estis muta kaj la
koro volis dis\^siri\^gi al \^si de mal\^gojo; \^si en\^steli\^gis
en la \^gardenon, derompis floron de la bedoj de \^ciuj siaj
fratinoj, \^{\j}etis al la palaco milojn da kisoj per la fingroj kaj
levi\^gis tra la mallume-blua maro supren.

   La suno ankora\u u ne estis levi\^ginta, kiam \^si ekvidis la palacon de
la re\^gido kaj supreniris sur la belega marmora \^stuparo. La luno
lumis mirinde klare. La malgranda virineto de maro eltrinkis la
brule akran trinka\^{\j}on, kaj estis al \^si, kiel se amba\u uakra
glavo trairus tra \^sia delikata korpo, \^si perdis la konscion kaj
falis kiel mortinta. Kiam la suno eklumis super la maro, \^si
veki\^gis kaj sentis fortan doloron, sed rekte anta\u u \^si staris
la aminda juna re\^gido, kiu direktis sur \^sin siajn okulojn
nigrajn kiel karbo, tiel ke \^si devis mallevi la siajn, kaj tiam
\^si rimarkis, ke \^sia fi\^sa vosto perdi\^gis kaj \^si havis la
plej graciajn malgrandajn blankajn piedetojn, kiujn bela knabino nur
povas havi. Sed \^si estis tute nuda, kaj tial \^si envolvis sin en
siajn densajn longajn harojn. La re\^gido demandis, kiu \^si estas
kaj kiel \^si venis tien \^ci, kaj \^si ekrigardis lin per siaj
mallumebluaj okuloj kviete sed anka\u u tiel malgaje, \^car paroli
\^si ja ne povis. Tiam li prenis \^sian manon kaj kondukis \^sin en
la palacon. \^Ce \^ciu pa\^so, kiun \^si faris, estis al \^si, kiel
la sor\^cistino anta\u udiris, kiel se \^si pa\^sus sur akrajn
pinglojn kaj tran\^cilojn, sed \^si elportadis \^gin volonte. Sub la
mano de la re\^gido \^si supreniris facile kiel akva veziko sur la
\^stuparo, kaj li kaj anka\u u \^ciuj aliaj admiris \^sian gracian
flugetantan iradon.

   Multekostaj vestoj el silko kaj muslino estis nun donitaj al \^si, en
la palaco \^si estis la plej bela el \^ciuj, sed \^si estis muta,
povis nek kanti nek paroli. Belaj sklavinoj, vestitaj en silko kaj
oro, eliradis kaj kantadis anta\u u la re\^gido kaj liaj gepatroj.
Unu kantis pli agrable ol \^ciuj aliaj, kaj la re\^gido apla\u udis
kaj ridetis al \^si; tiam la virineto de maro fari\^gis mal\^goja,
\^si sciis, ke \^si mem kantus multe pli bele. Ho! \^si diris al si
mem, se li nur scius, ke, por esti \^ce li, mi fordonis je eterne
mian vo\^con!

   Jen la sklavinoj ekdancis belegajn flugetajn dancojn sub la plej bela
muziko; tiam la virineto de maro levis siajn belajn blankajn
brakojn, stari\^gis sur la pintoj de la fingroj piedaj kaj
ekflugetis sur la planko, dancante kiel ankora\u u neniu dancis.
\^Ce \^ciu movo \^sia beleco pli multe elmontri\^gadis, kaj \^siaj
okuloj parolis pli interne kaj pli profunde al la koro, ol la
kantado de la sklavinoj.

   \^Ciuj estis ensor\^citaj de \^gi, aparte la re\^gido, kiu nomis la
re\^gidinon lia amata trovitino, kaj \^si dancis senhalte, kvankam
\^ciufoje, kiam \^sia piedo tu\^sis la plankon, \^sajnis al \^si,
kiel \^si ekpa\^sus sur akran tran\^cilon. La re\^gido diris, ke
\^si estu \^ciam apud li, kaj \^si e\^c ricevis la permeson dormi
sur velura kuseno anta\u u lia pordo.

   Li lasis fari al \^si viran vesta\^{\j}on, por ke \^si povu akompanadi
lin anka\u u sur \^cevalo. Ili rajdadis tra la bonodoraj arbaroj,
kie la verdaj bran\^coj tu\^sadis iliajn \^sultrojn kaj la birdetoj
kantadis inter la fre\^saj folioj. \^Si levadis sin kun la re\^gido
sur la altajn montojn, kaj kvankam \^siaj graciaj piedetoj sangadis,
\^si tamen ridadis je tio \^ci kaj sekvadis lin \^gis ili ekvidadis
la nubojn preterflugantaj profunde sub ili, kiel se \^gi estus amaso
da birdoj, iranta al la fremdaj landoj.

   En domo, en la palaco de la re\^gido, \^si iradis, kiam en la nokto la
aliaj dormis, sur la lar\^gan marmoran \^stuparon kaj malvarmigadis
siajn brulantajn piedojn en la malvarma akvo de la maro kaj tiam
\^si pensadis pri siaj parencoj en la profunda\^{\j}o.

   En unu nokto \^siaj fratinoj venis, reciproke tenante sin per la
brakoj; ili kantis, na\^gante super la akvo, tre malgajajn
melodiojn. \^Si faris al ili signon, kaj ili \^sin rekonis kaj
rakontis, kiel profunde \^si ilin \^ciujn mal\^gojigis. De tiu tempo
ili vizitadis \^sin \^ciun nokton, kaj en unu nokto \^si rimarkis en
granda malproksimeco la maljunan avinon, kiu jam de tre longa tempo
ne estis sin levinta al la supra\^{\j}o de la maro, kaj la re\^gon
de la maro kun sia krono sur la kapo. Ili etendis al \^si la manojn,
sed amba\u u ne kura\^gis veni tiel proksime al la tero, kiel la
fratinoj.

   De tago al tago la re\^gido \^sin \^ciam pli ekamadis, li amis \^sin,
kiel oni povas nur ami bonan belan infanon, sed fari \^sin re\^gino
tio \^ci neniam venis al li en la kapon, kaj tamen \^si ja devis
fari\^gi lia edzino, \^car alie \^si ne ricevus senmortan animon,
sed devus en la mateno de lia edzi\^go fari\^gi \^sa\u umo sur la
maro.

   \^Cu vi ne amas min pli ol \^ciun? kvaza\u u paroladis la okuloj de la
re\^gidineto, kiam li \^sin prenadis en siajn brakojn kaj kisadis
\^sian belan frunton.

 --- Jes, vin mi amas la plej multe! diris la re\^gido, \^car vi havas
la plej bonan koron, vi estas al mi la plej multe aldonita kaj vi
estas simila je unu juna knabino, kiun mi unu fojon vidis, sed jam
kredeble neniam revidos. Mi estis sur \^sipo, kiu rompi\^gis; la
ondoj alpelis min al tero proksime de sankta pre\^gejo, en kiu
multaj junaj knabinoj faris la meson. La plej juna el ili trovis min
tie sur la bordo kaj savis al mi la vivon; nur du fojojn mi \^sin
vidis. \^Si estus la sola knabino, kiun mi povus ami en tiu \^ci
mondo, sed vi estas simila je \^si, vi preska\u u forpremas \^sian
figuron en mia animo. \^Si apartenas al la sankta pre\^gejo, kaj
tial vin sendis al mi mia bona feli\^co; neniam ni disi\^gos unu de
la alia!

 --- Ha! li ne scias, ke mi savis al li la vivon! pensis la
re\^gidineto, mi portis lin trans la maro al la arbaro, kie staras
la pre\^gejo; mi rigardis lin post la \^sa\u umo kaj atendis, \^cu
ne venos al li ia homo. Mi vidis la belan knabinon, kiun li amas pli
ol min! La virineto de maro profunde \^gemis, plori \^si ne povis.
La knabino apartenas al la sankta pre\^gejo, li diris, neniam \^si
elvenos en la mondon, ili neniam renkontos unu la alian, mi estas
apud li, mi vidas lin \^ciutage, mi lin flegos, mi lin amos, mi
oferos al li mian vivon!

   Sed jen la re\^gido devis edzi\^gi kaj ricevi kiel edzino la filinon
de la najbara re\^go; tial oni pretigis tian belan \^sipon. La
re\^gido veturas, oni diras, por ekkoni la landojn de la najbara
re\^go, sed efektive \^gi estas farata por rigardi la filinon de la
najbara re\^go; grandan sekvantaron li devis kunpreni. La virineto
de maro balancis la kapeton kaj ridetis; \^si konis la pensojn de la
re\^gido pli bone ol \^ciuj aliaj. "Mi devas veturi! li diris al
\^si, mi devas rigardi la belan re\^gidinon; miaj gepatroj \^gin
postulas, sed por preni \^gin kiel fian\^cino, tion ili ne volas min
devigi. Mi ne povas \^sin ami! \^Si ne estas simila je la bela
knabino en la pre\^gejo, je kiu vi estas simila. Se mi iam devus
elekti fian\^cinon, tiam la elekto pli volonte falus sur vin, mia
muta trovitineto kun la parolantaj okuloj!" \^Ce tio li \^sin kisis
sur \^sia ru\^ga bu\^so, ludis kun \^siaj longaj haroj kaj metis
sian kapon al \^sia koro, kaj pli vive \^si ekrevis pri homa
feli\^co kaj pri senmorta animo.

 --- Vi ja ne timas la maron, vi, muta infano, li demandis, kiam ili
staris sur la belega \^sipo, kiu devis lin veturigi al la lando de
la najbara re\^go, kaj li rakontis al \^si pri ventegoj kaj
senventeco, pri mirindaj fi\^soj en la profunda\^{\j}o kaj kion la
subakvi\^gistoj tie vidis, kaj \^si ridis \^ce lia rakontado, \^car
\^si sciis ja pli bone ol \^ciu alia pri la fundo de la maro.

   En luna nokto, kiam \^ciuj dormis, krom la direktilisto apud sia
direktilo, tiam \^si sidis sur la rando de la \^sipo kaj senmove
rigardis malsupren tra la klara akvo, kaj \^sajnis al \^si, kvaza\u
u \^si vidas la palacon de sia patro; sur la plej alta turo de tiu
palaco staris la maljuna avino kun la ar\^genta krono sur la kapo
kaj rigardadis sendeturni\^ge tra la ondanta akvo al la kilo de la
\^sipo. Jen \^siaj fratinoj elna\^gis el la interna\^{\j}o de la
maro, rigardadis \^sin malgaje kaj kunerompadis la blankajn manojn.
\^Si faris al ili signojn, ridetis kaj volis rakonti, ke \^cio iras
bone kaj feli\^ce, sed la servanto de la \^sipo alproksimi\^gis al
\^si kaj la fratinoj subakvi\^gis, tiel ke li pensis, ke la blanka,
kion li vidis, estis nur \^sa\u umo sur la maro.

   Je la sekvanta mateno la \^sipo enna\^gis en la havenon de la belega
\^cefurbo de la najbara re\^go. \^Ciuj sonoriloj de pre\^gejoj
sonoris, kaj de la altaj turoj eksonis trumpetoj, dum la soldatoj
kun etenditaj standardoj kaj brilantaj bajonetoj staris orditaj en
parado. \^Ciu tago estis pasigata feste. Baloj kaj kolekti\^goj
sekvis unu la alian, sed la re\^gidino ankora\u u ne estis, \^si
estis edukata, kiel oni diris, malproksime en unu sankta mona\^hejo,
kie \^si lernadis \^ciujn re\^gajn virtojn. Fine \^si alveturis.

   La virineto de maro brulis de deziro vidi \^sian belecon kaj devis
konfesi, ke personon pli belan kaj \^carman \^si neniam ankora\u u
vidis. \^Sia ha\u uto estis delikata kaj travidebla kaj post la
longaj mallumaj okulharoj ridetis paro da nigrabluaj fidelaj okuloj.

 --- Vi \^gi estas! \^goje ekkriis la re\^gido, vi, kiu min savis, kiam
mi ku\^sis malviva sur la bordo de la maro! kaj li premis \^sin kiel
sian ru\^gi\^gantan fian\^cinon en siaj brakoj. Ho, mi estas tro
feli\^ca, li diris al la virineto de maro. La plej bona, kiun mi
neniam esperis atingi, estas al mi plenumita. Vi \^gojos je mia
feli\^co, \^car vi min amas la plej multe! La virineto de maro kisis
al li la manon, kaj jam nun estis al \^si, kiel \^si sentus, ke
\^sia koro dis\^siri\^gas. La mateno de la edzi\^go de la princo
devis ja alporti al \^gi la morton kaj turni \^sin en \^sa\u umon
sur la maro.

   \^Ciuj sonoriloj de la pre\^gejoj sonoris, la heroldoj rajdadis tra
la tuta urbo kaj sciigadis la fian\^ci\^gon. Sur \^ciuj altaroj
brulis bonodora oleo en multekostaj ar\^gentaj lampoj. La pastroj
balancadis la fumilojn kaj la gefian\^coj donis unu al la alia la
manon kaj ricevis la benon de la episkopo. La virineto de maro
staris en silko kaj oro kaj tenis la trena\^{\j}on de la vesto de la
fian\^cino, sed \^sia orelo ne a\u udis la festan muzikon, \^sia
okulo ne vidis la sanktan ceremonion, \^si pensis pri sia nokto de
la morto, pensis pri \^cio, kion \^gi perdis en tiu \^ci mondo.

   Ankora\u u en tiu sama vespero la gefian\^coj iris sur la \^sipon; la
pafilegoj tondris, \^ciuj flagoj sin movadis en la aero, kaj en la
mezo de la \^sipo estis konstruita re\^ga tendo el oro kaj purpuro
kaj provizita je la plej belaj kusenoj, sur kiuj devis ripozi la
gefian\^coj en la trankvila malvarmeta nokto.

   La vento blovis la velojn kaj la \^sipo glitis facile kaj sen forta
\^sanceli\^gado sur la klara maro. Kiam fari\^gis vespero, oni
ekbruligis kolorajn lampojn kaj la maristoj dancis gajajn dancojn
sur la ferdeko. La virineto de maro tiam ekmemoris tiun vesperon,
kiam \^si je la unua fojo suprenna\^gis el la maro kaj vidis tian
saman belegecon kaj \^gojon, kaj nun \^si sin turnadis en la danco,
flugetante kiel flugetas la hirundo, kiam \^gi estas persekutata,
kaj \^ciuj admirante apla\u udadis al \^si, neniam ankora\u u \^si
dancis tiel belege. Kvaza\u u akraj tran\^ciloj tran\^cadis en
\^siaj delikataj piedoj, sed \^si \^gin ne sentadis, \^car pli multe
\^gi tran\^cadis al \^si la koron. \^Si sciis, ke \^gi estas la
lasta vespero, ke \^si lin vidas, lin, por kiu \^si forlasis la
amikojn kaj hejmon, fordonis sian belegan vo\^con kaj \^ciutage
suferadis malfacilajn dolorojn, dum li \^gin e\^c la plej malmulte
ne scietis. \^Gi estis la lasta nokto, ke \^si enspiradis tiun saman
aeron, kiel li, kaj vidadis la profundan maron kaj la bluan stelan
\^cielon. Eterna nokto sen pensado kaj son\^gado \^sin atendis,
\^car \^si ne havis animon kaj jam neniam povis \^gin ricevi. \^Cio
sur la \^sipa estis plena je \^gojo kaj gajeco, multe ankora\u u
post la mezo de la nokto; \^si ridetadis kaj dancadis kun pensoj de
morto en sia koro. La re\^gido kisis sian belan fian\^cinon kaj \^si
ludis kun liaj nigraj haroj kaj brako en brako ili iris en la
belegan tendon por ripozi.

   Fari\^gis trankvile kaj malla\u ute sur la \^sipo, nur la direktilisto
staris apud la direktilo; la virineto de maro metis siajn blankajn
brakojn sur la randon de la \^sipo kaj rigardadis al oriento,
atendante la \^cielru\^gon de mateno. \^Si sciis, ke la unua radio
de la suno estos \^sia alportanto de morto. Tiam \^si ekvidis siajn
fratinojn levi\^gantajn el la maro, ili estis palaj kiel \^si mem,
iliaj longaj belaj haroj jam ne movadis sin en la vento, ili estis
detran\^citaj.

 --- Ni donis \^gin al la sor\^cistino, por ke \^si donu helpon kaj por ke
vi ne devu morti en tiu \^ci nokto! \^Si donis al ni tran\^cilon,
jen \^gi estas! Vi vidas, kiel akra \^gi e\^stas? Anta\u u ol la
suno levi\^gos, vi devas \^gin enpu\^si en la koron de la re\^gido,
kaj kiam lia varma sango \^sprucos sur viajn piedojn, tiam viaj
piedoj kunkreski\^gos en unu fi\^san voston kaj vi denove fari\^gos
virineto de maro, vi povos veni al ni en la maron kaj vivi viajn
tricent jarojn, anta\u u ol vi fari\^gos malviva sala \^sa\u umo de
la maro. Rapidu, rapidu! Li a\u u vi devas morti anta\u u la
levi\^go de la suno. Nia maljuna avino mal\^gojas tiel, ke la
blankaj haroj al \^si elfalis, kiel niaj falis sub la tondilo de la
sor\^cistino. Mortigu la re\^gidon kaj revenu! Rapidu! \^cu vi vidas
la ru\^gan strion sur la \^cielo? Post kelkaj minutoj levi\^gos la
suno kaj vi devos morti! Kaj ili strange kaj profunde ek\^gemis kaj
mallevi\^gis en la akvon.

   La re\^gidineto de maro fortiris la purpuran kurtenon anta\u u la tendo
kaj vidis, kiel la bela fian\^cino dormas, tenante sian kapon sur la
brusto de la re\^gido, kaj \^si klini\^gis, kisis lin sur la bela
frunto, suprenrigardis al la \^cielo, kie la ru\^go de la mateno
ricevadis \^ciam pli brulantan koloron, ekrigardis la akran
tran\^cilon kaj direktis ree siajn okulojn sur la re\^gidon, kiu en
la son\^go elparoladis la nomon de sia fian\^cino, --- \^si sola
vivis en liaj pensoj --- kaj la tran\^cilo tremis en la manoj de la
virineto de maro\dots sed jen \^si \^{\j}etis \^gin malproksime for
en la ondojn, kiuj ru\^ge eklumis, kie falis la tran\^cilo;
elrigardadis, kiel se gutoj de sango \^sprucus supren el la akvo.

   Ankora\u u unu fojon \^si ekrigardis la re\^gidon per rigardo duone
estingita, \^{\j}etis sin de la \^sipo en la maron kaj sentis, kiel
\^sia korpo sin turnis en \^sa\u umon.

   Nun la suno levi\^gis el la maro, kviete kaj varme falis \^giaj radioj
sur la malvarman malvivan \^sa\u umon de la maro kaj la re\^gidineto
de maro sentis nenion de la morto. \^Si ekvidis la sunon kaj
proksime super \^si flugetis centoj da travideblaj belegaj
esta\^{\j}oj; \^si povus tra ili vidi la blankajn velojn de la
\^sipo kaj la ru\^gajn nubojn sur la \^cielo. Ilia vo\^co estis kiel
sonorado de sferoj, sed tiel spiritia, ke nenia homa orelo povis
\^gin a\u udi, kiel anka\u u nenia tera okulo povis vidi tiujn \^ci
\^cielajn esta\^{\j}ojn. Sen flugiloj ili sin portadis tra la aero
danke sian propran facilecon. La virineto de maro vidis, ke \^si
havas korpon tian kiel ili, kiu \^ciam pli kaj pli sin levadis el la
\^sa\u umo.

 --- Al kiu mi venis? \^si demandis, kaj \^sia vo\^co sonis kiel la vo\^co
de la aliaj esta\^{\j}oj, tiel spirite, ke nenia tera muziko povas
\^gin reprezenti.

 --- Al la filinoj de la aero! respondis la aliaj. La virino de maro
ne havas senmortan animon kaj neniam povas \^gin ricevi, se ne
prosperis al \^si akiri la amon de homo. De fremda potenco dependas
\^sia eterna ekzistado. La filinoj de la aero anka\u u ne havas
senmortan animon, sed ili povas \^gin akiri al si mem per bonaj
faroj. Ni flugas al la varmaj landoj, kie la brulanta spiro de la
pesta aero mortigas la homojn; tie ni blovetados malvarmeton. Ni
disetendas la bonodoron de la floroj tra la aero kaj alportas
refre\^si\^gon kaj sani\^gon. Se ni en la da\u uro de tricent jaroj
penadis fari \^cion bonan, kion ni povas, tiam ni ricevas senmortan
animon kaj prenas parton en la eterna feli\^co de la homoj. Vi,
malfeli\^ca virineto de maro, el la tuta koro \^ciam laboradis al
tiu sama celo kiel ni, vi suferis kaj paciencis, vi levi\^gis nun al
la mondo de la spiritoj de la aero, nun vi povas mem akiri al vi
post tri centjaroj senmortan animon per bonaj faroj!

   La virineto de maro levis siajn travideblajn brakojn al la luma suno
kaj je la unua fojo \^si sentis larmojn. Sur la \^sipo denove regis
bruo kaj vivo; \^si rimarkis, kiel la re\^gido kun sia bela
fian\^cino \^sin ser\^cas, malgaje ili direktis siajn okulojn sur la
ondantan \^sa\u umon, kiel se ili scius, ke \^si \^{\j}etis sin en
la maron. Nevidate \^si kisis la frunton de la fian\^cino, ridetis
al la re\^gido kaj levis sin kun la aliaj infanoj de la aero al la
rozeru\^ga nubo, kiu na\^gis en la aero.

 --- Post tricent jaroj ni tiel transflugetos en la regnon \^cielan!

 --- E\^c pli frue ni povas tien veni! diris unu el la spiritoj de
la aero. Nevidate ni enflugetas en la domojn de la homoj, kie estas
infanoj, kaj por \^ciu tago, en kiu ni trovas bonan infanon, kiu
faras \^gojon al siaj gepatroj kaj meritas ilian amon, Dio
mallongigas al ni nian tempon de provado. La infano ne scias, kiam
ni flugas tra la \^cambro, kaj kiam ni el \^gojo pro tiu infano
ridetas, tiam niaj tricent jaroj perdas unu jaron, sed se ni vidas
malbonkondutan kaj malbonan infanon, tiam ni devas plori larmojn de
mal\^gojo, kaj \^ciu larmo aldonas al nia tempo de provado unu
tagon!

\smallrule{}
