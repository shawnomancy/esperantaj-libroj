\begin{center}
\footnotesize (Rakonto de V. \fsc{Devjatnin}.)
\end{center}

\begin{center}
\textbf{I}
\end{center}

   Sinjorino Anneto, kvankam sufi\^ce matura en sia a\^go (\^si havis
preska\u u 50 jarojn), tre bone ankora\u u sin konservis, almena\u u
la\u u \^sia ekstera vido oni ne povis \^sin nomi maljunulino: la
koloro de \^sia viza\^go estis \^ciam tiel roza, la dentoj tiel
blankaj kaj ebenaj, ke iafoje e\^c nevole naski\^gadis tre ofenda
por \^si suspekto. Sinjorino Anneto, la\u u la volo de la sorto,
\^gis nun ankora\u u estas fra\u ulino, sed\dots \^si estas jam
vidvino!\dots \^Si mem rakontis al mi (mi \^gis nun estas \^sia
amiko), ke \^si perdis sian junan belan edzon tuj post la
edzini\^go; apena\u u la junaj geedzoj revenis el la pre\^gejo,
\^sia edzo falis kaj mortis! Al li fari\^gis kredeble dis\^siro de
koro, tro multe plenigita de amo. Dank' al tiu \^ci okazo la
malfeli\^ca sinjorino restis vidvino, travivinte kun la edzo nur
proksimume duonon da horo. De tiu tempo pasis jam tridek jaroj, la
amo de sinjorino Anneto havis jam tempon tute malvarmi\^gi, sed
edzini\^go entute tiom \^sin timigis, ke \^si decidis senrefare jam
eterno resti fra\u ulino kaj e\^c \^{\j}uris sankte plenumi tiun
\^ci decidon. Sed la petolema sorto subite devigis \^sin ankora\u u
unu fojon facile tu\^si la katenojn de Himeneo. Jen kio okazis.

\begin{center}
\textbf{II}
\end{center}

   Unu fojon sinjorino Anneto veturis sur fervojo, revenante el la
malgranda urbeto K. post enterigo de sia kuzo. Okupinte en la vagono
tre oportunan kanapeton, \^si volis ekdormi, dezirante almena\u u
per tio \^ci sin liberigi de \^sin premanta \^sar\^ga impreso de
morto, --- sed subite, rigardinte okaze la plankon, \^si vidis ian
libron, forgesitan kredeble de elirinta veturanto. Sinjorino Anneto
levis \^gin kaj malfermis: tio \^ci estis franca tradukita romano
sub la titolo: "Terura sekreto". \^Si estis tre \^goja je tiu \^ci
neatendita trova\^{\j}o kaj komencis avide legi pa\^gon post pa\^go.
La romano estis tre interesa: en \^gi estis teruraj mortigoj kaj
sekretaj malaperoj kaj malfeli\^ca amo, per unu vorto, en \^gi estis
\^cio, kio devas esti en \^gusta romano, --- kaj entute en la romano
estis tia miksa\^{\j}o, ke sinjorino Anneto nur kun granda
malfacileco povis kompreni kvankam malmulte tiun \^ci mirindan
produkta\^{\j}on de la genia a\u utoro. Tiu \^ci malfacila laborado
de la kapo lacigis la malfeli\^can legantinon, kaj \^si
ekdormis\dots

   Jen \^si son\^gas, kvaza\u u \^si estas ankora\u u tre juna knabino. \^Si
senfine amas lin kaj li, bela, flama junulo, varmege, pasie \^sin
amas. Sed la kruela patro elektis por \^si malbelegan, maljunan,
senharan baronon, kiu scias ian teruran familian sekreton de \^sia
patro kaj per tio \^ci tenas lin en siaj manoj. La terura tempo
proksimi\^gas: jam estas difinita la tago de edzini\^go kun la
barono\dots La patro \^sin en\^slosis en malgrandan \^cambreton kaj
alstarigis al \^si maljunan diablistinon servantinon, kiu atente
\^sin gardas kaj ne permesas al \^si e\^c pa\^son fari libere,
--- kaj \^si, malfeli\^ca ofero de la patra krueleco kaj de la
malbonega barona volupto, ne povas e\^c sciigi lin, sian amatan, pri
sia terura situacio!\dots En malespero \^si \^siras siajn harojn,
sin \^{\j}etadas en la \^cambro, kaj fine, tute frenezi\^ginte, \^si
saltas kun kapo malsupren el la fenestro sur la pavimon kaj\dots ho
mirinda\^{\j}o!\dots \^si enfalis en la brakojn de sia amanto! Ili
flugas, flugas, flugas kun nekredebla rapideco en ia sor\^ca
kale\^so al la rando de la mondo, al plej feli\^ca Arkadujo\dots Jen
jam estas proksime, al \^si jam blovetis feli\^co\dots Sed subite
post ili fari\^gis forta bruo. Ili rigardas kaj\dots ho, teruro:
persekuto!\dots Sur nigra flugila \^cevalo rapide flugas la
barono!\dots Minace brilas liaj verdaj okuloj, liaj haroj estas
disblovitaj, liaj longaj malgrasaj manoj kun akraj kurbaj ungegoj
estas etenditaj anta\u uen\dots Lia forta \^cevalo flugas kun
rapideco de sago, distran\^cante kun fajfo la aeron per sia lar\^ga
fortika brusto\dots Ankora\u u unu momento, kaj la geamantoj nepre
devas perei!\dots La juna fra\u ulino pli proksime sin alpremis al
sia karulo, \^cirka\u uprenis lin, li forte, dol\^ce \^sin kisis
kaj\dots \^si veki\^gis.

   En la preska\u u tute malplena vagono estis mistera duonlumo: kelkaj
lanternoj estis estingitaj, kaj la ankora\u u brulantaj estis
kovritaj per mallumaj verdaj kurtenetoj, tra kiuj ili apena\u u
\^cirka\u ulumis du a\u u tri dormantajn veturantojn.

   Sinjorino Anneto eksentis, ke iu delikate premas \^sian manon, kaj nur
nun \^si ekvidis starantan anta\u u \^si sur genuoj ian nekonatan
viron, kiun \^si, mem ne komprenante per kia maniero, forte
\^cirka\u uprenis per la maldekstra libera mano\dots \^Si eksaltis,
kvaza\u u pikita de serpento, kaj kun teruro rigardis \^cirka\u uen:
en la vagono estis tute silente.

 --- Kion vi volas, sinjoro?

   La nekonato malrapide sin levis kaj kun petego etendis al \^si la
manojn.

 --- Pardonu! li murmuretis: pardonu, se mi vin nevole ofendis\dots
Sed vi tiel afable respondadis miajn karesojn, ke mi pensis\dots Ho,
pardonu, pardonu min!\dots

   Kaj li ree stari\^gis sur la genuoj, kaptis \^sian manon kaj alpremis
\^gin al siaj lipoj. Al sinjorino Anneto \^sajnis, ke tio \^ci estas
ankora\u u son\^go\dots Fine \^si malla\u ute liberigis de li sian
manon.

 --- Levu vin, pro Dio! kion vi faras? \^si diris al la nekonato: ni ja
povas fari\^gi objekto de atento por \^ciuj veturantoj!

   Li senvorte obeis. Sinjorino Anneto iom post iom komencis
trankvili\^gadi, kaj vidante, ke la stranga sinjoro tute ne estas
por \^si dan\^gera, sed, kontra\u ue, li mem timas ion, \^si tute
trankvili\^gis kaj e\^c ekinteresi\^gis je tiu \^ci neatendita
aventuro kaj anka\u u je la mistera nekonato, kiu silente sidis
anta\u u \^si sur la kanapeto. Sinjorino Anneto sin levis,
for\^sovis la kurteneton de la plej proksima lanterno kaj, dank' al
\^gia lumo, ekvidis anta\u u si belan kaj eksterordinare simpatian
viron en la a\^go de \^cirka\u u 35 jaroj. Liaj densaj nigraj ondaj
haroj kun intenca nezorgeco estis for\^{\j}etitaj posten kaj bone
ombris la agrablan palecon de la bela maldika viza\^go kun malgranda
barbeto; la vivaj esprimaj okuloj de ia nedifinita koloro lumis de
sento, kiun tuj oni ne povas kompreni, --- sed tiu \^ci sento havis
ian forlogantan forton\dots Sinjorino Anneto nun rememoris, ke \^si
sufi\^ce ofte renkontadis tiun \^ci belulon sur la strato en K., ---
kaj sekve li estas al \^si iom konata. \^Si decidis ekparoli kun li.

 --- Pardonu, \^si komencis, mian ne tute modestan demandon, sed mi tre
volas scii, kiu vi estas. Al mi \^sajnas, ke mi vin renkontadis en
K., kaj tial\dots tial via viza\^go estas al mi iom konata. De kie,
kien kaj pro kio vi veturas?

   La nekonato ek\^gemis.

 --- Tro malgaja kaj, la plej grava, tro longa estas la historio,
respondis li nevolonte: post duono da horo ni estos jam en K., kaj
mi ne havos sufi\^ce da tempo, por kontentigi vian scivolecon,
--- sekve oni ne devas e\^c komenci. Mi diros nur sole, ke mi estas
la plej malfeli\^ca homo en la tuta mondo, mi estas de \^ciuj
pelata, de \^ciuj malamata kaj de neniu amata\dots \^Ciujn viajn
demandojn: de kie mi veturas, kien kaj pro kio, mi nur povas
respondi: mi ne scias, ne scias, ne scias!\dots Kaj la nekonato,
ka\^sinte la viza\^gon per la manoj, restis en profunda silento. La
koro de sinjorino Anneto tropleni\^gis per kompato al tiu \^ci
malfeli\^ca homo: \^si jam estis preta ekplori.

 --- Rakontu al mi vian mal\^gojon, \^si diris kore: per tio \^ci vi eble
faciligos vian animon.

   La nekonato levis la kapon, kaj liaj okuloj ekbrilis.

 --- Ho, kiel mi estas feli\^ca! li ekkriis kun profunda sento: Mi
a\u udas kompatajn vortojn de virino, kiun mi tiel longe kaj tiel
pasie amas! kiun mi\dots

 --- Kion vi diras? interrompis lin sinjorino Anneto: rekonscii\^gu!
rigardu min: \^cu oni povas enami\^gi en tielan maljunulinon?

 --- Mi ne diris, ke mi enami\^gis, rediris la nekonato, denove prenante
\^sian manon: ne, mi jam longe \^cesis enami\^gadi, sed mi ne perdis
ankora\u u la kapablon de amo\dots Jes, sinjorino Anneto\dots

 --- De kie vi scias mian nomon? mirege ekkriis sinjorino Anneto: kiu
vi estas?

 --- Vi min ne konas kaj kredeble neniam a\u udis mian nomon. Mi estas
Arturo de\dots

 --- Via nomo estas Arturo?! Dio! kia stranga okazo!

   Sinjorino Anneto rememoris, ke la \^cefa heroo de la romano, kiun \^si
trovis anta\u u nelonge tie \^ci en la vagono, anka\u u estas
"Arturo". \^Si kredis je diversaj anta\u usignoj, kaj tiu \^ci
egaleco de la nomoj \^sin forte ekfrapis: \^si vidis en \^gi ian
anta\u usignon de la \^cielo mem.

 --- Jes, da\u urigis la nekonato, mi estas Arturo de Buler. Mi vin
ekkonis preska\u u anta\u u du jaroj (per kia maniero --- mi ne
diros), kaj de tiu sama tempo mi amas vin per \^ciuj fortoj de mia
animo! De tiu tempo mi \^ciam ser\^cis renkonti vin, kaj al tiu \^ci
celo mi tre ofte veturadis en la urbon K. Dank' al multaj ka\u uzoj,
pri kiuj mi ne kura\^gas paroli, mi ne volis konati\^gi kun vi\dots
Sed fine mi decidis fari tiun \^ci dan\^geran por mi pa\^son
kaj\dots nun mi jam tute ne beda\u uras tion \^ci, \^car mi estas
multe rekompencita!

   Arturo kun fajro kisis la manon de sinjorino Anneto, kiu kun tremanta
koro a\u uskultis liajn flamajn parolojn.

 --- Mi ne volas lacigi, li da\u urigis, vian atenton per rakontado pri
mia tuta malgaja vivo, --- kaj tio \^ci e\^c ne estas necesa: nia
okaza konateco fini\^gos, kiam ni alveturos en K., kaj disiros en
diversajn flankojn, por eble neniam jam nin renkonti\dots Post
kelkaj minutoj estos jam K\dots Tial ne estas \^gustatempe e\^c
komenci!\dots

 --- Ha, ne, pro Dio! vive rediris sinjorino Anneto: mi tiel multe
partoprenas en via sorto, tiel korege vin kompatas\dots Mi petas
vin, sinjoro, diru al mi vian mal\^gojon!\dots

   En tiu sama minuto la vagonaro brue venis al la K-a stacidomo. Arturo
etendis al \^si la manon. --- Adia\u u! li diris kun tremo en la
vo\^co: ankora\u u unu fojon mi petas vin: pardonu kaj ne rememoru
min malbone!

 --- \^Cu efektive nia konateco nur per tio \^ci fini\^gos? kun sincera
mal\^gojo demandis sinjorino Anneto, retenante kaj facile premante
lian manon.

 --- Tio \^ci dependas de vi, respondis Arturo.

 --- Sed kion do mi devas fari? Mi ne lo\^gas aparte en mia propra domo,
kaj sekve mi ne povas inviti vin rekte al mi\dots


 --- Tio \^ci tute ne estas necesa, tiom pli ke mi kun la unua vagonaro
veturas reen en B., kie mi lo\^gas. Sed se vi deziras fari min
senfine feli\^ca, donu al mi vorton respondadi miajn leterojn kaj
iam min viziti en mia malri\^ca hejmo: B. estas ja tiel
proksima!\dots

   Sinjorino Anneto kun plezuro plenumis lian modestan deziron, --- \^si
donis al li sian adreson, prenis de li lian, --- kaj ili disi\^gis
amike.

\begin{center}
\textbf{III}
\end{center}

   Pasis semajno. Sinjorino Anneto preska\u u forgesis sian vojan
aventuron, kaj e\^c la viza\^go mem de \^sia voja kolego komencis
iel mallumi\^gi en \^sia memoro, --- kaj jen subite \^si ricevis
leteron. \^Sian koron kvaza\u u io pikis. Per manoj tremantaj de
interna eksciti\^go \^si dis\^siris la koverton kaj rigardis la
subskribon; jes, \^sia anta\u usento ne trompis \^sin: la letero
estis de li! \^Si avidege komencis legi. Jen kion skribis Arturo:
"Kara mia Anneto! Pardonu al mi tiun \^ci nevolan senceremoniecon,
sed la amo ne scias paroli alie. Vi tiel forte deziris ekscii mian
historion. Mi plenumas vian deziron, kiun mi estimas, kiel plej
sanktan le\^gon. Sed mi korege petas pardonon, mia an\^gelo, se tiu
\^ci malgaja historio vin mal\^gojigos. Mi devas komenci de
malproksime. Mi naski\^gis en la \^cirka\u ua\^{\j}o de Moskvo en
genta patra bieno. Mia patro estis tre ri\^ca homo, natura
aristokrato, bonege edukita, sed beda\u urinde senkaraktera. Li
estis vidvo, kaj mi --- lia sola filo kaj heredanto de lia tuta
grandega ri\^ceco. Li ekiris voja\^gadi kaj prenis kun si anka\u u
min, tiam ankora\u u dekkvinjaran knabon. Kiam ni estis en Parizo,
li enami\^gis en unu belan kantistinon, edzi\^gis je \^si kaj
a\^cetis por \^si ri\^can bienon en la \^cirka\u ua\^{\j}oj de
Parizo, kie ni anka\u u enloki\^gis. Mia vivo en tiu tempo estis
neelportebla: mia duonpatrino jam de la unuaj tagoj de nia kuna
lo\^gado komencis min malami kaj balda\u u ekscitis kontra\u u min
mian patron, kiu \^sin senfine amis kaj plenumadis \^ciujn \^siajn
kapricojn. Post duono da jaro mia patro povis jam rimarki, ke lia
edzino kondutis tiel same, kiel \^si kondutis, estante ankora\u u
kantistino, en li do \^si vidis ne edzon, sed simple homon, kiu
devas \^sin provizadi: \^si sen\^cese postuladis de li monon, kiun
\^si elspezadis eksterordinare rapide, dank' al baloj, koncertoj,
maskobaloj k. t. p. \^Si preska\u u \^ciutage mendadis por si \^ciam
novajn ornamojn kaj \^ciusemajne devigadis mian patron pagi
grandegajn kalkulojn de tajlorinoj, meblistoj, kale\^sistoj,
ora\^{\j}istoj. La patro kun teruro rimarkis, kiel rapide malaperis
lia ri\^ceco, sed li sin tenis, li silentis. Fine li tute perdis la
paciencon kaj decidis serioze interparoli kun la edzino, --- sed
\^si tiel bone lin renkontis, ke li por \^ciam forlasis la deziron
interparoli kun \^si kaj, eksilentinte tute, humili\^gis al sia
sorto. Per tia maniero lia tuta ri\^ceco post unu kaj duono da jaro
malaperis tute, kaj mia malfeli\^ca patro flnis la vivon per
memmortigo, --- kaj lia edzino fari\^gis kromvirino de iu bankiero.
--- Post la morto de la patro mi, dank' al helpo de bonaj homoj,
forveturis Rusujon, kie estis ankora\u u unu nia genta bieno, en kiu
mi enloki\^gis kaj komencis min okupi per vila\^ga mastra\^{\j}o.
Tiel pasis du jaroj, en la da\u uro de kiuj mi, tute ne sciante
mastrumi, tiel malbonigis \^ciujn miajn aferojn, ke fine mia bieno
pro \^suldoj estis vendita per publika vendo, mi fari\^gis preska\u
u almozulo, kaj nur kun helpo de unu mia parenco ricevis servon sur
fervojo. Nun mi venis al la plej terura momento de mia vivo!\dots
Kiam mi havis la a\^gon de dudek du jaroj, mi edzi\^gis je unu bela
juna fra\u ulino\dots Ho, kiel doloras mia malfeli\^ca koro \^ce tiu
\^ci \^sar\^ga rememoro! Dio vidas, kiel multe mi amis mian edzinon!
Sed kredeble ne estis al mi sorto longe \^gui tiun \^ci
feli\^con\dots Post du monatoj de nia edzi\^go (per \^ciuj fortoj de
mia animo mi malbenas tiun \^ci teruran momenton!) mi tute okaze
trovis mian edzinon en la brakoj de malnobla amanto!\dots Ho, mia
kara! En tiu \^ci loko la plumo elfalas el miaj manoj, mi ne havas
forton da\u urigi kaj mi proponas, ke vi mem alplenigu tiun \^ci
teruran pentra\^{\j}on\dots Mi devas nur diri, ke mi forkuris de tiu
\^ci malnobla virino, kvaza\u u de pesto, kaj de tiu tempo mi jam
\^sin ne renkontadis, sed proksimume post tri jaroj mi tute okaze
eksciis, ke \^si estis mortigita ie en Germanujo per la mano de
\^sia amanto, kiun \^si trompis tiel same, kiel min!\dots Jen estas
la malgaja rakonto de mia disrompita vivo. --- Pardonu, mia kara!
Donu al vi la potenca Dio multe da feli\^co. Varmege vin amanta via
Arturo."

   Traleginte tiun \^ci leteron, sinjorino Anneto eksentis ion tian,
kion \^si mem komence ne povis kompreni. Tiu \^ci sento ne estis
ordinara partopreno en la sorto de proksimulo, \^gi ne estis anka\u
u sento de kompato al malfeli\^ca homo, --- ne, \^gi estis io multe
pli forta: \^gi estis amo, jes, amo!\dots Sinjorino Anneto kontra\u
u sia propra volo en la profunda\^{\j}o de la animo tion \^ci
konfesis. \^Si decidis tuj respondi la leteron de Arturo, respondi
en tia sama senco, kaj sendi al li pro signo de sia amo antikvan
diamantan ringon, kiun \^si ricevis, kiel heredon, ankora\u u de sia
avino. Jen \^si en tiu sama tago vespere sin en\^slosis en sia
\^cambro kaj tute plenskribis du grandajn foliojn da po\^sta papero.
Kiel Arturo, nur pli detale, \^si en la komenco priskribis sian
tutan vivon per tiel fortaj vortoj, ke e\^c \^si mem ekploris,
tralegante la produkton de sia libera fantazio, --- kaj poste jam
\^si alpa\^sis al vasta klarigo de sia amo, kiun \^si esprimis
senfine pli bone kaj elokvente, ol Arturo. La duan tagon la sigelita
letero kaj la atente \^cirka\u ukudrita multekosta senda\^{\j}o
estis forportitaj en la po\^stan oficejon, --- kaj post tri a\u u
kvar tagoj matene \^si ricevis jam de Arturo respondon: la tuta
letero de la komenco \^gis la fine estis plenigita de signoj de
ekkrio, de punktaroj, de signoj de demando kaj ekkrio kune kaj
simple de demandoj, --- sed ordinaraj punktoj kaj komoj preska\u u
tute forestis. Li petegis sinjorinon Anneton alveturi al li en B-n,
por "per sia apero \^cirka\u ulumigi kaj sanktigi la malluman
neston de malfeli\^ca ermito, kaj doni al li la eblon vidi sian
diinon"\dots Traleginte tiun \^ci leteron, \^si decidis la
sekvantan tagon kun la matena vagonaro veturi en B-n, kaj per
telegramo \^si sciigis pri tio Arturon, por ke li povu \^sin
renkonti.

\begin{center}
\textbf{IV}
\end{center}

   Suferante de senpacienco, \^si pasigis la tagon kaj la sekvantan
nokton, kaj matene frue \^si alveturis al la stacidomo, kiam
ankora\u u iris nenia vagonaro: timante malfrui, \^si atendis du
horojn la tempon de forveturo, --- kaj, enirinte fine en la vagonon,
\^si sidis kvaza\u u sur pingloj, \^ciuminute elrigardadis tra la
fenestro kaj malbenadis la neelporteble malrapidan, la\u u \^sia
opinio, iradon de la vagonaro kaj la ma\^siniston kaj la konduktoron
kaj entute \^ciujn servantojn de fervojo, forgesinte en sia
malpacienco, ke \^sia Arturo anka\u u servas sur fervojo. Fine la
vagonaro haltis, kaj tra la malfermitaj fenestroj de la vagono \^si
eka\u udis la longe atendatan vokon de la konduktoro "stacio B., la
vagonaro staras dekkvin minutojn!" Kun eksterordinara viveco, kiun
eble envius e\^c dekkvinjara knabino, \^si elsaltis el la vagono,
--- kaj \^si tuj ekvidis Arturon, kiu anka\u u \^sin rimarkis kaj jam
rapidis al \^si renkonte. Li \^sin \^cirka\u uprenis kaj karese
kisis.

 --- Ho, mia kara, fine vi alveturis! li ekkriis \^goje. Sinjorino
Anneto de ekscito ne povis e\^c paroli. \^Si a\u utomate sin apogis
sur la brakon de Arturo, kaj ili eliris sur la veturilejon. Arturo
zorge \^sin sidigis en malnovan disbatitan veturigistan kale\^son,
li mem sidi\^gis kaj delikate \^cirka\u uprenis \^sian dikan rondan
talion. Sinjorino Anneto sentis kapturnon de ravo. La maljuna
\^cevalo uzis \^ciujn siajn fortojn, por kontentigi sian tro
postulantan mastron, kiu \^ciuminute batadis \^giajn malgrasajn
flankojn, --- \^gi svingadis la voston, spirblovadis kaj, balancante
la kapon, vane penis kuri pli rapide. La kale\^so kun la\u uta
krakado ruli\^gadis sur la malbona pavimo kaj fine, la\u u la montro
de Arturo, haltis apud la pordego de malnova ligna dometo. Arturo
helpis al sinjorino Anneto elrampi el la kale\^so kaj kondukis \^sin
tra malpura malgranda korteto, plenigita per iaj bareloj, al
flankodomo, klarigante al \^si, ke la \^cefan parton de la domo oni
en tiuj \^ci tagoj komencos rebonigi kaj ke, dank' al tiu \^ci
cirkonstanco, li devas dume "iel" loki\^gi en flanka\^{\j}o.

 --- Por mi, diris Arturo, kiel por fra\u ulo, estas tute egale, kie ajn
mi lo\^gas, --- vi do min ne malla\u udos. Ili aliris al la pordo,
kiu estis fermita. Arturo ekfrapis. La pordon malfermis ia stranga
ekzista\^{\j}o --- viro a\u u virino, tre malfacile estis ekkoni
--- malpurigita, \^cifona. Li trairis malbonodoran kuirejon kaj
eniris en negrandan malaltan \^cambreton kun de\^siritaj tapetoj kaj
kun ne blankigitaj fornoj; tiu \^ci \^cambreto prezentis kredeble la
gasto\^cambron, \^car apud la muro inter la fenestroj staris kanapo,
kovrita per nigra sen\^seligita vakstolo, anta\u u la kanapo estis
ronda tablo kaj apud \^gi de unu flanko se\^go tute ligna, de la dua
flanko --- se\^go kun leda kuseno; ankora\u u unu se\^go, en kiu
mankis unu piedo, staris apud la muro en angulo. Krom tiuj \^ci en
la tuta \^cambro estis nenia meblo. Sinjorino Anneto kun mirego
rigardis \^cirka\u uen.

 --- Vi ne atendis vidi min en tia malri\^cegeco? demandis Arturo.

 --- Malri\^ceco ne estas malvirto, respondis Anneto.

 --- Jes, sed \^gi estas pli malbona, ol malvirto, kun \^gemo diris
Arturo: almena\u u mi plivolus esti ri\^ca. En la nuna materiala
tempo ri\^ceco estas unu el la plej necesaj kondi\^coj de feli\^co,
\^car nur \^gi donas sendependan situacion, sen kiu tute ne estas
ebla vera feli\^co.

   Sinjorino Anneto singardeme sidi\^gis sur la kanapon, kaj Arturo
eliris, por ordoni tagman\^gon. Post kelkaj minutoj li revenis kaj
sidi\^gis apud sinjorino Anneto. \^Si karese prenis lian manon kaj
diris kun sincera partopreno:

 --- Mi bone prezentas al mi, kiel malfacile estis por vi, mia karulo,
post tiu lukso, al kiu vi alkutimis de infaneco, humili\^gi al via
nuna sorto!\dots

 --- Jes, mia amikino, kun profunda \^gemo respondis Arturo: komence
estis tre malfacile, sed poste nenio\dots mi alkutimis. Homo estas
tia ekzista\^{\j}o, ke \^sajne ekzistas nenio, al kio li ne povus
alkutimi: mi mem tion \^ci spertis. Nun mi tiel kunvivi\^gis kun mia
situacio, ke mia tuta pasinta vivo, al mi \^sajnas, estis nur ia
mirinda son\^ga\^{\j}o, sendita al mi de iu sor\^cisto.

 --- Kaj \^cu efektive vi neniam beda\u uris tiun \^ci son\^ga\^{\j}on? Kaj
la ri\^cegeco, kiun posedis via patro, \^cu efektive vi e\^c \^gin
ne beda\u uris?

 --- Anta\u ue mi neniam pensis pri ri\^ceco, \^car \^gi ne estis al mi
necesa. Sed nun\dots nun estas alia afero\dots

 --- Kial do nun estas alia afero?

 --- Arturo pasie \^cirka\u uprenis sinjorinon Anneton kaj komencis
senfine \^sin kisi.

 --- Nun, ekkriis li kun fajro, nun mi havas celon en la vivo, celon,
kiun mi povas atingi nur ekri\^ci\^ginte, --- tial mi nepre devas
ri\^ci\^gi, nepre, nepre!\dots

 --- Ho, mia kara, rediris sinjorino Anneto, karesante lin: per
honesta vivo estas tre malfacile ri\^ci\^gi.

 --- Sed mi volas kaj esperas per honesta vivo tion \^ci atingi.

 --- Ne, mia amiko, tion \^ci vi ne povos fari.

 --- Kaj mi pensas, ke mi povos, almena\u u la espero min ne forlasas.

 --- Kion do vi volas entrepreni? demandis sinjorino Anneto multe
ekinteresite.

 --- Mi e\^c jam entreprenis. Kaj se mi havus kvankam malmulte pli
bonajn monajn rimedojn, mia entrepreno sendube havus plenan
sukceson.

 --- Sed kia do estas tiu \^ci entrepreno.

 --- Vidu, mia kara, komencis Arturo: kiel vi jam scias, mi havis tre
bonan bienon, kiu estis vendita pro \^suldoj. Tiun \^ci bienon
a\^cetis unu mia malproksima parenco, kiu, kiel mi anta\u u nelonge
eksciis, pruntis de mia patro kvindek mil rublojn la\u u kambioj.
Tiujn \^ci kambiojn mia patro, ankora\u u anta\u u sia forveturo
Parizon, sendis por enkasigo al la Moskva Granda Ju\^gejo, kiu
decidis, ke tiu \^ci enkasigo estis komencita tute regule. Sed mia
parenco petis mian patron, ke li permesu al li pagi tiun \^ci
\^suldon post tri jaroj. La patro plenumis lian peton; poste li
balda\u u forveturis el Rusujo kaj kredeble forgesis tiun \^ci
\^suldon: almena\u u al mi li nenion parolis pri \^gi. Mi eksciis
pri tio \^ci tute okaze: iu mia bondeziranto sciigis min per anonima
letero, ke tiu \^ci afero \^gis nun ankora\u u trovi\^gas en la
Moskva Ju\^gejo. Kaj jen mi decidis relevi tiun \^ci aferon. Sed jam
\^ce la unuaj pa\^soj mi renkontis preska\u u nevenkeblan baron: mi
devas dungi bonan advokaton kaj mi tute ne havas monon\dots Mi volis
prunti la necesan sumon, sed al mi oni rifuzis, kaj nun mi estas en
tia situacio\dots

   Arturo ne finis sian penson, \^car lin interrompis sinjorino Anneto,
fajre lin ekkisinte.

 --- Karulo mia, \^si diris: plenumu mian unuan peton!

 --- Ho, jam anta\u ue mi donas mian vorton de honoro! rapide respondis
Arturo: por vi mi estas preta \^cion fari, --- nur ordonu!

 --- Prenu de mi la necesan sumon da mono!\dots Mi ne estas ri\^ca,
sed mi \^goje, kun plezurego helpos al vi kvankam malmulte\dots kiom
vi bezonas?

   Arturo ne respondis. Li videble ne atendis tian peton; sed, anta\u ue
doninte vorton, li \^sajne ne sciis, kion li devas respondi.

 --- Ne, mia bona, mia kara, li diris fine: mi ne povas, mi ne havas
la rajton preni de vi monon\dots Vi postulas ion neeblan\dots

 --- Sed via vorto?

 --- Pardonu, mia amiko, mi donis \^gin en rapideco, mi e\^c ne povis
pensi, ke vi min petos pri tio \^ci\dots Ne, ne, mi ne povas, tute
ne povas!\dots

 --- Arturo, amata mia, mia karulo, petegis sinjorino Anneto: se vi
min efektive amas, vi ne rifuzos al mi en tiaj bagateloj. Mi ja ne
donacas al vi tiun \^ci monon, mi \^gin pruntas. \^Cu estas al vi
pli agrable kuradi al fremdaj homoj, de kiuj vi facile povas ricevi
rifuzon? Kaj fine --- \^cu povas esti inter ni iaj kalkuloj?! La
afero, kiun vi entreprenas, estas ja por mi tiel same interesa kaj
grava, kiel anka\u u por vi mem, --- tiu \^ci afero estas ja nia
komuna\dots Karulo mia, diru, kiom vi bezonas?

   Arturo ankora\u u \^sanceli\^gis, sed post kelka tempo, cedante al \^siaj
admonoj kaj petegoj, li diris, ke li bezonas mil rublojn. Sinjorino
Anneto kvankam ne atendis a\u udi pri tia tre granda por \^si sumo,
tamen, vidante, ke \^sia Arturo sendube gajnos la aferon, \^si tute
ne doma\^gis la monon, --- kontra\u ue, \^si estis e\^c \^goja, ke
\^si povas helpi al Arturo, kiun \^si havis jam tempon senfine
ekami. La tuta tago pasis en viva gaja parolado pri la estonteco,
--- sed venis la vespero, kaj sinjorino Anneto forveturis en K-n.

   En tiu sama semajno Arturo ricevis per la po\^sto mil rublojn, kaj
de \^gojo li e\^c iom pli, ol ordinare, dibo\^cis en unu el la B-aj
restoracioj kun siaj bonaj kolegoj.

\begin{center}
\textbf{V}
\end{center}

   Pasis monato. Arturo nenion skribis. Sinjorino Anneto tute ne
provis kompreni la ka\u uzon de tiel longeda\u ura silento kaj forte
maltrankvili\^gis. Fine Arturo skribis al \^si longegan leteron, en
kiu li dankis por la sendita mono, kun kies helpo lia afero rapide
ekiris anta\u uen, kaj diris, ke nun li jam estas tute konvinkita
pri plena sukceso; poste li petis pardoni lian longan silenton, kiu
okazis pro tiu ka\u uzo, ke li devis persone veturi Moskvon kaj ne
havis e\^c unu minuton liberan; poste li promesis post unu monato
alveturi en K-n, por kunigi eterne sian sorton kun la sorto de la
amata virino. En la fino de la letero li skribis, ke li jam elspezis
la tutan monon, kiun li havis, kaj petis sian "belan an\^gelon"
alsendi al li ankora\u u mil rublojn, por ke li povu prepari
diversan necesa\^{\j}on por la edzi\^go. --- Sinjorino Anneto estis
tre kontenta, ke \^sia honesta amiko estas tiel senceremonia rilate
\^sin, kaj tuj sendis al li la deziritan sumon da mono, skribinte
kune kun tio \^ci plej aman, plej karesan leteron, --- kaj \^si mem
solene anoncis al \^ciuj siaj parencoj kaj konatoj, ke \^si jam
estas fian\^cino. \^Ciuj \^sin gratulis, \^ciuj tre ekinteresi\^gis
kaj demandis \^sin, kiu estas \^sia fian\^co, kie \^si konati\^gis
kun li, kie li lo\^gas k. t. p. \^Ciujn tiujn \^ci demandojn
sinjorino Anneto respondis, ke tio \^ci estas \^sia sekreto, kaj nur
kontente ridetis. Post kelkaj tagoj la feli\^ca, gaja sinjorino
Anneto komencis preti\^gadi al la edzini\^go. \^Si kuradis de unu
magazeno al alia, \^si a\^cetadis \^stofon por vestoj, tolon; \^si
mendis edzi\^gofestan veston, a\^cetis edzi\^gofestajn kandelojn.
Post du semajnoj \^si jam \^cion pretigis, kaj kun senfina
malpacienco komencis atendi sian fian\^con. Sed la stranga fian\^co
denove nenion skribis. Sinjorino Anneto en la komenco vidis en tio
\^ci nenian signifon, pensante, ke li estas nun okupita. Sed pasis
ankora\u u unu monato, pasis dua, --- sinjorino Anneto skribis al li
leteron post letero, li nenion respondis. \^Si jam forte
maltrankvili\^gis. En \^sia imago sin prezentis plej teruraj
pentra\^{\j}oj: jen \^si vidis, ke \^sia Arturo estas malsana
 --- eble e\^c mortas, kaj apud li neniu alestas; jen \^si pensis, ke
li perfidis al \^si kaj ekamis alian virinon. \^Si per \^ciuj fortoj
penis forpeli de si tiujn \^ci mallumajn pensojn, kaj \^ciam
konsolis sin per tio, ke Arturo kredeble en la klopodoj ne havas
tempon, por skribi al \^si, a\u u eble li denove forveturis Moskvon.
Tiel en senfrukta turmenta atendado pasis ankora\u u du monatoj, en
la da\u uro de kiuj la malfeli\^ca fian\^cino ne havis e\^c unu
minuton trankvilan. Fine, tute perdinte la paciencon, \^si mem
forveturis en B-n. --- Alveturinte tien, \^si venis al la konata
dometo. Kia do estis \^sia mirego, kiam \^si ekvidis, ke la
fenestroj en tiu \^ci dometo estis tute albatitaj per tabuloj kaj
sur la pordo estis malgranda tabuleto, sur kiu per grandaj literoj
estis skribita: "tiu \^ci domo estas vendata". \^Si staris senmove
kaj eble jam centan fojon legis tiun \^ci mallongan anoncon. Subite
\^si eka\u udis ies fortan malagrablan vo\^con kun hebrea akcento:

 --- Eble vi volas a\^ceti tiun \^ci dometon? Bone, tre bone, la dometo
estas inda, ke vi \^gin a\^cetu.

   Sinjorino Anneto ekrigardis returnen. Anta\u u \^si staris ia hebreo,
li ridetis kaj pin\^cis sian akran barbeton. --- \^Cu iu lo\^gas en
la flankoparto? \^si demandis la hebreon.

 --- Ne, nun tie lo\^gas nur la gardisto, sed anta\u u du a\u u tri monatoj
tie lo\^gis unu sinjoro\dots

 --- Kie do estas nun tiu \^ci sinjoro? kun viva malpacienco interrompis
lin sinjorino Anneto: kien li forveturis?\dots

 --- Fi! malestime diris la hebreo: Sinjoro!\dots Li e\^c ne estas
sinjoro, sed fripono, \^stelisto! Dio lin malbenitan forbatu!\dots

   Sinjorino Anneto ne kredis al siaj propraj oreloj. \^Si staris,
kvaza\u u frapita de tondro. Ne! ne povas esti!\dots \^Sia
Arturo\dots Dio, Dio!\dots \^Si volis kredi, ke la hebreo parolas
pri iu alia.

 --- \^Cu vi povas al mi priskribi lian ekstera\^{\j}on?

 --- Ekstera\^{\j}on? Malbonan ekstera\^{\j}on li havas, respondis la hebreo,
kaj li komencis plej detale pentri la portreton de la fripono kaj
\^stelisto, en kiu sinjorino Anneto tuj ekkonis sian Arturon!\dots

   La hebreo rakontis al \^si, ke Arturon arestis la polico, ke li estis
du a\u u tri semajnojn en la malliberejo, el kiu li lerte forkuris
kaj malaperis, oni ne scias kien.

   En tiu sama tago vespere sinjorino Anneto, malgaja, ofendita, kun
dispremita koro, revenis en K-n. \^Si multe kaj longe suferis: en la
da\u uro de kelkaj monatoj \^si preska\u u tute ne eliradis el sia
\^cambro, malbenante la sorton, kiu tiel kruele mokis \^sian amon,
kaj beda\u urante sian senrevene perditan monon.

\asterism{}

   Pasis du jaroj. Sinjorino Anneto tre \^san\^gi\^gis: \^si fari\^gis pli
maljuna, \^sia viza\^go flavi\^gis, kaj sur \^gi aperis multaj
maljunulaj sulkoj; en la bu\^so mankis multaj dentoj; \^si
\^gibi\^gis; \^sia kapo kaj manoj \^ciam tremadis, la okuloj
malforti\^gis. Du malgrandaj \^cambraj hundetoj fari\^gis la escepta
objekto de \^sia amo kaj aldoniteco. \^Ciutage post la tagman\^go
\^si kondukadis siajn amatajn bestojn promenadi.

   Unu fojon \^si, promenante, renkontis okaze amason da arestitoj,
enfor\^gitaj per katenoj kaj \^cirka\u uitaj de forta grandanombra
gardo. Vidante en tiu \^ci renkonto malbonan signon, sinjorino
Anneto volis jam reiri, kaj subite el la amaso da arestitoj \^si
eka\u udis konatan mokan vo\^con: --- Kiel vi fartas, sinjorino
Anneto? Jam longe mi vin ne vidis. Kiam do estos nia edzi\^go? \^Si
ekrigardis kaj kun plej granda teruro ekvidis Arturon!\dots

   Sin mem ne komprenante de tumulto, \^si alkuris hejmen, kaj de tiu
tempo \^si jam por \^ciam \^cesis promenadi, plivolante restadi dome
kaj amuzi\^gadi kun siaj kvarpiedaj amikoj.

\smallrule{}
