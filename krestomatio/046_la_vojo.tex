\begin{verse}
\begin{center}
\footnotesize (De barono B. N. \fsc{Delvig}.)
\end{center}

                  Tre \^carme \^gi lumis, la suno ravanta,\\
                  En mia de vivo mateno;\\
                  En brusto kura\^go kaj forto bolanta,\\
                  Espero kaj kredo en pleno\dots

                  La kamp' estas lar\^ga kaj rekta la vojo\dots\\
                  "Anta\u uen sen halto kaj timo!"\\
                  La trio \^cevala ekkuris kun \^gojo\\
                  Al celo en la malproksimo.

                  "Pru! halt'!" "Kio estas?" "Risorto
                  rompi\^gis."\\
                  "Anta\u uen! malhelpo ne granda!"\\
                  Denove nun la veturil' ekruli\^gis,\\
                  Saltante sur vojo rubanda.

                  Jen mar\^co subite! Denove ni haltis\dots\\
                  Tagmezo jam\dots Kia \^cagreno!\\
                  La veturigist' kun murmuro desaltis,\\
                  Mi pelas kun la\u uta malbeno.

                  Tri horojn ni tie sen helpo batalas,\\
                  \^Gis fine ni kun malfacilo\\
                  Elrampas. Subite--la suno jam falas ---\\
                  Jen kavo kaj nova barilo!\dots

                  "Returne! kredeble tro dekstren ni iras!"\\
                  Ni rampas\dots Jam nokt'\dots De l' turmento\\
                  Ni celon forgesis, ni nun nur sopiras\\
                  Ripozon sub ia tegmento\dots

                  Halt'! Muro! kaj mi al la pordo rapidas\dots\\
                  Nur tomboj\dots silenta mal\^gojo\dots\\
                  La veturigist' al mi montras kaj ridas:\\
                  "Ni venis! Finita la vojo!"

                  "La Tempo" sin nomas la veturigisto;\\
                  Sur voj' malfacila, facila,\\
                  Li \^ciujn alportas, fidela servisto,\\
                  Al fina ripozo trankvila.

%L. L. ZAMENHOF.
\end{verse}

\citsc{L. Zamenhof.}

\smallrule{}
