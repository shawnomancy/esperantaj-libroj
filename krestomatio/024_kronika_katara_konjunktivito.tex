\begin{center}
\footnotesize (el la libro de profesoro E. Fuchs).
\end{center}

{\sl Simptomoj.} \^Ce la kronika kataro de la konjunktivo la
objektive trovataj \^san\^goj estas entute malmulte signifaj. Estas
modera ru\^geco de la konjunktivo, a\u u nur super la tarso, a\u u
anka\u u en la parto transira. La konjunktivo estas glata kaj ne
\^svelinta; nur en malnovaj okazoj aperas hipertrofio de la
konjunktivo kun \^gia diki\^go kaj velursimila karaktero. La
sekrecio estas malgranda kaj montras sin precipe per kungluiteco de
la palpebroj matene. La blanketa \^sa\u umo, kiun oni ofte trovas en
la anguloj de la palpebroj, venas de tio, ke en sekvo de ofta
palpebrumado la larma fluida\^{\j}o kun la eliga\^{\j}o de la
Meibomaj glandetoj per bati\^gado fari\^gas speco de \^sa\u uma
emulsio. La konstanta malseki\^gado de la ha\u uto en la anguloj de
la palpebroj kondukas al la formi\^gado de ekskoriacioj. En kelkaj
okazoj la sekrecio, anstata\u u esti pliigita, \^sajnas e\^c
malpliigita. Pro la tre malgranda a\u u e\^c tute mankanta plii\^go
de la sekrecio kelkaj a\u utoroj difinas multajn okazojn de kronika
kataro ne kiel tian, sed kiel hiperemion de la konjunktivo. \^Ce la
sensignifeco de la simptomoj objektive rimarkeblaj, tiom pli granda
atento devas esti turnata al la plendoj de la paciento. Efektive la
{\sl subjektivaj simptomoj} pleje estas tiel karakteraj, ke el ili
oni facile povas meti la diagnozon de kronika kataro de la
konjunktivo. La plendoj ordinare estas la plej grandaj vespere. La
pezeco de la palpebroj, tage apena\u u rimarkebla, vespere estas
tiel granda, ke al la pacientoj estas tre malfacile teni la okulojn
nefermitaj; ili havas la senton, kvaza\u u ili volus dormi. Per la
nemulta eliga\^{\j}o, kiu en formo de mukaj fadenoj restas en la
konjunktiva sako, naski\^gas teda sento de fremda korpo, kvaza\u u
en la okulo estus polvero; kiam tiaj mukaj fadenoj meti\^gas sur la
korneon, tiam la vidado malklari\^gas a\u u tiam \^cirka\u u la
flamo de kandelo montri\^gas \^cielarkaj koloroj; krom tio la
pacientoj rakontas pri malagrablaj sentoj de diversaj specoj, nome
pri brulado kaj jukado, pri blinduli\^gado per lumo, rapida
laci\^gado de la okuloj \^ce la laborado, ofta palpebrumado k. t. p.
Matene la okuloj estas iom kungluitaj a\u u iom da flava seki\^ginta
eliga\^{\j}o trovas sin en la interna angulo de la okulo. En aliaj
okazoj ekzistas turmenta sento de sekeco kaj la okuloj povas esti
malfermataj nur kun malfacileco, kaj la paciento havas la senton
kvaza\u u la palpebroj pro manko de malsekeco estas algluitaj al la
okulpomo (kataro seka). --- Tiuj \^ci tiel diversaspecaj plendoj
tute ne trovas sin en difinita interrilato al la objektiva stato.
\^Ce multaj homoj oni vidas la konjunktivon sufi\^ce forte
ru\^gi\^ginta kaj tamen ili plendas pri nenio e\^c plej malgranda,
dum \^ce aliaj, kiuj la kuraciston simple elturmentas per siaj
piendoj, oni ofte apena\u u rimarkas iajn \^san\^getojn en la
konjunktivo.

{\sl Irado de la malsano.} La kronika kataro de konjunktivo estas
unu el la plej oftaj malsanoj de la okuloj, kiu estas renkontata
precipe \^ce personoj granda\^gaj kaj nome pli maljunaj. \^Ce
maljunuloj \^gi estas preska\u u regulo, ke ni trovas facilan gradon
da kronika konjunktivito, kiun oni nomas kataro maljunula. La da\u
urado de la konjunktivito estas ordinare longa; multaj homoj suferas
je \^gi grandan parton de sia vivo. La malsano povas konduki al
komplika\^{\j}oj, kiuj parte donas nerebonigeblajn \^san\^gojn. Al
la plej oftaj komplika\^{\j}oj apartenas la brulumo de la palpebra
rando --- blefarito --- sekve de la ofta malseki\^gado de la randoj
de la palpebroj per la pli ri\^ce eligataj larmoj. Anka\u u sekve de
la malseki\^gado per la larmoj la ha\u uto de la malsupra palpebro
estas atakata de ekzemo, a\u u \^gi fari\^gas rigida kaj
mallongi\^gas, tiel ke la libera rando jam ne perfekte almeti\^gas
al la okulpomo. Sekve de tio la larma punkto jam ne trempi\^gas en
la larman lagon, per kio la forkondukado de la larmoj en la larman
sakon estas malhelpata, la fluado de la larmoj estas pligrandigata
kaj tio \^ci denove efikas malutile sur la staton de la ha\u uto.
Tiamaniere formi\^gas rondo da eraroj, kiu kondukas al \^ciam plua
mallevi\^go de la malsupra palpebro, al la ektropio. Tiu \^ci eliro
estas ankora\u u akcelata per tio, ke la pacientoj ofte devi\^sas la
defluantajn larmojn kaj \^ce tio pasigas la tukon de supre
malsupren, per kio la malsupra palpebro estas tirata malsupren. Se
la mallongi\^go de la palpebra ha\u uto, malsekigata de larmoj,
aperigas sin pli en direkto horizontala, tiam naski\^gas
blefarofimozo. Fine la kataro ofte ka\u uzas ulceretojn de la
korneo.

{\sl Etiologio.} La ka\u uzoj de la kronika kataro estas: 1. Anta\u
uirinta kataro akra, kiu, anstata\u u plene sani\^gi, transiris en
la stadion kronikan. 2. Komunaj malutila\^{\j}oj de diversaj specoj.
Al ili apartenas anta\u u \^cio malbona aero, difektita per fumo,
polvo, varmego, kunestado de multe da homoj k. t. p. La laboristoj
en fabrikoj, kie estas multe da polvo, la kelneroj en la fumoplenaj
restoracioj k. t. p. suferas tre ofte je kronika konjunktivito. Al
tio disponas anka\u u malfrua irado dormi, nokta maldormado, troa
\^guado de alkoholaj trinka\^{\j}oj. \^Ce personoj, kiuj jam suferas
je kronika konjunktivito, \^gi multe malboni\^gas post \^cia
tiaspeca malutila\^{\j}o, ekzemple post vespero pasigita en teatro
a\u u en fumoplenaj \^cambroj. La da\u ura efikado de vento kaj
malbona vetero ka\u uzas \^ce la vila\^ganoj, veturilistoj k. s.
tiel oftan kataron. Pro tiu sama ka\u uzo anka\u u okuloj forte
elstarantaj, a\u u kies palpebroj estas mallongigitaj (lagoftalmo),
estas atakataj de kataro, \^car ili tro malmulte estas defenditaj
kontra\u u la aero. La efiko, kiun faras sur la konjunktivon la
konstanta kuntu\^si\^gado kun la aero, montri\^gas la plej bone \^ce
la ektropio, \^ce kiu la neka\^sita konjunktivo de la tarso
fari\^gas forte ru\^ga kaj dika, velursimila a\u u e\^c tuberplena.
Tiom same malbone kiel longeda\u uran kuntu\^si\^gadon kun la aero,
la konjunktivo elportas anka\u u la konstantan foreston de tiu \^ci
lasta, tial \^ce longe da\u uranta \^cirka\u uligo de la okulo
anka\u u aperas kronika kataro. 3. Troa stre\^cado de la okuloj,
precipe \^ce personoj hipermetropiaj kaj astigmatismaj, povas ka\u
uzi kronikan kataron. 4. Lokaj malutila\^{\j}oj. Al ili apartenas
incitado de la konjunktivo per fremdaj korpoj, kiuj restas en la
konjunktiva sako; al tiaj fremdaj korpoj oni devas alkalkuli en pli
vasta senco de la vorto anka\u u okulharojn, kiuj estas turnitaj
kontra\u u la okulpomo. En la plimulto da okazoj la loka
malutila\^{\j}o konsistas en ia alia malsano de la okulo, kiu
aperigas post si la kataron kiel suferon sekvan, kiel ekzemple
blefarito a\u u infarktoj en la glandetoj Meibomaj. Reteni\^go de
larmoj en sekvo de blenoreo de la larma sako a\u u de nesufi\^ca
trempi\^gado de la larmaj punktoj en la larman lagon estas ofta ka\u
uzo de la kataro, tial oni neniam devas forgesi \^ce unuflanka
kataro esplori, \^cu ne ekzistas ia sufero de la larmaj vojoj. La
kataroj ka\u uzitaj de cirkonstancoj lokaj distingi\^gas de la
kataroj ka\u uzitaj de malutila\^{\j}oj komunaj per tio, ke ili tre
ofte estas unuflankaj, dum en la lasta okazo la\u u la naturo de la
afero pleje suferas amba\u u okuloj.

{\sl Terapio.} Estas kompreneble, ke la kuracado devas anta\u u
\^cio turni atenton al la ka\u uza momento, per konforma reguligo de
la komunaj kondi\^coj de la vivo, kiom tion \^ci permesas la
profesio de la paciento, per forigo de ekzistantaj lokaj ka\u uzoj
de la kataro k. t. p. Por la kuracado de la konjunktivo mem ni
anta\u u \^cio, kiel \^ce la kataro akra, havas por nia dispono la
nitran ar\^genton, kiun oni uzas por \^smirado per peniko (en solvo
2 p. 100) a\u u por engutado (en solvo 1/4 --- 1/2 p. 100). Oni uzas
\^gin nur en tiuj okazoj, en kiuj la kataro estas akompanata de pli
forta sekrecio kaj do malfirmi\^go de la konjunktivo, \^car tiaj
periodoj de akra malsani\^go ofte okazas en la da\u uro de \^ciu
kronika kataro, kaj anka\u u tiam, kiam ekzistas jam hipertrofio de
la konjunktivo. En la ceteraj okazoj sufi\^cas adstringaj gutoj
(okulakvoj), kiujn la paciento mem povas al si engutadi. La plej
uzataj el tiuj \^ci okulakvoj estas: flava adstringa okulakvo\footnote{
Tiu \^ci okulakvo, nomata anka\u u okulakvo de Horst, nun en la plimulto 
da landoj jam plu ne estas oficinala; tamen \^gi faras bonegajn servojn 
kaj en kelkaj okazoj oni povas \^gin anstata\u uigi per neniu alia rimedo.
La\u u la nova A\u ustruja farmakopeo \^gi devas esti preparata en la
sekvanta maniero
\begin{center}
\begin{tabu} to 0.7\textwidth{l@{ }X@{}S}
Rp. & Amonio klora \dotfill & 0,5\\
 & Zinko sulfura  \dotfill & 1,25\\
 & \hspace*{1cm}Solvu en: &  \\
 &  Akvo distilita \dotfill & 200,0\\
 & \hspace*{1cm}Aldonu: &  \\
 &  Kamforo \dotfill & 0,4\\
 & \hspace*{1cm}Solvita en: &   \\
 & Spirito vina dissolvita \dotfill & 20,0\\
 & \hspace*{1cm}Aldonu: & \\
 &  Safrano \dotfill & 0,1\\
\end{tabu}

Tenu 24 horojn ofte skuante, filtru.\par
\end{center}
La okulakvo de Romershausen, kiu anka\u u estas multe uzata \^ce
kronika kataro de la okuloj, konsistas el miksa\^{\j}o do fenkola
tinkturo kaj fenkola akvo.} (collyrium adstringens luteum), a\u u safrana tinkturo de opio
(tinctura opii crocata), amba\u u ordinare estas porskribataj ne
pure, sed miksite kun egala kvanto da akvo; kupro aluna, zinko
sulfura, amba\u u en 1/2-1 p. 100 solvo; plue aluno, tanino, bora
acido kaj aliaj adstringanta\^{\j}oj.

La vica ordo, en kiu tiuj \^ci okulakvoj estas cititaj, respondas
proksimume al ilia la\u ugradeco, de la plej fortaj \^gis la plej
delikataj. Ili devas esti engutataj \^ciutage unu a\u u du fojojn,
sed ne vespere. Ni elkalkulis tiom da ili, \^car estas bone havi pli
grandan elekton el ili, \^car \^ce la longa da\u urado de la kataro
oni ofte devas \^san\^gi la rimedojn. \^Ciu rimedo, uzata tro longe,
perdas sian efikecon, \^car la konjunktivo al \^gi alkutimi\^gas.
Kontra\u u la kunglui\^gado de la palpebroj, kiel anka\u u kontra\u
u ekzistantaj ekskoriacioj oni igas vespere anta\u u la endormi\^go
froti sur la fermitaj palpebroj \^smira\^{\j}on kun blanka hidrarga
precipitato (1/2-1 p. 100).

      

\smallrule{}
