\begin{verse}
\begin{center}
\footnotesize (El \fsc{Heine}.)
\end{center}

\begin{center}
\textbf{I}
\end{center}

                  \vin  La \^sipo\^sar\^gisto Minheer van Kek\\
                 Sidadas en longa medito,\\
                 Li taksas sur \^sipo la koston de l' \^sar\^g',\\
                 La sumon de ebla profito. \\
                 \vin   "Tricent barelegoj da pipro, da gum'\\
                 Kaj poste la sablo de oro\dots\\
                 \^Gi estas tre bona, sed tamen al mi\\
                 Pli pla\^cas la nigra trezoro.\\
                  \vin  "Ses centojn da negroj mi \^ce Senegal\\
                 Akiris je prezo profita.\\
                 Malmola viando, simila al \^ston',\\
                 La membroj --- el \^stalo for\^gita.\\
                  \vin  "Mi donis por tiu \^ci bela akir'\\
                 Brandakvon, glasperlojn kaj feron;\\
                 Gajnonte en \^cio ok centojn je cent,\\
                 Mi faros bonegan aferon.\\
                  \vin  "Se restos el ili ne pli ol tricent\\
                 \^Ce veno al Rio-\^Janejro,\\
                 Mi havos cent orajn dukatojn por pec'\\
                 De l' domo Gonzales Perejro."\\
                 \vin   Sed jen deturni\^gas Minheer van Kek\\
                 De sia pensado persista:\\
                 Al li proksimi\^gas la \^hirurgiist'\\
                 Van Smisen, la \^sip-kuracisto.\\
                  \vin  Van Smisen, maldika malgrasa figur',\\
                 La nazo kun ru\^gaj verukoj.\\
                 "He, kiel nun fartas --- ekkrias Van Kek,\\
                 La negroj kun la anta\u utukoj?"\\
                  \vin  Van Smisen salutas. Hodia\u u al vi,\\
                 Ne estos agrabla raporto:\\
                 Beda\u ure en tiuj \^ci tagoj sur \^sip'\\
                 Ofti\^gis l' apero de morto.\\
                  \vin  "\^Gis nun \^ciutage mortadis po du,\\
                 Hodia\u u --- ne malpli ol seso,\\
                 Du viroj, kvar inoj: la morto \^ce ni\\
                 Laboras kun granda sukceso.\\
                 \vin   "Mi la malvivulojn esploris kun pen',\\
                 \^Car ofte la nigraj satanoj\\
                 Trompadas, por esti \^{\j}etitaj de l' \^sip',\\
                 Estante plenvivaj kaj sanaj.\\
                 \vin   "Mi lasis depreni de \^ciu mortint'\\
                 La feran multpezan katenon,\\
                 Ke ili ne estu barataj de \^gi,\\
                 Faronte la maran promenon.\\
                  \vin  "Kaj tuj el la maro elna\^gis areg'\\
                 Da skvaloj sangame-malsataj,\\
                 La bestoj avidas je negroviand'\\
                 Kaj estas de mi pensiataj.\\
                  \vin  "De l' tago ke ni nin demetis de l' bord',\\
                 Nin sekvas \^ci tiuj gajuloj\\
                 Kaj flaras l' aeron, turni\^gas al mi\\
                 Kun mutaj, sed klaraj postuloj.\\
                 \vin   "\^Gi estas amuzo a\u uskulti de l' \^sip'\\
                 La dentokrakadon de l' bestoj;\\
                 La unuj profitas je kapo a\u u man',\\
                 L' aliaj je sangaj intestoj.\\
                 \vin   "Post sata man\^gado la bestoj al ni\\
                 Alna\^gas en aro solena\\
                 Kaj min rigardadas kun preska\u u-dezir'\\
                 Min danki por tia festeno."
                 \vin   Sed lin interrompas, \^gemante, van Kek\\
                 Kun krio: "Malbeno de sorto!\\
                 Kiele malhelpi je tia malbon'\\
                 Kaj savi la negrojn de morto?"\\
                  \vin  Van Smisen rediras: "En tiu \^ci mort'\\
                 La negroj mem estas la ka\u uzo,\\
                 \^Car ili difektas l'aeron de l' \^sip'\\
                 Per spiro kaj mortas pro na\u uzo.\\
                  \vin  "Aliaj pereas pro melankoli'\\
                 Kaj fine pro granda enuo,\\
                 Kaj farus al ili utilon aer',\\
                 Dancado, muziko kaj bruo."\\
                  \vin  Kaj diras van Kek: "La konsilo, doktor',\\
                 Min ravas, \^gi estas tre bela,\\
                 Vi montras vin, mi \^gin konfesas al vi,\\
                 Pli sa\^ga ol Aristotelo.\\
                 \vin   Se la prezidanto de la Societ'\\
                 "Tulipo-nobligo Holanda"\\
                 Sin nomas sa\^gulo, --- kompare je vi\\
                 Li estas azeno multgranda.\\
                  \vin  "Muzikon, muzikon! la nigra viand'\\
                 Tuj sur la ferdekon de l' \^sipo!\\
                 Kaj kiu ne trovos amuzon en danc\\
                 \^Ci tiu \^gin trovos en vipo!"

\begin{center}
\textbf{II}
\end{center}

                 \vin   Milaroj da steloj de l' blua \^ciel'\\
                 Bril-lumas per fajr-iluminoj,\\
                 Rigardas kun varmo, kun amo, kun sa\^g'\\
                 Kiele okuloj virinaj.\\
                 \vin   Rigardas la maron, kaj tiu \^ci mar'\\
                 Fosfore radias purpuron\\
                 Kaj balanci\^getas, kaj ondo post ond'\\
                 El\^gemas voluptan plezuron.\\
                 \vin   La veloj ne bruas \^ce l' masto de l' \^sip',\\
                 Rulitaj de la \^sipanaro,\\
                 Sed lumas lanternoj sur \^gia ferdek'\\
                 Kaj sonas muzik' sur la maro.\\
                  \vin  \^Sipano muzikas per granda tambur'\\
                 Kaj la kuiristo per fluto,\\
                 La \^sipa piloto --- per violon\^cel',\\
                 Kaj Smisen --- trumpet' per dufuta.\\
                  \vin  Kaj cento da negroj, la seksoj en kun',\\
                 Krietas kaj saltas kaj dancas,\\
                 Post \^cia eksalto la fera katen'\\
                 Sin brue kaj takte balancas.\\
                  \vin  La negroj saltadas kun bruo kaj \^goj'\\
                 Kaj iam negrin-belulino\\
                 Alpremas plen-nudan amikon al si\\
                 Kun voluptemeco senfina.\\
                  \vin  La ekzekutisto kun gaja rigard'\\
                 Gvidadas la ordon sur \^sipo,\\
                 Li estas de l' balo la "Maitre des plaisirs"\\
                 Kaj danci invitas per vipo.\\
                 \vin   Sonadas trumpeto, krakadas tambur',\\
                 La\u uti\^gas la bruo sova\^ga,\\
                 Kaj \^cie veki\^gas la monstroj de mar'\\
                 Dormintaj en dormo malsa\^ga.\\
                  \vin  Duone-dormante alvenas skvalar'\\
                 En centoj, pikata de miro,\\
                 Rigardas la \^sipon kun granda konfuz',\\
                 \^Cu eble montri\^gos akiro.\\
                  \vin  Sed ili rimarkas: la horo de l' man\^g'\\
                 Ne estas ankora\u u sonita,\\
                 Kaj \^ciu oscedas per lar\^ga bu\^seg',\\
                 De dentoj-segiloj plantita.\\
                 \vin   Sonadas trumpeto, krakadas tambur',\\
                 La negroj dancadas sen fino,\\
                 La skvaloj mordegas la vostojn al si\\
                 Pro la malsatego obstina.\\
                 \vin   Mi pensas, al ili ne pla\^cas muzik'\\
                 Kiele al \^cia fripono.\\
                 "Ne kredu al malmuzikema brutul',"\\
                 Eldiris poet' Albiona.\\
                 \vin   Sonadas trumpeto, krakadas tambur',\\
                 La negroj senlace baletas,\\
                 \^Ce l' alta fokmasto Minheer van Kek\\
                 Rigardas \^cielon kaj petas:\\
                 \vin   "Pro Kristo! konservu, plej sankta Sinjor',\\
                 La vivon al miaj sova\^gaj,\\
                 Se ili ekpekis, vi scias, ho Di',\\
                 Ke ili ja estas malsa\^gaj.\\
                 \vin   "Pardonu Vi ilin, pro amo al Krist',\\
                 Mortinta por savo de l' tero, ---\\
                 \^Car se al mi restos ne pli ol tricent,\\
                 Pereos la tuta afero."

\end{verse}

\citsc{A. Kofman.}

\smallrule{}

