\begin{center}
\footnotesize (El Andersen.)
\end{center}

   Anta\u u multaj jaroj vivis unu re\^go, kiu tiel amis belajn novajn
vestojn, ke li elspezadis sian tutan monon, por nur esti \^ciam bele
ornamita. Li ne zorgadis pri siaj soldatoj, nek pri teatro kaj
\^caso, esceptinte nur se ili donadis al li okazon montri siajn
novajn vestojn. Por \^ciu horo de la tago li havis apartan surtuton,
kaj kiel pri \^ciu alia re\^go oni ordinare diras: "li estas en la
konsilanejo", oni tie \^ci \^ciam diradis: "la re\^go estas en la
vestejo".

   En la granda urbo, en kiu li lo\^gis, estis tre gaje; \^ciun tagon tien
venadis multaj fremduloj. Unu tagon venis anka\u u du trompantoj,
kiuj diris, ke ili estas teksistoj kaj teksas la plej belan
\^stofon, kiun oni nur povas al si prezenti; ke ne sole la koloroj
kaj desegnoj de tiu \^ci \^stofo estas eksterordinare belaj, sed la
vestoj, kiujn oni preparas el tiu \^ci \^stofo, havas la mirindan
econ, ke al \^ciu, kiu ne ta\u ugas por sia ofico a\u u estas tro
malsa\^ga, ili restas nevideblaj.

 --- Tio \^ci estas ja bonegaj vestoj! pensis la re\^go; havante tian
surtuton, mi ja povus scii\^gi, kiu en mia regno ne ta\u ugas por la
ofico, kiun li havas; mi povus diferencigi la sa\^gajn de la
malsa\^gaj! Jes, la \^stofo devas tuj esti teksita por mi! Kaj li
donis al la amba\u u trompantoj grandan sumon da mono anta\u ue, por
ke ili komencu sian laboron.

   Ili starigis du teksilojn, faris mienojn kvaza\u u ili laboras, sed
havis nenion sur la teksiloj. Tamen en la postuloj ili estis tre
fervoraj kaj postuladis la plej delikatan silkon kaj la plej bonan
oron. Tion \^ci ili metadis en siajn proprajn po\^sojn kaj laboradis
super la malplenaj teksiloj, kaj e\^c \^gis profunda nokto.

 --- Mi volus scii, kiom de la \^stofo ili jam pretigis! ekpensis
la re\^go, sed kaptis lin kelka timo \^ce la penso, ke tiu, kiu
estas malsa\^ga a\u u ne bone ta\u ugas por sia ofico, ne povas vidi
la \^stofon. Li estis kvankam konvinkita, ke li pro si ne devas
timi, tamen li preferis anta\u ue sendi alian personon, por vidi,
kiel la afero staras. \^Ciuj homoj en la tuta urbo sciis, kian
mirindan forton la \^stofo havas, kaj \^ciu kun senpacienco jam
volis vidi, kiel malsa\^ga lia najbaro estas.

 --- Mi sendos al la teksistoj mian maljunan honestan ministron! pensis
la re\^go, li la plej bone vidos, kiel la \^stofo elrigardas, \^car
li estas homo sa\^ga kaj neniu pli bone ta\u ugas por sia ofico, ol
li!

   Tiel la maljuna bonkora ministro iris en la salonon, en kiu la amba\u u
trompantoj sidis anta\u u la malplenaj teksiloj kaj laboris. "Dio,
helpu al mi! ekpensis la maljuna ministro, lar\^ge malfermante la
okulojn, mi nenion povas vidi!" Sed li tion \^ci ne eldiris.

   La amba\u u trompantoj petis lin alveni pli proksime kaj demandis, \^cu
\^gi ne estas bela desegno kaj belegaj koloroj. \^Ce tio \^ci ili
montris la malplenan teksilon, kaj la malfeli\^ca ministro uzis
\^ciujn fortojn por malfermi bone la okulojn, sed li nenion povis
vidi, \^car nenio estis.

 --- Mia Dio! li pensis, \^cu mi estas malsa\^ga? tion \^ci mi neniam
supozis kaj tion \^ci neniu devas scii\^gi! \^Cu mi ne ta\u ugas por
mia ofico? Ne, neniel mi povas rakonti, ke mi ne vidas la
teksa\^{\j}on!

 --- Nu, vi ja nenion diras! rimarkis unu el la teksantoj.

 --- Ho, \^gi estas bonega, tre \^carma! diris la maljuna ministro kaj
rigardis tra siaj okulvitroj. Tiu \^ci desegno kaj tiuj \^ci
koloroj! Jes, mi raportos al la re\^go, ke \^gi tre al mi pla\^cas!

 --- Tre agrable al ni! diris la amba\u u teksistoj kaj nomis la kolorojn
kaj komprenigis la neordinaran desegnon. La maljuna ministro atente
a\u uskultis, por povi diri tion saman, kiam li revenos al la
re\^go; kaj tiel li anka\u u faris.

   Nun la trompantoj postulis pli da mono, pli da silko kaj oro, kion
ili \^ciam ankora\u u bezonis por la teksa\^{\j}o. Ili \^cion metis
en sian propran po\^son, en la teksilon ne venis e\^c unu fadeno,
sed ili, kiel anta\u ue, da\u urigadis labori super la malplenaj
teksiloj.

   La re\^go balda\u u denove sendis alian bonkoran oficiston, por revidi,
kiel iras la teksado kaj \^cu la \^stofo balda\u u estos preta.
Estis kun li tiel same, kiel kun la ministro, li rigardadis kaj
rigardadis, sed \^car krom la malplena teksilo nenio estis, tial li
anka\u u nenion povis vidi.

 --- Ne vere, \^gi estas bela peco da \^stofo? diris la trompantoj kaj
montris kaj klarigis la belan desegnon, kiu tute ne ekzistis.

 --- Malsa\^ga mi ja ne estas! pensis la sinjoro, tial sekve mi ne
ta\u ugas por mia bona ofico. Tio \^ci estas stranga, sed almena\u u
oni ne devas tion \^ci lasi rimarki! Tiel li la\u udis la \^stofon,
kiun li ne vidis, kaj certigis ilin pri sia \^gojo pro la belaj
koloroj kaj la bonega desegno. Jes, \^gi estas rava! li diris al la
re\^go.

   \^Ciuj homoj en la urbo parolis nur pri la belega \^stofo.

   Nun la re\^go mem volis \^gin vidi, dum \^gi estas ankora\u u sur la
teksiloj. Kun tuta amaso da elektitaj homoj, inter kiuj sin trovis
anka\u u la amba\u u maljunaj honestaj oficistoj, kiuj estis tie
anta\u ue, li iris al la ruzaj trompantoj, kiuj nun teksis per
\^ciuj fortoj, sed sen fadenoj.

 --- Nu, \^cu tio \^ci ne estas efektive belega? diris amba\u u honestaj
oficistoj. Via Re\^ga Mo\^sto nur admiru, kia desegno, kiaj koloroj!
kaj \^ce tio \^ci ili montris sur la malplenan teksilon, \^car ili
pensis, ke la aliaj kredeble vidas la \^stofon.

 --- Kio tio \^ci estas! pensis la re\^go, mi ja nenion vidas! Tio \^ci
estas ja terura! \^Cu mi estas malsa\^ga? \^cu mi ne ta\u ugas kiel
re\^go? tio \^ci estus la plej terura, kio povus al mi okazi. Ho,
\^gi estas tre bela, diris tiam la re\^go la\u ute, \^gi havas mian
plej altan aprobon! Kaj li balancis kontente la kapon kaj observadis
la malplenan teksilon; li ne volis konfesi, ke li nenion vidas. La
tuta sekvantaro, kiun li havis kun si, rigardadis kaj rigardadis,
sed nenion pli rimarkis, ol \^ciuj aliaj; tamen ili \^ciam ripetadis
post la re\^go: ho, \^gi ja estas tre bela! Kaj ili konsilis al li
porti tiujn \^ci belegajn vestojn el tiu \^ci belega materialo la
unuan fojon \^ce la solena irado, kiu estis atendata. Rava, belega,
mirinda! ripetadis \^ciuj unu post la alia kaj \^ciuj estis tre
\^gojaj. La re\^go donacis al la amba\u u trompantoj kavaliran
krucon kaj la titolon de sekretaj teksistoj de la kortego.

   La tutan nokton anta\u u la tago de la parado la trompantoj pasigis
maldorme kaj ekbruligis pli ol dekses kandelojn. \^Ciuj povis vidi,
kiel okupitaj ili estis je la pretigado de la novaj vestoj de la
re\^go. Ili faris mienon, kvaza\u u ili prenas la \^stofon de la
teksiloj, tran\^cadis per grandaj tondiloj en la aero, kudradis per
kudriloj sen fadenoj kaj fine diris: "nun la vestoj estas pretaj!"

   La re\^go mem venis al ili kun siaj plej eminentaj korteganoj, kaj
amba\u u trompantoj levis unu manon supren, kvaza\u u ili ion tenus,
kaj diris: "Vidu, jen estas la pantalono! jen estas la surtuto! jen
la mantelo! kaj tiel plu. \^Gi estas tiel malpeza, kiel
aranea\^{\j}o! oni povus pensi, ke oni nenion portas sur la korpo,
sed tio \^ci estas ja la plej grava eco!"

 --- Jes! diris \^ciuj korteganoj, sed nenion povis vidi, \^car nenio
estis.

 --- Via Re\^ga Mo\^sto nun volu plej afable demeti Viajn plej altajn
vestojn, diris la trompantoj, kaj ni al Via Re\^ga Mo\^sto tie \^ci
anta\u u la spegulo vestos la novajn.

   La re\^go demetis siajn vestojn, kaj la trompantoj faris, kvaza\u u
ili vestas al li \^ciun pecon de la novaj vestoj, kiuj kvaza\u u
estis pretigitaj; kaj ili prenis lin per la kokso kaj faris kvaza\u
u ili ion alligas --- tio \^ci devis esti la trena\^{\j}o de la
vesto --- kaj la re\^go sin turnadis kaj returnadis anta\u u la
spegulo.

 --- Kiel belege ili elrigardas, kiel bonege ili sidas! \^ciuj kriis.
Kia desegno, kiaj koloroj! \^gi estas vesto de granda indo!

 --- Sur la strato oni staras kun la baldakeno, kiun oni portos super
Via Re\^ga Mo\^sto en la parada irado! raportis la \^cefa
ceremoniestro.

 --- Nu, mi estas en ordo! diris la re\^go. \^Cu \^gi ne bone sidas?
Kaj ankora\u u unu fojon li turnis sin anta\u u la spegulo, \^car li
volis montri, ke li kvaza\u u bone observas sian ornamon.

   La \^cambelanoj, kiuj devis porti la trena\^{\j}on de la vesto, eltiris
siajn manojn al la planko, kvaza\u u ili levas la trena\^{\j}on. Ili
iris kaj tenis la manojn eltirite en la aero; ili ne devis lasi
rimarki, ke ili nenion vidas. Tiel la re\^go iris en parada mar\^so
sub la belega baldakeno, kaj \^ciuj homoj sur la stratoj kaj en la
fenestroj kriis: "Ho, \^cielo, kiel senkomparaj estas la novaj
vestoj de la re\^go! Kian belegan trena\^{\j}on li havas al la
surtuto! kiel bonege \^cio sidas!" Neniu volis lasi rimarki, ke li
nenion vidas, \^car alie li ja ne ta\u ugus por sia ofico a\u u
estus terure malsa\^ga. Nenia el la vestoj de la re\^go \^gis nun
havis tian sukceson.

 --- Sed li ja estas tute ne vestita! subite ekkriis unu malgranda
infano. --- Ho \^cielo, a\u udu la vo\^con de la senkulpeco! diris
la patro; kaj unu a! la alia murmuretis, kion la infano diris.

 --- Li estas tute ne vestita; tie staras malgranda infano, kiu diras,
ke li tute ne estas vestita! Li ja tute ne estas vestita! kriis fine
la tuta popolo. Tio \^ci pikis la re\^gon, \^car al li jam mem
\^sajnis, ke la popolo estas prava; sed li pensis: nun nenio helpos,
oni devas nur kura\^ge resti \^ce sia opinio! Li prenis teni\^gon
ankora\u u pli fieran, kaj la \^cambelanoj iris kaj portis la
trena\^{\j}on, kiu tute ne ekzistis.

\smallrule{}
