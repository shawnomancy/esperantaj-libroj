\begin{verse}
\begin{center}
\footnotesize (El K. A. \fsc{Nikander}.)
\end{center}
                     Benite vi venas, vespero,\\
                     Vi, klara, trankvila, al ni!\\
                     Pleni\^gas la kor' de espero;\\
                     Feli\^con ni havas de vi.

                     Kun oraj flugiloj an\^gelo\\
                     Malsupren venetas vesper':\\
                     Vivantojn pacigas \^cielo\\
                     Kaj donas ripozon al ter'.

                     Filino de nokto kaj tago,\\
                     Konsola, paciga liber'!\\
                     Jen! bele sur akvo de l' lago\\
                     Pentri\^gas \^cielo kaj ter'.

                     Vi pentras bluantajn montetojn,\\
                     Kverkaron brilantan en or';\\
                     Vi kisas ru\^gantajn rozetojn,\\
                     Dormetas senzorge la flor'.

                     Kaj birdo libere \^gojadas,\\
                     Espero fidela en \^gi,\\
                     La "Ave Maria'n" kantadas\\
                     En valo \^gi en harmoni'.

                     La ter' per mallumo kovri\^gas,\\
                     Pensema amik'; kion pli?\\
                     La sun' de \^ciel' mallevi\^gas,\\
                     Sed morga\u u revenos al ni.

                     \^Gi sidas sur nubo, la brila,\\
                     Ka\^sita en la okcident';\\
                     Tuj, homan esperon simila,\\
                     Tuj levos sin en l'orient'.

                     Pleni\^gas nun kor' de espero,\\
                     Feli\^con ni havas de vi!\\
                     Vi, klara, trankvila vespero,\\
                     Benite vi venas al ni.

%O. ZEIDLITZ.
\end{verse}
\citsc{O. Zeidlitz.}

\smallrule{}
