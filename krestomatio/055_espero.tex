\begin{verse}
\begin{center}
\footnotesize (El \fsc{Schiller}.)
\end{center}
                  Tre multe homar' kun espero en koro\\
                  Pri vivo estonta parolas,\\
                  Kaj \^ciam al celo feli\^ca kaj ora\\
                  Aliri plej balda\u u \^gi volas.\\
                  Jen mondo ekvelkas, jen ree disfloras, ---\\
                  Sed homo esperon sen\^cese adoras.

                  Espero kun li sen-aparte vivadas;\\
                  \^Gi knabon dorlotas, junulon\\
                  Per sor\^ca radio \^gi gaje logadas,\\
                  Konsolas \^ce tomb' maljunulon:\\
                  Li tre lacigita per vojo de tero,\\
                  Foriras en tombon kun dol\^ca espero.

                  Ho, ne! \^gi ne estas elpenso malvera,\\
                  Per revoj malsa\^gaj naskita!\\
                  Ni scias, ni sentas kun kredo sincera,\\
                  Ke estos esper' plenumita.\\
                  Kaj tiu \^ci sento \^cu estas kapabla\\
                  Nin trompi en nia reva\^{\j}o agrabla?
\end{verse}

\citsc{V. Devjatnin.}

\smallrule{}
