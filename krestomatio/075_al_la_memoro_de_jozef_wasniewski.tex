\begin{verse}
            \vin   Floro falis ankora\u u! seki\^gis herbero kaj mortis,\\
            Velkis pro vintra la frosto \^gia verdanta folio!\\
            Morto senkora, ho kial estas al vi permesite\\
            Fosi en fosan profundon \^cion la belan en mondo?\\
            Kial, ho kial fali\^gas la herbo, verdanta sur kampo,\\
            Kial disi\^gas de vivo fre\^sa, floranta \^gis nun?\\
            Dol\^ca la morto por la maljunulo, la laca de vivo;\\
            Ku\^sas ar\^genta la haro bele sub tomba cipreso.\\
            Tiun ne plendu, \^car lia ja estas feli\^ca la sorto,\\
            \^Car en la tombon li mem iras kun \^gojo, sen timo;\\
            Dol\^ca e\^c morto junula, la mort' en mateno de l' vivo,\\
            Anta\u u ol frosta la nokto detruis bur\^gonon en \^germo!\\
            Dol\^ca la morto, kiam ne pasis la kredo je vivo,\\
            Kiam trompa vivado ne kovris flugilon per polvo!\\
            Sed jam kiam junulo viri\^gis kaj lasis la ludon,\\
            Vivo nun estas ja \^carmo, nun malfeli\^co la morto.\\
            Tiam ja estas maldol\^ce disi\^gi de faro farata:\\
            \^Gojus la koro, vidante frukton de longa laboro.\\
            Kiu do \^ciam por noblaj ideoj vivadis, batalis,\\
            Tiu kun akra mal\^gojo iras en valon de ombroj,\\
            Lasas la komencita\^{\j}on kun pleje mordanta doloro:\\
            Estas malgranda la aro, kiu batalis kun li!

             \vin  Akra estas funebro! Neniam pli noblan animon\\
            Ol la \^{\j}us forirantan trovos vi sur nia tero.\\
            E\^c se vi iros, simile al Diogeno, tra l'urboj\\
            Kun la lumilo en mano, {\sl homon} ser\^cante sur strato,\\
            Tamen neniun en tuta la mondo renkonti vi povos,\\
            Kiu pli inde ol li portas la nomon de homo,\\
            Vivon vere pli {\sl homan} travivis en senco profunda,\\
            Kaj dedi\^citan al \^cio alta kaj nobla en mondo.\\
            Amo al \^cio, al \^ciuj, jen fundo de lia esenco,\\
            Amo en faro kaj vorto, jen kio estis li mem!\\
            Por la premitaj, por \^ciuj suferoj, por la mizeruloj\\
            Sentis la nobla animo flame kaj varme \^ce li.\\
            Ne e\^c iu al li konfidis vane doloron,\\
            Vane neniu \^ce li frapis la pordon de l'kor'.

            \vin   For li estas! Ho, kiu plenigos la lokon lasitan?\\
            Kiu plenumos en mondo faron farotan de li?\\
            For li estas. Sed tamen \^ce ni li vivas ankora\u u,\\
            Kiuj en akra doloro staras \^ce nova la tombo.\\
            Ni, kiuj konis l'animon poetan ne nur per la kantoj,\\
            Kiujn li donis al ni, sed per la tono la varma\\
            El lia brust' eliranta, portanta varmon \^cirka\u uen,\\
            Kie trovi\^gis la kordo, kiu agordis al \^gi.\\
            Nin neniam forlasos la bela sunhela memoro\\
            De la mortinta amiko, \^gi lumos belege en koro!\\
            Tiel li vivas kaj tiel li estos vivanta, \^gis kiam\\
            Restos neniu jam plu, kiu lin konis en ver'.

%V. LANGLET.
\end{verse}
\citsc{V. Langlet.}

\smallrule{}
