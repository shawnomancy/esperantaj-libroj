\begin{verse}
\begin{center}
\footnotesize (El \fsc{Herder}.)
\end{center}
                        Fini\^gis varmega jam tago,\\
                        Kaj nokto agrabla fari\^gis.\\
                        Sub ombra cipreso, \^ce lago\\
                        Diino de Zorgo sidi\^gis.

                        El ru\^ga argilo Diino\\
                        Malgaje figuron skulptadis, ---\\
                        Kaj luno, de nokto re\^gino,\\
                        De blua \^ciel' rigardadis.

                        Kaj Ze\u uso al bela Diino\\
                        Parolis kun dia favoro:\\
                        "Rakontu, malgaja knabino,\\
                        Pri via doloro en koro."

                        "Ho, Ze\u uso!" Diino rediras:\\
                        "Alportu vi al mi plezuron!\\
                        Nur solan sen fin' mi deziras:\\
                        Vivigu vi mian figuron!"

                        --- Volonte mi vin kontentigos:\\
                        Vivi\^gas figuro jam via!\dots\\
                        Sed tamen de nun mi devigos,\\
                        Ke estu eterne li mia. ---

                        "Ho, ne!" al li Zorgo parolas:\\
                        Figuron mi longe skulptadis,\\
                        Al vi mi lin cedi ne volas,\\
                        \^Car lin mi por mi preparadis!

                        Por mi lin restigu! Korege,\\
                        Sed vane \^si Ze\u uson admonis.\\
                        --- Ho, ne! mi lin amas varmege,\\
                        Al li \^car animon mi donis! ---

                        "Por kio disputon vi tenas?"\\
                        Subite la Tero ekdiras:\\
                        "Al mi li sen dub' apartenas,\\
                        \^Car tera en teron foriras!"

                        Kaj dioj disputis senfine,\\
                        \^Car cedi neniu deziris\\
                        Figuron argilan, --- sed fine\\
                        Al Kron' ili \^ciuj ekiris.

                        "Nu, \^car reciproke ne volas\\
                        Figuron vi tiun \^ci cedi,"\\
                        Grizhara Saturn' ekparolas:\\
                        "Vi kune lin povas posedi.

                        A\u uskultu do, dioj: nur kiam\\
                        Li mortos, anim' apartenos\\
                        Al Ze\u uso potenca, --- kaj tiam\\
                        Jam Ter' lian korpon alprenos.

                        Kaj vi, ho, de Zorgo Diino,\\
                        Posedu lin \^gis lia morto!\dots"\\
                        Mi pensas, ke tia difino\\
                        Similas je homa la sorto?

%V. DEVJATNIN.
\end{verse}

\citsc{V. Devjatnin.}

\smallrule{}


