\begin{verse}
                        Lin vidis en \^gardeno ---\\
                        Ekpla\^cis li al \^si.\\
                        \^Si nomis sin Heleno,\\
                        Anton' sin nomis li.

                        Ekamis la konato\\
                        Kaj reciproke \^si;\\
                        Post paso de monato\\
                        \^San\^gi\^gis "Vi" per "ci".

                        Someraj tri monatoj\\
                        Trapasis dol\^ce for,\\
                        Kaj niaj geamatoj\\
                        Jam estis kor' \^ce kor';

                        Kaj per solena beno\\
                        De l' pastro en la fin'\\
                        Antono kaj Heleno\\
                        Jam estis edz'-edzin'.

                        En \^carma harmonio\\
                        Ekvivis edz'-edzin';\\
                        Najbaroj kun envio\\
                        Rigardis lin kaj \^sin.

                        Sed balda\u u edzo estis\\
                        Plej malfeli\^ca hom',\\
                        Kaj jam tre ofte restis\\
                        Li ekster sia dom'.

                        Kaj fine eksedzi\^gis\\
                        Antono kaj Helen';\\
                        Belul' ia ali\^gis\\
                        Al \^si en la \^garden'.

                        Lin vidis en \^gardeno ---\\
                        Ekpla\^cis li al \^si.\\
                        \^Si nomis sin Heleno,\\
                        A\u ugust' sin nomis li.

                        Ekamis la konato\\
                        Kaj reciproke \^si;\\
                        Post paso de monato\\
                        \^San\^gi\^gis "Vi" per "ci".

                        Someraj tri monatoj\\
                        Trapasis dol\^ce for,\\
                        Kaj niaj geamatoj\\
                        Jam estis kor' \^ce kor'.

                        Kaj per solena beno\\
                        De l' pastro en la fin'\\
                        A\u ugusto kaj Heleno\\
                        Jam estis edz-edzin'.

                        En \^carma harmonio\\
                        Ekvivis edz'-edzin';\\
                        Najbaroj kun envio\\
                        Rigardis lin kaj \^sin.

                        Sed balda\u u edzo estis\\
                        Plej malfeli\^ca hom',\\
                        Kaj jam tre ofte restis\\
                        Li ekster sia dom'.

                        Kaj fine eksedzi\^gis\\
                        A\u ugusto kaj Helen';\\
                        Belul' ia ali\^gis\\
                        Al \^si en la \^garden'.

                        Lin vidis en \^gardeno ---\\
                        Ekpla\^cis li al \^si.\\
                        \^Si nomis sin Heleno,\\
                        Henrik' sin nomis li\\
                        K. t. p., k. t. p. sen fino.

%F. ZAMENHOF.
\end{verse}
\citsc{F. Zamenhof.}

\smallrule{}
