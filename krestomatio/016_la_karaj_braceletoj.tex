
   Estis en la plej gloraj tagoj de la unua franca imperio. Parizo estis
tre gaja. Festoj kaj baloj sekvis unu post la alia, kaj \^sajnis, ke
la stelo de la imperiestro brilas la plej hele, anta\u u ol \^gi por
\^ciam estingi\^gos. \^Cio, kion Parizo havis en si da brilo kaj
beleco, devis hodia\u u sin kolekti en la opero, \^car oni sciis, ke
la imperiestro intencas honori \^gin hodia\u u per sia alesto, kaj
pro tio la opera teatro estis plenigita de la plej brilanta Pariza
societo.

   La uverturo pasis; la imperiestro, akompanata de la imperiestrino
brilanta de beleco kaj diamantoj, \^{\j}us eniris en sian lo\^gion;
lia sekvantaro en uniformoj de \^ciuj koloroj de \^cielarko staris
en la posto de la lo\^gio. Post minuto la kurteno devis esti levita
kaj la opero komenci\^gi. Sed en tiu sama momento, kiam \^ciu
retenis la spiron en atendado, oni eka\u udis la bruon de vesto; la
dua lo\^gio, dekstre de la lo\^gio imperiestra, malfermi\^gis, kaj
eniris la belega edzino de la ambasadoro N. Ne miro, ke neniu plu
observadis la levadon de la kurteno; ne miro, ke \^ciu okulo kiel
ensor\^cita direkti\^gis sur la virinon, kiu \^{\j}us okupis sian
se\^gon kaj trankvile, kun aristokrata mal\^sateco, rigardis
\^cirka\u uen, \^car sur \^siaj brakoj, radiante kiel fulmo, brilis
braceletoj, pri kiuj Parizo jam tiom multe a\u udis kaj kiujn la
imperiestro vane provis a\^ceti. Murmuro de admiro kuris tra la
teatro, kaj nur poste oni sin turnis al la agado sur la sceno. Kiam
la kurteno post la unua akto falis, unu servanto en imperiestra
livreo eniris en la lo\^gion de la ambasadoro N.

 --- \^Sia Imperiestra Mo\^sto rimarkis la braceletojn kaj estis frapita
de admiro; \^si demandas, \^cu sinjorino la dukino volos havi la
bonecon permesi al la imperiestrino rigardi unu el ili de proksime?

   En momento la bela brako estis nudigita de la multekosta\^{\j}o, kaj
kun ekkrio de ravi\^go la imperiestra servanto salutis kaj eliris el
la lo\^gio, portante kun si la braceleton, kiu kostis pli ol
milionon da frankoj.

   La kurteno falis post la tria akto, levi\^gis denove al la kvara,
kaj \^ciam ankora\u u la edzino de la ambasadoro kun boneduka
\^gentileco atendis la redonon de siaj multekostaj juveloj. La
kortego imperiestra sin levis kaj foriris, kaj \^ciam ankora\u u la
braceleto ne estis redonita.

   La duko fine fari\^gis malpacienca, veturis en la imperiestran palacon
kaj petis pri la redono de la diamantoj. La afero klari\^gis, kaj la
duko konvinki\^gis, ke la imperiestrino neniam sendis peti la
braceleton kaj ke la homo en la imperiestra livreo estis unu el la
plej kura\^gaj \^stelistoj de la \^cefurbo. Li ordonis al sia
veturigisto veturi al la estro de la polico, kaj, anta\u u ol la
nokto pasis, centoj da plej lertaj policaj oficistoj traser\^cis la
tutan Parizon pro la \^stelitaj juveloj. La duko, plena de timo,
dume restis en la polica oficejo, dum la dukino en la domo
maltrankvile atendis la reporton de \^sia braceleto.

   \^Jus batis la sesa horo, kiam \^ce la pordo de la palaco de la duko
forte eksonis la sonorilo kaj unu polica oficisto deziris paroli kun
la dukino. Profunde salutante, tiu \^ci rakontis, ke oni kaptis la
\^steliston kaj trovis \^ce li la braceleton. Sed la fripono
persistas, ke li ne estas \^stelisto, kaj la braceleto jam de multaj
jaroj sin trovas en posedo de lia familio. Tial la duko petas
sinjorinon la dukinon, ke \^si transsendu al li la duan braceleton,
por ke oni povu kompari la amba\u u.

   Ne dirante vorton, la dukino malfermis sian juvelujon kaj donis al la
policisto la duan braceleton. Tiu \^ci kun profunda saluto forlasis
la \^cambron, kaj la sinjorino iris dormi kaj son\^gi pri siaj
braceletoj.

   Kiam la horlo\^go batis la na\u uan horon, la ambasadoro, laca de
maldormo kaj en malbona humoro, venis en la \^cambron de sia edzino
kaj \^{\j}etis sin malespere sur se\^gon. La sinjorino malfermis la
okulojn kaj kun gaja rideto demandis pri siaj braceletoj.

 --- Malbenita bando! ekkriis la duko, ni nenion povas scii\^gi pri \^gi.

 --- Kio! ekkriis la sinjorino, \^cu vi \^gin ne ricevis returne? La
oficisto, kiu prenis la duan braceleton, diris, ke la \^stelisto
estas kaptita kaj la braceleto trovita \^ce li.

   La duko eksaltis kun krio de teruro kaj petis sian edzinon kun ra\u uka
vo\^co, ke \^si donu al li klarigon.

   \^Si \^gin faris, kaj \^gemegante la duko falis sur la se\^gon.

 --- Mi \^cion komprenas! li ekkriis. La friponoj \^stelis de vi anka\u u
la duan braceleton, \^car ni neniun sendis tien \^ci. La homo, al
kiu vi \^gin donis, ne estis oficisto, sed ankora\u u pli aroganta
\^stelisto, ol la unua.

   Kaj tiel \^gi efektive estis. La braceletoj neniam estis retrovitaj,
kaj nur per \^gemo ofte la ambasadoro rememoradis la paradan operon,
kiu faris lin je kelkaj milionoj malpli ri\^ca, ol li estis, kiam li
sian belegan edzinon levis en la kale\^son kaj ordonis al la
veturigisto veturi al la opero.

\smallrule{}
