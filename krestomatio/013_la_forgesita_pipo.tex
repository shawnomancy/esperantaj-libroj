\begin{center}
\footnotesize Rakonto de A. \fsc{Edelmann}. Tradukis I. \fsc{Kaminski}.
\end{center}

   En unu el la pluvaj tagoj de Marto per grandaj pa\^soj iris sur la
kota strato de la vila\^go Vindirkovo unu vila\^gano, havanta la
a\^gon de \^cirka\u u tridek jaroj; sed \^sajnis, ke li estas multe
pli maljuna. Li portis dis\^siritajn lankitelon kaj \^cemizon, tra
kiuj elrigardis la malgrasa brusto; lia piedvesto ne estis pli bona
--- unu fliko sur alia; el la truaj plandoj estis vidata la pajla
substerno; sur la kapo maldece sidis dis\^sirita, grasmakulita lana
\^capo. Lia bruna senforma malsobra viza\^go, sur kiu ankora\u u
restis signoj de anta\u ua beleco, estis tuta kovrita per ru\^gaj
\^svelaj aknoj; la okuloj estis ru\^gaj kaj elstarantaj; la kapo jam
brilis en kelkaj lokoj per griza\^{\j}o\dots Per unu vorto, jam la\u
u la sola ekstera\^{\j}o de la vila\^gano oni povis ju\^gi pri la
dibo\^ca vivo, kiun li kondukis.

   Kaj efektive, granda drinkanto estis tiu \^ci vila\^gano, Doroteo
Nilov. Jen pasis jam pli ol jaro, ke li komencis drinkadi; sed en
tiu \^ci jaro li jam havis tempon fordrinki sian tutan havon kaj la
havon de sia edzino. Lian domon oni anta\u u nelonge vendis publike
por depagaj mankorestoj. Ne dezirante prilabori mem la pecon da
tero, kiu restis \^ce li, li farmigis \^gin, kaj la farman monon li
fordrinkis en unu semajno.

   De frua mateno \^gis malfrua nokto li pasigadis la tempon en la
drinkejo, kaj \^cion, kio enfalis en liajn manojn, li forportadis al
la drinkejmastro Vavilo Panfilovi\^c, kiu pro sia boneco ne abomenis
e\^c \^cifonon. Nia Doroteo Nilov anka\u u estis malbona edzo kaj
patro. Li neniam pensis pri sia malsatanta familio, kaj lia kvieta,
bona edzino Malanja devis akiradi la mizeran \^ciutagan
nutra\^{\j}on por si, sia edzo kaj infano --- dujara bubeto --- per
kudrado, lavado a\u u ia alia laborado. Sed Doroteo ne sole ne estis
danka al \^si por tio, sed batadis \^sin preska\u u \^ciutage,
revenante hejmen malsobre. Pri sia infano li tute ne zorgis kaj, oni
povas diri, tute \^gin ne konis.

   Ankora\u u anta\u u nelonge Doroteo Nilov estis kalkulata la plej ri\^ca
mastro en Vindirkovo. Li havis vastan dometon, multon da tero, bruto
kaj \^cia havo. Sian edzinon Malanja'n, kiu fra\u uline estis la
plej ri\^ca kaj plej bela fian\^cino en la tula vila\^go, li amis
pasie. Laborante fervore, li vivis kviete kaj estis estimata de
\^ciuj Vindirkovanoj. Sed "nek eterne da\u uras la \^gojoj, nek
senfinaj estas la mal\^gojoj", diras rusa proverbo. La \^gojoj
pasis anka\u u \^ce nia Doroteo, sed li mem estis la ka\u uzo de tio
\^ci: li amiki\^gis kun malbonaj kolegoj kaj komencis drinkadi,
dibo\^cadi, kaj \^cio transturni\^gis la fundo supren. Post unu jaro
\^ce Doroteo jam nenio restis --- \^cio transiris al la
drinkejmastro Vavilo.

   Nun ni vidas Doroteon, sin direktantan al sia kaduka dometo,
kiu staris sur la rando mem de la vila\^go, modeste alpreminte sin
al la plektobaro. Lia viza\^go esprimis \^cagrenon. Kaj efektive,
ekzistis ka\u uzo por \^cagreno: li devis forlasi en la drinkejo la
gajan drinkantaron kaj foriri hejmen, por preni la pipon, kiun li
tie forgesis sur la breto\dots

   La dometo estis fine atingita, kaj li eniris. La mastrino forestis;
\^si foriris ien al najbaro, por preni prunte kelke da terpomoj por
sia infano.

   La interna\^{\j}o de la dometo konsistis el malgranda kaverno, kies tri
kva\-ro\-nojn okupis granda rusa forno. Duo da makulitaj se\^goj,
tablo kun diskrevinta tabulo, malgranda kesteto, en kiu estis
konservata la tuta havoresto de la malfeli\^culoj, kaj tre malgranda
malnova sanktfiguro --- jen estas la tuta ornama\^{\j}o de tiu \^ci
kaverno. Krom tio estis alligita al la plafono lulilo, en kiu la
infano de Doroteo ripozis per trankvila dormo. Sur \^cio ku\^sis la
sigelo de malri\^ceco, tamen \^cio estis pura kaj bonorda kaj sur
\^cio estis vidata la mano de bona kaj zorgema mastrino.

   Enirante, Doroteo rekte sin direktis al la breto, prenis la pipon,
sed anka\u u kunprenis la hakilon, por \^ce la okazo forporti en la
drinkejon, kaj li estis jam kaptinta la fermilon de la pordo, por
eliri, sed en tiu \^ci momento eksonis en la \^cambro sonora infana
rido. Doroteo rapide sin returnis\dots En la lulilo, eltirante la
manetojn, nur en \^cemizo sole, staris blonda infano, vekita de la
pezaj pa\^soj de la patro. Sur la malgrandegaj lipetoj ludis \^goja
rideto, kaj gaje balancante la vilan kapeton, la infano babiladis:
"Patreto! Pa\^cjo!"\dots

   Io ekmovi\^gis en la malmola animo de Doroteo. Malrapide li aliris al
la lulilo, elprenis per siaj maldelikataj manoj la infanon kaj
ekkisis \^gin. La infano \^cirka\u uprenis per siaj manetoj la kolon
de la patro kaj alglui\^gis al lia brusto, gaje krietante kaj
sen\^cese babilante: "Patreto! Pa\^cjo!", la solajn vortojn, kiujn
\^gi jam povis libere elparoli. Doroteo ekkisis la infanon ankora\u
u unu fojon kaj volis ku\^sigi \^gin ree en la lulilon. Sed la
infano ne deprenis siajn manetojn kaj sendeturne rigardadis la
patron per siaj klaraj nigraj okuletoj. Denove io ekpikis la koron
de Doroteo, kvaza\u u la rigardo de la senkulpa infano trabruligus
lin trae kaj disfluidigus la glacion, per kiu estis kovrita lia
koro. Al li \^sajnis, ke la infano rigardas lin ripro\^ce kaj ke
\^gi kvaza\u u volas diri per tiu \^ci rigardo: --- Patreto! vi
volas forlasi min, por foriri en la drinkejon, fordrinki la lastan
hakilon, tute ne zorgante pri la malfeli\^ca panjo, kiu nun
malaltigas sin \^ce la najbaroj\dots Kaj kion donas al vi la
malbenita drinkado?

   Li mem iel nevole ekrigardis en la okulojn de la infano\dots kaj
ektremis, kvaza\u u en tiu \^ci spegulo de la senkulpeco kaj pureco
li ekvidis sian vivon en \^gia tuta abomeninda nudeco.

   Malvarma \^svito montris sin sur lia viza\^go; lia koro forte ekfrapis.
Li estis kiel en febro: la kapo al li brulis kaj tra la korpon
trakuris tremfrosto, liaj piedoj ne volis lin teni kaj la tero
kvaza\u u \^sanceli\^gis sub ili. En tiu \^ci minuto Doroteo sentis
teruran internan bataladon. Li deziregis denove foriri en la
drinkejon, por dibo\^ci kun siaj gajaj senzorgaj kolegoj, sed la
konscienco, vekita de la infana rigardo, kvaza\u u alfor\^gis lin al
la tero, al la loko, kie li haltis\dots En tiu \^ci minuto la pordo
malfermi\^gis kaj eniris Malanja, portante basta\^{\j}on kun
terpomoj.

   Kiam \^si ekvidis sian edzon, starantan kun la infano sur la manoj,
\^si pretervole haltis sur la sojlo, kiel alfor\^gita. Dume li
pretervole ekrigardis \^sin kaj ekvidis tion, kion li anta\u ue tute
ne rimarkadis, tiel bone, tiel klare, kvaza\u u liaj okuloj subite
malfermi\^gis. Li ekvidis \^sian delikatan, malgajan viza\^gon,
\^siajn grandajn, belajn okulojn, \^siajn palajn, velkintajn lipojn
kaj la vangojn, kiuj anta\u u nelonge estis ru\^gaj kiel papava
floro. Li ekvidis \^cion \^ci, kaj lia viza\^go ekru\^gi\^gis de
honto. Sed de tiu \^ci momento \^cesi\^gis la interna batalo en lia
animo, li subite eksentis ian abomenon kontra\u u si mem kaj iel
instinkte komprenis, ke li nun por nenio en la mondo revenus en la
drinkejon. En atako de \^gojo kaj konfuzo, kun koro trankvila, sed
konsumata de amo kaj kompato, li rigardis la elturmentitan viza\^gon
de Malanja kaj ne sciis, kion diri.

 --- Bonan tagon, Malanjeto, kial vi ektimi\^gis? fine apena\u u elparolis
Doroteo kaj \^ce tiuj \^ci vortoj ekrigardis la edzinon per tia
rigardo, kiu rememorigis al la malfeli\^ca virino la pasintan
feli\^can tempon. \^Si komprenis, ke \^si nun ne devas lin timi;
forte, sed \^goje ekfrapis \^sia koro, kaj klara rideto eklumigis
\^sian viza\^gon.

   Dume Doroteo aliris al \^si, ekkisis \^sin kaj poste, montrante per la
okuloj la infanon, li ekmurmuris:

 --- Li estos bona bubo. Li ne ektimi\^gis anta\u u mi, Malanjeto. Li
estos bubo-bravulo! Kaj denove li \^cirka\u uprenis per la dekstra,
libera mano la edzinon, altiris \^sin al si kaj malla\u ute diris al
\^si:

 --- Mi scias, Malanja, ke mi multe estas kulpa anta\u u vi\dots sed nun
estas fino al \^cio, Malanja\dots Neniam pli\dots Mi \^{\j}uras, ke
mi pli ne drinkos\dots \^Cu vi a\u udas, Malanja\dots neniam,
neniam\dots La infano estu atestanto, ke mi \^gis la morto ne prenos
en la bu\^son tiun malbenitan trinka\^{\j}on\dots

   \^Si estis pala, kiel tolo, kaj ne povis elparoli e\^c unu vorton; \^si
nur \^cir\-ka\u u\-pre\-nis Doroteon kaj forte, forte ekkisis lin.

 --- Mi estas peka anta\u u vi, Malanja; mi devas peti vin kaj stari sur
la genuoj anta\u u vi, \^car mi, malbenita, pereigis, tre
malfeli\^cigis vin. Kaj li efektive, ankora\u u tenante la infanon,
mallevis sin anta\u u la edzino sur la genuojn kaj forte klini\^gis
anta\u u \^siaj piedoj. Kaj li tiel longe ne levis sin, \^gis \^si,
la\u ute ploregante, levis lin kaj sidigis lin sur la benkon.

 --- Ha, mi malsa\^gulo\dots malhonorulo\dots granda malsobrulo\dots
 parolis Do\-ro\-te\-o, sidante sur la benko kaj direktinte siajn okulojn
al la planko. Jen kion ka\u uzis al mi la drinkado! Vi nur pardonu
min, kaj mi \^{\j}uras per Dio, mi \^cesos drinki! Li, li estos
atestanto! La lastajn vortojn li elparolis, montrante per unu mano
la infanon kaj la duan manon kunpreminte en pugnon. Malanja ploris;
\^si forte beda\u uris sian edzon.

 --- Mi pardonas vin, kara; Dio kaj la homoj anka\u u vin pardonos. Nur
\^cesu, Do\^cjo, drinki la malbenitan vinon; \^gi ja pereigis vin!

 --- Ho, ho, mi, malfeli\^culo, ne kvieti\^gis Doroteo. Kial mi vin
senkulpan pereigis!\dots Mi nun nenion havas, nek \^cevalon, nek
bovinon\dots Kie estis mia konscienco!

   Kaj Doroteo la\u ute ek\^gemis. Longe konsoladis lin Malanja, kiel \^si
povis. Sed li tute trankvili\^gis nur en la vespero.

   Kaj longe ankora\u u ili parolis inter si pri tio, kiel komenci
novan vivon, kaj nur tiam, kiam la vila\^ga sonorilo malgaje eksonis
noktomezon, ili sin levis de apud la tablo. Varmege ili pre\^gis
anta\u u sia sanktfiguro, kaj kun feli\^caj viza\^goj, sur kiuj
estis esprimita ia an\^gela kvieteco, ili iris dormi\dots

   Tiel fini\^gis la feli\^ca tago de niaj malri\^culoj.

\asterism{}

   Pasis kelkaj jaroj. Sur tiu loko, kie staris anta\u ue la mizera dometo
de Doroteo, staris nun vasta pina domo kun peroneto. Apud la domo,
gaje babilante, ludis du malgrandaj infanoj, kaj sur la peroneto
sidis, okupita de kudrado, juna bela virino en ru\^ga sarafano.
Estis malfacile rekoni en \^si la anta\u uan malsaneman, palan
Malanja'n. \^Gi estis la familio de Doroteo, kiu severe plenumis la
donitan promeson kaj \^cesigis sian anta\u uan dibo\^can vivon.

   Kiel ofte rememoradis la geedzoj pri tiu feli\^ca tago, kiam Doroteo
forgesis en la domo la pipon kaj, por preni \^gin, venis el la
drinkejo, por jam neniam tien reveni.

\smallrule{}
