\begin{verse}
\begin{center}
\footnotesize (El \fsc{Lermontov.})
\end{center}
                  En mezo de nokto, en blua \^cielo,\\
                  Traflugis kaj kantis plej bela an\^gelo.\\
                  Kaj nuboj kaj steloj kaj lun' en irado\\
                  Atentis kun \^gojo je l' sankta kantado.

                  Li kantis feli\^cajn, neniam pekantajn\\
                  Spiritojn, kun Di' en \^cielo lo\^gantajn;\\
                  Li kantis pri Patro \^ciela kaj tera ---\\
                  Kaj lia la\u udado ne estis malvera.

                  Animon tre junan en brakoj li tenis;\\
                  En mondo malgaja naski\^gi \^gi venis.\\
                  Kaj sono de l' kanto en juna animo\\
                  Restadis sen vort', sed kun viva estimo.

                  Tre longe en mondo \^gi estis premata,\\
                  Je revo mirinda pri Di' plenigata!\\
                  Kaj \^san\^gi la kanton de l' sankta sincero\\
                  Ne povis por \^gi \^ciuj kantoj de l' tero.

%V. DEVJATNIN.
\end{verse}

\citsc{V. Devjatnin.}

\smallrule{}