\titleformat{\chapter}[display]{\centering\fontspec{Arial}[LetterSpace=5]}{\chaptertitlename}{0pt}{\LARGE}
\chapter*{KOMENTO PRI KOMPOSTI.}

\begin{center}
\emph{Jen versio \laversio{} de ĉi tiu} \XeLaTeX{} \emph{aranĝaĵo.}
\end{center}

Mi unue volas danki Einstein G. dos Santos pro la kompleza helpo, en la kreado de tiu ĉi \LaTeX{}-a aranĝaĵo.  S-ro dos Santos bonkore donis sian originalan \LaTeX{}-an fontkodon, farita el la enkomputiligita teksto de David Starner kaj William Patterson kun Project Gutenberg.  La teksto estas laŭ la dua eldono de la \emph{Fundamenta Krestomatio}.

Provante imiti la fruan 20a-jarcentan tipografion laŭ la originala \emph{Krestomatio}, certe mi povus enplekti multajn erarojn, kaj ĉi tiujn estos sole la miajn.

Mi provas imiti la originalan duan eldonon tiel proksime kiel eble, sed mi restigis la similaĵon de D-ro Zamenhof, kiu estis en la aranĝaĵo de S-ro dos Santos.  Ankaŭ al li, ni devas doni niajn plej elkorajn dankojn.

Finfine, mi volas danki miajn amikojn, kiuj deziras lerni pri Esperanto.  Por ili pli ol iujn ajn, mi dediĉas la aranĝaĵon.

\vspace{1em}

{\setlength{\parindent}{0em}
Shawn KNIGHT (angle elparolata \emph{ŝan najt})\\
\hodiau}

\newpage
\thispagestyle{empty}
\vspace*{\fill}
\begin{center}
Ĉi tiu verko estas permesita per Creative Commons \\
Attribution-NonCommercial-ShareAlike 4.0 \\
Internacia Permesilo.

La originala verko de D-ro Zamenhof \\
estas senkopirajta.\\[1ex]

\ccbyncsa\\[1ex]

This work is licensed under a Creative Commons \\
Attribution-NonCommercial-ShareAlike 4.0 \\
International License.

The original work by Dr. Zamenhof \\
is in the public domain.
\vspace*{\fill}
\end{center}