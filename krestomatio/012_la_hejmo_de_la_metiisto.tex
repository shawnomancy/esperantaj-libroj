\begin{center}
\footnotesize (De Nov-Jorka kuracisto. Rerakontita de E. Weilshäuser).
\end{center}

   Unu vesperon, en la unua parto de la vintro, la sonorilo estis
forte ektirita, kaj la servantino, kiu malfermis la pordon, raportis
al mi, ke ia homo deziras paroli kun mi. Servanto faras certajn
diferencigojn inter "homo", "sinjoro" kaj "persono". La homo
staris en la anta\u u\^cambro, sed mi miris, kial li ne estis
raportita kiel "sinjoro". Lia vesto estis tre pura, sed simpla,
kaj el ne tre delikataj \^stofoj. Lia supra \^cemizo, tiu \^ci signo
de pli delikata stato, estis blanka, en plena ordo kaj preska\u u
eleganta. \^Cio en li montris certan solidecon, sed nenio prezentis
al mi klarigon pri lia situacio en la vivo. La\u u sia
ekstera\^{\j}o li \^sajnis konvena homo. Lia parolado estis simpla,
klara, rekta kaj kun certa nuanco de memfideco, kion oni \^ce simpla
homo ordinare tre malofte trovas.

 --- Sinjoro doktoro, li diris, mi volus vin peti veni al mia infano;
ni timas, ke al \^gi minacas atako de krupo.

   Mi prenis mian \^capelon kaj premis lin anta\u uen, \^car, se lia supozo
estus \^gusta, oni tie \^ci ne devis perdi tempon. \^Ce tiu \^ci
malsano unu horo povas decidi inter vivo kaj morto --- Post minuto
ni estis sur la strato kaj pa\^sis rapide tra unu el niaj lar\^gaj
aleoj. La infano, li diris, ludis anta\u u la pordo, man\^gis kun
apetito sian vesperman\^gon, iris poste dormi, kaj anta\u u mallonge
\^gi veki\^gis tre ra\u uka kaj kun sufoka tusado. Tiu \^ci okazo
prezentis preska\u u nenian dubon, kaj mi rapidigis miajn pa\^sojn
kaj post kelkaj minutoj mi atingis kun mia akompananto la pordon de
la domo. Ni levadis nin \^ciam pli kaj pli alte \^gis la kvara
eta\^go. La lasta eta\^go de la \^stuparo estis kovrita per
tapi\^soj kaj de supre lumis al ni malgranda lampo. Anta\u u la
pordo ku\^sis bonega kaj tre fortika mato. Vi poste vidos, kial mi
turnas atenton sur tiujn \^ci malgrandajn detala\^{\j}ojn. --- Mi
eniris en la malfermitan pordon kaj estis salutita de sufi\^ce bela,
pure kaj orde vestita virino, kiu ne povis esti iu alia, ol la
edzino de mia akompananto.

 --- Mi \^gojas, ke vi tiel balda\u u venis, \^si salutis min kun mola,
pura akcento. Vil\^cjo, kiel \^sajnas, estas tiel elturmentita, ke
li apena\u u povas spiri. Kaj post minuto, kiam ni aliris al lia
lito, mi a\u udis la senduban sonon de krupo, kiu tute prave
plenigis per tia timo la korojn de la gepatroj.

 --- \^Cu tio \^ci estas krupo, sinjoro doktoro? demandis la patro kun
ekscitita vo\^co, kiam mi klinis min super la infanon, belan
trijaran knabon.

 --- Sendube, tre serioza atako. De kiam la afero ek\^sajnis al vi
suspekta?

 --- Anta\u u \^cirka\u u unu horo, estis la respondo, en kiu sonis forteco
de karaktero. Mi ekrigardis la patrinon. \^Si estis tre pala, sed ne
kura\^gis paroli.

 --- En tia okazo estas kredeble nur tre malmulte da dan\^gero, mi
diris, sed ni devas tamen ion fari. \^Cu vi havas akvon en la domo?

   La viro aliris al unu kabineto, malfermis du pordojn, kaj oni
povis nun vidi belan banan kuvon, parte plenigitan de akvo. Tio \^ci
estis pli, ol mi kura\^gis esperi, sed mi ne havis tempon, por miri.
La malgranda knabo estis en forta febro kaj malfacile bataladis pro
spiro. Mi elprenis lin el lia malvarma lito, en kiu li ku\^sis sur
bela hara matraco, je kiu nenia princo bezonus honti, senvestigis
lin de liaj noktaj vestoj, starigis lin en la kuvon kaj ordonis al
lia patro ver\^si tri sitelojn da malvarma akvo super lian kolon kaj
bruston, dum mi per mia mano energie lin frotadis. Poste ni lin
sekigis kaj frotadis tiel longe, \^gis la tuta korpo fari\^gis flame
ru\^ga. Poste mi eltordis grandan vi\^silon, trempitan en malvarman
akvon, metis \^gin \^cirka\u u lian kolon kaj envolvis lin en lanajn
litkovrilojn. La malgranda brava knabo \^cion trankvile elportis,
kvaza\u u li komprenus, ke en apudestado de lia patro povus fari\^gi
al li nenio malbona. Post dekkvin minutoj aperis ri\^ca \^svitado,
la infano falis en sanan dormon kaj povis denove libere spiri. La
dan\^gero pasis; tiel rapide pasas tiu \^ci malsano kaj tiel facile
\^gi estas kuracata.

   La zorgoplena viza\^go de la patro beli\^gis, kaj super la viza\^gon
de la patrino kuris radio da feli\^co. Mi direktadis miajn rigardojn
de unu al la alia, kaj pli ol iam mi estis en dubo, kian pozicion en
la vivo ili povas okupi. Signoj de eminenta deveno a\u u pli alta
eduko ne estis videblaj; ili ne faris la impreson de bonklasaj
homoj, kiuj falis en malri\^cecon. Pli \^guste \^sajnis, ke ni havas
tie \^ci okazon kontra\u uan: \^sajnis pli \^guste, ke ili de pli
malalta \^stupo de la vivo levis sin al pli alta. Mi rigardis
\^cirka\u uen en la \^cambro, kiu okaze estis la dormo\^cambro.
\^Cio estis en plej bona ordo. La litoj estis puraj, sed simplaj, la
blanka stebita litkovrilo ne povis kosti pli ol 10 \^silingojn, sed
kiel bele \^gi elrigardis! La blankaj flankkurtenoj estis el malkara
muslino, sed iliaj faldoj pendis tiel dense, kvaza\u u ili estus el
damasko --- kaj kiel konvene ili elrigardis! La banan kuvon kun la
dense falditaj kurtenoj mi ne povis taksi pli ol 10 dolaroj. La
tualeta tablo de eleganta formo kaj tute kovrita estis sendube el
pina ligno kaj kostis duonon da dolaro. La pentra\^{\j}oj sur la
muro estis bone koloritaj litografa\^{\j}oj --- pli belaj, multe pli
belaj, ol oleaj pentra\^{\j}oj, kiujn mi vidis en domoj de
milionuloj: ili povis kosti po tri \^gis kvin \^silingoj kaj la
kadroj po unu dolaro. La planko estis kovrita per belaj tapi\^soj,
kaj la muroj estis de hela koloro. Per unu vorto, la \^cambro estis
tuta ornamujeto --- en \^ciuj partoj tiel en ordo, kvaza\u u artisto
\^gin aran\^gis.

   Lasante la knabon al trankvila dormado kaj donante la necesajn
klarigojn pri la bano post lia veki\^go, mi iris kun la gemastroj en
la duan \^cambron, kiu posedis alian, sed egale belan aran\^gon. Oni
povus \^gin preni por gasto\^cambro, por laborejo de artisto a\u u
por man\^go\^cambro. La muroj estis kovritaj per pentra\^{\j}oj ---
portretoj, historiaj skizoj kaj pejza\^goj --- \^cio pentra\^{\j}oj,
kiujn povis elekti nur homo kun gusto \^ce modestaj rimedoj, kiuj
tamen havas tiom same grandan indon, kiel bonaj libroj. Kaj
parolante pri tiuj \^ci lastaj, mi devas tie \^ci rimarki, ke apud
la kameno pendis malgranda biblioteko, kiu \^ce la unua rigardo
montris al mi, ke \^gi enhavas la plej elektitajn trezorojn de la
angla literaturo.

   La mastro iris al la skribotablo, malfermis unu fakon kaj elprenis
monon. --- Kiom mi \^suldas al vi, sinjoro doktoro? li demandis,
tenante prete la monon.

   Levante min sur la \^stuparon, mi pensis, ke mi devos atendi je mia
honorario, a\u u ke mi \^gin neniam ricevos, sed nun \^cio
\^san\^gi\^gis. Mi ne bezonis nun, kiel \^gi ofte okazas, scii\^gi
pri la pli detalaj cirkonstancoj kaj mezuri la\u u tio \^ci mian
postulon. La homo staris anta\u u mi preta, por pagi; tamen videble
li apartenis al la klaso de laboristoj kaj estis tre malproksima de
bonhaveco; mi tial povis difini al li nur la plej malaltan sumon.

 --- Unu dolaro \^sajnas al mi ne sufi\^ca, li respondis. Vi havis pli
grandan laboron, ol la solan skribadon de recepto.

 --- \^Cu vi laboras por via sintenado? mi demandis, esperante iom
malkovri la sekreton. Li ridetis kaj montris al mi sian manon, kiu
prezentis la signojn de honesta laborado. Vi estas metiisto, mi
diris kun la intenco scii\^gi pli multe.

 --- Prenu tion \^ci, li diris kaj metis kun nerifuzebla gesto dudolaran
banknoton en mian manon, kaj mi kontentigos vian scivolecon, \^car
vi ja ne povas ka\^si, ke vi estas iom scivola.

   En \^cio tio \^ci estis ia honesta, estiminda sincereco, kiu havis por
mi apartan \^carmon. Mi metis la banknoton en la po\^son, dum la
mastro iris al la pordo, malfermis \^gin kaj montris al mi kabineton
de meza grandeco, en kiu mi de la unua rigardo ekkonis metiejon de
botisto.

 --- Vi estas kredeble eksterordinara laboristo, mi diris,
\^cirka\u urigardante la \^cambron, kiu \^sajnis al mi preska\u u
lukse meblita, dum \^ce pli proksima observado de \^ciu objekto
fari\^gis al mi klare, ke \^gi ne povas multe kosti.

 --- Ne, vi eraras. Mi perlaboras nur iom pli ol unu dolaron en tago.
Mia edzino iom helpas. Krom la domaj laboroj kaj la zorgado pri la
infano \^si perlaboras tiom, ke nia semajna enspezo prezentas
mezonombre ok dolarojn. Ni komencis per nenio, kaj ni nun vivas
tiel, kiel vi vidas.

   Tiu \^ci komforto, tiu \^ci konvena aran\^go, kiu prezentis preska\u u
lukson, \^cio tio \^ci por ok dolaroj semajne! Mi esprimis mian
miregon.

 --- Mi estus en granda timo, se ni tiom devus elspezi, li rimarkis. Ni
kun tio \^ci ne sole vivis \^gis nun, sed ni havas anka\u u jam ion
en la \^spara banko. --- \^Cu vi volos esti tiel bona kaj klarigi al
mi, kiel vi tion \^ci faras? mi demandis, \^car mi efektive forte
volis scii\^gi, kiel botisto kun edzino kaj infano, perlaborante ok
dolarojn semajne, povas vivi en komforto kaj eleganteco kaj ankora\u
u kolekti monon.

 --- Kun plezuro, li respondis, \^car eble vi aliajn, kiuj ne estas en
pli bona stato, ol mi, povos konvinki, ke ili faru al si sian
situacion kiel eble plej oportuna.

   Mi prenis la donitan al mi se\^gon, kaj ni sidi\^gis, dum lia edzino,
a\u uskultinte la facilan kaj regulan spiradon de sia infano,
sidi\^gis, por kudri.

 --- Mia nomo estas William Carter, li diris. Mia patro mortis, kiam
mi estis ankora\u u juna, kaj, posedante ordinaran lernejan sciadon,
mi estis donita al botisto, por lerni. Mi estis granda amanto de
legado, kaj mian liberan tempon mi pleje uzadis tiel, ke mi
konati\^gadis kun la libroj el la biblioteko de la metiolernantoj.
La plej multe pla\^cis al mi la plenaj je sa\^go de la vivo verkoj
de W. Cobbet. Mi decidis sekvi lian ekzemplon kaj labori, kiom mi
povas, por mia kleri\^go, kaj mi pensas, ke mi ne vane laboris. Sed
la edukado de la homo da\u uras la tutan vivon, kaj ju pli mi
lernas, des pli multe mi vidas, ke mi devas aukora\u u lerni.

   Mi estis ankora\u u de nelonge submajstro, kiam mi enami\^gis en mian
Marion tie, kiun multaj nomas tre bela, kiun mi tamen ekkonis kiel
tre bonan.

   Mario ekrigardis kun tia gaja, \^carma rideto, ke la opinio de multaj
\^sajnis al mi tute prava.

 --- Laborinte unu jaron kiel submajstro kaj \^sparinte kelkajn dolarojn
(mi havis gravan ka\u uzon por \^spari), mi edzi\^gis kun mia Mario.
Mi lo\^gis \^ce \^sia patro, kaj \^si \^cirka\u ukudradis \^suojn
por la magazeno, por kiu mi laboradis. Tiel ni vivis kelkajn
semajnojn en la domo de \^siaj gepatroj, sed tio \^ci ja ne estis
nia propra hejmo, kaj tiel ni decidis aran\^gi propran
mastra\^{\j}on. Ni povis pensi nur pri modesta lo\^gejo, kaj mi
tutan semajnon ser\^cis tian, \^car jen lo\^gejo estis por mi tro
kara, jen tro mizera. Fine mi trovis tiun \^ci lo\^gejon. \^Gi estis
nova kaj pura, alta kaj kun bona aero, kaj mi pensis, ke estus
oportune lo\^gi en \^gi. Mi luis \^gin por kvindek dolaroj por jaro,
\^car kvankam la lo\^gejoj \^cirka\u u ni \^ciuj estas pli karaj,
nia mastro kontenti\^gas je tiu \^ci sumo, \^car li estas kontenta
je ni kiel luantoj. La lo\^gejo tiel jam estis, sed \^gi estis
malplena, kaj krom ni mem ni havis nur tre malmulte por enporti, sed
ni gaje komencis labori, ni penadis perlaboradi kiel eble pli multe,
ni \^sparis, kiom ni povis, --- kaj vi vidas, kion ni atingis.

 --- Mi vidas, sed mi ne komprenas, mi diris kun la intenco a\u udi ion
pri la mastrumado de tiu \^ci modesta kaj bela hejmo.

 --- Nu, tio \^ci estas sufi\^ce simpla. Kiam mi kun mia edzino tie \^ci
enlo\^gi\^gis kaj ni okupis nian lo\^gejon kun unu tablo, du
se\^goj, unu forno por kuirado, unu a\u u du patoj kaj unu
mallar\^ga lito kun pajla matraco, ni faris konsili\^gon de milito.
Nu, kara Mario, mi diris, ni estas tie \^ci; dume ni havas ankora\u
u nenion kaj devas \^cion ankora\u u akiri, kaj por tio \^ci ni
povas kalkuli nur je niaj dek fingroj.

 --- Ni trovis, ke ni povas mezonombre perlabori ok dolarojn. Ni tial
decidis vivi tiel malkare, kiel nur eble, \^spari, kiom ni nur
povos, kaj kiom eble --- \^cion fari mem. Nia pago por la lo\^gejo
prezentis la sumon de unu dolaro semajne, nia brula materialo,
lumigo, uzado de akvo kaj aliaj bagateloj ankora\u u unu dolaron.
Tian saman sumon ni difinis por vestoj, kaj per tio, ke ni a\^cetas
la plej bonajn \^stofojn kaj penas konservi la vestojn, kiom ni nur
povas, ni povas \^ciam esti bone vestitaj. E\^c mia edzino estas
kontenta je sia vestaro kaj trovas, ke kruda silko po 6 \^silingoj
por ulno estas kompare kun la tempo de uzado pli malkara, ol
kalikoto po 1 \^silingo. Tio \^ci faras 3 dolarojn semajne, kaj nun
restas ankora\u u la man\^ga\^{\j}oj. Por tio \^ci mi kalkulas por
ni tri unu dolaron semajne. --- Tio estas unu dolaron por \^ciu
persono? --- Ne, unu dolaron por ni \^ciuj. \^Sajnas, ke vi estas
surprizita, sed la kalkulo estas \^gusta.

 --- En la komenco ni elspezadis por tio \^ci pli multe, sed kun la
tempo ni lernis vivi pli bone kaj pli malkare, tiel ke post la
pagado de \^ciuj elspezoj por lo\^gejo, hejtado, lumigado, vestoj,
akvo kaj man\^ga\^{\j}o restas ankora\u u pura superfluo de 4
dolaroj semajne. Mi ne kalkulis kompreneble flankajn elspezojn,
ekzemple se ni iam vizitas la teatron a\u u koncerton, a\u u havas
gastojn \^ce ni.

   Rimarkinte rideton sur miaj lipoj, li da\u urigis: --- Jes, ni akceptas
anka\u u gastojn kaj amuzas nin \^ce tio \^ci tre bone. Iafoje ni
havas dekon da gastoj, kaj ilia regalado per \^cokolado, kukoj k. t.
p. kostas al ni ne pli ol 2 dolarojn; sed tio \^ci anka\u u ne tro
ofte okazas. Tiel al ni en \^cia okazo restas ankora\u u 200 dolaroj
jare, por kiuj ni nin aran\^gis, kiel vi vidas, kaj rezervis
kapitaleton en la \^spara banko.

 --- Mi vidas \^cion tion \^ci, mi diris, nur ne kiel vi vivas. Multaj
metiistoj elspezas pli ol tion \^ci por cigaroj, ne parolante jam
pri drinkoj. Mi petas vin, rakontu al mi, kiel vi vivas. --- Kun
plezuro. Mi ne fumas cigarojn, anka\u u ne ma\^cas tabakon kaj nek
mi nek mia edzino flaras tabakon.

   La edzino eniris kun agrabla rideto, sed ne interrompis nin, \^car \^si
pensis, kiel \^sajnas, ke \^sia edzo ja anka\u u sen \^sia helpo
komprenas paroli.

   De la tago de mia edzi\^go mi uzis nenian alkohola\^{\j}on, nur kvar
fojojn en jaro mi trinkas glaseton da vino, kaj nome: en la festo de
Kristonasko, en la Nova Jaro, la 4-an de Julio \footnote{La jartago
de la sendependi\^go de la Nord-Amerikaj \^Statoj.} kaj en la
naskotago de nia filo. Tiu \^ci tago estas por ni aparta festo. Mi
legis sufi\^ce da higienaj libroj kaj scias, ke teo kaj kafo estas
narkotikaj trinka\^{\j}oj sen nutra enhavo, kaj mi sufi\^ce longe
esploris la man\^gon kreska\^{\j}an kaj mi lernis dece estimi \^gin;
mi ekkonis \^gian pli grandan indon en komparo kun la dieto miksita
kaj trovis, ke \^gi pli bone servas al mi, ol tiu \^ci lasta, kaj
\^car ni kune legas kaj eksperimentas, tial mia edzino pensas tiel
same, kiel mi.

 --- Sed kion do vi man\^gas kaj trinkas? mi demandis scivole, por
scii\^gi, kiel malproksime la memedukita filozofo progresis en la
le\^goj de higieno.

 --- Venu, kaj mi montros al vi, li diris, prenante kandelon kaj
kondukante min en vastan provizejon. Jen vi vidas anta\u u \^cio
muelilon, kiu kostas al mi 10 \^silingojn. \^Gi muelas mian tutan
grenon, donas al mi la plej fre\^san kaj belan farunon kaj \^sparas
al mi la monon por muelado. Jen estas bareleto da tritiko. Mi
a\^cetas la plej bonan, kaj mi estas certa, ke \^gi estas pura kaj
bona. \^Gi kostas malpli ol 3 cendoj por funto, kaj funto da tritiko
estas sufi\^ca, kiel mi scias, por la taga nutrado de homo. Ni
preparas el \^gi panon, kukon kaj ka\^con. Jen estas sako da
terpomoj. Jen estas maizo. Jen estas faboj, kesto da rizo, tapioko
kaj makaronoj. Jen estas bareleto da plej bonaj pomoj. Jen skatolo
kun sukero kaj jen nia poto butera. Ni prenas \^ciutage kvarton da
vila\^ga lakto, kaj niajn ceterajn man\^ga\^{\j}ojn mi a\^cetas tie,
kie mi ilin ricevas la plej bone kaj plej malkare. Se vi kalkulos,
vi facile trovos, ke unu dolaro semajne por tiaj rimedoj de nutrado
ne sole estas sufi\^ca, sed ke al ni estas permesita \^ce tio \^ci
sana kaj preska\u u luksa malsameco. Ekster tio \^ci ni man\^gas
verdajn legomojn, fruktojn kaj berojn, kiujn alportas \^ciufoje la
responda jartempo. En la somero ni havas fragojn kaj persikojn, kiam
ili estas maturaj kaj bonaj; el \^ciuj tiuj \^ci simplaj materialoj
mia edzino preparas man\^gojn, kiujn mi certe preferas al la
man\^goj el la kuirejoj de la plej bonaj hoteloj.

   Mi a\u udis sufi\^ce. Tie \^ci mi trovis komforton, prudentecon, guston
kaj modestan lukson \^ce simpla metiisto, kiu komprenis vivi kun
malgrandaj elspezoj. Kiom da senbezonaj plendoj povus esti evititaj,
kiom da \^cagreno kaj suferoj povus esti forigitaj, se \^ciuj
laboristoj vivus tiel prudente, kiel William Carter. Neniam mi
premis al iu la manon kun pli sincera estimo kaj koreco, ol tiam,
kiam mi diris "bonan nokton" al tiu \^ci feli\^ca paro, kiu en tiu
\^ci kara urbo kun 8 dolaroj de semajnaj enspezoj vivas lukse kaj
ri\^ci\^gas kaj levas la benketon de botisto al la indo de instrua
se\^go de praktika filozofo.

   Leganto, se vi volas profiti de tiu \^ci malgranda historio, sekvu nur
la vortojn de la Biblio: "iru kaj faru tion saman."

\smallrule{}
