Prezentante pure kondi\^can rimedon de reciproka komuniki\^gado, la
lingvo internacia, simile al \^ciu lingvo nacia, povos bone atingi
sian celon nur en tiu okazo, se \^ciuj uzos \^gin plene {\it egale};
kaj por ke \^ciuj povu uzi la lingvon egale, estas necese, ke
ekzistu iaj {\it modeloj}, le\^gdonaj por \^ciuj. Tio \^ci estas la
ka\u uzo, pro kiu, cedante al la peto de multaj esperantistoj, mi
eldonis la {\it Fundamentan Krestomation}, kiu povos servi al \^ciuj
kiel modelo de esperanta stilo kaj gardi la lingvon de pereiga
disfalo je diversaj dialektoj.

   Lerni la lingvon \^ciu povas la\u u \^ciuj libroj, kiujn li deziros; sed
\^car multaj esperantaj libroj estas verkitaj de personoj, kiuj
ankora\u u ne posedas bone la lingvon Esperanto, kaj komencanta
esperantisto ne povus rilati al ili sufi\^ce kritike, tial estas
dezirinde, ke \^ciu, anta\u u ol komenci la legadon de la esperanta
literaturo, tralegu atente la {\it Fundamentan Krestomation}. Ne
deprenante de la lernanto la eblon kritike proprigi al si \^ciujn
ri\^cigojn kaj regule faritajn perfektigojn, kiujn li trovas en la
literaturo, la {\it Fundamenta Krestomatio} por \^ciam gardos lin de
blinda kaj senkritika alproprigo de stilo {\it erara}.

   Atentan tralegon de la {\it Fundamenta Krestomatio} mi rekomendas al
{\it \^ciu}, kiu volas skribe a\u u parole uzi la lingvon Esperanto.
{\it Sed precipe atentan kaj kelkfojan trategon de tiu \^ci libro mi
rekomendas al tiuj, kiuj deziras eldoni verkojn en Esperanto}; \^car
tiu, kiu eldonas verkon en Esperanto, ne koni\^ginte anta\u ue
fundamente kun la spirito kaj la modela stilo de tiu \^ci lingvo,
alportas al nia afero ne utilon, sed rektan malutilon.

   \^Ciuj artikoloj en la {\it Fundamenta Krestomatio} estas a\u u skribitaj
de mi mem, a\u u --- se ili estas skribitaj de aliaj personoj ---
ili estas korektitaj de mi en tia grado, ke la stilo en ili ne
deflanki\^gu de la stilo, kiun mi mem uzas.

\begin{flushright}
\begin{minipage}{5cm}
\begin{center}
L. \fsc{Zamenhof},\\
\footnotesize A\u utoro de la lingvo {\it Esperanto}.
\end{center}
\end{minipage}
\end{flushright}
\begin{flushleft}
\small Varsovio, en Aprilo 1903.
\end{flushleft}
