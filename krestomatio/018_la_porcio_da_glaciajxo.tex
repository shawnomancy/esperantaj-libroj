\begin{center}
\footnotesize (Rakonto de L. \fsc{Dilling}, tradukita de \\ \fsc{Eduard Hall} el \fsc{Joensuu} (Finnlando)).
\end{center}

   Estis tre agrable en la buduareto de sinjorino Breine. La meblaj
kovriloj estis el blua atlaso, kaj la ligna\^{\j}o de la se\^goj
estis origita. Tie sin trovis anka\u u multaj eta\^garetoj, egale
kovritaj per blua \^stofo kaj ornamitaj per bluaj fran\^goj, kiujn
najletoj origitaj firme tenis. \^Cirka\u ue en la \^cambro estis
distrita ankora\u u multo da sendifinaj puf-se\^goj, a\u u kiel ili
estas nomataj, tiuj remburitaj, rondaj, kvarangulaj kaj ovalaj
objektoj kun bluaj atlasaj kovriloj kaj amaso da nekompreneblaj
la\^ca\^{\j}oj kaj penika\^{\j}oj. En unu angulo de la buduaro, apud
la fenestro, staris pianeto kun la postflanko turnita al la interno
de la \^cambro. Kiam oni sidi\^gis al la piano, oni sidis tie
efektive ka\^sita, \^car grandaj foliokreska\^{\j}oj sur la amba\u u
flankoj de la instrumento formigis vere tropikan la\u ubon en tiu
\^ci malgranda Edeno de la sinjorino.

   Estis trankvile kaj pace en la buduareto, \^car neniu tie sin trovis;
kaj duonhela kvieta lumo sin vastigis tra la tuta \^cambro. Vere, la
origita lampo, pendanta de la plafono, brulis, sed tiel super \^gi,
kiel anka\u u super la kandeloj sur la skribotablo estis koketaj
\^sirmiloj. La blua silka pordokurteno estis tirita anta\u u la
pordo, sed el la flankaj \^cambroj estis a\u udata dancmuziko kaj
viva interparolado; \^car tiun \^ci vesperon estis balo \^ce la
komercegisto Breine, la sinjoro de la domo.

   La muziko silenti\^gis momente, la pordokurteno ek\^{\j}eti\^gis flanken,
kaj sur la sojlo de la pordo sin montris alta, gracia virina figuro.
Kelkaj frake vestitaj sinjoroj, kun la haroj kombitaj sur la frunto
kaj la klak\^capelo sub la brako, \^cirka\u uis \^sin; sed \^si
forigis ilin kun mieno de re\^gino, anta\u utiris rapide la kurtenon
anta\u u la pordo kaj falis poste laca sur unu el la silkokovritaj
kanapoj. \^Sia viza\^go estis tre delikate formita kaj pala, la
okuloj mallumaj, grandaj kaj radiantaj, kaj la nigraj haroj,
frizitaj a la Titus, portis neniajn aliajn ornamojn, ol paro da
diamantaj steloj. \^Sia tre dekoltita vesto, el salmokolora atlaso
kaj brodita silko, estis unu el tiuj nepriskribeblaj, majstraj
tualetoj el Parizo, kie puntoj kaj bantrozetoj, faldoj kaj festonoj
sin kuni\^gis en mirinda bonordita senordeco, dum kelkaj grandegaj
girlandoj da artaj rozoj faris surprize belan efikon, precipe tial,
ke ili estis metitaj en la plej neimageblaj lokoj.

   \^Si estis bela, kiel \^si ku\^sis tie, lace etendita sur la kanapo,
ludante kun monstra ventumilo; kaj pli bela ankora\u u \^si estus,
se \^sia viza\^go ne havus tiun mal\^sateman fieran esprimon. \^Si
estis laca de la bala muziko kaj vanila glacia\^{\j}o, de amindumaj
rigardoj kaj sengusta flata\^{\j}o de la frake vestitaj sinjoroj kaj
na\u uzaj salonparfumoj.

   \^Si estis fiera, \^car \^si estis \^ciam kutiminta, ke oni \^sin starigu
sur piedestalon, \^cirka\u uatan de genufleksantaj kavaliroj; sed
\^si nun estis jam tute laca de tiu \^ci adorado, \^si estis laca
esti diino de balo. Vere, \^si bone sciis, ke \^si prezentis tre
belan figuron sur la piedestalo, \^car \^si estis bela statuo; sed
\^si sciis anka\u u, ke oni admiradis la statuon pleje tial\dots ke
\^gi havis veran origa\^{\j}on.

   Stella Adler --- tia estis \^sia nomo --- restis ne longe ne malhelpata.
La pordokurteno kraketis, Stella levis senpacience siajn lacajn
okulojn, sed kiam \^si ekvidis, kiu envenis, \^goja rideto eklumis
sur \^sia bela viza\^go. La eniranto estis belega sinjorino de
\^cirka\u u tridek jaroj, kun plena figuro kaj floranta gajema
viza\^go. \^Siaj nigraj haroj estis kunvolvitaj en pezaj plektoj
super la kapo kaj ornamitaj per kufeto da blankaj plumoj kaj ru\^gaj
rozoj. \^Si havis sur si veston el vinru\^ga veluro kaj atlaso kun
ri\^ca garnituro da kremkoloraj puntoj. \^Si estis sinjorino.
Breine, la edzino de la komercegisto Breine kaj la regantino de tiu
\^ci eleganta domo.

 --- Mi sciis bone, ke mi vin tie \^ci trovos, diris sinjorino Breine,
sin sidigante apud sia amikino. Kiel mi \^gojas, ke mi vidas vin
\^ce mi tiun \^ci vesperon. Pensu nur, ke ni du ne parolis jam unu
kun la dua de unu tuta longa jaro! Kaj tio estis en la bona urbo
Parizo, kiam mi, post mia edzini\^go, voja\^gis tien kun mia kara
edzo, kaj vi estis ankora\u u en pensio. Nun ni havos ree agrablan
horon por babili.

 --- Se ni nur ne faras maljusta\^{\j}on al viaj aliaj gastoj, Fanny.

 --- Ne, nun mi estas absolute ekster dan\^gero, \^car nun la tuta
bestaro estas en la man\^gejo por sin nutri, a\u u pli delikate
esprimite, por vesperman\^gi. Kiam ni laste estis kune en Parizo,
tio estis anka\u u sur balo, \^cu vi tion memoras? Tion devus
patreto Breine scii, ke lia edzino veturis al opera maskobalo, dum
li estis en la kolekti\^go nacia kaj ku\^sis en Versailles! Kaj \^cu
vi ankora\u u memoras, ke ni perdis unu la duan en la premado kaj
devis reveni hejmen \^ciu aparte?

   Stella ek\^gemis kaj murmuretis duone por si mem: --- Ha, jes! tiun
maskobalon mi ne facile forgesos!

 --- Sed vi ne volus iom man\^gi, Stella?

 --- Ne, mi dankas, mi ne estas malsata.

 --- Jes, malka\^se dirate, vi ne multon perdas per tio \^ci, \^car tie
estas nek birdo, nek fi\^so, a\u u pli \^guste, estas de la amba\u u
specoj. Kiel tiuj bufedoj estas tamen bonega elpensa\^{\j}o! Oni
pakas kune \^ciujn eblajn kaj neeblajn homojn en la plej malgranda
loko, oni \^sar\^gas el \^ciuj eblaj kaj neeblaj pladoj sur unu kaj
tiu sama telero, kaj tiam oni tion man\^gas en la plej maloportuna
situacio! Sed unu glason da vino vi povas tamen trinki?

 --- Ne, mi estas tiel laca, mi bezonas nur iom ripozi.

 --- Sur viaj la\u uroj, jes! Ha, kian feli\^con, a\u u pli \^guste
malfeli\^con, vi faris en la bala salono! Sur la tuta planko estas
dis\^{\j}etitaj koroj, kiujn la malfeli\^caj sinjoroj perdis! Mi
devos lasi ilin elbalai, anta\u u ol ni komencos danci, \^car alie
oni povas faleti sur ili! La sinjoroj flugetis ja \^cirka\u u
vi\dots

 --- Jes, kiel mu\^soj \^cirka\u u sukero a\u u teologo \^cirka\u u paro\^ho!
diris Stella kun mal\^satema trajto \^cirka\u u sia bela bu\^so. La
vivopano, mia kara, estas por ili la grava demando. Sed la
edzi\^go\^casantaj sinjoroj faras laboron vane! Mi volas esti amata
pro mi mem kaj ne pro mia mono!

 --- Malsa\^gulineto! diris Fanny, donante al sia amikino nefortan
frapon per la ventumilo; \^cu vi estas do tiel malbela, a\u u \^cu
vi pensas, ke la sinjoroj estas tute blindaj? mi opinias \^ciam, ke
juvelo estas duoble pli bela, kiam \^gi estas enkadrigita en oro.

 --- Ili estas por mi tute indiferentaj.

 --- Ha, mi komencas kompreni. Via koro ne estas plu libera.

 --- Jes, sincere dirate, kara Fanny, mi estas efektive enamita; kaj
kiam oni havas unu en la koro, oni ne havas lokon tie por multaj.

 --- Tio dependas de la konstruo de la koro. Estas koroj, kiuj estas
tiel elastaj, ke en ili dudek trovus bone lokon! Sed kiel la
feli\^ca homo sin nomas? \^Cu mi lin konas?

   Stella mallevis la kapon.

 --- Mi lin ne konas mem! mi ne scias lian nomon, kaj eble mi lin
revidos neniam!

 --- Kiaj sentimentaj kapricoj tio estas? demandis sinjorino Breine kaj
skuis la kapon. Sed oni devigas sin \^ciam havi ion por sin
turmenti! Se oni ne havas zorgojn pro pano, oni tiam faras al si
zorgojn de amo! Sed \^cu vi permesos al mi demandi\dots

   La sinjorinoj estis interrompitaj de frizita kelnero, kiu eniris
dancante sur la pintoj de la piedoj. Kelnero Olsen estis \^ciam
melankolia. Li suferis konstante doloron de kapo kaj de malfeli\^ca
amo, kaj amba\u u doloroj naski\^gis de lia penema partoprenado en
la societa vivo; \^car li trinkadis \^ciam grandan kvanton da restoj
de vino en la nokto, kaj tio \^ci malsanigis lian kapon, kaj li
vidadis multon da belaj sinjorinoj, kaj tio \^ci malsanigis lian
koron. La lastajn estis permesate al li nur rigardi, sed ne tu\^si,
kaj tion li nomis "vandalaj turmentoj", volante diri "Tantalaj
turmentoj".

 --- \^Cu mi povas servi per io al la sinjorinoj? demandis Olsen kun
malgaja rideto, dum liaj akvobluaj okuloj glitis super la rondaj
\^sultroj de sinjorino Breine.

 --- Alportu al mi porcion da glacia\^{\j}o, diris Fanny.

 --- Post kvin minutoj mi \^gin metos \^ce viaj piedoj, sinjorino.

 --- Mi preferas, ke vi \^gin metu tie \^ci, sur la tablon!

 --- Tuj, sinjorino, tuj.

   Li fordancis. Fanny rigardis post lin kun rideto.

 --- La interkomuniki\^go estas infekta, \^si diris. Nun la kelneroj e\^c
volas esti kavaliroj.

 --- Tio estas ja tute natura, respondis Stella, kiam \^ciuj miaj
kavaliroj volas esti kelneroj por mi!

   \^Si levis la brovojn kaj apogis sian Titus-kapon al la klarebluaj
silkaj kusenoj. En tiu \^ci sama minuto la pordokurteno estis
\^{\j}etita flanken, kaj la komercegisto eniris en la buduareton.

   La komercegisto Breine estis malgranda, grasa sinjoro kun maldikaj
lume-brunaj haroj kaj lume-bruna vangbarbo iom griza. La
komercegisto, kiel \^sajnis, ne estis en bona humoro.

 --- Kiel vi fartas, sinjorinoj? li demandis, penante doni al sia
viza\^go komplezan esprimon.

 --- Ho, bonege! respondis Stella.

 --- Estas por mi plezuro vidi, ke vi nun iom ripozas, li diris al sia
edzino. Vi kredeble sentas vin jam tre laca esti tiel \^{\j}etata el
la brakoj de unu sinjoro al la brakoj de alia!

   La sinjorino ludis kun sia ventumilo kaj kantetis valson.

 --- Vi estas, kiel \^sajnas, en tre bona humoro, diris Breine ofendite.

 --- Tion oni vere ne povas diri pri vi, sinjoro Breine, rimarkis
Stella ridetante.

 --- Kontra\u ue, mi estas en la plej brilanta humoro\dots

 --- Ne kredu al li, Stella, diris la sinjorino. Mi vidas la\u u lia
nazo, ke li estas forte atakema. Li venis tien \^ci nur por aran\^gi
disputojn.

 --- Tiam estas plej bone sin savi per forkuro.

 --- Vi ja komprenas, ke mia edzino nur \^sercas?

 --- Mi devas tamen iri en la salonon. Mi \^suldas, jes, la dioj scias,
kiel multajn dancojn, kaj se mi ne pagos miajn \^suldojn tiel
balda\u u kiel eble, mi timas, ke miaj kreditoroj min trenos tien
per forto.

   \^Si foriris ridetante, lasante la edzan paron sola.

 --- Estu tiel bona, komencu, ju pli frue, des pli bone! Ne \^genu vin!
Mi estas preparita por la plej malbona.

 --- Sed, Fanny kara, mi ne diris e\^c unu vorton.

 --- Ne, sed vi estas kolerega. Tio estas ja tute videbla la\u u via
nazo, kiu estas pli staranta, ol ordinare.

 --- Fanny!

 --- Mi estas tro bela tiun \^ci vesperon, ne vere?

 --- Vi estas jam tro modesta!

 --- Mi koketis tro multe kun la junaj sinjoroj, ne vere? mi estis tro
\^{\j}etata el la brakoj de unu sinjoro en la brakojn de alia, kiel
vi vin tiel bonguste esprimis!

 --- La tuta nia edzeco estis neinterrompita dancado.

 --- Mi certe ne povas diri, ke mi dancis super rozoj.

 --- Ha, Fanny, se vi nur scius, kia turmenti\^go estas vidi sian
edzinon, prematan al la brusto de alia, \^cirka\u uprenatan de lia
brako, sin ruli en kapturnanta valso. Ho, tiuj baloj, tiuj baloj!

 --- Sendanka! kaj tamen \^gi estis sur unu el tiuj \^ci baloj, ke vi
trovis vian feli\^con! \^Cu vi memoras? Tio estis ne pli ol unu kaj
duono jaro anta\u ue. Tiam vi sidis tie kun bela meblita lo\^gejo
kaj tamen sen la plej belega meblo en domo --- juna edzino; kaj tiam
vi komencis trovi, ke esti tute sola en ok \^cambroj kaj kuirejo ne
estas tute sane por la homo. Kie tio \^ci estis tiam, kiam ni
renkontis unu la duan?

 --- Jes, estis sur balo.

 --- Kaj kie \^gi estis, ke vi petis mian manon?

 --- Anka\u u sur balo.

 --- Tial vi havas nenian ka\u uzon plendi. Ni vivas bone kune.

 --- Ho, oni vivas tre bone, kiam oni \^ciam veturas al baloj!

 --- Mi estas laborema edzino.

 --- Jes, vi man\^gas vian panon en la \^svito de via frunto sur \^ciuj
baloj de la urbo.

 --- \^Sercu nur! Vi pentos iam vian konduton, kiam mi estos for!

 --- Mi jam tute kutimis vidi vin forkuradi al baloj, diris la
komercegisto, nerve ludante kun la ventumilo de sia edzino, kiun
\^si estis metinta sur la tablon.

 --- Hodia\u u vere vi superas vin mem, diris Fanny, sin levante
ofendite. La komercegisto sin levis anka\u u.

 --- Jes, mi ne povas plu tion suferi. Tie \^ci devos fari\^gi alia
danco! Li prenis la ventumilon en amba\u u manojn kaj disrompis
\^gin en du pecojn.

 --- Vi fari\^gas \^ciam pli kaj pli bona, mia amiko. Nun mi ne mirus, se
vi komencus disrompi spegulojn kaj vitrojn de fenestroj.

 --- Mi estus sufi\^ce inklina por tio \^ci.

 --- Ne \^genu vin, mi petas. Mi volonte helpos al vi. Jen estas forna
fero! vi povas ja komenci rompi la spegulojn, dum mi ekspedos la
porcelanajn vazojn kaj la malgranda\^{\j}ojn!

   Hontigita kaj duone disarmita per la trankvileco de sia edzino,
la ko\-mer\-ce\-gis\-to for\^{\j}etis la rompitan ventumilon sur la
tablon kaj volis eliri, kiam la kelnero Olsen en tiu sama minuto
enkondukis en la kabineton junan sinjoron.

 --- Estu tiel bona, la gesinjoroj estas tie \^ci.

   Li estis altkreska, bela fra\u ulo kun kura\^ga, fiera viza\^go, buklaj
lumaj haroj kaj paro da dikaj blondaj lipharoj kun al tiuj
ali\^ganta pinta barbo.

 --- Mi ne malhelpas?

   Breine sin returnis vive.

 --- Georgo, vi, mia amiko? Neniel, neniel! Mia edzino kaj mi estis
nur okupitaj je kelkaj mastruma\^{\j}oj. Lasu, mi vin prezentos, mia
plej bona amiko, le\u utenanto Georgo Felsen, \^{\j}us reveninta
hejmen el la eksterlando --- Fanny, mia edzino.

 --- Mia edzo tiel ofte parolis pri vi, ke mi vin jam tute konas.

 --- Estas por mi anka\u u plezuro fari\^gi via konato, mia sinjorino!

 --- Mi a\u udis, ke vi revenis, kaj mi tiam tuj sendis al vi inviton.
En ia alia vespero vi kredeble ne renkontus mian edzinon; sed tiun
\^ci vesperon \^si estas okaze hejme, \^car \^si donas mem balon.

 --- Ne a\u uskultu lin, sinjoro le\u utenanto, li estas nur tiel malica.

 --- Mi lin bone konas, mia sinjorino. Tia li estis jam anta\u u mia
veturo Parizon.

 --- Parizon? ekkriis Breine. Mi pensis, ke vi estis en Italujo;
mirinde, ke ni ne renkontis unu la duan en Parizo. Tie mi estis en
la anta\u ua jaro.

 --- En tiel grandega urbo ne estas mirinde.

 --- Ni forveturis kelkajn tagojn post la granda opera maskobalo, diris
Fanny.

   Georgo sin returnis vive.

 --- Vi estis en la opera balo, sinjorino?

 --- Ne, kial\dots kial vi povas tion supozi?

 --- Tiu maskobalo, kiel \^sajnas, faris eksterordinaran impreson sur
vin, diris Breine.

 --- Sinjoro Felsen estas tiel flami\^gema.

 --- Tio\dots tio estis nenio!

 --- Ne, nun mi devas fari viziton en la salono kaj ne pli longe
mal\^sati miajn aliajn gastojn\dots

 --- Faru al mi la honoron, danci kun mi, sinjorino.

 --- Malfeli\^ce mi jam estas invitita; sed mi provos elzorgi por vi
fra\u ulinon, kaj mi pensas, ke vi ne perdos per la \^san\^go. \^Si
estas juna, bela, kaj ri\^ca. La lasta eco estas ja tio, kion la
plej multaj sinjoroj estimas pleje. Mi revenos balda\u u.

   Felsen apena\u u a\u udis \^siajn lastajn vortojn. Li staris en profundaj
pensoj kaj veki\^gis el sia revado nur tiam, kiam lia amiko lin
frapis sur la \^sultron kaj lin premis malsupren sur la sofon.

   La komercegisto, metinte siajn amba\u u manojn sur la \^sultrojn de la
le\u u\-te\-nan\-to, rigardis al li rekte en la okulojn.

 --- Kio estas al vi? Kiajn rememorojn vekis tiu opera balo \^ce vi,
kiam vi fari\^gis tiel ekscitita?

 --- Mi havis tie la plej mirindan aventuron de mia vivo. Mi tiel bone
kiel fian\^ci\^gis kun fra\u ulino, ne vidinte e\^c \^sian
viza\^gon.

 --- Kion vi babilas! Kiel tio povis okazi?

 --- Dum mi promenadis \^cirka\u ue en la premado de la maskoj, mi
eka\u udis fra\u ulinan vo\^con post mi, dirantan en pura norvega
lingvo: --- Dio, kion mi faros! Mi min returnis kaj ekvidis gracian
virinan figuron en nigra domeno kun nigra masko anta\u u la
viza\^go. Ke \^si estis juna kaj bela, pri tio mi estis tute
konvinkita. La rondaj brakoj, la marmorblanka kolo, la purpuraj
lipoj, kiuj vidi\^gis inter la puntoj de la masko, kaj la brilantaj
okuloj --- \^ciuj tiuj malgrandaj \^carmaj detaloj estis sufi\^caj
por mi, por auta\u usenti ravantan tuta\^{\j}on. Mi iris kura\^ge
anta\u uen kaj, prezentinte min mem kiel samlandulo, proponis, ke mi
estos \^sia kavaliro. \^Si akceptis kun danko mian brakon: mi estis
ja senmaska. Eble \^si konis min el Kristianio. Mian nomon \^si ne
demandis. Ni iradis \^cirka\u ue, por retrovi \^sian amikinon, kiun
\^si estis perdinta en la premado. Ili amba\u u havis nigrajn
domenojn sen iaj signoj, kaj tie nature sin trovis amaso da tiaj
nigraj domenoj. Sekve ni promenadis longan tempon kune kaj prenis
lokon en lo\^gio. \^Si estis sprita, edukita kaj aminda; mi estis
duone malsobrigita de amo, \^campano kaj dancmuziko; kaj ni amba\u u
enspiradis per plenaj pulmoj la aromon de romaneco, kiu trovi\^gis
en nia tuta aventuro. Mi prenis \^sian manon en mian kaj petegis
\^sin levi la maskon, sed \^si estis nemovebla. --- Ne dis\^siru la
misteran vualon, kiu estis \^{\j}etita sur nian renkonton kaj kiu
aldonis al \^gi \^gian \^carmon. Ni renkonti\^gos en Kristianio la
venontan vintron, \^si diris. --- Lasu min nun iri, akompanu min nur
al veturilo kaj ne penu scii, kiu mi estas, a\u u sekvi min. Mi
konfidas al via kavalireco. Mi prenis diamantan ringon de mia
malgranda fingro kaj metis \^gin sur \^sian manon. --- Per tiu \^ci
ringo mi vin katenas, mi diris; kiam mi vin revidos, ni disi\^gos
neniam. \^Cu vi volas \^gin porti? --- Jes, \^ciam! \^si respondis.
Kiam \^si estis en la veturilo, \^si sin klinis super min kaj kisis
min sur la frunto. --- Ni renkontos nin en Kristiano, estis \^siaj
lastaj vortoj. Kaj nun mi tien \^ci venis. Mi iros en teatron kaj al
koncertoj, al baloj kaj vesperman\^goj, kaj la diamanta ringo sur
\^sia mano estos la stelo, kiu montros al mi la vojon al la
Bethlehemo de la amo.

 --- Tion oni povas nomi romano, diris la komercegisto, sin levante.
Oni povus pensi, ke vi estas naskita en la kavalira tempo. Sed kien
la sinjorinoj foriris? \^Cu ni ne iros al ili?

 --- Mi preferus resti en paco tie \^ci. Estas tiel agrable en tiu \^ci
buduaro.

 --- Bone, tiam mi sendos la sinjorinojn al vi. Tie \^ci estas la
plejsanktejo de Fanny, tien \^ci venas nur la endedi\^citaj!

   Kiam la komercegisto foriris, Felsen sin amuzis per rigardado de
\^ciuj belaj malgranda\^{\j}oj kaj floroj, kiuj estis dismetitaj
\^cirka\u ue en la plej bongusta senordeco. Oni rimarkis tuj en
\^ciuj tiuj elegantaj, kokete distritaj luksa\^{\j}oj la delikatan
manon de edukita virino. Li haltis anta\u u la piano.

 --- Kia bongusta aran\^go, li murmuretis. La piano en la angulo estas
\^cirka\u uata de tropikaj kreska\^{\j}oj.

   Li sidi\^gis sur la se\^go anta\u u la piano kaj komencis transturnadi la
foliojn en la notolibro.

   Li ne rimarkis, ke Fanny kaj Stella eniris.

 --- Tie \^ci estas ja neniu, diris Stella. Des pli bone.

 --- Mi estas certa, ke li pla\^cos al vi, diris Fanny vive. Li estas
juna, bela, edukita kaj eleganta. Li vivis multajn jarojn en la
alilando.

 --- Li kredeble vivis tie tiel multe, ke li nun revenis hejmen, por
trovi ion, per kio li povus da\u urigi vivi! Li fari\^gos kredeble
unu el miaj plej varmaj adorantoj!

 --- Kia senhonteco! murmuris Felsen.

 --- Ho, li havas tamen ion alian, per kio li povas vivi krom vi, diris
Fanny incitita.

 --- Tio estas la plej bona vojo, kiun li povus elekti. Dudek mil
kronoj da procentoj en jaro kaj nenion alian fari, ol esti la edzo
de sia edzino!

 --- Fi, Stella!

   Felsen levi\^gis, pala, kun kunpremitaj dentoj.

 --- Nu, diris Stella, irante al la tablo, elser\^cu la pekulon, konduku
lin tien \^ci kaj starigu lin en la vicon de miaj adorantoj. Li
estas ja le\u utenanto. Eble mi povos lin uzi, por komandi la
aliajn!

   Felsen elpa\^sis.

 --- Mi estas jam tie \^ci kaj tuj, kiam mi estos prezentita, fari\^gos
por mi plezuro, akcepti la deciditan por mi lokon.

   \^Ce la sono de lia vo\^co Stella turnis la kapon. \^Si estis mortpala
kaj apogis sin konvulsie al la marmora tablo. \^Gi estis li, la
idealo de \^sia vivo, \^sia sekreta fian\^co, kiun \^si nun tiel
dolore insultis kaj por \^ciam forpu\^sis! \^Si rigardis me\^hanike
sian manon. Feli\^ce \^si portis gantojn kun sep butonoj, tiel la
ringo estis bone ka\^sita.

 --- Pardonu, se mi vin ektimigis, miaj sinjorinoj, diris Georgo, Sed
mi sidis tiel absorbita per la notoj, ke mi ne rimarkis vian eniron.

   Sinjorino Breine sa\^ge decidis rigardi la tutan aferon kiel \^sercon.

 --- La situacio estas vere malagrabla, \^si diris ridetante. Nun estas
jam tro malfrue, por sveni. Mi faras, kiel mia amikino, kaj ser\^cas
mian rifu\^gon en la ventumilo. Kiam \^gi estas malfermita, \^gi
faras la servon de kurteno, kiu estas anta\u utirita, dum la
viza\^go faras tualeton. Nu, nun mi pensas, ke \^gi pli-malpi
rehavas la \^gustan esprimon! Lasu min prezenti: le\u utenanto
Felsen --- fra\u ulino Adler\dots Sed, kara Stella, vi tiel
tremas\dots Sidi\^gu!

 --- Tio balda\u u pasos. Mi estas iom nerva.

 --- Mi beda\u uras, diris Georgo, ke mi estas la ka\u uzo\dots

 --- Nu, interrompis lin Fanny, pro puno mi vin kondamnas resti tie
\^ci kaj flegi la fra\u ulinon, \^gis vi povos konduki \^sin,
perfekte refortigitan, en la salonon. Mi devas forlasi vin. Miaj
\^suldoj de mastrino min vokas.

   Fanny foriris. Stella sidis kun mallevitaj okuloj, Felsen staris
malvarmega kaj apogis sin al la dorso de unu se\^go. Stella levis la
kapon kaj diris kun devigita trankvileco:

 --- Sidigu vin, sinjoro Felsen.

 --- Mi dankas.

 --- Diru al mi, \^cu ni parolos pri la vetero kaj blovo, a\u u \^cu ni
estos sinceraj?

 --- Kiel al vi pla\^cos.

 --- Nu, lasu nin esti sinceraj. Vi min abomenas, ne vere?

 --- Lasu nin unue konsenti, ke mi vin ne adoras, mia fra\u ulino. Se
mi vin komprenis \^guste, vi timas min pli kiel adoranton, ol kiel
malamikon.

 --- Sed tiam mi ju\^gis ne konante. Nun, kontra\u ue\dots
kontra\u ue\dots

 --- Nun, vi min konas egale malbone.

 --- Jes, vere, sed\dots

 --- Sed?

 --- Sed\dots

 --- \^Cu vi estos efektive sincera, fra\u ulino?

 --- Mi deziras nenion pli volonte.

 --- Sed, trovinte, ke mi estas sufi\^ce malhonta ne tuje fali sur miajn
genuojn, vi eble havus la deziron meti min venkitan \^ce viaj
piedoj, por poste havi la plezuron --- min forpu\^si.

 --- Sinjoro le\u utenanto! ekkriis Stella vive kaj levis sin de sia
se\^go.

 --- \^Cu la gesinjoroj ion deziras? eksonis murmuretanta vo\^co post
\^si. \^Gi estis la melankolia kelnero Olsen, kiu, enirinte senbrue,
nun staris tie kun kapo klinita kaj kun porcio da glacia\^{\j}o en
la mano.

   Stella re\^{\j}etis sin sur la se\^gon kaj ka\^sis sian viza\^gon en la
manoj.

 --- Ne, iru nur!

 --- Mi alportis nur tiun \^ci glacia\^{\j}on al sinjorino Breine.

 --- Metu \^gin tie \^ci sur la tablon kaj alportu al mi anka\u u unu
porcion, diris Georgo senpacience. Mi bczonas min iom malvarmetigi,
li murmuretis al si mem.

 --- Tuj, sinjoro, tuj! Nenion alian, sinjoro?

 --- Ne.

   Stella sidis da\u urige kun sia buklita Titus-kapo en la manoj, \^sia
blanka kolo estis fleksita, kaj parto de la delikate formita orelo
rigardetis el la grandaj salmokoloraj rozoj, kiuj estis
alfortikigitaj sur \^sia \^sultro. Georgo sentis, ke li estis tro
kruda.

 --- Fra\u ulino!

   \^Si relevis la kapon kun la malnova fiera esprimo en la mallumaj
okuloj.

 --- \^Cu vi havas ankora\u u ion nediritan?

 --- Mi estis tro ekflami\^ga, mi estis rapidema, mi volus nur\dots

 --- Vi volus nur pagi mokon per moko kaj min humiligi. Gratulu vin
mem, le\u utenanto Felsen, vi atingis perfekte vian celon!

 --- Fra\u ulino!

 --- Mi kura\^gis preni vin por \^casisto de edzi\^go, vi kura\^gis \^{\j}eti
en mian viza\^gon, ke mi estas senkora koketulino, kiu sin amuzas
per enjungado de malfeli\^caj amantoj al sia veturilo de triumfo.
\^Cu ni nun estas kvitaj?

 --- Mi povas nenion diri por mia pravigo.

 --- Mi povus eble diri multe por la mia, sed mi ne scias vere, kiel ni
tiun \^ci interparolon komencis, kaj mi scias ankora\u u pli
malmulte, kial ni \^gin\dots da\u urigus.

 --- Jes, vi estas prava, fra\u ulino. Mia efektiva intenco estis inviti
vin por la sekvanta danco.

 --- Mi dankas. En tiu \^ci minuto mi estas tiel laca, ke mi devas iom
ripozi.

 --- Tiam mi vin ne pli longe maloportunos.

   Li salutis malvarme kaj eniris rapide en la salonon.

   Stella levi\^gis. Kiam li estis ekster la pordo \^si falis sur la
sofon, \^{\j}etis sin en unu angulon kaj ka\^sis la kapon en la
klarbluaj silkaj kusenoj.

   Multaj kaj doloraj pensoj trairis la kapon de Stella, dum \^si ku\^sis
tie kun la viza\^go enpremita en la silkaj kusenoj. Tio estis do la
unua renkonto, kiun \^si en sia fantazio tiel bele elpentris al si
mem, kaj tamen --- la \^suldo estis ja \^sia propra. Sed eble li
e\^c ne memoris ilian malgrandan aventuron, --- kaj \^si, kiu
konsideradis tiun \^ci ringon, kiun li metis sur \^sian fingron,
kiel ringon de fian\^ci\^go, \^si, kiu tiel ame flegis la puran
floron de sia unua amo kaj \^gin nutris per sia revado\dots kia
malsa\^gulino \^si estis! Kaj \^si povis vere kredi, ke en nia proza
tempo iom da poezio ankora\u u povus ekzisti! \^Si levi\^gis, kaj,
detirinte la gantojn, \^si longan tempon rigardis la ringon. \^Sia
amo estis profunda kaj pura, kiel tiu \^ci diamanto. En ties kvieta
fajro ku\^sis mira sor\^co, \^car \^gi pruntis sian brilon el la
poezio de la animo. Tiun \^ci vesperon \^gi ne brilis kiel kutime.
Eble la roso en \^siaj okuloj ka\u uzis, ke \^gia brilo perdi\^gis.
Nun \^gi estis nur la lasta larmo sur la tombo de \^sia amo!

 --- Sed li neniam scios, ke mi estas tiu, kiun li renkontis en la
balo, \^si murmuretis. \^Si ja tenis en sia mano la \^slosilon de la
enigmo, \^slosilon, pri kiu \^si kredis, ke \^gi malfermos al \^si
la teran paradizon. Neniu \^gin vidis, des pli bone, e\^c ne Fanny.

 --- For, for!

   \^Si kuris al la fenestro, por \^{\j}eti la ringon sur la straton, sed
\^si sin detenis. La malgranda vipero tiel fortike enmordi\^gis en
\^sian koron, ke \^si ne povis suferi, ke \^gi estu dispremata sub
la piedoj de la preterirantoj. Kiam \^si revenis de la fenestro,
\^sia rigardo falis okaze sur la porcion da glacia\^{\j}o, kiu
staris sur la tablo, kaj tiam subite venis al \^si ideo. Li petis ja
la kelneron alporti al li porcion da glacia\^{\j}o; \^si lasos lin
ricevi \^gin --- kune kun la ringo. \^Si sin sidigis apud la
skribotablo kaj iom pensinte, \^si skribis la sekvantajn liniojn:

\begin{quote}
\footnotesize Tiun signon portu\\
 Sur mano via!\\
 En mistero restu\\
 La nomo mia!
\end{quote}

   Kunvolvinte la papereton, \^si \^sovis la ringon super \^gin kaj
enpremis \^gin en la glacia\^{\j}on. \^Si lasis la telereton stari
sur la tablo kaj eniris en la salonon, por elser\^ci la kelneron.

   Momenton post la eliro de Stella Fanny eniris en la kabineton, pri
kiu la kelnero Olsen diris al \^si, ke tie la glacia\^{\j}o staras
sur la tablo. \^Jus kiam \^si volis porti la kuleron al sia bu\^so,
\^si rimarkis la papereton kun la ringo, elprenis \^gin el la
glacia\^{\j}o kaj legis la versa\^{\j}on. Traleginte \^gin multajn
fojojn, pensinte pri la afero kaj skuinte konfuzite la kapon,
sinjorino Breine metis la ringon sur sian fingron, prenis la
glacia\^{\j}an telereton en la manon kaj iris en la man\^gejon al la
kelnero.

 --- Kiu donis al vi tiun \^ci glacia\^{\j}on?

 --- La kuiristino, li respondis kun sia kutima melankolia rideto.

 --- Ne, li scias nenion, pensis Fanny. \^Cu \^gi staras longan tempon en
la kabineto?

 --- Mallonge, sinjorino. La bela le\u utenanto --- mi pensas, ke li estas
nomata Felsen --- petis min \^gin meti tien. Li tiel fervoris, ke mi
foriru, kaj sendis min ser\^ci por li mem porcion da glacia\^{\j}o;
sed jam nenio restis.

 --- Li povas ricevi tiun \^ci.

 --- \^Cu mi \^gin devas porti al li?

 --- Ne, mi \^gin faros mem. Servu vi la limonadon al la gastoj.

 --- Nu, pensis Fannv, donaco de Felsen! Delikata \^gentileco! Sed post
tiel mallonga konateco! \^Gi estas nature pro mia edzo. Mi eniros,
por lin danki en ia delikata maniero, kaj mi ne metos la ganton, por
ke li vidu, ke mi ricevis la ringon.

   \^Si eniris en la salonon kaj alproksimi\^gis al Felsen, kiu staris
sola en unu angulo de la \^cambro.

 --- Vi deziris glacia\^{\j}on, diris la kelnero. Permesu al via humila
servantino doni \^gin propramane al vi.

 --- Vi estas tro bonega!

   \^Si anta\u uportis al li kokete la glacia\^{\j}on kaj lasis samtempe la
diamanton ekbrili. Li ektremis kaj preska\u u faligis la telereton,
dum liaj okuloj ripozis sur la ekbrilinta \^stono. Neniaj duboj plu!
Li trovis la ringon; sed la posedantino, la sekreta adoratino de
liaj revoj, estas la edzino de lia plej bona amiko! Ho, tio estis
terura!

 --- Vi rigardas mian ringon, diris Fanny kun fripona rideto. \^Gi estas
bela, ne vere?

 --- Jes\dots jes\dots \^gi estas\dots tre bela.

 --- \^Gi estas memoro de unu amiko, kiu estas al mi tre kara, kvankam
mi lin konas nur mallongan tempon. Mi konservos \^ciam tiun \^ci
memora\^{\j}on kaj mi \^ciam \^gin alte \^satos, \^si diris,
forirante kun koketa klineto de la kapo.

   Felsen staris tute surdigita kun la telereto en la mano.

   Dume Stella estis irinta al la kelnero kaj petinta lin doni la
glacia\^{\j}on al Felsen.

 --- Li \^gin jam ricevis, diris Olsen. Li staras kun \^gi en la mano.
Jen li estas.

   Li estis prava. Felsen tenis la telereton en la mano kaj \^sajnis esti
ekscitita. Fanny staris apude kaj parolis vive kun li. Io brilis
forte sur \^sia mano. \^Si portis ordinare nur sian fian\^ci\^gan
ringon. Kion tio signifas?

   Nun \^si forlasis lin kaj iris al la buduaro. \^Si estis haltigita sur
la vojo de paro da kavaliroj. Stella preterglitis tien nerimarkita
kaj lokis sin post la piano, kie \^si sidis, ka\^sata de la
\^cirka\u ustarantaj grupoj da kreska\^{\j}oj. Momenton post tio
eniris Fanny, kaj, starante sub la lustro, \^si lasis la ringon
ekbrili kaj admiradis \^gin. Stella rigardadis \^sin tra la
kreska\^{\j}oj.

   Jes, estis tute certe. \^Gi estas \^sia ringo; sed kiel Fanny \^gin
ricevis? \^Cu li vere \^gin senpere fordonacis, kiel senvaloran
objekton? Fanny estis tiel profundigita en la konsideradon de la
ringo, ke \^si ne rimarkis la eniron de sia edzo.

 --- Nu, Fanjo, \^cu vi estas tute sola? diris la komercegisto. \^Cu vi
estas aukora\u u kolera?

   Li prenis \^sian manon, kiun \^si penis fortiri.

 --- Vi ja scias, ke se mi estas anka\u u iafoje iom viva kaj \^{\j}aluza,
tio estas \^car mi vin tiel amas kaj \^car mi timas, mi maljuna
urso, ke mi ne povas fari vin tiel feli\^ca, kiel vi meritas.

   Li karesis \^sian manon, sed subite ektremis.

 --- Kia\dots kia ringo estas tio?

 --- \^Gi\dots \^gi estas anonima donaco, respondis Fanny iom konfuzite.

 --- Kiu \^gin donis al vi?

 --- Oni ne povas koni la donanton de anonimaj donacoj, diris Fanny
petole.

 --- Sed mi lin konas. \^Cu diri al vi tiun nomon?

 --- Vi scias do?

 --- \^Gi estas Georgo Felsen. Mi scias \^cion.

   Stella sidis, kiel falinta el nuboj. \^Cu Felsen vere estas enamita en
Fanny'n, kaj \^cu li, ne oferante e\^c unu solan penson al ilia
renkonto, ne penante elser\^ci, el kie la ringo venis, senpere \^gin
metis sur \^sian fingron? A\u u \^cu Fanny trovis la ringon kaj
metis \^gin sur la fingron, por pensigi lin, ke \^si estas la
elektita? Sed ne, tio \^ci estis malsa\^ga penso. Fanny nenion sciis
pri la historio de la ringo, \^si neniam \^gin vidis. Oni povus
frenezi\^gi.

   La komercegisto falis malespere sur se\^gon.

 --- Kaj \^gi estas mia plej bona amiko kaj mia propra edzino, kiuj min
trompas!

 --- Jes, la edzinoj de aliaj homoj tute ja ne povas havi intereson vin
trompi! diris Fanny trankvile.

 --- Ne elturnu vin per malkaraj \^sercoj! mi scias nun \^ciujn viajn
sekretojn. Vi volas eble nei, ke vi estis sen mia scio en la opera
balo en Parizo?

 --- Ha, Stella babilis?

 --- Stella nenion diris, sed tiu \^ci malfeli\^ca ringo malfermis miajn
okulojn kaj montris al mi, kiel facilanime kaj mensoge vi agis.

 --- Diru al mi, kara Breine, \^cu vi efektive ne trinkis tro multe post
la man\^go?

 --- Ho, Fanny, Fanny, kiel vi povas esti tiel kruela?

 --- Pro Dio, Breine, mi ne bone komprenas, kion mi faris. Ni devas
trovi sinjoron Felsen, li klarigos al ni \^cion.

   En tiu minuto envenis la kelnero Olsen, por sciigi, ke la dancado
ko\-men\-ci\-\^gis kaj ke oni demandas pri la gemastroj.

 --- \^Cu sinjoro Felsen estas tie?

 --- Ne, li prenis sian superveston kaj forlasis la balon anta\u u kelkaj
minutoj, diris Olsen.

 --- Ne dirinte adia\u u al mi! Sed kion do signifas \^cio \^ci? murmuris
Fanny.

   La interparolado estis interrompita de bela kavaliro, al kiu la
sinjorino estis promesinta la unuan dancon post la vesperman\^go.
Fanny prenis lian brakon kaj eniris en la salonon.

   La komercegisto ripozis, tute dispremita, en sia apoga se\^go. Stella
sidis post la piano kaj ploris. El la salono sonis danca muziko,
ridado kaj \^sovado de piedoj.

 --- Jes, \^gi estas efektive animita balo kaj luksa vesperman\^go, kiujn
ko\-mer\-ce\-gis\-to Breine donas tiun \^ci vesperon, li \^gemis.
Tiel vigla dancado da\u uri\^gas tie interne, kaj tamen la
komercegisto mem sidas tie \^ci sur la ruinoj de sia feli\^co!

   Li ka\^sis la viza\^gon en siajn manojn.

 --- \^Cu ekzistas en la mondo pli malfeli\^ca homo ol mi? li \^gemis
la\u ute.

   Mola mano estis metita sur lian \^sultron. Stella staris apud lia
flanko.

 --- Ekzistas ankora\u u unu alia, \^si diris.

 --- \^Cu vi estas anka\u u malfeli\^ca, fra\u ulino?

 --- Ho, jes!

 --- Sed vi ne havas edzinon, kiu vin trompas.

 --- Ne, sed unu kaj tiu sama homo estas la ka\u uzo de nia malfeli\^co.

 --- Felsen?

 --- Jes. Mi lin amis tiel profunde, tiel kore, kaj nun li min
forpu\^sis kaj forgesis pro Fanny, kaj \^cio tio post konateco de
nur duono da horo. \^Cio estas tiel freneze malsa\^ga, ke mia kapo
preska\u u turni\^gas.

 --- Ili konas unu la duan jam pli ol unu jaron. Mi estas certa pri
tio.

 --- Edzinigita virino --- kia senfunda malta\u ugeco!

 --- \^Gi estas la francaj ideoj. Sed kion vi pensas fari?

 --- Mi? mi almena\u u neniam edzini\^gos!

 --- Mi anka\u u neniam faros \^gin plu, diris la komercegisto. Mi petos
eks\-ed\-zi\-\^gon.

 --- Sed eble Fanny estas senkulpa?

 --- Ne, mi scias, kiel facilanime \^si komencis tiun \^ci rilaton, kaj
mi neniam povas havi plu konfidon al \^si.

 --- \^Cio povos fari\^gi ankora\u u bona, sinjoro.

 --- Neniam, neniam.

   Li donis al \^si signon, ke \^si sidi\^gu. \^Si sin sidigis kontra\u u li.

 --- Diru al mi, \^cu mi ne estas sufi\^ce maljuna, por povi esti via
patro?

 --- Mi vin ne komprenas\dots

 --- Mi opinias, ke ni amba\u u staras tute sole en la mondo. Vi ne havas
plu gepatrojn. \^Cu vi ne volus fari\^gi mia adopta filino?

 --- Tiu \^ci propono estas tiel subita, ke mi trovas, ke ni amba\u u
devas precize pripensi \^gin.

   Sinjorino Breine dume venis el la salono kaj, vidante sian edzon kaj
Stella'n en viva kaj konfida interparolado, stari\^gis en la mezo de
la \^cambro.

 --- Vidu, da\u urigis la komercegisto, kiam la eksedzi\^go kun Fanny
estos atingita, mi sentos min mem tiel sola.

 --- \^Cu ni nun eksedzi\^gos! pensis Fanny.

 --- Kiel feli\^ca mi estus tiam, se amika an\^gelo volus vivigi mian
malplenan hejmon kaj flegi min en la maljunaj tagoj.

   Fanny staris kolere kaj ludis kun sia ventumilo.

 --- Nun ili do kune vivos!

   Stella donis al li sian manon.

 --- Estos \^ciam agrable al mi, se mi iel povos helpi al via feli\^co,
sed\dots

 --- Ho, ne faru al vi penojn, diris Fanny, rapide aperante. Mi liveros
mem al li la feli\^con, kiun li bezonas.

   Fanny sin turnis vive kaj kolere al sia amikino:

 --- Vi, Stella, vi estis do la serpento en mia edzeca paradizo? \^Gi
estas tial vi, kiu elpensis mi ne scias kiel malhonoran historion
pri mi kaj Felsen, kiun mi tamen tiun \^ci vesperon vidas la unuan
fojon anta\u u miaj okuloj! Kaj kun kia intenco vi \^gin faris? Pro
malamo al mi \^gi ne povas esti, \^car mi neniam faris
malbona\^{\j}on al vi; kaj pro amo al Breine \^gi anka\u u ne povas
esti, \^si da\u urigis preska\u u plorante, \^car \^cu iu alia ol mi
mem povus esti sufi\^ce malprudenta, por ami tian sova\^gan beston,
kiel mia edzo?

 --- Fanny, a\u uskultu min, diris Stella.

 --- Sed nek al vi, nek al alia prosperos for\^siri lin de mi, --- mi
amas lin tro forte, \^si diris kaj \^{\j}etis la brakojn \^cirka\u u
lian kolon.

 --- Lasu min, mi volas disi\^gi!

 --- Jes, sed mi ne volas!

   En tiu minuto eniris Georgo Felsen.

 --- Pardonu al mi, se mi malhelpas; sed mi estos mallonga. Mi intencis
komence skribi al vi, sed mi komprenis, ke mi pli bone povos doni al
vi bu\^san klarigon.

 --- Mi foriros, diris Stella.

 --- Ne, mia fra\u ulino, kion mi diros, vi povas libere a\u uskulti. \^Gi
eble klarigos mian konduton kontra\u u vi. \^Car kiam mi nun ree
forlasos mian patrujon, kaj kredeble por \^ciam, mi ne dezirus, ke
vi pensu pri mi kun tro granda maldol\^ceco.

 --- Vi forveturos? demandis Fanny.

 --- Jes, sinjorino; post tio, kio okazis inter ni, mi ne volas resti
pli longe tie \^ci.

 --- Kio okazis inter ni\dots ?

 --- Mi konas la tutan historion de la diamanta ringo, diris Breine.

 --- Tiu historio ne estas tamen tiel dan\^gera, respondis Fanny
trankvile.

 --- Estas eble, ke vi, sinjorino, prenas nian aventuron kiel \^sercon,
diris Georgo. Mi rigardas \^gin tute alie. Mi estis tu\^sita de la
romaneco kaj pikanteco en la situacio, mi kreis al mi mem
figura\^{\j}on de vi en mia fantazio, kaj, ne konante la originalon,
mi adoris tiun \^ci fantazian portreton.

   Stella ektremis, komencante kompreni, ke tie \^ci estis ia
malkompreni\^go kaj ke \^si estis maljusta kontra\u u amba\u u,
Georgo kaj Fanny. La trankvileco de la lasta ne povis esti
\^sajniga.

 --- Sed, granda Dio, kio min tu\^sas viaj fantaziaj portretoj? ekkriis
Fanny surprize.

 --- \^Sajnas, ke vi \^cion forgesis, diris Georgo. Nun, de la momento,
kiam mi vidis, ke la edzino de mia amiko portas tiun \^ci fatalan
ringon, mi decidis anka\u u forgesi \^cion kaj mi komencis kompreni,
ke mia amo estis nur romanaj kapricoj kaj ke mi efektive vin neniam
amis.

 --- Ho, kiel eksterordinare mal\^goje! ekkriis Fanny, ridegante. Jes,
ke Breine perdis la malgrandan porcion da prudento, kiun li posedis,
tio estas jam longe klara al mi, sed nun mi vere komencas kredi,
sinjoro Felson, ke via supra eta\^go anka\u u estas iom en malordo.

 --- Sed \^cu vi povas nei, ekkriis Breine, ke tiu \^ci diamanta
ringo\dots

 --- Jes, tiu \^ci malfeli\^ca ringo estas la kulpo de \^cio, diris Fanny.
Estu tiel bona, sinjoro Felsen, jen mi \^gin redonas al Vi. Mi \^gin
ricevis en porcio da glacia\^{\j}o kune kun versa\^{\j}o. Mi metis
la ringon sur la fingron, kaj de tiu \^ci momento \^cio estas
freneza. Breine estas freneza, Stella estas freneza, Felsen estas
freneza, kaj nun mi mem fari\^gos balda\u u anka\u u freneza, tio
estas kredeble la plej prudenta, kion mi povas fari.

 --- Mi neniam sendis al vi ian glacia\^{\j}on a\u u ian versa\^{\j}on. Vi scias
tamen, ke mi metis la ringon sur vian fingron anta\u u unu jaro.

 --- Ne, nun estas jam tro freneze, ekkriis Fanny.

   Stella elpa\^sis pli proksime.

 --- Permesu al mi enporti kelkan klarigon en tiun \^ci senordigitan
aferon. Mi estas tiu, kiu ricevis de vi la ringon en la opera balo.

 --- Vi?

 --- Jes, kaj mi ne hontas konfesi, ke mi \^gin rigardadis kun tiaj
samaj profundaj sentoj, kiel vi. Poste ni nin rerenkontis en kolero,
kaj mi decidis, ke vi neniam sciu, kiu mi estas. Mi metis la ringon
en la vanilan glacia\^{\j}on, kaj \^gi nun erare venis en la manojn
de Fanny.

 --- Tiel, tamen\dots Ha, fra\u ulino, kiel vi min faras feli\^ca!

 --- Sed kia operbala historio tio estas? demandis Fanny.

   Tion mi rakontos al vi poste, diris la komercegisto. Ni iru en la
balan salonon kaj ne plu mal\^satu niajn gastojn. Georgo estos jam
tiel bona amuzi fra\u ulinon Stella dume\dots

   Kaj tiel bona estis sinjoro Felsen volonte. La interparolado estis
kredeble tre viva kaj interesa, \^car kiam la gastoj \^ciuj jam
estis for kaj Breine kaj la sinjorino post unu horo tien eniris,
Georgo kaj Stella sidis ankora\u u en animita interparolado. Ili
sidis sur la klarblua silka kanapo --- tute proksime, tiel proksime,
ke la kolosaj salmokoloraj rozoj sur la \^sultro de Stella estis
tute depremitaj. La diamanta ringo estis nun ree sur la fingro de
Stella, sed oni ne povis \^gin vidi, \^car Felsen tenis \^sian manon
en la amba\u u siaj.

 --- Nu, diris Breine, \^sajnas, ke ni povas gratuli.

 --- Jes, nun estas \^cio en ordo.

 --- Kaj la gastoj foriris, diris la komercegisto kaj frotis kontente
la manojn, tial nun ni estas la mastroj en la domo, kaj nun ni havos
ekstran festenon.

   Li vokis Olsen'on, kiu estingadis la lumojn en la salono.

 --- Alportu al ni botelon da \^campano.

 --- Kun glacio?

 --- Ne, mi dankas. Mi kredas, ke ni \^ciuj ricevis jam sufi\^ce de tiu
\^ci artikolo. Tiu lasta glacia\^{\j}o estis tiel bruliga!

   La \^campano estis alportita, la glasoj plenigitaj kaj la sano de la
novaj gefian\^coj trinkita.

 --- Mi petas anka\u u la permeson tre humile gratuli, diris la kelnero
Olsen; mi estas anka\u u fian\^cigita!

 --- Ho, kun kiu?

 --- Kun la vidvino Kaspersen, la kuiristino.

 --- Kiam tio fari\^gis?

 --- Anta\u u kelkaj minutoj, sed ni konati\^gis unu kun la dua jam anta\u u
longe, \^car en la lastaj tempoj ni estis tiel ofte kune \^ce
invita\^{\j}oj.

 --- Sed tiun \^ci vesperon vi vin klarigis? demandis Stella.

 --- Jes, \^si staris en la kuirejo, kaj \^si estis tiel dol\^ca, kaj tiam
mi diris, ke mi \^sin amas, kaj tiam \^si min \^cirka\u uprenis, kaj
tiam mi tiel blanki\^gis, \^car \^si estis faruna, kaj tiam ni kisis
unu la duan. Ho, estis tiel Die bonege!

   Olsen mallevis la okulojn kun dol\^ca melankolia rideto kaj flugetis
en la salonon, por estingi la lastajn lumojn.

   Kiam Felsen kondukis sian amatinon al la veturilo, li premis kison
sur \^sian manon kun la brilanta \^stono, kiu nun fari\^gis la stelo
de lia feli\^co.

   Nia reala, proza tempo eble skuos malestime la \^sultrojn pro la
romanaj fantazioj de la juna paro; sed tiel longe, kiel amo ekzistos
en la mondo --- kaj \^gi ekzistos tie \^gis la mondo pereos --- tiel
longe anka\u u la romaneco vivos.

\smallrule{}
