\begin{center}
\footnotesize (Hinda legendo.)
\end{center}

   Kiam la \^ciopova Mahadeva kreis la belegan Hindujon, li deflugis sur
la teron, por \^gin admiri. De lia flugo eklevis sin varma, bonodora
vento. La fieraj palmoj klinis anta\u u Mahadeva siajn suprojn, kaj
ekfloris sub lia rigardo la puraj, blankaj, delikataj, aromaj
lilioj. Mahadeva de\^siris unu el la lilioj kaj \^{\j}etis \^gin en
la lazuran maron. La vento ek\^sancelis la kristalan akvon kaj
enkovris la belegan lilion per blanka \^sa\u umo. Minuto --- kaj el
tiu \^ci bukedo de \^sa\u umo ekfloris la virino --- delikata,
bonodora kiel la lilio, facila kiel la vento, \^san\^ga kiel la
maro, kun beleco, brilanta kiel la \^sa\u umo mara, kaj rapide
pasanta, kiel tiu \^ci \^sa\u umo.

   La virino anta\u u \^cio ekrigardis en la kristalajn akvojn kaj
ekkriis:

 --- Kiel mi estas belega!

   Poste \^si ekrigardis \^cirka\u uen kaj diris:

 --- Kiel la mondo estas bela!

   La virino eliris sur la bordon seka el la akvo (de tiu \^ci tempo la
virinoj \^ciam eliras sekaj el la akvo).

   Je la vido de la virino ekfloris la floroj sur la tero, kaj el la
\^cielo sur \^sin ekcelis miljardoj da scivolaj okuloj. Tiuj \^ci
okuloj ekbrulis per ekstazo. De tiu \^ci tempo lumas la steloj. La
stelo Venus ekbrulis per envio --- pro tio \^si lumas pli forte ol
multaj aliaj.

   La virino promenadis tra belegaj arbaroj kaj herbejoj, kaj \^cio
silente estis ravita de \^si. Tio \^ci ekenuigis la virinon. La
virino ekkriis:

 --- Ho, \^ciopova Mahadeva! Vi kreis min tiel bela! \^Cio estas ravita
de mi, sed mi ne a\u udas, ne scias pri tiuj \^ci ravoj, \^cio estas
ravita silente!

   Eka\u udinte tiun \^ci plendon, Mahadeva kreis sennombrajn birdojn. La
sennombraj birdoj kantadis ravajn kantojn al la beleco de la belega
virino. La virino a\u uskultis kaj ridetis. Sed post unu tago tio
\^ci \^sin tedis. La virino ekenuis.

 --- Ho, \^ciopova Mahadeva! ekkriis \^si, al mi oni kantas ravantajn
kantojn, en ili oni parolas, ke mi estas belega. Sed kia beleco tio
\^ci estas, se neniu volas min cirka\u upreni kaj karese sin alpremi
al mi!

   Tiam la \^ciopova Mahadeva kreis la belan, fleksan serpenton. \^Gi
\^cir\-ka\u u\-pre\-na\-dis la belegan virinon kaj rampis apud
\^siaj piedoj. Duontagon la virino estis kontenta, poste ekenuis kaj
ekkriis:

 --- Ah, se mi efektive estus bela, aliaj penus min imiti. La
najtingalo kantas belege, kaj la kardelo \^gin imitas. Kredeble mi
ne estas jam tiel bela!

   La \^ciopova Mahadeva por kontentigo de la virino kreis la simion. La
simio imitis \^ciun movon de la virino, kaj la virino ses horojn
estis kontenta, sed poste kun larmoj \^si ekkriis:

 --- Mi estas tiel bela, tiel belega! Pri mi oni kantas, oni min
\^cirka\u uprenas, rampas apud miaj piedoj kaj min imitas. Oni min
admiras kaj min envias, tiel ke mi e\^c komencas timi. Kio do min
defendos, se oni ekvolos fari al mi de envio malbonon?

   Mahadeva kreis la fortan, potencan leonon. La leono gardis la
virinon. La virino tri horojn estis kontenta, sed post tri horoj
\^si ekkriis:

 --- Mi estas belega! Oni min karesas, mi --- neniun! Oni min amas, mi
 --- neniun! Mi ne povas ja ami tiun \^ci grandegan, teruran leonon,
por kiu mi sentas estimon kaj timon! Kaj en tiu \^ci sama minuto
anta\u u la virino, la\u u la volo de Mahadeva, aperis malgranda,
beleta hundeto.

 --- Kiel aminda besto! ekkriis la virino kaj komencis karesi la
hundeton. Kiel mi \^gin amas!

   Nun la virino havis \^cion, \^si pri nenio povis peti. Tio \^ci \^sin
ekkolerigis. Por ellasi la koleron, \^si ekbatis la hundeton, la
hundeto ekbojis kaj forkuris, \^si ekbatis la leonon --- la leono
ekmurmuregis kaj foriris --- \^si surpa\^sis per piedo sur la
serpenton, --- la serpento eksiblis kaj forrampis. La simio forkuris
kaj la birdoj forflugis, kiam la virino ekkriis je ili\dots

 --- Ho, mi malfeli\^ca! --- ekkriis la virino, rompante la manojn,
 --- oni min karesas, la\u udas, kiam mi estas en bona humoro, kaj
\^ciuj forkuras, kiam mi fari\^gas kolera! Mi sola! Ho, \^ciopova
Mahadeva! Je la lasta fojo mi vin petas: Kreu al mi tian
ekzista\^{\j}on, sur kiun mi povus ellasi la koleron, kiu ne havus
la kura\^gon forkuri de mi, kiam mi estas kolera, kiu estus devigita
pacience elportadi \^ciujn batojn!

 --- Mahadeva enpensi\^gis kaj --- kreis al \^si\dots la edzon!

\begin{flushright}
\footnotesize Tradukis A. \fsc{Grabowski}.
\end{flushright}

\smallrule{}
