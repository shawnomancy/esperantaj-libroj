\begin{center}
\footnotesize (de L. de \fsc{Beaufront}).
\end{center}

\koresp{Sinjoro M\dots, en Brian\c{c}on.}

   Kun la rekomendo de honorinda negocisto de via regiono, mi permesas
al mi adresi al vi mian prezaron \^generalan kaj samtempe kelkajn
specimenojn de la vendobjektoj paperaj kaj kartonaj, kiujn mi havas
ordinare en mia magazeno. Mi esperas, ke vi \^satos la kvaliton tute
rimarkindan de la komerca\^{\j}oj, kiujn mi proponas al vi, kaj ke
la kondi\^coj profitaj, en kiuj mi povas liveri ilin al vi,
decidigos vin doni al mi viajn mendojn.

   Vi povos anka\u u vidi en mia katalogo, ke, krom la paperoj, mi
anka\u u makleras pri \^ciuj komerca\^{\j}oj de la urbo Roubaix,
speciale pri la drapoj kaj teksa\^{\j}oj \^ciuspecaj. Miaj malnovaj
kaj konstantaj interrilatoj kun \^ciuj firmoj \^ci-tieaj donas al mi
la eblon trakti la negocojn de miaj mendantoj kun la tuta kompetento
dezirinda kaj per tia maniero, ke ili estu plene kontentaj.

   Esperante, ke vi favoros min per viaj mendoj, mi petas, Sinjoro, ke
vi akceptu mian sinceran saluton. \hfil R.

\koresp{Sinjoro R\dots, en Roubaix.}

   Mi ricevis vian leteron de la 10-a de la kuranta monato kun viaj
proponoj pri liveroj kaj la anonco pri viaj specimenoj, kiuj anka\u
u bone alvenis al mi. En aliaj cirkonstancoj fari uzon el tio estus
malfacile por mi, \^car mi ne havas motivon por forlasi miajn
kutimajn liverantojn, kiuj kontentigas min. Tamen, \^car la prezoj,
kiujn vi prezentas al mi, estas iom pli malkaraj, mi mendas de vi,
por provo, cent rismojn da papero konforma je specimeno n\textsuperscript{o} 1, kaj
tiom same da n\textsuperscript{o} 2.

   Se viaj liveroj estos kontentigaj, kiel mi esperas, mi sendos al vi
pli grandajn mendojn.

   Por la pago volu prezentigi kambion al mia kaso, post tridek tagoj,
kun 2\,\% da diskonto, konforme je viaj kondi\^coj.

   Koncerne vian proponon pri drapoj, mi beda\u uras, ke mi ne povas
profiti \^gin, \^car mi tute ne okupas min je tiuj komerca\^{\j}oj.

   Atendante vian raporton pri la ekspedo, mi prezentas al vi, Sinjoro,
mian sinceran saluton. \hfil M.

\koresp{Sinjoro H\dots, en Marseille.}

   Ni havas la honoron sciigi vin, ke ni ekspedas al vi hodia\u u,
malrapidire, keston enhavantan 60 botelojn da vinoj diversaj,
konforme je la mendo, kiun vi donis al nia voja\^gisto, S-ro L. Vi
trovos tie \^ci enfermitan la detalan fakturon. Volu akcepti \^gin
kiel \^gustan kaj enskribi la sumon en nian krediton.

   \^Ciam sindonemaj por viaj mendoj, ni petas vin, k. t. p.

\koresp{Sinjoro K\dots, en Parizo.}

   Ni sendas al vi tie \^ci aldonite la staton de via kalkulo fermita \^ce
la fino de decembro. Ni petas, ke vi ekzamenu \^gin kaj diru al ni,
\^cu ni konsentas.

\koresp{Sinjoro B\dots, en Lille.}

   Mi rapidas respondi je via letero kaj doni al vi la petitan informon
pri la firmo N. \^Gi \^guas tie \^ci la plej bonan reputacion. Mi
pensas, ke vi povas liveri al \^gi en plena trankvilo. Sen garantio
nek respondeco, kiel kutime.

   Tre sincere via\dots

\begin{center}
 KVITANCO
\end{center}

   (Mi certigas per tio \^ci, ke) mi ricevis de Sinjoro R. la sumon
de mil frankoj, kiujn al li mi pruntis por unu jaro en la dato 1\textsuperscript{a} de
januaro 1900, kune la sumon de dudek kvin frankoj kiel la procentojn
po 5\,\% de la dirita sumo dum la dua duonjaro.

   Aprobante por mil dudek kvin frankoj.

   Parizo, la l\textsuperscript{an} de januaro 1901.


\smallrule{}
