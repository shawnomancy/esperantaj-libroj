\emph{Unu advokato}, tre malgranda, venis ju\^gejon, por defendi la
aferon de sia kliento. Alia advokato, vidante lin, demandis, kiu li
estas. Tiu respondis. Tiam la unua ekkriis: --- Kio? tia malgranda
advokato? mi ja povas vin ka\^si en mian po\^son! --- Vi povas,
trankvile diris tiu, kaj tiam en via po\^so estos pli da sa\^go, ol
en la kapo. [M. Solovjev.]

\emph{Unu malgrandruso} veturis sur veturilo. Subite la veturilo
klini\^gis: \^gi trafis en kavon. La malgrandruso desaltis kaj
provis levi la veturilon, sed ne povis. Tiam li kolektas siajn
fortojn kaj vokas pro helpo la sanktan Nikolaon. Sed anka\u u tio
\^ci ne helpis. Tiam en malespero li krias: --- \^Ciuj sanktuloj
helpu! kaj pu\^sas la veturilon kun tia forto, ke \^gi renversi\^gas
sur alian flankon. --- Jen vi havas! li ekkriis kolere: vi ne devis
pu\^si \^ciuj kune! [M. Solovjev.]

\emph{Unu generalo} volis rigardi, kiel oni nutras la soldatojn.
Neatendite li venas en la soldatan kuirejon. Sur la korto li
renkontas du soldatojn, kiuj portas kaldronon. --- Haltu! li
ordonas: alportu kuleron! --- Sed, Via Generala Mo\^sto\dots
komencas la alkurinta adjutanto\dots --- Silentu! krias la generalo.
Li prenas kuleron. --- Levu la kovrilon! Lia ordono estas plenumata.
Li \^cerpas per la kulero la fluida\^{\j}on, provas kaj kun abomeno
kra\^cas. --- Kio tio \^ci estas? \^gi estas tia supo? \^gi estas
kota\^{\j}o!! --- Jes, Via Generala Mo\^sto, murmuretas la timigita
adjutanto, mi tion \^ci ja volis diri al vi. [M. Solovjev.]

\emph{Unu sprita mastro de budo} elpensis la sekvantan ruzan
\^sercon. Sur la pordo de sia budo li skribis: "eniro senpaga".
Granda amaso da publiko plenigis balda\u u la budon. Sed kiam la
gastoj volis eliri, ili renkontis du grandajn gardistojn kaj super
la pordo la surskribon: "eliro kostas kvindek centimojn". La
sukceso estis brilanta. Preska\u u \^ciuj pagis, la\u ute ridante
pro tiu \^ci ideo. [M. Solovjev.]

\emph{La re\^go Macedona} Filipo respondis al siaj amikoj, kiuj
diradis al li, ke la grekoj, kvankam \^cirka\u u\^sutataj per liaj
favoroj, tamen insultas lin: --- Kio rezultus, se mi agadus kun ili
pli malbone?! [J. Seleznev.]

\emph{Temistoklo} edzinigis sian filinon je homo tre bona, sed
malri\^ca, "\^car, li diris, mia filino pli volas homon sen
hava\^{\j}o, ol hava\^{\j}on sen homo". [J. Seleznev.]

\emph{La Roma oratoro} Cicero diris al unu homo, kiu diradis, ke lia
edzino havas 30 jarojn: "\^gi estas sendube vera, \^car jam 10
jarojn mi a\u udas tion \^ci de vi". [J. Seleznev.]

\emph{Unu el la eminentaj oficiroj} petis A\u uguston eksigi lin de
la servo kaj lasi al li la pension. --- Al mi ne la mono estas
bezona, regnestro, li diris, sed mi volus, ke \^ciuj sciu, ke mi
ricevis tiun \^ci favoron el viaj manoj. A\u ugusto respondis: ---
Bonege, diradu \^cie, ke mi kvaza\u u donas al vi pension, mi tion
\^ci ne neados. [J. Seleznev.]

\emph{Pastro postulis}, ke Lizandro konfesu al li sian plej \^cefan
pekon. --- \^Cu vi a\u u la dioj ordonas al mi malkovri mian animon?
demandis Lizandro. --- La dioj ordonas al vi! diris la pastro. ---
Bone, rediris Lizandro; foriru de tie \^ci, kaj kiam la dioj min
demandos, mi respondos al ili. [J. Seleznev.]

\emph{Helvetius} en sia verko "De l'esprit" rakontas la sekvantan
okazon: Unu edzo konvinki\^gis pri malfideleco de sia edzino kaj
komencis \^sin ripro\^ci. La edzino respondis, ke li diras
sensenca\^{\j}on. --- Sed mi vidis per miaj propraj okuloj! ekkriis
la edzo. --- Ha, jen kiel vi min amas, rediris la edzino: vi pli
kredas al viaj okuloj, ol al miaj vortoj! [N. Borovko.]

\emph{El unu prediko}. Vivis iam homo tre malbona kaj peka. Li
premis \^ciun, kiun li povis, li al neniu helpis iam e\^c per unu
centimo. Banante sin en la larmoj de multaj siaj oferoj, li tamen
vivis feli\^ce, kaj la tera justeco lin ne atingis. Sed jen li
mortis. \^Gojaj, ke ili liberi\^gis de li, la heredantoj faris al li
belegan enterigon. Sed apena\u u oni lin metis en la teron, la tero
tuj el\^{\j}etis returne la korpon de la pekulo. Vidante, ke la tero
ne volas akcepti la malbenitan korpon, oni decidis forbruligi \^gin
per fajro; sed anka\u u la fajro kun abomeno forsaltis de la korpo
kaj ne volis e\^c tu\^si \^gin. Ne povante al si helpi, oni
\^{\j}etis la korpon al hundoj, ke ili \^gin dis\^siru; sed anka\u u
la hundoj kun indigna bojado forkuris de la korpo kaj ne tu\^sis
\^gin. Oni \^{\j}etis la korpon en profundan mar\^con, por ke la
koto \^gin kovru kaj ripozigu, sed la korpo restis super la mar\^co
kaj e\^c unu kotero ne volis ali\^gi al la peka korpo\dots Nun, miaj
a\u uskultantoj, tiu \^ci terura ekzemplo servu al vi kiel instruo!
Estu bonaj, honestaj kaj piaj, kaj tiam vi povos esti tute certaj,
ke la tero vin prenos, fajro vin bruligos, hundoj vin dis\^siros kaj
koto kovros vin en granda amaso!

\emph{Konfeso de cigano}. Unu cigano venis al pastro peti lin, ke
tiu \^ci benu lian edzi\^gon. La pastro postulis, ke li anta\u u la
edzi\^go faru konfeson, kaj li difinis por tio \^ci la sekvantan
tagon. Veninte en la difinita tago al la pastro, la cigano vidis en
la kuirejo, kie neniu estis, barelon kun pizoj kaj en tiu \^ci
grandan pecon da porka sebo. Ciganoj entute estas grandaj amantoj de
porka sebo, tial anka\u u nia cigano ne povis sin deteni kaj,
forirante, li prenis \^gin kun si. Jam apud la pordo li vidis
ankora\u u \^capon, pendantan sur la muro, kaj, \^car lia propra
estis jam tute malnova kaj malbona, li prenis anka\u u la \^capon,
pensante en si: "estas tute egale, \^cu fari unu pekon a\u u du,
tiom pli, ke mi hodia\u u faros mian konfeson kaj puri\^gos de
\^ciuj pekoj." Forman\^ginte la sebon, li iris en la pre\^gejon,
por fari la konfeson. --- Nu, kiajn pekojn vi faris? demandas lin la
pastro. --- Hodia\u u mi vidis porkon en viaj pizoj kaj forigis
\^gin de tie, respondis la konfesanto. --- Tio \^ci tute ne estas
peko; kontra\u ue, vi faris, kiel vi devis fari. Kion vi ankora\u u
povas diri? --- Kiam mi estis hodia\u u en via domo, mi deprenis la
\^capon en la kuirejo. --- Anka\u u tio \^ci estas tute la\u udinda
ago. Se vi ne havas aliajn pekojn, iru en paco kaj estu trankvila!
La cigano foriris de la konfeso tute kontenta, \^car li ne ka\^sis
siajn pekojn kaj malgra\u u tio \^ci estis e\^c la\u udita de la
pastro. Kiam poste, reveninte hejmen, la pastro scii\^gis pri la
malapero de la sebo kaj la \^capo, li tuj divenis, kiu estis kulpa
en tio \^ci. --- Ha, li diris al si mem, li e\^c konfesis al mi
siajn pekojn kaj mi mem lin la\u udis\dots Sed estis jam tro malfrue
revenigi la perdita\^{\j}on. [I. Lojko].

\emph{Malkompreni\^go}. Prezentu al vi, amiko, mian \^cagrenon! mi
forgesis en domo la monujon\dots Pruntu al mi dek rublojn \^gis
morga\u u. --- Pardonu, mi \^gin ne povas, sed mi povas konsili al
vi certan rimedon, por ricevi tiun \^ci monon. --- Ho, mi dankas
vin, vi estas vera amiko\dots --- Jen dudek kopekoj; prenu
veturigiston kaj veturu hejmen, por preni la monujon. [I. Lojko.]

\emph{Forto de la scienco}. Mi komprenas, diris iu, ke oni povis
fari instrumentojn kaj esplori per ili la stelojn kaj planedojn, tio
\^ci estas farebla; sed kiel la instruituloj scii\^gis pri la nomo
de \^ciu stelo --- tion \^ci mi jam neniel povas kompreni.

\emph{Rimedo kontra\u u la Esperantismo}. En unu urbeto en gaja
societo oni parolis pri la \^holero kaj pri la novaj rimedoj
kontra\u u \^gi. En la societo sin trovis anka\u u unu kuracisto kaj
unu juna homo, kies sola celo en la vivo estis bone man\^gi, bone
trinki kaj amuzi fra\u ulinojn, kaj kiu pensis pri si, ke li estas
eksterordinare sprita, kaj amis \^cion kritiki, nenion sciante. Kiam
la kuracisto rakontadis pri la novaj rimedoj kontra\u u la \^holero,
la junulo interrompis lin kaj diris: --- Vi havas rimedojn kontra\u
u \^ciuj malsanoj; sed \^cu via scienco trovis jam anka\u u efikajn
rimedojn kontra\u u la plej nova malsano, kiu nun vasti\^gas en la
mondo --- kontra\u u la esperantismo? --- Kio estas esperantismo?
demandis unu fra\u ulino. --- Esperantismo, respondis la junulo kun
mieno de granda scienculo, estas ali\^gado al la nove elpensita
lingvo, kiu havas la nomon "Esperanto". Tiu \^ci malsano konsistas
en tio, ke homoj, kiuj ofte \^gis nun estis tute prudentaj, ricevas
atakon de frenezo kaj komencas lerni la novan lingvon; ili ricevas
varmegon, kaj en la deliro, ka\u uzita de tiu \^ci varmego, ili
komencas paroli pri la "estonteco", pri la "frateco de la
popoloj" k. t. p., k. t. p.; ekster tio ili ricevas la tre
dan\^geran pasion infekti per sia malsano kiel eble pli da aliaj
homoj, kaj en tio \^ci ku\^sas la plej granda dan\^gero de tiu \^ci
malsano. Oni diras, ke tiu \^ci malsano infektis jam multajn urbojn
kaj landojn. \^Cu vi elpensis jam ian rimedon kontra\u u tiu \^ci
malsano, sinjoro doktoro? --- Jes, respondis la kuracisto (kiu okaze
mem estis esperantisto), mi profunde esploris tiun \^ci malsanon kaj
mi havas kontra\u u gi tre efikan rimedon. La baciloj de tiu \^ci
malsano portas sin en la radioj de la suno, kaj en mallumo ili ne
povas vivi; tial la plej bona rimedo kontra\u u la esperantismo
estas: sidi \^ciam en profunda mallumo kaj allasi al si nenian
radion de la suno. [P. K.]

\emph{Suboficiro}. Homoj, \^ciam kura\^ge kaj diligente, kaj vi
\^cion atingos! La ovo de Kolumbo anka\u u ne estas metita en unu
tago!

\emph{\^Serca rifuzo}. La verkisto Gibeau, kiu \^ciam sin trovadis
en mona embaraso, skribis unu fojon al la \^cefo de la konata
fabriko de \^campano Roederer leteron kun la sekvanta enhavo:
"Sinjoro! Mi ne havas e\^c unu centimon kaj mi adoras la
\^campanon. Estu tiel bona kaj sendu al mi korbon da boteloj de via
dia trinka\^{\j}o. Kun \^gi mi esperas forgesi mian mizeron".
Apena\u u Roederer ricevis tiun \^ci leteron, li tuj sendis la
sekvantan respondon: "Via rimedo, por forgesi vian mizeron, nenion
ta\u ugas. La sen\^cesa kaj obstina prezentado de mia kalkulo
rememorigus vin \^ciuminute denove pri via mal\^goja situacio."

\emph{Mi ne komprenas}, kiel oni povas plendi \^ciam pri la tro
karaj viandokostoj! Mi kaj mia familio estas kune dektri personoj,
kaj tamen sufi\^cas por ni 1 1/2 funtoj da viando por tago. Mia
edzino ne man\^gas \^gin, miaj na\u u infanoj ne ricevas \^gin kaj
la du servantinoj ne bezonas \^gin. Jen en tia maniero la 1 1/2
funtoj tute sufi\^cas por mi por la tuta tago. [Dumpert].

\emph{Instruitulo kaj ri\^culo}. Unu instruitulo entreprenis gravan
sciencan la\-bo\-ron, sed ne havis la rimedojn por \^gin efektivigi.
Li vizitas unu ri\^culon kaj petas lin pri helpo. La ri\^culo
rifuzas, kaj inter ili komenci\^gas la sekvanta dialogo: {\sl
Ri\^culo}: Mirinde estas, ke nur la instruituloj \^ciam venadas al
la ri\^culoj kaj ke la lastaj, kontra\u ue, neniam venas al la
instruituloj. --- {\sl Instruitulo}: \^Car la instruituloj
komprenas, ke al ili mankas mono, sed neniam la ri\^culoj komprenas,
ke al ili mankas scienco.
--- {\sl Ri\^culo}: Kial do la ri\^culoj volonte oferas al blinduloj,
lamuloj kaj similaj malfeli\^culoj, sed ne amas helpi al malri\^caj
instruituloj? --- {\sl Instruitulo}: \^Car ili timas, ke fari\^gi en
la estonteco blindaj, lamaj k. t. p. ili povas iam mem, sed fari\^gi
iam instruitaj ili neniam timas! [N. Ku\^snir.]

\emph{Demando}. Kiu en la XVII-a centjaro portis la plej grandan
\^capelon? --- Tiu, kiu havis la plej grandan kapon.

\emph{Kura\^gulo}. Sinjoro! krias unu maljuna fra\u ulino el la
vagono al unu sinjoro, kiu volas tien eniri: tie \^ci estas la
vagono por sinjorinoj! --- Ho! respondas la sinjoro, enirante kaj
dismetante siajn paka\^{\j}ojn, mi ne estas timemulo!

\emph{Memfarita homo}. Jes, miaj sinjoroj, predikas sinjoro A.,
trinkante sian glason en gaja kolegaro, mia devizo \^ciam estis:
"la homo mem devas \^cion al si ellabori". Kiu mem al si helpas,
al tiu anka\u u Dio helpas! La 50\,000 frankojn, kiujn mi posedas,
neniu al mi donacis kaj anka\u u de neniu mi ilin heredis, --- ne,
mi mem ilin gajnis en la loterio!

\emph{Malfeli\^ca komercisto}. Juna homo, kiu en sia urbo havis
nenian okupon, venis Londonon, por ser\^ci helpon \^ce unu sia
parenco. Tiu \^ci lasta donis al li kelkan nombron da \^capoj kaj
konsilis al li stari sur la strato kaj vendi ilin. \^Goja, ke li nun
povos iom perlabori, la junulo prenis la \^capojn kaj iris kun ili
sur unu homplenan straton kaj sidi\^gis en unu oportuna anguleto.
Vespere li revenas al la parenco, kaj tiu \^ci demandas: Nu, \^cu vi
multe vendis? --- Ha, malgaje respondas la junulo, e\^c unu \^capon
mi ne vendis! --- \^Cu vi al neniu proponis? kion do vi faris la
tutan tagon? --- Mi tenis la \^capojn bone ka\^sitajn en mia korbo,
por ke la polvo ilin ne malbonigu, sed proponi al iu mi ne trovis
okazon en la da\u uro de la tuta tago, \^car el la granda amaso da
homoj, kiuj pasis anta\u u mi, \^ciuj havis jam \^capojn sur la
kapoj.

\emph{Apetito}. Kial vi petas almozojn? --- \^Car mi volas man\^gi,
mia bona sinjoro. --- Kial do vi ne laboras? --- Ha, kiam mi
laboras, mi ankora\u u pli volas man\^gi.

\emph{Niatempa amo}. Mi vin amas\dots --- Sed mi estas malri\^ca.
--- Pardonu, vi ne lasis min elparoli \^gis la fino\dots mi amas vin
ne tiel, por edzi\^gi je vi\dots --- Ha, ha, ha! mi volis nur
elprovi vin, mi havas grandegan kapitalon!---Pardonu, vi tamen ne
lasis min fini; mi amas vin ne tiel, por edzi\^gi je vi pro via
mono.

\emph{Zorgo}. Vi estas tiel malgaja, al vi kredeble faras zorgojn
viaj kreditoroj? --- Jes, miaj kreditoroj {\sl estontaj}, \^car mi
\^ciam zorgas, de kiu mi nun povos ankora\u u prunti.

\emph{Proceso}. En unu societo oni demandis pentriston, kiel oni
povas la plej reliefe prezenti du procesantojn, el kiuj unu gajnis
la proceson kaj la dua \^gin perdis. --- Mi pentrus la unuan en
\^cemizo kaj la duan nuda, respondis la pentristo.

\emph{El la pasintaj tempoj}. La imperiestro Pa\u ulo ordonis, ke
\^ciuj preterveturantaj sur la strato, renkontante lin, eliru el la
kale\^so kaj donu al li la re\^gan honoron per saluto. Escepto ne
estis farita anka\u u por sinjorinoj. Unu fojon la imperiestro
rajdis promene sur la strato en kota tago. Sur la strato montri\^gas
rapide veturanta kale\^so, en kiu sidas elegante vestita sinjorino.
Venante preter la imperiestro, la veturigisto haltigas la
\^cevalojn, kaj la sinjorino rapide elrampas el la kale\^so. La
imperiestro, doma\^gante \^sian ri\^can veston, ekkriis:
"sidi\^gu!" La sinjorino ektimigita rapidas plenumi la ordonon kaj
momente sidi\^gas\dots sur la koton de la strato. Pa\u ulo rapide
desaltis de la \^cevalo, alkuris al la sinjorino kaj, preninte \^sin
sub la brakon, alkondukis kaj sidigis \^sin en la kale\^son. Oni
diras, ke al sia ordono la imperiestro poste faris rimarkon, ke
virinoj estas liberaj de tia donado de honoro. [M. Solovjev.]

\emph{Brava voja\^gisto}. Voja\^gisto, rakontante kelkajn el siaj
aventuroj, diris al la societo, ke li kaj lia servanto kurigis 50
sova\^gajn arabojn; kaj kiam tio \^ci ekscitis miron, li aldonis, ke
\^gi ne estas mirinda, "\^car, li diris, ni eniris, kaj ili kuris
post ni." [B. G. Jonson.]

\emph{Du tajloroj}. Barono N., reveninte hejmen de sia somera
veturado en diversaj landoj, venigis kun si kelkajn metrojn da
\^stofo por surtuto. La \^stofo estis la plej kara, kiun li povis
trovi en kompetentaj magazenoj, kaj kompreneble \^gi estis bonspeca
kaj belkolora. Kiam la barono elripozis de la voja\^go, kiun oni en
tiu tempo faradis per \^cevaloj, li vokis sian konatan tajloron, por
kudri la surtuton. Tiu \^ci mezuris la longecon kaj dikecon de la
barono, mezuris la \^stofon, signis per kreto unu fojon kaj duan kaj
fine, kvaza\u u beda\u urante, diris: --- Via barona mo\^sto! el tiu
\^ci \^stofo surtuto ne povas esti eltran\^cita, \^car por gi
malsufi\^cas ankora\u u unu metro\dots La barono, kiu sciis, ke la
\^stofo estas venigita de malproksime kaj a\^ceti de \^gi ankora\u u
metron estas preska\u u io neebla, \^cagreni\^gis. Post kelkaj tagoj
li ekpensis kaj vokis alian, novan tajloron kaj montris al li la
\^stofon. Tiu \^ci pripensis, mezuris kaj fine li promesis ellabori
oportunan surtuton. Kelkaj tagoj pasis, kaj la tajloro efektive
plenumis sian vorton: li alportis pretan surtuton sufi\^ce vastan
kaj tiel longan, ke li devis ankora\u u subtran\^ci. Kun mirego la
barono vestis la surtuton, kaj \^goja li bone rekompencis la
tajloron, kiu iris domen. Post kelkaj semajnoj, en unu tago de
festo, la barono, promenante sur la strato, vidis knabon, havantan
la a\^gon de \^cirka\u u sep jaroj, kiu portis manteleton el tia
sama \^stofo, kiel lia surtuto. Li ekkoleris, sed lia miro estis pli
granda, ol la kolero. Demandinte, kies estas tiu \^ci knabo, li
scii\^gis, ke li estas filo de la tajloro, kiu kudris lian surtuton.
Veninte domen li sendis voki tiun \^ci tajloron. Tiu \^ci venis, kaj
la barono diris al li: --- Ke vi \^stelis pecon da \^stofo kaj ke vi
faris al via filo veston el tiu sama \^stofo, kiel mia surtuto, mi
pardonas al vi; sed diru al mi: mi ja vidas, ke de mia \^stofo
ankora\u u restis, kial do mia konata tajloro, kiu bone komprenas
sian laboron, diris, ke neniel \^gi povas esti sufi\^ca por surtuto?
La tajloro rediris: --- Tute simple, via barona mo\^sto! \^car lia
filo havas jam la a\^gon de dekdu jaroj. [E. Ne\u umark.]

\emph{La ju\^gisto de Reading}. El la anekdotoj, kiuj rakontas pri
Henriko VIII de Anglujo, estas nenia, kiu montrus lin de aminda
flanko; sed nenia estas tiel karakteriza, kiel la sekvanta. La
re\^go unu tagon sur la \^caso perdis la vojon kaj venis \^cirka\u u
tagmezo en la vila\^gon Reading. Malsata li iris al la ju\^gisto kaj
petis man\^gon kaj trinkon. La ju\^gisto, kiu prenis lin por simpla
gvardiano, akceptis lin kore kaj donis al li sur la tablon bovan
langon kaj kru\^con da biero. La re\^go man\^gis kun apetito, kaj la
mastro amike rimarkis: --- Mi donus cent funtojn da sterlingoj, se
bova lango povus havi por mi tian bonan guston kiel por vi. Post
paso de unu semajno la ju\^gisto estis vokita Londonon kaj metita en
malliberejon. En da\u uro de ok tagoj li ricevadis nur panon kaj
akvon, fine la na\u uan tagon oni donas al li langon de bovo kaj
kru\^con da biero. La malliberulo esprimas sian miron; sed la
gardisto de la malliberejo restas muta por \^ciuj liaj demandoj.
Tiel la ju\^gisto, ne ricevinte klarigon, komencas man\^gi la langon
de bovo, kiu efektive havas por li tre bonan guston. Subite pordo
malfermi\^gas, kaj la re\^go eniras. --- Mi estis via kuracisto,
diras Henriko VIII al la surprizita ju\^gisto; mi sanigis vian
malfortan stomakon. Pagu al mi sekve mian honorarion de cent funtoj
da sterlingoj, kiun vi mem difinis, alie vi restos tie \^ci la tutan
vivon. La ju\^gisto pagis kaj forlasis Londonon. Kion li pensis pri
la dankemeco de la re\^go, la historio al ni ne rakontas.

\emph{Malgrandruso kaj soldato}. Oni diras, ke la soldatoj estas
sa\^gaj homoj --- rakontas malgrandrusa vila\^gano, veninta el
Kievo, kien li iris pre\^gi, al siaj kunvila\^ganoj --- sed tio \^ci
estas mensogo; jen mi donos al vi ekzemplon. Unu fojon mi iras en
Kievo sur la strato; ekvidinte tre altan domon, mi levis la kapon,
rigardante \^gian supron. --- Kion vi rigardas, hundinido? ekkriis
alsaltinta al mi soldato. --- Mi kalkulas la korvojn, kiuj staras
sur la tegmento, mi timege respondis, kaptante la \^capon en la
manon. --- Kiom do vi kunkalkulis, hundinido? li denove ekkriis. ---
Dek tri, mi respondis. --- Pagu do tuj dek tri rublojn, hundinido,
\^car, se vi ne pagos, mi tuj vin transdonos al la guberniestro, li
kolere ekkriis. Mi estis \^goja liberi\^gi de li kun tia malgranda
sumo, kaj, paginte, mi forkuris, e\^c ne rerigardante posten. Sed vi
pensas, ke mi kunkalkulis nur dek tri korvojn? ho, ho, mi trompis la
soldaton: mi kunkalkulis pli ol kvardek korvojn kaj pagis nur dek
tri rublojn! [N. Ku\^snir.]

\emph{Tran\^canta komplimento}. A: Kiel pla\^cas al vi mia
versa\^{\j}o? Ne vere, feli\^ca ideo? --- B: Jes, jes, la ideo estas
jam per tio feli\^ca, ke \^gi {\sl elsaltis} el via kapo.

\emph{El la historio}. A: Kial Hanibalo iris trans la Alpojn? --- B:
\^Car tiam la tunelo ne estis ankora\u u preta.

\emph{En la ju\^gejo}. {\sl Ju\^ganto}: Kiel oni vidas el la
preparaj aktoj, la \^stelado estas via {\sl profesio}! --- {\sl
Ju\^gato}: Kion do vi volas, sinjoro ju\^ganto, ke mi \^steladu por
{\sl plezuro}?

\emph{Telegrama stilo}. Juna edzino naskis filinon. La edzo volis
sciigi la patron de sia edzino pri tiu \^ci fakto, aldonante, ke la
fakto havis lokon en la sepa horo matene kaj ke poste per letero li
skribos pli detale. Li telegrafas: "Hodia\u u matene sepa filino
naskita. Poste pli."

\emph{La pafanto}. {\sl Le\u utenanto}: Ni supozu, ke la malamiko
staras tie \^ci anta\u u la arbo. La\u u la komando "tri" vi
ekpafos sur la arbon. Sekve atentu: unu --- du --- tri!\dots Ha,
mallerta urso, vi pafis ja {\sl preter} la arbon! --- {\sl Rekruto}:
Nu, kion do \^gi malutilas, sinjoro le\u utenanto? Kiam la malamiko
efektive venos, tiam ja certe ne \^ciuj staros {\sl anta\u u} la
arbo, kelkaj staros anka\u u {\sl apud} la arbo!

\emph{Verkistino}: --- Ha, kara Julio, se mia artikolo "Kontra\u u
la ornami\^gamo de la virinoj" estos presita, tiam mi por la
honorario tuj a\^cetos al mi plu\^san matenveston kun pelta
\^cirka\u ukudro.

\emph{Juna mastrino} (aran\^gante kun la kuiristino la
man\^gotabelon por vespera kolekti\^go): Por la dua man\^go ni
prenos angilon! --- {\sl Kuiristino}: Kiom vi ordonas, ke mi prenu,
sinjorino? --- {\sl Mastrino}: Mi pensas, ke estos sufi\^ce
\^cirka\u u dek metroj!

\emph{Mastrino} (invitante la gaston man\^gi): --- Mi petas, prenu!
Estu tute kiel dome; mi amas, ke miaj gastoj estu dome!

\emph{Fra\u ulino}: Vidu, estimata sinjorino, tio \^ci estas \^cio,
kion ni volas! Ne vere, nun vi jam scias, kio estas la tiel nomata
virina demando? --- {\sl Sinjorino}: Mi konas nur unu virinan
demandon, kaj tiu \^ci estas: \^cu li jam estas edzigita?

\emph{Raporto}. En Londono unu krimulo devis esti mortigita. Anta\u
u sia morto li malsani\^gis kaj oni lin sendis en la malsanulejon.
Post unu semajno la \^cefa kuracisto skribis al la ju\^gejo: "La
arestito N. sani\^gis, kaj oni povas lin mortigi sen malutilo por
lia sano." [S. B-n.]

\emph{La \^suldo}. La malgranda Niko\^cjo, promenante kun sia
patrino sur la bordo de rivero, per nesingardo tien enfalis. Unu
junulo, vidinte tion \^ci, kura\^ge sin \^{\j}etis en la akvon kaj
eltiris la knabon. La patrino dankas la savinton: --- Ha,
sinjoro!\dots mi ne scias, kiel vin danki!\dots vi savis mian
fileton!\dots --- Ho, sinjorino, li rediris, \^gi ne estas inda je
danko, \^gi estas \^suldo de \^ciu. --- Ha, kion vi diras!\dots \^Cu
vi estos tiel bona anka\u u eltiri lian pajlan \^capon? Jen \^gi
na\^gas. [Malgranda poeteto.]

\emph{Konjektemeco}. En la pafado unu soldato memvolulo neniel povis
trafi la celon. La oficiro prenas lian pafilon kaj, rigardante lin
kun ripro\^co, diras: --- Nu, vi estus trafema pafisto! La oficiro
ekcelas. Paf!\dots maltrafo. --- Jen kiel vi pafas! Li ekcelas la
duan fojon --- paf!\dots denove maltrafo. --- Jen kiel vi pafas! La
oficiro ekcelas la trian fojon---paf! li trafis. --- Tiel oni devas
pafi, diras la oficiro, redonante la pafilon. [Malgranda poeteto.]

\emph{Advokato kaj kuracisto} disputis pri tio, kiu el ili staras
pli alte. Oni demandis tiam la poeton Piron'on kaj petis lin solvi
la disputon. --- La fripono iras \^ciam anta\u ue, kaj la
ekzekutisto lin sekvas, estis la respondo.

\emph{Jonathan Swift kaj la servanto}. La glora a\u utoro de la
"Voja\^goj de Gullivero" estis granda amanto de ple\u uronektoj
(speco de fi\^soj = germana "Steinbutte"). Tion \^ci sciis lia
adoranto kaj bonfaranto, lordo Temple, kiu ofte sendadis al li
grandajn fi\^sojn de tiu \^ci speco. La satiristo \^ciam kun plezuro
ilin akceptadis, sed al la alportanta servanto li neniam donadis
trinkmonon. Tio \^ci kolerigis la servanton, kaj, kiam li unu fojon
denove devis iri al Swift, por alporti al li grandan ple\u
uronekton, li montris sian nekontentecon iom maldelikate. Tuj Swift
levi\^gis, sidigis la servanton sur sian se\^gon kaj diris: --- Mia
amiko, vi ne scias, kiel oni devas plenumi komision; mi montros
\^gin al vi. Swift prenis la ple\u uronekton kaj alpa\^sis malrapide
kaj respektege al la servanto. --- Mia sinjoro, li diris, salutas
vin kaj petas vin, sinjoro Dekano, akcepti tiun \^ci malgrandan
donacon. Nun la servanto sin levis kaj respondis: --- Mi petas tre
kore danki vian sinjoron. Kun tiuj \^ci vortoj li metis la manon en
la po\^son, eltiris duonon da krono kaj premis \^gin al Jonathan
Swift en la manon. --- Goddam! ekkriis tiu \^ci, vi estas pli
sa\^ga, ol mi pensis. Mi rimarkos al mi la instruon kaj jam pli ne
forgesos pri la trinkmono.

\emph{Kiam Swift} prenadis novajn servantinojn, li \^ciam diradis al
ili, ke ili en lia domo devas anta\u u \^cio observadi du aferojn:
unue, fermi post si la pordon, kiam ili venas en \^cambron, kaj due
--- denove fermi la pordon, kiam ili eliras. Unu fojon unu servantino
petis de li la permeson iri al la festo de edzi\^go de \^sia
fratino. --- Tre volonte, diris Swift, mi e\^c donos al vi \^cevalon
kaj servanton por akompani, kaj vi povas amba\u u kune veturi. Tute
ekster si de \^gojo, la servantino, elirante, lasis la pordon ne
fermita. Kvaronon da horo post ilia forveturo Swift ordonis seli
alian \^cevalon kaj sendis sur \^gi rapide alian servanton kun la
ordono revenigi ilin. Tiu \^ci trovis ilin en la mezo de vojo, kaj
\^cu ili volis a\u u ne --- ili devis veturi returne. Tute
depremita, la knabino eniris en la \^cambron de sia sinjoro kaj
demandis, kion li ordonos. --- Nenion pli ol nur ke vi fermu post vi
la pordon, li diris kaj lasis \^sin post tio \^ci denove forveturi.

\emph{\^Ce tagman\^go}. --- Mirinde! mi neniam povas man\^gi bonan
pecon da meleagro\dots --- Kial do? \^cu vi havas tian strangan
stomakon? --- Ne, sed \^car mia edzino \^gin \^ciam forman\^gas.

\emph{En klubo}. --- Pardonu, sinjoro, \^cu ne kun sinjoro Fredo mi
havas la honoron paroli? --- Ne. --- Mi tiel anka\u u tuj pensis al
mi! Vi ne estas e\^c simila je sinjoro Fredo.

\emph{Instruita mimikisto}. En la tempo de la servuta rajto S.
Peterburgon alveturis unu instruita mimikisto, kiu publike anoncis,
ke li povas per diversaj signoj de manoj paroladi ne sole pri
ordinaraj objektoj de la \^ciutaga vivo, sed e\^c pri objektoj de
filozofio, kaj li fanfaronis, ke neniu povos kun li e\^c iom
interparoli. En tiu sama tempo en S. Peterburgo estis iu bienhavanto
kun sia servutulo. Eksciinte pri la anonco de la mimikisto, la
bienhavanto sendis al li sian servanton, al kiu li ordonis nepre
paroli kun la mimikisto. Tiu venis. --- \^Cu vi efektive povas
libere paroli per mimiko? oni demandis la vila\^ganon. --- Mi e\^c
ne scias, kio \^gi estas "per mimiko", li respondis. Je tiu \^ci
respondo \^ciuj multe ridis, tamen oni klarigis al la vila\^gano, ke
"paroli per mimiko" estas tio sama, kio "paroli per signoj de
manoj". --- Ha, mi komprenas! diris la vila\^gano, per manoj\dots
per manoj mi povas paroli tre bone: mi tiel ofte parolas, \^car mi
havas mutan fraton.
 --- Kiam la instruitulo eksciis, kun kiu li devos paroli per mimiko,
li forte ekridis kaj anta\u u atestantoj promesis doni al la
vila\^gano tricent rublojn, se li efektive povos klari\^gadi per
mimiko a\u u almena\u u ion kompreni el tio, kion li, instruitulo,
al li parolos; tiun \^ci promeson la instruitulo konsentis e\^c doni
per skribo. La bienhavanto, al kiu apartenis la nova mimikisto,
anka\u u promesis doni al li plenan liberon kaj dudek orajn
monerojn, se li kompreni\^gos kun la instruitulo per mimiko. --- Sed
se vi ne povos interparoli, vi ricevos dudek kvin bastonojn, aldonis
la bienhavanto, \^cu vi volas a\u u ne? --- Jes, mi konsentas kun
plezuro, respondis la vila\^gano. Kaj jen en difinita horo la du
mimikistoj eniris en la salonon kaj sidi\^gis unu kontra\u u la dua.
Tie \^ci anka\u u sidis multegaj atestantoj, kun sciemo atendante la
interesan paroladon. La instruitulo komencis la unua. Dezirante
provi la scion de la vila\^gano, li montris al li unu fingron (tio
\^ci en la mimiko de la instruitulo devis signifi: "Dio estas
unu"). La vila\^gano, respondante, montris du fingrojn. La
instruitulo ekmiris, sed li pensis, ke tio \^ci estas simpla
okazeco. Li montris al la vila\^gano manplaton (signo de regno); la
vila\^gano tuj montris al li pugnon. La instruitulo kaj \^ciuj
alestantoj estis tre mirigitaj, precipe la instruitulo. Li
ru\^gi\^gis de \^cagreno kaj kun kolero li montris supren kaj
malsupren (tio \^ci signifas: "Dio estas en la \^cielo kaj sur la
tero"). \^Ciuj ekrigardis la vila\^ganon. Li en tiu sama minuto
eltiris anta\u uen la manojn kaj faris signon, kvaza\u u li ion
trenas. \^Ciuj estis senfine mirigitaj, la instruitulo e\^c
eksaltis; li sentis sin mortigita! Li rapide elprenis el la po\^so
monujon kaj tuj elkalkulis al la vila\^gano tricent rublojn, kiujn
lia kunparolanto prenis kaj trankvile ka\^sis en sian po\^son. Neniu
komprenis, pri kio parolis la mimikistoj, kaj tial oni decidis
demandi aparte la instruitulon kaj la vila\^ganon, pri kio ili
interparolis. Anta\u ue estis forigita la vila\^gano. Tre
ekinteresitaj, la alestantoj kun sciemo kaj malpacienco atendis,
kion diros la instruitulo. Tiu \^ci rapidis trankviligi ilian
malpaciencon: --- Mi neniam ankora\u u renkontis homon, kiu povus
tiel libere klari\^gi per mimiko! Mi mem tre ofte pensas, anta\u u
ol mi ion diras, kvankam mi lernis tiun \^ci malfacilan sciencon
preska\u u tridek jarojn! Tio \^ci estas tre mirinda! simpla, tute
ne instruita vila\^gano, kaj tiel libere respondas per mimiko tiajn
pure filozofiajn demandojn! Tre mirinde, tre mirinde! --- Sed diru
do al ni, sinjoro, pri kio vi interparolis? --- Vidu, sinjoroj:
unue, en la komenco de la parolado, mi diris al la vila\^gano, ke
Dio estas unu. Li al mi respondis, ke Dio estas kvankam unu, sed Li
tamen estas duobla, \^car Li estas samtempe Dio kaj homo. Tiun \^ci
ideon la vila\^gano esprimis per signo de du fingroj. Mi pensis, ke
li okaze respondis al mi vere. Tial mi \^san\^gis la temon de la
parolado, kaj mi montris al li manplaton, esprimante per tiu \^ci
signo, ke Rusujo estas bonkonstruita regno, --- kaj li respondis al
mi, ke Rusujo estas regno unupotenca kaj estas sub la sceptro de
Imperiestro (pugno signifas sceptron). Tiam mi ree min turnis al la
unua temo, kaj montrinte supren kaj malsupren, mi diris per tiu \^ci
signo, ke Dio estas en la \^cielo kaj sur la tero; kaj li flnis mian
penson, dirante, ke Dio estas \^cie. \^Cio tio \^ci min konvinkis,
ke tiu \^ci simpla vila\^gano, tiu \^ci malklerulo, tute libere
povas paroli per mimiko, kaj mi anta\u u li min klinas. --- Oni
vokis la vila\^ganon kaj anka\u u lin petis rakonti, pri kio li
parolis kun la instruitulo. "Ni preska\u u nenion parolis, diris la
vila\^gano; ni nur minacis unu al la dua: li al mi, kaj mi al li.
Mi, estimataj sinjoroj, povas tute bone paroli per signoj de manoj,
\^car mi havas mutan fraton, kun kiu mi \^ciam tiel interparolas.
Tiu \^ci sinjoro (li montris la mimikiston) montris al mi fingron,
kaj mi tuj komprenis, ke li volas al mi elpiki okulon; kaj mi
respondis, ke mi povas elpiki al li e\^c amba\u u okulojn (du
fingroj). La sinjoro kredeble ekkoleris pro tiu \^ci respondo kaj
diris al mi, ke li donos al mi survangon (manplato); kaj mi
respondis, ke mi mem preferas pugnon, kiu iafoje tre bone efikas.
Tiam la sinjoro, tute ekkolerinte kaj dezirante kredeble min timigi,
diris, ke li min supren levos kaj poste \^{\j}etos al la tero. Mi
respondis al li, ke se li provus tion \^ci fari, mi prenus lin per
la haroj kaj trenus sur la planko. Tio \^ci estas \^cio, pri kio ni
parolis". En la salono eksonis la\u uta ridego de \^ciuj
alestantoj, kaj la malfeli\^ca instruita mimikisto tute ne sciis,
kion fari: tre konfuzita, li nur mordis la lipojn kaj silentis. En
tiu \^ci sama tago li forveturis el S. Peterburgo. [V. Devjatnin.]

\emph{Profesoro} de zoologio N. tre ne amis, kiam la studentoj
malfruis al la komenco de la lekcio kaj tiam, interrompante sian
legadon, li \^ciam esprimadis sian malplezuron al la malfruinta
studento. Unu fojon, kiam la profesoro legis pri \^cevalo, eniris en
la legejon iu malfruinta studento. Al la miro de la studentoj,
kontra\u u sia kutimo la profesoro nenion diris al la studento kaj
da\u urigis sian legadon. Fininte la legadon pri \^cevalo, li diris:
--- Nun, sinjoroj, post la "\^cevalo" ni transiru al la "azeno", kaj,
turninte sin al la malfruinta, li diris: Mi petas, sidi\^gu. --- Ne
maltrankviligu vin, sinjoro profesoro, respondis la studento, mi
povas a\u uskulti azenon anka\u u starante. [A. Gruenfeld.]

\emph{Inter bopatroj}. \^Cu vi estas kontenta je via nova bofilo?
--- Ho, ne forte, mi faris tre malbonan elekton. --- Per kio? ---
Vidu, mia kara, li ne povas trinki, li ne scias ludi kartojn, kaj al
\^cio li havas ankora\u u mirindan talenton de parolado. --- Kion do
vi volas? mi vin ne komprenas! Tio \^ci estas ja \^ciuj nur tre
bonaj ecoj! --- Sed vidu: li ne povas trinki kaj trinkas, li ne
scias ludi kartojn kaj ludas; se li ne estus granda parolanto, tiam
mi sola scius, ke li estas malsa\^ga, --- nun la tuta mondo tion
\^ci scias.

\emph{Avarulo}, elirante el la pre\^gejo, kie la pastro parolis pri
bonfarado, diris: --- La prediko tiel min tu\^sis, ke mi mem estas
preta peti almozojn. [P. Ko\^cergov].

\emph{Gradeco de ripro\^coj}. La ministro diris al la direktoro: La
opero iris tre bone, nur la \^horoj lasis ion por deziri. --- La
direktoro iras al la re\^gisoro: --- Sinjoro re\^gisoro, mi havas
ka\u uzon esti nekontenta je la \^horo; mankas al \^gi energio;
mirinde estus, se lia ministra mo\^sto ne akceptus \^gin de malbona
flanko. La re\^gisoro iras al la kapelestro: --- Sinjoro kapelestro,
mi devas diri al vi, ke la \^horoj estis hodia\u u ekstreme
malbonaj, tiel malbonaj, ke mi timis la falon de la opero. Mi vin
petas, observu, ke \^gi estonte iru pli bone. La ministro severe
punos. La kapelestro rapidas al la direktanto de la \^horoj. --- La
\^horoj estis hodia\u u absolute abomenaj. Unu rapidas, alia restas,
unu kantas tro alte, alia tro malalte, tute kiel strataj buboj. La
re\^gisoro orde vin regalos kaj bone faros. La sekvantan tagon la
direktanto, alestante \^ce la provo de la \^horanoj, diras: --- Vi
kriegis hiera\u u kiel brutaro; estas honto kaj abomeno! \^Cu vi ne
havas orelojn, ne havas ideon pri takto, ke vi kantas kiel
sova\^guloj? Mi miras, kial la kapelestro ne \^{\j}etis al vi la
notojn en la viza\^gojn kaj ne forpelis \^ciujn al la diablo! Mi
ripetas, vi kantis kiel brutaro, kaj se tiel okazos ankora\u u unu
fojon, mi vin \^ciujn dispelos kiel bestojn! [P. Ko\^cergov.]

\emph{Francisko I}, re\^go de Francujo vizitis la papon Leonon X. La
re\^go, mirigita de la lukso de la kortego de la papo, diris, ke
la\u u la rakonto de la biblio la religiaj kondukantoj vivis
malri\^ce kaj simple. --- Vero, respondis la papo, sed tio \^ci
estis tiam, kiam la re\^goj pa\^stis brutarojn. [P. Ko\^cergov.]

\emph{Kion oni amas plej multe}? Havante unu jaron --- sian
nutristinon; kvin jarojn --- la patrineton; dek jarojn --- la lernan
libertempon; dekses jarojn --- la liberecon; dudek jarojn --- sian
amatinon; tridek jarojn --- sian edzinon; kvardek jarojn --- siajn
infanojn; sesdek jarojn --- sian oportunecon; en \^ciuj tempoj ---
sin mem. [K. O. S-m.]

\emph{Senkulpi\^go. Sinjorino}, forlasante sian servantinon: Jes, mi
beda\u uras, sed mi ne povas repreni el via serva libro mian ateston
pri via nepuremeco. Rigardu do mem ekzemple la malpura\^{\j}on de
mu\^soj tie \^ci en la angulo. --- {\sl Servantino}: Tio \^ci ja
estas ne mia kulpo! tiu \^ci malpura\^{\j}o jam estis, kiam mi
anta\u u jaro komencis mian servadon \^ce vi.

\emph{La le\^gisto}: Despota urbestro ofendis unu urbanon. La urbano
rakontis sian malfeli\^con al unu el siaj konatoj, kiu pretendis, ke
li estas granda konanto de la le\^goj. Indigne ekkriis la konato:
--- Kiel li kura\^gis tion \^ci fari! mi tuj iros al la urbestro kaj
montros al li, ke li ne konas la le\^gojn, kaj mi devigos lin, ke li
sur la genuoj petu de vi pardonon! Ili amba\u u iris al la urbestro;
la le\^gisto eniris en la lo\^gejon, la ofendito jam preparadis sin,
kiel li akceptos la pardonpetantan urbestron, tamen li ne estis
sufi\^ce kura\^ga kaj restis post la pordo. Post kelkaj minutoj la
kura\^gulo eliras. --- Nu, kia rezultato? demandis la atendanta. ---
Ha, respondis la kura\^gulo, mi tute ne atendis, ke li estos tia
maldelikatulo! Prezentu al vi, kiam mi komencis mian ripro\^can
parolon, li tuj volis doni al mi du survangojn! --- Kiel vi scias,
ke li volis tion \^ci fari? --- Se li ne volus, li ja tion \^ci ne
farus, kaj se li faris, tio \^ci ja montras tute sendube, ke li tion
\^ci volis. --- Kaj vi silentis? --- Enirinte al li, mi forgesis lin
bone titoli, kaj ekzistas le\^go, ke se oni iun salutas per pli
malalta titolo, ol li havas, li havas la rajton doni al vi tri
survangojn. --- Sed vi ja ricevis nur du! --- Ha, mi forgesis; vidu,
ekzistas le\^go, ke se la punata falas sur la teron, li jam pli da
survangoj ne devas ricevi. --- Kial do vi ne falis sur la teron tuj
post la unua survango? --- Ekzistas alia le\^go, ke tiel longe, kiel
oni \^ce la survangoj havas ankora\u u la forton stari sur la
piedoj, oni devas stari. --- Kaj kio estos kun mia ofendo? --- Iru
mem kaj provu \^sovi al li en la manon kelkan sumon da mono, eble li
al vi pardonos. Mi nenion kun li povas fari, \^car mi vidas, ke li
mem scias \^ciujn le\^gojn parkere kaj havas ilin \^ciam sub la
mano.

\emph{La inkujo}: Knabo a\^cetis boteleton da inko kaj rapidis
domen. En la mezo de la strato la inkujo elfalis el lia mano kaj
rompi\^gis kaj la tuta inko elver\^si\^gis sur la teron. Konfuzite
la knabo staras kaj malgaje rigardas la elver\^sitan inkon. --- Nu,
kion vi staras, mia knabo? diris unu preteriranto; vi jam elskribis
la tutan inkon kaj nun vi povas trankvilo iri domen!

\emph{Principo. Luigantino de \^cambro}: Anta\u u ol vi
enlo\^gi\^gas en mian \^cambron, mi devas al vi rimarki, ke mi amas,
ke oni akurate pagu la luan pagon. --- {\sl Studento}: Tio \^ci
estas anka\u u mia principo; a\u u akurate, a\u u tute ne!

\emph{Malbona memoro}. De kio vi havas tian embarasan mienon? ---
Ha, prezentu al vi mian teruran situacion: mi petis hiera\u u fra\u
ulinon Marion pri \^sia mano kaj mi ne memoras, kion \^si al mi
respondis: jes a\u u ne\dots

\emph{Neniom malhelpas}. Sinjoro venas domen laca kaj malsata. Lia
sesdekjara kuiristino donas al li la tagman\^gon, kiu estas
preparita tre malbone. --- {\sl Sinjoro}: A\u uskultu, Antonino,
diru al mi malka\^se, kial vi ne prenas oficon de nutristino? \^gi
estas ja multe pli enspeza, ol la ofico de kuiristino. --- {\sl
Kuiristino}: Ha, sinjoro, vi \^sercas! Kiel do \^gi estus ebla? \^cu
mi povus nun esti nutristino? --- {\sl Sinjoro}: Neniom malhelpas!
Kuiristino vi ja anka\u u tute ne povas esti kaj tamen estas!

\emph{Bonigo. Fian\^cino}: Diru, \^cu estas vero tio, kion mia
patrino al mi diris? --- {\sl Fian\^co}: Kion do? --- {\sl
Fian\^cino}: Ke vi edzi\^gas je mi, \^car mi poste havos grandan
sumon da mono. --- {\sl Fian\^co}: Sed mia infano, kontra\u ue!
estus al mi e\^c pli agrable, se vi \^gin havus jam nun.

\emph{Tro granda postulo}. --- For! al sanaj kaj fortaj mi almozon
ne donas. --- Kiel vi volas; pro viaj kelkaj centimoj mi al mi
piedon ne rompos.

\emph{Teruraj infanoj}: Onklino parolas kun malgranda nevo, kiu
\^{\j}us venis el la lernejo. --- Nu, \^cu vi lernis aritmetikon?
--- Certe! --- Kion do vi lernis? --- Deprenadon. --- Aha! sekve se mi
diros al vi, en kiu jaro mi naski\^gis, \^cu vi povos diri, kian
a\^gon mi havas? --- Oho! tiajn grandajn nombrojn ni en la lernejo
ankora\u u ne lernis.

\emph{Mirinda\^{\j}o en la medicino}. En unu societo oni parolis pri
medicino. Unu el la alestantoj diris, ke la unua montrilo en la plej
multaj malsanoj estas la lango kaj se la lango estas kovrita, tio
\^ci jam montras, ke la homo ne estas sana. --- Kaj tamen, rimarkis
alia, mi konas unu sinjorinon, kies lango \^ciam estas kovrita per
tre dika tavolo da \^cikanoj kaj kalumnioj kaj \^si tamen estas tute
sana.

\emph{Ne volonte. Ju\^ganto}: Viaj respondoj estas \^ciuj fonditaj
sur plena vereco? --- {\sl Ju\^gato}: Vorto post vorto, Via mo\^sto!
ke mi mortu, se mi iom mensogas! --- {\sl Ju\^ganto}: Sekve vi estas
preta anka\u u \^{\j}uri pri ili? --- {\sl Ju\^gato}: Hm\dots ne tre
volonte, sinjoro.

\emph{Neprospera aludo}. Kara edzo, la kuracisto diras, ke mi
bezonas min distri, vidi aliajn viza\^gojn \^cirka\u u mi\dots ---
Nu, en tia okazo mi tuj forlasos vian servantaron kaj dungos aliajn.

\emph{Naive}: La somerlo\^go, kiun mi luigas, sin trovas, kiel vi
vidas, apud la arbaro mem. La odoron de la abioj vi havos \^ciam en
la \^cambro. Provu nur la odoron! Belega, ravanta! Kaj kiel saniga!
\^Cu vi eble havas iun malsanan je la brusto en via estimata
familio? --- Ne! --- Ha, efektiva doma\^go!

\emph{Malegala legado}. Patrineto, mi ricevis leteron de sinjoro
Roberto, li petas mian manon; \^cu vi permesas, ke mi \^gin donu al
li? --- Sinjoro Roberto? Ho, mia naiva, \^cu vi bone komprenas la
dezirojn de sinjoro Roberto? --- Legu do mem, patrineto, kion li
skribas: "Fra\u ulino, donacu al mi vian manon! mi povas vivi nur
por vi! Per floroj kaj baloj mi plenigos vian tutan vivon! Kun via
kiso mi iros al la fino de la mondo!" --- Ha, ha, ha! sed vi ja
tute ne bone legis la leteron! Rigardu, kion li skribas: "Fra\u
ulino, donacu al mi vian monon! mi povas vivi nur per vi! Per ploroj
kaj batoj mi plenigos vian tutan vivon! Kun via kaso mi iros al la
fino de la mondo!"

\emph{Napoleono III kaj Benedetti}. Napoleono III ofte mokadis siajn
kor\-te\-ga\-nojn. Tiel li diradis al Benedetti, lia konata
ambasadoro, ke li havas bovan viza\^gon. --- Mi ne scias, Via
Imperiestra Mo\^sto, respondis unu fojon Benedetti al tiela \^serco,
\^cu mi efektive havas viza\^gon de bovo, sed mi tre bone scias, ke
mi multajn fojojn prezentadis vian personon, kaj \^ciam kun tiu \^ci
sama viza\^go, kiun mi nun havas. [I. Lojko.]

\emph{Neanta\u uvidita respondo}. Unu provincano, veninte Romon,
turnis al si la atenton de \^ciuj per frapanta simileco kun la
imperiestro A\u ugusto. Tiu \^ci ordonis, ke oni alkonduku al li la
similulon kaj, rigardante lin atente, li diris: --- Junulo, \^cu via
patrino neniam estis en Romo? --- Ne, Sinjoro, respondis la
provincano, sed mia patro ofte tie \^ci estadis. [I. Lojko.]

\emph{Kulereto}. En bona restoracio estas granda amaso da homoj,
kiuj man\^gas kaj trinkas. Apud du tabletoj, proksime de la enirejo,
sidas du sinjoroj: unu jam nejuna kun friponeta viza\^go; anta\u u
li sur la tablo staras malplena glaseto kaj telero kun duono da
pantran\^co kun \^sinko; la dua --- juna homo, dande vestista,
videble \^{\j}us sidi\^ginta apud la tableto; li detiris siajn
\^samajn gantojn kaj disbutonumas la superveston. La nejuna sinjoro
legas gazeton kaj de post \^gi li de tempo al tempo \^cirka\u
urigardas la publikon. --- Servanto! vokas la juna dando: tason da
kafo kaj paste\^ceton! Post kelkaj minutoj la servanto alportis. La
juna homo komencas trinki la kafon, rigardas \^cirka\u uen sur
\^ciuj flankoj kaj subite ka\^sas la kulereton de sia kafo en la
po\^son. Tion \^ci rimarkis la maljuna sinjoro. Li vokas la
servanton. Tiu \^ci aperas. --- Tason da kafo kaj paste\^ceton! La
postulita\^{\j}o estas alportita. La sinjoro man\^gas. --- Se vi
volas, mi montros al vi \^{\j}onglon, li diras al la servanto,
alvokinte lin. La servanto ridetas. --- Rigardu: \^cu tio \^ci estas
kulereto? --- Jes, sinjoro. --- Kie do \^gi estas nun? demandis la
sinjoro, rapide ka\^sante la kulereton en la po\^son.
--- En via po\^so. --- Ne, ne en mia, sed en la po\^so de tiu \^ci
juna sinjoro. La servanto senvole returnas sin al la dando. Tiu \^ci
suprensaltas. --- Volu, sinjoro, elpreni la kulereton el via po\^so,
turnas sin al li la nejuna homo. Vole-ne-vole la dando en\^sovas la
manon en la po\^son kaj eltiras la kulereton. --- Nu, vi estas vera
\^{\j}onglisto, li rimarkas konfuzite. --- Jes. Prenu por la kafo;
la brando estas pagita, diras la \^{\j}onglisto, \^{\j}etante
moneron sur la tablon, kaj fori\^gas, akompanata per salutoj de la
mirigita servanto kaj forportante la ka\^sitan kulereton. [I.
Lojko.]

\emph{La poeto Pope}. La angla re\^go, vidinte unu fojon sur la
strato Pope'on, konatan poeton, sed \^gibulon, diris al sia
sekvantaro: --- Mi volus scii, por kio ta\u ugas tiu \^ci hometo,
kiu iras tiel malrekte. Pope a\u udis tiujn \^ci vortojn kaj,
turninte sin al la re\^go, li ekkriis: Por igi vin rekte iradi. [I.
Lojko.]

\emph{Frideriko la Granda} amis disputojn kaj havis la kutimon fini
ilin per ekbato de piedo en la genuon de la kundisputanto. Unu
fojon, havante fortan deziron disputi, li demandis unu korteganon,
kial li ne diras al li sian veran opinion. --- Estas maloportune,
respondis tiu \^ci, diri siajn opiniojn al re\^go, kiu havas tiajn
fortikajn konvinkojn kaj tiajn solidajn botojn. [N. B.]

\emph{Unu generalo} demandis unu junan soldaton, \^cu estas sufi\^ca
por li la porcio da pano, kiun li \^ciutage ricevas. --- Jes, Via
Generala Mo\^sto. --- Kaj eble io ankora\u u restas? --- Jes, Via
Generala Mo\^sto. --- Kion do vi faras kun la resta\^{\j}o? --- Mi
\^gin forman\^gas, Via Generala Mo\^sto. [N. B.]

\emph{Rusa oficiro}, ricevinte de unu alilandulo elvokon al duelo,
sendis al li kiel sekundanton unu sian kolegon. Oni prezentas al li
la sekundanton de la kontra\u ua flanko, malgrandan kaj tre dikan
homon. --- \^Cu vi scias nian kutimon, demandis la lasta, ke se la
batalantoj maltrafas, la sekundantoj devas batali? --- Mi konsentas,
respondis la oficiro. --- Sed \^cu vi scias, da\u urigis la dika
sekundanto, ke mi je dudek kvin pa\^soj trafas en kvinkopekan
moneron? --- Mi ne scias, respondis la oficiro, \^cu mi trafos en
tia interspaco en kvinkopekon, sed en tian ventregon, kiel via, mi
trafos sendube. [N. B.]

\emph{Malri\^ca sa\^gulo} tagman\^gis \^ce avara ri\^culo. La
avarulo for\^sovis la panon sur la tablo tiel malproksimen, ke la
malri\^ca gasto ne povis \^gin atingi. Ne estante ankora\u u
kontenta de tio \^ci kaj volante pli \^cagreni sian gaston, la
avarulo turnis sin al li: --- Mi a\u udis, ke vi estas tre sa\^ga
homo, sed mi vidas, ke vi estas nelaborema: vi volas, ke aliaj
laboru, dum vi mem sidas sur ili kiel parazito kaj man\^gas kaj
vestas vin per fremda kosto. Se, kontra\u ue, vi estus laborema, vi
povus fari\^gi io, ekzemple \^ce mi kaj miaj komercoj. --- Ne, mia
bonfarema, respondis la sa\^gulo, kontra\u ue: mi volus io fari\^gi
\^ce vi, sed vi ne permesas al mi\dots --- Klarigu vin, diris la
avarulo, \^car mi vin ne komprenas. --- Tute simple, kun sarkasma
rideto sur la lipoj respondis la sa\^gulo: mi volis \^ce vi fari\^gi
sata, sed vi ne permesas al mi. [N. Ku\^snir.]

\emph{Kontra\u u la germana le\^go pri la drinkemeco} kelkaj gajaj
filoj de la Marburga universitato en unu nokto faris al si la
plezuron skribi per kreto sur la nigra tabulo de la universitata
domo la sekvantan sentencon: --- Kiu bone drinkas, tiu bone dormas;
kiu bone dormas, tiu ne pekas; kiu ne pekas, estas bona homo; sekve:
kiu bone drinkas, estas bona homo! Kian grandegan gajecon tiu \^ci
sentenco la sekvantan matenon elvokis \^ce la studentoj, oni facile
povas al si prezenti. [K. Hübert.]

\emph{Oni demandis} la filozofon Malebranche pri la ka\u uzo de la
suferoj de la bestoj. --- Ke la homoj suferas, tio \^ci estas
komprenebla: Adamo man\^gis la malpermesitan frukton; sed por kio
suferas la bestoj? --- Eble ili man\^gis malpermesitan fojnon,
respondis Malebranche. [N. B.]

\emph{Jubilea telegramo}. Mallongan tempon post la 25-jara jubileo
de la tago de apero de "Guberniaj Skizoj" de la rusa satiristo
\^S\^cedrin-Saltikov havis lokon la \^ciujara kolega tagman\^go de
la studentoj de la N...a universitato. Unu el la alestantoj
proponis, ke oni esprimu al la talenta satiristo la studentan
gratulon je la pasinta jubileo. Lia ideo estis unuvo\^ce akceptita
kaj tuj estis kunmetita telegramo kun konvena teksto kaj kun la
komuna subskribo: "\^Ciujare-tagman\^gantaj studentoj." Post unu
horo de \^S\^cedrin venis respondo: "Dankas! \^Ciutage
tagman\^ganta \^S\^cedrin."

{\emph Fortaj konvinkoj}. Kiam Napoleono I forlasis la insulon Elbo
kaj aliradis al Parizo, la Pariza gazeto "Moniteur" per la
sekvantaj esprimoj sciigis pri tio \^ci siajn legantojn: (tago 1):
{\sl La malbenita homman\^ganto} eliris el sia nesto; (tago 2): {\sl
La korsika diablo} forlasis la \^sipon en la golfo \^Juano; (tago
3): {\sl La tigro} alvenis al Gano; (tago 4): {\sl La monstro}
pasigis la nokton en Grenoblo; (tago 5): {\sl La tirano} trairis tra
Liono; (tago 6): {\sl La uzurpatoron} oni vidis en 60 mejloj de la
\^cefurbo; (tago 7): {\sl Napoleono} estos morga\u u apud niaj
fortika\^{\j}oj; (tago 8): {\sl La imperiestro} venis al Fonteneblo;
(tago 9): {\sl Lia Imperiestra Mo\^sto} enpa\^sis hiera\u u en la
palacon Tuilerian, \^cirka\u uita de siaj fidelaj regatoj.

\emph{Inter geedzoj}. --- La diablo prenu tian ordon! \^ciun fojon,
kiam mi volas vesti novan \^cemizon, \^gi \^ciam estas sen butonoj!
--- Kiel vi ne hontas, granda\^ga viro, tiel brui pro kelkaj
mankantaj butonoj! Rigardu la infanojn: \^ce ili la tutaj vestoj
estas dis\^siritaj, kaj ili ne diras e\^c unu vorton.

\emph{Solvita problemo}. Unu fojon la glora Buffon invitis al
tagman\^go kelkajn aliajn naturistojn. Post la tagman\^go la societo
deiris en la \^gardenon. Estis ankora\u u sufi\^ce varmege, malgra\u
u ke la suno staris jam sufi\^ce malalte. En la \^gardeno turnis sur
sin la atenton staranta sur postumento vitra globo, kion oni ofte
vidas en \^gardenoj. Iu, metinte sur \^gin la manon, trovis, ke per
ia strangeco \^gi estas sur la ombra flanko multe pli varma, ol sur
la suna. Li tuj komunikis sian eltrovon al sia najbaro. Unu post la
alia \^ciuj komencis metadi la manojn sur la globon kaj trovis, ke
la dirita estas vero. \^Ciuj kolekti\^gis \^cirka\u u la vitra
globo, kaj komenci\^gis alte scienca interparolo; \^ciu, por klarigi
la fenomenon, elpensis sian teorion: unu supozis engluton de la
radioj de varmeco, alia montris la liberi\^gon de ka\^sita
varmelemento, tria klarigis \^cion per rebato de la radioj. Kion unu
ne sciis, tion plenigis la alia, kaj fine, apogante sin sur la
le\^goj de la naturo, ili venis al la konkludo, ke la apero estas
tute natura kaj ke alie ne povas esti. Buffon, kiu ne volis ankora\u
u esprimi sian opinion pri tiu \^ci apero, alvokis la \^gardeniston
kaj demandis : --- Nu, mia kara, diru, kial la globo sur la ombra
flanko estas pli varma, ol sur la flanko turnita al la suno? ---
Kial? respondis la \^gardenisto, tial, \^car mi \^{\j}us turnis la
globon, por ke tiu \^gia flanko ne tro varmi\^gu. [M. Solovjev.]

\emph{Unu virino} plendis al kapitano, ke \^si estas \^cirka\u
urabita de liaj soldatoj. --- \^Cu ili \^cion forportis? demandis la
kapitano. --- Ne, ili restigis ion. --- Nu, sekve tio \^ci ne estis
miaj soldatoj; miaj prenas \^cion.

\emph{La imperiestrino Katerino} sonoras kaj neniu el \^siaj
servantoj venas. \^Si iras el la kabineto en la vestejon kaj pli
malproksimen, kaj fine en unu el la lastaj \^cambroj \^si vidas, ke
la hejtisto fervore ligas maldikan paka\^{\j}on. Ekvidinte la
imperiestrinon, li pali\^gis kaj \^{\j}etis sin anta\u u \^si sur la
genuojn. --- Kio estas? \^si demandis. --- Pardonu min, Via
Imperiestra Mo\^sto! --- Kion do vi faris? --- Jen, kara re\^gino,
mia valizo estas plena de diversaj bonaj objektoj el la palaco de
Via Mo\^sto. Tie \^ci sin trovas rosta\^{\j}o, kuka\^{\j}o, kelkaj
boteletoj da biero kaj kelkaj funtoj da bombonoj por miaj infanetoj.
Mi de\^{\j}oris mian semajnon, kaj nun mi iras hejmen, --- Tra kie
do vi volas eliri? --- Jen tie, per tiu \^stuparo. --- Ne, tie ne
iru; tie eble renkontos vin la \^cefo de la palaco; kaj mi timas, ke
tiam viaj infanoj nenion ricevos. Prenu do vian paka\^{\j}on kaj iru
post mi. \^Si elkodukis lin tra la salonoj sur alian \^stuparon kaj,
malferminte mem la pordon, diris: --- Nu, nun iru kun Dio.

\emph{La advokato de ratoj}. Chassie, unu el la plej gloraj juristoj
de Francujo kaj poste prezidanto de la parlamento en la Provenco,
fondis sian gloron per tio, ke li en la jaro 1522 elpa\^sis kiel
advokato de ratoj. Oni devas scii, ke en la fino de la mezaj
centjaroj regis la malsa\^go, ke la re\^ga prokuroro e\^c bestojn
tiradis al ju\^go. Tio \^ci havis lokon kun la ratoj en la dirita
jaro; ili estis invititaj al la ju\^gejo, por pravigi sin kontra\u u
la plendoj, levitaj kontra\u u ili. Kompreneble ili ne venis, kaj
\^ciu pensis, ke oni ju\^gos ilin en foresto, kiam subite Chassie
aperis kiel ilia defendanto kaj diris, ke liaj klientoj ne estis
invititaj en le\^ga ordo. --- Kaj kiel oni tion \^ci povas fari?
demandis la prokuroro; kaj kun \^sajna seriozeco la advokato
respondis: --- Oni albatu la inviton al la pordoj de la pre\^gejoj,
sed oni forgesu nenian lokon, oni anka\u u en\^slosu ie la katojn,
la terurajn malamikojn de miaj klientoj, por ke la vojoj fari\^gu
liberaj, kaj oni bone klarigu al la katoj, ke ili estos severe
punataj, se ili malgra\u u la malpermeso pekos kontra\u u la
libereco de la vojoj. --- Tio \^ci estas ja neebla, ekkriis la
prokuroro. --- En tia okazo mi esprimas proteston kontra\u u \^cia
farita ju\^ga decido, respondis Chassie kaj forlasis triumfe la
salonon de ju\^go. De tiu momento lia gloro estis fondita; la tuta
Francujo penadis en okazo de bezono ricevi lin kiel advokaton.

\emph{Sur balo}. --- Invitu al valso tiun fra\u ulinon, kiun vi tie
vidas. --- Kun granda plezuro, fra\u ulino! Tio \^ci estas kredeble
via amikino? --- Ne, kontra\u ue, sinjoro: tio \^ci estas mia plej
granda malamikino, kaj vi almena\u u plene batos al \^si la piedojn.

\emph{Mirinda\^{\j}oj}. Okaze kunvenis kvar sinjoroj, grandaj
majstroj kaj amantoj de mensogado. Longan tempon ili sidis silente,
fine unu diras: --- \^Cu vi a\u udis, kian sukceson havis Aramburo?
Mi mem vidis, ke en la teatro estis tiel malvaste, ke oni ne povis
apla\u udi en horizontala direkto, sed apla\u udis en vertikala. ---
Tre povas esti, diris la dua; kaj mi unu fojon estis en teatro, kaj
prezentu al vi, tie estis tiel plenege, ke oni ne povis ridi en
horizontala direkto, sed ridis en vertikala\dots --- \^Cio \^ci
estas malvera, diris la tria; sed jen mi, kiam mi estis en Afriko,
vidis negron tiel nigran, ke oni bezonis, por lin rigardi, ekbruligi
kandelon. --- Nu, tio \^ci anka\u u estas malvera, diris la kvara;
sed jen mi, kiam mi estis en Anglujo, mi vidis tiel maldikan fra\u
ulinon, ke \^si bezonis du fojojn eniri la \^cambron, por ke oni
povu \^sin rimarki. [M. Solovjev.]

\emph{Severa bibliotekisto}. \^Ciu el la regimentoj, kiuj staras en
Parizo, posedas propran bibliotekon, kies gardisto estas ordinare ia
maljuna suboficiro, kiu estas multe pli kompetenta en aferoj
militaj, ol en literaturo. Unu le\u utenanto de la maristaro sendis
unu fojon en tian bibliotekon sian servanton kun la komisio, ke li
alportu al li el la biblioteko la 9-an volumon de la vortaro de
Larousse (kun la literoj I, J, K). Post unu horo la soldato alportas
al li la volumon unuan, raportante al li, ke la bibliotekisto ne
povis doni la na\u uan volumon al persono, kiu ne legis la ok
unuajn, kaj ke, traleginte tiujn \^ci, la le\u utenanto ricevos
tion, kion li deziras.

\emph{La Filo}. Malri\^ca hebreo volis kristani\^gi. La pastro
instruis lin pri la pre\^go, sed lia penado restis vana, \^car la
hebreo estis tre malinteligenta. --- Nu, mia filo, se vi morga\u u
scios fari senerare la signon de la kruco, vi ricevos de mi sakon da
terpomoj. --- En la sekvanta tago la hebreo prenis kun si sian
filon, portantan sakon por la donaco, kaj li eniris en la domon de
la pastro lasinte la filon en la korto. --- Nu, faru la signon de la
kruco, diris la pastro. --- En la nomo de la Patro kaj de la Sankta
Spirito. Amen. --- Ne, kie do restis la "Filo" --- Pardonu, via
pastra mo\^sto, mi lasis lin ekstere; li tenas la sakon por la
terpomoj. [P. Lengyel].

\emph{La\u u la lokoj}. --- Ha, via grafa mo\^sto, vi estas tie
\^ci, en la \^cefurbo? Kaj en tia malnova, eluzita vesto! --- Kial
ne? Estas tre komforte portadi \^gin, kaj en la \^cefurbo neniu min
konas. --- Sed vi \^gin portadas anka\u u hejme. --- Nu, kial ne?
hejme ja \^ciu min konas. [P. Lengyel.]

\emph{Kurioza trompo}. Tri sinjoroj, elegante vestitaj, venis en
hotelon, prezentante sin kiel komisaroj Amerikaj de la ekspozicio
Antverpena, kaj ekvivis tie tiel lukse, ke ilia kalkulo en la da\u
uro de tri tagoj kreskis \^gis kelkaj centoj da frankoj. En vespero
de la tria tago anta\u u la komenci\^go de la hotela tagman\^go
aperis en la hotelo kvara gasto, kiu legitimi\^gis \^ce la mastro
kiel kriminala policisto el Parizo, sciigante, ke li intencas
ser\^ci tri dan\^gerajn krimulojn. Dume li montris al la mastro
fotografa\^{\j}ojn de tri personoj, en kiuj oni tuj ekkonis la tri
komisarojn Amerikajn. Nun la kriminala policisto proponis la
sekvantan planon al la mastro kun la peto helpi al li en la plenumo
de \^gi: la mastro zorgu, ke neniu el la tri personoj malaperu el la
domo. La policisto sidi\^gis al tablo kaj \^guis kun granda apetito
man\^gojn kaj trinkojn. Kiam oni alportis la deserton, la policisto
sin levis kaj ordonis silenton, sciigante la alestantajn
terurigitajn gastojn, ke la tri kontra\u usidantaj sinjoroj estas
tri dan\^geraj krimuloj, kiujn li devas aresti. La tri sinjoroj
provis forkuri, sed la mastro kune kun sia servantaro staris jam
anta\u u la pordo kontra\u u ili. La\u u la ordono de la policisto
alveturis dro\^sko en kiun sidi\^gis \^ciuj kvar sinjoroj. --- \^Cu
la trompistoj pagis sian kalkulon? demandas la policisto la mastron.
--- Ne. --- Kiom ili \^suldas? --- 295 frankojn. --- Bone! Ni
traser\^cos ilin en la polico kaj pagos vian kalkulon per la trovita
mono. Mian kalkulon vi povas anka\u u sendi tien. Kaj nun,
veturigisto, anta\u uen, al la polico! Sed la mastro vidis \^gis
hodia\u u nek la krimulojn nek la kriminalan policiston. [G.
Dumpert.]

\emph{Ruso kaj tataro}. Ruso kaj tataro veturis kune. En la vojo
fari\^gis nokto, kaj ili restis sur la kampo nokti. Ili disbruligis
fajron, kuiris ka\^con kaj, man\^gante la ka\^con, komencis disputi
pri tio, kiu el ili devas gardi la \^cevalojn. La ruso havis
\^cevalon malbonan kaj de koloro blanka, la tataro havis bonan
nigran \^cevalon. La nokto estis nigra, kaj jen la ruso diras al la
tataro: mi ne havas bezonon gardi mian \^cevalon: mi mian \^cevalon
tuj ekvidos, kiam mi veki\^gos, \^car \^gi estas blanka; sed vi ne
devas dormi, sed, gardi vian nigran \^cevalon. La tataro ne havis
grandan deziron maldormi pro la \^cevalo, kaj li faris kun la ruso
inter\^san\^gon kun la \^cevaloj. La ruso prenis la bonan \^cevalon
kaj, ridante je la tataro, diris al li: --- Nun mi tute ne gardos
mian \^cevalon, \^car \^stelisto en la nigra nokto ne ekvidos mian
nigran \^cevalon, sed vian blankan li tuj ekvidos kaj \^stelos. La
tataro devis ne dormi, sed gardi sian \^cevalon. La nokto pasis. La
ruso kaj tataro veturis plu. Por man\^gi ili bavis nur unu kokinon.
Por du personoj \^gi ne estis sufi\^ca, kaj tial ili decidis, ke
man\^gos nur unu --- nome tiu, al kiu la sorto \^gin decidos. La
tataro diras: --- Ni endormi\^gu, kaj tiu, kiu vidos pli bonan
son\^gon, man\^gos la kokinon. Ili ku\^si\^gis. La tataro ne dormas,
sed elpensas son\^gon, vekas la ruson kaj demandas: --- Kion vi
vidis en son\^go? La ruso postulas, ke la tataro anta\u ue diru,
kion li vidis en son\^go. La tataro diras: --- Ha, ruso! bonan
son\^gon mi vidis: oni prenis min en la \^cielon; en la \^cielo mi
vidis multajn spiritojn\dots ha, kiel bone tie estis! --- Jes, diras
la ruso, mi anka\u u vidis, kiel oni prenis vin en la \^cielon; mi
pensis, ke vi de tie jam ne revenos, kaj mi forman\^gis la kokinon.
La tataro restis tiel sen bona \^cevalo kaj sen man\^go kaj ne volis
jam pli iradi iam kun ruso. [J. Polupanov.]

\emph{Singardema komuniko}. Kiam nia mortinta ju\^gisto Begli
gliti\^gis en la ju\^gejo, falis de la \^stuparo kaj rompis al si la
kolon, ni ekmeditis pri grava demando, kiel komuniki tiun \^ci
mal\^gojan sciigon al lia malfeli\^ca edzino. Fine ni decidis tiel:
ni metis la korpon en la veturilon de nia maljuna honesta
veturigisto kaj ordonis al tiu \^ci lasta liveri la mortinton en la
domon de sinjorino Begli, sed \^ce tio \^ci agi kun granda takto kaj
singardemo. Precipe ni turnis lian atenton sur tion, ke li ne
komuniku la okazintan malfeli\^con al la maljunulino subite, sed ke
li anta\u ue preparu \^sin al tio \^ci. Alveturinte al la loko de la
difino, nia veturigisto komencis kriegi el la tuta gor\^go, \^gis
sur la sojlo aperis la edzino de la ju\^gisto. Tiam li demandis
\^sin: --- \^Cu tie \^ci lo\^gas la vidvino Begli? --- Vidvino
Begli?\dots Ne \^si ne lo\^gas tie \^ci. --- Kaj mi povas veti, ke
\^si lo\^gas tie \^ci!\dots Sed tio \^ci ne estas grava. Diru, \^cu
tie \^ci lo\^gas la ju\^gisto Begli? --- Jes, li lo\^gas tie \^ci.
--- Kaj mi povas veti, ke li ne lo\^gas tie \^ci. Cetere tio \^ci ne
estas grava. Mi ne disputos kun vi. Diru, \^cu li ne estas hejme?
--- Ne, li ne estas hejme. --- Sed mi povas veti, ke li estas
hejme\dots Sinjorino, mi konsilas al vi apogi vin al la muro, mi
diros al vi ion tion, ke vi kredeble ne eltenos sur la piedoj.
Okazis malfeli\^co! Jen en mia veturilo ku\^sas la maljuna
ju\^gisto. Ekrigardu lin, kaj vi vidos, ke li jam neniam sin levos.
[El Mark Twain. Tradukis S. Borovko.]

\emph{Dupieda leono}. Juna kaj tre bela dresistino de leonoj
permesis al leono preni pecon da sukero el \^sia bu\^so. --- Tion
povas anka\u u mi fari, diris unu rigardanto. --- Kio, vi, malforta
junulo? respondis la bela dresistino. --- Jes, egale bone kiel la
leono. [O. J. Olson.]
