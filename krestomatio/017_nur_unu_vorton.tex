\begin{center}
\footnotesize (Noveleto de L. E. \fsc{Meier}.)
\end{center}

   Estis la mateno post dancado. Tio \^ci da\u uris \^gis la kvara horo;
tiam ankora\u u oni kunesidi\^gis, por trinki nigran kafon \^gis
\^cirka\u u la kvina horo. Post tio \^ci oni disiris hejmen.

   Ho, kiel bela estis tiu dancado! Amuzoj ne mankis kaj, kontra\u u la
ordinaro, la nombro de la dancantoj estis granda; e\^c sufi\^ce
maljunaj sinjoroj estis en\^siritaj per la plezurego kaj alportis
oferojn al la diino de la dancado. Kiam matene oni reiradis hejmen,
la plej junaj el la mondo sinjora estis la \^gentilaj akompanantoj
de la sinjorinoj kaj fra\u ulinoj. La maljunaj sinjoroj jam longan
tempon estis dormintaj, \^car la mateno \^ciam apartenas al la
junularo.

   Tiel anka\u u Ella, la belega filino de la komerca konsilano Balding,
feli\^ce venis al la pordo de la gepatra lo\^gejo, kaj \^si estis
kune kun la patro kaj patrino, \^cirka\u uita de amaseto da junaj
herooj de danco. Oni \^sercadis kaj ridadis dum la irado; la komerca
konsilano estis gaja, kaj la patrino havis sian "bonan tagon".

   La disi\^gado iom longi\^gis: \^ciu el la sinjoretoj penis diri ion
aparte agrablan, kaj \^ciu el ili pensis, ke li ricevis de Ella
premon de mano aparte signifan, --- \^ciu kredis, ke li estas sole
preferita.

   \^Cu \^ciuj?

   La plej poste tie restis unu juna sinjoro, kiu atendis \^gis \^ciuj
anta\u uirintoj ricevos sian "maneton"; tiam anka\u u li mem povus
diri adia\u u al Ella.

   Tio estis la doktoro Gering, juna ju\^gisto. Li estis tre trankvila
kaj modesta. "Trankvilaj akvoj estas profundaj", diras proverbo
germana, kaj modesteco ofte ka\^sas firman certecon.

   Kiam la lasta anta\u ustaranto diris adia\u u --- la konsilano kun lia
edzino jam staris en la nefermita pordo, kaj post ili servanto kun
brulanta kandelo --- Gering iris anta\u uen, por komenci sian
preparitan parolon de adia\u u. Sed en la minuto kiam li levis la
manon por saluti Ella'n, tiu \^ci sin turnis al la amaseto de la
foririntaj sinjoroj:

 --- Vi ne forgesu do, kion vi promesis, sinjoro de Stetten!

   Kaj de tie voko eksonis:

 --- Certe ne, mia fra\u ulino, mi ne bezonas ja pli ol unu vorton!


 --- Ella, nun venu do! eksonis voko de en la pordo, kaj la sinjorino
konsilanedzino jam preninte la veston de sia filino altiris tiun
\^ci en la domon.

 --- Bonan nokton! ankora\u u ekkriis Ella. Bonan nokton! resonigis la
amaseto. Gering ankora\u u staris kun levita mano --- Ella jam
eniris en la domon, la pordo fermi\^gis --- jen li staris sen
"maneto" kaj sen adia\u udiro, kaj la parolo restis neparolita!

   Trankvile kaj modeste Gering reiris en sian lo\^gejon.

   Li iris tre meditante kaj tre malrapide. \^Ciuj aliaj jam de longa
tempo estis foririntaj.

   Veninte en sian lo\^gejon, li fine memori\^gis je si mem. Li ek\^gemis
kaj parolis al si mem, kiel post la eldono de ia ju\^go mortiga:

 --- Nur unu vorton --- nur unu vorton se \^si dirus al mi! La ju\^gisto
Gering ankora\u u la duan fojon ek\^gemis kaj poste ekdormis en la
veninta mateno.

   Estis luma tago, kiam Ella maldormi\^ginte vidis la sunon brilantan
tra la kurtenoj de la \^cambro.

   \^Si pripensis la farita\^{\j}ojn de la tago pasinta kaj ridetis kontente.
Sed jen per unu fojo \^si malgaji\^gis. Jen unu rememoro malagrabla.
\^Si rememoris la disi\^gon, rememoris la ju\^giston Gering, kiu
estis la sola, kiu ne diris al \^si e\^c unu vorton! \^Ciuj sinjoroj
premis \^sian manon, \^ciuj diris ion amindan, sed li sola silentis.
Li e\^c \^sajne intence restis poste!

 --- Mi tamen kredas, ke li estas senkulpa, diris \^sia plej bona
amikino Mathilde, kiu, vizitinte Ella'n en la vespero, sidis kun
\^si en la \^cambro de Ella, por paroleti pri la dancado pasinta.
--- Vi scias, ke tro alproksimi\^gi estas tute kontra\u u la
karaktero de Gering --- kvankam mi devas konfesi, ke tio \^ci estas
eco stranga \^ce ju\^gisto; sed kiu scias, \^cu li ne estos tute
alia en lia ju\^ga servado?

 --- Tro alproksimi\^gi? respondis Ella. Inter tro alproksimi\^gi kaj
{\sl intence retiri\^gi} estas ja granda diferenco!

 --- Certe! Sed\dots

 --- Kio povas lin fari tia? Certe li bone scias, ke mi lin amas. \^Cu
mi ne kun li dancis la unuan valson? \^Cu ne li sidis kune kun mi
\^ce la vesperman\^go? Kion pli mi povus fari? \^Cu mi mem devas
alfor\^gi lin per ia aparta atento? \^Cu e\^c ne estas lia propra
afero diri\dots kio devos esti dirita? Ho! tia modesteco! Nur unu
vorton li bezonas diri, kaj tiam \^cio estos farita. Sed neniel li
diras tiun vorton!

 --- Kara mia Ella! Ne \^ciuj homoj havas egalan karakteron! Ofte en ia
viro oni trovas ian econ, kiu konvenus multe pli bone al virino.
Diru mem: kio hodia\u u pli multe devigas la virinojn plendi: la
modesteco a\u u la tro granda proksimi\^gado de la sinjora mondo?
Certe! Gering estas escepto el la regulo, kaj tio \^ci distingas
lin!

 --- Kaj kiel fine ni venos al la celo? mi ne povas rekte diri la
vorton liberigontan, kaj li mem --- estas tro modesta!

   Ella parolis iom kolere: \^siaj vangoj ru\^gi\^gis, kaj Mathilde vidis,
ke la okuloj de la amikino malseki\^gis.

 --- Jen estas la tempo por agi, diris Mathilde interne kaj
pripensadis, kiel \^si mem povus helpi la aferon. Kun la tempo vi
alvenos al la celo, \^si konsolis la amikinon. Eble Walter iel povos
helpi!

   Tiuj \^ci vortoj videble reanimis Ella'n. --- Walter, \^si respondis,
jam promesis al mi sian helpon. Kiam ni hodia\u u matene disi\^gis,
mi diris al li: "Sed ne forgesu, kion vi promesis!"

 --- Kaj kion li diris?

 --- Certe ne! Mi bezonas nur unu vorton!

   Walter de Stetten, oficiro husara, estis la fian\^co de Mathilde, kaj
pro tio li estis anka\u u la konfidulo de la du amikinoj. Li
koni\^gis kun la ju\^gisto, kiam ili estis kolegoj-studentoj, kaj li
nun volonte akceptis la laboreton, kiun Ella al li transdonis \^si
mem kaj per sia amikino: certi\^gi, \^cu Gering efektive amas
Ella'n.

   Post tagman\^go, kiun Stetten kaj Gering kutime prenadis kune en
restoracio, la oficiro --- fidela al sia principo ne longe pripensi,
sed rekte aliri al la celo --- komencis sen ia anta\u uparolo pri
Ella.

   Stetten estis rekte la kontra\u ua\^{\j}o de sia amiko. Dum tiu \^ci pro
multa pripensado kaj granda modesteco \^sanceli\^gadis frapi al la
pordo de sia sanktejo, Stetten iam estis preta enbati tiun \^ci
pordon.

 --- Kiam do estos via fian\^ci\^go? li demandis la amikon.

   Gering lin miregante ekrigardis; li intencis respondi, sed ne povis.

 --- Sen ia \^serco, jen estas la tempo por fine konfesi vian intencon!
La tuta urbo parolas pri vi du kiel pri estontaj geedzoj. Nun aliru
al la celo! Vi estas enamita en Ella'n --- vi mem min tion sciigis.
\^Si anka\u u amas vin. Tion mi scias, kaj vi \^gin scias ankora\u u
pli bone!

   Gering lin a\u uskultadis ne ferminte la bu\^son.

 --- Ho! ne rigardu min do tiel miregante! parolu fine! --- Stranga
ju\^gisto vi estas --- al vi mankas la parolo por defendi\^gi!
\^Ciam nur ian unuan vorton vi bezonas por komenci!

 --- Jes, la unuan vorton! diris Gering malgaje. Kial \^si ne decidi\^gas
paroli?

 --- Jen! diris nun kvaza\u u miregante Stetten, la unuan parolon \^ciam
devas diri vi! \^Cu la fra\u ulino eble devus sin turni al vi:
Sinjoro ju\^gisto, mi intencas havi vin, --- \^cu vi volas, jes a\u
u ne!

 --- Tion mi ne pensas, ne!

 --- Kion do? --- Mi vin ne komprenas: Nun vi du jam unu kun la dua
koni\^gis anta\u u kelkaj monatoj; \^ciu homo ellegas el viaj
viza\^goj, ke vi amas unu la duan --- sed nek vi nek \^si
decidi\^gas \^gin konfesi! Vi mem ankora\u u pli malmulte ol Ella
--- kvankam vi estas viro! Mi estas tute certa, ke vi ne ricevos
rifuzon --- kaj vi mem \^gin scias pli bone ankora\u u.

 --- Mi? --- Ho, se mi \^gin scius! Vi mem diru: kian certecon mi havas
por \^gi? \^Cu Ella parolis nur unu solan fojon e\^c plej malgrandan
vorteton, kiu min certe esperigus? E\^c ne unu solan vorton! Kaj mi
timas esti rifuzita!

 --- Grandegan estimon mi sentas por ju\^gisto, kiu tiel malmulte konas
la pruvojn efektivajn! --- Ni ne \^sercu, Herman, --- la afero devas
esti finita. Kian pruvon vi atendas de Ella por fari vin certa je
via feli\^co?

 --- Nur unu vorton!

 --- Tion saman Ella atendas de vi!

 --- Ho!

 --- Mi certigas vin per mia honoro!

 --- Ej! Tiam\dots

 --- Tiam via sorto dependas de tiu, kiu el vi du diros la unuan
vorton? a\u u pli bone vi amba\u u devos diri unu vorton --- tio
estas tiel komprenebla, kiel unuoble unu estas unu!

   Gering restis pripensanta: jen li staras anta\u u tiu foso, kiun li
devus transsalti! Lia malgaja viza\^go ridigis lian amikon. --- Mi
nun bone komprenas, ke mi mem devos helpi la aferon, diris Stetten.

 --- Vi? demandis mirante Gering, kaj en tiu sama minuto la espero
reanimis lian viza\^gon.

 --- Jes, respondis Stetten, mi mem estos sekundanto por vi, kaj
Mathilde por Ella. Ho, la duelo estos gajega! Mi estas sciema, kiu
el vi unue vundos la duan! Sed mi estas certa, ke la frapo estos sen
malbona rezultato, de kiu ajn flanko \^gi venos!

   Gering ne multe esperis de la provo, sed ridetis.

   Stetten sin levis kaj kune kun Gering foriris el la man\^gejo.

 --- Mi vin sciigos pri la sekvanta\^{\j}o, estu certa! diris Stetten,
premante la manon de Gering kaj disi\^gante kun li.

   Stetten iris al sia militistejo, Gering iris al sia ju\^gejo. Li
tiam skribis parolon defendan, kaj kiu el liaj konatoj legus la
lertan parolon, kiun Gering skribis, tiu neniel kredus, ke la
skribanto tiel vane ser\^cis nur unu vorton!

 --- Post kelkaj tagoj --- estis la lasta semajno de la monato Majo,
kaj la vetero estis plej bela --- Herman Gering sidis anta\u u siaj
skriba\^{\j}oj. Estis \^cirka\u u la sepa horo de vespero, kaj la
ju\^gisto sin preparis eliri. Tiam la skribisto de Gering venis en
la \^cambron kaj donis al Gering ian leteron. Tiu \^ci letero venis
de Stetten, kiu per \^gi invitis sian amikon fari kun li promenon en
la morga\u ua diman\^co. La lastaj vortoj de tiu letero estis la
jenaj:

 --- Diru al la transdonanto nur unu vorton: "jes" a\u u "ne" estos
sufi\^ce.

   Gering komisiis la skribiston diri al la alportinto nur "Jes", kaj li
eliris el la ju\^gejo.

   Je la mateno de la veninta diman\^co la du amikoj kune promenis tra
la kampoj en la \^cirka\u ua\^{\j}o de la urbo kaj eniris en belegan
arbaron. La tago estis belega, aero kaj lumo plenaj je gajeco kaj
fre\^seco. Stetten mem estis aparte gaja kaj tiel li faris sian
amikon Gering anka\u u gajeta. La du amikoj iradis, parolante pri
diversaj objektoj.

   Stetten parolis pri multaj objektoj, sed ne tu\^sis e\^c per unu vorto
la aferon de kelkaj tagoj anta\u ue.

 --- Stranga \^gi estas, interne pensis Gering; li do komencu!

   Gering de tempo al tempo penis igi la amikon tu\^si la aferon, sed
Stetten \^sajnis ne kompreni kaj intence parolis pri aliaj objektoj.

   Tiel ili amba\u u fine venis al loko, kie la arbaro estis iom
travidebla. Preterirante, Gering rimarkis ne malproksime anta\u u
ili du sinjorinojn, kiuj nerapide iradis anta\u uen. Li ne povis
ilin rekoni, sed io nekonata faris pli rapide bati lian koron.

   Stetten ne diris e\^c unu vorton, \^sajnis, ke li ne vidas la virinojn.
Gering intence silentis. Post kelkaj pa\^soj la sinjorinoj ne estis
pli videblaj. La parolado estis finita, kaj Gering meditante iradis
sur la nelar\^ga vojo post sia amiko, kiu, dis\^sovante ie kaj ie la
bran\^cojn, nela\u ute kantadis la kanton de Heine: "Mi ne scias,
kion signifas\dots"

   Fine Gering rekomencis paroli:

 --- Tie \^ci estas pli kaj pli interese. Ni iras tra efektiva densa\^{\j}o
anta\u umonda!

 --- Jes, respondis Stetten, estos ankora\u u pli interese, kiam fine la
arbaro fini\^gos. Estis beleta vojo, certe!

   La arbaro iom post iom maldensi\^gis, kaj kiam Stetten dis\^sovis la
lastajn bran\^cojn, montri\^gis herba loko \^cirka\u uita de la
arbaro, kaj en la fino de tiu loko estis bela dometo arbarista.

 --- \^Cu tie \^ci ne estas belege? demandis Stetten.

 --- Belege, certe! respondis Gering; sed Stetten aldonis:

 --- Mi diras al vi, amiko, ni trovos ankora\u u ion pli belegan!

 --- Kion do li intencas? sin demandis Gering.

   La du amikoj aliris al la dometo. En la mezo de la vojo Stetten sin
turnis al la dekstra flanko.

 --- Ni eniros en la dometon per poste, li diris. Ili tiel \^cirka\u uiris
la dometon tra la arbaro kaj proksimi\^gis al \^gardeno. Stetten
estis kelkan pecon pli anta\u ue.

   Jen du sinjorinoj elirintaj el la pordo alvenis al la du amikoj.

 --- Ho! kontra\u uvole ekkriis Gering, \^car li rekonis la samajn du
virinojn, kiujn li anta\u ue vidis en la arbaro.

   Kiam la du sinjorinoj eliris el la \^gardeno, Stetten rapide
alproksimi\^gis al ili, salutinte ilin \^gentile.

   \^Gi estis Mathilde kun Ella.

   Stetten malla\u ute diris al Ella kelkajn vortojn, kaj tiam tiu \^ci
ru\^gi\^gis. --- Certe! diris Stetten la\u ute, demandu do lin vi
mem!

   Gering dume alvenis. Anta\u u ol li povis saluti la sinjorinojn, Ella
alpa\^sis al li kaj diris:

 --- \^Cu tio estas efektive vera, kion sinjoro de Stetten min sciigis?

 --- Jes a\u u ne! rapide ekkriis Stetten al Gering. Nur unu vorton
 --- jes a\u u ne! Kaj li rigardis sian amikon kiel sciante, ke tiu \^ci
ne povus respondi ion alian krom "jes".

   La rigardo de Ella nun ekbrilis. \^Gi estis tia rigardo, ke Gering
\^gin sentis en sia koro --- tiel magnete Ella nenie ankora\u u lin
rigardis! \^Si atendis la respondon; \^sajnis al li, ke li vidas kaj
sentas, ke \^si atendas nur la unu vorton: "jes". Kaj tamen li ne
sciis, pri kio propre \^si lin demandas. Sed en la demando estis io,
kion li komprenis tre bone; \^si \^gin elparolis per sia malseka
rigardo; \^sia brusto sin levadis kaj mallevadis, kaj jen \^sia mano
estis donita al li.

 --- \^Cu estas vere, Herman? ekparolis \^si nun tiel karese kaj plene je
amo, ke Gering tute forgesis la \^cirka\u ua\^{\j}on. Li vidis pli
nenion krom \^si, li a\u udis nur \^sian vo\^con; la tuta mondo
\^sajnis foresti. Li nun sentis feli\^con, tiu \^ci sento lin
venkis, li prenis \^sian manon. Tiel varme kiel nenie ankora\u u li
ekparolis al \^si, li respondis:

 --- Jes, Ella, jes!

   Kaj la du amantoj sin \^cirka\u uprenis. --- Jen! ekkriis Stetten. \^Cu mi
ne plenumis mian promeson? Mi plenumis e\^c pli ol tion, \^car mi
promesis nur unu vorton, sed Gering nun elparolis e\^c tri vortojn!

   Kiam la sekvantan tagon Stetten venis en la lo\^gejon de sia amiko,
li mirante lin vidis vestitan en nigra surtuto kun kravato kaj
gantoj blankaj.

   Stetten ne bezonis longe demandi.

 --- \^Cu vi iras al la komerca konsilano Balding?

 --- Jes.

 --- Kaj kion vi respondos, kiam tiu \^ci vin demandos sinjoro ju\^gisto,
kion vi deziras?

 --- Nur unu vorton, Stetten!

 --- Kaj tiu \^ci vorto estos?

 --- Ella'n!

\smallrule{}
