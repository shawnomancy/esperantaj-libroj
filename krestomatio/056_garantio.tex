\begin{verse}
\begin{center}
\footnotesize (El \fsc{Schiller}.)
\end{center}
                  Damon' unu fojon malla\u ute trairis\\
                  Kun akra tran\^cilo sub bask' al tirano ---\\
                  Al Dionizio, --- sed unu gardano\\
                  Lin kaptis subite. Tiran' al li diris:\\
                  "Pro kio vi al mi, fripono, aliris?"\\
                  --- Mi sanktan promeson ekvolis plenumi\dots ---\\
                  "Krimulo! mi tuj vin ordonos krucumi!"

                  --- Jam longe mi estas al mort' pretigita,\\
                  Kaj, kredu, ne diros mi e\^c unu vorton\\
                  Pro mia pardono, \^car havas mi forton,\\
                  Por morti sur kruco; sed se inklinita\\
                  Vi estas al bono, --- ho, estu donita\\
                  Al mi kelka tempo, por fraton edzigi, ---\\
                  \^Ce vi do mi povas amikon restigi. ---

                  Kruele ekbrulis tirana rigardo,\\
                  Kaj diris li post la silento momenta:\\
                  "Tre bone, kun tio mi estas konsenta;\\
                  Sed tamen memoru vi tion, bastardo,\\
                  Ke vian amikon mi tenos sub gardo\\
                  Ne pli, ol tri tagoj, --- kaj poste li mortos,\\
                  Al vi li per tio pardonon alportos."

                  Al sia amiko Damono alvenis\\
                  Kaj diras: "tiran' min alju\^gis krucumi,\\
                  \^Car volis mi sanktan promeson plenumi\dots\\
                  Kaj mi nur tri tagojn ricevi elpenis,\\
                  Por fraton edzigi\dots" kaj li \^cirka\u uprenis\\
                  Amikon kaj petas: "ho, restu, kurulo,\\
                  Vi por garantio \^ce malbonegulo!"

                  Amiko volonte konsentis kaj restis\\
                  En malliberejo li por garantio,\\
                  Kaj tuj liberigis Damonon per tio\dots\\
                  Damono edzi\^gon de frato forfestis,\\
                  En tria mateno tre frue sin vestis\\
                  Kaj vojon rapidas, malfrui timante\\
                  Kaj \^ciam pri sia amiko pensante\dots

                  Subite tre forta ventego fari\^gis,\\
                  Kaj tondro kaj fulmo kaj granda pluvego;\\
                  Pli forte, pli brue koleras ventego, ---\\
                  Kaj jen riveretoj de mont' ekruli\^gis,\\
                  Riveron plenigis, kaj \^gi ekondi\^gis\dots\\
                  Damono al ponto rapidas de monto\\
                  Kaj vidas, ke estas rompita jam ponto.

                  Li vagas sub pluvo sur bord' de rivero,\\
                  Kaj tre malproksime li vidas dometon;\\
                  Li krias, li vokas, li petas \^sipeton:\\
                  Sed vane, --- neniu \^car kun malespero\\
                  Ekvolis batali kun brua rivero\dots\\
                  Kaj balda\u u rivero jam aliformi\^gis:\\
                  \^Gi tute je maro simila fari\^gis.

                  Kaj jen lian koron plenigis doloro,\\
                  Li ploras, anta\u uen li manojn etendas,\\
                  Kaj pre\^gon varmegan al Ze\u uso li sendas:\\
                  "Kompatu!" li petas kun granda fervoro ---\\
                  "Kaj min ekrigardu kun dia favoro!\\
                  Se al Sirakuzoj vi min ne alportos,\\
                  Ho!\dots mia amiko senkulpe ja mortos!"

                  Sed vane kompaton de Ze\u uso li petas:\\
                  Ventego kaj ondoj kaj pluv' ne \^cesi\^gas.\\
                  Sed \^ciam plifortaj, teruraj fari\^gas, ---\\
                  Kaj tempo forpasas\dots Damono formetas\\
                  De si sian veston, kura\^ge sin \^{\j}etas\\
                  En bruan riveron kaj na\^gas, tran\^cante\\
                  Per brusto nur ondon, kun mort' batalante.

                  Kaj jen li sur bordo jam estas, --- el\^siris\\
                  Sin tute el brakoj teruraj de morto,\\
                  Kun granda dankem' al favoro de sorto,\\
                  Rapidas li plu\dots Sed subite eliris\\
                  El granda arbaro rabistoj. Aliris\\
                  Al ili Damono, --- kaj ili malpace\\
                  \^Cirka\u uas lin tute, rigardas minace.

                  "Ho, kion vi volas, krimuloj? li diris:\\
                  Ankora\u u ne scias, friponoj, vi tion,\\
                  Ke mi, nur krom vivo, jam havas nenion!\\
                  Sed vivo --- ne mia!" Kaj tuj li aliris\\
                  Al ili, bastonon de unu el\^siris,\\
                  Per \^gi li eksvingis, kaj tri li mortigis,\\
                  Aliajn do \^ciujn facile forigis.

                  Sed suno post pluvo varmege bruligis\\
                  Migranton, kaj tute li perdis jam forton;\\
                  Kun granda teruro atendis li morton, ---\\
                  Kaj pre\^gis li Ze\u uson: "vi min liberigis\\
                  El ondoj bruegaj, de mi vi forigis\\
                  Rabistojn, --- \^cu tie \^ci devas suferi\\
                  Mi vane kaj morti, amikon oferi?!\dots"

                  Subite li a\u udas, kvaza\u u murmuretas\\
                  Ne tre malproksime malgranda rivero!\\
                  Rapide sin turnas al \^gi kun espero\\
                  Damono --- kaj vidas, ke fonto fluetas.\\
                  Al \^gi superforte li tuj alvagetas,\\
                  Soifon brulegan kun \^goj' kvietigas\\
                  Kaj korpon malsanan en \^gi li fre\^sigas.

                  Jam suno varmega majeste subiras\\
                  Kaj ombroj tre longaj ku\^si\^gas sur tero:\\
                  Tre balda\u u jam venos trankvila vespero.\\
                  Sur vojo al urb' du migrantoj jen iras,\\
                  Damono do ilin en voj' preteriras.\\
                  Migrantoj parolas, Damono a\u uskultas\\
                  Kaj a\u udas: "lin oni jam nun ekzekutas!"

                  Tumulto piedojn al li flugiligas,\\
                  En lia animo --- senfinaj teruroj\dots\\
                  Kaj jen Sirakuzoj ! Nur suprojn de turoj\\
                  Malalta jam suno facile origas:\\
                  Jam estas vespero!\dots kaj li rapidigas\\
                  Ankora\u u pli pa\^sojn, --- kaj vidas --- en strato\\
                  Renkontas lin lia servant' Filostrato.

                  --- Revenu, \^car \^cio jam estas finita! ---\\
                  Li diris malgaje: en urbon ne iru,\\
                  Mi petas vin, el Sirakuzoj foriru!\\
                  Li estas jam al ekzekuto metita.\\
                  Li kredis sen fin', ke li estos trompita\\
                  Neniam de sia amik', kaj nenio\\
                  Lin povis devigi ne kredi je tio. ---

                  "Ho, se jam malfrui al mi estas sorto,\\
                  Kaj se mi ne povas de mort' liberigi\\
                  Amikon, --- mi devas min anka\u u mortigi:\\
                  Tirano ne diru, ke kuris de morto\\
                  Mi nur dank' al mia anima malforto!\\
                  Al vi mian vivon volonte mi cedas,\\
                  Sed sciu, ke homoj kaj amas kaj kredas!"

                  Vespere Damon' Sirakuzojn eniras\\
                  Kaj vidas, ke kruco sur placo jam staras,\\
                  Kaj oni jam tute amikon preparas,\\
                  Por lin ekzekuti\dots Kun forto tra\^siras\\
                  Amason popolan Damono kaj diras:\\
                  "Mi morton deziris de Dionizio", ---\\
                  Mi estas krimulo, li --- nur garantio !

                  En granda mirego popol' silenti\^gis\dots\\
                  En \^gojo amikoj sin tuj \^cirka\u uprenas\\
                  Kaj ploras, --- pri kio --- neniu komprenas,\\
                  Sed larmoj el \^ciuj okuloj ruli\^gis\dots\\
                  Minaca tirano pri \^cio scii\^gis;\\
                  Li je amikeco ilia tre miras\\
                  Kaj verajn amikojn li vidi deziras.

                  Kaj jen ili venas. Kun vido afabla\\
                  Tirano renkontas amikojn kaj diras:\\
                  "Vi venkis min tute, je vi mi tre miras!\\
                  Mi estas ankora\u u por amo kapabla,\\
                  Kaj se mi ne estas al vi malagrabla,\\
                  Permesu al mi, ke en ligo mi via,\\
                  Amikoj varmegaj, por vi estu tria!\dots"

%V. DEVJATNIN.
\end{verse}

\citsc{V. Devjatnin.}

\smallrule{}

