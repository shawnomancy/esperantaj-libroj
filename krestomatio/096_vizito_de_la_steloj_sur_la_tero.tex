\begin{verse}
                        \^Ciuj steloj ekdecidis\\
                        Ekvoja\^gi sur la teron.\\
                        Al mastrino ili diris\\
                        Sian volon kaj esperon:\\
                        "Ni ja tie tre mallonge\\
                        Gastos --- unu, du monatojn;\\
                        Ni vizitos regnon homan,\\
                        Vidos niajn koramatojn\dots"\\
                        Sed mastrino al la steloj\\
                        Tiam donis la respondon:\\
                        "Restu tie \^ci en domo ---\\
                        Vi malbenos poste mondon!"\\
                        Sa\^gajn vortojn ne a\u uskultis\\
                        Tamen steloj la trudemaj\\
                        Kaj sen\^cese ili tedis\\
                        La mastrinon --- nespertemaj:\\
                        "Sed komprenu nin, mastrino,\\
                        Sur \^cielo ni sidante\\
                        Tute ja ne konas tiujn,\\
                        Kiuj kantas nin konstante!\\
                        \^Ciu ja el ni sur tero\\
                        Havas siajn adorantojn,\\
                        Kiuj sendas al \^cielo\\
                        Himnojn, \^gemojn, amokantojn!"\\
                        --- "Vere", diris la mastrino,\\
                        "Ili kantas vin la\u udinde;\\
                        Sed, vin ili ne konante,\\
                        Vin adoras ja --- en blinde!\\
                        Ili, dum vi sur \^cielo,\\
                        \^Juras ami vin eterne,\\
                        \^Car neniu ja atingos\\
                        La ka\^sitan en interne!\\
                        Se vi tamen volas scii\\
                        Malfeli\^can vian sorton,\\
                        Iru --- sed vi rememoros\\
                        Mian nunan patran vorton".\\
                        Kaj la steloj al voja\^go\\
                        Tuj pretigis sin kun \^gojo. ---\\
                        Kaj jen balda\u u ili estis\\
                        Al la tero sur la vojo\dots

\asterism{}

                        Kiam a\u udis nun la tero,\\
                        Ke atendas \^gin honoro.\\
                        \^Cie en plej dol\^caj tremoj\\
                        Tuj ekbatis homa koro.\\
                        Tuta tera junularo\\
                        De la \^goj' ne sciis finon\\
                        Kaj sen pacienc' atendis\\
                        \^Ciu sian amatinon.\\
                        \^Ciu kantis, \^ciu saltis\\
                        Kun feli\^c' en la mieno,\\
                        Ke li havos de proksime\\
                        "\^Sin" en sia \^cirka\u upreno\\
                        \^Ciu homo dankas Dion,\\
                        \^Ciu benas la \^cielon,\\
                        Ke li fine nun ekkonos\\
                        Sian karan, sian stelon!

\asterism{}

                        Dum la tera junularo\\
                        Estas plena de espero,\\
                        Belaj steloj de l' \^cielo\\
                        Alveturis al la tero.\\
                        Tuj sin ili jam dismetis\\
                        En plej granda la hotelo;\\
                        \^Gis vespero \^ce la pordoj\\
                        \^Ciuj staris --- sed sen celo.\\
                        Je l'vespero ili a\u udis,\\
                        Ke gastinoj la belegaj\\
                        Morga\u u montros sin matene,\\
                        Nun estante tro lacegaj.\\
                        Tiam tre mal\^gojigitaj\\
                        \^Ciuj iris for la domon;\\
                        Post mallonge la Morfeo\\
                        \^Cirka\u uprenis \^ciun homon.\\
                        Kio estis en la son\^go,\\
                        Ni priskribi e\^c ne provos,\\
                        \^Car ni sentas, ke precize\\
                        Tion fari ni ne povos.\\
                        Ni en tiu \^ci loketo\\
                        Iom nur priskribi volas,\\
                        Kiel\dots tamen tss! mi a\u udas,\\
                        Oni el la son\^g' parolas:\\
                        "Ho, vi!\dots ho, vi!\dots mia bela,\\
                        Mia kara, ho, Heleno!\dots\\
                        Viaj kisoj kaj karesoj\dots"\\
                        Unu pre\^gas al kuseno.\\
                        Kaj alia, flamemulo,\\
                        \^Sin jam havas \^ce l' altaro,\\
                        Kaj solene --- en la lito ---\\
                        Sonas \^{\j}ura promesaro\dots\\
                        --- "Ho!\dots karega\dots ho, an\^gelo,\\
                        Fian\^cino\dots ho, Mario!\dots\\
                        Mia vi nun je eterne,\\
                        Anta\u u homoj, anta\u u Dio!"\\
                        Tiel li karesas, \^{\j}uras\\
                        Amon veran kaj brulantan,\\
                        Tiel kisas tiu homo\dots\\
                        Lampon apud li starantan.\\
                        Sed subite al li \^sajnas,\\
                        Ke Marion iu tenas\\
                        \^Cirka\u uprene!\dots "Ho, pro Dio!\dots"\\
                        (Lampon tuj diabloj prenas).\\
                        Kaj cetere, kaj cetere\\
                        Da\u uris tio \^gis mateno. ---\\
                        Elrapidas el la domo\\
                        \^Ciu fre\^sa kun mieno.\\
                        La gastinojn de l' hotelo\\
                        Ni jam vidas sur la stratoj,\\
                        Kaj jen balda\u u promenadas\\
                        Brak' en brak' la geamatoj\dots\\
                        Sed neniu tie a\u udas\\
                        De la amo dol\^can vorton!\\
                        \^Ciu el la junularo\\
                        Nun malbenas sian sorton\dots

\asterism{}

                        Kio faris tian \^san\^gon,\\
                        Ke anta\u ue tre amataj\\
                        Steloj, tiel deziritaj ---\\
                        Nune estas malbenataj?\\
                        Kio faris, vi demandas,\\
                        Tian \^san\^gon nun subitan?\\
                        Kio ka\u uzis malfeli\^con\\
                        Kaj esperon renversitan?\\
                        \^Cefa ka\u uzo, ke malbenoj\\
                        Nun el \^cies bu\^so sonis,\\
                        Estas --- ke la "stelon" \^ciu\\
                        Pli proksime, pli ekkonis\dots

\asterism{}

                        \^Ciuj steloj ofenditaj\\
                        Forveturis la \^cielon\\
                        Kaj post kelkaj larmaj tagoj\\
                        Jam atingis sian celon.\\
                        --- "Sa\^ga nia stelmastrino!\\
                        Sankta estis via vorto!\\
                        Prenu nun la tutan teron\\
                        La diabloj kaj la morto!\\
                        Tamen kiel vi, mastrino,\\
                        Anta\u usentis nian sorton?\\
                        Kio gvidis vian sa\^gan\\
                        Kaj profetan vian vorton?"\\
                        -- "Se vin iu ne rigardas\\
                        Tra \^cielaj tra speguloj,\\
                        Sed rigardas vin per sobraj\\
                        Ne trompantaj sin okuloj, ---\\
                        Tiu vidos ja sendube,\\
                        Ke amata lia "stelo"\\
                        Estas sen la luma krono\\
                        Nur diablo --- ne an\^gelo."

%FELIKS ZAMENHOF.
\end{verse}
\citsc{Feliks Zamenhof.}

\smallrule{}
