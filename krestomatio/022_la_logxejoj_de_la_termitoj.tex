   La diversaj specoj de termitoj (ekzistas multaj specoj) konstruas
nestojn de tre diversa formo. Kelkaj konstruas tureton el tero, de
alteco iom pli ol 50 centimetroj, \^cirka\u uitan de elstaranta
konusa tegmento, tiel ke la konstruo estas tre simila je fungo;
interne sin trovas multego da \^celoj de diversa formo kaj grandeco.
Aliaj preferas staton pli altan kaj konstruas siajn nestojn, kiuj
havas diversan grandecon, de la grandeco de \^capelo \^gis la
grandeco de barelo de sukero, kaj konsistas el kungluitaj pecetoj da
ligno --- en bran\^coj de arboj, ofte en la alteco de 20-25 metroj.
Sed senkompare la plej rimarkindaj lo\^gejoj estas tiuj, kiujn
konstruas la "Termes fatalis", speco, vivanta en Gvineo kaj aliaj
partoj de la Afrikaj bordoj de maro kaj pri kies nestoj skribas
Smeathman en la 71 volumo de la "Phisos. Transactions". Tiuj \^ci
nestoj estas farataj tute el tero, havas ordinare la altecon de 3-4
metroj kaj respondan lar\^gecon, tiel ke, se amaso da tiaj nestoj
sin trovas kune, kio ofte havas lokon, oni povas \^gin preni por
vila\^go de la sova\^guloj; kaj efektive tiuj \^ci nestoj ofte estas
pli grandaj, ol la dometoj de la tieaj homoj. En la komenco ili
konstruas du a\u u tri turetojn el tero, havantajn la altecon de
preska\u u 30 centimetroj kaj la formon de konuso da sukero. Tiuj
\^ci prezentas la trabaron de la estonta konstruo, ili balda\u u
fari\^gas pli multaj kaj pli altaj, poste oni ilin malsupre pli
lar\^gigas, supre ligas en kupolon, \^cirka\u ue fortikigas per dika
muro el tero, kaj tiel ili prezentas lo\^gejon de la grandeco kaj
formo de amaso da fojno, je kiu ili en kelka malproksimeco estas tre
similaj, kiam ili kovri\^gas per herbo, kio balda\u u fari\^gas.
Kiam la konstruo ricevis tiun \^ci lastan formon, tiam la internaj
turetoj, esceptinte la suprojn, kiuj en diversaj lokoj elstaras kiel
\^cambretoj, estas forigataj kaj la tero estas uzata por aliaj
celoj. Nur la malsupra parto de la konstruo estas okupata de la
lo\^gantoj. La supra parto a\u u la kupolo, kiu estas tre dika kaj
fortika, restas malplena kaj servas precipe kiel defendo kontra\u u
la \^san\^goj de la vetero kaj kontra\u u la atakoj de naturaj kaj
okazaj malamikoj, kiel anka\u u por konservi al la malsupra parto la
necesan varmecon kaj malsekecon por la disvolvi\^go de la ovoj kaj
por la flegado de la idoj. La parto lo\^gata enhavas la palacon
re\^gan a\u u la lo\^gejon de la re\^go kaj re\^gino, la nutrejon
por la idoj, la provizejon a\u u konservejon de la man\^ga\^{\j}o
kaj sennombrajn irojn, pasojn kaj malplenajn lokojn, aran\^gitajn
la\u u la sekvanta plano. En la mezo de la konstruo, rekte sub la
supro kaj pli-malpli sur egala alteco kun la tero, sin trovas la
re\^ga \^cambro, arka\^{\j}o de duone ovala formo a\u u simila je
forno de bakado. \^Gi estas ne pli longa, ol du centimetroj, sed oni
\^gin pligrandigas \^gis dekkvin centimetroj kaj pli multe, kiam la
re\^gino fari\^gas pli dika. En tiu \^ci \^cambro lo\^gas la re\^go
kaj la re\^gino konstante, kaj pro la malvasteco de la eniroj, kiuj
apena\u u sufi\^cas por tralasi la plej malgrandajn regatojn, le
gere\^goj neniel povas eliri; tiel ili, simile je multaj niaj
potenculoj, kare pagas por la re\^gado per la perdo de la libereco.
Tuj apud la \^cambro de la gere\^goj, \^cirka\u ue de \^ciuj flankoj
en la spaco de tridek centimetroj a\u u pli multe, sin trovas la
\^cambroj de la re\^ga kortego, konfuzita labirinto da sennombraj
\^cambroj de diversa formo kaj grandeco, el kiuj \^ciu sin malfermas
en alian kaj estas aran\^gita por oportuneco de la soldatoj kaj
servantoj, de kiuj \^ciam kelkaj miloj atendas la ordonojn de sia
re\^go kaj re\^gino. Post la \^cambroj de la re\^ga kortego iras
nutrejoj kaj provizejoj; la unuaj \^ciam estas plenaj de ovoj kaj
idoj, kaj en la komenco de la konstruado ili sin trovas tuj apud la
re\^ga \^cambro; sed kiam la kreskanta grandeco de la re\^gino
postulas pli grandan \^cambron kaj la apudaj \^cambroj por la
pligrandigita nombro da servantoj postulas la forigon de la ovoj,
tiam la malgrandaj \^cambroj de nutrado estas detruataj, kaj en
kelka malproksimeco estas konstruataj aliaj, iom pli grandaj kaj
anka\u u en pli granda nombro. Per sia \^stofo ili diferencas de
\^ciuj aliaj \^cambroj: ili estas konstruitaj el pecetoj da ligno,
kiuj estas gluitaj inter si kredeble per rezino. Amaso da tiuj
densaj, neregulaj kaj malgrandaj lignaj \^cambretoj, el kiuj neniu
atingas la largecon de du centimetroj, estas de sia flanko fermita
en komuna tera \^cambro, kiu ofte estas pli granda, ol kapo de
infano. La provizejoj sin trovas intermiksite kun la nutrejoj kaj
prezentas \^ciam terajn \^cambrojn plenajn de provizoj. La provizoj
konsistas el pecetoj da ligno kaj el densigitaj sukoj de
kreska\^{\j}oj. Tiuj \^ci provizejoj kaj nutrejoj estas apartigitaj
unuj de la aliaj por malgrandaj malplenaj \^cambretoj kaj pasoj,
kiuj komuniki\^gas inter si a\u u iras \^cirka\u ue, apartigitaj
unuj de la aliaj, havas de \^ciuj flankoj sian da\u ura\^{\j}on en
la ekstera muro de la konstruo kaj atingas tie \^gis du trionoj a\u
u tri kvaronoj de la alteco. Tamen ili ne okupas la tutan malsupran
parton de la monteto, sed trovas sin nur en la flankoj, kaj en la
mezo sub la kupolo ili lasas liberan kampon, kiu estas tre simila je
la mezospaco de malnova pre\^gejo; \^gia tegmento sin apogas sur tri
a\u u kvar grandaj gotaj arka\^{\j}oj, el kiuj tiu, kiu sin trovas
en la mezo de la kampo, ofte havas la altecon de unu metro kaj la
aliaj de \^ciu flanko estas \^ciam pli malataj, kiel serio da
arka\^{\j}oj en perspektivo. La \^cambroj, nutrejoj k. t. p. estas
kovritaj de plata sentrua tegmento, por reteni la malsekecon en la
okazo, se la kupolo estus difektita; kaj la libera kampo, kiu estas
iom pli alta, ol la \^cambro de la re\^go, havas anka\u u
netraflueblan plankon kaj estas tiel aran\^gita, ke la tutan pluvon,
kiu povus enpenetri, \^gi forkondukas en la subterajn irojn. Tiuj
\^ci iroj estas tre grandaj, kelkaj havas tridek centimetrojn en la
diametro, tute cilindraj, kaj servas en komenco, simile al la
katakumboj de Parizo, kiel la rompejoj, el kiuj estas prenataj la
materialoj konstruaj, kaj poste ili servas kiel grandaj eliroj, tra
kiuj la termitoj faras iliajn rabojn, kiujn ili entreprenas en kelka
malproksimo de sia lo\^gejo. Ili iras en malrekta direkto sub la
fundamento de la monteto en la profundecon de \^cirka\u u unu metro,
bran\^ci\^gas poste en \^ciun flankon kaj tiras sin sub la
supra\^{\j}o de la tero tre malproksime. \^Ce ilia eniro en la
interna\^{\j}on ili komuniki\^gas kun aliaj pli malgrandaj iroj,
kiuj sin levas spirale en la interna parto de la ekstera kovro, iras
ronde \^cirka\u u la lo\^gejo \^gis la supro, \^ce kio ili
tratran\^cas unu la alian sur diversaj vojoj, kaj \^ciu en diversaj
lokoj sin malfermas senpere en la kupolon kaj en la malsupran parton
de la konstruo a\u u komuniki\^gas kun \^ciu parto de tiu \^ci lasta
per aliaj, pli malgrandaj, rondaj a\u u ovalaj iroj de diversaj
diametroj. Ke la subteraj \^cefaj iroj devas esti tiel grandegaj,
venas kredeble de tio, ke ili estas la grandaj traveturejoj, per
kiuj la lo\^gantoj enkondukas sian teron, lignon, akvon a\u u
provizojn, kaj ilia spirala a\u u \^stupara levi\^go esta necesa por
oportuna aliro de la termitoj, kiuj nur kun granda malfacileco povas
sin levadi vertikale. Por eviti tiun \^ci malfacilecon en la
internaj vertikalaj partoj de la konstruo, ili ofte faras vojeton de
centimetro de lar\^geco, kiu sin levas \^stupare, kiel monta vojo,
kaj per tio \^ci ili tre facile sin levas sur alta\^{\j}ojn, kiujn
ili sen tio ne povus atingi. La sprita penado mallongigi sian vojon
elvokis, kiel \^sajnas, ankora\u u pli eksterordinaran elpenson. Tio
\^ci estas speco de ponto kun grandega arka\^{\j}o, kiu iras de la
planko de la kampo al la supraj \^cambroj en la flanko de la domo,
respondas al la celoj de \^stuparo kaj mallongigas la interspacon
\^ce la forportado de la ovoj el la re\^gaj \^cambroj al la supraj
teraj \^cambroj, kiuj en kelkaj montetoj sin trovas unuj de la aliaj
en malproksimeco de 1 - 1 1/2 metroj, kalkulante la vojon plej
rektan, kaj ankora\u u multe pli malproksime, se ni kalkulos per la
kurbaj kaj turnitaj iroj, kiuj kondukas tra la internaj \^cambroj.
Smeathman mezuris unu el tiaj pontoj; \^gi havis la lar\^gecon de
unu centimetro, la dikecon de duono da centimetro, la longecon de
dudek centimetroj, kaj prezentis la flankon de elipsa arko de
responda grandeco, tiel ke oni devas miri, ke \^gi ne rompi\^gis per
sia propra pezo, anta\u u ol \^gi estis ligita kun la supre staranta
kolono. \^Ce la fundamento \^gi estis fortigita per malgranda arko
kaj havis la\u ulonge de sia supra plata\^{\j}o sulkon, a\u u
aran\^gitan intence por pli granda sendan\^gereco de la irantoj, a\u
u per si mem elbatitan per la tro ofta irado. Rilate tiun \^ci
ponton grandan miron meritas tiu cirkonstanco, ke la gotajn
arka\^{\j}ojn kaj entute \^ciujn arka\^{\j}ojn de la diversaj iroj
kaj \^cambroj la termitoj ne kavernigas, sed dis\^siras. Por krei
tiujn \^ci efektive gigantajn konstruojn, tiuj \^ci malgrandaj
bestetoj devas uzi la plej nekredeblan diligentecon, la plej
sen\^cesan laboradon kaj la plej senlacan lertecon. Ke tiaj
malgrandaj insektoj, kiuj havas apena\u u la longecon de duono da
centimetro, povas en tri a\u u kvar jaroj pretigi konstruon de kvar
metroj de alteco kaj de responda dikeco, ornamitan per multego da
turetoj, kun miriadoj da \^cambroj de diversa grandeco kaj el
diversaj materialoj, konstruitaj sub \^giaj grandegaj arka\^{\j}oj;
ke ili en diversaj direktoj kaj diversa profundeco elfosas
sennombrajn subterajn vojojn, el kiuj multaj havas dudek kvin
centimetrojn en la diametro, a\u u dis\^siras arka\^{\j}on el
\^stono super aliaj vojoj, kiuj kondukas de la \^cefurbo en la
malproksimecon de kelkaj centoj da metroj; ke ili aran\^gas en la
\^cirka\u ua\^{\j}o kaj en la interna\^{\j}o grandegajn \^stuparojn
a\u u pontojn, kaj fine, ke la milionoj, kiuj estas necesaj por tiaj
herkulesaj laboroj kaj sen\^cese iras kaj reiras en diversaj
direktoj, neniam malhelpas unu al la alia, --- estas mirinda\^{\j}o
de la naturo, a\u u pli \^guste de la Kreinto de la naturo, kiu
multe superas la plej glorajn kreita\^{\j}ojn kaj konstruojn de la
homo; \^car se tiuj \^ci bestetoj estus egale grandaj, kiel li, kaj
konservus sian ordinaran instinkton kaj laboremecon, tiam iliaj
konstruoj atingus la miregindan altecon de tuta kilometro, iliaj
trakondukoj prezentus belegan cilindron kun diametro de cent metroj,
per kio la piramidoj de Egiptujo kaj la akvokondukoj de Romo perdus
sian tutan gloron kaj fari\^gus nuloj. La plej alta piramido ne
havas pli ol ducent metrojn, kio, se ni kalkulos, ke la homo havas
la altecon de nur 1 1/2 metroj, prezentos nur 133 fojojn la altecon
de la homo. Dume la nesto de la termitoj havas la altecon de
almena\u u kvar metroj kaj tiuj \^ci insektoj mem la altecon de ne
pli ol duono da centimetro, la lo\^gejo sekve estas okcent fojojn
pli alta, ol \^giaj konstruantoj.

\smallrule{}
