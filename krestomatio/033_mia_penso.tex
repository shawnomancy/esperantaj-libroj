\begin{verse}
                           Sur la kampo, for de l' mondo\\
                           Anta\u u nokto de somero,\\
                           Amikino en la rondo\\
                           Kantas kanton pri l' espero.\\
                           Kaj pri vivo detruita\\
                           \^Si rakontas kompatante, ---\\
                           Mia vundo refrapita\\
                           Min doloras resangante.

                           "\^Cu vi dormas? Ho, sinjoro,\\
                           Kial tia senmoveco?\\
                           Ha, kredeble rememoro\\
                           El la kara infaneco?"\\
                           Kion diri? Ne ploranta\\
                           Povis esti parolado\\
                           Kun fra\u ulino ripozanta\\
                           Post somera promenado!

                           Mia penso kaj turmento,\\
                           Kaj doloroj kaj esperoj!\\
                           Kiom de mi en silento\\
                           Al vi iris jam oferoj!\\
                           Kion havis mi plej karan ---\\
                           La junecon --- mi ploranta\\
                           Metis mem sur la altaron\\
                           De la devo ordonanta!

                           Fajron sentas mi interne,\\
                           Vivi anka\u u mi deziras, ---\\
                           Io pelas min eterne,\\
                           Se mi al gajuloj iras\dots\\
                           Se ne pla\^cas al la sorto\\
                           Mia peno kaj laboro ---\\
                           Venu tuj al mi la morto,\\
                           En espero --- sen doloro!

%L. L. ZAMENHOF.
\end{verse}

\citsc{L. Zamenhof.}

\smallrule{}
