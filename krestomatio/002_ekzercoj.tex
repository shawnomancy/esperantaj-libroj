\section*{\S\,1.}
Patro kaj frato. --- Leono estas besto. --- Rozo estas floro, kaj
kolombo estas birdo. --- La rozo apartenas al Teodoro. --- La suno
brilas. --- La patro estas sana. --- La patro estas tajloro.

\section*{\S\,2.}
Infano ne estas matura homo. --- La infano jam ne ploras. --- La
\^cielo estas blua. --- Kie estas la libro kaj la krajono? --- La
libro estas sur la tablo, kaj la krajono ku\^sas sur la fenestro.
--- Sur la fenestro ku\^sas krajono kaj plumo. --- Jen estas
pomo. --- Jen estas la pomo, kiun mi trovis. --- Sur la tero ku\^sas
\^stono.


\section*{\S\,3.}
Leono estas forta. --- La dentoj de leono estas akraj. --- Al leono
ne donu la manon. --- Mi vidas leonon. --- Resti kun leono estas
dan\^gere. --- Kiu kura\^gas rajdi sur leono? --- Mi parolas pri
leono.


\section*{\S\,4.}
La patro estas bona. --- Jen ku\^sas la \^capelo de la patro. ---
Diru al la patro, ke mi estas diligenta. --- Mi amas la patron. ---
Venu kune kun la patro. --- La filo staras apud la patro. --- La
mano de Johano estas pura. --- Mi konas Johanon. --- Ludoviko, donu
al mi panon. --- Mi man\^gas per la bu\^so kaj flaras per la nazo.
--- Anta\u u la domo staras arbo. --- La patro estas en la \^cambro.


\section*{\S\,5.}
La birdoj flugas.--- La kanto de la birdoj estas agrabla.--- Donu al
la birdoj akvon, \^car ili volas trinki. --- La knabo forpelis la
birdojn. --- Ni vidas per la okuloj kaj a\u udas perla oreloj.
--- Bonaj infanoj lernas diligente. --- Aleksandro ne volas lerni, kaj
tial mi batas Aleksandron. --- De la patro mi ricevis libron, kaj de
la frato mi ricevis plumon. --- Mi venas de la avo; kaj mi iras nun
al la onklo. --- Mi legas libron. --- La patro ne legas libron, sed
li skribas leteron.

\enlargethispage{-\baselineskip}
\section*{\S\,6.}
Papero estas blanka. --- Blanka papero ku\^sas sur la tablo. --- La
blanka papero jam ne ku\^sas sur la tablo. --- Jen estas la kajero
de la juna fra\u ulino. --- La patro donis al mi dol\^can pomon. ---
Rakontu al mia juna amiko belan historion. --- Mi ne amas obstinajn
homojn. --- Mi deziras al vi bonan tagon, sinjoro! --- Bonan
matenon! --- \^Gojan feston! (mi deziras al vi). --- Kia \^goja
festo! (estas hodia\u u). --- Sur la \^cielo staras la bela suno.
--- En la tago ni vidas la helan sunon, kaj en la nokto ni vidas la
palan lunon kaj la belajn stelojn. --- La papero estas tre blanka,
sed la ne\^go estas pli blanka. --- Lakto estas pli nutra, ol vino.
--- Mi havas pli fre\^san panon, ol vi. --- Ne, vi eraras, sinjoro:
via pano estas malpli fre\^sa, ol mia. --- El \^ciuj miaj infanoj
Ernesto estas la plej juna. --- Mi estas tiel forta, kiel vi. --- El
\^ciuj siaj fratoj Antono estas la malplej sa\^ga.

\section*{\S\,7.}
Du homoj povas pli multe fari ol unu. --- Mi havas nur unu bu\^son,
sed mi havas du orelojn. --- Li promenas kun tri hundoj. --- Li
faris \^cion per la dek fingroj de siaj manoj. --- El \^siaj multaj
infanoj unuj estas bonaj kaj aliaj estas malbonaj. --- Kvin kaj sep
faras dek du. --- Dek kaj dek faras dudek. --- Kvar kaj dek ok faras
dudek du. --- Tridek kaj kvardek kvin faras sepdek kvin. --- Mil
okcent na\u udek tri. --- Li havas dek unu infanojn. --- Sesdek
minutoj faras unu horon, kaj unu minuto konsistas el sesdek
sekundoj. --- Januaro estas la unua monato de la jaro, Aprilo estas
la kvara, Novembro estas la dek-unua, Decembro estas la dek-dua.
--- La dudeka (tago) de Februaro estas la kvindek-unua tago de la
jaro. --- La sepan tagon de la semajno Dio elektis, ke \^gi estu pli
sankta, ol la ses unuaj tagoj. --- Kion Dio kreis en la sesa tago?
--- Kiun daton ni havas hodia\u u? --- Hodia\u u estas la dudek sepa
(tago) de Marto. --- Georgo Va\^sington estis naskita la dudek duan
de Februaro de la jaro mil sepcent tridek dua.


\section*{\S\,8.}
Mi havas cent pomojn. --- Mi havas centon da pomoj. --- Tiu \^ci
urbo havas milionon da lo\^gantoj. --- Mi a\^cetis dekduon (a\u u
dek-duon) da kuleroj kaj du dekduojn da forkoj. --- Mil jaroj (a\u u
milo da jaroj) faras miljaron. --- Unue mi redonas al vi la monon,
kiun vi pruntis al mi; due mi dankas vin por la prunto; trie mi
petas vin anka\u u poste prunti al mi, kiam mi bezonos monon. ---
Por \^ciu tago mi ricevas kvin frankojn, sed por la hodia\u ua tago
mi ricevis duoblan pagon, t. e. (= tio estas) dek frankojn.
--- Kvinoble sep estas tridek kvin. --- Tri estas duono de ses. --- Ok
estas kvar kvinonoj de dek. --- Kvar metroj da tiu \^ci \^stofo
kostas na\u u frankojn; tial du metroj kostas kvar kaj duonon
frankojn (a\u u da frankoj). --- Unu tago estas
tricent-sesdek-kvinono a\u u tricent-sesdek-sesono de jaro. --- Tiuj
\^ci du amikoj promenas \^ciam duope. --- Kvinope ili sin \^{\j}etis
sur min, sed mi venkis \^ciujn kvin atakantojn. --- Por miaj kvar
infanoj mi a\^cetis dek du pomojn, kaj al \^ciu el la infanoj mi
donis po tri pomoj. --- Tiu \^ci libro havas sesdek pa\^gojn; tial,
se mi legos en \^ciu tago po dek kvin pa\^goj, mi finos la tutan
libron en kvar tagoj.


\section*{\S\,9.}
 Mi legas. --- Ci skribas (anstata\u u "ci" oni uzas ordinare "vi"),
 --- Li estas knabo, kaj \^si estas knabino. --- La tran\^cilo tran\^cas
bone, \^car \^gi estas akra. --- Ni estas homoj. --- Vi estas
infanoj. --- Ili estas rusoj. --- Kie estas la knaboj? --- Ili estas
en la \^gardeno. --- Kie estas la knabinoj? --- Ili anka\u u estas
en la \^gardeno. --- Kie estas la tran\^ciloj? --- Ili ku\^sas sur
la tablo. --- Mi vokas la knabon, kaj li venas. --- Mi vokas la
knabinon, kaj \^si venas. --- La infano ploras, \^car \^gi volas
man\^gi. --- La infanoj ploras, \^car ili volas man\^gi. --- Knabo,
vi estas ne\^gentila. --- Sinjoro, vi estas ne\^gentila.
--- Sinjoroj, vi estas ne\^gentilaj. --- Mia hundo, vi estas tre
fidela. --- Oni diras, ke la vero \^ciam venkas. --- En la vintro
oni hejtas la fornojn. --- Kiam oni estas ri\^ca (a\u u ri\^caj),
oni havas multajn amikojn.

\section*{\S\,10.}
Li amas min, sed mi lin ne amas. --- Mi volis lin bati, sed li
forkuris de mi. --- Diru al mi vian nomon. --- Ne skribu al mi tiajn
longajn leterojn. --- Venu al mi hodia\u u vespere. --- Mi rakontos
al vi historion. --- \^Cu vi diros al mi la veron? --- La domo
apartenas al li. --- Li estas mia onklo, \^car mia patro estas lia
frato. --- Sinjoro Petro kaj lia edzino tre amas miajn infanojn; mi
anka\u u tre amas (infanojn). --- Montru al ili vian novan veston.
--- Mi amas min mem, vi amas vin mem, li amas sin mem, kaj \^ciu
homo amas sin mem. --- Mia frato diris al Stefano, ke li amas lin
pli, ol sin mem. --- Mi zorgas pri \^si tiel, kiel mi zorgas pri mi
mem; sed \^si mem tute ne zorgas pri si kaj tute sin ne gardas.
--- Miaj fratoj havis hodia\u u gastojn; post la vesperman\^go niaj
fratoj eliris kun la gastoj el sia domo kaj akompanis ilin \^gis
ilia domo. --- Mi jam havas mian \^capelon; nun ser\^cu vi vian.
--- Mi lavis min en mia \^cambro, kaj \^si lavis sin en sia
\^cambro. --- La infano ser\^cis sian pupon; mi montris al la
infano, kie ku\^sas \^gia pupo. --- Oni ne forgesas facile sian
unuan amon.

\section*{\S\,11.}
Nun mi legas, vi legas kaj li legas; ni \^ciuj legas. --- Vi
skribas, kaj la infanoj skribas; ili \^ciuj sidas silente kaj
skribas. --- Hiera\u u mi renkontis vian filon, kaj li \^gentile
salutis min. --- Hodia\u u estas sabato, kaj morga\u u estos
diman\^co. --- Hiera\u u estis vendredo, kaj post-morga\u u estos
lundo. --- Anta\u u tri tagoj mi vizitis vian kuzon kaj mia vizito
faris al li plezuron. --- \^Cu vi jam trovis vian horlo\^gon? --- Mi
\^gin ankora\u u ne ser\^cis; kiam mi finos mian laboron, mi
ser\^cos mian horlo\^gon, sed mi timas, ke mi \^gin jam ne trovos.
--- Kiam mi venis al li, li dormis; sed mi lin vekis. --- Se mi estus
sana, mi estus feli\^ca. --- Se li scius, ke mi estas tie \^ci, li
tuj venus al mi. --- Se la lernanto scius bone sian lecionon, la
instruanto lin ne punus. --- Kial vi ne respondas al mi? \^Cu vi
estas surda a\u u muta? --- Iru for! --- Infano, ne tu\^su la
spegulon! --- Karaj infanoj, estu \^ciam honestaj! --- Li venu, kaj
mi pardonos al li. --- Ordonu al li, ke li ne babilu. --- Petu
\^sin, ke \^si sendu al mi kandelon. --- Ni estu gajaj, ni uzu bone
la vivon, \^car la vivo ne estas longa. --- \^Si volas danci. Morti
pro la patrujo estas agrable. --- La infano ne \^cesas petoli.

\section*{\S\,12.}
Fluanta akvo estas pli pura, ol akvo, staranta senmove.
--- Promenante sur la strato, mi falis. --- Kiam Nikodemo batas
Jozefon, tiam Nikodemo estas la batanto kaj Jozefo estas la batato.
--- Al homo, pekinta senintence, Dio facile pardonas. --- Trovinte
pomon, mi \^gin man\^gis. --- La falinta homo ne povis sin levi. ---
Ne ripro\^cu vian amikon, \^car vi mem pli multe meritas ripro\^con;
li estas nur unufoja mensoginto, dum vi estas ankora\u u nun \^ciam
mensoganto. --- La tempo pasinta jam neniam revenos; la tempon
venontan neniu ankora\u u konas. --- Venu, ni atendas vin, Savonto
de la mondo. --- En la lingvo "Esperanto" ni vidas la estontan
lingvon helpantan de la tuta mondo. --- A\u ugusto estas mia plej
amata filo. --- Mono havata estas pli grava ol havita. --- Pasero
kaptita estas pli bona, ol aglo kaptota. --- La soldatoj kondukis la
arestitojn tra la stratoj. --- Li venis al mi tute ne atendite.
--- Homo, kim oni devas ju\^gi, estas ju\^goto.

\section*{\S\,13.}
Nun li diras al mi la veron. --- Hiera\u u li diris al mi la veron.
--- Li \^ciam diradis al mi la veron. --- Kiam vi vidis nin en la
salono, li jam anta\u ue diris al mi la veron (a\u u li estis
dirinta al mi la veron). --- Li diros al mi la veron. --- Kiam vi
venos al mi, li jam anta\u ue diros al mi la veron (a\u u li estos
dirinta al mi la veron; a\u u anta\u u ol vi venos al mi, li diros
al mi la veron). --- Se mi petus lin, li dirus al mi la veron. ---
Mi ne farus la eraron, se li anta\u ue dirus al mi la veron (a\u u
se li estus dirinta al mi la veron). --- Kiam mi venos, diru al mi
la veron. --- Kiam mia patro venos, diru al mi anta\u ue la veron
(a\u u estu dirinta al mi la veron). --- Mi volas diri al vi la
veron. --- Mi volas, ke tio, kion mi diris, estu vera (a\u u mi
volas esti dirinta la veron).

\section*{\S\,14.}
Mi estas amata. Mi estis amata. Mi estos amata. Mi estus amata. Estu
amata. Esti amata. --- Vi estas lavita. Vi estis lavita. Vi estos
lavita. Vi estus lavita. Estu lavita. Esti lavita. --- Li estas
invitota. Li estis invitota. Li estos invitota. Li estus invitota.
Estu invitota. Esti invitota. --- Tiu \^ci komerca\^{\j}o estas
\^ciam volonte a\^cetata de mi. --- La surtuto estas a\^cetita de
mi, sekve \^gi apartenas al mi. --- Kiam via domo estis konstruata,
mia domo estis jam longe konstruita. --- Mi sciigas, ke de nun la
\^suldoj de mia filo ne estos pagataj de mi. --- Estu trankvila, mia
tuta \^suldo estos pagita al vi balda\u u. --- Mia ora ringo ne
estus nun tiel longe ser\^cata, se \^gi ne estus tiel lerte ka\^sita
de vi. --- La\u u la projekto de la in\^genieroj tiu \^ci fervojo
estas konstruota en la da\u uro de du jaroj; sed mi pensas, ke \^gi
estos konstruata pli ol tri jarojn. --- Honesta homo agas honeste.
--- La pastro, kiu mortis anta\u u nelonge (a\u u anta\u u nelonga
tempo), lo\^gis longe en nia urbo. --- \^Cu hodia\u u estas varme
a\u u malvarme? --- Sur la kameno inter du potoj staras fera
kaldrono; el la kaldrono, en kiu sin trovas bolanta akvo, eliras
vaporo; tra la fenestro, kiu sin trovas apud la pordo, la vaporo
iras sur la korton.

\section*{\S\,15.}
Kie vi estas? --- Mi estas en la \^gardeno. --- Kien vi iras? --- Mi
iras en la \^gardenon. --- La birdo flugas en la \^cambro (--- \^gi
estas en la \^cambro kaj flugas en \^gi). --- La birdo flugas en la
\^cambron (--- \^gi estas ekster la \^cambro kaj flugas nun en
\^gin). --- Mi voja\^gas en Hispanujo. --- Mi voja\^gas en
Hispanujon. --- Mi sidas sur se\^go kaj tenas la piedojn sur
benketo. --- Mi metis la manon sur la tablon. El sub la kanapo la
muso kuris sub la liton, kaj nun \^gi kuras sub la lito. --- Super
la tero sin trovas aero. --- Anstata\u u kafo li donis al mi teon
kun sukero, sed sen kremo. --- Mi staras ekster la domo, kaj li
estas interne. --- En la salono estis neniu krom li kaj lia
fian\^cino. --- La hirundo flugis trans la riveron, \^car trans la
rivero sin trovis aliaj hirundoj. --- Mi restas tie \^ci la\u u la
ordono de mia estro. --- Kiam li estis \^ce mi, li staris tutan
horon apud la fenestro. --- Li diras, ke mi estas atenta. --- Li
petas, ke mi estu atenta. --- Kvankam vi estas ri\^ca, mi dubas,
\^cu vi estas feli\^ca. --- Se vi scius, kiu li estas, vi lin pli
estimus. --- Se li jam venis, petu lin al mi. --- Ho, Dio! kion vi
faras! --- Ha, kiel bele! --- For de tie \^ci! --- Fi, kiel abomene!
--- Nu, iru pli rapide!

\section*{\S\,16.}
La artikolo "la" estas uzata tiam, kiam ni parolas pri personoj
a\u u objektoj konataj. \^Gia uzado estas tia sama, kiel en la aliaj
lingvoj. La personoj, kiuj ne komprenas la uzadon de la artikolo
(ekzemple rusoj a\u u poloj, kiuj ne scias alian lingvon krom sia
propra), povas en la unua tempo tute ne uzi la artikolon, \^car \^gi
estas oportuna, sed ne necesa. Anstata\u u "la" oni povas anka\u u
diri "l"' (sed nur post prepozicio, kiu fini\^gas per vokalo).
--- Vortoj kunmetitaj estas kreataj per simpla kunligado de vortoj;
oni prenas ordinare la purajn radikojn, sed, se la bonsoneco a\u u
la klareco postulas, oni povas anka\u u preni la tutan vorton, t. e.
la radikon kune kun \^gia gramatika fini\^go. Ekzemploj: skribtablo
a\u u skribotablo (= tablo, sur kiu oni skribas); internacia (= kiu
estas inter diversaj nacioj); tutmonda (= de la tuta mondo); unutaga
(= kiu da\u uras unu tagon); unuataga (= kiu estas en la unua tago);
vapor\^sipo (= \^sipo, kiu sin movas per vaporo); matenman\^gi,
tagman\^gi, vesperman\^gi; abonpago (= pago por la abono).


\section*{\S\,17.}
\^Ciuj prepozicioj per si mem postulas \^ciam nur la nominativon. Se
ni iam post propozicio uzas la akuzativon, la akuzativo tie dependas
ne de la prepozicio, sed de aliaj ka\u uzoj. Ekzemple: por esprimi
direkton, ni aldonas al la vorto la finon "n" ; sekve: tie (= en
tiu loko), tien (= al tiu loko); tiel same ni anka\u u diras: "la
birdo flugis en la \^gardenon, sur la tablon", kaj la vortoj
"\^gardenon", "tablon" staras tie \^ci en akuzativo ne \^car la
prepozicioj "en" kaj "sur" tion \^ci postulas, sed nur \^car ni
volis esprimi direkton, t. e. montri, ke la birdo sin ne trovis
anta\u ue en la \^gardeno a\u u sur la tablo kaj tie flugis, sed ke
\^gi de alia loko flugis al la \^gardeno, al la tablo (ni volas
montri, ke la \^gardeno kaj tablo ne estis la loko de la flugado,
sed nur la celo de la flugado); en tiaj okazoj ni uzus la fini\^gon
"n" tute egale, \^cu ia prepozicio starus a\u u ne. --- Morga\u u
mi veturos Parizon (a\u u en Parizon). --- Mi restos hodia\u u dome.
--- Jam estas tempo iri domen. --- Ni disi\^gis kaj iris en
diversajn flankojn: mi iris dekstren, kaj li iris maldekstren. ---
Flanken, sinjoro! Mi konas neniun en tiu \^ci urbo. --- Mi neniel
povas kompreni, kion vi parolas. --- Mi renkontis nek lin, nek lian
fraton (a\u u mi ne renkontis lin, nek lian fraton).


\section*{\S\,18.}
Se ni bezonas uzi prepozicion kaj la senco ne montras al ni, kian
prepozicion uzi, tiam ni povas uzi la komunan prepozicion "je".
Sed estas bone uzadi la vorton "je" kiel eble pli malofte.
Anstata\u u la vorto "je" ni povas anka\u u uzi akuzativon sen
prepozicio.
--- Mi ridas je lia naiveco (a\u u mi ridas pro lia naiveco, a\u u:
mi ridas lian naivecon). --- Je la lasta fojo mi vidis lin \^ce vi
(a\u u: la lastan fojon). --- Mi veturis du tagojn kaj unu nokton.
--- Mi sopiras je mia perdita feli\^co (a\u u: mian perditan
feli\^con).
--- El la dirita regulo sekvas, ke se ni pri ia verbo ne scias, \^cu
\^gi postulas post si la akuzativon (t. e. \^cu \^gi estas aktiva)
a\u u ne, ni povas \^ciam uzi la akuzativon. Ekzemple, ni povas diri
"obei al la patro" kaj "obei la patron" (anstata\u u "obei je
la patro"). Sed ni ne uzas la akuzativon tiam, kiam la klareco de
la senco tion \^ci malpermesas; ekzemple: ni povas diri "pardoni al
la malamiko" kaj "pardoni la malamikon", sed ni devas diri \^ciam
"pardoni al la malamiko lian kulpon".


\section*{\S\,19.}
Ia, ial, iam, ie, iel, ies, io, iom, iu. --- La montritajn na\u u
vortojn ni konsilas bone ellerni, \^car el ili \^ciu povas jam fari
al si grandan serion da aliaj pronomoj kaj adverboj. Se ni aldonas
al ili la literon "k", ni ricevas vortojn demandajn a\u u
rilatajn: kia, kial, kiam, kie, kiel, kies, kio, kiom, kiu. Se ni
aldonas la literon "t", ni ricevas vortojn montrajn: tia, tial,
tiam, tie, tiel, ties, tio, tiom, tiu. Aldonante la literon "\^c",
ni ricevas vortojn komunajn: \^cia, \^cial, \^ciam, \^cie, \^ciel,
\^cies, \^cio, \^ciom, \^ciu. Aldonante la prefikson "nen", ni
ricevas vortojn neajn: nenia, nenial, neniam, nenie, neniel, nenies,
nenio, neniom, neniu. Aldonante al la vortoj montraj la vorton
"\^ci", ni ricevas montron pli proksiman; ekzemple: tiu (pli
malproksima), tiu \^ci (a\u u \^ci tiu) (pli proksima); tie
(malproksime), tie \^ci a\u u \^ci tie (proksime). Aldonante al la
vortoj demandaj la vorton "ajn", ni ricevas vortojn
sendiferencajn: kia ajn, kial ajn, kiam ajn, kie ajn, kiel ajn, kies
ajn, kio ajn, kiom ajn, kiu ajn. Ekster tio el la diritaj vortoj ni
povas ankora\u u fari aliajn vortojn, per helpo de gramatikaj
fini\^goj kaj aliaj vortoj (sufiksoj); ekzemple: tiama, \^ciama,
kioma, tiea, \^ci-tiea, tieulo, tiamulo k. t. p. (= kaj tiel plu).


\section*{\S\,20.}
Lia kolero longe da\u uris. --- Li estas hodia\u u en kolera humoro.
--- Li koleras kaj insultas. --- Li fermis kolere la pordon. --- Lia
filo mortis kaj estas nun malviva. --- La korpo estas morta, la
animo estas senmorta. --- Li estas morte malsana, li ne vivos pli,
ol unu tagon. --- Li parolas, kaj lia parolo fluas dol\^ce kaj
agrable. --- Ni faris la kontrakton ne skribe, sed parole. --- Li
estas bona parolanto. --- Starante ekstere, li povis vidi nur la
eksteran flankon de nia domo. --- Li lo\^gas ekster la urbo. --- La
ekstero de tiu \^ci homo estas pli bona, ol lia interno. --- Li tuj
faris, kion mi volis, kaj mi dankis lin por la tuja plenumo de mia
deziro. --- Kia granda brulo! kio brulas? --- Ligno estas bona brula
materialo. --- La fera bastono, kiu ku\^sis en la forno, estas nun
brule varmega. --- \^Cu li donis al vi jesan respondon a\u u nean?
Li eliris el la dormo\^cambro kaj eniris en la man\^go\^cambron.
--- La birdo ne forflugis: \^gi nur deflugis de la arbo, alflugis al
la domo kaj surflugis sur la tegmenton. --- Por \^ciu a\^cetita
funto da teo tiu \^ci komercisto aldonas senpage funton da sukero.
--- Lernolibron oni devas ne trale\^gi, sed tralerni. --- Li portas
rozokoloran superveston kaj teleroforman \^capelon. --- En mia
skribotablo sin trovas kvar tirkestoj. --- Liaj lipharoj estas pli
grizaj, ol liaj vangharoj.


\section*{\S\,21.}
Teatramanto ofte vizitas la teatron kaj ricevas balda\u u teatrajn
manierojn. --- Kiu okupas sin je me\^haniko, estas me\^hanikisto,
kaj kiu okupas sin je \^hemio, estas \^hemiisto. --- Diplomatiiston
oni povas anka\u u nomi diplomato, sed fizikiston oni ne povas nomi
fiziko, \^car fiziko estas la nomo de la scienco mem. --- La
fotografisto fotografis min, kaj mi sendis mian fotografa\^{\j}on al
mia patro. --- Glaso de vino estas glaso, en kiu anta\u ue sin
trovis vino, a\u u kiun oni uzas por vino; glaso da vino estas glaso
plena je vino. --- Alportu al mi metron da nigra drapo (Metro de
drapo signifus metron, kiu ku\^sis sur drapo, a\u u kiu estas uzata
por drapo). --- Mi a\^cetis dekon da ovoj. --- Tiu \^ci rivero havas
ducent kilometrojn da longo. --- Sur la bordo de la maro staris
amaso da homoj. --- Multaj birdoj flugas en la a\u utuno en pli
varmajn landojn. --- Sur la arbo sin trovis multe (a\u u multo) da
birdoj. --- Kelkaj homoj sentas sin la plej feli\^caj, kiam ili
vidas la suferojn de siaj najbaroj. --- En la \^cambro sidis nur
kelke da homoj. --- "Da" post ia vorto montras, ke tiu \^ci vorto
havas signifon de mezuro.

\section*{\S\,22.}
Mia frato ne estas granda, sed li ne estas anka\u u malgranda: li
estas de meza kresko. --- Li estas tiel dika, ke li ne povas trairi
tra nia mallar\^ga pordo. --- Haro estas tre maldika. --- La nokto
estis tiel malluma, ke ni nenion povis vidi e\^c anta\u u nia nazo.
--- Tiu \^ci malfre\^sa pano estas malmola, kiel \^stono. --- Malbonaj
infanoj amas turmenti bestojn. --- Li sentis sin tiel malfeli\^ca,
ke li malbenis la tagon, en kiu li estis naskita. --- Mi forte
malestimas tiun \^ci malnoblan homon. --- La fenestro longe estis
nefermita; mi \^gin fermis, sed mia frato tuj \^gin denove
malfermis. --- Rekta vojo estas pli mallonga, ol kurba. --- La tablo
staras malrekte kaj kredeble balda\u u renversi\^gos. --- Li staras
supre sur la monto kaj rigardas malsupren sur la kampon. ---
Malamiko venis en nian landon. --- Oni tiel malhelpis al mi, ke mi
malbonigis mian tutan laboron. --- La edzino de mia patro estas mia
patrino kaj la avino de miaj infanoj. --- Sur la korto staras koko
kun tri kokinoj. --- Mia fratino estas tre bela knabino. --- Mia
onklino estas bona virino. --- Mi vidis vian avinon kun \^siaj kvar
nepinoj kaj kun mia nevino. --- Lia duonpatrino estas mia bofratino.
--- Mi havas bovon kaj bovinon. --- La juna vidvino fari\^gis denove
fian\^cino.


\section*{\S\,23.}
La tran\^cilo estis tiel malakra, ke mi ne povis tran\^ci per \^gi
la viandon kaj mi devis uzi mian po\^san tran\^cilon. --- \^Cu vi
havas korktirilon, por mal\^stopi la botelon? --- Mi volis \^slosi
la pordon, sed mi perdis la \^slosilon. --- \^Si kombas al si la
harojn per ar\^genta kombilo. --- En somero ni veturas per diversaj
veturiloj, kaj en vintro ni veturas per glitveturilo. --- Hodia\u u
estas bela frosta vetero, tial mi prenos miajn glitilojn kaj iros
gliti. --- Per hakilo ni hakas, per segilo ni segas, per fosilo ni
fosas, per kudrilo ni kudras, per tondilo ni tondas, per sonorilo ni
sonoras, per fajfilo ni fajfas. --- Mia skribilaro konsistas el
inkujo, sablujo, kelke da plumoj, krajono kaj inksorbilo. --- Oni
metis anta\u u mi man\^gilaron, kiu konsistis el telero, kulero,
tran\^cilo, forko, glaseto por brando, glaso por vino kaj
telertuketo. --- En varmega tago mi amas promeni en arbaro. --- Nia
lando venkos, \^car nia militistaro estas granda kaj brava. --- Sur
kruta \^stuparo li levis sin al la tegmento de la domo. --- Mi ne
scias la lingvon hispanan, sed per helpo de vortaro hispana-germana
mi tamen komprenis iom vian leteron. --- Sur tiuj \^ci vastaj kaj
herbori\^caj kampoj pa\^stas sin grandaj brutaroj, precipe aroj da
bellanaj \^safoj.


\section*{\S\,24.}
Vi parolas sensenca\^{\j}on, mia amiko. --- Mi trinkis teon kun kuko
kaj konfita\^{\j}o. --- Akvo estas fluida\^{\j}o. --- Mi ne volis
trinki la vinon, \^car \^gi enhavis en si ian suspektan
malklara\^{\j}on. --- Sur la tablo staris diversaj sukera\^{\j}oj.
--- En tiuj \^ci boteletoj sin trovas diversaj acidoj: vinagro,
sulfuracido, azotacido kaj aliaj. --- Via vino estas nur ia abomena
acida\^{\j}o. --- La acideco de tiu \^ci vinagro estas tre malforta.
--- Mi man\^gis bongustan ova\^{\j}on. --- Tiu \^ci granda alta\^{\j}o
ne estas natura monto. --- La alteco de tiu monto ne estas tre
granda. --- Kiam mi ien veturas, mi neniam prenas kun mi multon da
paka\^{\j}o. --- \^Cemizojn, kolumojn, manumojn kaj ceterajn
similajn objektojn ni nomas tola\^{\j}o, kvankam ili ne \^ciam estas
faritaj el tolo. --- Glacia\^{\j}o estas dol\^ca glaciigila
franda\^{\j}o. --- La ri\^ceco de tiu \^ci homo estas granda, sed
lia malsa\^geco estas ankora\u u pli granda. --- Li amas tiun \^ci
knabinon pro \^sia beleco kaj boneco. --- Lia heroeco tre pla\^cis
al mi. --- La tuta supra\^{\j}o de la lago estis kovrita per
na\^gantaj folioj kaj diversaj aliaj kreska\^{\j}oj. --- Mi vivas
kun li en granda amikeco.


\section*{\S\,25.}
Patro kaj patrino kune estas nomataj gepatroj. --- Petro, Anno kaj
Elizabeto estas miaj gefratoj. --- Gesinjoroj N. hodia\u u vespere
venos al ni. --- Mi gratulis telegrafe la junajn geedzojn. --- La
gefian\^coj staris apud la altaro. --- La patro de mia edzino estas
mia bopatro, mi estas lia bofilo, kaj mia patro estas la bopatro de
mia edzino. --- \^Ciuj parenooj de mia edzino estas miaj boparencoj,
sekve \^sia frato estas mia bofrato, \^sia fratino estas mia
bofratino; mia frato kaj fratino (gefratoj) estas la bogefratoj de
mia edzino. --- La edzino de mia nevo kaj la nevino de mia edzino
estas miaj bonevinoj. --- Virino, kiu kuracas, estas kuracistino;
edzino de kuracisto estas kuracistedzino. --- La doktoredzino A.
vizitis hodia\u u la gedoktorojn P. --- Li ne estas lavisto, li
estas lavistinedzo. --- La filoj, nepoj kaj pranepoj de re\^go estas
re\^gidoj. --- La hebreoj estas Izraelidoj, \^car ili devenas de
Izraelo. --- \^Cevalido estas nematura \^cevalo, kokido --- nematura
koko, bovido --- nematura bovo, birdido --- nematura birdo.


\section*{\S\,26.}
La \^sipanoj devas obei la \^sipestron. --- \^Ciuj lo\^gantoj de
regno estas regnanoj. --- Urbanoj estas ordinare pli ruzaj, ol
vila\^ganoj. --- La regnestro de nia lando estas bona kaj sa\^ga
re\^go. --- La Parizanoj estas gajaj homoj. --- Nia provincestro
estas severa, sed justa. --- Nia urbo havas bonajn policanojn, sed
ne sufi\^ce energian policestron. --- Luteranoj kaj Kalvinanoj estas
kristanoj. --- Germanoj kaj francoj, kiuj lo\^gas en Rusujo, estas
Rusujauoj, kvankam ili ne estas rusoj. --- Li estas nelerta kaj
naiva provincano. --- La lo\^gantoj de unu regno estas samregnanoj,
la lo\^gantoj de unu urbo estas samurbanoj, la konfesantoj de unu
religio estas samreligianoj. --- Nia regimentestro estas por siaj
soldatoj kiel bona patro. --- La botisto faras botojn kaj \^suojn.
--- La lignisto vendas lignon, kaj la ligna\^{\j}isto faras tablojn,
se\^gojn kaj aliajn lignajn objektojn. --- \^Steliston neniu lasas
en sian domon. --- La kura\^ga maristo dronis en la maro. ---
Verkisto verkas librojn, kaj skribisto simple transskribas paperojn.
--- Ni havas diversajn servantojn: kuiriston, \^cambristinon,
infanistinon kaj veturigiston. --- La ri\^culo havas multe da mono.
--- Malsa\^gulon \^ciu batas. --- Timulo timas e\^c sian propran
ombron. --- Li estas mensogisto kaj malnoblulo. --- Pre\^gu al la
Sankta Virgulino.

\section*{\S\,27.}
Mi a\^cetis por la infanoj tableton kaj kelke da se\^getoj. --- En
nia lando sin ne trovas montoj, sed nur montetoj. --- Tuj post la
hejto la forno estis varmega, post unu horo \^gi estis jam nur
varma, post du horoj \^gi estis nur iom varmeta, kaj post tri horoj
\^gi estis jam tute malvarma. --- En somero ni trovas malvarmeton en
densaj arbaroj. --- Li sidas apud la tablo kaj dormetas.
--- Mallar\^ga vojeto kondukas tra tiu \^ci kampo al nia domo. --- Sur
lia viza\^go mi vidis \^gojan rideton. --- Kun bruo oni malfermis la
pordegon, kaj la kale\^so enveluris en la korton. --- Tio \^ci estis
jam ne simpla pluvo, sed pluvego. --- Grandega hundo metis sur min
sian anta\u uan piedegon, kaj mi de teruro ne sciis, kion fari.
--- Anta\u u nia militistaro staris granda serio da pafilegoj.
 --- Johanon, Nikolaon, Erneston, Vilhelmon, Marion, Klaron kaj Sofion
iliaj gepatroj nomas Johan\^cjo (a\u u Jo\^cjo), Nikol\^cjo (a\u u
Niko\^cjo a\u u Nik\^cjo a\u u Ni\^cjo); Erne\^cjo (a\u u Er\^cjo),
Vilhel\^cjo (a\u u Vilhe\^cjo a\u u Vil\^cjo a\u u Vi\^cjo), Manjo
(a\u u Marinjo), Klanjo kaj Sonjo (a\u u Sofinjo).


\section*{\S\,28.}
En la kota vetero mia vesto forte malpuri\^gis; tial mi prenis
broson kaj purigis la veston. --- Li pali\^gis de timo kaj poste li
ru\^gi\^gis de honto. --- Li fian\^ci\^gis kun fra\u ulino Berto;
post tri monatoj estos la edzi\^go; la edzi\^ga soleno estos en la
nova pre\^gejo, kaj la edzi\^ga festo estos en la domo de liaj
estontaj bogepatroj. --- Tiu \^ci maljunulo tute malsa\^gi\^gis kaj
infani\^gis. --- Post infekta malsano oni ofte bruligas la vestojn
de la malsanulo. --- Forigu vian fraton, \^car li malhelpas al ni.
--- \^Si edzini\^gis kun sia kuzo, kvankam \^siaj gepatroj volis
\^sin edzinigi kun alia persono. --- En la printempo la glacio kaj
la ne\^go fluidi\^gas --- Venigu la kuraciston, \^car mi estas
malsana. --- Li venigis al si el Berlino multajn librojn. --- Mia
onklo ne mortis per natura morto, sed li tamen ne mortigis sin mem
kaj anka\u u estis mortigita de neniu; unu tagon, promenante apud la
reloj de fervojo, li falis sub la radojn de veturanta vagonaro kaj
morti\^gis. --- Mi ne pendigis mian \^capon sur tiu \^ci arbeto; sed
la vento forblovis de mia kapo la \^capon, kaj \^gi, flugante,
pendi\^gis sur la bran\^coj de la arbeto. --- Sidigu vin (a\u u
sidi\^gu), sinjoro! --- La junulo ali\^gis al nia militistaro kaj
kura\^ge batalis kune kun ni kontra\u u niaj malamikoj.


\section*{\S\,29.}
En la da\u uro de kelke da minutoj mi a\u udis du pafojn. --- La
pafado da\u uris tre longe. --- Mi eksaltis de surprizo. --- Mi
saltas tre lerte. --- Mi saltadis la tutan tagon de loko al loko.
--- Lia hiera\u ua parolo estis tre bela, sed la tro multa parolado
lacigas lin. --- Kiam vi ekparolis, ni atendis a\u udi ion novan,
sed balda\u u ni vidis, ke ni trompi\^gis. --- Li kantas tre belan
kanton. --- La kantado estas agrabla okupo. --- La diamanto havas
belan brilon. --- Du ekbriloj de fulmo trakuris tra la malluma
\^cielo. --- La domo, en kiu oni lernas, estas lernejo, kaj la domo,
en kiu oni pre\^gas, estas pre\^gejo. --- La kuiristo sidas en la
kuirejo. --- La kuracisto konsilis al mi iri en \^svitbanejon.
 --- Magazeno, en kiu oni vendas cigarojn, a\u u \^cambro, en kiu oni
tenas cigarojn, estas cigarejo; skatoleto a\u u alia objekto, en kiu
oni tenas cigarojn, estas cigarujo; tubeto, en kiun oni metas
cigaron, kiam oni \^gin fumas, estas cigaringo. --- Skatolo, en kiu
oni tenas plumojn, estas plumujo, kaj bastoneto, sur kiu oni tenas
plumon por skribado, estas plumingo. --- En la kandelingo sidis
brulanta kandelo. --- En la po\^so de mia pantalono mi portas
monujon, kaj en la po\^so de mia surtuto mi portas paperujon; pli
grandan paperujon mi portas sub la brako. --- La rusoj lo\^gas en
Rusujo kaj la germanoj en Germanujo.


\section*{\S\,30.}
\^Stalo estas fleksebla, sed fero ne estas fleksebla. --- Vitro
estas rompebla kaj travidebla. --- Ne \^ciu kreska\^{\j}o estas
man\^gebla. --- Via parolo estas tute nekomprenebla kaj viaj leteroj
estas \^ciam skribitaj tute nelegeble. --- Rakontu al mi vian
malfeli\^con, \^car eble mi povos helpi al vi. --- Li rakontis al mi
historion tute ne kredeblan. --- \^Cu vi amas vian patron? --- Kia
demando! kompreneble, ke mi lin amas. --- Mi kredeble ne povos veni
al vi hodia\u u, \^car mi pensas, ke mi mem havos hodia\u u gastojn.
--- Li estas homo ne kredinda. --- Via ago estas tre la\u udinda. --- Tiu
\^ci grava tago restos por mi \^ciam memorinda. --- Lia edzino estas
tre laborema kaj \^sparema, sed \^si estas anka\u u tre babilema kaj
kriema. --- Li estas tre ekkolerema kaj eksciti\^gas ofte \^ce la
plej malgranda bagatelo; tamen li estas tre pardonema, li ne portas
longe la koleron kaj li tute ne estas ven\^gema. --- Li estas tre
kredema: e\^c la plej nekredeblajn aferojn, kiujn rakontas al li la
plej nekredindaj homoj, li tuj kredas. --- Centimo, pfenigo kaj
kopeko estas moneroj. --- Sablero enfalis en mian okulon. --- Li
estas tre purema, kaj e\^c unu polveron vi ne trovos sur lia vesto.
--- Unu fajrero estas sufi\^ca, por eksplodigi pulvon.


\section*{\S\,31.}
Ni \^ciuj kunvenis, por priparoli tre gravan aferon; sed ni ne povis
atingi ian rezultaton, kaj ni disiris. --- Malfeli\^co ofte kunigas
la homojn, kaj feli\^co ofte disigas ilin. --- Mi dis\^siris la
leteron kaj dis\^{\j}etis \^giajn pecetojn en \^ciujn angulojn de la
\^cambro. --- Li donis al mi monon, sed mi \^gin tuj redonis al li.
--- Mi foriras, sed atendu min, \^car mi balda\u u revenos. --- La suno
rebrilas en la klara akvo de la rivero. --- Mi diris al la re\^go:
via re\^ga mo\^sto, pardonu min! --- El la tri leteroj unu estis
adresita: al Lia Episkopa Mo\^sto. Sinjoro N.; la dua: al Lia Grafa
Mo\^sto, Sinjoro P.; la tria: al Lia Mo\^sto, Sinjoro D. --- La
sufikso "um" ne havas difinitan signifon, kaj tial la (tre
malmultajn) vortojn kun "um" oni devas lerni, kiel simplajn
vortojn. Ekzemple: plenumi, kolumo, manumo. --- Mi volonte plenumis
lian deziron. --- En malbona vetero oni povas facile malvarmumi. ---
Sano, sana, sane, sani, sanu, saniga, saneco, sanilo, sanigi,
sani\^gi, sanejo, sanisto, sanulo, malsano, malsana, malsane,
malsani, malsanulo, malsaniga, malsani\^gi, malsaneta, malsanema,
malsanulejo, malsanulisto, malsanero, malsaneraro, sanigebla,
sanigisto, sanigilo, sanilo, resanigi, resani\^ganto, sanigilejo,
sanigejo, malsanemulo, sanilaro, malsanaro, malsanulido, nesana,
malsanado, sanila\^{\j}o, malsaneco, malsanemeco, saniginda,
sanilujo, sanigilujo, remalsano, remalsani\^go, malsanulino,
sanigista, sanigilista, sanilista, malsanulista k. t. p.
