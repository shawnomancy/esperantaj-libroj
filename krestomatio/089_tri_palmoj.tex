\begin{verse}
\begin{center}
\footnotesize (El \fsc{Lermontov}.)
\end{center}
                   \vin  En sabla dezerto de lando Azia\\
                  Tri palmoj fieris per kresko gracia.\\
                  Apude el tero senfrukta \^sprucanta\\
                  Murmuris fonteto per ondo frostanta.\\
                  \^Gi estis per verdaj folioj gardata\\
                  De suna radio kaj sablo portata.

                   \vin  Kaj tempo pasadis e\^c ne rimarkita.\\
                  Sed fremda migranto, de voj' lacigita,\\
                  La bruston flamantan al akvo malvarma\\
                  Ankora\u u ne klinis sub tendo la \^carma, ---\\
                  Komencis jam velki de sun' bruliganta\\
                  La belaj folioj kaj fonto sonanta.

                  \vin   La palmoj komencis je Dio plendadi:\\
                  "Ni estas naskitaj, por vane velkadi?\\
                  En ia dezerto ni vivon pasigas,\\
                  Skuadas nin ventoj kaj suno bruligas;\\
                  Kaj homo ne \^guas je flor' nia rava\dots\\
                  Ne, sankta \^ciel', via ju\^go --- ne prava!"

                  \vin   Kaj \^{\j}us ili finis --- en blua fora\^{\j}o\\
                  Sin turni komencis la ora sabla\^{\j}o,\\
                  Kaj venis de tie la son' sonorila,\\
                  Rigardis paka\^{\j}oj sub tolo kovrila,\\
                  Kaj iris kameloj, la sablon levante,\\
                  Sin, kvaza\u u \^sipeto en mar', balancante.

                  \vin   El inter la \^giboj sin skuis en pendo\\
                  La lar\^gaj kurtenoj de tendo kaj tendo;\\
                  Kaj ilin la brunaj manetoj levadis,\\
                  Kaj nigraj okuloj el tie briladis;\\
                  Arabo, talion al selo klinante,\\
                  Peladis \^cevalon, al kur' instigante.

                  \vin   Levadis sin tiam \^ceval' en kolero,\\
                  Kaj saltis \^gi, kvaza\u u vundita pantero.\\
                  De blanka mantelo la faldoj belegaj\\
                  Senorde ku\^sadis sur \^sultroj brunegaj.\\
                  Ludante, l'araboj ponardojn \^{\j}etadis\\
                  Kaj ilin en flugo kun \^gojo kaptadis.

                  \vin   Jen venas al palmoj kun bru' kamelaro;\\
                  En ombro etendis sin gaja tendaro.\\
                  La vazoj per akvo plenigis, sonante.\\
                  Fiere la kapojn foliajn svingante,\\
                  La palmoj salutas gastaron subitan\\
                  Kaj fonto donacas la akvon ka\^sitan.

                  \vin   Apena\u u do lumon krepusko forpremis,\\
                  Sub akra hakilo radikoj ek\^gemis, ---\\
                  Kaj falis sen vivo la palmoj alta\^gaj!\\
                  La veston for\^siris l'infanoj sova\^gaj.\\
                  La korpon de ili la homoj hakadis,\\
                  Kaj \^gis la mateno lignaroj bruladis.

                  \vin   Kaj kiam nebul' okcidenten foriris,\\
                  La\u u voj' difinita gastaro jam iris.\\
                  Kaj kie la palmoj centjarojn traestis,\\
                  Nur cindro malvarma kaj griza nun restis.\\
                  La suno \^gis fino resta\^{\j}on bruligis,\\
                  Kaj poste \^gin vent' en dezerton disigis.

                  \vin Kaj tie dezerta, senviva nun \^cio;\\
                  Ne kovras la fonton murmura folio;\\
                  \^Gi vane profeton pri ombro petadas, ---\\
                  Nur sablo varmega la akvon faladas,\\
                  Kaj sole vulturo, akiron portanta,\\
                  Dis\^siras \^gin super la fonto dormanta\dots

%M. SOLOVJEV.
\end{verse}
\citsc{M. Solovjev.}

\smallrule{}

