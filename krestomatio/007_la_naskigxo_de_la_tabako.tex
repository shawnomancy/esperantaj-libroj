\begin{center}
\footnotesize \fsc{araba legendo.}\\[1ex]

La\u u la hispana rakonto de \fsc{Cervera Bachiller}
 tradukis \\ \fsc{Jayme Heinlein Ferreira.}
\end{center}

   En la nomo de Allah, pardonema kaj bonkora, kiu donis al ni la kanon,
por skribi, kaj kiu \^ciutage instruas al la homoj ion, kion ili ne
scias, a\u uskultu:

   Li sola estas la grandulo, la potenculo, la sinjoro de la an\^geloj
kaj de la homoj. Sur Liaj lipoj estas la vero; la lumo de la sunoj,
kiuj brilas super la montegoj, venas de Liaj okuloj. Unu el Liaj
fingroj guvernas la ma\^sinon de la mondoj. La bloveto de Lia bu\^so
estas la vento, kiu pelas la sablojn de la dezerto. A\u uskultu:

   \^Gi ne estas la legendo pri la belulino Zobeido, nek la legendo pri
la sultano de Kanda\^horo, nek la historio pri la beduena belulino,
nek unu el la dol\^cegaj popolrakontoj kaj sor\^corakontoj, kiujn la
popolaj orientaj poetoj kantas en akompano de la sonoj de la guzlo
{\footnote{Orienta unukorda ludinstrumento.}, sidante apud la pordoj
de la kafejoj de Bagdado a\u u anta\u u la bazaroj de la ri\^cega
D\^{\j}edo. \^Gi ne estas unu el la rozaj legendoj, kiujn la
beduenoj kantas apud la "Puto de Beno" dum la plenigado de la
kru\^coj, kiam la suno dormas en la brakoj de la vespero, a\u u
kiujn la bestogardistoj de la dezerto kantas kune apud la
"\^Stonegoj koloritaj", kiam la kameloj ripozas sub la blanka
tendo kaj la luno levi\^gas sur la horizonto.

   \^Gi estas la legendo, kiun la veraj kredantoj \^ciam ripetas kun
la okuloj turnitaj al la "Sankta Kiblo" {\footnote{Mekko.} kaj
kiun rakontis al mi Ali Hasan el la gento "Beni el Vedaroj" unu
matenon, kiam ni promenadis amba\u u sur la bordo de la maro.

   Kiam la suno sin levis, Ali etendis la tapi\^son de pre\^go, falis sur
la genuojn kaj ekparolis la Fatah. Kiam lia pre\^go estis finita, li
sin levis kaj proponis al mi la "pipon de amikeco". Ni sidi\^gis
kaj komencis fumi.

 --- Kristano, li diris al mi, vi ne konas la historion de naski\^go de
tiu \^ci folio, kies odoron ni nun flaras kaj kies fumo supreniras
\^gis la trono de Allah, miksita kun la odoroj de la floroj?

 --- Mahometano, mi ne konas, mi respondis.

 --- La\u udata estu Allah, li ekkriis, kiu nur al la kredantoj, per la
bu\^so de la profeto, malkovris la misterojn de la objektoj
kovritaj. De Dio ni estas, kaj al Dio ni reiros\dots . Li estas
granda!

   Kaj preninte pli da tabako en sian pipon, li rakontis al mi la
sekvantan legendon, simplan, sed profunde religian kaj severan.

   Unu fojon voja\^gis la profeto Mahometo (Dio havu lin en sankta
gloro) tra la dezertoj de Jemen. Estis vintro, kaj pro la malvarmeco
la bestoj-rampa\^{\j}oj dormis la dormon de la longaj noktoj. La
\^cevalo, sur kiu la profeto rajdis, metis unu hufon sur
serpentejon, kaj tuj oni ekvidis serpenteton, dormantan de malvarmo.
Mahometo kompatis la rampa\^{\j}on, deiris de la \^cevalo, prenis la
serpenteton kaj metis \^gin en la manikon de sia mantelo, por ke la
varmo de lia korpo \^gin revivigu. Kaj la varmo de la korpo redonis
al \^gi la vivon. Balda\u u \^gi ekmovi\^gis, poste \^gi elmetis la
kapon el la maniko kaj diris:

 --- Profeto, mi volas mordi vian manon.

 --- Ne estu maldanka, respondis la profeto.

 --- Mi tiel volas.

 --- Kiam vi montros al mi la ka\u uzon, kiu vin igas fari malbonon al
mi, tiam mi permesos al vi mordi min.

 --- Via gento, murmuris la serpento, \^ciam militas kontra\u u nia: la
piedoj de la viaj kaj la hufoj de viaj \^cevaloj venkas \^ciam la
niajn, kaj mi nun volas ven\^gi al vi.

 --- Nun ni parolas nek pri via gento, nek pri nia, respondis la
profeto: nur inter vi kaj mi estas la afero. Kian do malbonon mi al
vi faris? \^Cu mi ne faris al vi bonon, redonante al vi la vivon per
la varmo de mia korpo?

 --- Malgra\u u tio mi volas vin mordi, por ke poste vi ne faru malbonon
al miaj filoj nek al aliaj de mia gento.

 --- Tio \^ci, malbona rampa\^{\j}o, estos maldanko: vi pagos malbonon por
bono. Ve al vi, kiu tiel malbone volas repagi la bona\^{\j}on, kion
oni faras al vi!

 --- Mi volas, ekkriis tiam la serpento kolere, mi volas, kaj mi \^{\j}uras
al vi per Dio la granda kaj potenca, ke mi mordos vin!

   A\u udinte la nomon de Allah, la profeto ne kura\^gis pli respondi. Li
klinis la kapon kaj diris:

 --- Lia nomo estu la\u udata! Ni estas de Li kaj de Li ni havas la
vivon!

   Kaj li malkovris la manon, ke la serpento tie mordu. Kaj la serpento
mordis la sanktan manon de Mahometo.

   Tiu \^ci tiam, eksentinte la doloron, el\^{\j}etis la serpenton
malproksimen kaj malbenis \^gin en la nomo de Allah, \^car \^gi
estis maldanka, kaj kun \^gi li malbenis anka\u u \^ciujn, kiuj
pagas malbonon por bono kaj ne estas dankaj por la bonoj, kiujn oni
faras al ili.

   Poste la profeto alpremis la lipojn al la vundo, forte eksu\^cis kaj
eltiris la venenon, kiun la rampa\^{\j}o tie lasis. Kaj li kra\^cis
sur la sablon de la dezerto. En tiu sama minuto sur la loko, kie
falis la kra\^ca\^{\j}o, naski\^gis kreska\^{\j}o, rapide
disvolvi\^gante kaj donante foliojn.

   La araboj, kiuj akompanis la senditon de Allah, ekbruligis kelkajn
foliojn, kiel oferon al Dio sola, pardonema kaj bonkora, kiu savis
la estron de la kredantoj de pereo per la veneno; kaj tiam ili
flaris la strangan kaj delikatan odoron, kiun tiu \^ci kreska\^{\j}o
donas, estante bruligata.

   De tiu tago \^ciuj bonaj mahometanoj fumas la foliojn de tiu herbo
mirinda kaj benita de Allah, plantas \^gin en la oazoj kaj flaras
\^gian odoron kun respekto kaj \^gojo, \^car \^gia gusto enhavas la
maldol\^cecon de la veneno serpenta kaj la dol\^cecon de la sankta
kra\^ca\^{\j}o de la profeto.

   De tiu malproksima tempo la tabako estas la plezuro de la had\^{\j}ioj,
kiuj voja\^gas al Mekko, de la ulemoj, kiuj instruas la sa\^gecon
apud la sojlo de la meskito de El-Hazar, fonto de gajeco kaj lumo,
kaj de la "filoj de la blanka tendo", kiuj estas la re\^goj de la
dezerto.

   Anka\u u de tiu tempo la kredanto, ricevinte de alia mahometano, sub
lia tendo, la "salon de gastameco", devas ami lin kaj defendi lin
\^gis la morto, se estus bezone, \^car la malbeno de la profeto
staras super la kapo de maldankulo, kiu ne povos vidi la klaran
lumon de la paradizo en la nokto de sia morto.

   Tio \^ci estas la legendo pri la tabako, kiu transiras de gento
al gento, rakontata de la maljunaj kredantoj, tra la centjaroj kaj
generacioj, pro la gloro de Allah, kies nomo estas benita. Li sola
estas granda!

\smallrule{}
