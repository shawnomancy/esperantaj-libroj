\begin{verse}
\begin{center}
\footnotesize (El \fsc{Krylov}.)
\end{center}
                  Se inter kolegoj jam en la komenco\\
                  En ia afero malestas konsento,\\
                  Senfrukta restados ilia intenco,\\
                  Anstata\u u afero nur estos turmento.

                  Okazis, ke kankro kun cigno, ezoko,\\
                  Por \^sar\^gon transporti de loko al loko,\\
                  Triope al veturigil' enjungi\^gis\\
                  Kaj tre diligente de l' loko tiri\^gis.\\
                  Sed malgra\u u ilia grandega fervoro\\
                  Nenion alportis ilia laboro.

                  Ekstreme facile \^gi estus por trio,\\
                  Se ili laborus nur en harmonio;\\
                  Sed al la \^cielo la cigno sin tiras,\\
                  La kankro obstine returne foriras,\\
                  L'ezoko al akvo la \^sar\^gon kun\^siras.

                  Nun kiun pravigi kaj kiun kulpigi,\\
                  Mi certe kaj tute ne povus klarigi, ---\\
                  Sed la veturilo e\^c pa\^son ne faris,\\
                  Ankora\u u \^gi staras nun, kie \^gi staris.

%M. GOLDBERG.
\end{verse}

\citsc{M. Goldberg.}

\smallrule{}
