\begin{verse}
                        \^Goju, \^goju ni, kolegoj,\\
                        Dum ni junaj estas!\\
                        Post plezura estanteco,\\
                        Post malgaja maljuneco ---\\
                        \vin   Sole tero restas.

                        Vivo estas tre mallonga,\\
                        Kuras ne tenate,\\
                        Kaj subite morto venos\\
                        Kaj rapide \^ciun prenos,\\
                        \vin   \^Ciun senkompate.

                        Kie niaj anta\u uuloj\\
                        En la mondo sidas?\\
                        Iru al la superuloj,\\
                        Ser\^cu ilin \^ce l' subuloj ---\\
                        \vin   Kiu ilin vidas?

                        Vivu la akademio\\
                        Kaj la profesoroj!\\
                        Vivu longe kaj en sano\\
                        \^Ciu akademiano,\\
                        \vin   Vivu sen doloroj!

                        Vivu, floru nia regno\\
                        Kaj regnestro nia!\\
                        Kaj amikoj mecenataj,\\
                        Protegantoj estimataj\\
                        \vin   De l' akademio

                        Vivu \^ciuj la knabinoj\\
                        Belaj kaj hontemaj!\\
                        Vivu anka\u u la virinoj,\\
                        Amikinoj kaj mastrinoj,\\
                        \vin   Bonaj, laboremaj.

                        Mortu, mortu, malgajeco,\\
                        Mortu la doloro!\\
                        Mortu \^ciu intriganto\\
                        Kaj malamon konservanto\\
                        \vin   Longe en la koro!

%L. L. ZAMENHOF.
\end{verse}

\citsc{L. Zamenhof.}

\smallrule{}

