\begin{center}
\footnotesize Hinda fabelo. Rakontis F. V. L.
\end{center}

   En la urbo Vataman vivis sa\^ga, sed malri\^ca bramano Kecava. Lia
edzino Karagara estis malbona al \^ciu; e\^c la diablo, kiu lo\^gis
en arbo apud tiu domo, el timo je Karagara forkuris en dezerton. La
bramano ne povis elporti la turmentojn, ricevatajn de sia edzino,
kaj foriris. En la vojo vidis lin la diablo kaj diris: "Hodia\u u
mi regalos vin. Ne timu! Mi lo\^gis en arbo apud via domo, sed el
timo je Karagara mi forkuris; mi volas fari al vi ion bonan. Iru en
la urbon Mrigavati; tie regas Madana; mi eniros en lian fllinon
Mrigalo\^cana'n kaj lasos min elirigi nur per vi".

   Tiel estis farita. La diablo eniris en la re\^gidinon, la bramano
venis kaj faris \^cion, kion faras la aliaj magiistoj, sed la diablo
ne eliris. Fine la bramano diris: "En la nomo de Karagara, eliru!"
kaj la diablo obeis.

   La re\^go donis al la bramano sian filinon kiel edzino kaj kun tio \^ci
anka\u u duonon de la regno. --- La diablo iris en la urbon
Karnavati kaj eniris en la re\^ginon Sulo\^cana'n. La re\^go sendis
al Madana kaj petis pri la magiisto Kecava. Tiu \^ci venis al la
turmentata re\^gino, sed la diablo, vidinte lin, diris: "Jam unu
fojon mi helpis al vi; nun gardu vin." La bramano alpa\^sis pli
proksime kaj ekmurmuretis al la re\^gino en la orelon: "Karagara
venis; mi volis nur diri \^gin al vi." La diablo, a\u udinte tion
\^ci, ektimis kaj eliris. La bramano, riceviate grandajn honorojn de
la re\^go, reiris en la urbon Mrigavati.

\smallrule{}
