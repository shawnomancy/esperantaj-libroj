\begin{verse}
                   Forte ni staru, fratoj amataj,\\
                   Por nia sankta afero!\\
                   Ni bataladu kune tenataj\\
                   Per unu bela espero!

                    \vin    Regas ankora\u u nokto sen luno,\\
                     \vin   La mondo dormas obstine,\\
                     \vin   Sed jam levi\^gos balda\u u la suno,\\
                     \vin   Por lumi, brili senfine.

                   Veku, ho veku, veku konstante,\\
                   Ne timu ridon, insulton!\\
                   Voku, ho voku, ripetadante,\\
                   \^Gis vi atingos a\u uskulton!

                     \vin   Dekon da fojoj vane perdi\^gos\\
                    \vin    La voko via ridata, ---\\
                    \vin    La dekunua alradiki\^gos,\\
                     \vin   Kaj kreskos frukto benata.

                   Tre malproksime \^ciuj ni staras\\
                   La unu de la alia\dots\\
                   Kie vi estas, kion vi faras,\\
                   Ho, karaj fratoj vi miaj?

                    \vin    Vi en la urbo, vi en urbeto,\\
                   \vin     En la malgranda vila\^go,\\
                    \vin    \^Cu ne forflugis kiel bloveto\\
                    \vin    La tuta via kura\^go?

                   \^Cu vi sukcese en via loko\\
                   Kondukas nian aferon,\\
                   A\u u eksilentis jam via voko,\\
                   Vi lacaj perdis esperon?

                    \vin    Iras senhalte via laboro\\
                    \vin    Honeste kaj esperante?\\
                    \vin    Brulas la flamo en via koro\\
                    \vin    Neniam malforti\^gante?

                   Forte ni staru, brave laboru,\\
                   Kura\^ge, ho nia rondo!\\
                   Nia afero kresku kaj floru\\
                   Per ni en tuta la mondo!

                    \vin    Ni \^gin kondukos ne ripozante,\\
                    \vin    Kaj nin lacigos nenio;\\
                     \vin   Ni \^gin traportos, sankte \^{\j}urante,\\
                    \vin    Tra l' tuta mondo de Dio!

                   Malfacileco, malrapideco\\
                   Al ni la vojon ne baros.\\
                   Sen malhonora malkura\^geco\\
                   Ni kion povos, ni faros.

                   \vin     Staras ankora\u u en la komenco\\
                    \vin    La celo en malproksimo, ---\\
                    \vin    Ni \^gin atingos per la potenco\\
                    \vin    De nia forta animo!

                   Ni \^gin atingos per la potenco\\
                   De nia sankta fervoro,\\
                   Ni \^gin atingos per pacienco\\
                   Kaj per sentima laboro.

                    \vin    Glora la celo, sankta l'afero,\\
                    \vin    La venko --- balda\u u \^gi venos;\\
                    \vin    Levos la kapon ni kun fiero,\\
                    \vin    La mondo \^goje nin benos.

                   Tiam atendas nin rekompenco\\
                   La plej majesta kaj ri\^ca:\\
                   Nia laboro kaj pacienco\\
                   La mondon faros feli\^ca!

\end{verse}

\citsc{L. Zamenhof.}

\smallrule{}

