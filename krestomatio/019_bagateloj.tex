
\emph{Rimarkinda horlo\^go}. Per kia originala maniero en \^Hinujo
malri\^caj homoj, kiuj ne posedas horlo\^gon, difinas la tempon, pri
tio la franca voja\^ganto Le Hube rakontas sekvantan okazon: "Unu
tagon, kiam ni volis viziti niajn \^hinojn, kiuj anta\u u nelonge
akceptis la kristan religion, ni envoje renkontis knabon, kiu
pa\^stis bovon. Preterirante, ni demandis lin, \^cu jam estas la
12-a horo. La knabo rigardis al la suno, sed \^gi estis ka\^sita
post densaj nuboj, tiel ke li ne povis konsili\^gi kun tiu \^ci
horlo\^go. --- La \^cielo estas kovrita tro plene per nuboj, li
diris, sed atendu unu momenton! Li kuris en la plej proksiman korton
de vila\^gano kaj post unu minuto revenis kun kato sur la brako. ---
Rigardu, li diris, ankora\u u ne estas la 12-a horo, kaj samtempe li
montris al ni la okulojn de la kato, supren\^sovante \^giajn
palpebrojn. Ni kun mirego rigardis la knabon, sed lia ekstera\^{\j}o
estis tute serioza, kaj la kato, kvankam la operacio \^sajne ne tre
pla\^cis al \^gi, tamen videble kutiminta al \^gi, tenis sin tre
trankvile, kvaza\u u \^gia speciala okupi\^go konsistus en tio, esti
horlo\^go. Ni diris: Tre bone, mia knabo, plej bonan dankon! Kaj ni
hontis, ricevinte instruon de la knabo. Kiam ni trovis niajn
amikojn, nia unua afero estis scii\^gi pri tiu kata orakolo. Ili tre
miris pro nia malscio kaj rapide kolektis kelkajn dekojn da katoj el
la tuta najbara\^{\j}o, por montri al ni, ke la horlo\^goj en iliaj
okuloj \^ciuj iras regule. La pupiloj de la okuloj de la katoj \^gis
tagmezo iom post iom malgrandi\^gas kaj tiam atingas sian plej
mallar\^gan kuntiri\^gon en formo de maldika streko, metita, kiel
hareto, perpendikulare sur la mezon de la okulo. Poste la pupiloj
iom post iom ree etendi\^gas, \^gis ili je noktomezo ricevas la
formon de rondo. Oni nin certigis, ke \^ciu infano en mallonga tempo
akiras grandan lertecon kaj akuratecon en la montrado de la tempo el
kataj okuloj. Ni konvinki\^gis tre balda\u u, ke tiuj \^ci
horlo\^goj iras tute regule kaj \^ciuj en preciza konsento."

\begin{flushright}
\footnotesize [K. Hübert.]
\end{flushright}

\emph{La homa vo\^co}. D-ro Delaunay en anta\u u nelonge publikigita
laboro pri la homa vo\^co esprimas la konvinkon, ke la antikvaj
lo\^gantoj de E\u uropo estis tenoroj. La nunaj iliaj posteuloj
estas baritonoj, iliaj nepoj posedos vo\^con basan. La racoj pli
malaltaj, kiel ekzemple negroj k. t. p., havas vo\^con pli altan, ol
la blankaj. La homa vo\^co kun la tempo \^ciam fari\^gas pli
malalta; tiel ekzemple tenoro deksesjara povas en la dudekkvina jaro
de la vivo fari\^gi baritono kaj en la tridekkvina --- baso. La
blonduloj havas vo\^con pli altan, ol homoj kun haroj mallumaj; la
unuaj havas ordinare vo\^con sopranan kaj tenoron, la lastaj ---
kontralton kaj bason. La tenoroj estas pli maldikaj, la basuloj pli
lar\^gaj kaj fortaj. Homoj pensantaj kaj inteligentaj havas pleje
vo\^con malaltan, sed malmulte pensantaj kaj supra\^{\j}emaj ---
altan. La vo\^co anta\u u la tagman\^go estas pli alta, ol post la
tagman\^go. Singardaj kaj prudentaj kantistoj evitas, kiel oni
scias, trinkojn alkoholajn, precipe la tenoroj; basuloj povas ilin
\^gui, kiom ili volas. Tenorojn oni pli renkontas en la landoj
sudaj, kaj en la nordo --- pli da basoj; tiel ekzemple la\u u la
vortoj de Delaunay \^ciuj gloraj francaj tenoroj elvenis el la suda
Francujo, dum \^ciuj basuloj --- el la departementoj nordaj.

\emph{Maso por kungluado de marmoro, gipso, porcelano k. t. p.}
Preni da estingita kalko pulvorigita 100 gramojn, da ovoblanko bone
batita 10 gramojn kaj da aluno pulvorigita 12 gramojn; miksi \^gin,
bone disfrotante, kaj ricevinte specon de nedensa pasto, \^smiri per
\^gi la randojn, kiujn oni volas kunglui, kunpremi ilin forte kaj
lasi tiel en la da\u uro de 48 horoj. Aldoninte al tiu \^ci
miksa\^{\j}o pli da ovoblanko a\u u da gluo solvita en nafto, por
\^gin pli maldensigi, oni povas \^gin uzi por laki vazojn kaj
diversajn objektojn, por fari ilin netramalseki\^gaj. Kartono, tiel
lakita, fari\^gas malmola, kiel ligno. \^Sipetoj, lakitaj per tiu
\^ci maso, estas belaj kaj fortikaj; tiel same anka\u u kuvoj kaj
aliaj lignaj vazoj kaj anka\u u vazoj argilaj, uzataj por konservado
de grasoj kaj fluida\^{\j}oj.

\emph{\^Cu lakto estas kun akvo}, oni povas scii\^gi en la sekvanta
maniero: bone poluritan trikilon (trikkudrilon) oni trempas en
profundan vazon da lakto, tuj \^gin elprenas returne kaj tenas \^gin
rekte malsupren. Se la lakto estas pura, tiam guto da lakto restos
pendanta sur \^gi; sed se e\^c tre malgranda kvanto da akvo estis
ver\^sita en la lakton, tiam nenia guto pendos sur la trikilo.

\emph{Ser\^cistoj de postesignoj} en Hindujo formas apartan kaston.
En kelkaj lokoj de la orienta Hindujo, precipe super la bordoj do
Gango kaj Hindo, la bruto prezentas la \^cefan ri\^cecon de la
lo\^gantoj; tial anka\u u \^stelado de \^cevaloj kaj bovoj estas tie
tre ofta. Sed tie al la posedantoj de brutaroj alportas helpon la
ser\^cistoj de postesignoj. Kiel preska\u u \^ciuj okupoj en
Hindujo, tiel la specialeco de postesignistoj transiras de patro al
filo; la "khojoj" --- tiel en la lingvo hinda sin nomas la
postesignistoj --- de frua infaneco sin ekzercas en sia metio. Bona
"khojo" scii\^gas facile, per kie la \^stelinto forkondukis la
bruton, kiel longe li haltis por nokti, a\u u por ripozi en tago,
kion li kondukis kun si k. t. p. Unu el la anglaj voja\^gantoj
rakontas, ke unu fojon oni \^stelis al li tola\^{\j}on kaj vestojn;
li tial sin turnis al ser\^cisto de postesignoj, kiu post unuhora
ser\^cado trovis la postesignon de la \^stelintoj kaj post dutaga
persekutado alkondukis la voja\^ganton al la loko, kie la
\^stelintoj ripozis post la migrado, jam tute certaj ke ili ne estos
punataj. La postesignisto klarigis al si diversajn detala\^{\j}ojn,
kiuj por la e\u uropanoj estis absolute nerimarkeblaj. Tiel ekzemple
en unu loko, kie la vojo disiris en kelkaj direktoj, la hindo per
\^cifitaj folioj sur la apudvojaj arboj rimarkis, ke per tie \^ci
kaj ne per alia vojo la \^stelistoj forkuris. La kronikoj de ju\^go
montras fakton, ke unu khojo iris la\u u la postesignoj de unu
mortiginto 300 kilometrojn, \^gis li fine trovis lin en la
malliberejo de provinca urbeto, kien la krimulo estis metita por ia
malgranda \^stelo, plenumita anta\u u kelkaj tagoj. Alian fojon
postesignisto, ricevinte la komision trovi \^steliston, kiu \^stelis
juvelojn el la trezorejo de la mahara\^gi, iris la\u u la
postesignoj de la krimulo \^gis la rivero Bias. Tie li perdis la
postesignon, \^car anta\u u unu horo trans la riveron iris
ta\^cmento da soldatoj el 200 personoj. \^Sajnis tute neeble, ke la
hindo inter la enpremoj de la piedoj de 200 homoj povus eltrovi la
bezonatan postesignon. Sed la khojoj ne perdas la kura\^gon anta\u u
malfacila\^{\j}oj. Tiel anka\u u nun la postesignisto iris returne
kelkajn kilometrojn kaj tiel bone konatigis sin kun la postesigno de
la \^stelisto, ke li helpis al si e\^c inter la postesignoj de
ducent homoj, kaj transirinte la riveron, li trovis la \^steliston.
Kompreneble la rezultatoj de tia sperto estas atingeblaj nur por la
kasto de khojoj kaj estas tenataj en plej granda sekreto anta\u u la
profanoj. La metio de postesignistoj tute ne estas sendan\^gera,
\^car la krimuloj uzas \^ciun okazon, por liberigi sin de la homoj
dan\^geraj por ili per helpo de veneno a\u u ponardo.

\emph{En la Malviva Maro}, kiel oni almena\u u pensis \^gis nun,
nenia viva organa kreita\^{\j}o povas ekzisti. La senvivecon de tiu
\^ci maro oni klarigadis al si per tio, ke la supraj partoj de la
akvo enhavas \^gis
25% da salo, tiel ke homo povas tie \^ci na\^gi sen movado de la manoj kaj
piedoj, kaj la malsupraj partoj estas en tre granda amaso sorbigitaj
de bromo, kio anka\u u malhelpas la vivon en la Malviva Maro. Dume
la esploroj de la germana instruitulo Cortet montris, ke malgra\u u
tiaj malfavoraj kondi\^coj en la sal-broma akvo de la Malviva Maro
ekzistas vivaj kreita\^{\j}oj, kiuj, estas vero, apartenas al la
organismoj de la plej malalta grado.

\emph{Esploroj de la maro}. Profesoro Agassiz publikigis la
rezultatojn de siaj esploroj pri la floro kaj fa\u uno de la Silenta
Oceano. Kiel oni vidas el la raporto, vivaj kreita\^{\j}oj ekzistas
en la maro nur \^gis certa profundeco, nome ne pli malsupre ol 171
klaftoj sub la supra\^{\j}o de la akvo. Pli malsupre komenci\^gas
jam la regno de absoluta morto, kiu sin etendas \^gis la spaco
trovanta sin en malproksimeco de 5O-60 klaftoj de la fundo de la
maro. Tie \^ci denove komencas montri\^gi vivaj kreita\^{\j}oj.
Kompreneble, la konstruo de tiuj \^ci bestoj kaj maraj
kreska\^{\j}oj estas konformigita al la grandega pezo de la akvo,
kiun tiuj \^ci kreita\^{\j}oj devas porti sur si.

\emph{Lumantaj briliantoj}. Jam en la jaro 1663 la siatempe glora
instruitulo, Robert Boyl, montris, ke ekzistas diamantoj, kiuj havas
la kapablon lumi en mallumo. Nun, kiel certigas la \^{\j}urnalo
"Nature", dedi\^cita al sciencoj naturaj, la \^hemiisto Kunz
konfirmas la supozojn de Boyl. Kunz elmetis diamantojn por la da\u
uro de pli a\u u malpli granda tempo al la efikado de la radioj de
la suno, post kio la briliantoj lumis en mallumo kun pli a\u u
malpli granda forto. Tiun \^ci saman aperon la dirita \^hemiisto
konstatis, kiam li la esploratajn diamantojn frotadis sur ligno,
drapo a\u u metalo. La cirkonstanco, ke la kapablon lumi la
diamantoj ricevas per frotado sur metaloj, pruvas, ke la lumo, kiun
ili ricevas, ne havas naturon elektran. La eco de la diamantoj,
montrita de Kunz, povas doni utilan pruvilon \^ce diferencigado de
briliantoj veraj de malveraj.

\emph{La fino de la mondo}. La penso pri fino de la mondo turmentis
la homojn jam longe kaj trovis esprimon en granda serio da anta\u
udiroj. Ne parolante jam pri la kredo je "millenium", t. e.
ekzistado de la mondo nur en la da\u uro de mil jaroj post Kristo
(kredo fondita kvaza\u u sur kelkaj esprimoj de la Biblio), ni
trovas en la mezaj centjaroj tre oftajn anta\u udirojn de tiu \^ci
speco. Bernardo el Turingujo en la jaro 960 anta\u udiris proksiman
finon de la mondo, li e\^c difinis la tempon, nome, kiam la festo de
Anunciacio falos sur la Grandan Vendredon. Tia fakto havis lokon en
la jaro 992, sed tiu \^ci jaro pasis, kaj la mondo ne esprimis e\^c
la plej malgrandan intencon perei. En la da\u uro de la X-a centjaro
\^ciuj re\^gaj decidoj komenci\^gadis per la vortoj: "\^Car la fino
de la mondo estas jam proksima\dots" En la jaro 1186 la astrologoj
ektimigis la popolojn per anta\u udiro, ke balda\u u fari\^gos
kuni\^go de \^ciuj planedoj. En la komenco de la XIV-a centjaro la
al\^hemiisto Villeneuve anoncis, ke en la jaro 1335 venos la
Antikristo. La glora hispana predikisto, Vincento Ferrier, certigis,
ke la mondo ekzistos nur tiom da jaroj, kiom da versoj sin trovas en
la Psalmoj, t. e. \^cirka\u u 2537 jaroj. La fino de la mondo estis
decidita por la jaro 1832, pri tio \^ci restis e\^c signo en la
kanteto de Beranger: "Finissons-en, le monde est assez vie\u u".
La anta\u udiro por la jaro 1840 faris grandegan impreson. La
fini\^go de la monda vivo devis havi lokon la 6-an de Januaro. Miloj
da homoj finis la aferojn terajn kaj atendis rezignacie la morton.
Ni tamen scias, ke anka\u u tiu \^ci anta\u udiro montri\^gis ne
tiel dan\^gera. Ni preterlasas multajn aliajn astrologiajn kaj
kabalajn anta\u udirojn kaj transiras al pririgardo de la teorioj,
sur kiuj liaj anta\u udiroj estis fondataj. La glora naturisto
Buffon elkalkulis, ke la tero iom post iom malvarmi\^gas, sed la
homaro povos vivi sur \^gi ankora\u u 93 291 jarojn, \^gis la \^selo
de la tero tiel forte malvarmi\^gos, ke \^cia vivo sur \^gi
malaperos. Sed de la tempo, kiam oni elmontris, ke la interna varmo
de la tero ne havas influon sur \^gian supra\^{\j}on kaj ke la vivo
sur \^gi dependas sole de la suno, tiu \^ci hipotezo havas jam
signifon nur historian. Alia teorio, kreita ankora\u u en tempoj
antikvaj, diras, ke la interna\^{\j}o de la tero konsistas el fajra
fandita maso kaj, se la vulkanoj, tiuj \^ci klapoj de sendan\^gereco
de la tera globo, \^stopi\^gos, la tero krevos, kaj \^giaj partetoj
malaperos en la universo. Per tia maniero ni devus perei ne de
malvarmo, sed de fajro. Ekzistas ankora\u u unu teorio, la\u u kiu
la mondo mortos malrapide kaj trankvile, en sekvo de ebeni\^go de la
tera supra\^{\j}o. La ventoj kaj pluvoj deportados la suprojn de la
montoj en la valojn, kaj la tero per la riveroj kaj riveretoj estos
iom post iom forportita en la maron. La neebena\^{\j}oj de la tera
supra\^{\j}o glati\^gos, la maro \^ciam pli superver\^sos la
bordojn, kaj kiam \^gi kovros la tutan supra\^{\j}on de la tero, la
tuta vivo \^cesi\^gos. La\u u la teorio de Adhemar la fino de la
mondo povas esti ka\u uzita de superakvego, elvokata de \^san\^go de
la centro de pezo de la tero kaj per tio \^ci anka\u u de
transna\^go de la glacio kaj akvo de la suda poluso al la norda. Tia
fakto havis lokon anta\u u 2020 jaroj kaj denove venos post 6307
jaroj. Ekzistas aliaj instruituloj, kiuj diras, ke ni povas perei en
la okazo, se ia el la kometoj kunpu\^si\^gos kun la tero. En tiu
\^ci okazo ne la ekbato estus terura, sed la \^hemia kuni\^go de la
densigita vosto de la kometo kun la oksigeno de nia atmosfero. Tiam
sur milionojn da mejloj en la spacego eksplodus belega bengala
fajro, kaj en tiu \^ci grandioza iluminacio estingi\^gus en unu
momento la tuta tera vivo. Fine en la sekvanta formo la glora franca
astronomo Kamilo Flammarion, pentras al ni la lastajn tagojn de la
tero: La suno estas stelo, kiu ne restas sen \^san\^goj. Jam nun sur
\^gia supra\^{\j}o montras sin multaj makuloj; ili sen\^cese
pligrandi\^gas, kaj la suno malvarmi\^gas. Fortrenante kun si la
teron kaj planedojn tra la frostaj dezertoj de la universo, \^gi
perdas sian varmon kaj lumon, kaj venos tempo, kiam \^gia
malvarmi\^ginta supra\^{\j}o ne radios pli lumon nek varmon, kiuj
donas vivon al la naturo. Sed la homaro ne alvivos \^gis tiu tempo
kaj ne rigardos la lastajn radiojn de la estingi\^ganta suno. En
sekvo de malgrandi\^go de la varmo de la suno \^ciam pli
vasti\^gados la regionoj glaciaj; la maroj kaj la teroj de tiuj \^ci
regionoj ne povos reteni vivon, kiu malrapide iom post iom
koncentri\^gos en la regionoj subekvatoraj, kie la lastaj infanoj de
la tero kondukos la lastan batalon kun la morto. Fine la tero,
senforta, seki\^ginta kaj dezerta, prezentos unu grandegan tombejon.
La suno fari\^gos ru\^ga, poste nigra, kaj nia tuta planeda sistemo
transformi\^gos en kolekton da nigraj masoj, turnantaj sin \^cirka\u
u tia sama nigra globego. Tia estas la hipotezo de Flammarion.

\emph{El \^Hinujo}. Trairante tra la plej \^cefaj stratoj de Pekino,
oni povas ofte en la loko de kruci\^go de du stratoj trovi grandegan
kovrilon, kiu sin etendas tra la tuta lar\^go de la strato, sur du
altaj kolonoj. Tiaj kovriloj estas metataj anta\u u detruitaj
sanktejoj a\u u en la lokoj de ia katastrofo, sed ordinare oni ilin
metas, kiam la regnestro devas traveturi. En tiaj okazoj kurieroj
anoncas al la popolo pri la proksimi\^go de la monar\^ho kaj la
urbanoj rapide sin ka\^sas en la domoj kaj ka\^sitaj anguloj. La
\^hina regnestro estas la sola persono en \^Hinujo, kiu ne scias,
kiel elrigardas \^hinoj, \^car liaj korteganoj ne estas
reprezentantoj de tiu \^ci popolo. Oni diras, ke anka\u u la plej
altaj provincestroj nenion scias pri la lo\^gantoj kaj rilatoj de la
regionoj konfiditaj al ili. --- Neniu el la \^hinoj savas iam
dronanton. \^Car en \^Hinujo oni kredas, ke la malbona spirito de
dronanto vagas super la supra\^{\j}o de la akvo, atendante novan
oferon, kaj ke tio \^ci estas la sola momento, kiam la demono havas
nenian servon. Tial neniu volas savi dronantan homon, el timo, ke la
kolerigita diablo ven\^gos por la for\^sirita de li certa akiro.

\emph{Jubileo de tabako}. La kvarcenta jubileo de la eltrovo de
Ameriko prezentas anka\u u tian saman jubileon de la eltrovo de la
tabako. Kristoforo Kolumbo en siaj memora\^{\j}oj lasis detalan
sciigon, en kia maniero la homoj konati\^gis kun tiu \^ci
kreska\^{\j}o. La glora maristo alvenis kun sia \^siparo al la
bordoj de Kubo. La hindoj, ekvidinte de malproksime blankajn homojn,
forkuris. Tiam Kolumbo sendis post ili du \^sipanojn, kiuj, veninte
en la hindan vila\^gon, ekvidis multajn hindojn, kiuj tenis en la
bu\^so iajn torditajn kaj ekbruligitajn foliojn, kies fumon ili
elspiradis. En tia maniero estas eltrovita la tabako, kiu de tiu
tempo ricevis multajn amikojn kaj kontra\u uulojn. Ankora\u u Las
Casas predikis kontra\u u la moro de fumado, kiun oni heredis de la
kolonianoj, sed la tuta elokventeco de la glora misiisto restis
senfrukta. La unua en E\u uropo kontra\u uulo de fumado de tabako,
Jakobo I, sin esprimis: "La pasio fumi tabakon estas abomena kaj
dan\^gera tiel por la kapo, kiel anka\u u por la pulmoj." En
Hispanujo la granda inkvizitoro Bartolomeo en la jaro 1659
malpermesis al la pastroj fumi tabakon en la da\u uro de unu horo
anta\u u la Diservo kaj du horoj post la Diservo, kaj sub minaco de
ekskomuniko kaj de la plej severaj punoj malpermesis al \^ciuj flari
tabakon en la da\u uro de la Diservo. La papoj Urbano VIII kaj
Innocento IV eldonis bullojn malpermesantajn la fumadon, kaj en tiuj
\^ci bulloj la bu\^so de persono fumanta estis komparata kun forno
de diablo. La Sorbono, invitita doni sian opinion, ne donis
respondon decidan, kaj la jezuitoj permesis fumi kaj flari tabakon
tiom, kiom \^gi servas al la digestado kaj al la sano. Ludoviko XIV
estis granda malamiko de tabako kaj malpermesis uzi \^gin \^ce la
kortego. La sultano Murado IV punadis la fumantojn per morto. En
Persujo oni la fumantojn batadis sur stangon. Tamen \^ciuj tiuj \^ci
malpermesoj neniom helpis kaj la pasio fumi tabakon disvolvi\^gis
\^ciam pli multe. Richelieu, kiu pasie amis flari tabakon, la unua
metis sur tabakon depagon en la alteco de 2 frankoj por \^ciuj 100
funtoj. En tia maniero la tabako de la jaro 1674 fari\^gis fonto de
enspezoj por la regno kaj alportis al la regna kaso milionojn.

\emph{Kio estas vegetarismo?} La vegetarismo, t. e. vivado per la
produkta\^{\j}oj de la regno kreska\^{\j}a kun escepto de \^ciuj
man\^goj, akirataj per difektado a\u u detruado de besta vivo,
ricevas en la lasta tempo \^ciam pli da partianoj. En Londono sin
trovas jam \^cirka\u u 40 restoracioj, en kiuj oni donas nur
man\^gon kreska\^{\j}an kaj kie \^ciutage \^cirka\u u 30 000
personoj akceptas malkaran kaj fortigan tagman\^gon. En Berlino en
Januaro de 1893 jam anka\u u ekzistis \^cirka\u u 20 restoracioj,
kaj en \^ciuj pli grandaj urboj de Anglujo, Ameriko kaj Germanujo
ekzistas por personoj solestarantaj la okazo vivi vegetare, ne
parolante pri la multegaj, distritaj en la tuta mondo familioj, kiuj
elstrekis la viandon el sia tabelo de man\^goj. Granda nombro da
societoj kaj gazetoj penas konatigi kun la afero de sensanga
nutri\^gado pli grandajn rondojn da homoj, kaj tiuj \^ci penoj estas
forte subtenataj de la \^ciam pli vasti\^ganta senmedikamenta
kuracado de malsanoj a\u u la sistemoj naturkuracaj de diversaj
direktoj, kiuj preska\u u \^ciuj faris el la for\^{\j}eto de
man\^gado de viando la unuan kondi\^con \^ce siaj malsanuloj. Estus
tial utile esplori pli proksime la motivojn de la vegetarismo. Ni
vidas, ke \^ce \^ciu viva esta\^{\j}o, de la plej malgranda fungo
\^gis la plej disvolvita besto laktosu\^canta, la partoj de la
korpo, kiuj estas difinitaj por la nutro, estas tute precize
aran\^gitaj por la prilaborado de la respondaj man\^goj: la dentoj,
stomako, intestaro de la viandoman\^ganta leono estas esence
malegalaj je la samaj organoj de la kreska\^{\j}oman\^ganta
elefanto, de la \^cioman\^ganta urso a\u u de la insektoman\^gantaj
birdoj. Ne povas esti alie anka\u u kun la homo, kiu la\u u la
aran\^go de sia korpo havas la plej grandan similecon je la
fruktoman\^ganta simio kaj sekve jam la\u u sia deveno devus anka\u
u esti fruktoman\^ganto. Ni vidas tamen, ke la nuntempa kultura homo
kontra\u ue preska\u u nur en esceptaj okazoj vivas de fruktoj kaj
ordinare uzas man\^gon miksitan, konsistantan el viando kaj
kreska\^{\j}oj, kaj tamen povas vivi kaj \^gis certa grado esti
sana. Sed tio \^ci montras nur, ke la naturo zorgis pri tio, ke ne
tuj pereu \^ciu ekzista\^{\j}o, kiu vivas ne precize la\u u \^giaj
reguloj. Sed difekto longe ne povas forresti, kaj la \^ciam pli
grandi\^ganta malsanemeco kaj korpa degenerado de nia gento, kiu
parte certe dependas anka\u u de aliaj mal\^gustaj kondi\^coj de
vivo, montras kun timiga klareco, ke la miksita man\^go tamen ne
povas esti la \^gusta. Kio do estas pli proksima, ol la reiro al la
maniero de vivado, dezirita de la naturo, t. e. al la vegetarismo?
Preska\u u \^ciuj bu\^cataj bestoj estas malsanaj, la grasigado
estas fondita sur agado malsaniga (manko de sango kaj grasa
degenero), la viando sekve estas preska\u u \^ciam malsana kaj
malbonigita; krom tio komenci\^gas \^gia putrado, kiam la vivo
forlasis la korpon. La homo kultura sekve en efektivo nutras sin per
mortinta\^{\j}o. Dume la man\^ga\^{\j}o de la vegetarano estas
ankora\u u plena de vivo. Se oni fruktojn kaj grenon elmetas al
respondaj influoj, tiam nova vivo naski\^gas el ili. Kaj al ni
\^sajnas, ke por la konservado de nia vivo, la plej bone ta\u ugas
nur tia man\^go, en kiu ne mortis ankora\u u la \^germo de vivo. Nur
tiam, kiam fruktoj kaj greno fari\^gas putraj, ili trovas sin en tia
stato, en kia la viando sin trovas jam tuj post la bu\^co! \^Cu
estas tial miro, ke la gazetoj konstante alportas sciigojn pri
veneni\^go de viando, kiu sekvigas a\u u malfacilajn malsanojn, a\u
u la morton?! Kontra\u ue, \^ce man\^go kreska\^{\j}a tiu \^ci
dan\^gero estas preska\u u absolute esceptita, kaj se nur la ceteraj
kondi\^coj de la vivo estas iom konformaj al la naturo, la
vegetarano \^guas pli bonan sanon, ne estas tiel forte elmetita al
malsanoj a\u u elportas tiujn \^ci multe pli bone, ol la
viandoman\^ganto. Nemezurebla estus la rekompenco, se man\^go
kreska\^{\j}a okupus la lokon de viando. Nun la vila\^gano estas
sklavo de sia bruto. La dekono de la nun bezonata tero estus
sufi\^ca por la nutrado de lia familio, se li plantus sur \^gi
fruktojn kaj legomojn. Al tio \^ci lia laborado estus pli pura, pli
facila kaj pli interesa, ol \^ce la malnova mastrumado bruta kaj
kampa, kie la plejmulto da laboro devas esti uzata por produktado de
man\^go por la bruto. Kun plantado de fruktoj anka\u u la homo klera
volonte sin okupados, \^car \^gi postulas lian tutan spriton, lasas
lin \^ciam pensi pri la estonteco kaj per la originaleco de la
laboroj ligas la printempon kun la a\u utuno, la someron kun la
vintro kaj en tia maniero faras el lia agado \^ciam laboron sub
espero. Tutan homan vivon, e\^c centjarojn da\u uras la fruktoj de
laborado de fruktoplantisto kaj ofte nur la infanoj atingas plenan
\^guon de la plantoj, kiujn iliaj patroj faris. Nenie la nun tiel
ofta, nur por la momento kalkulita kaj sekve nemorala raba
prilaborado havas pli malmulte lokon, ol \^ce plantado de fruktoj.
Tial la fruktoplantado devas influi noblige sur tiujn, kiuj sin
okupas je \^gi. Pli grandaj ankora\u u estas la moralaj rezultatoj
de la for\^{\j}etado de man\^go vianda, kiu estas ja la unua
kondi\^co por la disvolvi\^go de la plantado de fruktoj kaj legomoj,
dank' al la per tio \^ci forte pligrandigita bezono de nutra\^{\j}o
kreska\^{\j}a. La akirado de viando estas neebla sen barbareco; \^gi
povas esti farata nur per mortigado de vivaj, sentantaj
ekzista\^{\j}oj, kiuj estis kreitaj por vivo, ne por morto. En \^ciu
senpartia, sentanta homo vivas abomeno al detruado de vivo, kaj nur
la kutimo povas silentigi la internan vo\^con, kiu krias al ni: "ne
mortigu!" Multaj tabloj restus sen viando, se la mastrino de la
domo devus mem mortigi barbare la beston. En neklara konscio de la
nemoraleco de la bu\^sado oni ja sen tio forpelis jam en
malproksimigitajn bu\^cejojn la scenojn de mortigado de bestoj, kiuj
ne estas eblaj sen fluoj de sango, sen kompatinda kriado kaj agonio
de la mortantaj bestoj, sen sangitaj vestoj kaj manoj de la
bu\^cistoj. Riveroj da sango, kriado de mortantaj, odoro de putrado
--- kia kontrasto al la kor\^gojiga vido kaj bonodoro de \^sar\^gita
fruktarbo a\u u de pure aran\^gitaj magazenoj de fruktoj! La
ankora\u u ne malbonigita instinkto de la infanoj ne dubas, kion
\^gi devas elekti --- sangan \^cifonon de mortinta besta korpo a\u u
ridantan pomon, bongustan vinberon. Per la \^ceso de grasigado kaj
bu\^cado tiel delikati\^go de moroj estus necesa sekvo de la
vasti\^go de la vegetarismo. Ke la bestoj nin man\^gos, se ni ilin
ne man\^gus (la ordinara rediro de la nevegetaranoj), ni ne devas
timi; oni ja nur ekmemoru la grandan penon, kiun la brutedukisto
havas kun la edukado de bu\^caj bestoj. Balda\u u ekzistus nenia
bruto, se neniu plu zorgus pri tio \^ci. Kaj la laborado de la
bestoj ja sen tio \^ciam pli kaj pli estas anstata\u uata per forto
de ma\^sinoj, kaj anstata\u u la bu\^cado la homoj balda\u u
fordonus sin al la pli sana profesio de \^gardenisto, kiam ili
vidus, ke tio \^ci donas pli grandan profiton. Ke la homo tre bone
povas ekzisti kaj esti tre forta de man\^go kreska\^{\j}a, e\^c se
\^gi estas kunmetita tre unuforme, tion \^ci pruvas la milionoj da
vila\^ganoj, kiuj nur malofte ricevas sur sian tablon peceton da
viando kaj el kies mezo ja \^ciam nova fre\^sa sango venas en la
malsanemajn, sangomankajn familiojn de la urbo. La vila\^ganoj sekve
senkonscie estas praktikaj vegetaranoj. Sed kiel ri\^ce oni povas
kunmeti sian tabelon de man\^goj, oni vidas el la sekvanta tabelo de
kreska\^{\j}aj nutra\^{\j}oj, kiuj per ri\^ceco de sia enhavo ne
lasas ion por deziri: 1) Fruktoj: kun kernoj, kun ostetoj, kun
\^seloj, nuksoj, beroj, sudaj fruktoj; frukta\^{\j}oj, kiel ekzemple
fruktoj sekigitaj, fruktaj gelatenoj, marmeladoj, konfita\^{\j}oj,
fruktaj sukoj; oleo de olivoj, de nuksoj, de migdaloj, butero de
kokosoj. 2) Grenoj: tritiko, sekalo, hordeo, aveno, rizo, maizo,
poligono, spelto, milio k. t. p. en iliaj prilabora\^{\j}oj en formo
de faruno, grio, makaronoj, vermi\^celoj, pano, baka\^{\j}o kaj
faruna\^{\j}o de diversaj specoj. 3). Fruktoj \^seletaj: faboj,
pizoj, lentoj, kiel anka\u u la faruno el ili. 4). Legomoj: radikaj,
foliaj, trunketaj, floraj kaj fruktaj; fungoj, salatoj kaj supaj
herboj. Fine oni devas ankora\u u montri, ke la kuirejo de
vegetaranino estas esence pli simpla, ol la kuirejo vianda. En
\^ciuj tempoj memstaraj pensantoj levadis sian vo\^con kontra\u u la
mortigado de bestoj por la celoj de nia nutrado; e\^c tutaj popoloj,
kiel ekzemple la pli altaj kastoj de la hindoj-buddistoj, uzas pro
motivoj religiaj nur man\^gon kreska\^{\j}an. Ke por la sindefendo
la mortigo de al ni malutila besto estas permesata, e\^c \^suldo,
tion anka\u u la vegetarano ne neas; en tiaj okazoj esta\^{\j}o pli
malalta devas cedi al esta\^{\j}o pli alte organizita. Ni devus tro
multe paroli, se ni volus priskribi pli detale la \^ciuflankan
efikadon de la vegetarismo. Ni devas nin kontentigi je la supre
donitaj mallongaj rimarkoj kaj sendi la dezirantojn al la tre ri\^ca
literaturo, kiu aperis rilate tiun \^ci objekton. Kiu deziras ekkoni
la objekton pli proksime, tiu povas venigi per ia librejo la
katalogon de Max Breitkreuz en Berlino, de Theod. Grieben (L.
Fernau) en Lejpcigo, do Hartung et Sohn en Rudolstatt, a\u u li
turnu sin al ia el la kluboj vegetaranaj, kiuj ekzistas en \^ciuj
pli grandaj urboj.

\emph{La lingvo de \^Hinujo}. En E\u uropo oni ordinare pensas, ke
en la tuta \^Hinujo oni parolas nur unu lingvon --- la \^hinan.
Estas vero, ke la lo\^gantoj de Pekino, kiel anka\u u la lo\^gantoj
de Kantono, \^San\^hajo, Fut\^sano a\u u Amojo parolas \^hine, sed
de la dua flanko estas anka\u u vero, ke la plej granda parto de la
lo\^gantoj de unu el la diritaj urboj povus kompreni la lo\^ganton
de alia urbo ne pli bone, ol ekzemple la Berlinano la Londonanon a\u
u la Parizano la Holandanon. La naturo de la diversaj dialektoj de
\^Hinujo havas nenion komunan kun la "patois" a\u u la simpla
"dialekto de la ordinara vivo": ili estas parolataj de la plej
altaj statoj, kiel anka\u u de la simpla popolo, de la instruituloj,
kiel anka\u u de la malklera amaso, de la oficisto, kiel anka\u u de
la "kuli". La dialekto estas aparta lingvo, unu el la multaj kaj
tre malsamaj inter si lingvoj, kiujn oni trovas en \^Hinujo. Estas
vero, ke ili estas parencaj inter si kaj trovas sin inter si
reciproke en tia sama rilato, kiel ekzemple la araba al la hebrea,
sira kaj aliaj semitaj lingvoj, a\u u kiel la germana al la angla,
holanda, dana, sveda k. t. p. Tiujn \^ci multegajn "dialektojn"
oni povus la\u u la "Orient-Azia Blojdo" dividi en la sekvantajn
ok \^cefajn klasojn: la Kantona, Hakka, Amoja, Svatana, \^San\^haja,
Ningpoa, Hajnana kaj Mandarina. El tiuj \^ci lingvoj la lasta estas
la plej juna; tio \^ci renversas la tre vastigitan opinion, ke la
Mandarina dialekto estas la lingvo de \^Hinujo kaj ke la ceteraj
lingvoj estas nur dialektoj. La Kantona lingvo pli ol la Mandarina
estas simila al la antikva lingvo de \^Hinujo, kiu estis parolata
anta\u u \^cirka\u u 3000 jaroj. La plej multe vastigita lingvo
estas la Mandarina, kiu en tia a\u u alia formo estas parolata en
dek kvar a\u u dek kvin el la dek na\u u provincoj, en kiujn
\^Hinujo estas dividita. Malgra\u u la diversaj dialektoj oni povas
per la lingvo Mandarina komprenigi sin \^cie, kie tiu \^ci lingvo
estas parolata. Se oni kalkulos, ke la lo\^gantaro de \^Hinujo estas
360 000 000, oni povas diri, ke \^cirka\u u 300 000 000 el ili
parolas la lingvon Mandarinan. \^Ciuj mandarinoj devas koni tiun
\^ci lingvon, kaj \^ciuj, kiuj \^gin ankora\u u ne parolas, devas
\^gin lerni. La aliaj lingvoj de \^Hinujo estas parolataj de pli
malgranda nombro da homoj, tamen en \^cia okazo tiu \^ci nombro
estas ankora\u u sufi\^ce granda. Tiel ekzemple \^cirka\u u 20
milionoj parolas la lingvon Kantonan en tiu a\u u alia formo. La
enkonduko de unuforma lingvo en \^Hinujo anstata\u u la multegaj
tiel nomataj dialektoj estas, almena\u u por la plej proksima tempo,
ankora\u u nur revo. Anta\u u \^cirka\u u 200 jaroj la imperiestro
Kang-Hi ordonis aran\^gi por tiu \^ci celo en diversaj partoj de la
regno lernejojn, tamen la plano donis nenian rezultaton. Ke \^Hinujo
en la estonteco havos ian unuforman lingvon, estas tamen tre
kredeble, kaj la lingvo Mandarina kredeble fine fari\^gos la regna
kaj nacia lingvo de la tuta \^Ciela Imperio.

\emph{Kiel elrigardas la norda poluso?} La Amerika astronomo Jonson,
su\-po\-zan\-te, ke la ekspedicio de Nansen havos sukceson,
priskribas la aperojn, kiujn vidos la esplorantoj sur la norda
poluso. Tiel, ekzemple, la taga lumo da\u uros sen interrompo de la
21 de Marto \^gis la 22 de Septembro, la tuta resta parto de la jaro
konsistos el malluma nokto, simila al niaj a\u utunaj noktoj. La
steloj estos vidataj, sed Nansen ne vidos ilian levi\^gon nek
mallevi\^gon. \^Cirka\u ue regos silento premanta. Poste en tiu \^ci
maro glacia komenci\^gos ventegoj kaj huraganoj. Inter senlima
mallumo oni a\u udos la bruadon de ventegoj, \^gemojn, plorojn,
kvaza\u u en la mondon en\^siri\^gis \^ciuj fortoj de la infero, kaj
al \^cio tio \^ci la nokto mallumega, nigra, senfina\dots Sur la
malfortan \^sipon premas de \^ciuj flankoj montoj da glacio, sube
estas maro kaj nenio pli. La ventegoj bruas kaj fajfas, la \^sipo
krakas terure. Poste Nansen ekvidos naski\^gantan lumon, flaman
brilon de matena \^cielru\^go, anoncanta la venon de la suno. La
\^cielo kovri\^gos per ora koloro kaj en la da\u uro de tri monatoj
iom post iom levi\^gados la suno, kaj en la da\u uro de la tri
sekvantaj monatoj \^gi mallevi\^gados. Sed en la tuta da\u uro de
tiuj \^ci ses monatoj estos konstante lume. Poste denove fari\^gos
mallumo kaj nokto.

\emph{Rimedo kontra\u u ondegoj}. En unu el la lastaj kunsidoj de la
franca societo de savado de dronantoj estis montrita nova aparato,
kiu kvietigas la ondegojn kaj konsistas el reto, plektita el facila
sed fortika materialo. Tiu \^ci reto efikas kiel oleo, kiu, kiel oni
scias, havas la kapablon kvietigi la forton de la ondoj. La provoj,
kiujn oni faris kun tiu \^ci reto sub Quiberon, sur la spaco de 800
kvadrataj metroj, donis rezulta\^{\j}ojn tiel bonajn, ke la franca
ministro de maro difinis specialan komision, por esplori la efikadon
de la reto sur la ondojn de maro.

\emph{El la historio de la grafologio}. Jam longe oni anta\u uvidis
la eblon diveni la karakteron de la persono el la karaktero de \^gia
skribado. En la jaro 1602 la itala instruitulo Bilbo publikigis
verkon: "Pri la rimedo ekkoni la morojn kaj la ecojn de skribanto
la\u u lia skribado". Lavater okupadis sin je tiu \^ci sama
objekto, sed nur en la komenco de tiu \^ci centjaro (1806) la franco
Johanno Hipolito Michon alkondukis tiun \^ci arton, a\u u kiel aliaj
diras, "sciencon" al sufi\^ca grado da perfekteco, donis al \^gi
certan formon kaj metodon. Nun ekzistas en Parizo Societo Grafologia
(strato de Bonaparte N-ro 62); en la nombro de \^giaj honoraj
prezidantoj sin trovas: Aleksandro Dumas (filo) kaj Mgr. Barbier de
Montant; sub la redakcio de Varinard la societo eldonas monatan
gazeton "La Graphologie", kiu inter aliaj donas "portretojn
grafologiajn". Anta\u u nelonge la Luksemburga ju\^gistaro oficiale
konfesis la ekzistadon de la grafologio, turnante sin al la
redaktoro de la dirita gazeto en unu ju\^ga afero, en kiu ordinaraj
kaligrafoj ne povis helpi. S-ro Varinard solvis bone la problemon.
De tiu tempo la ju\^go Luksemburga fari\^gis eterna abonanto de la
"Graphologie". Krom tiu \^ci servo publika, la redakcio faras
multajn servojn privatajn, divenante ekzemple al enamitoj pri la
karaktero de iliaj idealoj, al bankieroj pri iliaj kontoranoj k. t.
p.

\emph{Higieno de longevivado}. D-ro Javal dissendis anta\u u nelonge
serion da demandoj al multaj centjaraj maljunuloj, dezirante el
iliaj respondoj kunmeti regulojn, kion oni devas fari, por longe
vivi. \^Cirka\u u 50 personoj sendis al li respondojn. Ilia esenco
estas la sekvanta: Man\^gado simpla kaj sufi\^ca; precipe
kreska\^{\j}a; la plejparto de la centjaruloj tute ne uzadis
alkoholon, kelkaj trinkis vinon. \^Ce la man\^gado esti varme
vestita; tabakon ne fumi. Estas strange, ke preska\u u \^ciuj ili
amas franda\^{\j}ojn, ekzemple sukeron. Fine \^ciuj senescepte
sciigis, ke ili tra la tuta vivo kiel eble plej pene evitadis
eksciti\^gojn.

\emph{El Kalifornio}. Kiel oni scias, Kalifornio apartenas al la
lokoj tre ri\^caj kiel per mineraloj, tiel anka\u u per
produkta\^{\j}oj de la regno kreska\^{\j}a. La kreska\^{\j}oj tie
ofte distingas sin per tiaj grandegaj mezuroj, ke ni, e\u uropanoj,
tute ne povus \^gin kredi. Tuj \^ce la eniro en la montojn, en la
loko Kalaveraso, sur la alteco de 1400 metroj super la nivelo de la
maro, trovi\^gas angulo de arbaro, en kiu kolekti\^gis tuta familio
da grandeguloj el la speco de la tiel nomataj "mamontaj arboj".
Grizaj, maljunaj, ornamitaj per nudaj bran\^coj nur sur la pintoj,
ili staras jam ne la unuan miljaron kaj estas \^cirka\u ukreskitaj
de densa musko. Flamaj floroj nesti\^gas en la fendoj de iliaj
radikoj kaj sur la sulkita \^selo. La fre\^sa muska supra\^{\j}o de
la tero estas tre mola sub ili. Iliaj folietoj estas tre malgrandaj,
kaj ilia korpo --- la ligno --- estas malmola, de ru\^geta koloro,
sed tre rapide nigri\^gas de la tempo. La plej granda el la
Kalaverasa familio de tiuj \^ci arboj, sub la nomo "Maljuna arbo",
jam pli ne ekzistas. Kiam \^gi ankora\u u vivis, \^gi havis la alton
de 90 metroj kaj \^cirka\u u 10 metrojn en la diametro. Anta\u u 40
a\u u 60 jaroj la e\u uropaj enmigrintoj-kolonianoj ekvolis faligi
tiun \^ci arbon. \^Cu \^gi malhepis al la konstruo de kolonio a\u u
al trameto de vojo, \^cu oni simple bezonis tiun arbegon por iaj
laboroj, ekzemple por fari el \^gi trabojn a\u u tabulojn, --- fakto
estas, ke la kolonianoj komencis \^gin elhaki. Trasegi a\u u trahaki
grandegulon, kiu havas 10 metrojn en la diametro, estas afero ne
facila. Kian grandegan multegon da ekbatoj devis fari la malgrandaj
hakiloj de la laborantoj \^gis la atingo de la interna\^{\j}o! Ili
submordetus \^gin tiel same, kiel mordetas la arbaraj skaraboj niajn
betulojn, alnojn kaj pinojn. Sed la laboro de la kolonianoj iris pli
rapide. Per grandegaj boriloj ili traboris la arbon en multaj lokoj,
hakadis en la da\u uro de ses semajnoj kaj fine \^gin faligis. La
arbo falis, kaj sur la supra\^{\j}o de la fre\^sa \^stipo, kiun oni
purigis kaj glatigis, estis konstruita --- areno por dancoj! La
a\^gon de la arbo oni difinis per la nombro de la tavoloj sur la
\^stipo, kaj oni trovis, ke la arbo havis la a\^gon de 3000 jaroj!
En la da\u uro de tri semajnoj oni de\^siradis de la subhakita
grandegulo la \^selon; la \^selo havis la dikecon de 2/3 da metro
kaj estis sendita al la ekspozicio en San Francisko. Ni montros
ankora\u u unu ekzemplon de arbara grandegulo. Tio \^ci estas la
tiel nomata "\^Cevala arbo", kiu mem falis kaj ku\^sas jam kelkajn
centjarojn. \^Gia grandega trunko havas inter la radikoj, kiuj estas
elturnitaj eksteren, \^cirka\u u 40 metrojn en la diametro, kaj en
la malplena interna\^{\j}o, en la spaco de 30 metroj, rajdanto sur
\^cevalo povas tute libere traveturi la tutan arbon trae! Kia
nevidebla forto, kia ventego a\u u tertremo povis el\^siri el la
tero tian grandegulon, kies a\^go dank' al la putreco kaj neklareco
de la tavoloj ne povas esti difinita e\^c proksimume! Pri aliaj
arboj de tiu loko ni diros nur kelkajn vortojn. Grandeguloj en tiu
loko ekzistas multaj ankora\u u nun, kaj oni nomas ilin \^ciun per
propra nomo. Inter ili estas: "Tri fratinoj", \^ciu po 90 metroj
de alto; "Geedzoj", malmulte pli malgrandaj ol la "Fratinoj".
Ekzistas tuta "familio", konsistanta el 26 arboj-kolosoj; la plej
maljuna el ili --- la patro --- havis la alton de \^cirka\u u 150
metroj; subputrinte kaj falinte, \^gi rompi\^gis sur unu el siaj
najbaroj en du partojn, kaj la pli malgranda el ili, havanta la
longon de 90 metroj, ku\^sas sur la tero en nia tempo. [N.
Ku\^snir.]

\emph{Nova nutra kreska\^{\j}o}. Sur unu el la insuloj de Japanujo
oni okaze eltrovis novan nutran kreska\^{\j}on, al kiu oni donis la
nomon "Polygonum Saghalae" de la nomo de la insulo Sagalo, sur kiu
oni \^gin eltrovis. \^Gi distingi\^gas per kreskado tiel rapida, ke
en la da\u uro de tri \^gis kvar monatoj \^gi atingas la altecon de
2 metroj kaj kovri\^gas per grandegaj folioj, kiuj prezentas bonegan
man\^gon por la bruto. Unu arbeta\^{\j}o de tiu \^ci kreska\^{\j}o
ombras la spacon de tri kvadrataj metroj, kaj la tuta maso da folioj
de unu arbeta\^{\j}o pezas 30 kilogramojn. La trunko kaj \^ciuj
aliaj partoj de la kreska\^{\j}o enhavas en si multe da amelo kaj
aliaj nutraj \^stofoj. Tiu \^ci kreska\^{\j}o, kiel montris la unuaj
provoj, ne postulas penan edukadon. Anta\u u nelonge kelkaj
arbeta\^{\j}oj estas elsenditaj Francujon, kie oni intencas fari
provojn de \^gia plantado en granda mezuro.

\emph{Konservado de vivaj fi\^soj}. Oni scias, ke kelkaj specoj da
fi\^soj enfosas sin por la vintro en \^slimon kaj restas en \^gi en
la stato de plena rigidi\^go. La \^hinoj turnis atenton sur tiun
\^ci fakton kaj, apogante sin sur \^gin, elpensis spritan rimedon
por konservado de vivaj fi\^soj. La kaptitan fi\^son ili tuj
\^cirka\u ukovras per malseka argilo kaj ka\^sas \^gin en glaciejo.
Post kelkaj, e\^c post dek monatoj la tiel konservita fi\^so,
enlasita en akvon, revivi\^gas. En tia maniero la pli ri\^caj
\^hinoj konservas en siaj glaciejoj grandajn provizojn da vivaj
fi\^soj. Tamen ne \^ciujn specojn da fi\^soj oni povas konservi en
tia maniero.

\emph{Kolonio komunista} de A\u ustraliaj elmigrantoj estas nun
fondata en Paragvajo. Multaj personoj kune kun siaj familioj
forlasis sian landon kaj formigris en la novan patrujon. La Belga
konsulo en Buenos-Ayres sendis al sia registaro la sekvantajn
sciigojn pri tiu \^ci kolonio: La registaro de Paragvajo fordonis al
la kolonianoj por \^ciam spacon da tero, kiu okupas 40 mejlojn
kvadratajn kaj ku\^sas sur la norda bordo de la rivero Tebikuara, en
la interspaco de 6 mejloj de la fervojo Villa Rica. \^Gis la jaro
1894 devas lo\^gi\^gi tie 400 familioj en la nombro de 2000
personoj. La profitoj, kiuj estos atingataj per komuna laborado,
devas \^ciujare esti dividataj inter \^ciuj plena\^gaj personoj,
tiel viroj, kiel anka\u u virinoj. La kolonio estos regata de
direktoro kaj de aldonitaj al li konsilanoj. \^Ciuj plena\^gaj
lo\^gantoj de amba\u u seksoj havas la rajton de vo\^cdonado. La
lernejoj kaj malsanejoj estos tenataj je la kalkulo de la komunumo.
\^Ciuj religioj havas egalajn rajtojn, kaj \^ciu familio povas havi
religion, kiun \^gi volas. La societo posedas sufi\^cajn rimedojn,
por atendi 1 l/2 jarojn la unuan rikolton de la kampoj. La spaco da
tero, donita de la Paragvaja registaro estas tre fruktoporta, sed
arbara kaj postulanta multe da prepara laborado.

\emph{La deveno de la kiso}. La konata itala instruitulo Lombroso
publikigis en la "Nouvelle Revue" artikolon pri la deveno de la
kiso. La\u u lia opinio \^ciuj popoloj sova\^gaj kaj e\^c
duone-civilizitaj, enkalkulinte anka\u u la Japanojn, ne vidas en la
kiso simbolon de amo. Tiel same anka\u u la Novo-Zelandanoj,
Somalisoj, Eskimosoj k. t. p. Inter kelkaj popoloj oni ne diras:
"kisu min", sed "flaru min". Lombroso diras, ke la kiso
naski\^gis iom post iom kaj havas sian komencon en la akto patrina,
t. e. en la nutrado de la brustaj infanoj de ilia patrino. \^Gis nun
kelkaj virinoj-patrinoj en tia maniero dorlotas siajn infanojn. Tiu
\^ci moro estas precipe disvastigita inter la lo\^gantoj de la
insuloj Fid\^gi. Ili ne havas vazojn por trinki, kaj la homoj
trankviligas tie sian soifon rekte el la rivereto, per helpo de
tubeto, kaj la infanoj tie mortus de soifo, se la patrinoj ne
trinkigadus ilin per enver\^sado al ili akvon en la bu\^son el sia
propra bu\^so. En la poemoj de Homero kaj Heziodo ne estas dirita
e\^c unu vorto pri la bu\^so nek pri la kiso en senco de seksa amo,
tie estas parolate pri ili nur kiel pri kareso patrina. Tiel
ekzemple Hektoro en la sceno kun Androma\^ho tute ne kisas sian
edzinon, sed nur karesas \^sin per la mano. Nenio anka\u u estas
dirita pri kiso en la rakontoj pri Venero kaj Marso, pri Uliso kaj
Kalipso, Uliso kaj Circeo, Pariso kaj Heleno a\u u Hero en la XIV
kanto de Iliado. Ne ekzistas tie e\^c unu epiteto, kiu karakterizus
la bu\^son de Heleno, Androma\^ho, Brizeido, Kalipso a\u u Circeo.
En la antikvaj Hindaj poemoj Lombroso ne trovis e\^c postesignon de
kiso erotika. \^Cio tio \^ci montras, ke en tiuj tempoj \^gi ne
ekzistis, sed naski\^gis nur en la tempo de pli granda disvolvi\^go
de la civilizacio.

\emph{Memvola vivisekcio}. Du Amerikaj kuracistoj anta\u u nelonge
publikigis en la gazeto "New-York Herald" originalan anoncon, en
kiu ili proponas 5000 dolarojn da rekompenco al la persono, kiu
konsentus, ke oni faru al \^gi operacion, konsistantan en farado de
truo en la stomako kaj fermado \^gin poste per vitro, por farado de
esploroj super la funkciado de la stomako. Al tiu \^ci propono
alsendis sian konsenton 142 homoj, precipe laboristoj. La "feli\^ca
elektito" fari\^gis unu atleto, kiu ne havis sukceson en la cirkoj.

\emph{Sekigado de mar\^coj}. Malgra\u u la grandegaj spacoj da
ankora\u u ne okupita prilaborebla tero, la Amerikanoj jam frutempe
pensas pri akirado de novaj pecoj da tero. En la \^stato Georgio oni
komencis la sekigadon de grandegaj mar\^coj, kiuj \^gis nun estis
nur la loko de lo\^gado de aligatoroj kaj aliaj akvaj bestoj. Unu
\^cefa kanalo kaj multo da kanaloj paralelaj flankaj liberigos de
sub la akvo spacon da 250 000 hektaroj da tero. Ekster tio \^ci
funkciados, kiel ordinare en Ameriko, ma\^sinoj vaporaj, pumpantaj
la akvon en la kvanto de 135 000 litroj en minuto. Sub la tavolo da
\^slimo kaj sur la nemultaj insuletoj de la mar\^co oni jam trovis
sendubajn postesignojn de ekzistado de la antikvaj lo\^gantoj de
Ameriko en tiu \^ci loko. La ar\^heologoj promesas al si ri\^can
rikolton, kiam la in\^genieroj finos la laborojn senakvigajn.

\emph{Nova diamanto}. La gloro de la Kohi-Noor, Regento, Orlovo kaj
aliaj eksterordinare belaj kaj grandaj diamantoj devas nun pali\^gi
anta\u u la diamanto-grandegulo, kiun oni anta\u u nelonge trovis en
la suda Afriko, en la libera regno Oran\^go, en la loko
Jagersfontein. La 30-an de Junio de la jaro 1893 unu kafro,
interparolante kun la observanto de laboro, rimarkis, ke proksime io
lumas sur la tero. Li kovris la lumantan objekton per la piedo, kaj
kiam la observanto foriris li elfosis grandegan \^stonon, havantan
la alton de \^cirka\u u 8 centimetroj kaj la lar\^gon de \^cirka\u u
5 centimetroj. \^Gia pezo estas 971 karatoj. \^Gia prezo superas
\^cion, kion oni \^gis nun pagadis por diamantoj. La kafro poste mem
fordonis \^gin al la entreprenanto; li ricevis por \^gi 150 funtojn
da sterlingoj, \^cevalon kaj selon kaj revenis hejmen forte
feli\^ca. Estas interese, ke ia societo farmis la tiean fosejon kun
la rajto a\^cetadi la\u u pezo en karatoj \^ciajn \^stonojn,
suprajn, malsuprajn kaj indiferentajn; la kontrakto fini\^gis la
30-an de Junio de tiu \^ci jaro, kaj tiu grandega diamanto estis
preska\u u la lasta \^stono, kiun oni trovis en tiu \^ci tago. Al la
posedantoj oni proponis jam por \^gi 1 500 000 frankojn, sed kelkaj
pensas, ke la prezo atingos la sumon de 15 000 000. La \^stono havas
koloron blanka-bluan kaj estas preska\u u tute pura. Sur \^gi
trovi\^gas malgranda nigra makuleto, pri kiu oni pensas, ke \^gi
malaperos post la facetado. Nun la \^stono sin trovas en Londono.

\emph{Timo}. En unu el la Amerikaj bestejoj oni faris anta\u u
nelonge provojn, por konvinki\^gi, en kia grado estas vera la
proverba timo de leonoj, tigroj, elefantoj kaj aliaj grandaj bestoj
anta\u u simpla muso. Anta\u ue oni enlasis muson en la ka\^gon, en
kiu sin trovis du grandegaj leonoj; tiuj \^ci forsaltis kun timego,
terure kriante kaj penante eli\^gi el la ka\^go. Iom post iom ili
trankvili\^gis tiom, ke ili \^cirka\u uflaris la gaston kaj poste
jam turnis sur lin nenian atenton. Tiel same tenis sin granda re\^ga
tigro. Elefanto tremis de teruro kaj maltrankvile movadis la
rostron; sed lia dresita kolego kun plej malvarma sango dispremis la
gaston per la piedego. Hienoj kaj lupoj, rigardante la aferon pli
praktike, sufokis tuj la ratojn kaj musojn, kiujn oni enlasis al ili
en la ka\^gon.

\emph{Viroj kaj virinoj}. Profesoro Bucher el Lejpcigo elkalkulis la
rilatan nombron de la viroj kaj virinoj en E\u uropo. Montri\^gas,
ke, esceptinte Grekujon, Italujon kaj la regnojn Danubajn, inter la
300 milionoj da lo\^gantaro en E\u uropo sin trovas 4 1/2 milionoj
pli da virinoj, ol da viroj. La plinombreco de la virinoj estas la
plej forte disvolvita en la a\^go, kiu estas la plej responda por
edzi\^go, nome en la a\^go de 20 \^gis 30 jaroj; la plej rimarkebla
\^gi estas en la urboj, precipe en tiuj urboj, kie staras tre
malgrandaj garnizonoj, ekzemple en Svisujo a\u u Skandinavujo. En
Portugalujo kontra\u u 1 000 viroj estas 1 091 virinoj, en Norvegujo
1 090, en Polujo 1 076, en Anglujo 1 060. Berlino havas kontra\u u
\^ciuj 1 000 viroj 1 080 virinojn, Dresdeno 1 113, Frankfurto sur
Majno 1 123. La ka\u uzo de tiu \^ci plinombreco estas la pli granda
mortado de knaboj en la a\^goj su\^cula kaj infana.

\emph{Internacia universitato por memlernuloj}. En la Amerika
\^stato Nov-Jorko, sur la bordo de la lago Chautauque, trovi\^gas la
universitato por memlernuloj, konata en Ameriko sub la nomo C. L. S.
C. (Chautauque Literary and Scientific Circle). Tiu \^ci
universitato estis fondita en la jaro 1867 de la Amerikano D-ro
Vincent. La unua penso de la organizatoro estis: doni la eblon al
pli maljunaj homoj plenigi la mankojn, devenantajn de nepreciza
lerneja edukado a\u u de tro frua \^cesi\^go de tiu \^ci edukado,
sed en tia maniero, ke la lernado ne alportu malhelpon al la
\^ciutagaj devoj kaj okupoj de la lernantaj homoj. La por tiu \^ci
celo fondita rondo estas \^gis nun la \^cefa akso de la societo C.
L. S. C. Organizata memlernado \^sajne estas ideo ne nova, sed nova
estas la sistemo de d-ro Vincent de popularigado de sciencoj inter
personoj pli maljunaj kaj lo\^gantaj malproksime de centro de
civilizacio. En Bostono ekzistas jam rilate de longe societo sub la
nomo "Study at Home Society", kiu por du dolaroj jare alsendadas
al la interesatoj instrukciojn kaj bonegajn konsilojn tu\^sante la
metodon de memedukado per legado; \^sajnus, ke tia societo respondas
tute al sia celo; sed "Study at Home Society" havis en la jaro
1886 nur 2 000 membrojn, dum C. L. S. C. posedis ilin 60 000!


   D-ro Vincent mem estis devigita de diversaj malfeli\^caj cirkonstancoj
forlasi en la a\^go de 20 jaroj la kolegion kaj fari\^gi Metodista
predikisto; tial, sciante, per kia malfacila maniero oni akiras en
pli malfrua a\^go la necesajn scia\^{\j}ojn, li el la tuta koro
beda\u uris tiujn, kiuj iras sur tiu vojo; la penson helpi al ili li
portadis en si tra 25 jaroj, kaj nur en la jaro 1867, trovinte en la
persono de L. Miller efikan kunlaboranton, li sian penson
transformis en fakton. La pretan materialon la organizatoroj trovis
en la societo "Teachers Retreat", kiu \^gis nun ekzistas en
Chautauque. La membroj de la dirita societo, instruistoj de
diman\^caj lernejoj de Metodistoj, kunvenas en Chautauque en somero
por trisemajna kunestado, por priparoli siajn demandojn kaj trovi
novajn rimedojn, por ri\^cigi siajn sciojn. Uzinte tiun \^ci
materialon, d-ro Vincent kaj Miller kreis la universitaton en la
plej simpla formo, intencante perfektigadi \^gin la\u u alvenontaj
bezonoj kaj demandoj. La sukceso estis tre granda. La nova plano de
universitato estis publikigita en la jaro 1878, kaj tuj ali\^gis al
la societo 700 personoj. La kredo je luma estonteco estis granda kaj
la entuziasmo eksterordinara: kiam la membroj revenis el Chautauque
hejmen kaj rakontis al la amikoj kaj konatoj pri la nova entrepreno,
\^suti\^gis tuj petoj pri akcepto en la societon de plej diversaj
flankoj de la lando. Tiel en la fino de la jaro 1878 la societo
posedis jam 8 000 membrojn, en 1885 la nombro de la membroj estis 35
000, kaj kun \^ciu jaro la nombro de la membroj rapide kreskas. La
membroj estas dis\^{\j}etitaj en urboj, urbetoj kaj vila\^goj de la
Unuigitaj \^Statoj, Kanadujo, Meksikujo, Anglujo, Svisujo, Italujo,
Hindujo kaj aliaj landoj.

   Por fari\^gi membro de C. L. S. C. estas nur necese sciigi pri sia
deziro per letero la sekretarion de la societo, kun aldono de 50
centimoj en signoj de po\^sto; tiu \^ci malgranda sumo, alportita de
la membro unu fojon por jaro, servas jam por elspezoj de
korespondado, sendado de programoj, tabeloj da demandoj k. t. p. La
sekretario, ricevinte la leteron de kandidato, sendas al li presitan
tabelon da demandoj, petante efektive veran respondon. La demandoj
estas sekvantaj: 1) kia estas via nomo; 2) adreso; 3) \^cu vi estas
fra\u ulo, fra\u ulino a\u u edzigita; 4) via a\^go; 5) se vi estas
edzigita, kiom da vivantaj infanoj vi havas; 6) je kio vi vin
okupas; 7) kia estas via religio; 8) \^cu vi estas decidita oferi 4
jarojn, por trairi la kurson de lernado; 9) \^cu vi promesas oferadi
4 horojn \^ciusemajne por legado de rekomendotaj al vi libroj; 10)
\^cu vi povas aldoni al tiuj \^ci 4 horoj ankora\u u iom da tempo,
kaj kiom nome? Profesoroj, esplorinte la respondojn de la
interesato, sendas al li katalogon de libroj, kiujn li devas legi en
la da\u uro de unu jaro. La nombro kaj enhavo de la libroj estas
tia, ke oni povas ilin tralegi en unu jaro, uzinte por tio \^ci
\^cirka\u u 40 minutojn \^ciutage. Por la rekomendataj libroj ne
ekzistas ia gradigado, tiel ke studanto de la unua kaj kvara jaroj
legas ofte tiujn samajn verkojn. Ekzemple por la jaro 1884/5 la
universitato rekomendis la sekvantajn kelkajn verkojn: "Historio de
Grekujo" de Burns; "Arto de elokventeco" de Toundsend; "Historio
de la reformacio" de Heret; "Ciro kaj Aleksandro de Macedonujo"
de Abbot; "La karaktero de Jezo" de Beschnel; "\^Hemio" de prof.
Apilton; "Facilaj lecionoj de biologio de la bestoj" de Uyeyt;
"Esploroj pri la kulinara scienco kaj arto", "Priskribo de la
antikva greka vivo", "Mitologio greka" kaj diversajn aliajn
librojn kaj gazetajn artikolojn. \^Sajne tia sistemo ne havas
sencon, \^car al persono, kiu komencas legi, ne sole estas malfacile
bone konati\^gi kun la legita objekto, sed li devas ankora\u u
anta\u u \^cio lerni legi la pli gravajn verkojn. Sed tiun \^ci
sistemon oni klarigas per tio, ke ekster la komuna kurso la plimulto
da membroj, pasinte la unuan a\u u duan jaron, ordinare jam ricevas
deziron studi ian specialan bran\^con de scienco. Tre multaj membroj
studas samtempe la specialan kurson de seminario por instruistoj. Al
\^ciu membro la sekretario de la universitato sendas \^ciujare krom
la katalogo de komuna kurso anka\u u la katalogojn de specialaj
verkoj, senpagan gazeton "Alma mater" kaj multon da diversaj
gravaj kaj interesaj katalogoj, bro\^suroj kaj dokumentoj. La
libroj, kiuj konsistigas la instruan materialon, kostas por jaro
\^cirka\u u 7 dolarojn (en librejoj 15 dolarojn). La malkarecon de
la libroj d-ro Vincent atingas per la sekvanta rimedo: li turnas sin
al libreja firmo kun la demando, \^cu \^gi ne donus rabaton, se \^gi
por tio \^ci ricevos la garantion, ke \^gi disvendos la donitajn
librojn en la nombro de 20-30 miloj da ekzempleroj; kompreneble la
firmo konsentas, \^car \^gia profito estas tiam tre granda; e\^c
ne-membroj, scii\^ginte, kiujn verkojn la universitato rekomendis,
a\^cetas ilin en granda nombro. Por la specialaj verkoj la membroj
devas anka\u u elspezi 6-8 dolarojn \^ciujare, sed tiujn \^ci ili
ricevas anka\u u por rabatita kosto. La plimulto da membroj,
lo\^gantaj profunde en la provinco, estas devigitaj a\^cetadi
\^ciujn verkojn kaj lerni mem, dum aliaj, lo\^gantaj en urboj,
pruntas la necesajn librojn el bibliotekoj, a\u u unu de alia, kaj
ofte ili kreas specialajn klubojn, kie ili kunvenas, por priparoladi
kaj klarigadi la legitan objekton. Membroj, kiuj distingas sin per
scienco a\u u specialeco, ekzemple in\^genieroj, instruistoj,
kuracistoj, pastroj k. c. donas senpage la respondajn klarigojn kaj
konsilojn al la membroj, kiuj lernas ilian specialecon.

   En la fino de lerna jaro \^ciu membro ricevas longan tabelon da
demandoj, kiujn li devas respondi, se li deziras posedi ateston de
C. L. S. C. La demandoj, malgra\u u la \^sajna komplikiteco, ne
prezentas tamen ian malfacilecon, se la studanto efektive atente
legis la donitan verkon. Bonaj 85 por 100 respondoj sufi\^cas, por
elteni la ekzamenon kaj ricevi ateston. Tia ekzameno ne estas
ekzameno en la propra senco; al la membroj e\^c estas permesite la
pli malfacilajn respondojn \^cerpi el libroj; sed ili devas esti
esprimitaj per propraj pensoj kaj vortoj de la studanto. La societo
opinias: nia celo estas posedi la eble plej grandan nombron da
membroj, sed ne doni ka\u uzon por malplii\^gado de ili; severaj do
ekzamenoj timigas nur tiujn membrojn, kiuj pleje nian helpon
bezonas; cetere regulaj ekzamenoj estas e\^c neeblaj pro tiu ka\u
uzo, ke multaj membroj trovi\^gas malproksime kaj alveni ne povas.
Tamen ekzistas en Chautauque tiel nomata "somera popola
universitato". Tiu \^ci nomo ne estas preciza, \^car la lerna jaro
limas sin per 10 monatoj kaj la somera kunveno sur la bordo de la
lago Chautauque servas sole kaj propre por spirita ripozo, ferioj,
plu por finado de la lerna jaro kaj por disdono de diplomoj al tiuj
membroj, kiuj trovas eblon alveni en Chautauque. Tiel la nombro de
\^ciuj finintaj la kurson ne prezentas sin tie en plena kompleto.
Ekzemple en la jaro 1885 finis la kurson 1 600 personoj, tamen
alvenis en Chautauque nur 600. Sed la nombro de la personoj flankaj,
kiuj en 1885 vizitis en somero Chautauque'on estis 75 000!

   La lago Chautauque, havanta longon de 20 anglaj mejloj, ku\^sas sur
la alteco de 1 400 futoj super la nivelo de la Atlanta Oceano, en
malproksimeco de 9 mejloj de la lago Erie, kaj estas \^cirka\u uita
de arbaroj, montoj, kun pura, fre\^sa aero. En la jaro 1878, kiam
d-ro Vincent fondis la someran universitaton, Chautauque okupis 150
akrojn da bela tero, sed trovis sin en tre praa stato. La alvenintaj
membroj devis lo\^gi en transportaj tendoj. Post 4 jaroj Chautauque
posedis jam hotelon por 500 personoj kaj multegon da domoj. En la
da\u uro de la supre diritaj du someraj monatoj en Chautauque estas
granda movado. Per vapor\^sipoj kaj fervojoj alvenas sen\^cese novaj
homoj el plej diversaj punktoj de la Unuigitaj \^Statoj kaj Kanado.
Krom la membroj alveturas multego da gastoj. La vizito en Chautauque
apartenas jam al la bona tono. Tiel en la jaro 1880 estis tie
Garfield, la universitaton vizitis anka\u u generalo Grandt, kaj
la\u u lia ekzemplo sekvis generalo Lohan en 1885, tiam ankora\u u
kandidato por vic-prezidanto.

   Por akiri prozelitojn, la societo faras, kion \^gi povas, ligante
utila\^{\j}on kun diversaj agrabla\^{\j}oj. En la da\u uro de la
tutaj 2 monatoj, kelkajn fojojn en tago, oni povas tie a\u udi
diversajn publikajn parolojn kaj lecionojn. \^Ciu povas havi
vo\^con; en la intertempo inter la paroloj oni produktas artajn
fajrojn, ilumina\^{\j}ojn. Artistoj ludas koncertojn, oni donas
spektaklojn, oni entreprenas procesiojn, promenadojn sur la lago per
en vespero lumigitaj \^sipetoj, k. t. p. Precipe la publikaj paroloj
havas altirantan forton. La societo por tio \^ci invitas parolantojn
kun nacia a\u u internacia gloro; la plejmulto de tiuj \^ci talentaj
publikaj paroloj havas por objekto E\u uropon, Azion kaj Afrikon. En
la jaro 1885 inter aliaj objektoj la sekvantaj personoj parolis pri
la sekvantaj objektoj: Syman Abbot parolis pri la homa naturo; d-ro
S. J. M. Eaton --- pri la planoj de konstruo de la piramidoj de
Egiptujo; A. M. Fairbaern, prezidanto de la Airedale College en
Anglujo --- pri la historio de la filozofio, pri John Locke, pri H.
Spencer, pri J. Bright, pri Comte kaj la pozitivistoj; d-ro Finch
--- pri la uzado de brando; Julius N. Seelye, prezidanto de la
Amherst College --- pri la potenco de la ideo, k. t. p. En la jaro
1886 grandan rolon tie \^ci ludis la oriento, pri kiu parolis: Babn
Keshub Chandra, Ram Chandra Bose kaj Sau Ah Brah.

   La specialajn sciencojn kaj artojn oni povas studi en la da\u uro de
la 4-jara komuna kurso a\u u poste. Tamen en la jaro 1884 la societo
organizis la unuan specialan fakultaton, konsistantan el profesoroj,
kiuj guvernas per korespondado; la profesoroj plenumas siajn devojn
en la loko de ilia lo\^gado, kaj unu fojon en jaro ili alveturas en
lokon difinitan por ekzamenoj; la studantoj do \^cerpas klarigojn
kaj sciencan helpon \^ce specialistoj, kiuj trovi\^gas tie, kie la
studantoj lo\^gas. La sekvanta eltiro el la cirkulero de la
profesoroj klarigos la celon de tiu \^ci universitato: "Oni kreas
universitaton, por doni la eblon atingi pli altan kaj praktikan
eduki\^gon al tiaj junaj homoj de amba\u u seksoj, kiuj ne povas
forlasi la domon, por vizitadi kolegion, kiuj estis devigitaj
forlasi universitaton, a\u u kiuj deziras plenigi siajn sciencajn
mankojn. La progresoj en la lernado kaj la grado de la akiritaj
scioj estos metitaj al severa esploro per skriba ekzameno \^ce
alesto de senpartiaj kaj konformaj atestantoj."

   En la jaro 1886 la universitato posedis jam 9 specialajn
fakultatojn: akademion de la greka kaj latina lingvoj, kolegion de
novaj lingvoj, fakultaton de matematiko, de historio, de literaturo,
kurson de mikroskopio, kurson teologian, kurson de retoriko kaj de
\^hemio. En la jaro 1887 oni malfermis: fakultaton de fiziko kaj
natur-historio, fakultaton de filozofio kaj logiko, instituton de
orientaj lingvoj, lernejon de industriaj sciencoj, kiu enhavas en si
telegrafiadon, konstruadon kaj diversajn fabrikadojn, lernejon de
komerco kaj de praktikaj aferoj (Business and practical affairs),
lernejon de agronomio, de belaj artoj, de aferoj ekleziaj, muzikan
kolegion, instituton de in\^genieroj de montolaboroj, pontoj kaj
vojoj, antropologian fakultaton por anatomio, flziologio, higieno,
psi\^hologio kaj sociologio.

   Persono, kiu deziras esti akceptita en la nombron de la studantoj,
enportas krom 5 dolaroj de enskribi\^go ankora\u u 10 dolarojn por
la universitato kaj 3 dolarojn kiel aldono, se li deziras studi kune
du specialajn sciencojn. Oni supozas, ke studanto povas lerni la
kurson de latina kaj greka lingvoj, uzante en la da\u uro de 4 jaroj
po unu horo \^ciutage, sed ankora\u u li povas unu horon oferi por
studadi samtempe la matematikon. \^Ciuj temoj kaj ekzercoj estas
korektataj de la profesoroj kaj resendataj al la studantoj.

   Mi finis. Imagu al vi, kara leganto, ke la dirita universitato
posedas tian ilon, kia estas la lingvo Esperanto\dots La komentarioj
ne estas necesaj.

\begin{flushright}
\footnotesize \fsc{Jan Janowski.}
\end{flushright}

\smallrule{}
