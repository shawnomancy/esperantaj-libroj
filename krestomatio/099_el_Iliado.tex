\begin{center}
\footnotesize (Tradukita de A. \fsc{Kofman})\\[2ex]

{\large KANTO UNUA}\\[2ex]
\footnotesize La pesto. --- La kolero.
\end{center}

\begin{verse}[0.9\textwidth]
          Kantu, diino, koleron de la Peleido A\^hilo,\\
          \^Gin, kiu al la A\^hajoj ka\u uzis mizerojn sennombrajn\\
          Kaj en Aidon de\^{\j}etis multegajn animojn kura\^gajn\\
          De herouloj kaj faris korpojn iliajn akiro\\
          Al rabobirdoj kaj hundoj --- fari\^gis la volo de Ze\u uso---\\
          De tiu tago, de kiu disigis sin ekdisputinte\\
          La ordonanto al viroj, Atrido, de l' dia A\^hilo.\\
           \vin  Kia do dio ekscitis en ili disputon malpacan?\\
          Filo Latona kaj Ze\u usa. Ekkolerigita de re\^go,\\
          Li malsanigis la militistaron, --- mortadis popoloj,\\
          \^Car malhonoris Atrido \^Hrizon, la pastron de l' dio.\\
          Por dea\^ceti filinon, li kun netaksebla depago\\
          Venis al \^sipoj A\^hajaj rapidena\^gantaj, en manoj,\\
          Sur ora sceptro, la florokronon de dio Apolo,\\
          La malproksimen\^{\j}etanta, kaj \^ciujn A\^hajojn petegis,\\
          Sed precipe la amba\u u Atridojn, la \^cefojn popolajn:\\
           \vin  "Ho vi, Atridoj, kaj \^ciuj A\^hajoj kura\^gaj! la dioj,\\
          Sur Olimpo lo\^gantaj, donu al vi ekdetrui\\
          Urbon Priaman kaj poste hejmen reveni feli\^ce.\\
          Al mi la karan filinon redonu vi pro dea\^ceto,\\
          Estimegante Apolon, la malproksimenpafantan."\\
           \vin  Jen tiam \^ciuj A\^hajoj la\u utege konsentis honoron\\
          Fari al pastro, de li dea\^ceton ri\^cegan ricevi.\\
          Tio ne pla\^cis nur sole al koro de Agamemnono;\\
          Li malhonore forpelis lin kaj al li diris minace:\\
          \vin"Gardu vin, ke, maljunulo, mi vin plu ne vidu \^ce \^sipoj\\
          Lar\^gaj, nun pro malrapido a\u u poste pro la reveno:\\
          \^Car malcerte vin helpos la sceptro kaj krono de l' dio.\\
          \^Sin do mi ne refordonos anta\u ue ol \^si maljuni\^gos\\
          Hejme, en Argo, \^ce mi, malproksime de \^sia patrujo,\\
          Laboradante teksa\^{\j}on, kaj mia kunku\^santino.\\
          Sed vi foriru kaj min ne ekscitu, ke sana vi restu."\\
          \vin   Tiel li diris, kaj \^Hrizo ektimis, la vorton obeis\\
          Kaj al la multebruanta maro silente foriris.\\
          Poste, malproksimi\^gante, li ekpetegis fervore\\
          Nun Apolon, la filon de la belabukla Latono:\\
           \vin  "A\u udu, Ar\^gentopafarka, apogo de \^Hrizo kaj Kilo\\
          Sankta, potence re\^ganta la Tenedoson, vi, glora!\\
          Se efektive mi vian sanktejon beligis per kronoj\\
          A\u u mi iam por vi de plej bonaj kaprinoj kaj bovoj\\
          Grasajn femurojn bruligis, vi mian deziron plenumu:\\
          Punu vi per viaj sagoj pro miaj larmoj A\^hajojn."\\
           \vin  Tiel pre\^gante li diris, atentis lin Febo Apolo.\\
          Jen li deiris, koleron en koro. Olimpan alta\^{\j}on,\\
          \^Cirka\u ufermitan sagujon kaj arkopafon sur \^sultroj.\\
          La\u ute sonadis la sagoj sur \^sultroj de ekkolerinto,\\
          Dum li kuradis. Li estis simila al nokto. Li poste\\
          Malproksime de \^sipoj sidi\^gis, kaj sagon ellasis,\\
          Kaj el pafarko ar\^genta jen sono terura eksonis.\\
          Li ekatakis anta\u ue nur mulojn kaj hundojn rapidajn,\\
          Sed jen la sagoj maldol\^caj atingis nun anka\u u la homojn,\\
          Kaj malvivuloj senhalte en lignaroj flamadis.\\
          Militistaron la sagoj flugadis na\u u tagojn, la dekan\\
          Al kolekti\^go A\^hilo la popolon kunvokis.\\
          Tion \^ci metis en liajn pensojn la Hero blankmana.\\
          \^Car pri Danaoj mal\^gojis \^si, ilin perei vidante.\\
          Kiam ili kunvenis kaj kunkolekti\^gis, sin levis\\
          Inter ili A\^hilo rapidapieda kaj diris:\\
           \vin  "Mi, ho, Atrido, nun pensas, revenos ni hejmen, irante\\
          Ree \^ci tien kaj tien, se nur ni evitos la morton,\\
          \^Car la milito kaj pesto A\^hajojn mortigas samtempe.\\
          Sed demandu ni pastron a\u u anta\u udiriston a\u u anka\u u\\
          Son\^gklarigiston, \^car anka\u u son\^goj elvenas de Ze\u uso,\\
          Ke al ni diru li, kial koleras nin Febo Apolo,\\
          \^Cu, eble, pro nefarita promeso a\u u pro hekatombo,\\
          \^Cu, el \^safidoj kaj el sendifektaj kaprinoj prenonte\\
          La odoron oferan, li nin liberigos de pesto."\\
           \vin  Tiel dirinte, sidi\^gis li. Kaj inter ili sin levis\\
          La Testorido Kal\^haso plej lerta el birddivenistoj,\\
          Kiu konadis estanton, estonton kaj anka\u u pasinton\\
          Kaj alkondukis la \^sipojn Danaajn al Trojo per sia\\
          Scio profeta, de Febo Apolo al li donacita.\\
          Li, bonpensanta, sin turnis al ili kaj tiel parolis:\\
           \vin  "Ho, Ze\u usamato A\^hilo, al mi vi ordonas klarigi\\
          La koleregon de malproksimegen\^{\j}etanta Apolo.\\
          Bone, mi estas dironta, sed vi al mi \^{\j}uru defendi\\
          Min efektive favore per viaj paroloj kaj manoj:\\
          Certe, mi ekkolerigos viron, re\^gantan potence\\
          \^Ciujn Arganojn, viron, al kiu obeas A\^hajoj:\\
          Re\^go ja estas tro forta kontra\u ue la vir' subpotenca;\\
          Se li e\^c tiun \^ci tagon sian malicon subpremos,\\
          Tamen en koro konservos li sian malicon, \^gis li \^gin\\
          Forellasos. Nun ju\^gu vi, \^cu vi min povas defendi."\\
          \vin   Lin respondante eldiris A\^hilo piedorapida:\\
          "Vi kura\^gi\^gu, sincere eldiru la dialuda\^{\j}on;\\
          \^Juras mi per Ze\u usamato Apolo, al kiu vi pre\^gas,\\
          Malfermante la volon de Dio al la Danaoj,\\
          Ke, dum mi estos vivanta kaj lumon de l' suno vidanta,\\
          Vin tie \^ci ektu\^sos per mano pezega neniu\\
          El A\^hajoj, se vi e\^c nomus mem Agamemnonon,\\
          Kiu glorigas sin esti plej forta el \^ciuj A\^hajoj."\\
          \vin   Kura\^gi\^ginte, la anta\u udiristo la nemala\u udebla\\
          Diris: "Koleras li nek pro promeso, nek pro hekatombo,\\
          Nur pro la pastro, la malhonorita de Agamemnono,\\
          Ne forlasinta filinon kaj ne alpreninta depagon.\\
          Jen kial nin mizerigas li, la malproksimenpafanto,\\
          Kaj mizerigos ankora\u u, ne forprenante multpezan\\
          Manon de l' pesto, \^gis estos la \^gojookula filino\\
          Ree al patro kaj kune kun sankthekatombo sendita\\
          En \^Hrizourbon. Nur tiam nin eble la dio kompatos."\\
           \vin  Tiel dirinte, sidi\^gis li, kaj inter ili sin levis\\
          Agamemnono la vastere\^ganta, l' heroo Atrido,\\
          Tute mal\^goje; la koron malluman kolero plenigis\\
          Kaj la okuloj fari\^gis similaj al fajro brilanta.\\
          Kontra\u u Kal\^haso anta\u ue li nun furioze ekdiris:\\
          \vin   "Ho, malbona\^{\j}oportisto, al mi eldirinta neniam\\
          Favora\^{\j}on! Vi \^gojas nur anta\u udiri mizerojn.\\
          Vi anta\u udiris neniam bonvorton kaj nek \^gin plenumis,\\
          Vi anka\u u nun anta\u udiras al kolekti\^gintaj A\^hajoj,\\
          Ke l' malproksimenpafanta Apolo ilin nur punas,\\
          \^Car mi ne volis pro \^Hrizofilino depagon belegan\\
          Preni. Ho, jes, mi preferas \^sin havi en mia domego,\\
          Mi \^sin preferas e\^c pli ol Klitemnestron, kun kiu\\
          Mi edzi\^gis dum \^sia virgeco. La \^Hrizofilino\\
          Al \^si ne cedas per kresko, talio, prudento, teksado.\\
          Sed mi konsentas redoni \^sin, se tio estas pli bona,\\
          \^Car mi la savon popolan, kaj ne la pereon deziras;\\
          Sed vi alie honore donacu min, ke mi ne restu\\
          Unu ne rekompencita, \^car tio ne estas konvena,\\
          \^Ciuj vi vidas, ke mia donaco foriras de mi nun."\\
          Al li kontra\u ue respondis A\^hilo piedorapida:\\
          "Ho, vi plej glora Atrido, kaj plej profitama el \^ciuj,\\
          Al vi pro kio A\^hajoj aldonos honoran donacon?\\
          Oni ne scias, \^cu restas ankora\u u ri\^ceco komuna,\\
          \^Car la akiron el urboj rabitaj ni jam disdividis,\\
          Estas ne bone \^gin rekolektigi por ree dividi.\\
          Sed vi ellasu \^sin pro la honoro do dioj; A\^hajoj,\\
          Ni, al vi pagos trioble, kvaroble, se Ze\u uso detrui\\
          Iam permesos al ni la Trojon murofortegan."\\
           \vin  Al li kontra\u ue respondis Agamemnono, la \^cefo:\\
          "Kia ajn estas kura\^ga vi, diosimila A\^hilo,\\
          Trompi ne penu min: vi ne admonos min nek superruzos.\\
          \^Cu vi, mem konservante donacon, vi volas, ke nur mi\\
          Estu rabita de mia, ke mi \^gin ellasu senpage?\\
          Bone, nur se anstata\u uos A\^hajoj \^gin per alia\\
          Rekompenco de kosto egala, la\u u mia deziro.\\
          Sed se ili rifuzos, mi prenos \^gin mem: la donacon\\
          Vian a\u u la donacon Ajaksan a\u u l' Odisean;\\
          Prenos min \^gin, --- senigita de \^gi ekkoleros la viro.\\
          Sed iafoje pri tio ni reparolos ankora\u u.\\
          Nun ni surmetu sur maro sankta la \^sipon nigretan,\\
          Ni kolektigu en \^gi da remistoj amason sufi\^can,\\
          Metu en \^gi l' hekatombon kaj \^Hrizofilinon ru\^gvangan.\\
          Unu el viroj konataj estos la \^cefo: Ajakso,\\
          Eble Idomeneo, a\u u Odiseo la dia,\\
          A\u u vi mem Peleido, vi, la plej terura el viroj ---\\
          Por oferante pacigi la Malproksimegen\^{\j}etantan."\\
           \vin  Al li respondis minace A\^hilo piedorapida:\\
          "Ve al mi! Senigita de honto, vi, profitamulo,\\
          Kiu A\^hajo de nun viajn vortojn obei ekvolos,\\
          Kun vi irante a\u u kontra\u u viroj batalon farante?\\
          \^Car mi alvenis ne pro ponardegojn\^{\j}etistoj Trojanoj\\
          Kontra\u ubatali, \^car ili kulpi\^gis je mi en nenio.\\
          Ili neniam \^cevalojn a\u u bovojn de mi ekdeprenis\\
          A\u u fruktportantajn kampegojn, en Ftio la multelo\^gita\\
          Fruktojn ruinis, \^car inter ni estas ja multe da ombraj\\
          Montoj kaj anka\u u la maro bruanta. Nur vin, plej senhontan,\\
          Sekvis ni, por ke vi \^goju, portante ekven\^gon al Trojo\\
          Pro Menelao kaj anka\u u pro vi mem, vi hundosimila.\\
          Sed tion \^ci ne atentas, nek zorgas vi tion \^ci tute;\\
          Min e\^c minacas vi preni donacon, de mi akiritan\\
          Per mia peno, kaj al mi de la A\^hajidoj donitan.\\
          Tamen do mi ja neniam ricevas donacon egalan\\
          Al la via post rabo de urbo per la A\^hajoj.\\
          Plej malfacilan laboron en laciganta batalo\\
          Miaj manoj faradas, sed kiam venas divido,\\
          La rekompenco plej granda venas al vi; mi, kontra\u ue,\\
          De malmulto kontenta, revenas laci\^ge al \^sipoj;\\
          Nun mi foriras en Ftion, \^car estas certe pli bone\\
          Hejmen reveni en \^sipoj fleksitaj. Mi ne deziras\\
          Malhonorita de vi plu kolekti por vi nun ri\^cecojn".\\
           \vin  Al li respondis nun Agamemnono, la \^cefo de viroj:\\
          "Nu do, vi kuru do, se via koro al tio vin igas;\\
          Mi vin pro mi ja ne petas restadi, mi havos aliajn,\\
          Por akiri honoron, precipe la Ze\u uson-zorganton.\\
          El diaj re\^goj vi estas por mi la plej malamata.\\
          \^Car vi \^ciam preferas malpacojn, bati\^gojn, batalojn.\\
          Se pli da forto vi havas, donita \^gi estas de dio.\\
          Hejmen revenu vi kun viaj \^sipoj kaj viaj kolegoj,\\
          Re\^gu vi Mirmidonojn pace, pri vi mi ne pensas,\\
          Mi ne atentas vian koleron, kaj mem mi minacas:\\
          Kiel Febo Apolo deprenas de mi \^Hrizoidon,\\
          Kiun forsendas mi per mia \^sipo kaj miaj amikoj,\\
          Tiel mi mem vian tendon eniros kaj forelkondukos\\
          Vian donacon, ru\^gvangan la Brizeidon: vi sciu,\\
          Kiom mi estas pli alte ol vi, kaj ke \^ciu teruru\\
          Al mi sin egaligi kaj al mi kontra\u uparoli."\\
           \vin  Tiel parolis li, kaj Pelopido forte doloris,\\
          Inter du pensoj \^sancelis la koro en brusto, en vila:\\
          \^Cu, eltirinte de flanko glavon akregan, aliajn\\
          Militistojn forpeli kaj la Atridon mortigi,\\
          A\u u humiligi koleron kaj sian koron ekbridi.\\
          Dum en animo kaj koro li pensis pri tio kaj grandan\\
          Glavon el ingo eltiris, alvenis al li el \^cielo\\
          La diino Ateno, sendita de Hero blankmana,\\
          Kiu amis kaj zorgis amba\u u heroojn egale.\\
          \^Si poste li nun stari\^gis, lin kaptis per blondkapoharoj,\\
          Sole al li aperinte: \^sin \^ciu alia ne vidis.\\
          Ektimi\^ginte, sin turnis A\^hilo kaj tuj ekrekonis\\
          Palas-Atenon, terure \^siaj okuloj brilegis.\\
          \vin   Kaj li komencis paroli kaj diris la vortojn flugilajn:\\
          "Kial vi venis, filino de Ze\u uso \^sildontenanta,\\
          \^Cu por ekvidi la aroganta\^{\j}on de Agamemnono?\\
          Sed mi eldiras al vi, kaj mi pensas, ke tio fari\^gos,\\
          La fiereco senbrida balda\u u lin pereigos."\\
          Al li respondis Ateno, la bluokula diino:\\
          "Por kvietigi koleron vian mi venis, se restos\\
          Vi obeema; min sendis Hero, diino blankmana,\\
          Kiu vin amba\u u egale samtempe amas kaj zorgas.\\
          Vi do nun finu malpacon, kaj ne elprenu glavegon,\\
          Certe, per vortoj vi povas insulti lin la\u u via volo.\\
          Mi al vi diras, kaj tio \^ci estos efektivigita:\\
          Iam vi havos trioble da luksaj, belegaj donacoj\\
          Pro tiu \^ci fiereco. Detenu vin nun kaj obeu."\\
          \vin   \^Sin respondante, eldiris A\^hilo rapidapieda:\\
          "Vere, estas bezone vian ordonon obei,\\
          Malgra\u u la koro kolera: \^car tio \^ci estas pli bona:\\
          La obeantan al dioj dioj volonte atentas."\\
          \vin   Diris li, kaj sur ar\^gentan tenilon li metis la manon,\\
          Ree enmetis glavegon en ingon, ne malobeante\\
          Vortojn diinajn. \^Si balda\u u ekflugis Olimpon, lo\^gejon\\
          De dio Ze\u uso \^sildontenanta, al dioj aliaj.\\
          \vin   Sed Peleido returnis sin kun parolo kolera\\
          Al Atrido: li sian koleron ne ek\^cesigis:\\
          \vin "Ho, vi drinkanto kun hunda rigardo, kun cerva kura\^go!\\
          Vi en la koro neniam havas kura\^gon vin armi,\\
          Por kune kun la popolo batali a\u u kun la A\^hajoj\\
          Plej eminentaj embuski vin: tio la\u u vi estas --- morto.\\
          Jes! prefereble ja estas en vasta tendaro A\^haja\\
          Vian kontra\u udiranton senigi de lia donaco.\\
          Re\^go-popolman\^gegulo, neta\u ugulojn re\^ganta!\\
          \^Car, Atrido, alie vi lastan fojon ofendus.\\
          Sed al vi mi nun diras kaj faras \^{\j}uron grandegan,\\
          \^Juras mi je tiu sceptro, neniam florojn kaj bran\^cojn\\
          Renaskonta, estante lasinta en montoj la trunkon,\\
          Kaj ne reekfloronta (\^car fero de \^gi ekdemetis\\
          \^Giajn foliojn kaj \^selon), kaj portadata en manoj\\
          De A\^hajidoj-ju\^gistoj, gardantaj la le\^gojn, venantajn\\
          De l' dio Ze\u uso; mi \^{\j}uras je tiu \^ci \^{\j}uro grandega,\\
          Ke eksopiros tuj \^ciuj A\^hajoj, ser\^cante A\^hilon\dots\\
          Tamen, malgra\u ue doloro via, ne povos vi helpi\\
          Ilin, ekmortigotajn amase per man' de Hektoro\\
          La virbu\^canto, kaj vian vi koron mordetos, ke estis\\
          Vi plej kura\^gan el \^ciuj A\^hajoj malestiminta".\\
          \vin   Tiel dirinte, sur teron ek\^{\j}etis li sian sceptron,\\
          Beligitan de najloj oraj, kaj balda\u u sidi\^gis.\\
          Sed Atrido koleris anka\u u. Kaj levis sin poste\\
          La dol\^calingva kaj tondravo\^ca Nestoro el Pilo,\\
          El kies bu\^so fluadis dol\^camiela parolo;\\
          Estis li nun travivinta jam du generaciojn,\\
          Kiuj naski\^gis kaj vivis kune kun li en la sankta\\
          Urbo Pilo, hodia\u u li superre\^gis la trian.\\
          Jen li, bonepensanta, eldiris al la kolekti\^go:\\
          "Ve, ho kia mal\^gojo surfalas sur teron A\^hajan!\\
          Certe, ek\^gojos Priamo kaj kune kun li Priamidoj,\\
          Anka\u u l' aliaj Trojanoj eksentos \^carmegon en koro,\\
          Se ili a\u udos, ke vi nun malpacas, vi, kiuj estas\\
          En konsiloj plej altaj kaj en bataloj unuaj.\\
          Sed vi a\u uskultu, \^car amba\u u pli junaj vi estas ja ol mi.\\
          Iam mi havis aferon kun viroj ankora\u u pli fortaj\\
          Ol vi, sed miajn konsilojn ili neniam mal\^satis.\\
          Vidis neniam mi kaj ne ekvidos jam virojn similajn,\\
          Kiel Pirito kaj Drio, kiu potencis popolojn,\\
          Kiel Kenco a\u u Eksadio a\u u Polifemo\\
          Diosimila kaj dia Tezeo, la filo Egea.\\
          Ili estis pli fortaj ol \^ciuj, sur tero naskitaj,\\
          Kaj, plej fortaj estinte, batalis nur kontra\u u plej fortaj,\\
          Kontra\u u montaj monstregoj, ilin fordetruante.\\
          Kune kun ili mi estis milita kolego el Pilo,\\
          El malproksima lando: alvokis min tiuj \^ci viroj:\\
          Kune kun ili batalis mi, kaj kontra\u u ili nenia\\
          Homo, sur tero naskita, havis kura\^gon batali.\\
          Sed anka\u u ili konsilon mian obee atentis.\\
          Nun vi anka\u u obeu: obei ja estas pli bone.\\
          Vi, Atrido potenca, ne prenu de li la knabinon,\\
          Sed \^ce li lasu donacon, donitan de la A\^hajidoj,\\
          Sed anka\u u vi, Peleido, \^cesu kun re\^go disputi\\
          Malpacante: konvenas ja honoro pli alta\\
          Al la sceptroportanta, kiun Ze\u uso glorigas;\\
          Kaj se vi estas pli forta kaj filo de dia patrino,\\
          Estas li ja pli potenca, \^car re\^gas li multajn popolojn.\\
          Vi, Atrid', humiligu la koron, mi mem vin petegas,\\
          Kontra\u u A\^hilo demetu vian koleron: li estas\\
          Forta apogo A\^haja en pereigema milito."\\
           \vin  Diris al li respondante Agamemnono potenca:\\
          "Vere! Vi, ho maljunulo, \^cion eldiris konvene,\\
          Sed tiu viro deziras esti pli alta ol \^ciuj,\\
          Volas estri\^gi je \^ciuj, volas ordoni al \^ciuj\\
          Kaj doni le\^gojn, al kiuj, certe, neniu obeos.\\
          Se lin la dioj eternaj instruis \^{\j}etadi ponardon,\\
          \^Cu ili anka\u u pro tio permesis al li --- nin insulti?"\\
           \vin  Lin interrompis kaj diris diosimila A\^hilo;\\
          "Ho, mi, certege, nur estus timulo kaj neta\u ugulo,\\
          Se mi vin kontentigadus je \^cio, kion vi volas.\\
          Vi de aliaj postulu tion, tamen al mi vi\\
          Ne ordonu, \^car mi ne intencas al vi humili\^gi.\\
          Tion mi diras al vi, kaj \^gin vi konservu en koro:\\
          Mi la manon armitan ne levos pro la knabino\\
          Kontra\u u vi a\u u aliaj: vi donis kaj vi \^sin deprenas.\\
          Sed el mia cetera havo en \^sipoj rapidaj\\
          Vi nenion deprenos kontra\u u mia bonvolo.\\
          Sed se vi volas, elprovu, ke \^ciuj kune ekvidu,\\
          Kiel el vi nigra sango \^cirka\u u ponardo ekfluos."\\
           \vin  Tiel ili disputis per malamikaj paroloj\\
          Kaj, levi\^ginte, dislasis \^ce \^sipoj la kolektigitajn.\\
          Peleido foriris al tendoj kaj \^sipoj rapidaj\\
          Kune kun Menetiido kaj kune kun siaj kolegoj,\\
          Sed Atrido surmetis sur maron \^sipon rapidan,\\
          Dudek remistojn elektis kaj metis en \^gin hekatombon\\
          Por la dio kaj anka\u u la \^Hrizofilinon ru\^gvangan,\\
          Kaj Odiseo multsa\^ga tie estis \^sipestro.\\
          \^Ciuj sidi\^gis kaj balda\u u ekna\^gis sur vojon malsekan.\\
          \vin   Kaj al popoloj ordonis purigi sin Agamemnono.\\
          Ili purigis sin kaj la malpuron en maron en\^{\j}etis,\\
          Kaj sendifektajn hekatombojn oferis al dio:\\
          Bovojn kaj kaprinojn \^ce bordo senfrukta de l' maro,\\
          Kaj l' odoro ofera en fumo suriris \^cielon.\\
          \vin   Tiel ili penadis, sed dume Agamemnono\\
          Tion ne lasis, per kio li \^{\j}us A\^hilon minacis:\\
          Vokis rapide li Taltibion kaj E\u uribaton,\\
          Kiuj estis liaj heroldoj kaj lertaj servantoj.\\
           \vin  "Iru vi tien, en tendon de la Peleido A\^hilo,\\
          Prenu vi la Brizeidon ru\^gvangan kaj \^sin alkonduku,\\
          Se do li \^sin ne redonos, mi mem \^sin jam tiam deprenos,\\
          Al li venonte kun aro, kaj tio lin plu ektimigos."\\
          \vin   Tiel dirinte, forsendis li ilin kun vorto minaca.\\
          Ili al bordo de l' maro senfrukta mal\^goje foriris\\
          Kaj al la tendoj kaj \^sipoj de Mirmidonoj alvenis.\\
          Lin do ili ektrovis \^ce tendo, \^ce \^sipo nigreta\\
          Tie sidanta, kaj ilin vidinte ne \^gojis A\^hilo.\\
          Ekkonfuzitaj pro timo kaj respektego pri l' re\^go,\\
          Ili stari\^gis sen ia parolo, sen ia demando,\\
          Sed li mem tion penetris per sia koro kaj diris:\\
          \vin   "Bonan venon, heroldoj, senditoj de Ze\u uso kaj homoj!\\
          Alproksimi\^gu, \^car kulpaj estas ne vi, sed Atrido,\\
          Kiu venigis vin pro la filino ru\^gvanga de \^Hrizo.\\
          Dia Patroklo, knabinon vi elkonduku, ke ili\\
          \^Sin ekdeprenu. Kaj estu ili mem la atestoj\\
          Anta\u u la dioj feli\^caj, anta\u u homoj mortemaj,\\
          Anta\u u la re\^g' furioza, ke kiam oni ankora\u u\\
          Mian helpon bezonos por malhonoron formeti\\
          De aliaj! Ho, certe, pro pereema malsa\^go\\
          Li, Atrido, ne povas rigardi anta\u uen, nek posten,\\
          Por fortikigi A\^hajojn, batalantajn \^ce \^sipoj."\\
          \vin   Tiel li diris. Patroklo obeis la karan amikon:\\
          Li elkondukis el tendo kaj donis knabinon ru\^gvangan\\
          Al la senditoj, kaj tiuj foriris al \^sipoj A\^hajaj;\\
          Kaj ne volonte kuniris knabino. Dume A\^hilo\\
          Iris plorante for de amikoj kaj tie sidi\^gis,\\
          Sur la bordo de l' maro; rigardis li maron malluman,\\
          Manojn eltiris kaj sian karan patrinon petegis:\\
          \vin   "Mia patrino, \^car vi min por tempo mallonga eknaskis,\\
          Devus almena\u u alju\^gi honoron al mi l' Olimpano\\
          Ze\u uso altetondranta, sed li tute min ne honoras!\\
          Jen Atrido, la vastere\^ganta Agamemnono,\\
          Min malhonoris estinte de mi forrabinta donacon."\\
          \vin   Tiel li diris plorante; atentis lin lia patrino,\\
          La estiminda, en profundega\^{\j}oj de maro, \^ce l' patro\\
          Maljuni\^ginta, kaj venis el maro \^si kiel nebulo,\\
          Kaj apud li, elver\^santa larmojn, \^si alsidi\^gis.\\
          Lin karesante per mano, \^si la parolon eldiris:\\
          \vin   "Filo, pro kio vi ploras? pro kio la koro mal\^gojas?\\
          Ho, vi ka\^su nenion, ke amba\u u tion ni sciu."\\
          \vin   Diris, profunde \^geminte, A\^hilo rapidapieda:\\
          "\^Cion vi scias, patrino, pro kio rakonti konaton?\\
          Tebon aliris ni, sanktan urbon de Eetiono,\\
          Kaj \^gin detruis kaj \^cion tien \^ci forelkondukis;\\
          La A\^hajidoj honeste l' akiron dividis kaj oni\\
          \^Hrizofilinon ru\^gvangan elektis por la Atrido.\\
          \^Hrizo do, pastro de malproksimegen\^{\j}etanta Apolo,\\
          Venis al \^sipoj rapidaj de kuproarmitaj A\^hajoj,\\
          Por dea\^ceti filinon per netaksebla depago,\\
          La florkronon portante de malproksimegen\^{\j}etanta\\
          Dio sur ora sceptro, kaj la A\^hajojn petegis,\\
          Sed l' Atridojn, la amba\u u \^cefojn popolajn --- precipe.\\
          Tiam \^ciuj A\^hajoj la\u utege konsentis honoron\\
          Fari al pastro, de li dea\^ceton ri\^cegan ricevi.\\
          Tio ne pla\^cis nur sole al koro de Agamemnono,\\
          Li malhonore forpelis kaj lin minacis per vortoj.\\
          La maljunulo foriris kolere, sed Febo Apolo\\
          Lian petegon atentis, \^car amis li multe la pastron.\\
          Sendis li al Argoanoj sagojn malbonajn amase,\\
          Tiam mortadis popoloj, \^car sagoj de l' dio flugadis\\
          \^Cie en vasta tendaro A\^haja. Sciigis nin tiam\\
          Anta\u udiristo scianta pri volo de dio Apolo;\\
          Mi mem unua konsilis rapide pacigi la dion.\\
          Sed Atrido koleris: subite levinte sin, diris\\
          Li la parolon minacan, hodia\u u plenumi\^gintan.\\
          La \^gojokulaj A\^hajoj \^sin, vere, venigas per \^sipo\\
          Hejmen en \^Hrizon kaj anka\u u donacojn por la Potenculo,\\
          Sed \^{\j}us deprenis l'heroldoj de mi el mia la tendo\\
          La Brizeidon, donacon honoran de A\^hajidoj.\\
          Helpu do vi vian filon noblan, se povas vi helpi.\\
          Vi sur Olimpon suriru kaj Ze\u uson petegu, se iam\\
          Vi lian koron \^gojigis per paroloj a\u u agoj,\\
          \^Car mi vin ofte a\u udadis glori\^gi en domo de l' patro,\\
          Ke l' \^cirka\u uitan de nigraj nubegoj Kronidon vi sola\\
          El senmortema diaro de malhonora pereo\\
          Igis for, dume kateni minacis lin \^ciuj aliaj:\\
          Hero kaj Posejdono kaj anka\u u Palas-Ateno.\\
          Sed vi venis, diino, formetis de li la katenojn,\\
          Sur Olimpon la vastan vokinte Centmanon, nomatan\\
          Breareo de dioj kaj --- Egeono de homoj,\\
          \^Car li estas multege pli forta ol lia patro;\\
          Apud Kronido sidi\^gis li fiera de gloro,\\
          Kaj lin teruris la dioj kaj timis Ze\u uson kateni.\\
          Rememorigu lin, anta\u u li sidi\^gu kaj liajn\\
          Vi \^cirka\u uprenu genuojn, ke volu li helpi Trojanojn\\
          A\u u \^gis tendaro kaj \^sipoj la A\^hajidojn forpeli\\
          Mortigatojn, ke \^ciuj scii\^gu la krimon de l're\^go,\\
          Ke li mem, Agamemnono Atrido, la \^cefo de viroj\\
          Sciu l' ofendon, faritan al plej kura\^ga A\^hajo."\\
          \vin   Larmojn ver\^sante, respondis al sia filo Tetido:\\
          "Ve al mi, ke edukis mi vin, por mizeroj naskitan!\\
          Ho, se vi povus \^ce \^sipoj sen larmoj kaj sen mal\^gojo\\
          Sidi, \^car estos nelonge vi, tute ne longe vivanta.\\
          Nun ne longtempa kaj pli malfeli\^ca ol \^ciuj vi estas\\
          Per unu fojo. Ho, mi vin por malfeli\^co eknaskis.\\
          Tion al tondron\^{\j}etanta Ze\u uso, se volos li a\u udi,\\
          Iras mi diri nun supren, sur multane\^gan Olimpon.\\
          Vi do, sidante \^ce \^sipoj rapidaj, ne \^cesu koleri\\
          La A\^hajidojn kaj tute detenu vin de la milito.\\
          Ze\u uso hiera\u u festeni al Etiopoj senpekaj,\\
          \^Ce l'Okeano, foriris kaj kune kun li --- la diaro,\\
          Tamen la tagon dekduan revenos li hejmen, Olimpon.\\
          Tiam eniros mi kupran domon de Ze\u uso, mi liajn\\
          Ek\^cirka\u uprenos genuojn, mi lin inklinigi esperas."\\
          \vin   Tiel dirinte, foriris \^si. Kaj li restigis koleron\\
          En la koro pro sia ru\^govanga knabino,\\
          Malgra\u u li forkondukita perforte. Sed Odiseo\\
          En \^Hrizourbon alvenis kun hekatombo la sankta.\\
          En la havenon profundan veninte, tuj ili formetis\\
          Velojn kaj ilin kunligis en \^sipoj kaj al la mastejo\\
          Maston altiris, rapide \^gin sur \^snuregojn mallevis\\
          Kaj en havenon enpelis per remiloj la \^sipon,\\
          Tie \^ci ankron el\^{\j}etis kaj ligis al bordo \^snuregojn.\\
          Mem ili teron suriris kaj l'hekatombon kunprenis\\
          Por dio Febo Apolo malproksimegen\^{\j}etanta,\\
          Anka\u u eliris \^Hrizido el \^sipo la martrana\^ganta,\\
          \^Sin al altaro kondukis nun Odiseo multsa\^ga,\\
          Kaj redoninte en brakojn de l'patro, li tiel eldiris:\\
          \vin   "\^Hrizo, venigis min Agamemnono, la \^cefo de viroj,\\
          Por la filinon redoni kaj hekatombon oferi\\
          Al Apolo, ke por A\^hajidoj paci\^gu la dio,\\
          Kiu pezegajn mizerojn alsendis nun al Argoanoj."\\
          \vin   Tion li diris, \^sin donis en liajn manojn; la patro\\
          \^Goje ricevis la karan filinon. L'A\^hajoj al dio\\
          La hekatombon majestan ordigis \^cirka\u ue l'altaro\\
          Bonkonstruita, eklavis la manojn kaj prenis l'hordeon.\\
          \^Hrizo do levis la manojn supren kaj la\u ute ekpre\^gis:\\
          \vin   "A\u udu, Ar\^gentopafarka, apogo de \^Hrizo kaj Kilo\\
          Sankta, potence re\^ganta la Tenedoson. Kiele\\
          Vi jam anta\u ue atentis mian petegon vokantan\\
          Kaj min trankviligis, mizeriginte l'A\^hajojn,\\
          Tiel same vi ree mian deziron plenumu\\
          Kaj de A\^hajoj forigu la teruregajn mizerojn."\\
           \vin  Tiel li diris, pre\^gante, atentis lin Febo Apolo.\\
          Sed post la pre\^go dis\^{\j}etis ili la sanktan hordeon,\\
          Kolojn defleksis al bestoj, ekbu\^cis kaj felojn deprenis\\
          Kaj femurojn eltran\^cis, ilin envolvis per graso\\
          Duobligite, por pecoj da kruda viando kovrinte;\\
          Ilin bruligis la pastro sur \^stipoj kaj nigran vinon\\
          Ver\^sis sur ili; lin knaboj kun kvindentiloj \^cirka\u uis.\\
          Forbruliginte femurojn kaj l' interna\^{\j}ojn provinte,\\
          Ili distran\^cis la reston pece kaj metis sur stangojn,\\
          Rostis \^gin pene kaj fine demetis \^gin. Post la laboro\\
          Ili aran\^gis festenon, kaj tiam la koroj iliaj\\
          En tiu bonaran\^gita festeno mankis nenion.\\
          Post kvietigo de la malsato kaj de soifo\\
          Knaboj plenigis \^gis randoj kalikojn per la trinka\^{\j}o\\
          Kaj, komencinte de dekstre, al \^ciuj pokalojn disdonis.\\
          Da\u ure la tago l'A\^hajoj per kantoj pacigis la dion,\\
          Belan peanon kantadis la maljunuloj A\^hajoj,\\
          Glorigadante Apolon, kaj \^gojis li en sia koro.\\
           \vin  Kiam la suno subiris kaj mallumo fari\^gis,\\
          Ili \^ce la \^snuregoj, tenintaj la \^sipon, ekdormis.\\
          Sed kiam la rozafingra Eos', el nebuloj naskita,\\
          Levis sin, ili forna\^gis al vasta tendaro A\^haja,\\
          Venton favoran al ili donis Febo Apolo.\\
          Ili starigis la maston, disstre\^cis la velojn blankegajn,\\
          Vento ekblovis en mezon de veloj, kaj ondo purpura\\
          \^Cirka\u u la kilo de l' \^sipo na\^ganta fortego ekbruis,\\
          Kaj \^gi ekkuris sur ondoj, rapide la vojon farante.\\
          Poste, veninte al vasta tendaro de la A\^hajidoj,\\
          Ili eltiris la \^sipon nigretan sur bordon la sablan\\
          Kaj, \^gin subapoginte alte per traboj longegaj,\\
          Ili ne malrapidante disiris en tendojn kaj \^sipojn.\\
          Li do koleris, sidante \^ce \^sipoj rapidena\^gantaj,\\
          Li, Peleido kura\^ga, la dia piedorapida.\\
          \vin   Li en konsilojn, virojn glorigadantajn, ne venis,\\
          Nek en batalojn. Sed lian koron mordetis mal\^gojo\\
          Pro la resti\^go senaga: soifis li bruon, batalon.\\
          \vin   Jen apena\u u dekdua maten\^cielru\^go aperis,\\
          Sur Olimpon revenis l'eternevivanta diaro.\\
          Ze\u us anta\u uiris. Tetido do ne forgesis la peton\\
          De \^sia filo, \^si iris el maro kaj levis sin kune\\
          Kun frumatena nebulo \^cielon kaj altan Olimpon;\\
          Tie \^ci trovis \^si Ze\u uson, aparte de dioj aliaj\\
          Sur la supra\^{\j}o plej alta de multekapa Olimpo.\\
          \^Si alsidi\^gis kaj \^cirka\u uprenis per mano maldekstra\\
          Liajn genuojn, per dekstra lian mentonon ektu\^sis,\\
          Kaj potenculon Kronidon-Ze\u uson \^si ekpetegis:\\
          \vin   "Ho, patro Ze\u uso, se mi al vi inter diaro ekpla\^cis\\
          Iam per vorto a\u u faro, plenumu vi mian deziron:\\
          Vi mian filon ekven\^gu, mortonton pli frue ol \^ciuj;\\
          Agamemnono, la \^cefo de viroj, lin malhonoris,\\
          De li preninte donacon kaj mem \^gin forekrabinte.\\
          Sed vi lin ven\^gu, Anta\u upripensanta, ho, Ze\u uso potenca!\\
          Vi al Trojanoj donacu venkadon, \^gis A\^hajidoj\\
          Mian filon honoros kaj rekompencos lin glore."\\
          Tiel \^si diris. Al \^si ne respondis la nubopelanto\\
          Ze\u uso kaj longe silentis. Tetido do forte alpremis\\
          Sin al \^cirka\u uprenitaj genuoj kaj reekpetegis:\\
          "Ne \^sanceli\^gu: promesu kaj donu al mi la konsenton\\
          A\u u do --- rifuzu (ne timas ja vi), ke tuj nun mi sciu,\\
          \^Cu mi el \^ciuj estas plej malestimata diino."\\
          \vin   Al \^si respondis indigne Ze\u uso la nubkolektulo:\\
          "Vere, malbona afero estos, se vi kontra\u u Hero\\
          Igos min, kaj \^si per vortoj insultaj min ekkolerigos;\\
          \^Si jam sen tio en rondo de \^ciuj dioj senmortaj\\
          Kontra\u u mi \^ciam disputas, dirante --- mi helpas Trojanojn.\\
          Tamen vi nun forrapidu, ke Hero vin ne ekvidu.\\
          Mi jam prizorgos, por vian deziron plenumi, kaj al vi\\
          Mi la kapon balancos, por ke al mi vi konfidu:\\
          Tio \^ci estas de miaj promesoj por \^ciuj senmortaj\\
          Garantia\^{\j}o sanktega, \^car estas ne redonebla,\\
          Ne refarebla la vorto, donita dum kapobalanco."\\
          \^Ce tiuj vortoj mallevis Ze\u uso la brovojn mallumajn,\\
          Kaj l' ambroziaj buklharoj de l' Potenculo subfalis\\
          De la kapo senmorta, la vasta Olimpo ektremis.\\
          \vin   Tiel konsili\^ginte, ili disiris nun. \^Si do\\
          Maron profundan rapidis de la brilega Olimpo.\\
          Ze\u uso eniris palacon sian, kaj eklevi\^ginte\\
          Iris la dioj renkonte al patro ilia, neniu\\
          Kura\^gi\^gis atendi lin: \^ciuj renkonte aliris.\\
          \vin   La Olimpano sidi\^gis sur trono sia. Sed Hero\\
          Ne preterlasis, ke li konsili\^gis kun la filino\\
          De maljunulo Nereo, l'ar\^gentapieda Tetido,\\
          Kaj tiajn vortojn pikemajn al Ze\u uso potenca eldiris:\\
          \vin   "Kiu el dioj, vi ruza, al vi sian donis konsilon?\\
          \^Ciam pla\^cas al vi ricevadi sekretajn decidojn,\\
          Dum mi malestas de vi, kaj neniam al mi vi diras\\
          Vole e\^c unu vorton de tio, pri kio vi pensas."\\
          \vin   Al \^si respondis la patro de la homaro kaj dioj:\\
          "Hero, ne penu vi \^ciujn miajn pensojn scii\^gi,\\
          \^Car tio estas ne ebla por vi e\^c, por mia edzino.\\
          Certe, neniu el dioj kaj homoj scii\^gos pri tio,\\
          Anta\u u ol vi, kio estas por vi permesata scii\^gi;\\
          Tamen do, kion mi volas aparte de dioj decidi,\\
          Vi min ne devas demandi, vi min ne devas esplori."\\
          \vin   Hero la bovookula ree respondis al Ze\u uso:\\
          "Ho, kiajn vortojn eldiris vi, Kronido terura!\\
          Mi ja neniam \^gis nun vin demandis a\u u esploradis,\\
          Tute trankvile decidas vi \^cion, kion vi volas.\\
          Sed mi nun timas en koro, \^cu vin ne ekkonvinkis\\
          L'ido de griza Nereo, l'ar\^gentapieda Tetido:\\
          \^Ce vi sidante, \^si viajn \^cirka\u uprenadis genuojn.\\
          Al \^si vi eble promesis per kapobalanco A\^hilon\\
          Ekglorigi kaj multajn A\^hajojn ekstermi \^ce l' \^sipoj?"\\
          \vin   Al \^si respondis dirinte Ze\u uso la nubkolektulo:\\
          "Stranga, vi \^ciam suspektas, ne povas de vi mi ka\^si\^gi,\\
          Sed vi nenion atingos, nur malproksimi\^gos de koro\\
          Mia, kaj tio por vi estos ja pli terura ankora\u u.\\
          Se tio estos, fari\^gos tio la\u u mia deziro,\\
          Sidu do vi nun trankvile kaj miajn ordonojn obeu!\\
          \^Car malcerte tuta diaro Olimpa vin helpos,\\
          Se kontra\u u vin nevenkeblajn miajn manojn mi uzos".\\
          \vin   Tiel li diris, kaj Hero la bovookula ektimis,\\
          Sidis silente, koleron en sian koron bridinte,\\
          Kaj eksopiris la Uranidoj en Ze\u usa palaco.\\
          Jen nun eldiris Hefesto, la glora artisto, favore\\
          Pri sia kara patrino Hero, diino blankmana:\\
          \vin   "Estos certege malbone kaj fine ne elsufereble,\\
          Se vi tiel malice malpacos por la mortemaj\\
          Kaj inter dioj vi semos ribelon. Ne \^guos ja pli ni\\
          \^Gojon festenan, se re\^gos malpaco. Al mia patrino\\
          Mi nun konsilas (kvankam \^si mem havas multan prudenton)\\
          Al kara patro obee submeti sin, por ke li ree\\
          Ne ekkoleru, kaj pli ne konfuzi nin dume festeno.\\
          \^Car se li, la Olimpano tondron\^{\j}etanta, ekvolos,\\
          Li nin de\^{\j}etos de tronoj; li estas pli forta ol \^ciuj.\\
          Sed vi lin volu per vortoj delikataj moligi\\
          Kaj refari\^gos favora balda\u u al ni l'Olimpano".\\
          \vin   Tiel dirinte, levi\^gis li kaj pokalon dufundan\\
          Donis al sia patrino kara kaj diris la vortojn:\\
           \vin  "Kara patrino, elportu mal\^gojon malgra\u ue sufero,\\
          Ke mi vin, ho, la plej kara, ne vidu per propraj okuloj\\
          Balda\u u batata; \^car vana, kiel ajn volus mi, estos\\
          Mia helpo, \^car Ze\u uson kontra\u ui estas ne eble.\\
          \^Car jam anta\u ue, kiam fojon mi volis vin helpi,\\
          Kaptis li min \^ce l' piedo, de sojlo la sankta min \^{\j}etis.\\
          Tutan la tagon mi flugis suben kaj dum sunsubiro\\
          Mi sur Lemnoson surfalis, havante malmulte da vivo,\\
          Sed la Sintoj ricevis la subenfalinton amike."\\
          \vin   Tiel eldiris li, kaj ekridetis Hero blankmana\\
          Kaj ridetante \^si prenis de sia filo pokalon.\\
          Li do al \^ciuj ceteraj senmortaj, irante de dekstre,\\
          Dol\^can nektaron enver\^sis, el la kaliko \^cerpante,\\
          Kaj inter dioj feli\^caj eksonis ridado senfina,\\
          Kiam ili lin vidis kuradi kaj rekuradi.\\
           \vin  \^Gis sunsubiro festenis ili dum tuta la tago,\\
          Kaj nenio mankis al koroj dum festenado\\
          Bonaran\^gita: nek sonoj belegaj de liro Apola.\\
          Nek l' harmonia kantado de Muzoj, kantantaj alterne.\\
          \vin   Sed kiam ekestingi\^gis la lumo de l' suno brilega,\\
          \^Ciu el ili foriris dormi en sian lo\^gejon,\\
          Kiun al \^ciu lamapieda artisto Hefesto\\
          Ekkonstruis per sia multescianta prudento.\\
          Ze\u uso do tondron\^{\j}etanta iris al sia ku\^sejo,\\
          Kie li ofte ku\^sadis sub ago de dormo la dol\^ca.\\
          Tien veninte, li dormis, kaj apud li --- Hero orkrona.
\end{verse}
