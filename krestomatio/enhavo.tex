\selectlanguage{esperanto}
\begin{center}
\narrow{\Large{TABELO DE LA ENHAVO}}
\thispagestyle{plain}
\end{center}

{ % environment for parskip settings

\setlength{\parindent}{0pt}
\setlength{\parskip}{1em}

{\centering \rule{13mm}{0.4pt}\par}

{\sansfont EKZERCOJ} \dotfill \pageref{ekzercoj}

{\centering \sansfont FABELOJ KAJ LEGENDOJ\par}

La novaj vestoj de la reĝo \dotfill \pageref{novajvestoj} \\
Aleksandro Macedona (V. Jabec) \dotfill \pageref{aleksandro} \\
Kiu estas la plej bona amiko de la viro? \dotfill \pageref{plejbona} \\
La deveno de la virino \dotfill \pageref{deveno} \\
La naskiĝo de la tabako (C. Bachiller) \dotfill \pageref{tabako}\\
Karagara \dotfill \pageref{karagara} \\
La virineto de maro \dotfill \pageref{virineto}

{\sansfont ANEKDOTOJ} \dotfill \pageref{anekdotoj}

{\centering \sansfont RAKONTOJ\par}

Nokto \dotfill \pageref{nokto} \\
La hejmo de la metiisto \dotfill \pageref{metiisto}\\
La forgesita pipo (A. Edelmann) \dotfill \pageref{pipo}\\
Arturo \dotfill \pageref{arturo}\\
La nigra virino \dotfill \pageref{nigra}\\
La karaj braceletoj \dotfill \pageref{braceletoj}\\
Nur unu vorton \dotfill \pageref{unuvorton}\\
La porcio da glaciaĵo \dotfill \pageref{porcio}

{\centering \sansfont EL LA VIVO KAJ SCIENCOJ\par}

Bagateloj \dotfill \pageref{bagateloj} \\
Fingra kalendaro \dotfill \pageref{fingra} \\
El la poŝto \dotfill \pageref{posxto} \\
La loĝejoj de la termitoj \dotfill \pageref{termitoj} \\
Korespondado komerca \dotfill \pageref{komerca} 

\newpage

Kronika katara konjunktivito \dotfill \pageref{kronika} \\
La sunhorloĝo en Dijon \dotfill \pageref{dijon} 

{\centering \sansfont ARTIKOLOJ PRI ESPERANTO\par}
\vspace{1em}

{ % outdent this section because the Leibnitz title is nuts

\setlength{\leftskip}{1em}
\setlength{\parindent}{-1em}
\setlength{\parskip}{0pt}

El la unua libro de la lingvo Esperanto \dotfill \pageref{unualibro}

Plena gramatiko de Esperanto  \dotfill \pageref{plena} 

Al la historio de la provoj de lingvoj tutmondaj de Leibnitz \\ ĝis la
  nuna tempo  \dotfill \pageref{leibnitz} 
  
Esenco kaj estonteco de la ideo de lingvo internacia  \dotfill \pageref{esenco} 

} % end outdented bit

{\centering \sansfont POEZIO\par}

% author, dash, poem, dots, page
\begin{longtabu} to\textwidth{@{}l@{ }c@{ }X@{}r@{}}
\bf L. Zamenhof. & --- & La espero \dotfill & \pageref{laespero}\\
\hfil--- & & La vojo \dotfill & \pageref{lavojo}\\
\hfil--- & & Al la fratoj \dotfill & \pageref{allafratoj}\\
\hfil--- & & Mia penso \dotfill & \pageref{miapenso}\\
\hfil--- & & Ho mia kor' \dotfill & \pageref{homiakor}\\
\hfil--- & & La vojevodo (Mickiewicz) \dotfill & \pageref{vojevodo}\\
\hfil--- & & La rozeto (Goethe) \dotfill & \pageref{rozeto}\\
\hfil--- & & En nord' unu pino (Heine) \dotfill & \pageref{unupino}\\
\hfil--- & & En sonĝo (Heine) \dotfill & \pageref{songxo}\\
\hfil--- & & Lorelej' (Heine) \dotfill & \pageref{lorelej}\\
\hfil--- & & Kanto de studentoj \dotfill & \pageref{studentoj}\\
\hfil--- & & Al la reĝo \dotfill & \pageref{regxo}\\
\hfil--- & & Nokta kanto de soldato (W. Hauff) \dotfill & \pageref{soldato}\\
\hfil--- & & Kanto de l'ligo (A. Mozart) \dotfill & \pageref{ligo}\\
\hfil--- & & La kapelo (L. Uhland) \dotfill & \pageref{kapelo}\\
\hfil--- & & La gaja migranto \dotfill & \pageref{migranto}\\
\hfil--- & & La vojo (B. N. Delvig) \dotfill & \pageref{lavojo2}\\
\vspace*{-38pt}
\end{longtabu}
\begin{longtabu} to\textwidth{@{}l@{ }c@{ }X@{}r@{}}
\bf Léo Belmont. &--- & Nokt' en la koro (Byron) \dotfill & \pageref{noktenlakoro}\\
\hfil--- & & Al brusto, al min (Heine) \dotfill & \pageref{albrusto}\\
\vspace*{-38pt}
\end{longtabu}
\begin{longtabu} to\textwidth{@{}l@{ }c@{ }X@{}r@{}}
\bf V. Devjatnin. & --- & Printempo \dotfill & \pageref{printempo}\\
\hfil--- & & Ŝipeto \dotfill & \pageref{sxipeto}\\
\hfil--- & & Sciuro kaj papago \dotfill & \pageref{sciuro}\\
\hfil--- & & Anĝelo (Lermontov) \dotfill & \pageref{angxelo}\\
\hfil---  &  & Husaro (Puŝkin) \dotfill & \pageref{husaro}\\
\bf V. Devjatnin. & --- & Infano de zorgo (Herder) \dotfill & \pageref{infano}\\
\hfil--- & & Espero (Schiller) \dotfill & \pageref{espero}\\
\hfil--- & & Garantio (Schiller) \dotfill & \pageref{garantio}\\
\vspace*{-38pt}
\end{longtabu}
\begin{longtabu} to\textwidth{@{}l@{ }c@{ }X@{}r@{}}
\bf A. Dombrowski. & --- & Mia mizero \dotfill & \pageref{mizero}\\
\hfil--- & & Nova kanto \dotfill & \pageref{nova}\\
\hfil--- & & La malliberulo \dotfill & \pageref{malliberulo}\\
\vspace*{-38pt}
\end{longtabu}
\begin{longtabu} to\textwidth{@{}l@{ }c@{ }X@{}r@{}}
\bf M. Goldberg. & --- & La turo babilona \dotfill & \pageref{turo}\\
\hfil--- & & Nova Dio (S. Frug) \dotfill & \pageref{novadio}\\
\hfil--- & & Cigno, ezoko kaj kankro (Krylov) \dotfill & \pageref{cigno}\\
\vspace*{-38pt}
\end{longtabu}
\begin{longtabu} to\textwidth{@{}l@{ }c@{ }X@{}r@{}}
\bf A. Grabowski. & --- & Tri Budrysoj (Mickiewicz) \dotfill & \pageref{tri}\\
\hfil--- & & Revaĵo (P. Dalman) \dotfill & \pageref{revajxo}\\
\hfil--- & & Excelsior (Longfellow) \dotfill & \pageref{excelsior}\\
\hfil--- & & La pluva tago (Longfellow) \dotfill & \pageref{pluva}\\
\hfil--- & & La sago kaj la kanto (Longfellow) \dotfill & \pageref{sago}\\
\hfil--- & & Sonoriloj de vespero (T. Moore) \dotfill & \pageref{sonoriloj}\\
\vspace*{-40pt}
\end{longtabu}
{\bf E. Haller.} --- Mi rakontis (Vejnberg) \dotfill  \pageref{rakontis}\\
{\bf G. Janowski.}  ---  Mi eliras (Lermontov) \dotfill \pageref{eliras}\\
{\bf D. Jegorov.}  ---  Aŭtuno \dotfill  \pageref{auxtuno}\\
{\bf F. de Kanaloŝŝy-Lefler.}  ---  Kuŝas somero (Heine) \dotfill  \pageref{somero}\\
\vspace*{-21pt}
\begin{longtabu} to\textwidth{@{}l@{ }c@{ }X@{}r@{}}
\bf A. Kofman. & --- & Filino de Iftah \dotfill & \pageref{filino}\\
\hfil--- & & La sklavoŝipo (Heine) \dotfill & \pageref{sklavo}\\
\vspace*{-40pt}
\end{longtabu}
{\bf V. Langlet.} --- Al la memoro de Jozef Wasniewski \dotfill \pageref{memoro}\\
{\bf R. Libeks.}  ---  Latva popola kanto \dotfill  \pageref{latva}\\
\vspace*{-21pt}
\begin{longtabu} to\textwidth{@{}l@{ }c@{ }X@{}r@{}}
\bf I. Lojko & --- & Malgrand'-rusa kanteto \dotfill & \pageref{malgrand}\\
\hfil--- & & Profeto (Lermontov) \dotfill & \pageref{profeto}\\
\vspace*{-40pt}
\end{longtabu}
{\bf F. Lorenc.}  ---  Alaŭdeto (Boheme) \dotfill  \pageref{alauxdeto}\\
\vspace*{-21pt}
\begin{longtabu} to\textwidth{@{}l@{ }c@{ }X@{}r@{}}
\bf A. Naumann. & --- & Mi amis vin \dotfill & \pageref{miamasvin}\\
\hfil--- & & Printempo venos \dotfill & \pageref{printempovenos}\\
\vspace*{-40pt}
\end{longtabu}
{\bf Poeteto.}  ---  La Ĉaso \dotfill  \pageref{cxaso}\\
\vspace*{-21pt}
\begin{longtabu} to\textwidth{@{}l@{ }c@{ }X@{}r@{}}
\bf I. Seleznet. & --- & Kanto. \dotfill & \pageref{kanto}\\
\hfil--- & & Ne riproĉu la sorton \dotfill & \pageref{riprocxu}\\
\hfil--- & & Revo \dotfill & \pageref{revo}\\
\vspace*{-40pt}
\end{longtabu}
{\bf E. Smetanka.}  ---  La poeto (V. Halek) \dotfill  \pageref{poeto}\\
{\bf L. Sokolov.}  ---  La velo (Lermontov) \dotfill  \pageref{velo}\\

\newpage
\begin{longtabu} to\textwidth{@{}l@{ }c@{ }X@{}r@{}}
\bf M. Solovjev. & --- & Plendo \dotfill & \pageref{plendo}\\
\hfil--- & & Tri palmoj (Lermontov) \dotfill & \pageref{palmoj}\\
\vspace*{-40pt}
\end{longtabu}
{\bf K. Svanbom.}  ---  Laŭ sveda melodio \dotfill  \pageref{sveda}\\
{\bf S. Satunovski.}  ---  Ne diru (Nadson) \dotfill  \pageref{nediru}\\
{\bf W. Waher.}  ---  Spirita Ŝipo \dotfill  \pageref{spirita}\\
{\bf J. Wasniewski.}  --- El la paperujo de miaj fabloj \dotfill  \pageref{paperujo}\\
{\bf E. de Wahl.}  ---  Ĉe l' maro (Heine) \dotfill  \pageref{maro}\\
\vspace*{-21pt}
\begin{longtabu} to\textwidth{@{}l@{ }c@{ }X@{}r@{}}
\bf F. Zamenhof. & --- & Versaĵo sen fino \dotfill \pageref{senfino}\\
\hfil--- & & Vizito de la steloj sur la tero \dotfill \pageref{vizito}\\
\vspace*{-40pt}
\end{longtabu}
{\bf O. Zeidlitz.}  ---  Vespero (A. Nikander) \dotfill  \pageref{vespero}\\
{\bf L. Zamenhof.}  ---  El Hamleto \dotfill  \pageref{hamleto}\\
{\bf A. Kofman.} --- El Iliado \dotfill  \pageref{iliado}

{\centering \sansfont ALDONO\par}

Preĝo sub la verda standardo \dotfill \pageref{standardo}

} % end environment for parskip settings

\vspace*{\fill}

\begin{center}
\rule{13mm}{0.4pt}\\
\footnotesize 1432-05. --- Coulommiers. Imp. \fsc{Paul} BRODARD. --- 12-05.
\end{center}
\cleardoublepage
