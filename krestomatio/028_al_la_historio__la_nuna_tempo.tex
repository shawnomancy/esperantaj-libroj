\begin{center}
\footnotesize Publika parolado, havita en la Nurnberga klubo de instruistoj la
11 de Novembro 1884 do L. \fsc{Einstein}. (El la "Bayerische Lehrer-
                     Zeitung" No. 11 kaj 12 1885.)
\end{center}

   Jam de 200 jaroj kun rilate malgrandaj interrompoj estis farataj
provoj krei lingvon internacian, kaj la deziro de komuna lingvo por
\^ciuj homoj estis esprimita e\^c jam anta\u u pli ol 2000 jaroj de
la profeto Zefaniah. Tiu \^ci deziro havas anka\u u tre gravan
fundamenton; \^car de \^ciam la babilona konfuzo de lingvoj estis
granda malhelpo al la spirita komuniki\^go inter la nacioj. En nia
tempo, kiam la interkomuniki\^go de la popoloj per la telegrafoj,
vapor\^sipoj kaj fervojoj grandi\^gis en maniero neniam supozita,
anka\u u la ellernado de la lingvoj de \^ciuj tiuj \^ci popoloj
fari\^gis bezono de la bone edukitaj kaj precipe de la komercistoj;
sed tiu \^ci lernado estas tiel malfacila kaj tempon deprenanta, ke
\^gi pleje devas esti farata kun perdo por la aliaj regionoj de
scienco, kaj tial la postulado de unu lingvo komuna, akceptita de
\^ciuj edukitaj homoj, anstata\u uanta \^ciujn aliajn fremdajn
lingvojn, montras sin des pli pravigita. Oni povus tial atendi, ke
tia elpenso estos de la homaro akceptita kun la plej granda plezuro!
Sed tiu \^ci ideo entute en la publiko ne estas tiel hejme, kiel
\^gi devus esti, kontra\u ue, ni renkontas, e\^c en sferoj, kie ni
devus \^gin atendi la plej malmulte, anta\u u \^cio la anta\u
uju\^gon, ke enkonduki unu lingvon por \^ciuj homoj estas simple
afero ne ebla. Tamen la argumentoj, per kiuj oni kontra\u u tio \^ci
batalas, estas aluzeblaj nur por lingvoj naturaj kaj neniel por
lingvo arta, kiun oni devas rigardi nur kiel arte la\u u la modeloj
de la lingvoj naturaj kreitan "elekton de la plej bona". Cetere mi
trovis, ke vera kompreno de tiu \^ci objekto nur tiam povas esti
atingita kun utilo, kiam ni anta\u ue okupos nin iom je la preparaj
laboroj, kiuj anta\u uiris al la nuna sistemo de lingvo tutmonda.

   Tie \^ci ne estas al mi eble doni detalan priskribon de tiuj \^ci
\^cirka\u u 60 provoj; estas sufi\^ce, se ni nur enkomune vidos, kio
\^cio estis farita en tiu \^ci rilato, kaj kiajn rimedojn oni
\^ciufoje volis uzi, por \^ciujn homojn lingve unuigi. Dume el
\^ciuj tiuj \^ci penoj montri\^gas la instruo, kiu anka\u u en aliaj
aferoj pruvi\^gas, ke la grandajn verojn oni atingas ne per tre
malgranda kosto, sed ke al tio, kion oni fine konfesas kiel
"mirinde simpla", oni proksimi\^gis \^ciam per malrapidaj kaj
malfacilaj flankaj vojoj nur iom post iom kaj pa\^so post pa\^so.

   Oni vidas ordinare en Leibnitz la patron de la "ideo de lingvo
tutmonda"; tiu \^ci publikigis sian ideon pri la "Pazigrafio a\u u
la arto fari sin komprenebla per komunaj skribaj signoj al \^ciuj
nacioj de la tero, en kiaj ajn malegalaj lingvoj ili parolas, se ili
nur konas tiujn \^ci komunajn signojn" la unuan fojon en la jaro
1666 en latina disertacio (Dissertatio de arte combinatoria, Lipsiae
1666). Tiu \^ci ideo anka\u u okupis la grandan pensanton \^gis la
fino de lia vivo, kvankam li en tio \^ci venis al nenio, krom
senfruktaj iniciatoj. Tamen oni povas jam la hieroglifojn rigardi
kiel specon de pazigrafio, kiel anka\u u la \^hinajn signojn de
skribado, kiujn uzas pli ol triono de la tuta homaro, malgra\u u
iliaj preska\u u nevenkeblaj malfacila\^{\j}oj. \^Car tiu \^ci idea
skribado, konsistanta el \^cirka\u u 40 000 signoj, en kiu per unu
komuna simbola signo estas esprimata tiu sama penso a\u u
komprena\^{\j}o, kiu, estante elparolata, en \^ciu lingvo sonus
alie, povas tial esti legata en \^ciu lingvo, kaj la \^Hino tial
anka\u u legas siajn skribajn signojn \^ciam en sia dialekto, kaj la
aliaj popoloj de Azio legas tiujn samajn signojn en ilia lingvo.
Tiuj \^ci popoloj reciproke sin tiel ne komprenas per sia lingvo,
sed komprenas sin per sia skribado, oni povus diri, per la Azia
pazigrafio\footnote{V. D-ro Albert Wild: "Ueber Geschichte der
Pasigraphie und ihre Fortschritte in der Neuzeit". München 1864.
(Aparta represo el la "Chronik der Gegenwart".)}.

   Tiaj kredeble estis la konsideroj, kiuj jam en la jaro 1668 igis la
episkopon John Wilkins\footnote{John Wilkins, An Essay towards a real character and
philosophical language, 1868.} eldoni sian grandiozan pazigrafian provon,
kiun prezentas al ni en bonega maniero Max Müller en la 2a volumo
de siaj "Lekcioj pri la scienco de la lingvo", el kiuj ni vidas,
ke tie la ideoj kun siaj signoj estis orditaj la\u u specoj kaj
dividitaj en klasojn, tiel ke de la komuna oni venadis al la aparta
kaj de tie \^ci per \^ciam pli precizaj specialigadoj oni venadis
fine al la plej speciala, kion oni volis esprimi en tia maniero.


   Sed jam Leibnitz esprimas sin malfavore pri \^gi pro la multego da
signoj kaj pro la malfacileco de ilia uzado por la intencita celo,
kaj Max Müller mem montras al ni, kiel Wilkins fine, kvaza\u u
kondukita de natura instinkto, venis al la vera, nome al "lingvo
vorta", kaj tiun \^ci vojon oni devus nur iri plu, por \^spari al
la homaro ducentjaran vanan penadon kun fantomo pazigrafia.

   Sed kiel e\^c tia granda pensanto, kiel Leibnitz, sin movadis ankora\u u
en mallumaj anta\u usentadoj, oni vidas el tio, ke li ankora\u u
e\^c ne povis klare sin esprimi pri ia ideo de lingvo tutmonda;
\^car li diris pri la laboroj de Wilkins kaj de Dalgarno, ke la
\^cefa afero, kiu tie \^ci estas bezona, la eltrovo de komunaj
signoj por \^ciu objekto kaj \^ciu ideo, al ili ankora\u u ne
prosperis. Tiaj signoj la\u u lia opinio devis esti similaj je la
signoj de la algebro, kaj \^sajnis al li necese, se oni volis tute
atingi la celon, elpensi ion en la maniero de alfabeto de la homaj
pensoj\footnote{D. S. Chofrin, amusements liter. vol. I p. 28.
--- En aparta traktato: "Geschichte und Empfehlung einer allgemeinen
Schriftzeichensprache" (v. Leibnitz Werke nach Raspe, vol II p.
615-653). Leibnitz e\^c klare komprenigas, ke por \^ciuj ideoj oni
devas akcepti karakterajn nombrojn k.c.}, el kio cetere Immanuel Niethammer (1808) volas tiri la
konkludon, ke Leibnitz per tio \^ci jam pensis "fonetikan lingvon
vortan".

   Jam anta\u u Wilkins D-ro Joh. Joach. Becher (1661) proponis numeri
la vortojn de tuta vortaro kaj tiujn \^ci nombrojn uzi kiel komunan
lingvon de skribado. Sed sendube estas pruvite, ke nur de la tempo
de Leibnitz la ideo krei pazigrafion ricevis dece radikojn kaj
okupis parte apartajn instruitulojn, parte instruitajn akademiojn
kaj registarojn. En Germanujo, Francujo, Anglujo, Hispanujo,
Hungarujo, Rusujo, Danujo k. c. estis faritaj multaj provoj kaj
proponoj por la atingo de tiu \^ci celo, kaj ankora\u u en la jaro
1811 la akademio de sciencoj en Kopenhago difinis premion por la
plej bona prezento de facila kaj praktike efektivigebla pazigrafio.Müller

   Sed tiel same kiel Wilkins kaj Dalgarno, ne venis al la celo anka\u u
Anastasius Kirchner (1665), Peter Porele (1667) a\u u Joh. Upperdorf
(1679-80), anka\u u Andreas Müller (1681), kiu havis la intencon
krei universalan lingvon, fonditan sur la \^hina lingvo kaj \^giaj
signoj de skribado, kaj ne pli bone prosperis al Joh. Caramuel von
Lobkowitz (1687), kiel anka\u u al lia anta\u uiranto la jezuito
Besuier (1684), a\u u al lia postiranto David Solbrig (1725). En la
jaro 1772 la hungaro Kalmar de Taboltzafo en sia verko reduktis la
tutan sumon de la homaj ideoj al \^cirka\u u 500 fundamentaj kaj
komunaj, \^ce kio li uzis skribajn signojn de \^ciuj popoloj, sed
precipe malabarajn. Tiu \^ci universala signa skribado, kiu sekve
prave portis la nomon "pazigrafio", estis tamen tiel malfacila, ke
nur la senlaca Boyle povis \^gin tute ekposedi.

   Post tiu \^ci verko sekvis ankora\u u multaj, kiel ekzemple de
Chr. G. Berger "Plan zu einer allgemeinen Rede --- und
Schriftsprache fuer alle Nationen" (Berlin 1779) kaj Delormel
(1795), la pazigrafio de Vater (Wien 1795), la pazigrafio de M. de
Maimie\u u (Paris 1797), kiu anka\u u estas fondita sur la numerado
de la vortoj, kiel anka\u u en tiu sama tempo M. Budet kaj M.
Chambry.

   \^Ciuj tiuj \^ci havis nenian sukceson, \^gis en la jaro 1796 la glora
instruisto de surdamutaj, Sicard, kun granda pompo anta\u uanoncis
oportunan pazigrafion, kiu, atendita kun granda senpacienco, estis
eldonita en la oficejo de pazigrafio en Parizo en la lingvoj germana
kaj franca du jarojn post \^gia anta\u uanonco. Kvankam li certigis,
ke li bezonas nur 12 signojn, nomitajn "gamoj", li tamen uzis da
ili multe pli multe, por ordigi multajn vortojn, helpajn verbojn kaj
ideojn, kiujn li dividis en 3 \^cefajn aparta\^{\j}ojn kaj poste
ankora\u u en klasojn kaj subklasojn, kaj al kiuj estis ankora\u u
bezonaj diversaj linioj kaj punktoj. La vortaro servas al \^ciu en
lia propra lingvo, sed estas en sia konstruo tre vasta, kaj kiel ajn
genia la sistemo estas --- \^car nenia havis tian difinitecon --- ,
anka\u u tie \^ci la ellernado kostis multan kaj grandan laboradon.

   Post tio \^ci la bone konata en sferoj pedagogaj lingvisto Wolke\footnote{C. H. Wolke,
Erklärung, wie die wechselseitige Gedankenmittheilung allen
kultivirten Völkern des Erdkreises oder die Pasigraphie möglich
und ausüblich sei, ohne Erlernug irgend einer neuen, besonderen
oder einer allgemeinen Wort- oder Zeichensprache. Dessau 1797.},
profesoro de la universitato en S. Peterburgo, publikigis en la jaro
1797 en Dessa\u u sian propran elpenson de interkomuniki\^ga skriba
lingvo, kiu konsistis en tio, ke \^ciu lingvo postulis apartan
leksikonon, kiu devis enhavi \^ciun vorton kun \^giaj formoj
deklinaciaj kaj konjugaciaj; la vortoj sur \^ciu pa\^go estis
numeritaj per 1, 2, 3 k.c. kaj al \^ciuj sur la flanko estis
aldonitaj la nombroj de la pa\^goj kaj numeroj, sub kiuj tiu sama
vorto estas trovebla en \^ciuj aliaj vortaroj. Oni tiel en kiu ajn
dezirita lingvo devis ion nur anstata\u u per literoj skribi per
nombroj de pa\^goj kaj numeroj de la vortoj, tiam \^ciu, kiu volis
legi la skribitan en alia lingvo, devis ser\^ci la enhavon en la
vortaroj de tiu a\u u alia lingvo. Sed se ni konsideros, kiel
mallerte, malfacile kaj multekoste tio \^ci devis esti, ni ne devas
miri, ke tia sistemo ne povis trovi amikojn. \^Car se oni volus
pretigi vortarojn de nur \^cirka\u u 16 lingvoj, tiam, alkalkulante
la nombron kaj la komon necesan apud \^ciu el ili, estus necesa loko
de 135 literoj tipografiaj. Kaj tiu \^ci laboro, plendas Wolke, pli
multe lin okupis, ol kiom li volus pro siaj aliaj laboroj. Al tio
\^ci li e\^c ne pensis pri la sintakso de la lingvoj. El la ceteraj
pazigrafiaj sistemoj, kiuj aperis \^cirka\u u la fino de la 18
centjaro, de Fry en London, de G. E. Busch en la "Jahrbuch des
Fortschrittes der Wissenschaften", kiel anka\u u de Grotenfeld en
Goettingen, la verko "Pasigraphie und Antipasigraphie" de J. S.
Vater (Weissenfels kaj Leipzig 1799) estas la sola, kiu havas indon;
\^car la a\u utoro jam esprimas en \^gi la opinion, ke la pazigrafio
estas rimedo por la interkomuniki\^go kaj komerco en malproksimaj
landoj, kies lingvojn ni ne komprenas, kaj ke tio \^ci estas la plej
bona parto, kiun la pazigrafio povas atingi.

   En la komenco de la 19 centjaro ni trovas anta\u u \^cio Naether'on en
Görlitz, kiu en la jaro 1805 publikigis verkon\footnote{J. Zach. Naether,
Versuch einer ganz neuen Erfindung von
Pasigraphie oder die Kunst zu schreiben und zu drucken, dass es von
allen Nationen der ganzen Welt in allen Sprachen eben so leicht
gelesen werden kann als die Zahlencharaktere 1, 2, 3; in Form einer
Sprachlehre oder Grammatik nebst 20 pasigraphischen Uebungen.
Görlitz 1805.}, en kiu li faras la
proponon uzi karakteran skribadon, kiu estus figura kaj prenita el
la naturo. Sed \^car ne facile estas ellerni senfinan nombron da
figuraj signoj, tial tiu \^ci simbola skribado, tute malegala je la
\^gistiamaj sistemoj, estas tre maloportuna kaj malfacile uzebla.
Prave sin levas kontra\u u tiu \^ci, kiel anka\u u kontra\u u \^ciuj
aliaj \^gistiamaj sistemoj, en bonege skribita kritika polemiko, la
tiama re\^ga Bavara centra konsilano de instruado \^ce la sekreta
ministra\^{\j}o de interna\^{\j}oj en Mun\^heno, Friedr. Immanuel
Niethammer\footnote{F. J. Niethammer: Ueber Pasigraphik und
Ideographik, Nürnberg, bei Karl Felssonecker, 1808.}, kiu tute vere
venas al la konkludo, ke sur vojo simbola, \^cu \^gi estos signoj
hieroglifaj a\u u nombraj a\u u signoj de sonoj (literoj), uzataj
kiel signoj de pensoj, oni neniam povos veni al la celo. Sed tute ne
intencante for\^{\j}eti la penson de norma lingvo, li fine venas al
la prudenta opinio, ke nur imitita al la lingvoj naturaj "fonetika
vorta lingvo" povas solvi la problemon de lingvo tutmonda; anka\u u
pli: tiu \^ci natura iro de ideoj alkondukas lin e\^c al la penso,
ke devas esti eble elpensi skribadon, kiu permesus, "pensadi sur la
papero", t. e. skribadi kun la rapideco de la pensado, --- sekve
"tempo\^sparantan ideoskribadon a\u u pensodesegnadon", kiun li
nomis "Ideografiko". Kiel oni tamen vidas el la pluaj partoj de
lia traktato, tiu \^ci ideografiko devus kvankam elveni el la
ideofoniko kaj per tio \^ci tuj fari\^gi ideolaliko, tamen anka\u u
lia idealo \^sajnas iri pli malproksimen, ol nia nuna stenografio
kaj la Volap\"uk de Schleyer; sed estas sendube, ke per sia skribado
li pensas nian alfabetan skribadon de sonoj, kiel per sia lingvo li
pensas lingvon vortan, similan je tiu, por kiu ni batalas.

   \^Ciuj tiuj \^ci kaj aliaj provoj rilate la praktikan utilon alkondukis
tiel same malmulte al ia rezultato, kiel la jam dirita voko de la
Kopenhaga akademio, kaj de tiu tempo (1811) \^gis la nuna
(esceptinte: "Le polyglotte improvise ou l'art d'ecrire les langues
sans les apprendre" de A. Renzi en Parizo 1840 kaj la pazigrafio de
Sunderwall en Svedujo kaj la "New universal cipher language" 1874
de unu ne nomita a\u utoro en Londono) ni havas nenion pli por noti.
Por tio la plej novaj pazigrafiistoj, pri kiuj ni nun parolos, nome:
barono de Gablenz, Moses Paic, Don Sinibaldo de Mas kaj Anton
Bachmaier, kiel anka\u u Albert Walter --- staras pli alte, ol iliaj
anta\u uirantoj, a\u u en rilato de la konstruo, a\u u en rilato de
facileco de la metodo.

   La konstruo de la "Gablenzographia kaj Gablenzolalia" de barono
de Gablenz estas grandioza. Anta\u u \^cio li penis per aparta
alfabeto la\u u 33 diversaj lingvoj en tiom same da \^slosiloj solvi
la problemon por la mondo lingvista skribadi tiel, kiel oni parolas.
Li verkis gramatikon kaj vortaron, kiu pleje konsistas el vortoj
unusilabaj. Sed kiel ajn bonega lia laboro estus, tre granda estas
la malfacileco ellerni la 33 kaprompantajn \^slosilojn. Al tio \^ci
li tamen ne turnis atenton, ke la popoloj orient-Aziaj ne tre facile
povas elparoladi la e\u uropajn konsonantojn, nome, la "r", kiun
li precipe multe uzas en siaj nomoj de nombroj (ra 1, re 2, ri 3, ro
4, ru 5). Kvankam la provoj de la pazigrafio de Gablenz por la
praktika vivo ne ta\u ugas, li tamen montris, ke li bone konas la
spiriton de la lingva scienco. En la jaro 1863 samtempe kun la
proponoj de Grimm en Konstantinopolo aperis la verko de Don
Sinibaldo de Mas, pri kiu anka\u u estas iom parolite en la jam
diritaj lekcioj de Max Müller; \^gi aperis samtempe en Londono,
Parizo kaj Lejpcigo kaj meritas atenton pro sia sistemo. Sed jam la
difino, kiun li donas al ni pri ideografio: "L'ideographie est
l'art d'ecrire avec des signes qui representent des idees et non
avec des mots (sons) d'une langue quelconque" ankora\u u unu fojon
montras, ke oni \^ciam ankora\u u sin movadis sur la malvera vojo,
kiu \^gis nun ne kondukis al la celo kaj anka\u u neniam al \^gi
kondukos. Elirante el la penso, ke la signoj de nombroj en la
aritmetiko kaj algebro por tiuj popoloj, kiuj ilin uzas, estas nenio
alia, ol signoj ideografiaj, li penas elmontri, ke 500 milionoj da
homoj, la Japanoj, Ko\^hin\^hinanoj, Tonkinanoj sin komprenas
reciproke nur per unu sama skribado kaj uzas \^gin \^ciutage. Kaj
kio \^gi estas alia, ol signoj ideografiaj? \^Cu oni tiel povos
ankora\u u dubi la eblecon de tiu \^ci sistemo? \^Cu ni estus
nekapablaj fari tion, kion faras la Azianoj, kiam ni ja kredas, ke
ni staras multe pli alte, ol ili? --- La eblecon de la uzado anka\u
u ni ne dubas, sed ni anka\u u ne dubas la malfacilecon de la
enkondukado. La \^Hinoj laboru multajn jarojn super la lernado de
sia skriba lingvo kun \^giaj 40\,000 diversaj signoj --- ni
okcidentuloj jam de anta\u ue havas nian hereditan de la
feniko-hebreoj fonetikan skribadon alfabetan, kiu donas al ni la
eblon el 23 literoj formi multajn milionojn da sonoj, kiuj, estas
vere, ne povas esti nomataj vortoj, se kun ili ne estas ligataj
difinitaj ideoj (komprena\^{\j}oj). Sed el tio \^ci almena\u u
sekvas, ke ni povas krei al ni neniajn pli ri\^cajn kaj pli
praktikajn reprezentantojn por la grandioza diverseco de niaj ideoj,
ol la vortojn de lingvo, kiuj \^ciuj estas kunmetitaj nur el
malmultaj sonoj fundamentaj. Kaj ni devus nun ree nin turni al la
senmovaj egiptaj a\u u \^hinaj signoj figuraj, kiuj estas fonditaj
ankora\u u sur la primitiva senta pririgardado kaj nur kun grandaj
malfacila\^{\j}oj permesas la prezentadon de abstrakta\^{\j}o! ---
Don Sinibaldo de Mas prenas por \^ciu "ideo" unu signon, pruntitan
el la muzika nota sistemo, nur kun la diferenco, ke li donas ne
efektivajn notojn en \^ciuj iliaj nuancoj. La signon fundamentan
formas noto kvarono, kaj la\u u tio, \^cu \^gi staras sur tiu a\u u
alia linio, \^gi \^san\^gas sian signifon kiel substantivo, verbo,
adjektivo k. c., kaj en simila maniero estas esprimataj \^ciuj
formoj de la gramatiko. Kiel ajn facila tiu \^ci sistemo \^sajnas je
la unua rigardo, tamen tre balda\u u montrigas, ke la signoj por
universala uzado estas tro komplikitaj.

   Unu jaron poste Moses Paic, Serbo el Zemlin, publikigis pazilalion
kaj pazigrafion. (Pli detale v. en la verko de D-ro Wild.) Li uzis
sole ciferojn, por esprimi \^ciujn vortojn la\u u iliaj elementaj
ideoj, tiel ke difinita vorto aperas \^ciam esprimata per difinita
cifero. Tiel li uzas la nombrajn signojn de 1 \^gis 999 por \^ciuj
gramatikaj fleksioj; la nombroj komencante de 1\,000 estas la
pazigrafiaj signoj de ideoj. La apudaj komprena\^{\j}oj de komuna
ideo, la vortaj aliforma\^{\j}oj kaj devena\^{\j}oj estas formataj
per aldonado de pluaj nombroj al la nombroj de la ideoj per signoj
de aldonado a\u u deprenado. Tiel ekzemple 3243 signifas la komunan
ideon de a\^cetado, kaj tiam 3243 + 10 = a\^cetanto, 3243 + 13 = la
a\^cetanto, 3243 + 101 = la a\^cetantoj k. c. El sia pazigrafio M.
Paic faras pazilalion per tio, ke li por la apartaj nombroj metas
literojn kaj por "+" = "m", por "--" = "n"; ekzemple: 3243 +
10 = a\^cetanto = "fegimanos", 3243 + 20 = a\^cetantino =
"fegimenos" , 3243 + 40 = a\^ceto = "fegimonos" k. c. k. c.

   Oni tiel vidas, ke anka\u u Paic fine en simila maniero kiel Wilkins
venus al la \^gusta, nome al skribado kaj lingvo sona, kvankam
Wilkins tuj en la komenco laboris kun literoj, dum Paic kun ciferoj.
Li anka\u u pensis, ke por la ordinaroj aferoj de la vivo estas
sufi\^ca la ellernado de 1\,000 signoj de ideoj, kiuj kun la plej
granda facileco per helpo de la devenigitaj apudaj ideoj povas esti
alkondukitaj al 10\,000. Sed jam D-ro Wild rimarkis, ke tiu \^ci
sistemo postulas tro grandan pripensadon, por povi esti enkondukita
en la ordinaran interkomuniki\^gon kun malproksimaj popoloj. Li
opiniis anka\u u, ke se la pazigrafio estus enkondukebla \^ce la
Aziaj popoloj, oni devus preni por fundamento ne la ellaboritajn
lingvojn E\u uropajn, sed ni devus por tio \^ci krei pazigrafion
multe pli facilan, ol e\^c la \^hina; kaj, apogante sin sur tiun
\^ci principon, la lasta partiano de la ideo pazigrafia, Anton
Bachmaier en Mun\^heno, \^cefo de unu tiea granda komerca firmo, jam
en la jaro 1852 konstruis novan sistemon de pazigrafio, kiu en
komparo kun la anta\u ue nomitaj distingas sin per ekstrema
simpleco.

   Bachmaier anka\u u fondis sian sistemon sur la signoj de la nombroj,
\^car \^ciuj komercaj popoloj posedas la sistemon de la 10 signoj de
nombroj, kvankam iliaj formoj e\^c estas malegalaj. Anka\u u tiuj
\^ci signoj povas esti skribataj la\u u \^ciuj direktoj, ne sole
horizontale de maldekstre al dekstre, kiel \^ce la E\u uropanoj, sed
anka\u u de dekstre al maldekstre, kiel \^ce la Mahometanoj, a\u u
de supre al malsupre, kiel \^ce la Orientazianoj, kaj al tio \^ci
ili estas facile prezenteblaj per skribado, presado kaj telegrafo.
Bachmaier esprimas \^ciun komprena\^{\j}on per nombro, ne uzas
difinitan artikolon, sed por la artikolo nedifinita la nombron 1,
kaj signas la multenombron per substrekado de la nombro de la ideo.
La substantivoj estas sen sekso kaj deklinacio. La gradoj de komparo
estas distingataj per unu a\u u du punktoj metitaj super la nombro
de la ideo, la nombronomoj fundamentaj per supre metitaj, la
nombronomoj ordaj per malsupre metitaj punktoj. La verbo estas uzata
nur en la modo nedifinita (infinitivo), la estonta tempo estas
esprimata per superstrekado, la tempo pasinta per trastrekado de la
nombro de la ideo k. c. La plej konataj nomoj propraj kaj geografiaj
ricevis en la vortaroj apartajn nombrojn. Bachmaier lasis jam en 18
lingvoj prepari provajn ekzemplerojn de tiaj vortaroj, super kiuj
laboris la konata lingvisto profesoro Ignatz Gaugengigl kaj la
privata instruitulo Wilh. Stephanus. Al tio \^ci formi\^gis en
Mun\^heno centra societo por pazigrafio, al kiu apartenis unuaklasaj
instruituloj, kiel la germana Mezzofanti, Pf. Richter, kiu paroladis
en 34 lingvoj, Pf. Lauth, la glora egiptologo, D-ro Wild, Ja bonega
naciekonomiisto kaj statistikisto kaj multaj aliaj, kiel anka\u u la
barono von Gablenz en Dresdeno, Don Sinibaldo de Mas kaj Paic estis
honoraj membroj de \^gi.

   Malgra\u u ke la sistemo de Bachmaier estis ekstreme simpla, uzante nur
9 komunajn kaj 6 apartajn signojn, kaj e\^c jam turnis sur sin la
okulojn de la registaroj, tiel ke jam estis intencita universala
kongreso, kiu devis havi lokon en Parizo, tamen la entrepreno ree
perdi\^gis sen rezultato. Kaj kial? --- Se \^gin anka\u u sonoraj
ka\u uzoj, la elseki\^go de materialaj fontoj, anta\u u \^cio
pereigis, tamen la veran ka\u uzon oni devas ser\^ci pli profunde,
\^gi ku\^sas en la sistemo mem. \^Gi nome estis konstruita sur
sablo, kiel la sistomoj de \^ciuj anta\u uirantoj de 200 jaroj,
\^car la plej simplan kaj plej facilan oni la plej malfacile trovas.

   Tamen oni ne devas pensi, ke sistemo en tiel alta grado simpligita,
kiel la sistemo de Bachmaier, kvankam \^gi ne portis en si la
fundamenton por lingvo tutmonda, foriris el la mondo, alportinte
nenian utilon. Bachmaier diris mem en la anta\u uparolo al siaj
pazigrafiaj vortaroj: "Estas kompreneble, ke tiu \^ci skriba
maniero de komuniki\^go neniam estos egala al la bona\^{\j}oj de
lingvo; tamen por la komuniki\^go kun tiuj, kies lingvon oni no
komprenas (kiu homo komprenas \^ciujn lingvojn!) \^gi estas ekstreme
grava helpo". Efektive tiu \^ci sistemo por mallongaj notoj
komercaj, por sciigoj de gazetoj kaj precipe por telegramoj al
\^ciuj landoj de la mondo montri\^gis uzebla. La "Sistemo \^cifrada
kaj telegrafada de A. Walter en Winterthur", kiu en trimembraj
kunigoj de literoj de "aaa" \^gis "zzz" kune kun kelkaj ciferoj
por la plej necesaj gramatikaj rilatoj, en maniero de sekreta
skribado prezentas \^ciujn necesajn ideojn en formo tabela, donas en
\^ciu okazo por \^cifraj telegramoj \^sparon de kostoj \^gis 40 0/0
en komparo kun vortaj telegramoj en la lingvo germana; sed por
lingvo tutmonda, kiel por parolado tiel anka\u u por skribado,
anka\u u al \^gi mankas la natura fundamento, \^car e\^c Mezzofanti
ne povus teni en la kapo tiujn \^ci kunligojn de literoj, kiuj
anka\u u tie \^ci devas servi anstata\u u vortoj. Al tio \^ci ili
estas neelparoleblaj kaj tial por la bu\^sa interkompreni\^gado tute
ne uzeblaj. Anka\u u la verkado de vortaroj jam en unu lingvo estas
ligita kun grandegaj laboroj kaj malfacila\^{\j}oj, ne parolante jam
pri kelkaj kaj multaj lingvoj.

   (Tie \^ci la a\u utoro finas la historion de la diversaj provoj kaj
transiras al la priskribo de la sistemo Volap\"uk, de kiu li en tiu
tempo estis ankora\u u varmega partiano, trovante \^gin la plej bona
el \^ciuj tiam faritaj provoj. Ni jam rakontis [v. No. 10 de "La
Esperantisto" 1890], kiel post la apero de la sistemo "Esperanto"
la a\u utoro forlasis Volap\"ukon kaj transiris kun plena entuziasmo
al "Esperanto", de kiu \^gis la fino de sia vivo li restis varmega
batalanto. --- Ni devas ankora\u u aldoni, ke la nun alportita
artikolo ne estas plena historio de \^ciuj provoj de lingvo
tutmonda; volante skribi nur gazetan artikolon kaj ne dikan detalan
verkon, la a\u utoro parolis nur pri parto de la proponitaj
sistemoj, kaj anka\u u pri \^ciuj el tiuj \^ci li diris nur kelkajn
vortojn. La efektiva nombro de la diversaj projektoj, faritaj en la
lastaj 200 jaroj, estas pli ol 150).

\smallrule{}
