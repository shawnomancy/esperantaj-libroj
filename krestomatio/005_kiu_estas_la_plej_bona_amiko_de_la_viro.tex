\begin{center}
\footnotesize Sudrusa popol-rakonto, tradukita de K. \fsc{Hübert}.
\end{center}

   La supre metitan demandon respondas la malgrand'rusoj per la
sekvanta historieto. Iam, rakontas ili, lupo persekutis la diablon
en la la\u udinda intenco lin kapti kaj forman\^gi. Okaze tion \^ci
rimarkis malgrand'ruso, kiu laboris sur la kampo, kaj, \^cu pro tio,
ke la malgrand'ruso ne povas vivi sen la diablo, a\u u \^cu pro ia
alia ka\u uzo --- sufi\^ce, li eksentis kompaton por la diablo kaj
fortimigis la lupon per fortega "tju!". La diablo, \^gojante, ke
li liberi\^gis de la lupo, dankis la malgrand'ruson kaj invitis lin
veni la estontan vendredon en tian kaj tian profunda\^{\j}on, kie li
anka\u u estos kaj montros sin al li aparte dankema. Li nur ne venu
sole, sed alvenu kun sia plej bona amiko. La malgrand'ruso pensis,
ke la plej bona amiko de viro estas lia edzino, kaj tial li iris
tien kune kun sia edzino. Kiam ili alvenis al la signita loko, la
diablo ankora\u u ne estis tie kaj, por iel pasigi la tempon, la
malgrand'ruso ku\^sigis sian kapon sur la bruston de sia edzino,
kaj, dum \^si \^gin purigadis de certaj malpuraj bestetoj, li
dol\^ce ekdormis. Kiam li \^{\j}us ekdormis, tiam la diablo,
aliformi\^ginte en belegan junan viron, proksimi\^gis al la edzino,
amindumis kun \^si per la okuloj, kaj en kelkaj minutoj tute \^sin
gajnis tiel, ke \^si volonte ekkaptis la altiritan al \^si
tran\^cilon, por tratran\^ci la gor\^gon de sia edzo. Sed en tiu
sama momento, kiam \^si tion \^ci volis fari, la diablo, repreninte
sian veran ekstera\^{\j}on, la\u ute ekkriis, "tju!" tiel, ke la
malgrand'ruso veki\^gis kaj \^gustatempe ankora\u u povis eviti la
dan\^geron, kiu lin minacis.

   En sekvo de tiu \^ci okazo la diablo, farante al li fortan predikon
pro lia maltrafo, admonis lin kaj donis al li la klarigon, ke ne la
edzino, kiel li ja tuj vidis sur si mem, estas la plej bona amiko de
la viro, sed la hundo. La edzinon oni povas malsa\^gigi, delogi;
\^si ofte forlasas la edzon, dum la hundo restas fidela al sia
mastro, neniam lin forlasas, dividas kun li \^ciujn dan\^gerojn,
\^ciun mizeron kaj, kiam mortas \^gia mastro, \^gi pli sincere
mal\^gojas je li, ol kiu ajn el liaj parencoj.

 --- Tion \^ci, diris la diablo, mi komunikas al vi, por turni vian
atenton sur viajn efektivajn, verajn amikojn, el dankemeco por la
helpo, kiun vi alportis al mi.

\smallrule{}
