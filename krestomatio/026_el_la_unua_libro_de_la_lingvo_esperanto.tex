
   La nun proponatan bro\^suron la leganto kredeble prenos en la manojn
kun malkonfido, kun anta\u ue preta penso, ke al li estos proponata
ia neefektivigebla utopio; mi devas tial anta\u u \^cio peti la
leganton, ke li formetu tiun \^ci anta\u uju\^gon kaj ke li pripensu
serioze kaj kritike la proponatan aferon.

   Mi ne parolos tie \^ci vaste pri tio, kian grandegan signifon havus
por la homaro la enkonduko de unu komune akceptita lingvo
internacia, kiu prezentus egalrajtan propra\^{\j}on de la tuta
mondo, apartenante speciale al neniu el la ekzistantaj nacioj. Kiom
da tempo kaj laboroj estas perdata por la ellernado de fremdaj
lingvoj, kaj malgra\u u \^cio, elveturante el la limoj de nia
patrujo, ni ordinare ne havas la eblon kompreni\^gadi kun similaj al
ni homoj. Kiom da tempo, laboroj kaj materialaj rimedoj estas
perdata por tio, ke la produktoj de unu literaturo estu aligitaj al
\^ciuj aliaj literaturoj, kaj en la fino \^ciu el ni povas per
tradukoj konati\^gi nur kun la plej sensignifa parto de fremdaj
literaturoj; sed \^ce ekzistado de lingvo internacia \^ciuj tradukoj
estus farataj nur en tiun \^ci lastan, kiel ne\u utralan, al \^ciuj
kompreneblan, kaj la verkoj, kiuj havas karakteron internacian,
estus eble skribataj rekte en \^gi. Falus la \^hinaj muroj inter la
homaj literaturoj; la literaturaj produktoj de aliaj popoloj
fari\^gus por ni tiel same atingeblaj, kiel la verkoj de nia propra
popolo; la legata\^{\j}o fari\^gus komuna por \^ciuj homoj, kaj kune
kun \^gi anka\u u la edukado, idealoj, konvinkoj, celado, --- kaj la
popoloj interproksimi\^gus kiel unu familio. Devigataj dividi nian
tempon inter diversaj lingvoj, ni ne havas la eblon dece fordoni nin
e\^c al unu el ili, kaj tial de unu flanko tre malofte iu el ni
posedas perfekte e\^c sian patran lingvon, kaj de la dua flanko la
lingvoj mem ne povas dece ellabori\^gi, kaj, parolante en nia patra
lingvo, ni ofte estas devigataj a\u u preni vortojn kaj esprimojn de
fremdaj popoloj, a\u u esprimi nin neprecize kaj e\^c pensi lame
dank' al la nesufi\^ceco de la lingvo. Alia afero estus, se \^ciu el
ni havus nur du lingvojn, --- tiam ni pli bone ilin posedus kaj tiuj
\^ci lingvoj mem povus pli ellabori\^gadi kaj perfekti\^gadi kaj
starus multe pli alte, ol \^ciu el ili staras nun. Kaj la lingvo ja
estas la \^cefa motoro de la civilizacio: dank' al la lingvo ni tiel
alti\^gis super la bestoj, kaj ju pli alte staras la lingvo, des pli
rapide progresas la popolo. La diferenco de la lingvoj prezentas la
esencon de la diferenco kaj reciproka malamikeco de la nacioj, \^car
tio \^ci anta\u u \^cio falas en la okulojn \^ce renkonto de homoj:
la homoj ne komprenas unu la alian kaj tial ili tenas sin fremde unu
kontra\u u la alia. Renkonti\^gante kun homoj, ni ne demandas, kiajn
politikajn konvinkojn ili havas, sur kiu parto de la tera globo ili
naski\^gis, kie lo\^gis iliaj prapatroj anta\u u kelke da miljaroj:
sed tiuj \^ci homoj ekparolas, kaj \^ciu sono de ilia parolo
memorigas nin, ke ili estas fremdaj por ni. Kiu unu fojon provis
lo\^gi en urbo, en kiu logas homoj de diversaj reciproke batalantaj
nacioj, tiu eksentis sendube, kian grandegan utilon alportus al la
homaro lingvo internacia, kiu, {\sl ne entrudi\^gante en la doman
vivon de la popoloj}, povus, almena\u u en landoj kun diverslingva
lo\^gantaro, esti lingvo regna kaj societa. Kian, fine, grandegan
signifon lingvo internacia havus por la scienco, komerco --- per unu
vorto, sur \^ciu pa\^so --- pri tio mi ne bezonas vaste paroli. Kiu
almena\u u unu fojon serioze ekmeditis pri tiu \^ci demando, tiu
konsentos, ke nenia ofero estus tro granda, se ni povus per \^gi
akiri al ni lingvon komunehoman. Tial \^ciu e\^c la plej malforta
provo en tiu \^ci direkto meritas atenton. Al la afero, kiun mi nun
proponas al la leganta publiko, mi oferis miajn plej bonajn arojn;
mi esperas, ke anka\u u la leganto, pro la graveco de la afero,
volonte oferos al \^gi iom da pacienco kaj atente tralegos la nun
proponatan bro\^suron \^gis la fino.

   Mi ne analizos tie \^ci la diversajn provojn, faritajn kun la celo
krei lingvon internacian. Mi turnos nur la atenton de la legantoj al
tio, ke \^ciuj tiuj \^ci provoj a\u u prezentis per si sistemon da
signoj por mallonga interkomuniki\^go en okazo de granda bezono, a\u
u kontenti\^gis je plej natura simpligo de la gramatiko kaj je
anstata\u uigo de la vortoj ekzistantaj en la lingvoj per vortoj
aliaj, arbitre elpensitaj. La provoj de la unua kategorio estis tiel
komplikitaj kaj tiel nepraktikaj, ke \^ciu el ili mortis tuj post la
naski\^go; la provoj de la dua kategorio jam prezentis per si {\sl
lingvojn}, sed da {\sl internacia} ili havis en si {\sl nenion}. La
a\u utoroj ial nomis siajn lingvojn "tutmondaj", eble nur pro tio,
ke en la tuta mondo estis neniu persono, kun kiu oni povus
kompreni\^gi per tiuj \^ci lingvoj! Se por la tutmondeco de ia
lingvo estas sufi\^ce, ke unu persono \^gin nomu tia, en tia okazo
\^ciu el la ekzistantaj lingvoj povas fari\^gi tutmonda la\u u la
deziro de \^ciu aparta persono. \^Car tiuj \^ci provoj estis
fonditaj sur la naiva espero, ke la mondo renkontos ilin kun \^gojo
kaj unuanime donos al ili sankcion, kaj tiu \^ci unuanima konsento
\^guste estas la plej neebla parto de la afero, pro la natura
indiferenteco de la mondo por kabinetaj provoj, kiuj ne alportas al
\^gi senkondi\^can utilon, sed kalkulas je \^gia preteco pionire
oferi sian tempon, --- tial estas kompreneble, kial tiuj \^ci provoj
renkontis plenan fiaskon; \^car la plej granda parto de la mondo
tute ne interesis sin je tiuj \^ci provoj, kaj tiuj, kiuj sin
interesis, konsideris, ke ne estas inde perdi tempon por la lernado
de lingvo, en kiu neniu nin komprenos krom la a\u utoro; "anta\u ue
la mondo", ili diris, "a\u u kelkaj milionoj da homoj ellernu tiun
\^ci lingvon, tiam mi anka\u u \^gin lernos". Kaj la afero, kiu
povus alporti utilon al \^ciu aparta adepto nur tiam, se anta\u ue
jam ekzistus multego da aliaj adeptoj, trovis nenian akceptanton kaj
montri\^gis malvive naskita. Kaj se unu el la lastaj provoj,
"Volap\"uk", akiris, kiel oni diras, certan nombron da adeptoj,
tio \^ci estas nur tial, ke la ideo mem de lingvo "tutmonda" estas
tiel alta kaj alloga, ke homoj, kiuj havas la inklinon
entuziasmi\^gi kaj dedi\^ci sin al pionireco, oferas sian tempon en
la espero, ke {\sl eble} la afero sukcesos. Sed la nombro de la
entuziasmuloj atingos certan ciferon kaj haltos, kaj la malvarma
indiferenta mondo ne volos oferi sian tempon por tio, ke \^gi povu
komuniki\^gadi kun tiuj \^ci nemultaj,
--- kaj tiu \^ci lingvo, simile al la anta\u uaj provoj, mortos,
alportinte absolute nenian utilon.\footnote{Tiuj \^ci vortoj estis
skribitaj en la komenco de la jaro 1887, kiam Volap\"uk havis en la
tuta mondo grandegan gloron kaj rapidege progresadis! La tempo
balda\u u montris, ke la anta\u udiro de la a\u utoro de Esperanto
ne estis erara.}

   La demando pri lingvo internacia okupadis min jam longe; sed sentante
min nek pli talenta, nek pli energia, ol la a\u utoroj de \^ciuj
senfrukte pereintaj provoj, mi longan tempon limigadis min nur per
revado kaj nevola meditado super tiu \^ci afero. Sed kelke da
feli\^caj ideoj, kiuj aperis kiel frukto de tiu \^ci nevola
meditado, kura\^gigis min por plua laborado kaj igis min ekprovi,
\^cu ne prosperos al mi sisteme venki \^ciujn barojn por la kreo kaj
enkonduko en uzadon de racia lingvo internacia.

   \^Sajnas al mi, ke tiu \^ci afero iom prosperis al mi, kaj tiun \^ci
frukton de longatempaj persistaj laboroj mi proponas nun al la
priju\^go de la leganta mondo.

   La plej \^cefaj problemoj, kiujn estis necese solvi, estis la
sekvantaj:

   I) Ke la lingvo estu eksterordinare facila, tiel ke oni povu ellerni
\^gin ludante.

   II) Ke \^ciu, kiu ellernis tiun \^ci lingvon, povu tuj \^gin uzi por
la kompreni\^gado kun homoj de diversaj nacioj, tute egale \^cu tiu
\^ci lingvo estos akceptita de la mondo kaj trovos multe da adeptoj
a\u u ne, --- t.e. ke la lingvo jam de la komenco mem kaj dank' al
sia propra konstruo povu servi kiel efektiva rimedo por internaciaj
komuniki\^goj.

   III) Trovi rimedojn por venki la indiferentecon de la mondo kaj
igi \^gin kiel eble plej balda\u u kaj amase komenci uzadi la
proponatan lingvon kiel lingvon vivan, --- ne kun \^slosilo en la
manoj kaj en okazoj de ekstrema bezono.

   El \^ciuj projektoj, kiuj en diversaj tempoj estis proponitaj al
la mondo, ofte sub la la\u uta, per nenio pravigita nomo de "lingvo
tutmonda", neniu solvis pli ol {\sl unu} el la diritaj problemoj,
kaj e\^c tiun \^ci nur {\sl parte}. Krom la supre montritaj tri
\^cefaj problemoj, mi devis, kompreneble, solvi ankora\u u multajn
aliajn, sed pri ili, \^car ili estas ne esencaj, mi ne parolos tie
\^ci. Anta\u u ol mi transiros al la klarigo de tio, kiel mi solvis
la supre diritajn problemojn, mi devas peti la leganton mediti iom
pri la signifo de tiuj \^ci problemoj kaj ne preni tro facile miajn
rimedojn de solvo sole nur tial, \^car ili aperos al li eble kiel
tro simplaj. Mi petas tion \^ci tial, \^car mi scias la inklinon de
la plimulto da homoj rigardi aferon kun des pli da estimego, ju pli
\^gi estas komplikita, ampleksa kaj malfacile-digestebla. Tiaj
personoj, ekvidinte la malgrandegan lernolibron kun plej simplaj kaj
por \^ciu plej kompreneblaj reguloj, povas preni la aferon kun ia
malestima mal\^sato, dum \^guste la atingo de tiu \^ci simpleco kaj
mallongeco, la alkonduko de \^ciu objekto el la formoj komplikitaj,
el kiuj ili naski\^gis, al la formoj plej facilaj --- prezentis la
plej malfacilan parton de la laboro.

\begin{center}
\textbf{I}
\end{center}

   La unuan problemon mi solvis en la sekvanta maniero:

   \emph{a}) Mi simpligis \^gis nekredebleco la gramatikon, kaj al tio de unu
flanko en la spirito de la ekzistantaj vivaj lingvoj, por ke \^gi
povu facile eniri en la memoron, kaj de la dua flanko --- neniom
deprenante per tio \^ci de la lingvo la klarecon, precizecon kaj
flekseblecon. {\sl La tutan gramatikon de mia lingvo oni povas
bonege ellerni en la da\u uro de unu horo}. La grandega faciligo,
kiun la lingvo ricevas de tia gramatiko, estas klara por \^ciu.

   \emph{b}) Mi kreis regulojn por {\sl vortofarado} kaj per tio \^ci mi enportis
grandegan ekonomion rilate la nombron de la vortoj ellernotaj, ne
sole ne deprenante per tio \^ci de la lingvo \^gian ri\^cecon, sed
kontra\u ue, farante la lingvon --- dank' al la eblo krei el unu
vorto multajn aliajn kaj esprimi \^ciujn eblajn nuancojn de la penso
--- pli ri\^ca ol la plej ri\^caj naturaj lingvoj. Tion \^ci mi
atingis per la enkonduko de diversaj prefiksoj kaj sufiksoj, per
kies helpo \^ciu povas el unu vorto formi diversajn aliajn vortojn,
ne bezonante ilin lerni. (Pro oportuneco al tiuj \^ci prefiksoj kaj
sufiksoj estas donita la signifo de memstaraj vortoj, kaj kiel tiaj
ili estas lokitaj en la vortaro.) Ekzemple:

   1) La prefikso "mal" signifas rektan kontra\u ua\^{\j}on de la ideo;
sekve, sciante la vorton "bona", ni jam mem povas formi la vorton
"malbona", kaj la ekzistado de aparta vorto por la ideo
"malbona" estas jam superflua; alta --- malalta; estimi ---
malestimi k. t. p. Sekve, ellerninte unu vorton "mal", ni jam
estas liberigitaj de la lernado de grandega serio da vortoj, kiel
ekzemple "malmola" (sciante "mola"), malvarma, malnova, malpura,
malproksima, malri\^ca, mallumo, malhonoro, malsupre, malami,
malbeni k. t. p., k. t. p.

   2) La sufikso "in" signifas la virinan sekson: sekve, sciante
"frato", ni jam mem povas formi "fratino", patro --- patrino.
Sekve superfluaj jam estas la vortoj "avino, filino, fian\^cino,
knabino, kokino, bovino" k. t. p.

   3) La sufikso "il" signifas instrumenton por la donita farado.
Ekzemple tran\^ci --- tran\^cilo; superfluaj estas: "kombilo,
hakilo, sonorilo, plugilo, glitilo" k. t. p. Kaj similaj aliaj
prefiksoj kaj sufiksoj.

   Krom tio mi donis komunan regulon, ke \^ciuj vortoj, kiuj jam fari\^gis
internaciaj (la tiel nomataj "fremdaj vortoj"), restas en la
lingvo internacia ne\^san\^gataj, akceptante nur la internacian
ortografion; tiamaniere grandega nombro da vortoj fari\^gas
superfluaj por la lernado; ekzemple: lokomotivo, redakcio,
telegrafo, nervo, temperaturo, centro, formo, publiko, platino,
bo\-ta\-ni\-ko, figuro, vagono, komedio, ekspluati, deklami,
advokato, doktoro, teatro k. t. p., k. t. p.

   Dank' al la supre montritaj reguloj kaj ankora\u u al kelkaj flankoj de
la lingvo, pri kiuj mi trovas superflue tie \^ci detale paroli, la
lingvo fari\^gas eksterordinare facila, kaj la tuta laboro de \^gia
ellernado konsistas nur en la ellernado de tre malgranda nombro da
vortoj, el kiuj la\u u difinitaj reguloj, sen apartaj kapabloj kaj
stre\^cado de la kapo, oni povas formi \^ciujn vortojn, esprimojn
kaj frazojn, kiuj estas necesaj. Cetere e\^c tiu \^ci malgranda
nombro da vortoj, kiel oni vidos malsupre, estas tiel elektita, ke
ilia ellernado por homo iomete klera estas afero eksterordinare
facila. La ellernado de tiu \^ci lingvo sonora, ri\^ca kaj por
\^ciuj komprenebla (la ka\u uzojn vidu malsupre) postulas sekve ne
tutan serion da jaroj, kiel \^ce la aliaj lingvoj, --- por \^gia
ellernado sufi\^cas {\sl kelke da tagoj}. Pri tio \^ci \^ciu povas
konvinki\^gi, \^car al la nuna bro\^suro estas aldonita {\sl plena
lernolibro}.\footnote{En la originalo de la unua libro pri
Esperanto en la fino estis presita la tuta gramatiko kaj vortaro el
\^cirka\u u mil vortoj.}

\begin{center}
\textbf{II}
\end{center}

   La duan problemon mi solvis en la sekvanta maniero:

   \emph{a}) Mi aran\^gis plenan {\sl dismembrigon} de la ideoj en memstarajn
vortojn, tiel ke la tuta lingvo, anstata\u u vortoj en diversaj
gramatikaj formoj, konsistas sole nur el {\sl sen\^san\^gaj} vortoj.
Se vi prenos verkon, skribitan en mia lingvo, vi trovos, ke tie
\^ciu vorto sin trovas {\sl \^ciam} kaj {\sl sole} en unu konstanta
formo, nome en tiu formo, en kiu \^gi estas presita en la vortaro.
Kaj la diversaj formoj gramatikaj, la reciprokaj rilatoj inter la
vortoj k. t. p. estas esprimataj per la kunigo de sen\^san\^gaj
vortoj. Sed \^car simila konstruo de lingvo estas tute fremda por la
E\u uropaj popoloj kaj alkutimi\^gi al \^gi estus por ili afero
malfacila, tial mi tute alkonformigis tiun \^ci dismembri\^gon de la
lingvo al la spirito de la lingvoj E\u uropaj, tiel ke se iu lernas
mian lingvon la\u u lernolibro, ne traleginte anta\u ue la anta\u
uparolon (kiu por la lernanto estas tute senbezona), --- li e\^c ne
supozos, ke la konstruo de tiu \^ci lingvo per io diferencas de la
konstruo de lia patra lingvo. Tiel ekzemple la devenon de la vorto
"fratino", kiu en efektiveco konsistas el tri vortoj: {\sl frat}
(frato), {\sl in} (virino), {\sl o} (kio estas, ekzistas) (--- kio
estas frato-virino = fratino),
--- la lernolibro klarigas en la sekvanta maniero: frato = {\sl frat};
sed \^car \^ciu substantivo en la nominativo {\sl fini\^gas} per
"o" --- sekve {\sl frat'o}; por la formado de la sekso virina de
tiu sama ideo, oni enmetas la vorteton "in"; sekve fratino ---
{\sl frat'in'o}; kaj la signetoj estas skribataj tial, \^car la
gramatiko postulas ilian metadon inter la apartaj konsistaj partoj
de la vortoj. En tia maniero la dismembri\^go de la lingvo neniom
embarasas la lernanton; li e\^c ne suspektas, ke tio, kion li nomas
fini\^go a\u u prefikso a\u u sufikso, estas tute memstara vorto,
kiu \^ciam konservas egalan signifon, tute egale, \^cu \^gi estos
uzata en la fino a\u u en la komenco de alia vorto a\u u memstare,
ke \^ciu vorto kun egala rajto povas esti uzata kiel vorto radika
a\u u kiel gramatika parteto. Kaj tamen la rezultato de tiu \^ci
konstruo de la lingvo estas tia, ke \^cion, kion vi skribos en la
lingvo internacia, tuj kaj kun plena precizeco (per \^slosilo a\u u
e\^c sen \^gi) komprenos \^ciu, kiu ne sole ne ellernis anta\u ue la
gramatikon de la lingvo, sed e\^c neniam a\u udis pri \^gia
ekzistado.

   Mi klarigos tion \^ci per ekzemplo:

   Mi trovi\^gis en Rusujo, ne sciante e\^c unu vorton rusan; mi bezonas
turni min al iu, kaj mi skribas al li sur papereto en libera lingvo
internacia ekzemple la jenon:

   "Mi ne sci'as, kie mi las'is mi'a'n baston'o'n; \^cu vi \^gi'n ne
vid'is?"

   Mi proponas al mia interparolanto vortaron internacia-rusan kaj mi
montras al li la komencon, kie per grandaj literoj estas presita la
sekvanta frazo: {\sl \^Cion, kio estas skribita en la lingvo
internacia, oni povas kompreni per helpo de tiu \^ci vortaro.
Vortoj, kiuj prezentas kune unu ideon, estas skribataj kune, sed
dividataj unu de la alia per signeto; tiel ekzemple la vorto
frat'in'o, prezentante unu ideon, estas kunmetita el tri vortoj, el
kiuj \^ciun oni devas ser\^ci aparte}.

   Se mia interparolanto neniam a\u udis pri la lingvo internacia, li
komence rigardos min tre mirigite, sed li prenos mian papereton,
ser\^cos en la montrita maniero en la vortaro kaj trovos jenon:

{\small
\begin{center}
\begin{longtblr}[theme=plain,label=none]{lX@{}l@{}l}
 {\sl Mi} & Ja \Dotfill & & ja\\
 {\sl ne} & nje, njet \Dotfill & & nje\\
 {\sl sci} & znatj \Dotfill & \SetCell[r=2]{c} {\scalebox{2}[2.5]\}\ } & \SetCell[r=2]{l} znaju \\
 {\sl as} & ozna\^cajet nastoja\^s\^ceje vremja glagola \Dotfill & &  \\
 {\sl kie} & gdje \Dotfill & & gdje\\
 {\sl mi} & ja \Dotfill & & ja\\
 {\sl las} & ostavljatj \Dotfill & \SetCell[r=2]{c} {\scalebox{2}[2.5]\}\ } & \SetCell[r=2]{l} ostavil \\
 {\sl is} & ozna\^cajet pro\^sed\^seje vremja \Dotfill & & \\
 {\sl mi} & ja \Dotfill & \SetCell[r=3]{c} {\scalebox{2}[3.5]\}\ } & \SetCell[r=3]{l} (moj) moju\\
 {\sl a} & ozna\^cajet prilagatelnoje \Dotfill & & \\
 {\sl n} & ozna\^cajet vinitelnij pade\^{\j} \Dotfill & &\\
 {\sl baston} & palka \Dotfill & \SetCell[r=3]{c} {\scalebox{2}[3.5]\}\ } & \SetCell[r=3]{l} palku \\
 {\sl o} & ozna\^cajet su\^s\^cestvitelnoje \Dotfill & & \\
 {\sl n} & ozna\^cajet vinitelnij pade\^{\j} \Dotfill & & \\
 {\sl \^cu} & li \Dotfill & & li\\
 {\sl vi} & vi, ti \Dotfill & & vi\\
 {\sl \^gi} & ono \Dotfill & \SetCell[r=2]{c} {\scalebox{2}[2.5]\}\ }  & \SetCell[r=2]{l} (jego) jejo \\
 {\sl n} & ozna\^cajet vinitelnij pade\^{\j} \Dotfill & & \\
 {\sl ne} & nje \Dotfill & & nje\\
 {\sl vid} & vidjetj \Dotfill & \SetCell[r=2]{c} {\scalebox{2}[2.5]\}\ } & \SetCell[r=2]{l} vidjel (i)\\
 {\sl is} & ozna\^cajet pro\^sed\^seje vremja \Dotfill & & \\
\end{longtblr}
\end{center}}

   En tia maniero la ruso klare komprenos, kion mi de li deziras. Se li
volos respondi al mi, mi montras al li la vortaron rusa-internacian,
en kies komenco estas presite: {\sl Se vi deziras esprimi ion en la
lingvo internacia, uzu tiun \^ci vortaron, ser\^cante la vortojn en
la vortaro mem kaj la fini\^gojn por la gramatikaj formoj en la
gramatika aldono, en la paragrafo de la responda parto de parolo}.
\^Car en tiu aldono, kiel oni vidas en la lernolibro, la plena
gramatiko de \^ciu parto de parolo okupas ne pli ol kelke da linioj,
tial la trovado de la fini\^go por la esprimo de responda gramatika
formo okupos ne pli da tempo, ol la trovado de vorto en la vortaro.

   Mi turnas la atenton de la leganto al la klarigita punkto, kiu
\^sajne estas tre simpla, sed havas grandegan praktikan signifon.
Estas superflue diri, ke en alia lingvo vi kun persono, ne posedanta
tiun \^ci lingvon, ne havas la eblon kompreni\^gadi e\^c per la
helpo de la plej bona vortaro, \^car, por povi fari uzon el la
vortaro de ia el la ekzistantaj lingvoj, oni devas anta\u ue pli a\u
u malpli scii tiun \^ci lingvon. Por povi trovi en la vortaro la
donitan vorton, oni devas scii \^gian fundamentan formon, dum en la
interligita parolado \^ciu vorto ordinare estas uzita en ia
gramatika \^san\^go, kiu ofte estas neniom simila je la fundamenta
formo, en kunigo kun diversaj prefiksoj, sufiksoj k. t. p.; tial, se
vi anta\u ue ne konas la lingvon, vi preska\u u neniun vorton trovos
en la vortaro, kaj e\^c tiuj vortoj, kiujn vi trovos, donos al vi
nenian komprenon pri la signifo de la frazo. Tiel ekzemple, se mi la
supre donitan frazon skribus germane (Ich weiss nicht wo ich meinen
Stock gelassen habe; haben sie ihn nicht gesehen), tiam persono, ne
scianta la lingvon germanan, trovos en la vortaro la jenon:

   "Mi --- blanka --- ne --- kie --- mi --- pensi --- bastono a\u u eta\^go
--- kvieta --- havo --- havi --- \^si --- ? --- ne --- ? --- ".

   Mi ne parolas jam pri tio, ke la vortaro de \^ciu el la ekzistantaj
lingvoj estas treege vasta kaj ser\^ci en \^gi 2-3 vortojn unu post
alia jam lacigas, dum la vortaro internacia, dank' al la dismembra
konstruo de la lingvo, estas tre malgranda kaj oportuna; mi ne
parolas jam anka\u u pri tio, ke en \^ciu lingvo \^ciu vorto havas
en la vortaro multe da signifoj, el kiuj oni devas divenprove elekti
la \^gustan. Kaj se vi e\^c imagos al vi lingvon kun la plej ideala
simpligita gramatiko, kun konstanta difinita signifo por \^ciu
vorto, --- en \^ciu okazo, por ke la adresito per helpo de la
vortaro komprenu vian skribon, estus necese, ke li anta\u ue ne sole
ellernu la gramatikon, sed ke li anka\u u akiru en \^gi sufi\^can
spertecon, por facile helpi al si, distingi vorton radikan de vorto
gramatike \^san\^gita, devena a\u u kunmetita k. t. p., --- t. e. la
utilo de la lingvo denove dependus de la nombro da adeptoj, kaj \^ce
manko de la lastaj \^gi prezentus nulon. \^Car, sidante ekzemple en
vagono kaj dezirante demandi vian najbaron, "kiel longe ni atendos
en N.", vi ja ne proponos al li anta\u ue ellerni la gramatikon de
la lingvo! Sed en la lingvo internacia vi povas esti tuj komprenita
de membro de \^cia nacio, se li ne sole ne posedas tiun \^ci
lingvon, sed e\^c neniam a\u udis pri \^gi. \^Ciun libron, verkitan
en la lingvo internacia, libere povas, kun vortaro en la mano, legi
\^ciu, sen ia anta\u uprepari\^go kaj e\^c sen bezono anta\u ue
tralegi ian anta\u uparolon, klarigantan la uzadon de la vortaro;
kaj homo klera, kiel oni vidos malsupre, e\^c la vortaron devas uzi
tre malmulte.

   Se vi deziras skribi, ekzemple, al iu hispano Madridon, sed nek vi
scias lian lingvon, nek li vian, kaj vi dubas, \^cu li scias la
lingvon internacian a\u u \^cu li e\^c a\u udis pri \^gi, --- vi
povas tamen kura\^ge skribi al li, kun la plena certeco, ke li vin
komprenos! \^Car, dank' al la dismembra konstruo de la lingvo
internacia, la tuta vortaro, kiu estas necesa por la ordinara vivo,
okupas, kiel oni vidas el la almetita ekzemplero, ne pli ol
malgrandan folieton, eniras oportune en la plej malgrandan koverton
kaj povas esti ricevita por kelke da centimoj en kia ajn lingvo,
--- tial vi bezonas nur skribi leteron en la lingvo internacia,
enmeti en la leteron hispanan ekzempleron de la vortareto, --- kaj
la adresito vin jam komprenos, \^car tiu \^ci vortareto ne sole
prezentas oportunan plenan \^slosilon por la letero, sed \^gi mem
jam klarigas sian difinon kaj manieron de uzado. Dank' al la plej
vasta reciproka kunigebleco de la vortoj, oni povas per helpo de tiu
\^ci malgranda vortaro esprimi \^cion, kio estas necesa en la
ordinara vivo; sed, kompreneble, vortoj renkontataj malofte, vortoj
te\^hnikaj (kaj anka\u u vortoj "fremdaj", supozeble konataj al
\^ciuj, ekzemple "tabako", "teatro", "fabriko" k. t. p.) en
\^gi ne estas troveblaj; tial se vi bezonas nepre uzi tiajn vortojn
kaj anstata\u uigi ilin per aliaj vortoj a\u u tutaj esprimoj estas
neeble, tiam vi devos jam uzi vortaron {\sl plenan}, kiun vi tamen
ne bezonas transsendi al la adresato: vi povas nur apud la diritaj
vortoj skribi en krampoj ilian tradukon en la lingvon de la
adresato.

   b) Sekve, dank' al la supre montrita konstruo de la lingvo, mi povas
kompreni\^gadi per \^gi kun kiu mi volas. La sola maloportuneco
(\^gis la komuna enkonduko de la lingvo) estos nur tio, ke mi
bezonos \^ciufoje atendi, \^gis mia interparolanto analizos miajn
pensojn. Por forigi kiom eble anka\u u tiun \^ci maloportunecon
(almena\u u \^ce komuniki\^gado kun homoj kleraj), mi agis en la
sekvanta maniero: la vortaron mi kreis ne arbitre, sed kiom eble el
vortoj konataj al la tuta klera mondo. Tiel ekzemple la vortojn,
kiuj estas egale uzataj en \^ciuj civilizitaj lingvoj (la tiel
nomatajn "fremdajn" kaj "te\^hnikajn"), mi lasis tute sen ia
\^san\^go; el la vortoj, kiuj en malsamaj lingvoj sonas malegale, mi
prenis a\u u tiujn, kiuj estas komunaj al du tri plej \^cefaj E\u
uropaj lingvoj, a\u u tiujn, kiuj apartenas nur al unu lingvo, sed
estas popularaj anka\u u \^ce la aliaj popoloj; en tiuj okazoj, kiam
la donita vorto en \^ciu lingvo sonas alie, mi penis trovi vorton,
kiu havus eble nur signifon proksimuman a\u u uzon pli maloftan, sed
estus konata al la plej \^cefaj nacioj (ekzemple la vorto
"proksima" en \^ciu lingvo sonas alie; sed se ni prenos la latinan
"plej proksima" ({\sl proximus}), tiam ni vidos, ke \^gi, en
diversaj \^san\^goj, estas uzata en \^ciuj plej \^cefaj lingvoj;
sekve se mi la vorton "proksima" nomos {\sl proksim}, mi estos pli
a\u u malpli komprenata de \^ciu klera homo); en la ceteraj okazoj
mi prenadis ordinare el la lingvo latina, kiel lingvo
duone-internacia. (Mi flanki\^gadis de tiuj \^ci reguloj nur tie,
kie tion \^ci postulis apartaj cirkonstancoj, kiel ekzemple la evito
de homonimoj, la simpleco de la ortografio k. t. p.). Tiamaniere
\^ce korespondado kun meze-klera E\u uropano, kiu tute ne lernis la
lingvon internacian, mi povas esti certa, ke li ne sole min
komprenos, sed e\^c sen bezono tro multe ser\^cadi en la vortaro,
kiun li uzos nur \^ce vortoj dubaj.

\smallrule{}

