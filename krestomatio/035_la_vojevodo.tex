\begin{verse}
\begin{center}
\footnotesize (El \fsc{Mickiewicz.})
\end{center}

                  En vespero somera vojevodo kolera\\
                  Al la hejma kastelo rapidas;\\
                  Al la lito edzina kun teruro senfina\\
                  Li alvenas, --- neniun li vidas.

                  En doloro brulanta kaj per mano tremanta\\
                  Sian grizan lipharon li prenis,\\
                  De la lito foriris, la manikojn retiris\\
                  Kaj ektondris --- kozako alvenis.

                  "Kial, best' abomena, mia pordo \^gardena\\
                  Restas nokte sen hundo, sen gardo?\\
                  Prenu tuj mian sakon kaj pafilon kozakan\\
                  Kaj silente min sekvu, bastardo!"

                  Kun pafil' en la mano al \^gardena altano\\
                  Ili amba\u u sen bru' al\^steli\^gas.\\
                  Sur la benka herba\^{\j}o ia lumas
                  blanka\^{\j}o:\\
                  En tola\^{\j}o virino vidi\^gas.

                  Unu manon levinte, la okulojn kovrinte,\\
                  Per la dua forpu\^si \^si penis\\
                  Unu viron petantan, surgenue starantan,\\
                  Kiu nun en la brakoj \^sin tenis.

                  Kaj en flama fervoro li parolis: "Ho, koro,\\
                  \^Cu jam \^cio por \^ciam perdita?\\
                  E\^c la premoj de l' mano per la mon' de l'
                  tirano\\
                  \^Ciuj estas jam fora\^cetitaj?

                  Mi vin tiel amadis, por vi tiel bruladis, ---\\
                  Malproksime nun plori mi devas;\\
                  Li ne amis, ne ploris, nur per mon' eksonoris, ---\\
                  Kaj li \^cion por \^ciam ricevas.

                  Sur la brusto an\^gela lia kapo malbela\\
                  En dorloto de nun ripozados,\\
                  De la roza bu\^seto, de la ru\^ga vangeto,\\
                  Li \^cielan feli\^con su\^cados.

                  Sur \^cevalo fidela, en vetero kruela,\\
                  Mi rapidas al mia an\^gelo ---\\
                  Por sopire foriri kaj al vi nur deziri\\
                  Bonan nokton en lia kastelo\dots"

                  \^Si silentas senmove; li komencas denove\\
                  Sian plendon kun petoj kaj ploro,\\
                  \^Gis, la brakojn lasinte, la konscion perdinte,\\
                  \^Si defalis al li al la koro.

                  En l'arba\^{\j}o silente, a\u uskultante atente,\\
                  Staras amba\u u gardantoj kovritaj,\\
                  Ili staras genue, en la manoj senbrue\\
                  La pafiloj ektremis \^sargitaj.

                  "Estro"! diris kozako, "ia stranga atako\\
                  Al mi ligas subite la manon;\\
                  Brulan sentis mi larmon kaj skuantan malvarmon,\\
                  Kiam tu\^si mi volis la \^canon."

                  "Pesto"! mi vin jam skuos, mi vin plori
                  instruos!\\
                  Jen saketo kun pulvo: sen vorto\\
                  Vi preparos, vi pafos, kaj se \^sin vi ne
                  trafos.\\
                  Mi edzigos vin mem kun la morto.

                  "Supren, dekstren, senskue! Mi ekpafos, ---
                  l'unue\\
                  De l' amanto diskrevu la koro."\\
                  La servanto ektiris, kaj la kuglo eniris\\
                  En la frunton de\dots lia sinjoro.

%L. L. ZAMENHOF.
\end{verse}
\citsc{L. Zamenhof.}

\smallrule{}

