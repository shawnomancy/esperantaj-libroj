\begin{verse}
\begin{center}
\footnotesize (El T. \fsc{Moore}.)
\end{center}
            Sonoriloj de vespero, sonoriloj de vespero!\\
            Kiom ili rakontadas pri juneco kaj espero,\\
            Pri la domo de gepatroj, pri la dol\^ca kora \^gojo,\\
            Kiam mi ilian sonon a\u udis je la lasta fojo!

            Longe, longe jam forpasis tiuj de l' \^gojeco horoj!\\
            Ekdorminte je eterne jam ne batas multaj koroj.\\
            En la tomboj ili lo\^gas post la \^gojo kaj sufero:\\
            Ne por ili la muziko, sonoriloj de vespero!

            Anka\u u kiam mi ripozos je eterne en trankvilo,\\
            Ne ekhaltos la batado de vespera sonorilo,\\
            Dum kun kanto novaj bardoj pa\^sos jam sur nia tero\\
            Kaj vin a\u udos kaj vin la\u udos, sonoriloj de vespero!

%A. GRABOWSKI.
\end{verse}

\citsc{A. Grabowski.}

\smallrule{}
