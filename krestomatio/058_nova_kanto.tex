\begin{verse}
                        Ho! eksonu nova kanto\\
                        Pri la lingvo Esperanto,\\
                        Pri ligil' internacia,\\
                        Revo nia, amo nia!\\
                        Kreita\^{\j}' la plej mirinda,\\
                        Vere estas \^gi la\u udinda\\
                        De verkistoj, de poetoj,\\
                        En poemoj kaj odetoj,\\
                        Pli ol tondro de bataloj,\\
                        Pli ol dol\^caj najtingaloj,\\
                        Pli ol belaj aktorinoj,\\
                        Pli ol fajfoj de ma\^sinoj,\\
                        Pli ol oro la plej brila,\\
                        Pli ol gloro senutila,\\
                        Pli ol \^ciu, pli ol \^cio,\\
                        Krom la amo kaj la Dio.\\
                        Dum venonta la centjaro,\\
                        Sciu esperantistaro,\\
                        En E\u uropo, Ameriko,\\
                        En Azio kaj Afriko,\\
                        Kie ajn vi veturados\\
                        Esperanton vi trovados,\\
                        Sur la strato, en vagono,\\
                        En hotelo, en salono\\
                        Kaj e\^c en privata domo:\\
                        \^Gin parolos \^ciu homo:\\
                        Laboristo, profesoro,\\
                        Kaj ju\^gisto, kaj doktoro,\\
                        Kaj hebreo, kaj kristano,\\
                        Kaj litovo kaj japano, ---\\
                        Kaj pereos la plendato:\\
                        "Mi vin ne komprenas, frato".\\
                        Por ke venu tiu horo ---\\
                        Kune fratoj, al laboro!\\
                        Jen per kanto, jen per vorto,\\
                        Jen agante \^gis la morto,\\
                        Servu ni al la afero,\\
                        La plej bela sur la tero,\\
                        Gardu \^gin de la forgeso\\
                        Per parolo en la preso;\\
                        Iru kiel apostoloj\\
                        \^Gin prediki por popoloj,\\
                        Kaj eksonu nia voko\\
                        Sur la ter' en \^ciu loko,\\
                        En vila\^goj, en urbetoj,\\
                        En lernejoj, en gazetoj,\\
                        Kaj servantajn al la vero\\
                        Nin fortigu la Espero.

%A. DOMBROWSKI.
\end{verse}

\citsc{A. Dombrowski.}

\smallrule{}

