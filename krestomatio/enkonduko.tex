%
% Enkonduko por la Fundamenta Krestomatio
%

% ni parolas esperanton ...
%
\usepackage[esperanto]{babel} 

% ... kaj la libro estis presita de francoj
%
\usepackage{csquotes}
\DeclareQuoteStyle{esperanto}
  {\guillemotleft}{\guillemotright}
  {\guillemotleft}{\guillemotright}
\MakeOuterQuote{"}
\setquotestyle{esperanto}

% figuroj kaj matematiko (en la sunhorloĝa artikolo)
%
\usepackage{tikz}
\usetikzlibrary{shapes,arrows.meta}
\usepackage[leqno]{amsmath}

% verso
%
\usepackage{verse}

% litera spaco
%
\usepackage{soul}
\sodef\spaceout{}{.2em}{0.6em}{0pt}
\sodef\spaceoutmed{}{.1em}{0.5em}{0pt}
\newcommand{\narrow}[1]{\scalebox{0.8}[1]{#1}}

% korespondantoj en "korespondado komerca"
%
\newcommand\koresp[1]{
\begin{flushleft}
\hspace{2em} #1
\end{flushleft}
}

% titoloj 
%
\usepackage{titlesec}
\renewcommand{\thechapter}{\Roman{chapter}}

% memoru:
% titleformat{command}[shape]{format}{label}{sep}{before-code}{after-code}
%
\titleformat{\chapter}[display]{\filcenter\Large}{\thechapter}{1ex}{\MakeUppercase}[{\rule{13mm}{0.4pt}}]

\titleformat{\section}[display]{\filcenter\large}{}{0pt}{\MakeUppercase}
\titlespacing{\section}{0pt}{1em}{1em}

\titleformat{\subsection}[display]{\filcenter\large}{}{0pt}{\MakeUppercase}

% papergeometrio
%
\usepackage{geometry}
\geometry{a5paper,top=21.5mm,bottom=20mm,left=21.5mm,right=21.5mm,}

%
% Creative Commons ikonoj
%
\usepackage{ccicons}

% datoprezento en mia komento
%
\def\hodiau{\number\day a~de~\ifcase\month\or januaro\or februaro\or marto\or aprilo\or majo\or junio\or julio\or aŭgusto\or septembro\or oktobro\or novembro\or decembro\fi, \number\year}

% la krommarĝeno por la unua lineo de ĉiu alineo devus esti kiel la aliaj
%
\usepackage{indentfirst} 

% ligiloj, kaj ilia koloro 
%
\usepackage{color}
\definecolor{verde_escuro}{rgb}{0,0.5,0}
\usepackage[colorlinks,linkcolor=verde_escuro]{hyperref}

% la xelatex-simbolo en mia komento
%
\usepackage{dtk-logos}

% por la belaj tiparoj
%
\usepackage{fontspec}

% de Google Fonts
%
\setmainfont{Old Standard TT}
\newfontfamily\nicefont{Cardo}

% de CTAN
%
\newfontfamily\csfont{TeX Gyre Schola}

% kutimaj
\newfontfamily\sansfont{Liberation Sans}[LetterSpace=5]

% fancy page headers, that is, in imitation of the original
%
\usepackage{fancyhdr}
\pagestyle{fancy}
\fancyhf{}
\renewcommand\headrule{}
\renewcommand\footrule{}

% citations
%
\newcommand\cited[1]{%
\begin{flushright}
\footnotesize #1
\end{flushright}}

\newcommand\citsc[1]{\cited{\fsc{#1}}}

% Big delimiters in the demo sentence table
\usepackage{multirow}

% asterisms
%
%
%     *
%   *   *
\newcommand{\aster}{*} % maybe \ding{97} ?
\newcommand\asterism{
   \begin{center}
     \vspace{-1em}
     \parbox{1in}{ % needed to prevent split across page boundaries
       \begin{center}
         \aster\\
         \aster\hspace{0.6em}\aster
       \end{center}
     } %\parbox
     \vspace{-1em}
   \end{center}
 }

%
% reset footnote counter per page 
%
\usepackage{perpage} 
\MakePerPage{footnote} 

% parolanto, en Hamleto
%
\usepackage{ifthen}
\newcommand\speak[2][]{\makebox[0.7\textwidth]{\footnotesize {\ifthenelse{\equal{#1}{}}{\fsc{#2}}{#2}}}

\vspace*{-0.9em}

}

% spaco en Hamleto
%
\newlength{\xxx}
\newcommand\psp[1]{\settowidth{\xxx}{#1}\hspace{\xxx}}

% teatra direkto
%
\newcommand\stg[1]{\makebox[0.8\textwidth][r]{\footnotesize #1}}

% trompaj malgrandaj majuskloj -- bedaŭrinde, Old Standard TT ne havas la verajn :(
%
\makeatletter
\newlength\fake@f
\newlength\fake@c
\def\fakesc#1{%
  \begingroup%
  \xdef\fake@name{\csname\curr@fontshape/\f@size\endcsname}%
  \fontsize{\fontdimen8\fake@name}{\baselineskip}\selectfont%
  \uppercase{#1}%
  \endgroup%
}
\makeatother
\newcommand\fsc[1]{\fauxschelper#1 \relax\relax}
\def\fauxschelper#1 #2\relax{%
  \fauxschelphelp#1\relax\relax%
  \if\relax#2\relax\else\ \fauxschelper#2\relax\fi%
}
\def\Hscale{.83}\def\Vscale{.72}\def\Cscale{1.00}
\def\fauxschelphelp#1#2\relax{%
  \ifnum`#1>``\ifnum`#1<`\{\scalebox{\Hscale}[\Vscale]{\uppercase{#1}}\else%
    \scalebox{\Cscale}[1]{#1}\fi\else\scalebox{\Cscale}[1]{#1}\fi%
  \ifx\relax#2\relax\else\fauxschelphelp#2\relax\fi}

% forigu la paĝnumero el la unua paĝo de ĉapitro
%
\fancypagestyle{plain}{%
  \renewcommand{\headrulewidth}{0pt}%
  \fancyhf{}%
}

% tabloj
%
\usepackage{longtable}
\usepackage{tabu}
\usepackage{siunitx}

% malgranda streko
%
\newcommand{\smallrule}{%
\begin{center}%
\rule{13mm}{0.4pt}\end{center}}

% La paĝtitoloj ŝanĝiĝas per ĉapitro, ne per sekcio aŭ subsekcio; do, 
% malpermesu al la (sub)sekcioj, ŝanĝi la paĝtitoloj
%
\renewcommand{\sectionmark}[1]{} 
\renewcommand{\subsectionmark}[1]{}

%
% Fino de Enkonduko
%
