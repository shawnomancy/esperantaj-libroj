\begin{center}
{\footnotesize (Tradukita de L. \fsc{Zamenhof}.)}\\[1ex]

\spaceout{\textbf{AKTO I}}

\rule[0.5ex]{5mm}{0.4pt}\\[1ex]

{\large SCENO I}\\[1ex]

\footnotesize Teraso anta\u u la palaco. Francisko staras sur la posteno; Bernardo venas.
\end{center}

\begin{verse}
\speak{Bernardo}
He! Kiu?\\
\speak{Francisko}
\psp{He! Kiu?} Halt'! Respondu: kiu iras? \\
\speak{Bernardo}
La reĝo vivu!\\
\speak{Francisko}
\psp{La reĝo vivu!} \^Cu Bernardo?\\
\speak{Bernardo}
\psp{La reĝo vivu! \^Cu Bernardo?} Jes!\\
\speak{Francisko}
 Vi akurate venis al la servo!\\
\speak{Bernardo}
 Dekdua horo sonis. Iru dormi.\\
\speak{Francisko}
 Mi dankas. La malvarmo estas tran\^ca,\\ 
 Kaj mi min sentas nun ne tute bone.\\
\speak{Bernardo}
 \^Cu \^cio estis orda kaj trankvila?\\
\speak{Francisko}
 E\^c mus' ne preterkuris.\\
\speak{Bernardo}
\psp{E\^c mus' ne preterkuris.} Bonan nokton!\\ Se vidos vi Marcellon, Horacion,\\ Kolegojn
miajn, rapidigu ilin!\\
\speak{Francisko}
 Jen, \^sajnas, ili. Haltu! Kiu iras?\\
\stg{(Horacio kaj Marcello venas.)}\\
\speak{Horacio}
 Amikoj de la lando.\\
\speak{Marcello}
\psp{Amikoj de la lando.} Kaj de l' reĝo.\\
\speak{Francisko}
 Nu, bonan nokton!\\
\speak{Marcello}
\psp{Nu, bonan nokton!} Bonan al vi dormon!\\ Vin kiu anstata\u uas nun?\\
\speak{Francisko}
\psp{Vin kiu anstata\u uas nun?} Bernardo.\\ Li gardas nun. Adia\u u! {\footnotesize (Foriras.)}\\
\speak{Marcello}
\psp{Li gardas nun. Adia\u u!} He! Bernardo!\\
\speak{Bernardo}
 \^Cu Horacio?\\
\speak{Horacio}
\psp{\^Cu Horacio?} Peco de li mem.\\
\speak{Bernardo}
 Salut' al vi, Marcello, Horacio!\\
\speak{Horacio}
 \^Cu la apero nun denove venis?\\
\speak{Bernardo}
 Nenion vidis mi.\\
 \speak{Marcello}
 \psp{Nenion vidis mi.} Jen Horacio\\ Parolas, ke \^gi estas nur imago,\\ Ne kredas li
pri la fantom' terura,\\ Kiun ni vidis jam la duan fojon.\\ Kaj tial
mi lin tien \^ci invitis,\\ Ke li kun ni maldormu nunan nokton\\
Kaj, kiam la fantomo reaperos,\\ Li vidu kaj kun la fantom' parolu.\\
\speak{Horacio}
 He, babila\^{\j}o! Kredu, \^gi ne venos.\\
\speak{Bernardo}
 Sidi\^gu do, kaj provos mi denove\\ Ataki per rakont' orelon
vian,\\ Ke \^gi for\^{\j}etu sian obstinecon\\ Kaj kredu, kion ni du
fojojn vidis.\\
\speak{Horacio}
 Nu, mi konsentas. Bone, ni sidi\^gu,\\ Kaj vi rakontu.\\
\speak{Bernardo}
\psp{Kaj vi rakontu.} En la lasta nokto,\\ En la momento, kiam tiu stelo\\ Tre luma,
kiun vidas vi, en sia\\ Kurado trans arka\^{\j}o la \^ciela\\ En tiu
sama loko, kiel nun,\\ Sin trovis, tiam ni --- mi kaj Marcello ---\\
Ekvidis\dots\\
\speak{Marcello}
\psp{Ekvidis\dots} Haltu! Jen \^gi mem aperas!\\
\stg{(Montras sin spirito en arma\^{\j}o.)}\\
\speak{Bernardo}
 Li mem, li mem, mortinta nia reĝo!\\
\speak[1]{\fsc{Marcello} (al Horacio).}
Vi estas instruita, Horacio,\\ Parolu vi kun li!\\
\speak{Bernardo}
\psp{Parolu vi kun li!} \^Cu ne simila\\ Li estas al la reĝo? Diru mem!\\
\speak{Horacio}
 Jes, jes! Li mem! Mi miras, timas, tremas!\\
\speak{Bernardo}
 Li volas, ke kun li oni parolu.\\
\speak{Marcello}
 Parolu, Horacio!\\
\speak[1]{\fsc{Horacio} (al la spirito).}
\psp{Parolu, Horacio!} Kiu estas\\ Vi, kiu vagas
en la nokta horo\\ En la majesta nobla ekstera\^{\j}o\\ De la
mortinta reĝo de Danujo?\\ Mi vin pri\^{\j}uras per \^cielo: diru!\\
\speak{Marcello}
 Li estas ofendita.\\
\speak{Bernardo}
\psp{Li estas ofendita.} Li foriras.\\
\speak{Horacio}
 En nomo de \^cielo! Restu! Diru!\\
\stg{(La spirito foriras.)}\\
\speak{Marcello}
 Li iris for, respondi li ne volis.\\
\speak{Bernardo}
 Ha, Horacio, kio al vi estas?\\ Vi estas pala, tremas?
\^Cu ne pli\\ Ol son\^go estas tiu \^ci apero?\\
\speak{Horacio}
 Per Dio! Mi ne povus tion kredi,\\ Se mi ne vidus \^gin
per miaj propraj\\ Okuloj.\\
\speak{Marcello}
\psp{Okuloj.} Ne simila al la reĝo?\\
\speak{Horacio}
  Ne malpli multe, ol vi al vi mem.\\ Jes, \^guste tian
portis li arma\^{\j}on\\ En la batalo kontra\u u la Norvegoj,\\ Li
\^guste tian vidon havis, kiam\\ Li en kolero sian pezan glavon\\
En la glacion batis. Strange!\\
\speak{Marcello}
\psp{En la glacion batis. Strange!} Tiel\\ Li jam du fojojn kun minaca vido\\ Aperis anta\u u
ni en sama horo.\\
\speak{Horacio}
 Ne povas mi precize \^gin klarigi;\\ Sed la apero, kiom
mi komprenas,\\ Al nia land' malbonon anta\u udiras.\\
\speak{Marcello}
 \^Cu ne militon? \^Cu vi scias, kial\\ Nun gardoj \^cie
staras, pafilegoj\\ Nun estas pretigataj \^ciam novaj,\\ El l'
eksterlando venas bataliloj,\\ Rapide \^sipoj estas konstruataj\\ En
granda nombro, kaj e\^c en diman\^coj\\ Ne \^cesas la prepara
laborado?\\ El kia ka\u uzo tiu rapidado\\ Kaj laborad' en tago kaj
en nokto?\\ \^Cu povas iu diri?\\
\speak{Horacio}
\psp{\^Cu povas iu diri?} Jes, mi povas, ---\\ Almena\u u tiom, kiom mi \^gin a\u
udis.\\ La lasta reĝo, kies la figuro\\ \^Jus anta\u u ni aperis,
estis iam\\ Vokita --- vi \^gin scias --- al batalo\\ De la Norvega
reĝo Fortinbras.\\ Hamleto, nia brava, glora reĝo,\\ En la
batalo venkis Fortinbrason.\\ La\u u la kontrakto anta\u usigelita\\
Kaj sankciita de la rajt' kaj moro\\ Post morto de l' venkita
Fortinbras\\ La tuta lando de l' venkita reĝo\\ Transiris en
posedon de l' venkinto.\\ Se la venkinto estus Fortinbras,\\ Al li
transiri devus nia lando.\\ Sed nun la juna fil' de Fortinbras,\\
Kura\^gigita eble per la morto\\ De nia reĝo, amasigis aron\\ Da
kura\^guloj kaj aventuristoj\\ Por entrepreno, kiun nia regno\\
Komprenas bone; lia celo estas:\\ Per forta mano de la bataliloj\\
El\^siri nun el nia man' la landon,\\ Perditan per batal' de lia
patro.\\ Kaj tio, kiel \^sajnas, estas ka\u uzo\\ De niaj
prepari\^goj kaj gardado\\ Kaj laborado en la tuta lando.\\
\speak{Bernardo}
 Mi pensas anka\u u, ke la ka\u uzo estas\\ Nun
tio \^ci. Pro tio anka\u u certe\\ Nun vizitadas nokte nian gardon\\
Terura la fantomo en arma\^{\j}o\\ De l' reĝo, kiu ka\u uzis la
militon.\\
\speak{Horacio}
 Ne! Mi alion timas de l' fantomo!\\ En la plej glora
temp' de l' granda Romo,\\ \^Jus anta\u u la pereo de Cezaro,\\
Mortintoj sin ellevis el la tomboj\\ Kaj promenadis tra la urbaj
stratoj;\\ Kaj steloj kun teruraj fajraj vostoj,\\ Kaj sanga roso,
makulita suno,\\ Mallumigita luno, --- \^cio tio\\ Aperis tiam kiel
anta\u usignoj\\ De granda malfeli\^co. Tiel ofte\\ Anta\u u la veno
de renversoj grandaj\\ Sin montras strangaj kaj teruraj signoj\\ En
la naturo. Al ni anka\u u certe\\ Nun la \^cielo sendas tian
signon\dots\\
\stg{(La spirito denove aperas.)}\\
\speak{Horacio}
 Sed ha! denove li aperas! Nepre\\ Mi nun kun li
parolos, se e\^c morto\\ Al mi per \^gi minacos! Halt', fantomo!\\
Se vo\^con vi posedas kaj parolon, ---\\ Parolu! Se per ia bona
faro\\ Al vi trankvilon povas mi alporti, ---\\ Parolu! Se dan\^gero
al la lando\\ Minacas kaj ankora\u u estas eblo\\ La landon anta\u
usavi, --- Ho, parolu\\ Kaj se en via vivo vi kolektis\\ Trezorojn
kaj enfosis en la teron\\ Kaj nun pro ili vagas en la noktoj, ---\\
Parolu! Diru! Restu kaj respondu!\\
\stg{(Koko krias.)}\\
 Haltigu \^gin, Marcello! \^Gi foriras!\\
\speak{Marcello}
 \^Cu piki lin per mia halebardo?\\
\speak{Horacio}
 Jes, piku lin, se li ne volas stari!\\
\speak{Bernardo}
 Jen \^gi!\\
\speak{Horacio}
\psp{Jen \^gi!} Jen \^gi! Ho, halt'!\\
\speak{Marcello}
\psp{Jen \^gi! Jen \^gi! Ho, halt'!} Ha, \^gi foriris! \\
\stg{(La spirito foriras.)}\\ Ni lin
ofendas! Kontra\u u majesteco\\ Ni venas kun sova\^ga atakado!\\
Kiel aer' li estas nevundebla,\\ Kaj nia atakado estas vana!\\
\speak{Bernardo}
 Li volis jam paroli, sed la koko \\Ekkriis.\\
\speak{Horacio}
\psp{Ekkriis.} Kaj subite li ektremis,\\ Simile al esta\^{\j}o
pekoplena\\ \^Ce krio de teruro. Oni diras,\\ La koko, trumpetisto
de l' mateno,\\ Per sia la\u uta, forta, hela vo\^co\\ El dormo
vekas dion de la tago,\\ Kaj la\u u la krio de la koko tuj\\ El
akvo, fajro, tero kaj aero\\ Spirito \^ciu, kie ajn li vagas,\\
Rapidas hejmen; la fantomo pruvis\\ Al ni, ke tio estas ne malvero.\\
\speak{Marcello}
 Li malaperis \^ce la koka krio.\\ Mi a\u udis, ke en
nokto Kristonaska,\\ En kiu sur la teron venas Kristo,\\ La koko
krias tra la tuta nokto:\\ Kaj ne kura\^gas tiam la spiritoj\\ Sin
montri, kaj la nokto estas pura,\\ La steloj ne minacas, la
koboldoj\\ Sin ka\^sas kaj la sor\^cistinoj dormas:\\ \^Car tiel
sankta estas tiu nokto.\\
\speak{Horacio}
 Mi anka\u u a\u udis, kaj mi iom kredas.\\ Sed vidu,
la mateno purpurvesta\\ Sin levas sur la roson de l'monteto:\\ Ni
povas jam forlasi la postenon,\\ Kaj mi konsilas: tion, kion vidis\\
Ni en la nokto, ni rakontu \^cion\\ Al la Hamleto juna; \^car mi
\^{\j}uras,\\ Ke la spirito, por ni tiel muta,\\ Al li parolos. \^Cu
konsentas vi,\\ Ke ni al li raportu, kiel amo\\ Kaj \^suldo al ni
fari \^gin ordonas?\\
\speak{Marcello}
 Mi petas vin, ni faru \^gin! Mi scias,\\ En kia
loko ni lin certe trovos. {\footnotesize (Ĉiuj foriras.)}
\end{verse}

\begin{center}
{\large SCENO II}\\[1ex]

\footnotesize Salono en la palaco. La reĝo, la reĝino, Hamleto, Polonio,
Laerto, Voltimand, Kornelio, korteganoj eniras.
\end{center}

\begin{verse}
\speak{Reĝo}
 
Ankora\u u fre\^sa estas la memoro\\
                Pri l'morto de la kara nia frato;\\
                Al nia kor' konvenus nun funebri,\\
                Kaj kvaza\u u unu frunto de mal\^gojo\\
                La tuta regno devus nun sulki\^gi:\\
                Sed tamen la prudento necesigas\\
                Nin venki la naturon; kun mal\^gojo\\
                Pri la mortinto ni decidis ligi\\
                Memoron pri ni mem kaj pri la regno.\\
                Kaj tial ni kun \^gojo depremita,\\
                Per unu el l' okuloj plezurante\\
                Kaj per la dua larmojn ver\^segante,\\
                Kun \^goj' funebra, plor' edzi\^gofesta,\\
                Pesante juste \^gojon kun doloro,\\
                Edzi\^gis kun l' edzin' de nia frato,\\
                Kun la reĝino nia kaj vidvino,\\
                Heredantino de la glora regno.\\
                Obeis ni per tio \^ci al via\\
                Konsilo tre prudenta kaj libera,\\
                Akceptu nian dankon kaj saluton!\\
                Kaj nun transiru ni al la aferoj.\\
                \vin Vi scias, ke la juna Fortinbras,\\
                Ne respektante dece nian indon,\\
                Pensante eble nun, ke per la morto\\
                De nia kara frato renversi\^gis\\
                La ordo kaj la fort' en nia lando, ---\\
                Ekpensis, ke li estas pli potenca,\\
                Kaj ekturmentis nin per la postulo\\
                Redoni nun al li la tutan landon,\\
                Perditan de mortinta lia patro\\
                La\u u plena rajt' al glora nia frato.\\
                Ni tial vin kunvokis nun. A\u uskultu\\
                Decidon nian: Jen ni skribis skribon\\
                Al l' onklo de la juna Fortinbras ---\\
                Malforta kaj malsana en la lito,\\
                L'intencon de la nevo li ne scias ---\\
                Ni skribis, ke retenu li la nevon\\
                De liaj krimaj, arogantaj agoj,\\
                \^Car ĉiuj prepari\^goj ja fari\^gas\\
                En lia propra lando kaj popolo.\\
                Vin, bravaj Voltimand kaj Kornelio,\\
                Ni sendas kun la skribo kaj saluto\\
                Al la maljuna estro de l' norvegoj.\\
                Sed ne ricevas vi la rajton trakti\\
                Kun la norvego pli, ol en mezuro\\
                De tio, kion \^{\j}us ni al vi diris.\\
                Adia\u u, kaj per rapideco montru\\
                Al ni fervoron vian!\\
\speak[1]{\fsc{Kornelio} kaj \fsc{Voltimand}}
\psp{Al ni fervoron vian!} Kredu, reĝo,\\
                              Fidele ni plenumos la ordonon.\\
\speak{Reĝo}
 Ne dubas ni pri \^gi. Feli\^can vojon!\\
\stg{(Voltimand kaj Kornelio foriras.)}\\
                Kaj nun, Laerto, al l'afero via!\\
                Vi diris, ke vi peti ion volas!\\
                Pri kio ajn prudenta vi al ni\\
                Parolos, viaj vortoj ne perdi\^gos.\\
                Volonte ni plenumos vian peton.\\
                Ne pli obea estas kap' al koro,\\
                Ne pli servema mano al la bu\^so,\\
                Ol la Danuja tron' al via patro.\\
                Sciigu nin pri la deziro via.\\
\speak{Laerto}
 Mi petas vin, permesu al mi, reĝo,\\
                Reiri nun Francujon. Mi volonte\\
                El tie venis tien \^ci, por fari\\
                La servon mian \^ce l' kronado via;\\
                Sed nun, jam plenuminte mian \^suldon,\\
                Sopiras mi denove al Francujo,\\
                Kaj nun de vi permeson mi petegas.\\
\speak{Reĝo}
 \^Cu via patro \^gin permesas? Kion\\
                Al tio diras Polonio?\\
\speak{Polonio}
\psp{Al tio diras Polonio?} Reĝo!\\
                Li eldevigis la permeson mian\\
                Per ne \^cesanta ripetado, tiel,\\
                Ke fine mi al lia peto donis\\
                Sigelon mian. Nun por li mi petas\\
                Permeson vian, via re\^ga mo\^sto.\\
\speak{Reĝo}
 Mi liberigas vin. Nun via tempo\\
                Libere estas al dispono via. ---\\
                Kaj nun al vi mi turnas min, Hamleto,\\
                Vi, kara nevo, kara filo mia!\\
\speak[1]{\fsc{Hamleto} (flanken)}
Sed certe tute fremda per la koro.\\
\speak{Reĝo}
 Ankora\u u \^ciam nuboj sur la frunto?\\
\speak{Hamleto}
 Ho, ne! Mi staras ja anta\u u la suno!\\
\speak{Reĝino}
 For\^{\j}etu la doloron, kara filo,\\
                Amikan frunton montru al la reĝo.\\
                Ne ser\^cu kun palpebroj mallevitaj\\
                Eterne vian patron en la polvo.\\
                Vi scias ja: la le\^goj de l' naturo\\
                Ne estas refareblaj! Kio vivas,\\
                Necese devas iam morti. Vivo\\
                Eterna post la tombo nur ekzistas.\\
\speak{Hamleto}
 Ho, jes, reĝino, ĉiuj devas morti.\\
\speak{Reĝino}
 Nu kial do la morto de la patro\\
                Al vi ek\^sajnis tiel eksterorda?\\
\speak{Hamleto}
 Ek\^sajnis? Ne, \^gi estas, ho, patrino.\\
                Ne mia nigra vesto, nek la \^gemoj,\\
                Nek la okuloj plenaj de larmaro,\\
                Nek la funebra vido de l' viza\^go,\\
                Nek ĉiuj moroj de senkonsoleco\\
                Min vere montras. \^Cio estas \^sajno.\\
                Funebron estas tre facile ludi.\\
                L' efektiva\^{\j}on portas mi interne;\\
                Eksteraj gestoj estas ja nur vestoj.\\
\speak{Reĝo}
 Tre bela kaj la\u udinda estas via\\
                Mal\^gojo pri la patro la mortinta.\\
                Sed sciu, anka\u u via patro perdis\\
                La patron sian, tiu --- anka\u u sian.\\
                La filo, la\u u la \^suldo de infano,\\
                Funebri devas kelkan tempon. Tamen\\
                Obstine kaj sen ia fino plendi, ---\\
                Agado tia estas granda peko\\
                Kaj tute ne konvenas al la viro.\\
                Ribel' \^gi estas kontra\u u la \^cielo,\\
                Signo de kor' sova\^ga kaj senbrida,\\
                De kapo nematura kaj malforta.\\
                Pri kio \^ciu scias, ke fari\^gi\\
                \^Gi devis; kio estas tute simpla,\\
                Plej ordinara fakto en la mondo:\\
                Pro kia ka\u uzo tion \^ci obstine\\
                Alpreni al la koro? Fi, \^gi estas\\
                Pekado kontra\u u Dio kaj naturo,\\
                Pekado kontra\u u la mortinto mem;\\
                Malsa\^go anta\u u pura la prudento,\\
                Kiu predikas: "patroj devas morti".\\
                Kaj kiu \^ciam, de l' mortint' unua\\
                \^Gis nuna tempo, diras kaj ripetas:\\
                "\^Gi devas tiel esti!" Mi vin petas,\\
                For\^{\j}etu la doloron la sencelan\\
                Kaj vidu patron en persono nia;\\
                La mondo sciu, ke al nia trono\\
                Vi estas la sendube plej proksima,\\
                Kaj, kiel plej amanta el la patroj,\\
                Mi portas por vi amon plej varmegan\\
                En mia koro. Ke vi nun reiru\\
                Al la lernejo alta Vittenberga,\\
                Ni tion ne dezirus; ni vin petas,\\
                Vi restu apud ni, amata nevo,\\
                Unua kortegano, kara filo.\\
\speak{Reĝino}
 Vi ne rifuzu al patrino via:\\
                Mi petas vin, Hamleto, restu hejme.\\
\speak{Hamleto}
 Al vi obei estas mia \^suldo.\\
\speak{Reĝo}
 Jen tio estas bele respondita.\\
                Mi \^gojas, ke vi restas. Nun, reĝino,\\
                La propravola cedo de Hamleto\\
                Plenigas mian koron per plezuro;\\
                Kaj tial ni aran\^gos nun festenon,\\
                Kaj kune kun la sono de l' pokaloj\\
                La pafilegoj tondru; \^ciun fojon,\\
                Kiam la reĝo levos la pokalon,\\
                De l' tero tondro iru al \^cielo,\\
                Tra l' mondo \^gojon disportante. --- Venu!\\
\stg{(La reĝo, reĝino, Laerto kaj korteganoj foriras.)}\\
\speak{Hamleto}
 Ho, kial ne fandi\^gas homa korpo,\\
                Ne disflugi\^gas kiel polv' en vento!\\
                Sin mem mortigi kial malpermesis\\
                La Plejpotenca! Dio mia granda!\\
                Ho, kiel bestaj kaj abomenindaj\\
                Aperas ĉiuj agoj de la mondo!\\
                Fi, fi! \^Gardeno plena de venenaj\\
                Malbelaj herboj, \^cie senescepte!\\
                Apena\u u pasis du monatoj! Ne!\\
                Nur ses semajnoj! Tia granda homo!\\
                Se lin kompari kun la nuna reĝo,\\
                Li estis Apolono \^ce Satiro!\\
                Kaj tiel amis li patrinon mian,\\
                Ke al la ventoj e\^c li ne permesis\\
                Viza\^gon \^sian tu\^si! Ho, \^cielo!\\
                Ho tero! \^Cu forgesi estas eble?\\
                Kaj \^si ja lin pasie tiel amis!\\
                Kaj tamen nun post kelke da semajnoj\dots\\
                Pri tio mi ne volas e\^c ekpensi!\\
                Malforto! via nom' estas: virino!\\
                Monato! \^Si ankora\u u ne eluzis\\
                La \^suojn, kiujn portis \^si, irante\\
                Funebre post la \^cerk' de mia patro.\\
                Ver\^sante larmojn kvaza\u u per riveroj.\\
                Ho Dio mia! Besto senprudenta\\
                Mal\^gojus ja pli longe! Nun edzino\\
                De mia onklo, frat' de mia patro,\\
                Sed ne simila pli al mia patro,\\
                Ol mi al Herkuleso! Nur monato!\\
                La signoj de mensogaj \^siaj larmoj\\
                Ankora\u u de l' viza\^g' ne malaperis, ---\\
                \^Si estas jam edzino de alia!\\
                Ho, malbenita rapideco flugi\\
                En kriman liton, liton de adulto!\\
                Ne bonon \^gi alportos! Tamen krevu\\
                Kor' mia, sed silentu mia bu\^so!\\
\stg{(Horacio, Bernardo kaj Marcello venas.)}\\
\speak{Horacio}
 Salut' al la re\^gido!\\
\speak{Hamleto}
\psp{Salut' al la re\^gido!} Ha, mi \^gojas\\
                Vin vidi! Se memoro min ne trompas,\\
                Vi estas Horacio?\\
\speak{Horacio}
\psp{Vi estas Horacio?}Jes, re\^gido,\\
                Kaj \^ciam tute preta al vi servi.\\
\speak{Hamleto}
 Ne nomu min "re\^gido", sed "amiko".\\
                Sed kial vi forlasis Vittenbergon?\\
                {\footnotesize (Al Marcello)} Marcello?\\
\speak{Marcello}
\psp{{\footnotesize (Al Marcello)} Marcello?} Jes, re\^gido.\\
\speak{Hamleto}
\psp{{\footnotesize (Al Marcello)} Marcello? Jes, re\^gido.} Mi tre \^gojas\\
                Revidi vin. {\footnotesize (Al Horacio)} Sed diru serioze,\\
                Por kio vi forlasis Vittenbergon?\\
\speak{Horacio}
 De maldiligenteco, ho, re\^gido\\
\speak{Hamleto}
 Al malamiko via ne permesus\\
                Mi tion diri, kaj vi vane penas\\
                Kredigi min, ke tio estas vero.\\
                Mi scias, vi ne amas sen laboro\\
                Vagadi. Kion tie \^ci vi faras?\\
                \^Cu trinki vi ankora\u u volas lerni?\\
\speak{Horacio}
 Mi alveturis al la enterigo\\
                De via patro.\\
\speak{Hamleto}
\psp{De via patro.} Ha, ne moku min,\\
                Kolego; diru, ke vi alveturis\\
                Al fest' edzi\^ga de patrino mia!\\
\speak{Horacio}
 Jes, vero, princ', la dua balda\u u sekvis\\
                Post la unua.\\
\speak{Hamleto}
\psp{Post la unua.} Vidu, mia kara, ---\\
                Afero simpla de ekonomio:\\
                De l' fest' funebra restis multaj man\^goj, ---\\
                Por ke la man\^goj vane ne pereu,\\
                Edzi\^gon fari oni nun rapidis.\\
                Ho, pli volonte mi en la infero\\
                Renkontus malamikon plej malbonan,\\
                Ol tagon tiun \^ci alvivi! \^Sajnas\\
                Al mi, ke mi la patron mian vidas\dots\\
\speak{Horacio}
 Ha, kie?\\
\speak{Hamleto}
\psp{Ha, kie?} En l' okuloj de l' animo.\\
\speak{Horacio}
 Mi konis lin, li estis brava reĝo.\\
\speak{Hamleto}
 Li estis homo en plej pura senco.\\
                Al li similan ni neniam vidos.\\
\speak{Horacio}
 Re\^gido, al mi \^sajnas, ke mi vidis\\
                Lin en la lasta nokto.\\
\speak{Hamleto}
\psp{Lin en la lasta nokto.} Vidis? Kiun?\\
\speak{Horacio}
 La reĝon, vian patron, mia princo.\\
\speak{Hamleto}
 La reĝon? Mian patron?\\
\speak{Horacio}
\psp{La reĝon? Mian patron?} Trankviligu,\\
                Re\^gido, por momento vian miron\\
                Kaj ela\u uskultu, kion mi raportos\\
                Pri mirinda\^{\j}o, kies la verecon\\
                Atestos amba\u u tiuj \^ci kolegoj.\\
\speak{Hamleto}
 Pro Dio, ho, rakontu pli rapide!\\
\speak{Horacio}
 Du noktojn al Marcello kaj Bernardo\\
                Aperis \^gi: postene ili staris, ---\\
                Kaj jen en noktomeza silentego\\
                Aperas anta\u u ili ia ombro,\\
                Simila tute al la patro via,\\
                Armita de la kapo \^gis piedoj;\\
                Per pa\^so serioza kaj majesta,\\
                Ne rapidante, \^gi sin preter\^sovas;\\
                Tri fojojn \^gi ripetas sian mar\^son\\
                Anta\u u la teruritaj oficiroj,\\
                Proksime tiel, ke per halebardo\\
                \^Gin ili povus tu\^si. Dispremitaj\\
                De granda timo, ili mute staras\\
                Kaj e\^c ne provas ion ekparoli.\\
                Pri la aper' mistera kaj terura\\
                Al mi ili rakontis. Mi kun ili\\
                En la sekvanta tria nokto gardis,\\
                Kaj jen tre vere en la sama horo,\\
                Pri kiu ili diris, en la sama,\\
                Perfekte tiu sama maniero\\
                Aperis la fantomo. Vian patron\\
                Mi konis, la fantom' al li similis,\\
                Kiel de l' sama akvo gut' al guto.\\
\speak{Hamleto}
 Sed kie \^gi okazis?\\
\speak{Marcello}
\psp{Sed kie \^gi okazis?} \^Gi okazis\\
                Sur la teras', kie ni staris garde.\\
\speak[1]{\fsc{Hamleto} (al Horacio)}
Al li parolis vi?\\
\speak{Horacio}
\psp{Al li parolis vi?} Jes, mia princo;\\
                Sed li al mi e\^c vorton ne respondis.\\
                Nur unu fojon \^sajnis, ke li levas\\
                La kapon kaj moveton ian faras,\\
                Por ekparoli: sed subite tiam\\
                Ekkriis la\u ute koko la matenon,\\
                Kaj tuje forrapidis la fantomo\\
                Kaj malaperis.\\
\speak{Hamleto}
\psp{Kaj malaperis.} Efektive strange!\\
\speak{Horacio}
 Mi \^{\j}uras, ke \^gi estas pura vero!\\
                Ni \^gin kalkulis kiel nian devon\\
                Al vi raporti pri l' afero.\\
\speak{Hamleto}
\psp{Al vi raporti pri l' afero.} Vere,\\
                Sinjoroj, tio min maltrankviligas.\\
                \^Cu vi hodia\u u staros gardon?\\
\speak{Ĉiuj}
\psp{\^Cu vi hodia\u u staros gardon?} Jes.\\
\speak{Hamleto}
 Kaj, diras vi, li estis en arma\^{\j}o?\\
\speak{Ĉiuj}
 Jes, princo.\\
\speak{Hamleto}
\psp{Jes, princo.} De la kapo \^gis piedoj?\\
\speak{Ĉiuj}
 De l' kapo \^gis piedoj, nia princo.\\
\speak{Hamleto}
 Kaj sekve la viza\^gon vi ne vidis?\\
\speak{Horacio}
 Levita estis lia viziero,\\
                Kaj ni tre bone vidis la viza\^gon.\\
\speak{Hamleto}
 \^Cu la rigardo estis malafabla?\\
\speak{Horacio}
 Mieno lia montris pli suferon,\\
                 Sed ne koleron.\\
\speak{Hamleto}
\psp{Sed ne koleron.} \^Cu li estis pala,\\
                A\u u \^cu koloron havis en viza\^go?\\
\speak{Horacio}
 Terure pala.\\
\speak{Hamleto}
\psp{Terure pala.} \^Cu li vin rigardis?\\
\speak{Horacio}
 Tre forte.\\
\speak{Hamleto}
\psp{Tre forte.} Kial mi kun vi ne estis!\\
\speak{Horacio}
 Por vi l' apero estus tro terura!\\
\speak{Hamleto}
 Tre povas esti. \^Cu li longe restis?\\
\speak{Horacio}
 Apena\u u centon povus vi kalkuli\\
                \^Ce nerapida kalkulad'.\\
\speak[1]{\fsc{Marcello} kaj \fsc{Bernardo}}
\psp{\^Ce nerapida kalkulad'.} Pli longe!\\
\speak{Horacio}
 Almena\u u tiam, kiam mi \^gin vidis.\\
\speak{Hamleto}
 La barbo estis griza, \^cu ne vere?\\
\speak{Horacio}
 \^Gi estis tia, kiel mi \^gin vidis\\
                \^Ce lia vivo: duonnigre-griza.\\
\speak{Hamleto}
 Hodia\u u mi kun vi en nokto staros:\\
                Denove eble venos la fantomo.\\
\speak{Horacio}
 \^Gi certe venos.\\
\speak{Hamleto}
\psp{\^Gi certe venos.} Kaj se \^gi aperos\\
                En l' ekstera\^{\j}' de mia nobla patro,\\
                Al \^gi mi ekparolos, se e\^c tuta\\
                Al mi minacus la infero. Vin\\
                Mi ĉiujn petas, se \^gis nun silentis\\
                Vi pri l' apero, tenu anka\u u plu\\
                \^Gin en sekreto; kaj al \^cio, kio\\
                Okazos eble en la nokto, havu\\
                Okulojn kaj orelojn, sed ne bu\^son.\\
                Per mia amo mi vin rekompencos.\\
                Adia\u u do! en la dekdua horo\\
                Sur la teraso mi vin revizitos.\\
\speak{Ĉiuj}
 Al via princa mo\^sto niaj servoj!\\
\speak{Hamleto}
 Ne, via amo, kiel al vi mia.\\
                Adia\u u do! {\footnotesize (Horacio, Marcello, kaj Bernardo foriras.)}\\
                \psp{Adia\u u do!} Spirit' de mia patro\\
                En bataliloj! Mi supozas ion\\
                Malbonan. Ho, se venus jam la nokto!\\
                \^Gis tiam do trankvile! Malbona\^{\j}oj\\
                Kaj krimoj venas al la lum' de l' tago,\\
                Se e\^c murego tera ilin kovras.\\
\stg{(Foriras.)}
\end{verse}

\begin{center}
{\large SCENO III}\\[1ex]

\footnotesize \^Cambro en la domo de Polonio. \fsc{Laerto} kaj \fsc{Ofelio} eniras.
\end{center}

\begin{verse}

\speak{Laerto}
 Paka\^{\j}o mia estas sur la \^sipo.\\
                Adia\u u, fratineto! Se vi havos\\
                Okazon, ne forgesu al mi skribi\\
                Pri via farto.\\
\speak{Ofelio}
 \^Cu vi tion dubas?\\
\speak{Laerto}
 Sed pri Hamleto kaj pri lia amo\\
                Konsilas mi al vi, vi \^gin rigardu,\\
                Kiel kapricon kaj nur kiel ludon\\
                De juna sango; bela violeto\\
                En la printempo: frue elkreskinta,\\
                Sed ne konstanta, --- dol\^ca, sed ne da\u ura,\\
                Odoro, \^guo de momento unu,\\
                Nenio pli.\\
\speak{Ofelio}
\psp{Nenio pli.} Nenio pli?\\
\speak{Laerto}
\psp{Nenio pli. Nenio pli?} Jes, tiel\\
                Vi \^gin rigardu. La natur', kreskante,\\
                Grandi\^gas pli ne sole per la korpo\\
                Kaj per la forto de la sama sento;\\
                Kun la kreskado de la templo ofte\\
                \^San\^gi\^gas la spirito kaj la servo\\
                Interne. Nun li eble amas vin;\\
                Nek trompo nek malico nun makulas\\
                La virton de l' animo lia: tamen\\
                Memoru vi, ke en la rango lia\\
                De sia volo li ne estas mastro.\\
                Li mem ja estas sklav' de sia stato;\\
                Ne povas li, kiel la simplaj homoj,\\
                Por si elekti: de elekto lia\\
                Dependas farto de la tuta regno, ---\\
                Li tial devas gvidi la elekton\\
                Per la aprob' kaj vo\^co de la korpo,\\
                Al kiu li mem servas kiel kapo.\\
                Se li nun diras, ke li amas vin,\\
                Prudente estas, ke vi al li kredu\\
                Nur tiom, kiom povas li plenumi\\
                La vorton sian, --- tio estas tiom,\\
                Kiom permesas la komuna vo\^co\\
                De tuta la Danujo. Ekmemoru,\\
                En kia grad' honoro via povas\\
                Suferi, se a\u uskultos tro kredeme\\
                Vi lian kanton, se la koron vian\\
                Vi perdos kaj al lia persistado\\
                Malkovros la trezoron vian \^castan.\\
                Vi timu \^gin, fratino mia kara,\\
                Evitu flaman mezon de la amo,\\
                Atakon kaj atencon de deziro!\\
                Knabin' plej \^casta perdis jam modeston,\\
                Se al la lun' \^si montris sian \^carmon.\\
                E\^c virto ne evitas kalumnion,\\
                L' infanojn de printempo ofte mordas\\
                La verm' ankora\u u en la bur\^goneco,\\
                Kaj al la frua roso de juneco\\
                Spiret' venena estas plej dan\^gera.\\
                Vin gardu do! Tim' donas garantion.\\
                Por la juneco \^cie staras retoj.\\
\speak{Ofelio}
 La sencon de l' instruo via bona\\
                Konservos mi, por gardi mian bruston;\\
                Sed, bona mia frato, vi ne agu\\
                Simile al la predikistoj, kiuj\\
                Predikas krutan vojon al \^cielo,\\
                Dum ili mem senzorge kaj volupte\\
                Sur flora vojo de gajeco pa\^sas,\\
                Mokante pri prediko sia propra.\\
\speak{Laerto}
 Ne timu! Tamen mi tro longe restis. ---\\
                La patro venas jen! {\footnotesize (Polonio eniras.)}\\
                \psp{La patro venas jen!} {\footnotesize (Al Polonio)} Duobla beno\\
                Sendube duobligas la feli\^con:\\
                Dank' al okazo mi denove povas\\
                Adia\u u al vi diri.\\
\speak{Polonio}
\psp{Adia\u u al vi diri.} Vi ankora\u u\\
                La domon ne forlasis? Al la \^sipo!\\
                La vento blovas helpe al la vojo,\\
                Kaj oni vin atendas. Nu, akceptu\\
                Denove mian benon.\\
                \stg{(Li metas la manon sur la kapon de Laerto.)}\\
                Kaj memoru\\
                Regulojn, kiujn mi al vi instruis:\\
                Ne \^cian penson metu sur la langon,\\
                Ne donu tuj al \^cia penso faron.\\
                Afabla estu, sed ne tro kredema.\\
                Al la amiko sa\^ge elektita\\
                Kunfor\^gu vin en fera fideleco,\\
                Sed gardu vian manon de la premo\\
                De \^ciu renkontota bona frato.\\
                Vi gardu vin de \^cia malpaci\^go;\\
                Se vi \^gin ne evitos, --- tiam agu\\
                Fortike, ke la malamik' vin timu.\\
                Al \^ciu servu per orelo via,\\
                Sed ne al \^ciu servu per la bu\^so.\\
                Konsilojn \^ciam prenu vi de \^ciu,\\
                Sed propran ju\^gon en la kapo tenu.\\
                La\u u via mon' mezuru vian veston,\\
                Sed \^cio estu takta kaj konvena;\\
                Vin vestu bone, sed ne kiel dando:\\
                La\u u vest' ekkonas oni ofte viron,\\
                La homoj altastataj en Francujo\\
                En tiu punkto estas tre zorgemaj.\\
                Ne prenu prunte kaj ne prunte donu:\\
                Per pruntedono ofte oni perdas\\
                Krom sia havo anka\u u la amikon,\\
                Kaj pruntepren' kondukas al ruino\\
                De la mastra\^{\j}o. Anta\u u \^cio estu\\
                Fidela al vi mem, --- de tio sekvos,\\
                Ke vi ne estos anka\u u malfidela\\
                Al la aliaj homoj. Nun adia\u u,\\
                Kaj mia beno vin akompanadu!\\
\speak{Laerto}
 Adia\u u, mia patro kaj sinjoro!\\
\speak{Polonio}
 Jam tempo. Iru, oni vin atendas.\\
\speak{Laerto}
 Adia\u u, Ofelio, kaj memoru\\
                Pri tio, kion diris mi al vi!\\
\speak{Ofelio}
 Mi bone \^slosis \^gin en mia kapo,\\
                Kaj la \^slosilon mi al vi fordonas.\\
\speak[1]{\fsc{Laerto} (al Polonio kaj Ofelio.)}
Adia\u u! {\footnotesize(Foriras.)}\\
\speak{Polonio}
\psp{Adia\u u! {\footnotesize(Foriras.)}} Kia estas la konsilo,\\
                Pri kiu li parolis?\\
\speak{Ofelio}
\psp{Pri kiu li parolis?} Pri Hamleto,\\
                Pri la re\^gido.\\
\speak{Polonio}
\psp{Pri la re\^gido.} Ha, jes, \^gustatempe!\\
                Mi a\u udis, ke en lasta temp' Hamleto\\
                Al vi komencis montri amikecon\\
                Kaj ke vi mem apartan afablecon\\
                Al li montradis. Se \^gi estas vero ---\\
                Mi a\u udis \^gin en formo de averto ---\\
                Mi devas al vi diri, ke vi mem\\
                Kredeble ne komprenas tute klare\\
                Dan\^geron, kiu al filino mia\\
                Kaj al honoro via nun minacas.\\
                Kiel vi estas unu kun la dua?\\
                La veron al mi diru!\\
\speak{Ofelio}
\psp{La veron al mi diru!} De mallonge\\
                Al mi li ripetadis kelkajn fojojn,\\
                Ke li estimas min.\\
\speak{Polonio}
\psp{Ke li estimas min.} Ha, ha! Estimo!\\
                Parolo de nesperta knabineto!\\
                Kaj kredas vi al la "estimo" lia?\\
\speak{Ofelio}
 Mi mem ne scias, kiel pri \^gi pensi.\\
\speak{Polonio}
 Nu, a\u udu do! Vi estas tre malsa\^ga,\\
                Ke vi akceptas por kontanta mono\\
                La vortojn, kiuj vere ne enhavas\\
                E\^c plej malgrandan indon. Vi vin tenu\\
                Prudente kaj singarde, \^car alie\\
                La malsa\^geco via la grandega\\
                Vin pereigos!\\
\speak{Ofelio}
\psp{Vin pereigos!} Li al mi proponis\\
                La amon sian pure kaj honeste.\\
\speak{Polonio}
 Jes, pure kaj honeste! Vi, senkapa!\\
\speak{Ofelio}
 Li \^{\j}uris al mi, patro mia kara,\\
                Per ĉiuj sanktaj \^{\j}uroj de \^cielo.\\
\speak{Polonio}
 Kaptiloj por la birdoj! Mi ja scias,\\
                Ke se la sango bolas, tiam lango\\
                La \^{\j}urojn ne avaras. Ho, filino,\\
                Ne prenu vi por fajro tiun flamon, ---\\
                \^Gi lumas, sed \^gi tute ne varmigas,\\
                \^Gi estingi\^gas tuj, ne supervivas\\
                E\^c la minuton de la promesado.\\
                De nun kun via virga afableco\\
                Avaru pli; la paroladon vian\\
                Vi \^satu pli, ne estu tute preta\\
                Tuj la\u u ordono! La re\^gid' Hamleto\\
                Ankora\u u estas juna, kaj li havas\\
                Pli grandan liberecon, ol al vi\\
                Donata povas esti. Ofelio,\\
                Mallonge mi sed klare al vi diras:\\
                Ne kredu al \^{\j}urado lia; \^{\j}uroj\\
                Similaj estas trompaj delogantoj,\\
                Anta\u uparoloj de tre pekaj petoj;\\
                Promesojn piajn ili hipokritas,\\
                Por pli sukcese veni al la celo.\\
                Per unu vorto: nun vi e\^c minuton\\
                Ne restu en intima parolado\\
                Kun la re\^gid' Hamleto! Ne kura\^gu\\
                Forgesi mian tiun \^ci ordonon!\\
                Nun iru!\\
\speak{Ofelio}
\psp{Nun iru!} Mi obeos, mia patro. {\footnotesize (Amba\u u foriras.)}
\end{verse}

\begin{center}
{\large SCENO IV}\\[1ex]

\footnotesize La teraso anta\u u la palaco. Hamleto, Horacio, kaj Marcello venas.
\end{center}

\begin{verse}
\speak{Hamleto}
 Ho, kiel akra estas la aero!\\
\speak{Horacio}
 Jes, princo, blovas vento malvarmega.\\
\speak{Hamleto}
 Kioma horo?\\
\speak{Horacio}
 Balda\u u la dekdua.\\
\speak{Marcello}
 Ne, ne, jam la dekdua horo batis.\\
\speak{Hamleto}
 \^Cu efektive? Mi ne a\u udis. Sekve\\
                Alproksimi\^gas jam la temp', en kiu\\
                Aperas ordinare la spirito.\\
\stg{(Post la sceno estas a\u udataj tamburado kaj ektondro de pafilego.)}\\
\speak{Horacio}
 Ha, kion tio \^ci signifas, princo?\\
\speak{Hamleto}
 La reĝo diligente nun pasigas\\
                La nokton en gajega festenado;\\
                Kaj \^ciun fojon, kiam li eltrinkas\\
                Pokalon, pafilegoj al la mondo\\
                Anoncas la grandfaron de la reĝo.\\
\speak{Horacio}
 Ekzistas tia moro?\\
\speak{Hamleto}
\psp{Ekzistas tia moro?} Jes, sendube.\\
                Sed pensas mi, ke estas pli honore\\
                Forgesi tian moron, ol \^gin sekvi.\\
                La brua kaj dibo\^ca festenado\\
                Alportis al ni tre malbonan gloron\\
                \^Ce la popoloj de la tuta mondo.\\
                Drinkistoj oni nomas nin insulte;\\
                Kaj tiu \^ci makulo malpurigas\\
                La gloron de plej grandaj niaj faroj.\\
                Similan sorton ofte anka\u u havas\\
                Privataj homoj, se sen propra kulpo\\
                Makulon ian ili de naturo\\
                Ricevis; se ekzemple de naski\^go ---\\
                En kio ne ilia vol' ja estis ---\\
                Ilia sango estas tro bolanta,\\
                Rompanta ofte digon de prudento;\\
                A\u u se kutimo ilin malbonigis, ---\\
                La mond' ilin atakas sen kritiko,\\
                Se e\^c iliaj virtoj estas puraj\\
                Kaj multenombraj. Grajno da malbono\\
                Por la okul' de l' mondo ofte kovras\\
                La tutan indon de plej bona homo.\\
\stg{(Aperas la spirito en arma\^{\j}o.)}\\
\speak{Horacio}
 Ho, vidu, princo, \^gi aperas!\\
\speak{Hamleto}
\psp{Ho, vidu, princo, \^gi aperas!} Dio!\\
                Defendu nin, an\^geloj de \^cielo!\\
\stg{(Li staras kelkajn minutojn senmove.)}\\
                Spirito sankta, a\u u demon' terura,\\
                \^Cu el \^cielo a\u u el la infero,\\
                \^Cu vi intencas bonon a\u u malbonon, ---\\
                Aperis vi en tia nobla formo,\\
                Ke mi paroli devas. Ho, Hamleto,\\
                Ho, patro, ho vi, reĝo de Danujo,\\
                Respondu! Ho, ne lasu min perei\\
                En nesciado! Diru al mi, kial\\
                Sin levis el la tombo viaj ostoj?\\
                Kaj la \^cerkujo, kien ni trankvile\\
                Vin metis, kial \^gi malfermis nun\\
                La pezan sian bu\^son de marmoro,\\
                Por vin el\^{\j}eti? Kion \^gi signifas,\\
                Ke vi, senviva korpo, en arma\^{\j}o\\
                Denove venis en la lunan lumon,\\
                Por ektimigi nin, malsa\^gajn homojn,\\
                Kaj nin ataki per teruraj pensoj\\
                Ne klarigeblaj por l' animo nia?\\
                Por kio? Diru! Kial? Kion fari?\\
\stg{(La spirito faras signojn al Hamleto.)}.\\
\speak{Horacio}
 Li vokas vin, ke iru vi kun li,\\
                Li kvaza\u u volas ion komuniki\\
                Nur al vi sola.\\
\speak{Marcello}
\psp{Nur al vi sola.} Kun afablaj gestoj\\
                Li vokas vin pli malproksimen, princo;\\
                Sed ho, pro Di', ne iru!\\
\speak{Horacio}
\psp{Sed ho, pro Di', ne iru!} Ne, ne iru!\\
\speak{Hamleto}
 Sed tie \^ci ne volas li paroli,\\
                Kaj mi lin sekvos.\\
\speak{Horacio}
\psp{Kaj mi lin sekvos.} Ho, ne, ne, re\^gido\\
\speak{Hamleto}
 Nu, kion do mi timos? Mia vivo\\
                Ne havas por mi indon, kaj l'animo,\\
                Senmorta, kiel la spirito mem,\\
                Ja ne bezonas timi la spiriton.\\
                Li vokas min denove; mi lin sekvos.\\
\speak{Horacio}
 Sed se li vin allogos al la maro,\\
                A\u u eble al la supro de l' \^stonego,\\
                Staranta super la senfunda akvo,\\
                Kaj tie per terura nova vido\\
                Subite nebuligos vian sa\^gon\\
                Kaj pereigos vin? Memoru, princo!\\
                Jam per si mem al \^ciu viva homo\\
                Terura estas tiu alta pinto,\\
                Pendanta super la bruantaj ondoj\\
                De plej profunda loko de la maro.\\
\speak{Hamleto}
 Li \^ciam vokas. {\footnotesize (Al la spirito.)} Iru, mi vin sekvas!\\
\speak[1]{\fsc{Marcello} (retenante Hamleton).}
Ho, princo, vi ne iros!\\
\speak{Hamleto}
\psp{Ho, princo, vi ne iros!} For la manojn!\\
\speak{Horacio}
 A\u uskultu nin, ne iru!\\
\speak{Hamleto}
\psp{A\u uskultu nin, ne iru!} Mia sorto\\
                Min vokas, kaj per \^gi mi sentas nun\\
                En \^ciu vejno de la korpo mia\\
                Potencan feran forton de leono. \\
                \stg{(La spirito faras signojn.)}\\
                Li \^ciam vokas! Lasu! Pro \^cielo! {\footnotesize (Li sin elŝiras.)}\\
                Fantomon faros mi el \^ciu, kiu\\
                Kura\^gos min reteni! For! {\footnotesize (Al la spirito.)} Ho, iru!\\
                Mi post vi iras. {\footnotesize (La spirito kaj Hamleto foriras.)}\\
\speak{Horacio}
\psp{Mi post vi iras.} Ha, li frenezi\^gis!\\
\speak{Marcello}
 Ni sekvu lin! La dev' al ni ordonas!\\
\speak{Horacio}
 Ni iru! Kiel tio \^ci fini\^gos?!\\
\speak{Marcello}
 Malbona io estas en Danujo.\\
\speak{Horacio}
 Nin gardos la \^cielo.\\
\speak{Marcello}
\psp{Nin gardos la \^cielo.} Ni rapidu!
\end{verse}

\begin{center}
{\large SCENO V}\\[1ex]


\footnotesize Izolita loko de la teraso. La spirito kaj Hamleto venas. 
\end{center}

\begin{verse}
\speak{Hamleto}
 Ho, kien vi kondukas min? Parolu!\\
                Mi plu ne iros jam!\\
\speak{Spirito}
 A\u uskultu!\\
\speak{Hamleto}
 Diru!\\
\speak{Spirito}
 Jam proksimi\^gas mia horo. Balda\u u\\
                Reiri devos mi al la sulfuraj\\
                Turmentaj flamoj.\\
\speak{Hamleto}
\psp{Turmentaj flamoj.} Malfeli\^ca patro!\\
\speak{Spirito}
 Min ne beda\u uru, sed atentu bone,\\
                Al tio, kion diros mi.\\
\speak{Hamleto}
\psp{Al tio, kion diros mi.} Parolu!\\
                Atenti estas mia sankta \^suldo.\\
\speak{Spirito}
 Kaj ven\^gi anka\u u, kiam vi ekscios.\\
\speak{Hamleto}
 Ho, Dio! Ven\^gi? Kion?\\
\speak{Spirito}
\psp{Ho, Dio! Ven\^gi? Kion?} A\u udu bone.\\
                Mi estas la spirit' de via patro,\\
                Mi estas kondamnita longan tempon\\
                Vagadi en la nokto kaj bruladi\\
                La tutan tagon en eternaj flamoj,\\
                \^Gis mi puri\^gos de la ĉiuj pekoj\\
                De mia tera vivo. Se ne estus\\
                Al mi malpermesite paroladi,\\
                Rakonton tiam povus mi komenci,\\
                De kiu \^ciu vort' al vi dispremus\\
                La koron, rigidigus vian sangon,\\
                El kapo elsaltigus la okulojn,\\
                Kaj \^ciun vian haron disstarigus\\
                Simile al haregoj de histriko.\\
                Sed la misteroj de posttomba vivo\\
                Ne devas soni al orelo tera.\\
                A\u uskultu! Se vi amis vian patron\dots\\
\speak{Hamleto}
 Ho, Dio!\\
\speak{Spirito}
\psp{Ho, Dio!} Ven\^gu por mortigo lia!\\
\speak{Hamleto}
 Mortigo?\\
\speak{Spirito}
\psp{Mortigo?} Jes, mortigo plej malnobla,\\
                Terura, nenatura, nea\u udita.\\
\speak{Hamleto}
 Ho, nomu lin! Ke povu mi tuj flugi,\\
                Simile al la penso de amanto\\
                Mi flugu ven\^gi tuj.\\
\speak{Spirito}
\psp{Mi flugu ven\^gi tuj.} Vi \^sajnas preta;\\
                Vi estus dorma, kiel pala herbo,\\
                Kreskanta sur la bordo de la Leto,\\
                Se restus vi trankvila. A\u udu, filo:\\
                La famon oni faris, ke en tempo\\
                De mia dormo en \^gardeno mia\\
                Serpento mordis min; kaj oni trompas\\
                Per la mensoga fam' pri mia morto\\
                L'orelon de la regno; tamen sciu,\\
                Ho, nobla mia filo: la serpento,\\
                De kies mordo mortis via patro,\\
                Nun portas lian kronon.\\
\speak{Hamleto}
\psp{Nun portas lian kronon.} Mia onklo!\\
                Ho, la profeta mia anta\u usento!\\
\speak{Spirito}
 Jes, tiu plej malnobla adultulo\\
                (Ho, malbenita sprito, kaj donacoj!)\\
                Delogis la reĝinon \^sajne virtan\\
                Al la plezuroj de malnobla amo.\\
                Hamleto mia! Ho, kia defalo!\\
                De mi, kies la amo estis sankta,\\
                Sendeklini\^ga \^ciam de la \^{\j}uro,\\
                Farita en la tago de l' edzi\^go, ---\\
                \^Si malalti\^gis al pekulo, kiun\\
                Per \^cio la naturo ja malbenis!\\
                Sed kiel virton peko ne delogas,\\
                Se \^gi e\^c en \^ciela vest' aperas,\\
                Tiel volupto, se e\^c kun an\^gelo\\
                Vi ligos \^gin, enuas tamen balda\u u\\
                Kaj ser\^cas ion novan\dots Sed silentu!\\
                Mi sentas jam aeron de mateno, ---\\
                Mi mallongigos la rakonton. Kiam\\
                En la \^gardeno la\u u kutimo mia\\
                Mi dormis post tagmezo, via onklo\\
                Malla\u ute al\^steli\^gis kaj enver\^sis\\
                Al mi en la orelon plej teruran\\
                Venenon el malgranda boteleto,\\
                Venenon tian, kiu tre rapide\\
                Trakuras tuj la tutan homan korpon\\
                Kaj, kvaza\u u acida\^{\j}o en la lakto,\\
                Momente malbonigas en la korpo\\
                La tutan puran sangon. Mi pereis,\\
                Kaj lepro mian tutan glatan korpon\\
                Tuj kovris per abomeninda \^selo,\\
                Kaj tiel mi perfide en la dormo\\
                Per frata mano estis senigita\\
                De l' vivo, krono kaj edzino kune,\\
                Mi estis mortigita en florado\\
                De miaj pekoj, sen konfesodono,\\
                Sen komunio sankta; la kalkulon\\
                Ne resuminte, estis mi sendita\\
                Kun ĉiuj miaj kulpoj sur la kapo\\
                Al granda, fina ju\^go.\\
\speak{Hamleto}
\psp{Al granda, fina ju\^go.} Ho, terure!\\
\speak{Spirito}
 Se havas vi animon, ne permesu,\\
                Ke re\^ga lito de Danujo servu\\
                Por sangomiks' adulta kaj volupto.\\
                Sed kiel ajn vi volos tion fari, ---\\
                Vi ne makulu vian filan koron:\\
                Nenion faru kontra\u u la patrino, ---\\
                \^Sin ju\^gos la \^cielo kaj la pikoj\\
                De \^sia propra koro. Nun adia\u u!\\
                La pali\^gado de l' lumanta vermo\\
                Anoncas la aperon de l' mateno.\\
                Adia\u u, filo, kaj memoru min! {\footnotesize (Foriras.)}\\
\speak{Hamleto}
 Ho, vi, \^cielo! Tero! Eble anka\u u\\
                L' inferon voki? --- Fi, halt', mia koro!\\
                Ne maljuni\^gu tuj, ho mia korpo,\\
                Fortike vi min portu! Vin memori?\\
                Jes, malfeli\^ca patro! Tiel longe,\\
                \^Gis mia kapo perdos la kapablon\\
                De \^cia memorado! Vin memori?\\
                Sed de l' tabulo de memoro mia\\
                Forvi\^sos mi nun \^cion, kio restis,\\
                Sentencojn ĉiujn el la libroj, ĉiujn\\
                Pentra\^{\j}ojn, postesignojn, kiujn lasis\\
                Sur \^gi la pasinta\^{\j}o kaj juneco,\\
                Observojn kaj la spertojn de la vivo;\\
                Ordono via vivos tute sola\\
                De nun en mia cerbo, ne miksita\\
                Kun io malpli inda. Ho, \^cielo!\\
                Virino malhonesta kaj perfida!\\
                Fripono! Vi, fripono ridetanta!\dots\\
                La tabuleton donu! Mi enskribos,\\
                Ke oni povas \^ciam ridetadi\\
                Kaj tamen esti friponeg'. Almena\u u\\
                En nia lando tio estas ebla. {\footnotesize (Li skribas.)}\\
                Mi vin enskribis, onklo! Nun al mia\\
                Deviz'! "Adia\u u kaj memoru min!"\\
                Mi \^{\j}uris.\\
\speak[1]{\fsc{Horacio} (post la sceno.)}
Princo! Princo!\\
\speak[1]{\fsc{Marcello} (post la sceno.)}
\psp{Princo! Princo!} Princ' Hamleto!\\
\speak[1]{\fsc{Horacio} (post la sceno.)}
Ho, Dio lin defendu!\\
\speak[1]{\fsc{Hamleto} (decide, al si mem.)}
\psp{Ho, Dio lin defendu!} Tiel estu!\\
\speak[1]{\fsc{Marcello} (post la sceno.)}
He! Princo! He!\\
\speak{Hamleto}
\psp{He! Princo! He!} He! kara mia, he!\\
                Al mi, birdeto mia, he! mi estas.\\
\stg{(Horacio kaj Marcello venas.)}\\
\speak{Marcello}
 Nu, kio do sinjoro?\\
\speak{Horacio}
\psp{Nu, kio do sinjoro?} Kio nova?\\
\speak{Hamleto}
 Ho, tre mirinde!\\
\speak{Horacio}
\psp{Ho, tre mirinde!} Diru, kara princo.\\
\speak{Hamleto}
 Vi elbabilos.\\
\speak{Horacio}
 \psp{Vi elbabilos.} Ne, pro la \^cielo!\\
\speak{Marcello}
 Mi anka\u u ne.\\
\speak{Hamleto}
 \psp{Mi anka\u u ne.} He, kion vi parolas?\\
                \^Cu oni povus kredi?! Vi silentos?\\
\speak[1]{\fsc{Horacio} kaj \fsc{Marcello}}
 Jes, jes, pro la \^cielo, kara princo!\\
\speak{Hamleto}
 En tuta la Danujo vi ne trovos\\
                Friponon, kiu estus hom' honesta.\\
\speak{Horacio}
 Spirito ne bezonis ja sin levi\\
                El tombo, por sciigi nin pri tio.\\
\speak{Hamleto}
 Vi estas prava. Tial mi nun pensas,\\
                Sen plua disputado ni jam povas\\
                Al ni la manojn premi reciproke\\
                Kaj iri hejmen. Iru, kien vokas\\
                Vin la aferoj kaj deziroj viaj ---\\
                Aferojn kaj dezirojn \^ciu havas ---\\
                Mi anka\u u, la mizera, iros fari\\
                Aferon mian, --- mi nun iros pre\^gi.\\
\speak{Horacio}
 Malklare kaj tre strange vi parolas.\\
\speak{Hamleto}
 Ha, mi beda\u uras, ke mi vin \^cagrenas,\\
                Mi \^gin beda\u uras el la tuta koro.\\
\speak{Horacio}
 Vi ne \^cagrenas nin.\\
\speak{Hamleto}
 \psp{Vi ne \^cagrenas nin.} Ne, Horacio,\\
                \^Cagren' \^gi estas, pro Patriko sankta\dots\\
                Pri la spirito do mi al vi diros,\\
                \^Gi estas tre senkulpa kaj honesta.\\
                La scivolecon pri la estinta\^{\j}o\\
                Vi venkos, kiel povos. Nun kolegoj,\\
                Amataj kunlernantoj kaj amikoj,\\
                Malgrandan unu peton mi vin petos.\\
\speak{Horacio}
 Parolu, princo, ni \^gin certe faros.\\
\speak{Hamleto}
 Neniu devas scii pri l' apero\\
                De nuna nokto.\\
\speak[1]{\fsc{Horacio} kaj \fsc{Marcello}}
\psp{De nuna nokto.} Ni promesas, princo.\\
\speak{Hamleto}
 Sed \^{\j}uru!\\
\speak{Horacio}
 \psp{Sed \^{\j}uru!} Pro honoro, mi silentos!\\
\speak{Marcello}
 Mi anka\u u, pro honor'!\\
\speak{Hamleto}
 \psp{Mi anka\u u, pro honor'!} Sur mia glavo!\\
\speak{Marcello}
 Ni \^{\j}uris jam, sinjoro.\\
\speak{Hamleto}
 \psp{Ni \^{\j}uris jam, sinjoro.} Sed mi petas,\\
                Sur mia glavo \^{\j}uru al mi.\\
\speak[1]{\fsc{Spirito} (sub la tero.)}
\psp{Sur mia glavo \^{\j}uru al mi.} \^Juru!\\
\speak{Hamleto}
 Ha, mia kara! Vi al ni alrampis?\\
                Nu, bone! La bravul' a\u uskultas. \^Juru!\\
\speak{Horacio}
 Sed kiel \^{\j}uri?\\
\speak{Hamleto}
 \psp{Sed kiel \^{\j}uri?} Ke neniam vi\\
                Parolos e\^c per unu sola vorto\\
                Pri tio, kion en la lasta nokto\\
                Vi vidis. \^Juru sur la glavo!\\
\speak[1]{\fsc{Spirito} (sub la tero.)}
\psp{Vi vidis. \^Juru sur la glavo!} \^Juru!\\
\speak{Hamleto}
 Hic et ubique? \^San\^gu ni la lokon!\\
                Al mi, sinjoroj! Metu viajn manojn\\
                Denove sur la glavon kaj ripetu,\\
                Ke vi neniam al neniu diros\\
                Pri la apero de la nokto. \^Juru\\
                Pri tio \^ci sur mia glavo!\\
\speak[1]{\fsc{Spirito} (sub la tero.)}
\psp{Pri tio \^ci sur mia glavo!} \^Juru!\\
\speak{Hamleto}
 Ha, brave, mia talp'! Vi bone fosas!\\
                Pli malproksimen do, amikoj!\\
\speak{Horacio}
 \psp{Pli malproksimen do, amikoj!} Dio!\\
                Tre strange kaj neniel kompreneble!\\
\speak{Hamleto}
 Plej bona estos, se vi \^gin forgesos!\\
                Ekzistas en \^ciel' kaj sur la tero\\
                Pli da aferoj, ol en la lernejoj\\
                Instruas filozofoj, Horacio.\\
                Sed venu! Kaj mi tie \^ci denove\\
                Vin petas, \^{\j}uru: kiel ajn mirindaj\\
                Al vi agado mia poste \^sajnos,\\
                Se eble poste trovos mi utila\\
                \^Sajnigi, ke mi perdis la prudenton, ---\\
                Neniam vi kunmetu viajn manojn\\
                En tia formo, nek la kapon vian\\
                Balancu tiel, nek per parolado\\
                Dusenca montru, ke vi ion scias, ---\\
                Ekzemple: "nu, ni scias", a\u u "ni povus,\\
                Se ni nur volus", a\u u "se nur paroli\\
                Ni povus", a\u u "ekzistas unu punkto"\dots\\
                Ne helpu al vi Dio en mizero,\\
                Se tion \^ci vi faros. --- \^{\j}uru!\\
\speak[1]{\fsc{Spirito} (sub la tero.)}
\psp{Se tion \^ci vi faros. --- \^{\j}uru!} \^Juru!\\
\speak{Hamleto}
 Ripozu, ho, spirito malfeli\^ca!\\
                Kaj nun, sinjoroj, mi min rekomendas\\
                Al vi kun mia tuta amikeco.\\
                Kaj \^cion, kion mi, malri\^ca, povas\\
                Al vi alporti aman kaj amikan,\\
                Kun Dia help' al vi mi ne rifuzos.\\
                Ni iru! Kaj denove mi vin petas,\\
                La fingron \^ciam tenu sur la bu\^so!\\
                Ni vivas en terura tempo! Ve,\\
                Ke mi naski\^gis esti la punanto!\\
                Nu, venu do! Ni iros ĉiuj kune.\\
                \stg{(Ĉiuj foriras.)}
\end{verse}
\smallrule{}
