\begin{verse}
                 \vin   Tra densa mallumo briletas la celo,\\
                Al kiu kura\^ge ni iras.\\
                Simile al stelo en nokta \^cielo,\\
                Al ni la direkton \^gi diras.\\
                Kaj nin ne timigas la noktaj fantomoj,\\
                Nek batoj de l' sorto, nek mokoj de l' homoj,\\
                \^Car klara kaj rekta kaj tre difinita\\
                \^Gi estas, la voj' elektita.

                 \vin   Nur rekte, kura\^ge kaj ne flanki\^gante\\
                Ni iru la vojon celitan!\\
                E\^c guto malgranda, konstante frapante,\\
                Traboras la monton granitan.\\
                L'espero, l'obstino kaj la pacienco ---\\
                Jen estas la signoj, per kies potenco\\
                Ni pa\^so post pa\^so, post longa laboro,\\
                Atingos la celon en gloro.

                 \vin   Ni semas kaj semas, neniam laci\^gas,\\
                Pri l' tempoj estontaj pensante.\\
                Cent semoj perdi\^gas, mil semoj perdi\^gas, ---\\
                Ni semas kaj semas konstante.\\
                "Ho, \^cesu!" mokante la homoj admonas, ---\\
                "Ne \^cesu, ne \^cesu!" en kor' al ni sonas:\\
                "Obstine anta\u uen! La nepoj vin benos,\\
                Se vi pacience eltenos".

                 \vin   Se longa sekeco a\u u ventoj subitaj\\
                Velkantajn foliojn de\^siras,\\
                Ni dankas la venton, kaj, repurigitaj,\\
                Ni forton pli fre\^san akiras.\\
                Ne mortos jam nia bravega anaro,\\
                \^Gin jam ne timigas la vento, nek staro,\\
                Obstine \^gi pa\^sas, provita, hardita,\\
                Al cel' unu fojon signita!

                  \vin  Nur rekte, kura\^ge kaj ne flanki\^gante\\
                Ni iru la vojon celitan!\\
                E\^c guto malgranda, konstante frapante,\\
                Traboras la monton granitan.\\
                L' espero, l' obstino kaj la pacienco, ---\\
                Jen estas la signoj, per kies potenco\\
                Ni pa\^so post pa\^so, post longa laboro,\\
                Atingos la celon en gloro.

\end{verse}

\citsc{L. Zamenhof.}

\smallrule{}
