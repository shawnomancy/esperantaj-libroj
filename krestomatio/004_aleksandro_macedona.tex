\begin{center}
\footnotesize Orienta historia legendo el V. \fsc{Jabec.}
\end{center}

   Granda re\^go ekregis en Grekujo; "Aleksandro Macedona" li estis
nomata. Li venkis popolojn kaj regnojn. Kaj liaj pensoj alte
levi\^gis super la nubojn kaj flugadis kiel per aglaj flugiloj tiel
alte sub la \^cielo, ke li el tie ekvidis, ke la tero estas en la
oceano, kiel ekzemple na\^ganta pomo en vazo da akvo.

   Unufoje Aleksandro ekvolis iri Afrikon, kie trovi\^gas la oro kaj kie
regas virinoj, kiuj anka\u u batalas kun siaj malamikoj.

   Li kolektis siajn maljunulojn kaj ilin demandis:

 --- Per kia maniero oni povas veni en la landon de oro?

 --- Vi ne povas transiri en tiun landon, respondis la maljunuloj, \^car
mallumaj montegaroj baras la vojon.

 --- Unu fojon mi decidis, kaj mian vorton mi ne \^san\^gos, kolere
respondis la re\^go; mi devas tie esti, tial donu al mi konsilon.

 --- Se vi ne\^san\^geble decidis iri tien, respondis la maljunuloj,
ordonu al viaj servantoj alkonduki azenojn de Libujo, por kiuj ne
ekzistas mallumo, kaj ili prenu anka\u u tre longajn \^snurojn kaj
alligu ilian unu finon al via palaco; veturu, ho re\^go, kun via
militistaro sur la azenoj tiun lokon kaj tenu la \^snurojn en la
manoj: tiam, se vi ne trovos la deziritan lokon, vi povos per la
helpo de la \^snuroj veni returne al via re\^ga palaco.

   Tio \^ci pla\^cis al Aleksandro. Li prenis siajn militistojn, kiuj
tenis la \^snurojn en la manoj, kaj ili forveturis al la lando de
oro, por batali kun la tie regantaj virinoj.

   Kiam li alvenis al tiu lando, la virinoj eliris al li kaj sin turnis
al li kun la sekvanta parolo:

 --- Se via honoro estas al vi kara, ne tu\^su nin; \^car se vi nin
venkos, la popoloj diros, ke nur virinojn vi venkis, kaj se vi estos
venkita, oni diros ke virino vin mortigis!

   Aleksandro lasis sian paroladon pri batalo kaj petis de ili panon. La
virinoj alportis al li oran tablon, sur kiu ku\^sis ora pano.

 --- \^Cu mi sati\^gos de ora pano? demandis Aleksandro, tion \^ci
vidinte.

 --- Se vi deziras panon, kial vi venis al ni? respondis la virinoj;
\^cu en via lando mankas pano, ke vi tiel malproksime venis \^gin
ser\^ci?

   Aleksandro ekhontis kaj eliris el la urbo, surskribinte sur videblaj
lokoj la sekvantan frazon: "Mi, Aleksandro, estis malsa\^gulo en la
da\u uro de mia tuta vivo \^gis la tempo, kiam la Afrikaj virinoj
instruis al mi sa\^gon kaj prudenton". Kaj li reiris kun sia
militistaro al sia lando.

   En la vojo li sidi\^gis man\^geti sur la bordo de ia rivero, kaj
lavetante salan fi\^son en la akvo de la rivero, li eksentis tre
agrablan odoron.

 --- Tiu \^ci rivero sendube elvenas el la paradizo, diris Aleksandro.

   Ne longe pensante, Aleksandro ekiris la\u u tiu \^ci rivero \^gis li
venis al la pordoj de ia palaco, kiu estis, kompreneble, la
paradizo.

 --- Malfermu al mi la pordon! li ekkriis.

 --- Tio \^ci estas la pordo de la domo de Dio, respondis al li vo\^co el
interne; nur sanktuloj havas la rajton enveni tien \^ci!

 --- Konfesu mian re\^gan majeston, ekkriis Aleksandro, kaj donacu al mi
ion, \^car mi estas re\^go !

   Apena\u u li finis siajn vortojn, --- jen ia mano donas al li homan
kapon, li \^gin prenis kaj reiris al sia lando.

   Alveninte hejmen, li ekvolis scii la pezon de la donacita kapo, metis
\^gin sur la teleron de pesilo, kaj la telero mallevi\^gis \^gis la
tero. Sur la duan teleron li metis milojn da pecoj da ar\^gento kaj
oro por \^gin mallevi --- , sed vane: li ne povis; la tuta ar\^gento
kun la oro ne pezis tiom, kiom la kapo.

 --- Kion \^gi signifas? demandis Aleksandro la sa\^gulojn.

 --- \^Gi estas tial, respondis la sa\^guloj, \^car en la kapo sin trovas
homa okulo, kiu neniam sati\^gas je ar\^gento kaj oro.

 --- Kion do fari en tia okazo? demandis Aleksandro.

 --- \^Sutu iom da tera polvo sur la okulon, tiam \^gi nenion vidos kaj
kompreneble nenion deziros.

   Kiam tiu \^ci konsilo estis plenumita, la telero, sur kiu ku\^sis la
kapo, tuj levi\^gis rapide supren.

 --- Nun mi vidas, diris Aleksandro, ke la sa\^guloj havas prudenton ne
homan, sed Dian!

\begin{flushright}
\footnotesize Tradukis N. \fsc{Kuŝnir.}
\end{flushright}

\smallrule{}
