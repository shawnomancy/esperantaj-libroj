\begin{verse}
\begin{center}
\footnotesize (De S. \fsc{Frug}.)
\end{center}

\begin{center}
\textbf{I}
\end{center}

                     Jam plena estas la festeno\\
                     De bril', sonoro, bru', soleno\\
                     En la palac' de faraono;\\
                     En vic' pentrinda anta\u u vastaj\\
                     Trapezaj tabloj sidas gastoj\\
                     Konvene, en mez' de salono.

                     De l' tron' antikva la apogoj ---\\
                     La sa\^geguloj, astrologoj,\\
                     En silk' brodita envestitaj ---\\
                     La militestroj, generaloj,\\
                     Armitaj kun hakiloj \^stalaj,\\
                     Per ringoj oraj ornamitaj.

                     Sin de kolono \^gis kolono\\
                     Etendas kiel tend' visono\\
                     Tre multekosta, multkolora;\\
                     Kaj de fumilo origita\\
                     \^Cirka\u ue estas disver\^sita\\
                     La timiano bonodora.

                     Al flut', liutoj, barbitonoj\\
                     Per rapidegaj belegsonoj\\
                     Bruega la tintil' ripetas;\\
                     Kaj peza la amfor' konsolas\\
                     Rigardon, koron vekas, bolas\\
                     Kaj mu\^gas, kaj fajrerojn \^{\j}etas.

                     Atentas faraon' sonoron\\
                     De l' himnoj kaj gracian \^horon.\\
                     Senlima l'ordonanto estas\\
                     Je la rigardo luma, klara;\\
                     \^Gi estas fest' neordinara:\\
                     Nomtagon sian li nun festas.

                     De subpotencaj al li princoj\\
                     Donacojn oni el provincoj\\
                     Al li alportas multekostajn ---\\
                     Smeraldojn, perlojn kaj arma\^{\j}ojn,\\
                     Tapi\^sojn, vazojn kaj teksa\^{\j}ojn,\\
                     Ar\^genton, elefantajn ostojn.

                     La gastoj ri\^cajn la donacojn,\\
                     Envenigitajn la palacojn,\\
                     Rigardas kun mireg' kaj ravo\dots\\
                     Tuj, defleksinte pordkovrilon,\\
                     Tremante, en salonon brilan\\
                     Eniris brunviza\^ga sklavo:

                     "A\u uskultu re\^g"', --- la sklavo diris ---\\
                     Mirindaj viroj du aliris\\
                     Al la pordegoj de la korto\dots\\
                     Traflugis preter la gardantoj\\
                     De la kortego, maldormantaj,\\
                     Tuj nevidate kaj sen vorto.

                     Vidante l'mirajn du personojn,\\
                     L'ibisoj flugis for\dots Leonojn\\
                     De l' \^cen' mi lasis --- senkura\^gaj,\\
                     Per la kolharoj balainte\\
                     La polvon, kapon mallevinte,\\
                     La besto-re\^goj la sova\^gaj

                     Timeme iris, proksimi\^gis\\
                     Al ili pa\^s' post pa\^s', stari\^gis\\
                     De l' flankoj sur la piedegoj;\\
                     Netera gardas ilin forto,\\
                     Malkovras vojon e\^c sen vorto\\
                     Al re\^gaj tiuj \^ci pordegoj

                     Sonor', paroloj eksilentas,\\
                     La gastoj kun mireg' atentas\dots\\
                     "Sed, --- krias la\u ute l'ordonanto, ---\\
                     De kie venas tiuj viroj?\\
                     \^Cu portas ili krom' la miroj\\
                     Donacon indan por reganto?"

                     --- Reganto granda, --- sklavo diras, ---\\
                     Per forto de la dioj spiras\\
                     Iliaj la viza\^goj brilaj\dots\\
                     Ne finis sklavo, --- jen subite\\
                     En la salonon nepetite\\
                     Eniris du de Amram filoj.

                     La gastoj vidas, anta\u u l' trono\\
                     Jen staras Mozes, Aarono\\
                     Sur la tapi\^s' purpura, ri\^ca ---\\
                     En simplaj kaj malri\^caj vestoj;\\
                     Severviza\^gaj kaj majestaj\\
                     Kun bastoneg' en man' malri\^ca.

\begin{center}
\textbf{II}
\end{center}

                     Re\^g', rigardinte tre severe\\
                     La novajn gastojn, nun fiere\\
                     Demandis: "Kiu vi? al festoj\\
                     De l' granda re\^go aran\^gitaj\\
                     Kial do venis vi vestitaj\\
                     En tiaj malri\^cegaj vestoj?

                     Al vi malbona fort', magio\\
                     Malfermis pordon\dots Sed do kio!\\
                     Mi kun kontent' en la palacon\\
                     Akceptas \^ciam gastojn tiajn;\\
                     Se indos sor\^c' atenton nian,\\
                     Ricevos ri\^can vi donacon."

                     La kapon alten eklevinte,\\
                     Al Di' humile klini\^ginte,\\
                     Al re\^g' respondis Aarono:\\
                     Sor\^cistoj estas ni neniaj,\\
                     Ne gardas fortoj nin magiaj\dots\\
                     A\u uskultu do, ho faraono!

                     Ni ne el re\^ga familio:\\
                     Ni estas filoj de nacio,\\
                     Per kies manoj humilegaj\\
                     Fari\^gis luksaj viaj landoj,\\
                     En vic' majesta turoj grandaj\\
                     Kaj fortika\^{\j}oj kaj kortegoj.

                     Per kies diligent' kaj peno\\
                     Ekfloris tiu kamp', \^gardeno,\\
                     Nun de dol\^ca\^{\j}o plenigita\dots\\
                     Sed jam sufi\^ce ni suferas!\\
                     Centjara la labor' esperas\\
                     Jam esti nun rekompencita!

                     La Di' de l' dioj kaj Reganto,\\
                     La sola Di', ne \^san\^gi\^ganta,\\
                     Alsendis nin al via trono.\\
                     Lia Spirit' en ni sin trovas:\\
                     \^Gi niajn lipojn, bu\^son movas\dots\\
                     Atentu Lin, ho faraono:

                     "Jehov'\dots" --- "Jehov'? mi lin ne scias," ---\\
                     Al Aarono la\u ute krias\\
                     Fiera faraon' kun rido: ---\\
                     Al mi di' de Moav', Edomo ---\\
                     Konataj estas, sed la nomo\\
                     Jehovo --- kiu estas li do?

                     El malproksima land' kaj limo\\
                     Popoloj \^ciuj por estimo\\
                     Elsendis ja al mi senditojn,\\
                     Sa\^gulojn kaj ar\^hitektistojn,\\
                     Kaj astrologojn kaj maristojn\\
                     Kaj pastrojn kaj sankt-oleitojn.

                     Kun la\u uda kant' palacon mian,\\
                     Kiel pre\^gejon sanktan, dian,\\
                     Eniris \^ciuj humilege,\\
                     Al faraon' donacon sian,\\
                     Kiel oferon sanktan, pian,\\
                     Sur sojlon metis respektege.

                     Kiel malri\^ca devas esti\\
                     Di' via nova, se, krom vesti\\
                     Vin pastrovestojn dis\^siritajn,\\
                     Ne povis li de malpotenco\\
                     Jam per alia rekompenco\\
                     Vin rekompenci elektitajn.

                     La di' de l' sklavoj malfeli\^caj,\\
                     \^Stonistoj tiuj \^ci malri\^caj,\\
                     La sova\^guloj kaj sen scioj,\\
                     De la nuda\^{\j}', mallumo blinda,\\
                     Ho, kiu estas di' ridinda,\\
                     La re\^g' de l' re\^goj, di' de l' dioj?

                     Kaj \^ce la tabloj kun bruego,\\
                     Ektondris la\u ute nunridego;\\
                     Kaj la komuna rida krio\\
                     Ruli\^gis e\^he, la\u ute bruis\\
                     La murojn de l' palaco skuis:\\
                     Kiu li estas, nova dio?

                     Kiu li estas?\dots Tagoj venis, ---\\
                     Kaj tuj ekkonis, ekkomprenis\\
                     La Dion Novan Egiptanoj;\\
                     Sed per turmentoj, malagrabloj\\
                     De la dezert' en brulaj sabloj,\\
                     En profunda\^{\j}' de l' oceanoj.

                     Jam pri la Granda Dio Nova,\\
                     Regant' de l' mondo, \^Ciopova,\\
                     De l' Nil' la ondoj, kantis, bruis;\\
                     Kaj kun tondrego kaj ventegoj\\
                     Inter granitaj la \^stonegoj\\
                     De l' maro ond' kantante fluis\dots

                     Kaj resonante al la kanto\\
                     De l' mar' senfunda kaj bruanta\\
                     Kun sankta mirsciig' konsola\\
                     Per triumfega kri: "Libero!"\\
                     Eksonis la\u ute sur la tero\\
                     Renaski\^ganta la popolo\dots

%M. GOLDBERG.
\end{verse}


\citsc{M. Goldberg.}

\smallrule{}

