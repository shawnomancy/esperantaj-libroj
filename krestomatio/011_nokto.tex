\begin{center}
\footnotesize (De Eube el Odeso.)
\end{center}

   Trankvila stela nokto kaj trankvila senlima maro. Kun meza rapideco
\^sovas sin sur la mara ebena\^{\j}o, unu post alia, tri hispanaj
"karaveloj" de la XV centjaro. Mizeraj, facile rompeblaj \^sipoj;
\^sipanaron oni ne vidas: \^gi jam longe dormas; sur la ferdekoj oni
vidas nur la senmovajn gardstarantojn. Sed nur en la meza \^sipo,
kiu portas sur la rando la surskribon "Santa Maria" kaj admiralan
flagon sur la masto, en la malvasta kajuto brulas lumo, kaj apud la
tablo, kovrita de geografiaj kartoj, sidas en profunda meditado la
granda admiralo de Hispanujo, Kristoforo Kolumbo. Dormo ne volas
tu\^si liajn okulojn. Jam la duan nokton la admiralo ne ku\^sigas
sin, la duan tagon turmentas lin unu sama premanta penso. Ankora\u u
du tagoj --- kaj, se li ne renkontos teron, la \^sipoj iros returne
Hispanujon. Hiera\u u la \^sipanaro arogante sciigis, ke \^gi ne
volas iri plu, kaj kun granda malfacileco la admiralo elpetis de
\^gi limtempon de tri tagoj. La suno mallevi\^gis kaj denove
levi\^gis, --- anta\u u li etendis sin tiu sama senlima ebena\^{\j}o
da trankvila mistera akvo. Estas vero, ke hodia\u u la maro alportis
du verdajn bran\^cetojn, ian pikan arbeta\^{\j}on, tabulon kaj
rabotitan bastoneton --- sendubajn signojn de proksimeco de tero.
Anka\u u sen tio \^ci li scias, ke tero sin trovas anta\u u li, ke
tiu \^ci tero estas proksima, sed \^cu li havos tempon veni al \^gi
en tiuj \^ci du tagoj? \^Guste anta\u u unu monato li renkontis ion
similan je rompita masto: la mara fluo forportis \^gin malproksimen;
post kelkaj tagoj --- akvokreska\^{\j}on kaj unu e\^c kun viva
kankro, kaj birdojn oni vidis ankora\u u anta\u u du semajnoj. La
tero, kompreneble, estas proksima, eble en la interspaco de nur
kelkaj tagoj da vojo, sed kie preni tiujn \^ci kelkajn tagojn? Kaj
ankora\u u anta\u u tri horoj la admiralo kaj kontrolisto de la
ekspedicio, Rodrigo Sanchez, vidis sur la maro iajn briletantajn
lumetojn, sed la \^sipoj, kompreneble, venis jam al tiu loko, kie la
lumoj estis viditaj, --- kaj \^ciam ankora\u u oni ne vidas la
atendatan bordon. Sufi\^ce da eraroj: jam du fojojn, la 25-an de
Septembro kaj la 7-an de Oktobro, ili eraris, vidante imagatan
teron, kaj amba\u u fojojn tiuj \^ci eraroj terure influis la
\^sipanaron. Hodia\u u la tutan tagon la admiralo estis forte
ekscitita, kaj nur la fiereco ne permesis al li pasigi la tutan
tagon sur la ferdeko, avide celante la rigardon en la puran
horizonton. Kaj anta\u u unu horo, \^{\j}etinte la lastan rigardon,
plenan de muta malespero, sur la malluman mason da akvo, li kun
ka\^sita turmenti\^go deiris en sian kajuton. Ankora\u u 48 horoj
--- kaj \^cio estas perdita!

   Jes, \^cio estas perdita. Rompita, neniigita, polvigita estas la
granda celo de lia vaga vivo. Oni ne povas iri, se la piedoj estas
ligitaj, oni ne povas batali, se oni el\^siras el la manoj la lastan
batalilon. La \^cielon kaj la homojn li povas voki kiel atestantojn,
ke li faris \^cion, kion li povis kaj devis fari. Kiam en lia kapo
plene formi\^gis la plano de lia entrepreno, li estis vigla, plena
de fortoj kvardek-ok-jara viro, --- kaj nur nun, kiam grizeco dense
ar\^gentigis liajn harojn, kiam li havas jam la a\^gon de 66 jaroj,
nur nun li ricevis la eblon alpa\^si al la efektivigo de tiu \^ci
plano. Tutan serion da jaroj li sendeklini\^ge iris al la elektita
celo, kaj ne estis tago, en kiu li ne pensus pri \^gi, kaj ne estis
ofero, kiun li ne alportus por \^gi. Li venkis \^ciujn malhelpojn,
kiujn tiel malavare prezentis al li la kruela sorto, li venkis e\^c
la tempon mem. Preska\u u sepdekjara maljunulo, li enportis en sian
aferon fre\^secon de la plej bonaj jaroj de juneco, kaj kun heroa
trankvileco li komencis la ekspedicion, kiu ektimigis e\^c la
kura\^gajn maristojn de Palos. Li faris \^cion de sia flanko, kaj
nun li estas preta pensi, ke super li pendas ia malbeno. Travivi
tion, kion li travivis, penante pri la efektivigo de siaj planoj
 --- estis jam tro multe; sed esti sur la vojo al ilia efektivigo,
trairi na\u u dekonojn de tiu \^ci vojo kaj ekiri returnen, nenion
farinte --- tio \^ci estis tro multe e\^c por liaj fortoj! Tio \^ci
estis nea\u udita, nekredebla moko de la sorto, tio \^ci estis bato,
kiu estis kapabla rompi e\^c lin!

   Kaj en la malvasta kajuto de "Santa Maria" en tiu \^ci turmenta nokto
la admiralo en pensoj travivis sian tutan pasintan vivon. Oni diras,
ke se homo komencas rememori sian tutan travivita\^{\j}on, tio \^ci
estas signo de lia proksima morto. Al li anka\u u minacas morto, se
ne fizika, tiam ankora\u u pli terura --- morala. Povas esti, ke pro
tio \^ci li senvole turnas sin anta\u uen. Jen li, plej maljuna el
la kvar infanoj de Dominiko Kolumbo, itala komercisto de drapoj,
estante ankora\u u preska\u u infano, la unuan fojon iras en la
maron. Li \^{\j}us forlasis la universitaton, li, dekkvarjara gracia
bela knabo. Li veturas sur italaj \^sipoj en Malgrand-Azion,
Anglujon, Portugalujon, al la bordo de Gvineo. La voja\^goj
disvolvas lian fortan naturan sa\^gon kaj kun la jaroj ellaboras el
li bravan mariston. La Genua respubliko konas lin, kiel bonan
mariston, kaj en la milito kun Venecio \^gi faras lin komandanto de
galeroj. Lia nomo estas sufi\^ce konata en Italujo, kaj la Neapola
re\^go faras lin estro de la ekspedicio kontra\u u la piratoj de
Tuniso. Kun sia kutima kura\^go li kondukas siajn \^sipojn kontra\u
u la sova\^gaj korsaroj. Sed la \^sipanaro montri\^gas ne inda je
sia estro: \^gi timas la rabistojn, kaj jam anta\u u la bordo de
Berberujo \^gi faras ribelon. La \^sipanoj postulas, ke oni iru
returnen. La estro de la ekspedicio konsentas plenumi ilian
postulon; li levas la velojn; la \^sipanoj, konvinkitaj, ke ili iras
returnen, trankvili\^gas, kaj la sekvantan matenon ili kun miro
ekvidas la bordon de Afriko. Tiel pasas dudek jaroj, plenaj de
voja\^goj, ventegoj, renkontoj, aventuroj, --- pasas preska\u u
nerimarkite. Havante la a\^gon de tridek kvar jaroj, en la tuta
floro de siaj fortoj, li venas Lisabonon. Tie li laboras super
kartoj kaj globoj geografiaj, kiuj turnas sur sin la atenton de la
instruita mondo, korespondas kun instruituloj. Poste li denove estas
sur la maro: li na\^gas al la malproksimaj bordoj de Islando.
Reveninte Lisabonon, li diras al fra\u ulino Felipa Perestrello
tion, kion li jam longe sentas, kiam li \^sin renkontas: ke li \^sin
amas. Fra\u ulino Felipa sentas tion saman: la majesta figuro de la
kura\^ga maristo, lia neka\^sita rigardo, kiu brilas per tia mirinda
fajro, lia nobla elokventeco --- \^cio en tiu \^ci neordinara
persono de longe profunde frapis \^sian imagon. Kaj \^si fari\^gas
lia edzino. En tiu tempo jam en lia kapo komencas formi\^gi la plano
de lia granda entrepreno. La patrino de lia edzino transdonas al li
la kartojn kaj tagolibrojn de sia mortinta edzo, Bartolomeo Munhiz
Perestrello, gubernatoro de la insulo Porto-Santo, kaj \^si ne povas
kompreni tiun \^gojon, kiun ka\u uzas al \^sia bofilo la amaso da
flavi\^gintaj paperoj. Kaj dume por li tio \^ci estas tuta trezoro,
kaj li tutajn tagojn sidas super ili kaj kun \^gojo trovas en ili
jesigon de liaj teoriaj supozoj. En tiu \^ci sama tempo li da\u
urigas labori super siaj kartoj, partoprenas en la portugalaj
ekspedicioj al la bordo de Gvineo, \^cion a\u uskultas, \^ciujn
eldemandas, korespondas kun sia instruita amiko Pa\u ulo Toscanelli,
al kies leteroj li tiom multe dankas. Poste la malgranda familio
forlasas Lisabonon kaj translo\^gi\^gas sur la insulon Porto-Santo.
Tie \^ci li konstante renkonti\^gas kun voja\^gantoj, kiuj veturas
al la bordo de Gvineo a\u u revenas de \^gi, kaj iliaj komunikoj, la
interparoloj kun ili \^ciam pli kaj pli ekflamigas lin. Lia multjare
ellaborita projekto jam ricevis precizajn, difinitajn formojn: siajn
supozojn, naskitajn per la legado de Aristotelo, Plinio, Strabono,
li fortigis per multo da faktoj, kiujn li kolektis; la karto de
Toscanelli, kiun tiu \^ci alsendis al li, tute konsentas kun liaj
konkludoj. \^Cio estas klara kiel tago: kiu havos sufi\^ce da
kura\^go, por tratran\^ci la misteran maron, kiu etendas sin post la
Gibraltara pordego, tiu trovos la ser\^catan plej proksiman vojon al
Hindujo. Li havos por tio \^ci sufi\^ce da kura\^go, sed li ne havas
la materialajn rimedojn. Oni devas trovi tiujn \^ci rimedojn, oni
devas trovi fortan manon, kiu povus subteni lin. Jen kial li denove
venas Lisabonon, --- kaj de tiu \^ci minuto komenci\^gas la granda
dramo, kiu da\u uras tutan serion da jaroj.

   Jen li, modeste vestita alilandulo, metas sian projekton al la
favora trarigardo de lia re\^ga mo\^sto, la re\^go de Portugalujo.
Tio \^ci estas lia unua pa\^so sur tiu pika\^{\j}a vojo, sur kiun
kondukis lin la malfeli\^co esti naskita neordinara homo. La sorto
volas, ke li, kiu tiel bone ellernis la maron kun \^giaj ondegoj kaj
ventegoj, akiru ankora\u u pli gravan scion --- scion de la homoj.
Lia re\^ga mo\^sto, Johanno II de Portugalujo, ekinteresi\^gis je la
propono de la malri\^ca Genuano kaj difinas komision por trarigardo
de tia nea\u udita projekto. Forta eraro: kion povas kompreni en tio
\^ci tiu \^ci komisio? \^Gi estas dispremita de \^gia grandeco, kaj
al la re\^go oni raportas, ke la projekto de la Genuano Kristoforo
Kolumbo estas sensenca kaj meritas nenian atenton. Kun tio \^ci li
povas ankora\u u paci\^gi: li laboris super sia projekto ne kun la
celo konvinki pri \^gia efektivigebleco malsa\^gulojn. Sed kiam la
re\^go, ne metinte atenton sur la raporton de la komisio,
malalti\^gas al \^stelado, kiam li, postulinte liajn paperojn
kvaza\u u por persona trarigardo, sekrete, sen lia scio preparas
ekspedicion, --- li estas profunde indignigita. Nur unu fojon la
sorto batalas por li: ventego apena\u u ne dronigis la \^sipojn,
kiuj volis \^steli lian gloron, kaj la timema ekspedicio venas
returnen. Lia indigno estas senlima; neniam lia kredo je homoj, je
ilia honesteco ricevis tian fortan frapon. Kaj kiam la re\^go,
komprenanta, kiu staras anta\u u li, proponas al li denove komenci
traktadon, li fiere rifuzas \^ciajn komuniki\^gojn kun tia re\^go.
Ne, jam pli bona estas malri\^cegeco kaj nekonateco, ol esti tiel
malhoneste trompata ankora\u u dek fojojn. Li rifuzas, kaj se ne
venus la morto de la edzino, kiu lasas al li malgrandan filon, li
trankvile paci\^gus kun tiu \^ci unua malsukceso. Sed anka\u u tiun
\^ci baton li elportas kun tia sama vireco, kun kia li renkontadis
kaj elportadis la marajn ventegojn. Tiam li rememoras, ke li estas
Genuano, kaj li forlasas Portugalujon, por proponi la efektivigon de
sia granda plano al tiu, kiu la plej multe tion \^ci meritas --- al
sia patrujo.

   Kaj jen li denove estas en la patrujo. Multo da jaroj pasis de la
tempo, kiam li forlasis la patran urbon, kaj kun forta espero li nun
eniras en \^gin. Kvankam li estas malri\^ce vestita, kvankam li
estas preska\u u almozulo, sed li alportas al la urbo re\^gan
donacon: li proponas al \^gi sian projekton. Sed la patrujo tute ne
povas nun pensi pri li. La urbon Kaffa en Krimo okupis la turkoj kaj
al la Genua \^siparo minacas ekstermo. Krom tio la projekto estas
tute fantazia kaj meritas nenian atenton. Kaj se en Portugalujo oni
volis \^cirka\u u\^steli lin, en la patrujo la renkonto montri\^gas
ankora\u u pli malbona: tie almena\u u trovi\^gis homoj, kiuj povis
lin kompreni, --- tie \^ci lia propono estas renkontita per
malsa\^ga mokado. La espero distri\^gis, kiel fumo, kaj montri\^gas,
ke li absolute ne havas, kion fari en la patrujo. Kaj kun maldol\^co
en la koro li forlasas Genuon por nova migrado kaj novaj provoj.

   \^Ciuj konas la mal\^gojan historion de tiuj \^ci provoj, sed kiu scias,
kion li travivis en la tempo de ili? Li estas en Hispanujo, en la
\^cirka\u ua\^{\j}oj de Palos. Laca kaj \^cirka\u u\^sirita,
preska\u u duonnuda, li vagas kun la malgranda filo sur la brakoj
kaj kun brulanta turmenti\^go a\u udas, kiel la infano, premata de
malsato kaj soifo, balbutas pri pano kaj akvo. Dank' al Dio, jen
estas mona\^hejo de franciskanoj, kaj eble tie li povos ion elpeti
por la malfeli\^ca infaneto. Kaj li, kiu iam komandis la \^sipojn de
la Genua respubliko, kiel almozulo haltas apud la pordego de la
mona\^hejo kaj pelas la pordiston pri kru\^ceto da akvo kaj peco da
pano por sia infano. En tiu \^ci minuto preteriras la abato de la
mona\^hejo, la nobla Juan Perez de Marchena. Estu por \^ciam benita
tiu \^ci okaza renkonto, kiu donis al li la plej bonan amikon, sen
kies helpo li eble neniam ekvidus tiun \^ci maron! Se li atingos
sian celon kaj plej mallonga vojo al Hindujo estos eltrovita, la
mondo dankos tion \^ci tiom same al Perez de Marchena, kiom al li,
Kristoforo Kolumbo. Ho, li rekompencos inde tiun \^ci luman noblan
kapon! Jen li, frapita de la majesta ekstera\^{\j}o de la \^cirka\u
u\^sirita migranto, haltas kaj demandas lin, kiu li estas. Kaj la
migranto komencas rakonti pri sia mal\^goja migrado kaj ne sen miro
trovas en la mona\^ho-franciskano homon, kiu tute lin komprenas. Oni
gastame invitas lin esti gasto en la mona\^hejo; la abato sendas en
la urbon Palos inviti sian amikon, la kuraciston Fernandez, kaj en
la modesta Andaluza mona\^hejo komenci\^gas varmaj priparoladoj de
lia projekto. Jes, tiuj \^ci homoj komprenas lin, anta\u u tiaj
homoj estas inde paroli. Li entuziasmigas ilin per sia flameco kaj
elokventeco; ili kune pentras belajn pentra\^{\j}ojn de la
estonteco, kreita per la efektivigo de lia ekspedicio, kaj Perez de
Marchena direktas lin al la re\^ga kortego. Lia filo restos tie \^ci
en la mona\^hejo por edukado, kaj li kun la letero de la abato al
Talabera, la konfesprenanto de la re\^gino, devas veturi Kordovon.
Talabera estas persono influa, kaj la amikoj disiras, plenaj de la
plej lumaj esperoj.

   Li venas Kordovon kaj prezentas sin al Talabera. Neniam li forgesos,
per kia fiera rigardo renkontis lin la konfesprenanto de la
re\^gino. Li estas vestita malri\^ce kaj malbone; lia parolo tuj
montras en li alilandulon: sendube tio \^ci estas unu el tiuj
multegaj aventuristoj, kiuj nur ser\^cas okazon kapti fi\^sojn en
malklara akvo. Lia projekto? Talabera bonkore a\u uskultas tiun \^ci
sensenca\^{\j}on kaj li ne dubas, ke la kreinto de la projekto mem
ne povas kredi tian absurdon, t. e. ke anta\u u li staras simple
lerta \^carlatano. Sed li tro bone konas la homan animon, por
permesi, ke oni lin tiel senceremonie trompu. Ne, li absolute per
nenio povas servi al la petanto: la projekto estas infane naiva, kaj
li kompreneble ne povas raporti pri tiaj aferoj al la re\^gaj
mo\^stoj. \^Cu Juan Perez pensis pri tia akcepto, kiam li skribis
sian leteron al Talabera? Ankora\u u unu fojon oni devas humili\^gi
al la forto de la cirkonstancoj kaj al la forto de la malklereco kaj
ser\^ci aliajn vojojn al la re\^ga trono. La someron kaj a\u utunon
li lo\^gas en Kordovo, faras konatecojn kaj trovas e\^c amikojn por
sia afero: la edukanton de la re\^gaj infanoj, lian fraton, la
nuncion de la papo kaj la regnan financestron de Kastilujo. Tiuj
\^ci personoj bone konas la kortegon, \^ciujn \^giajn enirojn kaj
elirojn. Estas necese prezenti lin al la \^cefepiskopo de Toledo,
Pedro Gonzalez de Mendoza, kies vorto \^ce la kortego havas pli
grandan signifon, ol la opinioj de Talabera. La gere\^goj nenion
entreprenas sen la konsilo de tiu \^ci grava persono. Tio \^ci estas
la plej certa vojo, kaj \^gi efektivo alkondukas al la dezirita
celo: Mendoza aran\^gas por li a\u udiencon.

   Trankvile kaj fiere li aperas anta\u u la re\^ga paro. Li estas malri\^ce
vestita, sed lia parolo spiras konscion de sia propra indo, liaj
certaj klarigoj montras en li rimarkindan mariston, e\^c
instruitulon, kiu longe kaj multe pensis pri tio, kion li parolas.
Li nenion petas: li faras al Kastilujo kaj Leonujo preska\u u
fabelan proponon --- ri\^ci\^gi per plej facila rimedo: per
efektivigo de lia projekto. Oni donu al li la eblon iri en la maron
--- kaj li solvos la eternan problemon, li trovos tiun \^ci novan
filozofian \^stonon --- la plej mallongan vojon al Hindujo, --- kaj
Hispanujo fari\^gos la plej ri\^ca lando de la mondo. Kaj tiu \^ci
propono estas farata tiel simple, tiel memfide, ke anta\u u la
gere\^goj staras a\u u frenezulo, a\u u efektive eksterordinara
homo. Tiun \^ci demandon devas decidi instruitaj homoj; Talabera
devas kunvoki komision, por prikonsiderado de tia neordinara propono
kaj raporti al la re\^gaj mo\^stoj pri la rezultatoj de la
konsili\^go. La plua iro do la afero dependos de la respondo de la
komisio.

   Kaj jen en la mona\^hejo de la Sankta Stefano la komisio komencas
la esploradon de lia projekto. Dio vidas, ke li estas bona katoliko:
\^cu li ne pensas pri konvertado de idolistoj al kristaneco? \^cu li
ne revas alporti \^ciujn siajn estontajn ri\^cecojn al ankora\u u
pli granda celo --- liberigo de la tombo de Dio? Sed la membroj de
la komisio estas profundaj malkleruloj. Ili citas Lactantius'on kaj
diras, ke la projekto kontra\u uparolas al li, sekve \^gi estas
sensenca kaj e\^c hereza. Jes, Lactantius! Sendube li estis granda
homo kaj bona kristano, sed tio \^ci ne malhelpis al li skribi
malsa\^ga\^{\j}ojn pri la formo de la tero. Li legis kaj relegadis
tiun \^ci malfeli\^can lokon el la III libro de la "Diaj
instruoj"; per tiu \^ci loko oni pikadis al li la okulojn, \^gi
apena\u u ne mortigis lian grandan entreprenon; li scias \^gin
parkere. "Kiu estas tiom senprudenta, ke li povas kredi, ke
ekzistas homoj, kiuj iras kun la piedoj supre kaj tenas la kapon
malsupren; ke \^cio, kio tie \^ci ku\^sas, tie pendas, ke la herbo
kaj arboj tie kreskas malsupren, ke pluvo kaj hajlo falas tie de
malsupre supren?" Kaj jen tiu \^ci mizera rimarko apena\u u ne
mortigis la celon de lia tuta vivo! Poste venas cito el A\u
ugustinus: "instruo pri antipodoj estas nekonforma al la principoj
de la religio, \^car tio \^ci signifus, ke ekzistas homoj, kiuj ne
devenas de Adamo, \^car estas neeble, ke ili estu transirintaj trans
la oceanon, kiu \^cirka\u uas la tutan teron!"

   Poste venas jam simple malsa\^ga\^{\j}oj, ekzemple de tia speco, ke
la ekspedicio ne povos veni returnen, \^car la \^sipoj, dank' al la
elfleksiteco de la tero, devus na\^gi de malsupre supren. Kaj li
devis disputi kun tiuj \^ci sova\^gaj malkleruloj kaj ankora\u u
danki Dion, ke oni ne kulpigis lin je herezo kaj ke prosperis al li
konvinki almena\u u kelkajn personojn. Sed la plej granda parto de
la membroj de la komisio firme decidas, ke la projekto senkondi\^ce
estas absurda. Kaj dum da\u uras la disputoj inter ili, la
gere\^goj, okupitaj je milito, forveturas Kordovon, kaj la
konsili\^goj de la komisio \^cesas.

   Nun komenci\^gas la tagoj de liaj plej pezaj suferoj. Li sekvas post
la kortego. Denove li komencas eldonadon de kartoj, kaj jaro post
jaro, kvar malfacilajn jarojn, iel trabatas sin, ne forlasante sian
celon, atendante, kiel almozon, kiam al la re\^gaj mo\^stoj pla\^cos
rememori pri la malri\^ca Genuano. Jes, kvar senfinaj, teruraj
jaroj, plenaj de transiroj de espero al malespero, de revo pri
estontaj ri\^cecoj al malsato, al malfacila batalado pro peco da
pano. Li sendas sian fraton Anglujon, al la re\^go Henriko VII, por
fari en lia nomo proponon elser\^ci vojon al Hindujo. Sed la frato
revenas kun mal\^goja respondo. Ankora\u u unu bato kaj ankora\u u
unu fojon oni devas humili\^gi. Grizeco, kiu jam longe montri\^gis
en liaj haroj, \^ciam pli kaj pli blankigas lian noblan kapon, sed
anta\u u lia energio estas senforta e\^c la tempo mem. Li havas jam
la a\^gon de 65 jaroj, kiam la re\^go denove kunvokas komision. Kaj
Talabera kun plena plezuro faras al la re\^go raporton en \^gia
nomo: la projekto de Kristoforo Kolumbo estas simple
sensenca\^{\j}o. Sed la konvinkitaj de la membroj de la komisio fine
elpa\^sas por li. Ili aperas anta\u u la re\^go kaj defendas lian
planon, kaj eble nur dank' al tiu \^ci defendo la re\^go promesas
denove trarigardi lian proponon --- post la fino de la milito kun la
Ma\u uroj.

   Tio \^ci estas super \^ciuj homaj fortoj! Malespero lin atakas: oni
povas pensi, ke tiuj \^ci sinjoroj esperas, ke li neniam mortos. Li
turnas sin al la ri\^ca grandsinjoro Medinaceli, sed tiu, kvankam
e\^c kunsentas al li, ne povas armi ekspedicion: tio \^ci estus
konkuri kun la re\^go mem. Nur li kaj la \^cielo scias, kion li
travivis en la da\u uro de tiu tempo! Fine el tiu \^ci mallumego
ekbriletas radio da lumo: la re\^go de Francujo afable invitas lin
al si. Lia energio denove veki\^gas en sia tuta forto. Sur hispanaj
a\u u francaj \^sipoj --- li enpenetros en tiun \^ci kvaza\u u
ensor\^citan por li maron! Li rapidas al Perez. Anta\u u ses jaroj
li forlasis la Palosan mona\^hejon, kaj forte maljuni\^ginta li nun
denove eniras en \^gin. Granda estas la mal\^gojo de la nobla Perez,
kiam li a\u udas la longan rakonton de sia amiko pri liaj
malsukcesoj. \^Cu efektive lia patrujo montri\^gos tiel blinda, \^cu
efektive \^gi permesos, ke alia nacio efektivigu la grandan planon
de lia amiko? Tio \^ci ne devas esti! Kaj li petas Kolumbon atendi,
\^gis li faros la lastan provon. Iam li estis konfesprenanto de la
re\^gino, kaj nun li skribe petas \^sin permesi al li veni al la
kortego por defendi la projekton de sia amiko. Ne sen maltrankvilo
la amikoj atendas la respondon de la re\^gino. Fine \^gi estas
ricevita: al Perez de Marchena estas permesite veni al la kortego
kaj Kolumbon oni petas atendi iom kun sia forveturo. Tiam Perez
aperas anta\u u la re\^gino. Kun flama elokventeco li defendas la
planon de sia amiko. Li montras al la re\^gino la grandegajn sekvojn
de la malkovro de nova vojo; li anta\u udiras al \^si, ke ondegoj da
oro enver\^si\^gos Hispanujon; li prezentas al \^si, la pia
re\^gino, pentra\^{\j}on de vasta kristana predikado, parolas al
\^si pri la milionoj da novaj servantoj de Kristo kaj pri tio, ke la
savo de tiuj \^ci milionoj da animoj apartenos anta\u u \^cio al
\^si kaj jam poste al Kristoforo Kolumbo, kiu montros la vojon al
tiuj \^ci pereantaj animoj. Anta\u u tia pentra\^{\j}o la re\^gino
ne povas sin reteni, kaj \^si difinas komisiulojn por trakti kun
Kolumbo.

   Pasis multe da tempo de la tago, kiam li respondis per fiera rifuzo
la proponon rekomenci traktadon kun la re\^go de Portugalujo, sed,
kiel anta\u ue, anta\u u la komisiuloj nun aperas ne petanto, sed
batalanto por la majesteco de sia ideo. Li metas siajn kondi\^cojn:
li, Kristoforo Kolumbo, estos farita granda admiralo de Hispanujo,
li estos farita vicre\^go de la nove malkovrotaj landoj, li ricevos
la dekonon de \^ciuj multekostaj produkta\^{\j}oj, kiuj estos
akiritaj en tiuj \^ci landoj. La rajton por siaj postuloj li pagis
per kara kosto, kaj vane la indignigita Talabera krias pri avideco
kaj aroganteco de la Genua \^carlatano. Li ne miras Talaberan. Sed
li estas frapita kaj profunde indignigita, kiam oni raportas al li
en la nomo de la re\^gino, ke liaj postuloj estas iom tro grandaj
kaj ke estus bone, ke li ilin malgrandigu. La re\^gino, kiel
\^sajnas, pensas, ke anta\u u \^si staras vendisto, kiu postulas tri
fojojn pli multe, por ke li havu, de kio fari rabaton. Sed li scias
pli ol \^si, kion li povas postuli, kaj li petas raporti al la
re\^gino, ke li povas konsenti neniajn cedojn. Lia rifuzo estas
preska\u u rifuzo de la celo de sia vivo; \^gi estu tiel, sed li ne
konsentos malnoblan malalti\^gon. Kaj li forveturas, kun firma
decido forlasi por \^ciam tiun \^ci sendankan landon.

   Sed dum li, kun premanta mal\^gojo en la koro, veturas sur la vojo
al Grenado, en Santa Fe estas decidata lia sorto. La regna
financestro de Kastilugo kaj Sant-Angel, la kolektisto de la depagoj
por pre\^gejoj --- du sa\^gaj homoj --- decidas alpa\^si al la lasta
rimedo. Ili petas a\u udiencon \^ce la re\^gino kaj petegas \^sin ne
rifuzi la postulojn de la fiera Genuano. Ili estas financistoj, kaj
neniu povas montri al Isabella pli bone ol ili \^ciujn profitojn de
la efektivigo de lia projekto. Kredeble iliaj paroloj estas
konvinkaj, se la entuziasmigita re\^gino ekkrias, ke \^si prodonos
siajn briliantojn, por armi ekspedicion. Tiam Sant-Angel respondas
al \^si, ke tia ofero ne estas bezona, \^car li estas preta doni al
la re\^gino rimedojn prunte el la pre\^gejaj enspezoj. Kaj el Santa
Fe al Grenado jam rapidas kuriero, atingas la malri\^ce vestitan
voja\^ganton apud Grenado kaj raportas al li, ke \^sia mo\^sto la
re\^gino de Kastilujo petas lin reveni en Santa Fe. Denove la
malvarma malespero cedas lokon al ekbrilo de luma espero, kaj li
turnas la \^cevalon returnen. Li estas en Santa Fe, li estas
akceptita de la re\^gino; la 17-an de Aprilo de la jaro 1492 li
subskribas la kontrakton. Nun li estas sur la vojo al sia celo. Dum
da\u uras la preparoj al la ekspedicio, li estas bona gasto \^ce la
gere\^goj; kun plezuro ili a\u uskultas lin, kiam li, plena de
junula flameco, parolas al ili pri la konvertado de la idolistoj de
Azio al kristaneco kaj modeste komunikas al ili siajn esperojn
efektivigi ankora\u u pli altan celon de sia vivo --- oferi la
kolektotajn de li ri\^cecojn por la liberigo de la tombo de Kristo.
La pia re\^gino a\u uskultas kun profunda intereso sian noblan
kunparolanton. Tiuj \^ci tagoj almena\u u iom rekompencas lin por la
malfacila pasinta\^{\j}o. Kaj fine venas la longe dezirita minuto:
li kolektas en Palos bravulojn por siaj tri karaveloj. Sed la bravaj
maristoj de Palos estas timigitaj de lia ekspedicio. Kun helpo al li
venas la kura\^ga maristo Marteno Alonzo Pinzon; li entuziasmigas
per sia ekzemplo la \^sipanojn, kaj en vendredo, la 3-an de A\u
ugusto, la tri karaveloj fine forpu\^sas sin de la bordo, plenigita
de popolo, kaj sub krioj de bondeziroj de la rigardantoj ili
kura\^ge elveturas renkonten al la malluma estonteco.

   Fine efektivi\^gis lia revo, fine li estas sur la maro! Sed anka\u u sur
la maro, kiel sur la tero, la sorto da\u urigas metadi al li
malhelpojn. La direktilisto de "Pinta", suba\^cetita de \^gia
mastro, difektas la direktilon, por ke la karavelo reiru Paloson.
Kvar semajnoj pasas por la bonigado de la karaveloj apud la Kanariaj
insuloj; la \^sipanoj krias, ke tio \^ci estas malbona anta\u
usigno. \^Cio timigas tiujn \^ci malsa\^gulojn --- e\^c la vulkano
Teneriffa. Ju pli profunde li penetras en la oceanon, des pli granda
\^ciufoje estas la teruro de la \^sipanoj. En la interspaco de
ducent mejloj de la insulo Ferro komenci\^gas deklini\^go de la
magneta montrilo --- nova teruro por la \^sipanoj. La admiralo devas
malalti\^gi al la plej malagrabla ruzo: konduki du taglibrojn, unu
--- veran --- por la gere\^goj, la duan --- kun pli malgrandigitaj
mezuroj de la traveturita vojo --- por la timema \^sipanaro. La
malkontenteco de tiu \^ci lasta kreskas kun \^ciu tago, kaj jen
hiera\u u \^gi akceptis la formon de neka\^sita ribelo. Kion povas
fari li sola kontra\u u cent dudek homoj? Kiel dronanto kaptas
pajleton, tiel li ser\^cas savon en tritaga limtempo. Se en la da\u
uro de tiuj \^ci 72 horoj li ne ekvidos teron --- li ekiros
returnen. \^Cio estas metita sur la karton: lia tuta vivo. Dek ok
jarojn li kun fera energio iradis al sia alta celo, dek ok jarojn,
plenigitajn de obstina laborado de la kapo, malsukcesoj,
disrevi\^goj, malri\^cegeco kuj e\^c mokoj kaj ofendoj. Jes, e\^c
ofendoj. \^Cu en Genuo kaj en Hispanujo oni ne nomadis lin revisto
kaj frenezulo? \^Cu e\^c la strataj buboj de Kordovo, renkontante
lin, ne montradis sur sian kapon, amuzi\^gante je li, kiel je
frenezulo? La envio kaj senka\u uza blinda malamo de liaj malamikoj
iris ankora\u u pli malproksimen: lin, kiu oferis sian tutan vivon
al granda afero, lin oni malhonoris per la nomo \^carlatano! Sed li
\^cion elportis, \^cion! La malri\^cegeco, malsato, fatalaj
malsukcesoj, sova\^ga malklereco, mokoj, kalumnioj --- \^cio tio
\^ci estis venkita de la superhoma forto de lia persisteco, kaj fine
venis tiu granda tago, kiam li levis la velojn en la haveno de
Palos. Granda tago. Kaj jen nun, kiam \^ciuj rigardoj estas
direktitaj sur lin, kiam la tuta mondo kun pasia scivoleco atendas
la rezultaton de la heroa entrepreno, kiam restas fari nur kelkajn
pa\^sojn por pravigi siajn esperojn kaj la esperojn de siaj nemultaj
varmegaj amikoj, por kroni tiujn \^ci dek ok jarojn --- nun li estos
devigita ekiri returnen! Kun honto kaj malhonoro li revenos
Hispanujon, kie la ekflamema re\^gino povas \^{\j}eti al li en la
viza\^gon la nomon de trompisto, kie la malamikoj mokados lin, la
mizeran aventuriston, kaj jam neniam, neniam li havos la eblon
efektivigi sian amatan revon! Malhonoro, malri\^cegeco kaj rompita
vivo --- jen estas \^cio, kio lin tie atendas, --- kaj \^cio tio
\^ci estas nur tial, ke tiu \^ci timema brutaro, kiu ku\^sas sur la
ferdeko, estas pli forta ol li sola, nur tial, ke en tiu \^ci okazo
maldelikata besta forto havas pli grandan signifon, ol la spirita
forto de genio! Se en liaj manoj sin trovus la egido de Ze\u uso, se
per unu ekbato li povus ekstermi tiun \^ci tutan malkura\^gan
bestaron, li ekstermus \^gin kaj li sola irus anta\u uen. Sed la
forto estas sur ilia flanko, kaj post du tagoj la karaveloj ekiros
returnen. La admiralo kun malespero apogis la kapon sur la manojn,
kaj la pla\u udado de la akvo sur la flankojn de "Santa Maria"
sonis en liaj oreloj kiel funebra sonorado.

   Jes, \^cio, \^cio pereis. La tuta vivo, \^ciuj laboroj kaj esperoj
--- \^cio estas neniigita, kaj neniigita \^guste tiam, kiam \^cio
kuni\^gis, por pravigi ilin, por elmontri la \^gustecon de liaj
supozoj! Serio da faktoj pruvis, ke tero sin trovas en la okcidento,
la transflugo de birdoj klare parolas pri la proksimeco de tiu \^ci
tero, la hodia\u uaj trovoj povus konvinki e\^c liajn plej obstinajn
kontra\u uulojn. \^Cio diras, ke li divenis la grandan sekreton de
la oceano, --- kaj en tiu \^ci sama tempo li, kiel ligita aglo,
devas tordi sin en siaj ligiloj, ne povante eksvingi la potencajn
flugilojn, por \^{\j}eti sin renkonte al tiu \^ci tero, kiun oni
forprenas de li! \^Cu tiuj \^ci malsa\^guloj komprenas, kion ili
faras, \^cu ili komprenas, ke returni\^go \^guste nun estas por li
pli malbona ol morto, --- \^cu ili povas tion \^ci kompreni? \^Cu
ili povas lin kompreni?! Sur la tuta tera globo lin komprenas nur
kelkaj dekoj da kapoj tiel penetremaj, kiel li, kaj ekster ili
--- seninterrompa muro da malamikoj, enviantoj kaj malkleruloj, kiuj
malamas \^ciun provon eliri al lumo el tiu sova\^ga nokto, al kiu
ili memvole sin kondamnis. Kiel \^goje krios tiu \^ci sova\^ga
amaso, kiam li malhonore venos returne, per kiaj mokoj \^gi
super\^{\j}etos lin, kun kia triumfo \^gi kriados, ke \^gi estis
prava kaj ke li estas simple frenezulo! Sed ne grava ankora\u u
estas tio, ke li estos premita en koton kaj neniigita: terura estas
tio, ke lia ideo, lia infano, elnutrita per lia \^svito kaj sango,
devos perei kaj perei \^guste tiam, kiam \^gia praveco estas metita
ekster \^cian dubon. \^Cu efektive li devas eltrinki ankora\u u tiun
\^ci kalikon? \^Cu estis ne sufi\^ca tio, kion li elsuferis --- kaj
por kio elsuferis? por ke en la anta\u utago de granda venko fali
kiel ofero de blinda besta forto?! \^Cu efektive estis absolute
nenia senco en tiuj tentoj, tra kiuj lin kondukis la enigma sorto,
kaj vane li kondamnis sin al la suferoj de vaga vivo, malri\^cegeco
kaj mokado de malsa\^guloj?! Ne povas esti! Ankora\u u restis du
tagoj, ankora\u u ne \^cio pereis. Ankora\u u kvardek ok horoj
restas \^gis la fino de tiu \^ci batalo kun la lin persekutanta fato
--- kaj li batalos \^gis la fino mem! La admiralo fiere sin
elrektigis\dots kaj subite li ek\^sanceli\^gis, kvaza\u u en la
tablon, apud kiu li sidis, ekbatus tondra sago. De la anta\u ua
karavelo ektondris pafego.

   Tero!!!

   Estis la dua horo je mateno, la 12-an de Oktobro de la jaro 1492. Sur
la ferdeko de "Santa Maria" oni eka\u udis pa\^sojn de homoj,
veki\^gintaj el dormo, kaj subite sur \^ciuj tri \^sipoj ekbruis la
\^goja krio: "Tero, tero!"

   Li venkis. La admiralo --- kaj de tiu \^ci minuto vicre\^go --- malrapide
falis sur la genuojn, kaj neniam en la vivo liaj lipoj murmuris tian
varmegan pre\^gon. "Tero! tero!" \^goje ripetadis dekoj da
vo\^coj. La admiralo malla\u ute pre\^gis. "Te Deum laudamus, Te
Dominum confitemur", murmuris la lipoj de la admiralo. "Salve
Regina" \^hore bruis la \^sipanaro. Kaj kiam, klara kaj majesta, la
admiralo aperis sur la ferdeko, la \^sipanaro kun estimego liberigis
al li vojon, la \^sipanoj salutis lin per entuziasmaj krioj kaj,
falante anta\u u li sur la genuojn, kisis liajn manojn. Kaj li
staris inter ili, kiel duondio, lumigita de la unuaj radioj de la
levi\^ganta suno, salutante per trankvila rideto "sian" teron, kaj
en liaj okuloj brulis la fajro de genio, kiu venkis sian sorton. La
nokto de lia vivo fini\^gis, kaj komenci\^gis hela, solena tago.

\smallrule{}
