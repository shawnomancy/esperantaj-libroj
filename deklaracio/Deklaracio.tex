\documentclass[11pt]{article}
\usepackage[margin=2cm]{geometry}
\usepackage[esperanto,english]{babel}
\newcommand\cc[1]{{\hfill #1 \hfill}}
\usepackage{longtable}
\usepackage{parskip}
\usepackage{ccicons}
\usepackage{fontspec}
\usepackage{changepage}
\setmainfont{Open Sans}
\newfontfamily\traduko{Libre Baskerville}
\newcommand{\righty}[1]{\begin{adjustwidth}{100pt}{0pt}\it\traduko #1\end{adjustwidth}}
\usepackage{tabu}
\usepackage{pgfornament} 
\newcommand{\sectionline}{
\begin{center}
\pgfornament[width=0.5\textwidth]{89}
\end{center}}
\setlength\parindent{0pt}
\begin{document}

\begin{longtabu} to\textwidth{>{\LARGE}X[1,c]@{\hspace{3em}}>{\LARGE\it\traduko}X[1,c]}
Deklaracio & Declaration \\[1ex]
pri la esenco & about the essence \\[1ex]
de la Esperantismo & of “Esperantism”, \\[1ex]
de la 9-a Aŭgusto & 9 August \\[1ex]
1905 &  1905 \\
\end{longtabu}

\textbf{Ĉar} pri la esenco de la Esperantismo multaj havas tre malveran ideon, tial ni subskribintoj, reprezentantoj de la Esperantismo en diversaj landoj de la mondo, kunvenintaj al la internacia kongreso Esperantista en Boulogne-sur-Mer, trovis necesa laŭ la propono de la aŭtoro de la lingvo Esperanto doni la sekvantan klarigon:

\righty{\textbf{Whereas} many have a very false idea of the essence of Esperantism, therefore we the undersigned, representatives of Esperantism in many countries of the world, convened at the international congress of Esperantists in Boulogne-sur-Mer, deemed it necessary, according to the proposal of the author of Esperanto, to give the following explanation:}

1. La Esperantismo estas penado disvastigi en la tuta mondo la uzadon de lingvo neŭtrale homa, kiu «ne entrudante sin en la internan vivon de la popoloj kaj neniom celante elpuŝi la ekzistantajn lingvojn naciajn», donus al la homoj de malsamaj nacioj la eblon kompreniĝadi inter si, kiu povus servi kiel paciga lingvo de publikaj institucioj en tiuj landoj, kie diversaj nacioj batalas inter si pri la lingvo, kaj en kiu povus esti publikigataj tiuj verkoj, kiuj havas egalan intereson por ĉiuj popoloj.

\righty{Esperantism is the effort to disseminate worldwide the use of a neutrally human language, which while “not intruding in the internal life of the peoples and never aiming to replace the existing national languages”, would give to people of different nations the possibility to understand each other, which could serve as a peacemaking language of public institutions of those countries where diverse nationalities fight among themselves about language, and in which those works could be published, which have equal interest for all peoples.}
 
Ĉiu alia ideo aŭ espero, kiun tiu aŭ alia Esperantisto ligas kun la Esperantismo, estos lia afero pure privata, por kiu la Esperantismo ne respondas.

\righty{Every other idea or hope, which this or that Esperantist associates with Esperantism, will be purely their private affair, for which Esperantism has no reply.}

2. Ĉar en la nuna tempo neniu esploranto en la tuta mondo jam dubas pri tio, ke lingvo internacia povas esti nur lingvo arta, kaj ĉar el ĉiuj multegaj provoj, faritaj en la daŭro de la lastaj du centjaroj, ĉiuj prezentas nur teoriajn projektojn, kaj lingvo efektive finita, ĉiuflanke elprovita, perfekte vivipova kaj en ĉiuj rilatoj pleje taŭga montriĝis nur unu sola lingvo Esperanto, tial la amikoj de la ideo de lingvo internacia, konsciante ke teoria disputado kondukos al nenio kaj ke la celo povas esti atingata nur per laborado praktika, jam de longe ĉiuj grupiĝis ĉirkaŭ la sola lingvo Esperanto kaj laboras por ĝia disvastigado kaj riĉigado de ĝia literaturo. 

\righty{Whereas already at the present time no researcher in the world doubts that an international language can only be an artificial language, and because out of all the many attempts made during the last two centuries, all have presented only theoretical projects, and only the one language Esperanto has proven to be an effectively complete language, tested in every way, perfectly viable, and most suitable in all relations; therefore, the friends of the idea of an international language, conscious that theoretical argument will lead to nothing, and that the goal can be attained only through practical work,  for some time already have gathered around the one language Esperanto and work for its dissemination and enrichment of its literature.}

3. Ĉar la aŭtoro de la lingvo Esperanto tuj en la komenco rifuzis unu fojon por ĉiam ĉiujn personajn rajtojn kaj privilegiojn rilate tiun lingvon, tial Esperanto estas «nenies propraĵo», nek en rilato materiala, nek en rilato morala. 

\righty{Whereas the author of the language Esperanto, at the very beginning, declined once and for all, all personal rights and privileges related to the language, therefore Esperanto is “no one’s property”, neither in material affairs, nor in moral affairs.}

Materiala mastro de tiu ĉi lingvo estas la tuta mondo kaj ĉiu deziranto povas eldonadi en aŭ pri tiu ĉi lingvo ĉiajn verkojn, kiajn li deziras, kaj uzadi la lingvon por ĉiaj eblaj celoj; kiel spiritaj mastroj de tiu ĉi lingvo estos ĉiam rigardataj tiuj personoj, kiuj de la mondo Esperantista estos konfesataj kiel la plej bonaj kaj plej talentaj verkistoj en tiu ĉi lingvo. 

\righty{The material master of this language is the entire world, and everyone who desires to can publish in, or about, this language all works which they please, and can use the language for all possible purposes; as spiritual masters of this language shall always be regarded those persons, who of the Esperantist world are acknowledged as the finest and most talented writers in the language.}

4. Esperanto havas neniun personan leĝdonanton kaj dependas de neniu aparta homo. Ĉiuj opinioj kaj verkoj de la kreinto de Esperanto havas, simile al la opinioj kaj verkoj de ĉiu alia Esperantisto, karakteron absolute privatan kaj por neniu devigan.  La sola unu fojon por ĉiam deviga por ĉiuj Esperantistoj fundamento de la lingvo Esperanto estas la verketo «Fundamento de Esperanto», en kiu neniu havas la rajton fari ŝanĝon.  

\righty{Esperanto has no personal lawgiver and depends on no particular person. All opinions and works of the creator of Esperanto have, similarly to the opinions and works of every other Esperantist, an absolutely private character and are obligatory for no one.  The single, once and for all obligatory for all Esperantists, foundation of the language Esperanto is the book \emph{Fundamento de Esperanto}, to which no one has the right to make a change.}

Se iu dekliniĝas de la reguloj kaj modeloj donitaj en la dirita verko, li neniam povas pravigi sin per la vortoj «tiel deziras aŭ konsilas la aŭtoro de Esperanto». 

\righty{If someone deviates from the rules and models given in the said work, they can never justify themselves with the words “so desires or advises the author of Esperanto”.}

Ĉiun ideon, kiu ne povas esti oportune esprimita per tiu materialo, kiu troviĝas en la «Fundamento de Esperanto», ĉiu Esperantisto havas la rajton esprimi en tia maniero, kiun li trovas la plej ĝusta, tiel same, kiel estas farate en ĉiu alia lingvo. Sed pro plena unueco de la lingvo al ĉiuj Esperantistoj estas rekomendate imitadi kiel eble plej multe tiun stilon, kiu troviĝas en la verko de la kreinto de Esperanto, kiu la plej multe laboris por kaj en Esperanto kaj la plej bone konas ĝian spiriton.

\righty{Every idea, which cannot be conveniently expressed by the material found in \emph{Fundamento de Esperanto}, every Esperantist has the right to express in that manner which they find the most correct, in the same way as is done in every other language. But for complete uniformity of the language, to all Esperantists it is recommended to emulate as much as possible that style which is found in the work of the creator of Esperanto, who has worked the most for and in Esperanto, and who best knows its spirit.}

5. Esperantisto estas nomata ĉiu persono, kiu scias kaj uzas la lingvon Esperanto tute egale por kiaj celoj li ĝin uzas. Apartenado al ia aktiva societo Esperantista por ĉiu Esperantisto estas rekomendinda, sed ne deviga.

\righty{Every person is called an Esperantist, who knows and uses the language Esperanto, entirely equally for whatever aim they use it.  Affiliation with some active Esperantist society is recommended for every Esperantist, but not obligatory.}

\sectionline

\begin{tabu} to\textwidth{*2{>{\footnotesize}X[c]}}
Ĉi tiu verko estas permesita per la Creative~Commons Attribution-NonCommercial-ShareAlike 4.0 Internacia Permesilo. & This work is licensed under the Creative~Commons Attribution-NonCommercial-ShareAlike 4.0 International License. \\[1ex]
Angla traduko per Shawn KNIGHT. & English translation by Shawn Knight.\\
\multicolumn{2}{c}{\ccbyncsa}
\end{tabu}


\end{document}
